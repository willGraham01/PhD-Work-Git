\chapter{Example Systems} \label{ch:ExampleSystems}
In this chapter we present how the theory of chapters \ref{ch:QuantumGraphs} and \ref{ch:VectorEqns} comes together, and how we can determine the spectrum of variational problems on singular structures.
We also aim to illustrate some of the difficulties in this process, both numerically and analytically, and provide a brief discussion of these issues as they arise.
Some potential avenues of exploration for bypassing these issues (particularly on the numerical side) are suggested, and will be revisited in chapter \ref{ch:Conclusion}. \tstk{make this chapter!}
Throughout, it may be useful to the reader to recall the physical setup that our graphs in the plane represent (section \tstk{MAKE IT!}).

\section{Preliminaries} \label{sec:ExamplePrelims}
Our focus in the examples of this chapter will be to determine the eigenvalues $\omega^2$ of the problem \eqref{eq:CurlCurlEquationDivFree}, which we recall below
\begin{align*}
	-\ktcurl{\ddmes}\bracs{\ktcurl{\ddmes}u} &= \omega^2 u, \quad u\in\ktcurlSobDivFree{\ddom}{\ddmes}.
\end{align*}
In the process we employ we will also determine the eigenfunctions $u$, however these are not the focus of our analysis.
We shall change the underlying (period) graph $\graph$ between sections, but will make clear any changes prior to beginning any examples.
Using the work of section \ref{sec:CurlReductionToQG}; we know that we can determine the eigenvalues by solving the quantum graph problem given in \eqref{eq:CurlEdgeEquations}-\eqref{eq:CurlVertexConditions}.
To save on notational clutter we drop the overhead tilde notation for $\widetilde{u}_{2,jk},\widetilde{u}_{3,jk}$ that was used in section \ref{sec:CurlReductionToQG} and also define some convenient constants, for a given $\qm$ and $\wavenumber$,
\begin{align*}
	\qm_{jk} &= \bracs{R_{jk}\qm}_2, \quad\forall I_{jk}\in E,\\
	\effFreq &= \sqrt{\omega^2 - \wavenumber^2}.
\end{align*}
We also make use of some of the properties of \eqref{eq:CurlEdgeEquations}-\eqref{eq:CurlVertexConditions}.
In particular any (general) solution pair $u_{2,jk},u_{3,jk}$ that solves any two of \eqref{eq:CurlEdgeEquations} necessarily satisfies the remaining equation.
As such we can substitute \eqref{eq:CurlEdgeEquations3} into \eqref{eq:CurlEdgeEquations2} to obtain a single equation in $u_{3,jk}$.
The (general) solution $u_{3,jk}$ that would be obtained from this equation would then allow us to recover $u_{2,jk}$ and satisfy the remaining \eqref{eq:CurlEdgeEquations1}.
Similarly we can also note that a solution pair to \eqref{eq:CurlEdgeEquations} that satisfies \eqref{eq:CurlVertexConditions1}, then satisfies \eqref{eq:CurlVertexConditionsSum2} if and only if it satisfies \eqref{eq:CurlVertexConditionsDeriv}.
Given that our objective is just determining the eigenvalues of \eqref{eq:CurlCurlEquationDivFree}, it is thus sufficient for us to solve the (quantum graph) problem
\begin{align} \label{eq:QGEquation}
	0 &= -\bracs{\diff{}{t} + i\qm_{jk}}^2 u_{3,jk} + \bracs{\wavenumber^2 - \omega^2}u_{3,jk}.
\end{align}
\begin{subequations} \label{eq:QGVertexConditions}
	\begin{align}
		u_3 &\text{ is continuous at each } v_j\in V, \\
		0 &= \sum_{j\con k} \bracs{\diff{}{t} + i\qm_{jk}}u_{3,jk}\bracs{v_j}, \quad\forall v_j\in V
	\end{align}
\end{subequations}
for the eigenpairs $\bracs{\omega^2, u_{3,jk}}$.
We shall take the system \eqref{eq:QGEquation}-\eqref{eq:QGVertexConditions} as our starting point for the examples in this section.
In addition, we compute the general form for $u_{3,jk}$ as
\begin{align} \label{eq:CurlEdgeEqnGenSol}
	u_{3,jk} &= e^{-i\qm_{jk}t}\bracs{ C_{+}^{(jk)}e^{-i\effFreq t} + C_{-}^{(jk)}e^{i\effFreq t} },
\end{align}
which will be useful for proving the result of proposition \ref{prop:M-MatrixEntries}. 
As discussed in chapter \ref{ch:QuantumGraphs}, and as the eigenfunctions $u$ are not our primary interest, we are interested in obtaining the $M$-matrix for the problem \eqref{eq:QGEquation}-\eqref{eq:QGVertexConditions} (see section \ref{sec:M-MatrixTheory} for definition and details).
In fact, we can actually compute the $M$-matrix for a given value of $\effFreq$ and $\qm$ simply by examining \eqref{eq:QGEquation}, resulting in the following proposition.
\begin{prop}[$M$-matrix entries] \label{prop:M-MatrixEntries}
	Let $\graph=\bracs{V,E}$ be an embedded graph on which the problem \eqref{eq:QGEquation}-\eqref{eq:QGVertexConditions} is posed.
	For each $I_{jk}\in E$ let $\qm_{jk} = \bracs{R_{jk}\qm}_2$ and $l_{jk} = \abs{I_{jk}}$.
	Suppose that $\dmap u = e_k$ where $e_k$ is the $k$\textsuperscript{th} canonical unit vector in $\reals^{\abs{V}}$.
	Then the $j$\textsuperscript{th} entry of $\nmap u$, and hence the $jk$\textsuperscript{th} entry in the $M$-matrix, is given by
	\begin{align*}
		\bracs{\nmap u}_j &= 
		\begin{cases}
			\!\begin{aligned}
				&0,
			\end{aligned}			
			& j \not\con k, \\
			\!\begin{aligned}
				&-\sum_{j\conLeft k} \effFreq e^{i\qm_{jk}l_{jk}} \csc\bracs{l_{jk}\effFreq} 
				\\ &\quad - \sum_{j\conRight k} \effFreq e^{-i\qm_{kj}l_{kj}} \csc\bracs{l_{kj}\effFreq},
			\end{aligned}
			& j\neq k, \ j\con k, \\
			\!\begin{aligned}
				&\sum_{j\con l} \effFreq\cot\bracs{l_{jl}\effFreq}
				\\ &\quad + 2\effFreq\sum_{j\conLeft j} \cot\bracs{l_{jj}\effFreq} - \cos\bracs{\qm_{jj}l_{jj}}\csc\bracs{l_{jj}\effFreq},
			\end{aligned}
			& j=k.
		\end{cases}
	\end{align*}
	Note the choice of $j\conLeft j$ in the contributions from loops is simply a convention, $j\conRight j$ is equivalent here.
	Also recall the convention for summing over $j\con k$:
	\begin{align*}
		\sum_{j\con k} \effFreq\cot\bracs{l_{jl}\effFreq} &= \sum_{j\conLeft k} \effFreq\cot\bracs{l_{jl}\effFreq}	+ \sum_{j\conRight k} \effFreq\cot\bracs{l_{lj}\effFreq}
	\end{align*}
\end{prop}
\begin{proof}
	The proof is an explicit computation, and follows the same idea as in \tstk{EKK paper result, we just have QM here too} with adjustments for the fact that there are $\wavenumber$ and $\qm$ terms floating around.
	For each $k$, setting $\dmap u = e_k$ provides us with sufficient Dirichlet data at each vertex to eliminate the constants $C^{(jk)}_{\pm}$ in \eqref{eq:CurlEdgeEqnGenSol}.
	This in turn enables us to explicitly write the solutions $u_{3,jk}$, differentiate them, and read off their values at any relevant vertices.
	This then provides us with the value of each term in the sum in $\bracs{\nmap u}_j$, for each $j$.
\end{proof}
Note that in proving this proposition one also obtains an analytic form for the edge solutions $u_{3,jk}$, and that the $M$-matrix can be thought of as a function of $\effFreq$ parametrised by $\qm$, $M_{\qm}\bracs{\effFreq}$.

\subsection{Approaches to Finding Spectra via the M-Matrix} \label{sec:NumericalMethodsDiscussion}
Section \ref{sec:M-MatrixTheory} briefly touched on two approaches to using the $M$-matrix as a tool for determining the spectrum of a quantum graphs problem, which we now discuss in light of proposition \ref{prop:M-MatrixEntries}. \newline

It is possible to determine the spectrum of a quantum graph problem numerically, by solving a generalised (matrix) eigenvalue problem involving the $M$-matrix.
Of course such a scheme requires the ability to evaluate the $M$-matrix as a function of $\effFreq$ (and in our case, parametrised by $\qm$ too).
It is entirely possible to do this using \eqref{eq:QGEquation}-\eqref{eq:QGVertexConditions} as the starting point - one simply goes through the computation in the proof of proposition \ref{prop:M-MatrixEntries} numerically, each time the $M$-matrix needs to be evaluated at a new value of $\effFreq$ and $\qm$.
This requires solving the ODE \eqref{eq:QGEquation} $\abs{E}\times\abs{V}$ times per evaluation of the $M$-matrix\footnote{Once for each column of the $M$-matrix, giving the factor $\abs{V}$. 
Then given a column, each edge has to be solved on with the appropriate boundary conditions, giving the factor $\abs{E}$. 
Of course, there are prudent ways to reduce this cost by ignoring edges with 0 Dirichlet data at each end, for example.}
then computing the values of the components of $\nmap u$ using the function values found by the numerical solver, and hence constructing the $M$-matrix.
Proposition \ref{prop:M-MatrixEntries} essentially bypasses the problems associated with solving \eqref{eq:QGEquation} many times, allowing us to skip directly to building the $M$-matrix for given $\omega^2$ and $\kt$.
With careful programming one can even make building the $M$-matrix a function call that is passed $\bracs{\omega^2, \wavenumber, \qm}$ and only relies on knowing the $\qm_{jk}$; and the lengths and adjacency matrix of $\graph$ to return the $M$-matrix. \tstk{hermitian so only need half too!}
Whilst this is a very helpful consequence of proposition \ref{prop:M-MatrixEntries}, there are other things that a numerical approach must consider.
\begin{itemize}
	\item One of the foremost problems is that we are actually looking to solve each member of a family of generalised eigenvalue problems (parametrised by $\qm$) to determine the spectrum of our original variational problem.
	Although there is always the naive approach of discretising the range of $\qm$ and solving $M_{\qm}\bracs{\effFreq}v=0$ for each discrete value, this raises further questions on how fine our discretisation needs to ensure we do not miss any features of the spectrum in our results, not to mention the increase in computational cost that comes with solving many generalised eigenvalue problems.
	\item Although the $M$-matrix has several useful analytic properties (section \ref{sec:M-MatrixTheory}) these do not translate into nice properties of the solutions to the generalised eigenvalue problem.
	In particular, for a given $\qm$ the $M$-matrix will in general have a (countably) infinite number of eigenvalues due to the fact that it is periodic in $\effFreq$ (this period is computable as the lowest common multiple of all the $l_{jk}$, or if there are loops in $\graph$ of all the $l_{jk}$ and $\qm_{jj}l_{jj}$).
	If we are to solve numerically we need a method for dealing with this caveat, for example working out the period of the $M$-matrix and only search for eigenvalues in that range.
	But even then this requires our numerical scheme to be able to find all the distinct eigenvalues in over one period of the $M$-matrix, and in general there is no easy check that we can perform to validate that the output of our numerical scheme has done this.
	\item Utilising symmetries in the quantum graph $\graph$ may help reduce the complexity of the process of building the $M$-matrix.
	In particular if both $I_{jk}$ and $I_{kj}$ are edges with the same length and $\qm_{jk}=\qm_{kj}$, the $j\neq k, j\con k$ case of proposition \ref{prop:M-MatrixEntries} reduces to $-2\sum_{j\conLeft k}\effFreq\cos\bracs{\qm_{jk}l_{jk}}\csc\bracs{l_{jk}\effFreq}$ (equivalently we could use $j\conRight k$).
	In this case the numerical scheme no longer has to deal with complex floats, and there are slightly fewer terms in $M$ to construct.
	If there is also a symmetry in $\qm$, then it is sufficient to fix the second component of $\qm$ as 0 and effectively ``halve" the size of the family of problems that need to be solved.
\end{itemize}
So whilst it is possible to conceive of a numerical method that begins from the equations or $M$-matrix and computes the spectrum of the original variational problem, in practice there are a number of complications that must be considered first.
The paragraph below briefly discusses how in some cases proceeding analytically can help, but this is usually limited to small or somewhat specialised graphs. \newline

Progress can be made analytically on finding the spectrum if one is willing to plough through the long and complex expressions that can come out of proposition \ref{prop:M-MatrixEntries}.
To this end, the starting point is often considering the equation $\det\bracs{M_{\qm}\bracs{\effFreq}}=0$.
For graphs with a lot of symmetries and a small number of vertices this approach becomes much more appealing as the size of $M$ is reduced, as is the complexity of it's components.
And even if exact solution is impossible, one can often arrive at a simpler problem that can be tackled numerically, without as many problems as the generalised eigenvalue problem faces.
This is the approach that we take in our examples, using proposition \ref{prop:M-MatrixEntries} to write out the $M$-matrix, then manipulating $\det\bracs{M_{\qm}\bracs{\effFreq}}=0$ to obtain an expression of the form $0 = \mathcal{F}\bracs{\effFreq, \qm}$, which we can then treat as the situation requires.

\section{Cross in the Periodic Plane} \label{sec:ExampleCrossInPlane}
As the title says, but make it more formal and fancy and not in reference to the TFR. 
Spectrum is whole real line, because no gaps open...

We begin with an example primarily for illustrative purposes, as we shall see that the spectrum we obtain doesn't possess any interesting features.
Consider the periodic graph defined as follows; for each $\bracs{n,m}\in\integers^2$ define
\begin{align*}
	v_1^{\bracs{n,m}} = \bracs{\recip{2},0} + \bracs{n,m}, 
	&\quad v_2^{\bracs{n,m}} = \bracs{0,\recip{2}} + \bracs{n,m}, \\
	v_3^{\bracs{n,m}} = \bracs{\recip{2},\recip{2}} + \bracs{n,m}. & \\
	I_{13}^{\bracs{n,m}} = \sqbracs{v_1^{\bracs{n,m}}, v_3^{\bracs{n,m}}},
	&\quad I_{23}^{\bracs{n,m}} = \sqbracs{v_2^{\bracs{n,m}}, v_3^{\bracs{n,m}}}, \\
	I_{31}^{\bracs{n,m}} = \sqbracs{v_3^{\bracs{n,m}}, v_1^{\bracs{n+1,m}}},
	&\quad I_{32}^{\bracs{n,m}} = \sqbracs{v_3^{\bracs{n,m}}, v_2^{\bracs{n,m+1}}}.
\end{align*}
Then with 
\begin{align*}
	V^* &= \clbracs{v_j^{\bracs{n,m}} \ \vert \ j\in\clbracs{1,2,3}, \bracs{n,m}\in\integers^2}, \\
	E^* &= \clbracs{I_{jk}^{\bracs{n,m}} \ \vert \ j,k\in\clbracs{1,2,3}, \bracs{n,m}\in\integers^2},
\end{align*}
$\graph^* = \bracs{V^*,E^*}$ is an embedded, periodic graph in $\reals^2$.
It's period graph occupies $\sqbracs{0,1}^2$ and can be visualised in figure \ref{fig:Diagram_TFRGraph}; consisting of 5 (although due to the association at the edges, effectively 3) vertices and 4 edges.
\begin{figure}[b!]
	\centering
	\begin{subfigure}[t]{0.45\textwidth}
		\centering
		\includegraphics[height=4.5cm]{Diagram_TFRGraph.pdf}
		\caption{\label{fig:Diagram_TFRGraph} The period graph that we are considering. All edges have length $\recip{2}$, and the quasi-momentum on horizontal edges is $-\qm_1$ and on vertical edges is $-\qm_2$.}
	\end{subfigure}
	~
	\begin{subfigure}[t]{0.45\textwidth}
		\centering
		\includegraphics[height=4.5cm]{Diagram_TFRQuantumGraph.pdf}
		\caption{\label{fig:Diagram_TFRQuantumGraph} The quantum graph that our problem corresponds to. Due to the identification of vertices on the boundary of the period graph, we are effectively dealing with a 3-vertex quantum graph.}
	\end{subfigure}
	\caption{\label{fig:5VertexCross} (\ref{fig:Diagram_TFRGraph}) The period graph that we are considering, corresponding to a waveguide with a cross-like pattern in the cross-section. (\ref{fig:Diagram_TFRQuantumGraph}) The equivalent quantum graph on which we pose \eqref{eq:QGEquation}-\eqref{eq:QGVertexConditions}, retaining the lengths $l_{jk}$ and $\qm_{jk}$ from the edges of the original period cell.}
\end{figure}
It is possible of course to draw the period graph with 3 vertices wholly within $\sqbracs{0,1}^2$ and 4 hanging edges.
After associating the edges of the period graph, we obtain it's associated quantum graph $\graph=\bracs{V,E}$ with $V=\clbracs{v_1,v_2,v_3}$, $E=\clbracs{I_{13},I_{23},I_{31},I_{32}}$ with lengths
\begin{align*}
	l_{13} = l_{23} = l_{31} = l_{32} = \recip{2}.
\end{align*}
Given that all the edges of $\graph^*$ are parallel to the co-ordinate axes, it is also fairly easy to compute the values of $\qm_{jk}$ for each $I_{jk}\in E$ and a given $\qm=\bracs{\qm_1,\qm_2}\in[-\pi,\pi)^2$;
\begin{align*}
	\qm_{13} = \qm_{31} = -\qm_2, &\quad \qm_{23} = \qm_{32} = -\qm_1.
\end{align*}

We now look to determine the spectrum of the problem \eqref{eq:CurlCurlEquationDivFree} on $\graph\subset\sqbracs{0,1}^2$ with respect to the singular measure $\ddmes$ on $\graph$.
We know that this is equivalent to determining the eigenvalues $\omega^2$ of the system \eqref{eq:QGEquation}-\eqref{eq:QGVertexConditions}, and using proposition \ref{prop:M-MatrixEntries} we can write down the $M$-matrix as
\begin{align*}
	M_{\qm}\bracs{\effFreq} &=
	\begin{pmatrix}
		2\effFreq\cot\bracs{\frac{\effFreq}{2}} & 0 & -2\effFreq\csc\bracs{\frac{\effFreq}{2}}\cos\bracs{\frac{\qm_2}{2}} \\
		0 & 2\effFreq\cot\bracs{\frac{\effFreq}{2}} & -2\effFreq\csc\bracs{\frac{\effFreq}{2}}\cos\bracs{\frac{\qm_1}{2}} \\
		-2\effFreq\csc\bracs{\frac{\effFreq}{2}}\cos\bracs{\frac{\qm_2}{2}} & -2\effFreq\csc\bracs{\frac{\effFreq}{2}}\cos\bracs{\frac{\qm_1}{2}} & 4\effFreq\cot\bracs{\frac{\effFreq}{2}}
	\end{pmatrix}.
\end{align*}
At this point we have the option of solving for the spectrum numerically or analytically, however because we have such a small and symmetric problem we can actually determine the spectrum of the problem analytically by solving for when the determinant of $M_{\qm}$ is 0.
After some calculation and cancellation we find that
\begin{align*} 
	\det\bracs{M_{\qm}\bracs{\effFreq}} = 0& \\
	&\Leftrightarrow\cos\effFreq = \cos\bracs{\frac{\qm_1+\qm_2}{2}}\cos\bracs{\frac{\qm_1-\qm_2}{2}}. \labelthis\label{eq:ExampleCrossInPlaneSolution}
\end{align*}
Note that \eqref{eq:ExampleCrossInPlaneSolution} has the symmetries in $\qm_1$ and $\qm_2$ that we expect from the geometry of $\graph^*$.
Given this symmetry and that $\qm\in[-\pi,\pi)^2$, the right hand side of \eqref{eq:ExampleCrossInPlaneSolution} attains every value in the interval $\sqbracs{0,1}$ and thus, for every $\effFreq>0$ there exists a $\qm\in[-\pi,\pi)^2$ such that \eqref{eq:ExampleCrossInPlaneSolution} holds.
And so we must conclude that the spectrum of the quantum graph problem on $\graph$, and hence the spectrum of the variational problem on the embedded graph $\graph^*$, is the region $\omega^2\in\bracs{\wavenumber^2, \infty}$. \newline

Whilst this result is not particularly exciting nor useful for wave-guidance in a photonic fibre context; this simple example highlights the reasons why we have adopted this approach.
Foremost is the ability to use the $M$-matrix to obtain a result like \eqref{eq:ExampleCrossInPlaneSolution}, where we can just read off the spectrum of the quantum graph problem and hence original variational problem.
Without the use of the $M$-matrix; determining the eigenvalues would have to be done by imposing the boundary conditions \eqref{eq:QGVertexConditions} on the general solutions \eqref{eq:CurlEdgeEqnGenSol}, which would lead to a complex system of simultaneous equations in the constants $C_{\pm}^{jk}$ that would need to be solved (then repeated for each $\qm$ if done numerically).
Another benefit that is made apparent is that simplifying the condition $\det\bracs{M_{\qm}\bracs{\effFreq}} = 0$, even if it yields an equation without a closed form, normally provides something that is easier to solve numerically.
Enough progress analytically can replace the need to solve a generalised eigenvalue problem with a (comparatively simple) root-finding problem, or (as is the case here) provide the entire spectrum immediately.
In sections \ref{sec:ExampleGeneralLengths} and \ref{sec:ExampleThickVertex} we will explore systems where we have to take this ``hybrid" mixture of working from $\det\bracs{M_{\qm}\bracs{\effFreq}} = 0$ to obtain an equation that we can then explore numerically.

\tstk{what about the chevron example?}

\section{General period cell with lengths system} \label{sec:ExampleGeneralLengths}
We now turn to a more general example, and assume the period graph $\graph=\bracs{V,E}\subset\sqbracs{0,1}^2$ as shown in figure \ref{fig:Diagram_5VertexGraph}.
\begin{figure}[b!]
	\begin{subfigure}[b]{0.45\textwidth}
		\centering
		\includegraphics[height=5cm]{Diagram_5VertexGraph.pdf}
		\caption{\label{fig:Diagram_5VertexGraph} The period graph that we are considering. The need for periodicity forces the placement of $v_4$ and $v_5$ on the opposite side of the unit cell to $v_2$ and $v_1$ respectively.}
	\end{subfigure}
	~
	\begin{subfigure}[b]{0.45\textwidth}
		\centering
		\includegraphics[scale=0.75]{Diagram_5VertexGraphQG.pdf}
		\caption{\label{fig:Diagram_5VertexGraphQG} The associated quantum graph for this problem. Due to the association of $v_3$ and $v_4$, we have two edges directed out of $v_2$ into the single vertex $v_3=v_4$, however this does not introduce any theoretical complexities.}
	\end{subfigure}
	\caption{\label{fig:GeneralLengthsDiagrams} The period graph and it's associated quantum graph for the problem in section \ref{sec:ExampleGeneralLengths}. }
\end{figure}
For formality, we set $l_1,l_2,l_3,l_4>0$ and place vertices
\begin{align*}
	v_1 = \bracs{0,l_1}, &\quad v_2 = \bracs{l_2,l_1+l_4}, \quad v_3 = \bracs{l_2+l_3, 0}, \\
	v_4 = \bracs{l_2+l_3, 1}, &\quad v_5 = \bracs{1, l_1}.
\end{align*}
Then we place edges $I_{12}$, $I_{23}$, $I_{24}$, $I_{35}$ corresponding to the segments joining the relevant vertices, and can compute the lengths
\begin{align*}
	l_{12} &= \sqrt{l_2^2 + l_4^2}, &\quad l_{23} = \sqrt{l_3^2+\bracs{l_1+l_4}^2}, \\
	l_{24} &= \sqrt{l_3^2 + \bracs{1-l_1-l_4}^2}, &\quad l_{35} = \sqrt{l_1^2+\bracs{1-l_2-l_3}^2}.
\end{align*}
We also have to compute the rotation matrices $R_{jk}$ for the edges and hence (given $\qm\in[-\pi,\pi)^2$), the parameters $\qm_{jk}$;
\begin{align*}
	R_{12} &= l_{12}^{-1}
	\begin{pmatrix}
		l_4 & -l_2 \\ 
		l_2 & l_4
	\end{pmatrix},
	&\quad
	R_{23} &= l_{23}^{-1}
	\begin{pmatrix}
		-\bracs{l_1+l_4} & l_3 \\
		-l_3 & -\bracs{l_1+l_4}
	\end{pmatrix},
	\\
	R_{24} &= l_{24}^{-1}
	\begin{pmatrix}
		1-l_1-l_4 & -l_3 \\
		l_3 & 1-l_1-l_4
	\end{pmatrix},
	&\quad
	R_{31} &= l_{31}^{-1}
	\begin{pmatrix}
		l_1 & l_2+l_3-1 \\
		1-l_2-l_3 & l_1
	\end{pmatrix},
	\\
	\qm_{12} &= l_{12}^{-1}\bracs{l_2\qm_1 + l_4\qm_2},
	&\quad \qm_{23} &= -l_{23}^{-1}\bracs{l_3\qm_1 + \bracs{l_1+l_4}\qm_2}, \\
	\qm_{24} &= l_{24}^{-1}\bracs{l_3\qm_1 + \bracs{1-l_1-l_4}\qm_2},
	&\quad \qm_{31} &= l_{31}^{-1}\bracs{\bracs{1-l_2-l_3}\qm_1 + l_1\qm_2}.
\end{align*}
Note that due to the association at the edges of $\graph$, $v_1$ and $v_5$ are associated, as are $v_3$ and $v_4$, so we again have an associated 3-vertex quantum graph, which is shown in figure \ref{fig:Diagram_5VertexGraphQG}.
Furthermore we have arrived at a quantum graph that possesses two edges directed out of $v_2$ into the same vertex $v_3=v_4$, and it is for this reason that we keep the distinct labels $I_{23}$ and $I_{24}$ for these edges and only relabel $I_{35}$ as $I_{31}$.
In terms of our existing notation, the only proviso we need to make is that (formally)
\begin{align*}
	\sum_{2\conLeft 3} = \sum_{I_{23}} + \sum_{I_{24}}.
\end{align*}
Seeking to determine the spectrum of \eqref{eq:CurlCurlEquationDivFree} via \eqref{eq:QGEquation}-\eqref{eq:QGVertexConditions}, we use proposition \ref{prop:M-MatrixEntries} to construct the ($3\times3$) $M$-matrix,
\begin{align*}
	M_{\qm}\bracs{\effFreq} &= \effFreq
	\begin{pmatrix}[3]
		\cot\bracs{l_{12}\effFreq} & -
		e^{-i\qm_{12}l_{12}}\csc\bracs{l_{12}\effFreq} &
		-e^{-i\qm_{31}l_{31}}\csc\bracs{l_{31}\effFreq} \\
		-e^{i\qm_{12}l_{12}}\csc\bracs{l_{12}\effFreq} &
		\!\begin{aligned} \cot\bracs{l_{12}\effFreq} & + \cot\bracs{l_{23}\effFreq} \\ & \ \ + \cot\bracs{l_{24}\effFreq} \end{aligned} &
		\!\begin{aligned} & -e^{-i\qm_{23}l_{23}}\csc\bracs{l_{23}\effFreq} \\ & \ \ - e^{-i\qm_{24}l_{24}}\csc\bracs{l_{24}\effFreq} \end{aligned} \\
		-e^{i\qm_{31}l_{31}}\csc\bracs{l_{31}\effFreq} &
		\!\begin{aligned} & -e^{i\qm_{23}l_{23}}\csc\bracs{l_{23}\effFreq} \\ & \ \ - e^{i\qm_{24}l_{24}}\csc\bracs{l_{24}\effFreq} \end{aligned} &
		\!\begin{aligned} \cot\bracs{l_{23}\effFreq} & + \cot\bracs{l_{24}\effFreq} \\ & \ \ + \cot\bracs{l_{31}\effFreq}  \end{aligned}
	\end{pmatrix}.
\end{align*}
Whilst not the nicest to look at, $M_{\qm}$ is still fairly simple to evaluate on a computer.
For those who would persist with an analytical approach, we can proceed further from $\det\bracs{M_{\qm}\bracs{\effFreq}}=0$ to deduce that
\begin{align*}
	0 &= -\bracs{2 + \cot_{12} + \cot_{23} + \cot_{24} + \cot_{31}} \\
	&\quad +2\csc_{12}\csc_{23}\csc_{31} \bracs{ \cos_{12}\cos_{23}\cos_{31} - \cos\bracs{\beta_{12}+\beta_{23}+\beta_{31}} } \\
	&\quad +2\csc_{12}\csc_{24}\csc_{31} \bracs{ \cos_{12}\cos_{24}\cos_{31} - \cos\bracs{\beta_{12}+\beta_{24}+\beta_{31}} } \\
	&\quad +2\csc_{23}\csc_{24} \bracs{ \cos_{23}\cos_{24} - \cos\bracs{\beta_{23}-\beta_{24}} },
\end{align*}
under the conventions
\begin{align*}
	\cot_{jk} = \cot\bracs{l_{jk}\effFreq}, &\quad \csc_{jk} = \csc\bracs{l_{jk}\effFreq}, \\
	\cos_{jk} = \cos\bracs{l_{jk}\effFreq}, &\quad \beta_{jk} = \qm_{jk}l_{jk}.
\end{align*}
Needless to say, having broken most symmetries of $\graph$ the equation we are rewarded with isn't particularly fun to work with\footnote{Understatement intended.}. 
\tstk{time to play with this system and plot the spectrum Will... :(}

have made some progress in the case when $l_1=l_2=0.5, l_3=l_4=0$ and numerics can confirm the existence of spectral band gaps. 
It's a tricky argument but essentially we end up with
\begin{align*}
	\cos\qm_1 + \cos\qm_2 + \cos\bracs{\qm_1+\qm_2} &=
	\recip{2}\cos\frac{\effFreq}{\sqrt{2}}
	+ \frac{3}{8}\cos\bracs{\frac{\sqrt{2}-1}{\sqrt{2}}\effFreq}
	+ \frac{9}{8}\cos\bracs{\frac{\sqrt{2}+1}{\sqrt{2}}\effFreq} \\
	&\quad + \recip{2}\sin\bracs{\frac{\sqrt{2}+1}{\sqrt{2}}\effFreq}
	- \recip{2}\sin\bracs{\frac{\sqrt{2}-1}{\sqrt{2}}\effFreq}.
\end{align*}
Call the RHS $\Xi$ again.
The LHS is bounded between $-3$ and $1.5$ for the range of $\qm$ (1.5 at $\qm_1=\qm_2=\frac{2\pi}{3}$, whilst min is at $\qm=0$).
This means that we can try to solve $-3\leq \Xi\leq 1.5$ numerically, which I have done in my Python scripts, and there are (tiny, tiny, TINY) bandgaps in the spectrum.
So we could produce another bandgap plot too, by sweeping through $\wavenumber$ too.
In terms of validation, $\Xi$ is sadly not periodic (even though it's the sum of periodic functions...see for example \url{https://math.stackexchange.com/questions/1079/sum-of-two-periodic-functions}) so it's NOT NECESSARILY a genuine set of bandgaps, it could just be a finite number of finite-bandgaps which culminates in one infinite-bandgap.
Sooooo.... fun times, I guess it would also be good to try and get the general NLII solver on this to try a more complicated system with the lengths not being so nice, but that, alas, proves nothing.
That of course, also requires you having a working version of the NLII, of course - guess you've now got an ITT project!

\section{Thick Vertex Case} \label{sec:ExampleThickVertex}
In our final example we return to the (period graph) defined in section \ref{sec:ExampleCrossInPlane}, and still examine \eqref{eq:QGEquation}.
However make a slight departure from the boundary conditions in \eqref{eq:QGVertexConditions} and take a non-zero coupling constant $\alpha$ (c.f. section \ref{sec:DEonQG}) at the vertex $v_3$, meaning that our new boundary conditions are
\begin{align} \label{eq:QGThickVertexConditions}
		u_3 &\text{ is continuous at each } v_j\in V, \\
		0 &= \sum_{j\con k} \bracs{\diff{}{t} + i\qm_{jk}}u_{3,jk}\bracs{v_j}, \quad j\in\clbracs{1,2,4,5}, \\
		\alpha u_3\bracs{v_3} &= \sum_{3\con k} \bracs{\diff{}{t} + i\qm_{3k}}u_{3,3k}\bracs{v_3}.
\end{align}
\tstk{why? What? How is this legal? Refer back to Olaf-Post and Kuchment work when it was discussed for as to why we care about this problem, and then explain how we can add point masses to singular measures and go through the motions of the previous chapters to get this system. }

The $M$-matrix itself is no different to that which we computed in section \ref{sec:ExampleCrossInPlane}, 
\begin{align*}
	M_{\qm}\bracs{\effFreq} &=
	\begin{pmatrix}
		2\effFreq\cot\bracs{\frac{\effFreq}{2}} & 0 & -2\effFreq\csc\bracs{\frac{\effFreq}{2}}\cos\bracs{\frac{\qm_2}{2}} \\
		0 & 2\effFreq\cot\bracs{\frac{\effFreq}{2}} & -2\effFreq\csc\bracs{\frac{\effFreq}{2}}\cos\bracs{\frac{\qm_1}{2}} \\
		-2\effFreq\csc\bracs{\frac{\effFreq}{2}}\cos\bracs{\frac{\qm_2}{2}} & -2\effFreq\csc\bracs{\frac{\effFreq}{2}}\cos\bracs{\frac{\qm_1}{2}} & 4\effFreq\cot\bracs{\frac{\effFreq}{2}}
	\end{pmatrix},
\end{align*}
save now we have a non-zero matrix of coupling constants $A=\mathrm{diag}\bracs{0,0,\alpha}$ and will need to solve
\begin{align*}
	\bracs{M_{\qm}\bracs{\omega^2} - \omega^2 A}v = 0.
\end{align*}
If we chose to proceed numerically we could make use of the suggestion in section \ref{sec:DEonQG} and write $\widetilde{M}_{\qm}\bracs{\omega^2} = M_{\qm}\bracs{\omega^2} - \omega^2 A$; however we will elect to continue our analysis analytically here because it can be shown that
\begin{align*}
	\det\bracs{\bracs{M_{\qm}\bracs{\omega^2} - \omega^2 A}} &= 0 \\
	\Leftrightarrow \cos\bracs{\frac{\qm_1+\qm_2}{2}}\cos\bracs{\frac{\qm_1-\qm_2}{2}} &= \cos\effFreq - \frac{\alpha}{4}\frac{\omega^2}{\effFreq}\sin\effFreq. \labelthis\label{eq:ExampleThickVertexSolution}
\end{align*}
As we might expect; there is a lot of similarity between \eqref{eq:ExampleCrossInPlaneSolution} and \eqref{eq:ExampleThickVertexSolution}, the only difference being the introduction of a term with a factor of $\alpha$.
The left hand side of \eqref{eq:ExampleThickVertexSolution} attains every value in the interval $\sqbracs{-1,1}$ over the range $\qm\in[-\pi,\pi)^2$, but the right hand side cannot be written as a function of $\effFreq$ alone.
So finding pairs\footnote{We could elect to find pairs $\bracs{\omega, \effFreq}$ of course; but as $\effFreq$ is a function of $\omega$ and $\wavenumber$, and the conventions that surround dispersion relations, it makes more sense to phrase things in terms of $\bracs{\omega, \wavenumber}$.} $\bracs{\omega, \wavenumber}$ amounts to finding all $\bracs{\omega, \wavenumber}$ such that
\begin{align} \label{eq:ExampleThickVertexDispExpr}
	-1 \leq \cos\effFreq - \frac{\alpha}{4}\frac{\omega^2}{\effFreq}\sin\effFreq =: \Xi\bracs{\omega, \wavenumber} \leq 1.
\end{align}
One can visualise the curve $\Xi$ for a fixed value of $\wavenumber=\pi$ (and with $\alpha=-1$) in figure \ref{fig:ThickVertex_CurlsTFRSetup_k1a1}.
\begin{figure}[b!]
	\centering
	\includegraphics[scale=0.75]{ThickVertex_CurlsTFRSetup_k1a1.pdf}
	\caption{\label{fig:ThickVertex_CurlsTFRSetup_k1a1} The function $\Xi$ for a fixed value of $\wavenumber=\pi$. Red regions indicate those $\omega$ that correspond to eigenvalues $\omega^2$, when $\Xi\bracs{\omega, \pi}$ takes values between $-1$ and $1$ (black lines). One can see the existence of spectral bands which shrink in size as $\omega\rightarrow\infty$.}
\end{figure}
We can observe that for a fixed $\wavenumber$ the points $\omega$ that satisfy \eqref{eq:ExampleThickVertexDispExpr} are divided into distinct ``spectral bands" which shrink as $\omega\rightarrow\infty$. 
Of course the value of $\alpha$ changes the shape of $\Xi$ which in turn will also effect the set of points $\bracs{\omega, \wavenumber}$ that solve \eqref{eq:ExampleThickVertexDispExpr}. \newline

More useful and relevant for wave-guidance purposes is that a dispersion- or band-gap-plot can be generated from \eqref{eq:ExampleThickVertexDispExpr}, as presented in figure \ref{fig:ThickVertex_CurlsTFRSetup}.
\begin{figure}[h]
	\centering
	\includegraphics[scale=0.75]{ThickVertex_CurlsTFRSetup.pdf}
	\caption{\label{fig:ThickVertex_CurlsTFRSetup} Dispersion plot for the system in section \ref{sec:ExampleThickVertex}, yellow regions correspond to $\omega, \wavenumber$ pairs that solve \eqref{eq:CurlCurlEquationDivFree}. We observe spectral ``band-gaps" in $\omega$ for each $\wavenumber$, and \tstk{maybe compare to physical plot from papers?}}
\end{figure}
In this figure points $\bracs{\omega, \wavenumber}$ in yellow correspond to solutions to \eqref{eq:ExampleThickVertexDispExpr}, whilst those in blue do not.
\tstk{compare to David Papers? Also either recall or explain why this is useful in PCFs :L}

\section{Summary}
Chapter summary of the results and how they might be physically interpreted.
Speculate on future developments.