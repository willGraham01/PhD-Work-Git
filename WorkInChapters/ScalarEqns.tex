\chapter{Scalar Equations} \label{ch:ScalarEqns}
In this chapter we look at scalar wave equations in our waveguide-like geometry, treating the cross-sectional structure as singular.
We base our approach largely off the work of Zhikov \cite{zhikov2000extension}, in that we consider what can be colloquially described as ``differential equations with respect to a measure $\ddmes$".
We will give a rigorous definition of the kinds of problems and spaces that this colloquialism covers in the following sections, and will demonstrate how this framework allows us to obtain a quantum graph problem from our variational formulation.
This will allow us to make use of the theory which was highlighted in \ref{ch:QuantumGraphs}, and additionally will make our variational problems open to numerical solution.
At the conclusion of this chapter we will have highlighted the techniques that are available to us, and how we shall adapt them for the more descriptive (or physical) vector systems we consider in chapter \ref{ch:3VectorEqns}.

\section{The Scalar Sobolev Spaces} \label{sec:ScalarSobSpaces}
In this section we look to construct the function spaces that we shall be working with throughout the chapter, as well as discussing some of the consequences of this construction.
What we present is a synopsis of the work of Zhikov presented in \cite{zhikov2000extension}, although with an adapted notation which will suit our needs when we later make a link to Quantum Graph problems. \newline

Let $n\in\naturals$, $D\subset\reals^n$ and $\nu$ be a (Borel) measure on $D$.
Denote the set of smooth functions on $D$ by $\smooth{D}$, and then let $W=W\bracs{D,\mathrm{d}\nu}$ be the closure of the set of pairs $\bracs{\phi,\grad\phi}$ in $\ltwo{D}{\nu}\times\ltwo{D}{\nu}^n$, where $\phi\in\smooth{D}$.
That is
\begin{align*}
	W = W\bracs{D,\nu} &= \overline{\clbracs{\bracs{\phi,\grad\phi} \ \vert \ \phi\in\smooth{D}}} \quad \text{in} \ \ltwo{D}{\nu}\times\ltwo{D}{\nu}^n.
\end{align*}
The notation above is chosen to provide similarities to the $W$-style construction of classical Sobolev spaces (in which integration is performed with respect to the $n$-dimensional Lebesgue measure).
An element of $W$ is a pair $\bracs{u,z}$; and as the above construction suggests we would like to associate the function $z$ with a distributional derivative of sorts, however there is a problem with making this association.
Suppose that $\bracs{u,z}\in W$ and $\bracs{0,y}\in W$, then we also have $\bracs{u,z+y}\in W$ too by taking an approximating sequence for each pair and forming another sequence by point-wise addition of terms.
As a result each $u\in\ltwo{D}{\nu}$ has multiple (distinct) functions $z\in\ltwo{D}{\nu}^n$ such that $\bracs{u,z}\in W$; and so as it stands it does not make sense to associate ``\textit{a} gradient" to $u$, because $u$ has multiple candidates for this role.
In what follows we will look into this quirk, and deduce that we can make sense of the idea of ``\textit{the} gradient". \newline

Let use denote the set of ``gradients of zero" by $\gradZero{D}{\nu}$, that is
\begin{align*}
	\gradZero{D}{\nu} &= \clbracs{z\in\ltwo{D}{\nu}^n \ \vert \ \bracs{0,z}\in W}, \\
		&= \clbracs{z\in\ltwo{D}{\nu}^n \ \vert \ \exists\phi_n\in\smooth{D} \text{ such that } \phi_n\lconv{\ltwo{D}{\nu}}0, \grad\phi_n\lconv{\ltwo{D}{\nu}^n}z}, \labelthis\label{eq:GradZeroSequenceDefinition}
\end{align*}
the construction of $W$ ensuring the two sets coincide.
A fact that follows immediately from this definition and will be used later is that $\gradZero{D}{\nu}$ is a closed linear subspace of $\ltwo{D}{\nu}^n$.
The illustration of the non-uniqueness of gradients earlier employed the fact that we can always add an element of $\gradZero{D}{\nu}$ to the second member of a pair $\bracs{u,z}\in W$ and produce another element of $W$.
Colloquially this can be expressed by the following statement; we can always ``add a gradient of zero" to an existing ``gradient", which will produce another ``gradient".
This hints at the possibility that whilst we any $u$ may not have a unique gradient, it might at least have only one relevant gradient in the context of a mathematical problem.
With this in mind, consider the following variational problem; find a pair $\bracs{u,z}\in W$ such that
\begin{align} \label{eq:TangentialGradientVariationalMotivation}
	\integral{D}{z \cdot \grad\overline{\phi} - u\overline{\phi}}{\nu} &= 0 \quad \forall\phi\in\smooth{D}.
\end{align}
Setting aside questions of existence and uniqueness of solutions to this problem for the purposes of illustration, take an element $g\in\gradZero{D}{\nu}$, and an approximating sequence $\phi_n$ as in the definition \eqref{eq:GradZeroSequenceDefinition}.
Substituting $\phi_n$ into \eqref{eq:TangentialGradientVariationalMotivation} and employing a limit-argument, we can see that
\begin{align*}
	\integral{D}{z \cdot g}{\nu} &= 0 \quad \forall g\in\gradZero{D}{\nu}.
\end{align*}
Namely that the member $z$ of the solution pair is orthogonal (in the $\ltwo{D}{\nu}$-norm) to $\gradZero{D}{\nu}$.
Stepping back for a moment, as $\gradZero{D}{\nu}$ is a closed linear subspace of $\ltwo{D}{\nu}^n$ we can decompose $\ltwo{D}{\nu}^n$ as
\begin{align*}
	\ltwo{D}{\nu}^n &= \gradZero{D}{\nu}^{\perp} \oplus \gradZero{D}{\nu}.
\end{align*}
But we have just seen that the member of the solution pair $z\in\gradZero{D}{\nu}^{\perp}$, hence $z$ is the \textit{unique} element of $\gradZero{D}{\nu}^{\perp}$ such that every pair $\bracs{u,\tilde{z}}\in W$ can be written as $\bracs{u,z+g}\in W$ for some $g\in\gradZero{D}{\nu}$.
Thus whilst the concept of a unique gradient does not make sense in this setting, once we consider variational problems we can obtain the concept of the \textit{tangential} gradient $z$, which is unique.
We therefore adopt the notation $z := \grad_\nu u$ for the second member of the solution pair $\bracs{u,z}\in W$, and can now define the (non-classical) ``Sobolev space"
\begin{align*}
	\gradSob{D}{\nu} &:= \clbracs{\bracs{u,\grad_\nu u}\in W \ \vert \ \grad_\nu u \perp \gradZero{D}{\nu}}.
\end{align*}
Furthermore we can now pose variational problems on this space; for example we would write \eqref{eq:TangentialGradientVariationalMotivation} as find $\bracs{u,\grad_\nu u}\in\gradSob{D}{\nu}$ such that
\begin{align} \label{eq:GeneralVariationProblem}
	\integral{D}{\grad_\nu u \cdot \grad\overline{\phi} - u\overline{\phi}}{\nu} &= 0 \quad \forall\phi\in\smooth{D}.
\end{align}
As $\grad_\nu u$ is unique for each $u$ it is sufficient to specify only the element $u$ to identify the pair $\bracs{u,\grad_\nu u}\in\gradSob{D}{\nu}$, so henceforth we adopt the shorthand $u\in\gradSob{D}{\nu}$ to refer to this element.
We also adopt a shorthand notation for the problem \eqref{eq:GeneralVariationProblem}, writing is as
\begin{align} \label{eq:GeneralDifferentialProblem}
	-\grad_\nu^2 u + u &= 0, \quad u\in\gradSob{D}{\nu}.
\end{align}
Of course this is just notation and has no meaning without \eqref{eq:GeneralVariationProblem}; one can envisage \eqref{eq:GeneralDifferentialProblem} as somehow being the weak form of \eqref{eq:GeneralVariationProblem} for analogy with classical Sobolev spaces and variational problems, however there is not formal link in this context (mainly because of the lack of the ``integration by parts" technique). \newline

In general for $f\in\ltwo{D}{\nu}$ and an elliptic matrix $A(x)$, we say that $u\in\gradSob{D}{\nu}$ is a solution to the (elliptic) equation
\begin{align} \label{eq:GeneralScalarStrongForm}
	-\grad_\nu \cdot \bracs{A(x)\grad_\nu u(x)} &= f(x), \quad x\in D
\end{align}
to mean that $u\in\gradSob{D}{\nu}$ solves the variational problem
\begin{align} \label{eq:GeneralScalarWeakForm}
	\integral{D}{A\grad_\nu u \cdot \grad\overline{\phi}}{\nu} &= \integral{D}{f\overline{\phi}}{\nu} \quad \forall \phi\in\smooth{D}.
\end{align}
Again it is critical to highlight that \eqref{eq:GeneralScalarWeakForm} is the only way for us to assign a meaning to the problem \eqref{eq:GeneralScalarStrongForm}.
However existence and uniqueness of the solution pair $\bracs{u,\grad_\nu u}$ is guaranteed by appealing to the Riesz Representation theorem and the bilinear form defined by \eqref{eq:GeneralScalarWeakForm}.
One can (see \cite{zhikov2000extension}) establish that the $\grad_\nu u$ in the solution pair coincides with the unique gradient of $u$ such that $A\grad_\nu u \perp \gradZero{D}{\nu}$; highlighting that understanding $\gradZero{D}{\nu}$ is crucial to determining solutions to \eqref{eq:GeneralScalarStrongForm}.
For completeness, we also note that the spectral problem (when $f$ is replaced by $\lambda u$ for some eigenvalue $\lambda\in\complex$) is written as
\begin{align*}
	-\grad_\nu \cdot \bracs{A(x)\grad_\nu u(x)} &= \lambda u(x), \quad x\in D
\end{align*}
to mean
\begin{align*}
	\integral{D}{A\grad_\nu u \cdot \grad\overline{\phi}}{\nu} &= \lambda\integral{D}{u\overline{\phi}}{\nu} \quad \forall \phi\in\smooth{D}.
\end{align*}
In which case solutions (if they exist) are triplets $\bracs{\lambda, u, \grad_\nu u}$, although it suffices to specify $\bracs{\lambda, u}$ only.

\section{An Example System} \label{sec:ScalarExample}
To compliment the theory outlined in section \ref{sec:ScalarSobSpaces}, and to highlight some of the considerations we shall be taking forward, we now provide an example and some analysis of it.
Let $\ddom\subset\reals^2$ be a bounded domain and $\graph = \bracs{V,E}$ be a finite graph embedded in $\ddom$. \tstk{may not need this clarification once the quantum graphs chapter is written up}
By this we associate each vertex $v_i\in V$ with a point $\vec{v}_i\in\reals^2$, and each edge $\bracs{v_i,v_j}\in E$ to the segment $I_{ij}=\sqbracs{\vec{v}_i,\vec{v}_j}$.
In a slight abuse of notation we drop the distinction between the vertices $v_i$ of $\graph$ and their associated points $\vec{v}_i\in\ddom$, and simply use the notation $v_i$ for both objects.
Similarly we adopt the notation $I_{ij}$ for both the edges of $\graph$ and their associated segments in $\ddom$.
Because of the embedding we can also consider $\graph$ as a subset of $\ddom$ itself, and so can make sense of expressions like $\graph\cap[0,1]^2$ by interpreting $\graph$ as the intersection of its segments $I_{ij}$. 
On $\ddom$ we define the measure $\ddmes$ as the measure which supports 1D Lebesgue measure on the edges of $\graph$, so for some $B\subset\ddom$ we have that 
\begin{align*}
	\ddmes\bracs{B} = \sum_{I_{ij}\in E}\lambda_{ij}\bracs{B \cap I_{ij}}
\end{align*}
where $\lambda_{ij}$ is the measure on $\reals^2$ that supports 1D Lebesgue measure down the edge $I_{ij}$. 
We are interested in the (spectral) problem
\begin{align} \label{eq:ScalarExampleStrongForm}
	-\grad_\ddmes \cdot \grad_\ddmes u &= \omega^2 u
\end{align}
for $u\in\gradSob{\ddom}{\ddmes}$ and eigenvalues ($\omega^2$).
\tstk{the provisos about considering the spectral problem, and the things we will have to consider in the vector case. This should include how we have already taken the Fourier transform of the 3D problem, and the Gelfand transform, and how this is the $k=0,\theta=0$ case - $\theta$ the quasi-momentum.}

To see how the setup just detailed relates to the wave propagation problems that we wish to consider in future, see figure \ref{fig:ScalarStrucDiagram}.
We interpret $\ddom$ as being the period cell of the cross-section of some waveguide that extends into 3D, whose waveguide axis is parallel to the $x_3$-axis.
The graph $\graph$ represents the (singular) structure within the period cell that is invariant down the waveguide, and the measure $\ddmes$ indicates that we are interested wave propagation on (the planes induced by) $\graph$ only\footnote{Later we will look to relate singular structure problems such as this to thin-structure problems, where the cross section has a graph-like structure but with edges of non-zero ``thickness".}.
\begin{figure}[h!]
	\centering
	\begin{tikzpicture}
		%actual waveguide
		\begin{scope}[shift={(5,-2)}]
			\draw (0,0) rectangle (2,2);
			\node[anchor=south] at (1,1) {$\ddom\subset\reals^2$};
			%waveguide axis
			\draw[red] (2,0) -- (2+3.5/2,1);
			\draw[dashed, red] (0,0) -- (3.5/2,1);
			\draw[red] (0,2) -- (3.5/2,3);
			\draw[red] (2,2) -- (2+3.5/2,3);
		\end{scope}		
		
		%axes labels
		\begin{scope}[shift={(5,2)}, scale=0.75]
			\draw[->] (0,1) -- (2,1) node[anchor=west] {$x_1$};
			\draw[->] (0,1) -- (0,3) node[anchor=south] {$x_2$};
			\draw[->] (0,1) -- (2,1+8/7) node[anchor=south west] {$x_3$};
		\end{scope}
		
		%graph stucture
		\begin{scope}[shift={(-3,-1.25)}]
			\draw (0,0) rectangle (5,5);
			%graph edges
			\draw (2.5,2.5) -- (4,3) -- (4.5,4);
			\draw (2.5,2.5) -- (0.5,4.5);
			\draw (2.5,2.5) -- (1,1) -- (3.5,1) -- (4,2) -- cycle;
			%nodes
			\filldraw (2.5,2.5) circle (1pt) node[anchor=south] {$v_i$};	
			\filldraw (4,3) circle (1pt);	\filldraw (4.5,4) circle (1pt);
			\filldraw (0.5,4.5) circle (1pt) node[anchor=north east] {$v_j$};
			\filldraw (1,1) circle (1pt);	\filldraw (3.5,1) circle (1pt);	\filldraw (4,2) circle (1pt);
			\node[anchor=south] at (2.5,5) {$\graph\subset\ddom$};
			%labelling for I_ij
			\node[anchor=north east] at (1.5,3.5) {$I_{ij}$};
			\draw[->] (2,3.2) -- (1.5,3.7) node[anchor=west] {$e_{ij}$};
		\end{scope}
		
		%lines between waveguide pic and graph illustration
		\draw[dashed, blue] (5,-2) -- (-3,-1.25);
		\draw[dashed, blue] (7,-2) -- (2,-1.25);
		\draw[dashed, blue] (7,0) -- (2,3.75);
		\draw[dashed, blue] (5,0) -- (-3,3.75);
	\end{tikzpicture}
	\caption{\label{fig:ScalarStrucDiagram} An illustration of the wider setting for $\ddom$ with singular structure. The 2D problem will arise from considering a structure that is periodic in the $\bracs{x_1,x_2}$-plane and translation invariant in the $x_3$ direction; and taking the (Fourier and) Gelfand transforms to arrive at a family of problems posed on the 2D-period (cross-section) period cell. \tstk{move tikz code into separate file so that diagram can be reused!}}
\end{figure}

\subsection{Analysis of $\gradZero{\ddom}{\ddmes}$}
We first pursue an understanding of $\gradZero{\ddom}{\ddmes}$, as this will be crucial for helping us reduce the problem \eqref{eq:ScalarExampleStrongForm} to something more manageable.
We begin by first considering what $\gradZero{\ddom}{\ddmes}$ looks like when $\graph$ consists of only a single edge $I$ parallel to the $x_1$-axis.
In this case $\ddmes = \lambda_I$ (Lebesgue measure down $I$) and it can be deduced that 
\begin{align*}
	\gradZero{\ddom}{\lambda_I} &= 
	\clbracs{
		\begin{pmatrix} 0 \\ f	\end{pmatrix}
		\ \vert \ f\in\ltwo{\ddom}{\ddmes}
	},
\end{align*}
see for example \cite{zhikov2000extension}.
Intuitively we can think of this result as follows \tstk{use the tikzpicture you created for this!}: the smooth gradient $\grad$ (from which $\grad_\ddmes$ is constructed) details the rate of change of the function it is applied to.
The measure $\ddmes$ however can only ``see" along (or alternatively ``only cares about") what's going on along the segment $I$, as this is it's entire support.
As such $\ddmes$ can only see the change along the segment $I$, and hence we find that $\gradZero{\ddom}{\ddmes}$ consists of all the gradients that are directed perpendicular to $I$.
Furthermore any gradient orthogonal to $\gradZero{\ddom}{\ddmes}$ is directed parallel to $I$, so $\grad_\ddmes$ can only ``see" rates of change parallel to the segment $I$.
This interpretation is further reinforced by the following analysis, which characterises $\gradZero{\ddom}{\ddmes}$ when the segment $I$ is not (necessarily) parallel to the $x_1$-axis. \newline

Consider the case when $\graph$ consists of a single segment $I\subset\ddom$ with orthogonal co-ordinate system $y=\bracs{y_1,y_2}$, with $y_1$ parallel to $I$.
Let $R$ be the orthogonal change of co-ordinates $x=Ry$ with $x=\bracs{x_1,x_2}$ the orthogonal co-ordinate system along the axes. \tstk{result on rotating gradients is kind of awkwardly placed, make it fit in better.}
\begin{lemma}[Gradients under Rotation]
	Let $D\subset\reals^n$, $\phi\in C^1\bracs{D}$, and let $x=\bracs{x_j}$, $y=\bracs{y_j}$ be two co-ordinate systems for $D$, related by $x=Ry$ for a rotation $R\in\mathrm{SO}\bracs{n}$.
	Let $\grad_x$ and $\grad_y$ denote the gradient operator in the $x$ and $y$ co-ordinate systems accordingly.
	Then
	\begin{align*}
		\grad_x u &= R \grad_y u, \\
		\grad_y u &= R^{\top} \grad_x u.
	\end{align*}
\end{lemma}
\begin{proof}
	We use index notation to save explicitly writing summation symbols;
	\begin{align*}
		\bracs{\grad_x u}_j &= \pdiff{u}{x_j} = \pdiff{u}{y_k}\pdiff{y_k}{x_j} \\
		&= R_{jk}\pdiff{u}{y_k} = \bracs{\grad_y u}_j.
	\end{align*}
\end{proof}
\tstk{check notation, and cross-reference with note-book - it's not all copy and paste from before!}
If $v\in\gradZero{\ddom}{\lambda_I}$ then there exists a sequence of functions $\phi_k\in\smooth{\ddom}$ such that $\phi_k\lconv{\ltwo{\ddom}{\lambda_I}}0$ and $\grad_y\phi_k\lconv{\ltwo{\ddom}{\lambda_I}^2} v \toInfty{k}$.
Define $\psi_k = \phi_k\bracs{R^{\top}x}$, and let $\lambda_{R^{\top}I}$ denote the measure obtained from the composition $\lambda_I\circ R^{\top}$ (note that this is just the measure that supports 1D Lebesgue measure down the rotated segment).
Then using the change of variables $x=Ry$, $\psi_k\lconv{\ltwo{\ddom}{\lambda_{R^{\top}I}}}0$, and
\begin{align*}
	\grad_x\psi_k &= \grad_x\phi_k\bracs{R^{\top}x} = R^{\top}\grad_y\phi_k\bracs{y}\big\vert_{y=R^{\top}x} \\
	&\lconv{\ltwo{\ddom}{\lambda_{R^{\top}I}}} R^{\top}v\bracs{R^{\top}x} =: w\bracs{x}.
\end{align*}
Now $w$ is a gradient of zero for a segment parallel to the $x_1$ axis, and hence has the form
\begin{align*}
	w &= \begin{pmatrix} 0 \\ f \end{pmatrix}, &\quad f\in\ltwo{\ddom}{\lambda_{R^{\top}I}} \\
	\Leftrightarrow v\bracs{R^{\top}x} &= R\begin{pmatrix} 0 \\ f \end{pmatrix} &\\
	\Leftrightarrow v\bracs{y} &= R\begin{pmatrix} 0 \\ \widetilde{f} \end{pmatrix}, &\quad \widetilde{f}\in\ltwo{\ddom}{\lambda_I}.
\end{align*}
This confirms that 
\begin{align*}
	\gradZero{\ddom}{\lambda_I} &= \clbracs{R\begin{pmatrix} 0 \\ \widetilde{f} \end{pmatrix} \ \vert \ \widetilde{f}\in\ltwo{\ddom}{\lambda_I}} \\
	&= \clbracs{g\in\ltwo{\ddom}{\lambda_I}^2 \ \vert \ g\cdot e_{I} = 0}
\end{align*}
where $e_I$ is the unit vector directed along the segment $I$.

%\subsubsection{Graphs with Multiple Edges}
%Of course the results so far only describe the set of gradients of zero for graphs with single edges, however we can use this understanding of single edges to build up an understanding of what happens to gradients of zero when we have a more interesting graph.
%In particular, we want to show that for a (multiple-edge) graph $\graph$ and $\ddmes$ supporting it's edges, we have that
%\begin{align*}
%	\gradZero{\ddom}{\ddmes} &= \clbracs{g\in\ltwo{\ddom}{\ddmes} \ \vert \ g\vert_{I_{ij}}\cdot e_{ij}=0 \ \forall I_{ij}\in E}.
%\end{align*}
%This provides an edge-wise characterisation of the set of gradients of zero; any gradient of zero for the whole graph has (upon restriction) the same form as a gradient of zero along each edge of the graph.
%Here $g\vert_{I_{ij}}$ denotes the restriction of the function $g$ to the edge $I_{ij}$, and $e_{ij}$ denotes the unit vector (directed $v_i$ to $v_j$) along $I_{ij}$.
%For convenience, we denote the set on the right hand side of the equality by $B$.
%The inclusion $\gradZero{\ddom}{\ddmes} \subset B$ follows quickly:
%\begin{prop} \label{prop:Grad0IncB}
%	For $B=\clbracs{g\in\ltwo{\ddom}{\ddmes} \ \vert \ g\vert_{I_{ij}}\cdot e_{ij}=0 \ \forall I_{ij}\in E}$, we have
%	\begin{align*}
%		\gradZero{\ddom}{\ddmes}\subset B
%	\end{align*}
%\end{prop}
%\begin{proof}
%	If $g\in\gradZero{\ddom}{\ddmes}$ then there exists a sequence of smooth functions $\phi_k$ such that $\phi_k\lconv{\ltwo{\ddom}{\ddmes}}0, \grad\phi_k\lconv{\ltwo{\ddom}{\ddmes}^2}g$.
%	Thus, due to the structure of $\ddmes$
%	\begin{align*}
%		\sum_{I_{ij}}\integral{I_{ij}}{\abs{\phi_k}^2}{\lambda_{I_{ij}}} &= \integral{\ddom}{\abs{\phi_k}^2}{\ddmes} \\
%		&\rightarrow 0 \toInfty{k}
%	\end{align*}
%	and as every term in the sum is non-negative, each term must also be tending to 0.
%	Thus 
%	\begin{align*}
%		\phi_k\lconv{\ltwo{\ddom}{\lambda_{I_{ij}}}}0 \ \forall I_{ij}\in E.
%	\end{align*}
%	Similarly
%	\begin{align*}
%		\sum_{I_{ij}}\integral{I_{ij}}{\abs{\grad\phi_k - g\vert_{I_{ij}}}^2}{\lambda_{I_{ij}}} &= \integral{\ddom}{\abs{\grad\phi_k - g}^2}{\ddmes} \\
%		&\rightarrow 0 \toInfty{k}
%	\end{align*}	
%	and hence 
%	\begin{align*}
%		\grad\phi_k\lconv{\ltwo{\ddom}{\lambda_{I_{ij}}}^2} g\vert_{I_{ij}} \ \forall I_{ij}\in E.
%	\end{align*}
%	Hence $g\in B$.
%\end{proof}
%
%The reverse inclusion requires some further preliminary results before it can be proven.
%\begin{lemma}[Extension of Gradients on Segments to Whole Graph] \label{lem:SegGradExtend}
%	Let $I_{ij}^{n} = \clbracs{x\in I_{ij} \ \vert \ \mathrm{dist}\bracs{x, \partial I_{ij}}\geq\recip{n}}$.
%	Suppose further that we have a function $g\in\ltwo{\ddom}{\ddmes}$ with $g=0$ on $\graph\setminus I_{ij}^{n}$ and $g\cdot e_{ij}=0$ on $I_{ij}^{n}$.
%	Then $g\in\gradZero{\ddom}{\ddmes}$.
%\end{lemma}
%\begin{proof}
%	As $g\cdot e_{ij}=0$ on $I_{ij}^{n}$ and $g=0$ on $I_{ij}\setminus I_{ij}^{n}$, we have that $g\cdot e_{ij}=0$ on $I_{ij}$ and hence $g\in\gradZero{\ddom}{\lambda_{I_{ij}}}$.
%	So we can find a sequence of smooth functions $\phi_k$ with the properties $\phi_k\lconv{\ltwo{\ddom}{\lambda_{I_{ij}}}}0$, $\grad\phi_k\lconv{\ltwo{\ddom}{\lambda_{I_{ij}}}^2}g\vert_{I_{ij}}$.
%	Now let $\chi_{ij}^{n}\in\smooth{\ddom}$ be the function such that
%	\begin{align*}
%		\chi_{ij}^{n}\bracs{x} &\in [0,1], \\
%		\chi_{ij}^{n} = 1 &\text{ whenever } \mathrm{dist}\bracs{x, I_{ij}^n}\leq \recip{3n} \\
%		\chi_{ij}^{n} = 0 &\text{ whenever } \mathrm{dist}\bracs{x, I_{ij}^n}\geq \frac{2}{3n}
%	\end{align*}
%	\begin{figure}
%		\centering
%		\begin{tikzpicture}
%			\begin{scope}[scale=2, rotate=-26.565]
%				\draw (0,0) -- (6,3);
%				\filldraw (0,0) circle (1pt) node[anchor=north] {$v_i$};
%				\filldraw (6,3) circle (1pt) node[anchor=south] {$v_j$};
%				\draw[red, line width=2pt] (1,0.5) -- (5,2.5);
%				\draw[dashed] (1-1/5,0.5+2/5) -- (5-1/5,2.5+2/5) to[out=26.565, in=26.565] (5+1/5,2.5-2/5) -- (1+1/5,0.5-2/5) to[out=180+26.565, in=180+26.565] cycle;
%				\draw[dashed] (1-2/5,0.5+4/5) -- (5-2/5,2.5+4/5) to[out=26.565, in=26.565] (5+2/5,2.5-4/5) -- (1+2/5,0.5-4/5) to[out=180+26.565, in=180+26.565] cycle;
%			\end{scope}
%			\node at (6,0.25) {$\chi_{ij}^{n}=1$};
%			\node at (6,1.25) {$0\leq\chi_{ij}^{n}\leq1$};
%			\node at (6,2.25) {$\chi_{ij}^{n}=0$};
%			\draw[->] (3,0) -- (3,-4/5);
%			\draw[->] (3,-4/5) -- (3,0);
%			\node[anchor=west] at (3,-2/5) {$\recip{3n}$};
%			\node[anchor=west] at (10,-6/5) {$\frac{2}{3n}$};
%			\draw[->] (10,0) -- (10,-9/5);
%			\draw[->] (10,-9/5) -- (10,0);
%			\draw[->] (13.416,-0.5) -- (11.18,-0.5);
%			\draw[->] (11.18,-0.5) -- (13.416,-0.5);
%			\node[anchor=north] at (12.5,-0.5) {$\recip{n}$};
%		\end{tikzpicture}
%		\caption{\label{fig:chiDiagram} The function $\chi_{ij}^n$ on the region surrounding the segment $I_{ij}^n$. Should another edge of $\graph$ lie in the region $\clbracs{x \ \vert \ \mathrm{dist}\bracs{x,I_{ij}^{n}}\leq\frac{2}{3n}}$, we can simply apply a rescaling to the argument of $\chi_{ij}^n$ to avoid this issue.}
%	\end{figure}
%	Note that since $\graph$ is finite, we can assume without loss of generality that the only edge of $\graph$ that lies in the support of $\chi_{ij}^n$ is $I_{ij}$, otherwise we apply a scaling to the argument of $\chi_{ij}^n$ to avoid this issue.
%	Furthermore, $\abs{\grad\chi_{ij}^{n}}$ is bounded by a constant that depends on $n$ only.
%	Now consider the sequence $\psi_k = \chi_{ij}^{n}\phi_k$.
%	We have that
%	\begin{align*}
%		\integral{\ddom}{\abs{\psi_k}^2}{\ddmes} = \integral{I_{ij}}{\abs{\chi_{ij}^{n}\phi_k}^2}{\lambda_{I_{ij}}}
%		\leq \integral{I_{ij}}{\abs{\phi_k}^2}{\lambda_{I_{ij}}} \rightarrow0 \toInfty{k},
%	\end{align*}
%	which is one of the desired convergence results for $\psi_k$.
%	For the other convergence result we need, observe that
%	\begin{align*}
%		\integral{\ddom}{\abs{\phi_k\grad\chi_{ij}^{n}}^2}{\ddmes} &= \integral{I_{ij}}{\abs{\phi_k\grad\chi_{ij}^{n}}^2}{\lambda_{I_{ij}}} \\
%		&\leq \sup_{I_{ij}}\bracs{\abs{\grad\chi_{ij}^{n}}^{2}}\integral{I_{ij}}{\abs{\phi_{k}}^{2}}{\ddmes} \\
%		&\rightarrow 0 \toInfty{k}
%	\end{align*}
%	because $\abs{\grad\chi_{ij}^{n}}$ depends on $n$ only.
%	Additionally
%	\begin{align*}
%		\integral{\ddom}{\abs{\chi_{ij}^n\grad\phi_k - g\vert_{I_{ij}}}^2}{\ddmes} &= \integral{I_{ij}}{\abs{\chi_{ij}^n\grad\phi_k - g\vert_{I_{ij}}}^2}{\lambda_{I_{ij}}} \\
%		&= \integral{I_{ij}\setminus I_{ij}^n}{\abs{\chi_{ij}^n\grad\phi_k - g\vert_{I_{ij}}}^2}{\lambda_{I_{ij}}} + \integral{I_{ij}^n}{\abs{\chi_{ij}^n\grad\phi_k - g\vert_{I_{ij}}}^2}{\lambda_{I_{ij}}} \\
%		&= \integral{I_{ij}\setminus I_{ij}^n}{\abs{\chi_{ij}^n\grad\phi_k}^2}{\lambda_{I_{ij}}} +  \integral{I_{ij}^n}{\abs{\grad\phi_k - g\vert_{I_{ij}}}^2}{\lambda_{I_{ij}}} \\
%		&\leq \integral{I_{ij}\setminus I_{ij}^n}{\abs{\grad\phi_k}^2}{\lambda_{I_{ij}}} +  \integral{I_{ij}^n}{\abs{\grad\phi_k - g\vert_{I_{ij}}}^2}{\lambda_{I_{ij}}} \\
%		&= \integral{I_{ij}\setminus I_{ij}^n}{\abs{\grad\phi_k - g\vert_{I_{ij}}}^2}{\lambda_{I_{ij}}} +  \integral{I_{ij}^n}{\abs{\grad\phi_k - g\vert_{I_{ij}}}^2}{\lambda_{I_{ij}}} \\
%		&= \integral{I_{ij}}{\abs{\grad\phi_k - g\vert_{I_{ij}}}^2}{\lambda_{I_{ij}}} \rightarrow0 \toInfty{k},
%	\end{align*}
%	where we have made use of the fact that $g\vert_{I_{ij}}=0$ on $I_{ij}\setminus I_{ij}^n$ and the various properties of $\chi_{ij}^n$.
%	Armed with these inequalities, we have that
%	\begin{align*}
%		\integral{\ddom}{\abs{\grad\psi_k - g\vert_{I_{ij}}}^2}{\ddmes} &= \integral{\ddom}{\abs{\chi_{ij}^n\grad\phi_k + \phi_k\grad\chi_{ij}^n - g\vert_{I_{ij}}}^2}{\ddmes} \\
%		&\leq 2\integral{\ddom}{\abs{\phi_k\grad\chi_{ij}^n}^2}{\ddmes} + 2\integral{\ddom}{\abs{\chi_{ij}^n\grad\phi_k - g\vert_{I_{ij}}}^2}{\ddmes} \\
%		&\rightarrow0 \toInfty{k}.
%	\end{align*}
%	Thus, $\psi_k$ is a sequence of smooth functions such that
%	\begin{align*}
%		\psi_k \lconv{\ltwo{\ddom}{\ddmes}} 0, &\quad
%		\grad\psi_k \lconv{\ltwo{\ddom}{\ddmes}^2} g
%	\end{align*}
%	and hence, $g\in\gradZero{\ddom}{\ddmes}$.
%\end{proof}
%
%The other result we need is a convergence result for a particular form of function we will utilise in the proof.
%Let the function $\eta\in\smooth{\ddom}$ with the properties
%\begin{align*}
%	\eta\bracs{x} &\in [0,1], \\
%	\eta = 0 &\text{ whenever } \abs{x}\leq 1, \\
%	\eta = 1 &\text{ whenever } \abs{x}\geq 2.
%\end{align*}
%Then for each $v_i\in V$ and $n\in\naturals$, we define
%\begin{align*}
%	\eta_i\bracs{x} = \eta\bracs{x-v_i}, &\quad \eta_i^n\bracs{x} = \eta_i\bracs{nx}
%\end{align*}
%which are clearly both smooth functions by composition.
%\begin{lemma}[Convergence of $\eta_i^n$ in $\ltwo{\ddom}{\ddmes}$] \label{lem:etaConv}
%	For any $v_i\in V$, $\eta_i^n \rightarrow 1$ in $\ltwo{\ddom}{\ddmes} \toInfty{n}$.
%\end{lemma}
%\begin{proof}
%	We can directly prove this convergence by estimating the integral from above:
%	\begin{align*}
%		\begin{split}
%			\integral{\ddom}{\abs{\eta_i^n-1}^2}{\ddmes} &= \integral{\graph\setminus B_{2/n}\bracs{v_{i}}}{\abs{\eta_{i}^{n}-1}^{2}}{\ddmes} + \integral{\graph \cap \bracs{B_{2/n}\bracs{v_{i}} \setminus B_{1/n}\bracs{v_{i}}}}{\abs{\eta_{i}^{n}-1}^{2}}{\ddmes} \\ + &\integral{\graph\cap B_{1/n}\bracs{v_{i}}}{\abs{\eta_{i}^{n}-1}^{2}}{\ddmes} \\
%			&= \integral{\graph\setminus B_{2/n}\bracs{v_{i}}}{0}{\ddmes} + \integral{\graph \cap \bracs{B_{2/n}\bracs{v_{i}} \setminus B_{1/n}\bracs{v_{i}}}}{\abs{\eta_{i}^{n}-1}^{2}}{\ddmes} \\ + &\integral{\graph\cap B_{1/n}\bracs{v_{i}}}{}{\ddmes} \\
%			&\leq \integral{\graph \cap \bracs{B_{2/n}\bracs{v_{i}} \setminus B_{1/n}\bracs{v_{i}}}}{}{\ddmes} + \integral{\graph\cap B_{1/n}\bracs{v_{i}}}{}{\ddmes} \\
%			&=\integral{\graph\cap B_{2/n}\bracs{v_{i}}}{}{\ddmes} = \ddmes\bracs{\graph\cap B_{2/n}\bracs{v_{i}}} \\
%			&\leq \frac{4 \abs{E}}{n} \rightarrow0 \toInfty{n}.
%		\end{split}
%	\end{align*}
%	The last line following because each edge of $\graph$ can intersect $B_{2/n}\bracs{v_i}$ on a segment of length $\leq\frac{4}{n}$.
%\end{proof}
%
%We are now ready to prove that $B\subset\gradZero{\ddom}{\ddmes}$.
%\begin{prop} \label{prop:BIncGrad0}
%	For $B=\clbracs{g\in\ltwo{\ddom}{\ddmes} \ \vert \ g\vert_{I_{ij}}\cdot e_{ij}=0 \ \forall I_{ij}\in E}$, we have
%	\begin{align*}
%		B \subset \gradZero{\ddom}{\ddmes}
%	\end{align*}
%\end{prop}
%\begin{proof}
%	Take $g\in B$, and define a family of functions $g_{n}$ by
%	\begin{align*}
%		g_{n}\bracs{x} &= \recip{2}\sum_{i\in V}\sum_{i\sim j}\eta_{i}^{n}\bracs{x}\eta_{j}^{n}\bracs{x}g\vert_{I_{ij}}\bracs{x}
%	\end{align*}
%	where the notation $i\sim j$ means that there is an edge $(i,j)\in E$, and the sum is taken over such edges.
%	Recall that $\graph$ was assumed finite, so there are no convergence issues with the double sum.
%	Then for each $i,j$ with $i\sim j$, the function $\eta_{i}^{n}\eta_{j}^{n}g\vert_{I_{ij}}$ satisfies the hypothesis of \ref{lem:SegGradExtend}, so $\eta_{i}^{n}\eta_{j}^{n}g\vert_{I_{ij}}\in\gradZero{\ddom}{\ddmes}$.
%	Furthermore, as $\gradZero{\ddom}{\ddmes}$ is a linear subspace of $\ltwo{\ddom}{\ddmes}^{2}$, $g_{n}\in\gradZero{\ddom}{\ddmes}$ too, for all $n$.
%	By closure of $\gradZero{\ddom}{\ddmes}$; $g_{n}$ converges in $\gradZero{\ddom}{\ddmes}$ provided it converges at all, so it remains to show that $g_{n}\lconv{\ltwo{\ddom}{\ddmes}^2} g \toInfty{n}$.
%	However with the result of \ref{lem:etaConv}, we have that $\eta_{i}^{n}\eta_{j}^{n}g\vert_{I_{ij}}\lconv{\ltwo{\ddom}{\ddmes}^2} g\vert_{I_{ij}}$ and hence
%	\begin{align*}
%		g_{n} \lconv{\ltwo{\ddom}{\ddmes}^2} &\recip{2}\sum_{i\in V}\sum_{j\sim i}g\vert_{I_{ij}} = g \toInfty{n},
%	\end{align*}
%	so $g\in\gradZero{\ddom}{\ddmes}$.
%\end{proof}
%
%\subsection{Reduction to a system of scalar equations} \label{sec:ScalarReduceEdgeEqns}
%We now turn our attention back to \eqref{eq:ScalarExampleStrongForm}.
%Recall that we interpret this in the weak sense, namely a pair $\bracs{u,\grad_\ddmes u}$ solves \eqref{eq:ScalarExampleStrongForm} if and only if
%\begin{align} \label{eq:ScalarExampleWeakForm}
%	\integral{\ddom}{\grad_\ddmes u \cdot \grad\phi}{\ddmes} &= \omega^2\integral{\ddom}{u\phi}{\ddmes} \quad \forall\phi\in\smooth{\ddom}.
%\end{align}
%\tstk{again reiterate Gelfand/Fourier and the case we are considering. Also how can we guarantee solutions to this system?}
%We want to find an alternative but equivalent system to \eqref{eq:ScalarExampleWeakForm} that we can use in numerical schemes (as the measure $\ddmes$ prevents implementations via finite elements, for example) or is more tractable to an analytic approach.
%To this end we look to work from \eqref{eq:ScalarExampleWeakForm} and employ our previous analysis to find an alternative problem. \newline
%
%We begin by re-writing \eqref{eq:ScalarExampleWeakForm} by appealing to the nature of $\ddmes$;
%\begin{align} \label{eq:ScalarExampleSumOfInts}
%	0 &= \recip{2}\sum_{I_{ij}\in E}\integral{\ddom}{\grad_\ddmes u \cdot \grad\phi - \omega^2 u\phi}{\lambda_{ij}}.
%\end{align}
%Note that $I_{ij},I_{ji}\in E$ are the same edge of the graph $\graph$, which is what gives us the factor of a half.
%Of course we may simply drop this factor of a half, and understand the sum as being over unique edges, which we henceforth do.
%Next recall the notation introduced for the edge $I_{ij}$; we denote by $e_{ij}$ the unit vector along $I_{ij}$ pointing from $v_i$ to $v_j$,
%\begin{align*}
%	e_{ij} = \frac{v_j-v_i}{\norm{v_j-v_i}_2}.
%\end{align*}
%We define the map 
%\begin{align*}
%	r_{ij}:\sqbracs{0,\abs{I_{ij}}} &\rightarrow I_{ij}, \\
%	r_{ij}\bracs{t} &= v_i + te_{ij}
%\end{align*}
%so $r'\bracs{t}=e_{ij}$.
%Note that $r$ is therefore an admissible change of variables from $\sqbracs{0,\abs{I_{ij}}}$ to $I_{ij}$.
%To prevent cluttered notation, we use an overhead tilde to denote composition with $r_{ij}$, so for example $\tilde{\phi}=\phi\circ r_{ij}$.
%We next look at how to deal with each integral term.
%Without loss of generality we can assume that each $I_{ij}$ is parallel to the $x_1$-axis by performing a rotation, and we denote by $u_{ij}\in\ltwo{\ddom}{\lambda_{ij}}$ the function such that $u_{ij}=u$ on the (potentially rotated) edge $I_{ij}$.
%$u_{ij}$ can be thought of as the $ij$\textsuperscript{th}-part of $u$, or the part of $u$ down the edge $I_{ij}$ - this separation of $u$ onto separate functions on the edges of $\graph$ will prove useful in what follows.
%Although we do not need to make this simplification since we have information about $\gradZero{\ddom}{\lambda_{ij}}$, it is helpful to do so in the following analysis.
%In fact the information about $\gradZero{\ddom}{\lambda_{ij}}$ is necessary to perform the rotation we would use, and retain equivalence of the resulting problem to the original. \newline
%
%With this simplification, we examine the term
%\begin{align*}
%	\integral{\ddom}{\grad_\ddmes u_{ij}\cdot\grad\phi}{\lambda_{ij}}.
%\end{align*}
%We know that the solution to \eqref{eq:ScalarExampleWeakForm} satisfies $\grad_\ddmes u \perp \gradZero{\ddom}{\ddmes}$ and so $\grad_\ddmes u_{ij} \perp \gradZero{\ddom}{\lambda_{ij}}$ by our analysis of gradients of zero.
%Thus let us suppose that $\grad_\ddmes u_{ij} = \bracs{\alpha, \beta}^{\top}$.
%Recalling the characterisation of $\gradZero{\ddom}{\lambda_{ij}}$ when $I_{ij}$ is parallel to the $x_1$-axis, we require that
%\begin{align*}
%	0 &= \integral{\ddom}{\grad_\ddmes u_{ij} \cdot \begin{pmatrix} 0 \\ f \end{pmatrix}}{\lambda_{ij}} \quad \forall f\in\ltwo{\ddom}{\lambda_{ij}}, \\
%	\Rightarrow 0 &= \integral{I_{ij}}{\beta f}{\lambda_{ij}}
%	= \int_{0}^{\abs{I_{ij}}}\widetilde{\beta}\widetilde{f}\md t.
%\end{align*}
%As $\widetilde{f}$ is arbitrary, we conclude that $\widetilde{\beta}=0$ and hence $\beta=0$.
%Thus $\grad_\ddmes u_{ij} = \bracs{\alpha, 0}^{\top}$; we now show that $\alpha$ is related to the distributional derivative of the function $\widetilde{u}_{ij}$.
%Notice that for any smooth function $\phi$ we have that
%\begin{align*} \labelthis\label{eq:PartialToTDiff}
%	\partial_t\widetilde{\phi}(t) &= \grad\phi\bracs{r_{ij}(t)}\cdot r'_{ij}(t) = \partial_1\phi\bracs{r_{ij}(t)} \\
%	&= \widetilde{\partial_1\phi}(t),
%\end{align*}
%as $r'_{ij}=e_{ij}=\bracs{1,0}^{\top}$.
%Now as $u_{ij}\in\ltwo{\ddom}{\lambda_{ij}}$ we can find smooth functions $\alpha_n$ such that
%\begin{align*}
%	\alpha_n\lconv{\ltwo{\ddom}{\lambda_{ij}}}u_{ij}, \quad \grad\alpha_n\lconv{\ltwo{\ddom}{\lambda_{ij}}^2}\begin{pmatrix} \alpha \\ 0	\end{pmatrix}.
%\end{align*}
%Exploiting the change of variables $r_{ij}$ and writing these statements using integrals converging in $\reals$ implies
%\begin{align*}
%	\int_0^{\abs{I_{ij}}}\abs{\widetilde{\alpha}-\widetilde{u}_{ij}}^2 \md t &\rightarrow 0, \\
%	\int_0^{\abs{I_{ij}}}\abs{\partial_t\widetilde{\alpha}_n - \widetilde{\alpha}}^2 \md t = \int_0^{\abs{I_{ij}}}\abs{\partial_1\widetilde{\alpha}_n - \widetilde{\alpha}}^2 \md t &\rightarrow 0, \\
%	\int_0^{\abs{I_{ij}}}\abs{\partial_2\widetilde{\alpha}_n}^2 \md t &\rightarrow 0, \toInfty{n}.
%\end{align*}
%In particular the first two convergences imply that $\widetilde{\alpha}=\widetilde{u}'_{ij}$; that is that $\widetilde{\alpha}$ is the distributional derivative of $\widetilde{u}_{ij}\in\gradSob{\sqbracs{0,\abs{I_{ij}}}}{t}$.
%Using the notation $u'_{ij}$ for the $\ltwo{\ddom}{\lambda_{ij}}$ function such that $u'_{ij} = \widetilde{u}'_{ij}\circ r_{ij}^{-1}$, we can write $\grad_\ddmes u_{ij} = \bracs{u'_{ij}, 0}^{\top}$.
%In light of this we see that
%\begin{align*}
%	\integral{\ddom}{\grad_\ddmes u_{ij} \cdot \grad\phi - \omega^2 u_{ij}\phi}{\lambda_{ij}}
%	&= \int_0^{\abs{I_{ij}}}\widetilde{u}'_{ij}\widetilde{\phi}' - \omega^2\widetilde{u}_{ij}\widetilde{\phi} \md t.
%\end{align*}
%
%Equipped with this knowledge, we are able to transform \eqref{eq:ScalarExampleSumOfInts} into
%\begin{align} \label{eq:ScalarExampleLebesgueIntegrals}
%	0 &= \sum_{I_{ij}\in E}\int_0^{\abs{I_{ij}}}\widetilde{u}'_{ij}\widetilde{\phi}' - \omega^2\widetilde{u}_{ij}\widetilde{\phi} \md t.
%\end{align}
%The choice of $\phi$ was arbitrary so in particular \eqref{eq:ScalarExampleLebesgueIntegrals} must hold when $\mathrm{supp}\bracs{\phi}$ contains precisely one edge $I_{ij}$, meaning that we have
%\begin{align*}
%	0 &= \int_0^{\abs{I_{ij}}}\widetilde{u}'_{ij}\widetilde{\phi}' - \omega^2\widetilde{u}_{ij}\widetilde{\phi} \md t, \quad \forall I_{ij}\in E, \ \forall \phi\in\smooth{\ddom}.
%\end{align*}
%That is to say, we have arrived at a system of differential equations (``edge equations") for $\widetilde{u}_{ij}$ (the ``edge solutions").
%This weak formulation admits numerical solution by (for example) finite elements, providing us with a method to recover the original function $u$.
%If we assume more regularly of $u$, for example assume that $u\in C^2\bracs{\ddom}$ so the second derivative of $u$ exists in the strong sense we can obtain a system of ODEs in the edge solutions $\widetilde{u}_{ij}$, complete with boundary conditions.
%We obtain one (second-order) differential equation for each edge $I_{ij}$ by integrating each edge equation by parts;
%\begin{align*}
%	0 &= \int_0^{\abs{I_{ij}}}\widetilde{u}'_{ij}\widetilde{\phi}' - \omega^2\widetilde{u}_{ij}\widetilde{\phi} \md t, \quad \forall\phi\in\smooth{\ddom} \\
%	&= \int_0^{\abs{I_{ij}}} \bracs{\widetilde{u}''_{ij} + \omega^2\widetilde{u}_{ij}}\widetilde{\phi} \md t. \\
%	\Rightarrow 0 &= \widetilde{u}''_{ij} + \omega^2\widetilde{u}_{ij}.
%\end{align*}
%As we have assumed at least $C^2$ regularity of $u$ we require that $u$ be continuous at the vertices $v_i$ and so we must have that
%\begin{align*}
%	\widetilde{u}_{ij}\vert_{v_i} &= \widetilde{u}_{ik}\vert_{v_i} \quad \text{whenever } i\sim j \text{ and } i\sim k,
%\end{align*}
%which is simply a matching condition of the solution at the vertices.
%The vertical bar notation is used to denote evaluation, so $\widetilde{u}_{ij}\vert_{v_i}$ denotes evaluation of $\widetilde{u}_{ij}\bracs{t}$ at whichever $t$ such that $r_{ij}\bracs{t} = v_i$. 
%The other boundary conditions can be found by considering $\phi$ such that $\mathrm{supp}\bracs{\phi}$ contains precisely one of the vertices $v_k$, and then integrating \eqref{eq:ScalarExampleLebesgueIntegrals} by parts and considering the boundary term:
%\begin{align*}
%	0 &= \sum_{I_{ij}\in E}\int_0^{\abs{I_{ij}}}\widetilde{u}'_{ij}\widetilde{\phi}' - \omega^2\widetilde{u}_{ij}\widetilde{\phi} \md t \\
%	&= \sum_{I_{ij}\in E}\int_0^{\abs{I_{ij}}} \bracs{\widetilde{u}''_{ij} + \omega^2\widetilde{u}_{ij}}\widetilde{\phi} \md t + \sum_{I_{ij}\in E}\sum_{i\sim j}\sqbracs{\widetilde{u}'_{ij}\widetilde{\phi}}\big\vert_{v_i} \\
%	&= \sum_{I_{ij}\in E}\sum_{i\sim j}\sqbracs{\widetilde{u}'_{ij}\widetilde{\phi}}\big\vert_{v_i} \\
%	&= \sum_{k\sim j}\widetilde{u}'_{kj}\vert_{v_k}\widetilde{\phi}\vert_{v_k}, \\
%	\Rightarrow 0 &= \sum_{k\sim j}\widetilde{u}'_{kj}\vert_{v_k}.
%\end{align*}
%That is we obtain a Kirchoff condition on the derivatives of the edge solutions at each of the vertices.
%These conditions provide us with a complete system of differential equations and boundary conditions; observe that for each edge equation we can select one of it's vertices and associate one matching condition and one Kirchoff condition at that vertex.
%
%\section{Chapter Summary}
%The work in this section largely builds off that of \tstk{references!!}, before pursuing an example that will be relevant in the later sections.
%We have introduced the theory behind the kinds of mathematical problems and their associated spaces.
%One constructs the Sobolev spaces $\gradSob{D}{\nu}$ using closure of the set of $\smooth{D}$ functions and their gradients, the cost of which is uniqueness of the $\nu$-gradients, $\grad_\nu$.
%However one can characterise all the gradients of a given function $u\in\gradSob{D}{\nu}$ by finding the unique $\grad_\nu u$ which is perpendicular to the set of $\nu$-gradients of 0, $\gradZero{D}{\nu}$.
%As such understanding the structure of $\gradZero{D}{\nu}$ is central to finding solutions to problems posed in $\gradSob{D}{\nu}$. \newline
%
%The example provided in section \ref{sec:ScalarExample} demonstrates how this theory can be employed to reduce an abstract problem involving $\nu$-gradients to an equivalent problem which involves more familiar (or ``classical") objects.
%Starting from the problem \eqref{eq:ScalarExampleStrongForm}, one can obtain the system \eqref{eq:ScalarExampleLebesgueIntegrals} which can be approached numerically via schemes like finite elements, and the solution to the original problem can be recovered.
%Under further regularly assumptions, one can even obtain systems of ODEs which may prove admissible to an analytic approach and even yield exact solutions. \newline
%
%We will be looking to extend the concepts introduced here when we move towards considering systems of vector-valued functions, which are widely used in the description of physical systems.
%Many of the argumentative techniques that we have employed in this section will provide inspiration for the arguments we employ in chapter \ref{ch:3VectorEqns}.
