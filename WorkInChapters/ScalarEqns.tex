\chapter{Scalar Equations} \label{ch:ScalarEqns}
In this chapter we look at scalar wave equations in our waveguide-like geometry, treating the cross-sectional structure as singular.
We base our approach largely off the work of Zhikov \cite{zhikov2000extension}, in that we consider what can be colloquially described as ``differential equations with respect to a measure $\ddmes$".
We will give a rigorous definition of the kinds of problems and spaces that this colloquialism covers in the following sections, and will demonstrate how this framework allows us to obtain a quantum graph problem from our variational formulation.
This will allow us to make use of the theory which was highlighted in \ref{ch:QuantumGraphs}, and additionally will make our variational problems open to numerical solution.
At the conclusion of this chapter we will have highlighted the techniques that are available to us, and how we shall adapt them for the more descriptive (or physical) vector systems we consider in chapter \ref{ch:VectorEqns}.

\section{The Scalar Sobolev Spaces} \label{sec:ScalarSobSpaces}
In this section we look to construct the function spaces that we shall be working with throughout the chapter, as well as discussing some of the consequences of this construction.
What we present is a synopsis of the work of Zhikov presented in \cite{zhikov2000extension}, although with an adapted notation which will suit our needs when we later make a link to Quantum Graph problems. \newline

Let $n\in\naturals$, $D\subset\reals^n$ and $\nu$ be a (Borel) measure on $D$.
Denote the set of smooth functions on $D$ by $\smooth{D}$, and then let $W=W\bracs{D,\mathrm{d}\nu}$ be the closure of the set of pairs $\bracs{\phi,\grad\phi}$ in $\ltwo{D}{\nu}\times\ltwo{D}{\nu}^n$, where $\phi\in\smooth{D}$.
That is
\begin{align*}
	W = W\bracs{D,\nu} &= \overline{\clbracs{\bracs{\phi,\grad\phi} \ \vert \ \phi\in\smooth{D}}} \quad \text{in} \ \ltwo{D}{\nu}\times\ltwo{D}{\nu}^n.
\end{align*}
The notation above is chosen to provide similarities to the $W$-style construction of classical Sobolev spaces (in which integration is performed with respect to the $n$-dimensional Lebesgue measure).
An element of $W$ is a pair $\bracs{u,z}$; and as the above construction suggests we would like to associate the function $z$ with a distributional derivative of sorts, however there is a problem with making this association.
Suppose that $\bracs{u,z}\in W$ and $\bracs{0,y}\in W$, then we also have $\bracs{u,z+y}\in W$ too by taking an approximating sequence for each pair and forming another sequence by point-wise addition of terms.
As a result each $u\in\ltwo{D}{\nu}$ has multiple (distinct) functions $z\in\ltwo{D}{\nu}^n$ such that $\bracs{u,z}\in W$; and so as it stands it does not make sense to associate ``\textit{a} gradient" to $u$, because $u$ has multiple candidates for this role.
In what follows we will look into this quirk, and deduce that we can make sense of the idea of ``\textit{the} gradient". \newline

Let use denote the set of ``gradients of zero" by $\gradZero{D}{\nu}$, that is
\begin{align*}
	\gradZero{D}{\nu} &= \clbracs{z\in\ltwo{D}{\nu}^n \ \vert \ \bracs{0,z}\in W}, \\
		&= \clbracs{z\in\ltwo{D}{\nu}^n \ \vert \ \exists\phi_n\in\smooth{D} \text{ such that } \phi_n\lconv{\ltwo{D}{\nu}}0, \grad\phi_n\lconv{\ltwo{D}{\nu}^n}z}, \labelthis\label{eq:GradZeroSequenceDefinition}
\end{align*}
the construction of $W$ ensuring the two sets coincide.
A fact that follows immediately from this definition and will be used later is that $\gradZero{D}{\nu}$ is a closed linear subspace of $\ltwo{D}{\nu}^n$.
The illustration of the non-uniqueness of gradients earlier employed the fact that we can always add an element of $\gradZero{D}{\nu}$ to the second member of a pair $\bracs{u,z}\in W$ and produce another element of $W$.
Colloquially this can be expressed by the following statement; we can always ``add a gradient of zero" to an existing ``gradient", which will produce another ``gradient".
This hints at the possibility that whilst we any $u$ may not have a unique gradient, it might at least have only one relevant gradient in the context of a mathematical problem.
With this in mind, consider the following variational problem; find a pair $\bracs{u,z}\in W$ such that
\begin{align} \label{eq:TangentialGradientVariationalMotivation}
	\integral{D}{z \cdot \grad\overline{\phi} - u\overline{\phi}}{\nu} &= 0 \quad \forall\phi\in\smooth{D}.
\end{align}
Setting aside questions of existence and uniqueness of solutions to this problem for the purposes of illustration, take an element $g\in\gradZero{D}{\nu}$, and an approximating sequence $\phi_n$ as in the definition \eqref{eq:GradZeroSequenceDefinition}.
Substituting $\phi_n$ into \eqref{eq:TangentialGradientVariationalMotivation} and employing a limit-argument, we can see that
\begin{align*}
	\integral{D}{z \cdot g}{\nu} &= 0 \quad \forall g\in\gradZero{D}{\nu}.
\end{align*}
Namely that the member $z$ of the solution pair is orthogonal (in the $\ltwo{D}{\nu}$-norm) to $\gradZero{D}{\nu}$.
Stepping back for a moment, as $\gradZero{D}{\nu}$ is a closed linear subspace of $\ltwo{D}{\nu}^n$ we can decompose $\ltwo{D}{\nu}^n$ as
\begin{align*}
	\ltwo{D}{\nu}^n &= \gradZero{D}{\nu}^{\perp} \oplus \gradZero{D}{\nu}.
\end{align*}
But we have just seen that the member of the solution pair $z\in\gradZero{D}{\nu}^{\perp}$, hence $z$ is the \textit{unique} element of $\gradZero{D}{\nu}^{\perp}$ such that every pair $\bracs{u,\tilde{z}}\in W$ can be written as $\bracs{u,z+g}\in W$ for some $g\in\gradZero{D}{\nu}$.
Thus whilst the concept of a unique gradient does not make sense in this setting, once we consider variational problems we can obtain the concept of the \textit{tangential} gradient $z$, which is unique.
We therefore adopt the notation $z := \grad_\nu u$ for the second member of the solution pair $\bracs{u,z}\in W$, and can now define the (non-classical) ``Sobolev space"
\begin{align*}
	\gradSob{D}{\nu} &:= \clbracs{\bracs{u,\grad_\nu u}\in W \ \vert \ \grad_\nu u \perp \gradZero{D}{\nu}}.
\end{align*}
Furthermore we can now pose variational problems on this space; for example we would write \eqref{eq:TangentialGradientVariationalMotivation} as find $\bracs{u,\grad_\nu u}\in\gradSob{D}{\nu}$ such that
\begin{align} \label{eq:GeneralVariationProblem}
	\integral{D}{\grad_\nu u \cdot \grad\overline{\phi} - u\overline{\phi}}{\nu} &= 0 \quad \forall\phi\in\smooth{D}.
\end{align}
As $\grad_\nu u$ is unique for each $u$ it is sufficient to specify only the element $u$ to identify the pair $\bracs{u,\grad_\nu u}\in\gradSob{D}{\nu}$, so henceforth we adopt the shorthand $u\in\gradSob{D}{\nu}$ to refer to this element.
We also adopt a shorthand notation for the problem \eqref{eq:GeneralVariationProblem}, writing is as
\begin{align} \label{eq:GeneralDifferentialProblem}
	-\grad_\nu^2 u + u &= 0, \quad u\in\gradSob{D}{\nu}.
\end{align}
Of course this is just notation and has no meaning without \eqref{eq:GeneralVariationProblem}; one can envisage \eqref{eq:GeneralDifferentialProblem} as somehow being the weak form of \eqref{eq:GeneralVariationProblem} for analogy with classical Sobolev spaces and variational problems, however there is not formal link in this context (mainly because of the lack of the ``integration by parts" technique). \newline

In general for $f\in\ltwo{D}{\nu}$ and an elliptic matrix $A(x)$, we say that $u\in\gradSob{D}{\nu}$ is a solution to the (elliptic) equation
\begin{align} \label{eq:GeneralScalarStrongForm}
	-\grad_\nu \cdot \bracs{A(x)\grad_\nu u(x)} &= f(x), \quad x\in D
\end{align}
to mean that $u\in\gradSob{D}{\nu}$ solves the variational problem
\begin{align} \label{eq:GeneralScalarWeakForm}
	\integral{D}{A\grad_\nu u \cdot \grad\overline{\phi}}{\nu} &= \integral{D}{f\overline{\phi}}{\nu} \quad \forall \phi\in\smooth{D}.
\end{align}
Again it is critical to highlight that \eqref{eq:GeneralScalarWeakForm} is the only way for us to assign a meaning to the problem \eqref{eq:GeneralScalarStrongForm}.
However existence and uniqueness of the solution pair $\bracs{u,\grad_\nu u}$ is guaranteed by appealing to the Riesz Representation theorem and the bilinear form defined by \eqref{eq:GeneralScalarWeakForm}.
One can (see \cite{zhikov2000extension}) establish that the $\grad_\nu u$ in the solution pair coincides with the unique gradient of $u$ such that $A\grad_\nu u \perp \gradZero{D}{\nu}$; highlighting that understanding $\gradZero{D}{\nu}$ is crucial to determining solutions to \eqref{eq:GeneralScalarStrongForm}.
For completeness, we also note that the spectral problem (when $f$ is replaced by $\lambda u$ for some eigenvalue $\lambda\in\complex$) is written as
\begin{align*}
	-\grad_\nu \cdot \bracs{A(x)\grad_\nu u(x)} &= \lambda u(x), \quad x\in D
\end{align*}
to mean
\begin{align*}
	\integral{D}{A\grad_\nu u \cdot \grad\overline{\phi}}{\nu} &= \lambda\integral{D}{u\overline{\phi}}{\nu} \quad \forall \phi\in\smooth{D}.
\end{align*}
In which case solutions (if they exist) are triplets $\bracs{\lambda, u, \grad_\nu u}$, although it suffices to specify $\bracs{\lambda, u}$ only.

\section{An Example System} \label{sec:ScalarExample}
To compliment the theory outlined in section \ref{sec:ScalarSobSpaces}, and to highlight some of the considerations we shall be taking forward, we now provide an example and some analysis of it.
Let $\ddom\subset\reals^2$ be a bounded domain and $\graph = \bracs{V,E}$ be a finite graph embedded in $\ddom$. \tstk{may not need this clarification once the quantum graphs chapter is written up}
By this we associate each vertex $v_j\in V$ with a point $\vec{v}_j\in\reals^2$, and each edge $\bracs{v_j,v_k}\in E$ to the segment $I_{jk}=\sqbracs{\vec{v}_j,\vec{v}_k}$.
In a slight abuse of notation we drop the distinction between the vertices $v_j$ of $\graph$ and their associated points $\vec{v}_j\in\ddom$, and simply use the notation $v_j$ for both objects.
Similarly we adopt the notation $I_{jk}$ for both the edges of $\graph$ and their associated segments in $\ddom$.
Because of the embedding we can also consider $\graph$ as a subset of $\ddom$ itself, and so can make sense of expressions like $\graph\cap[0,1]^2$ by interpreting $\graph$ as the intersection of its segments $I_{jk}$. 
On $\ddom$ we define the measure $\ddmes$ as the measure which supports 1D Lebesgue measure on the edges of $\graph$, so for some $B\subset\ddom$ we have that 
\begin{align*}
	\ddmes\bracs{B} = \sum_{I_{ij}\in E}\lambda_{ij}\bracs{B \cap I_{ij}}
\end{align*}
where $\lambda_{ij}$ is the measure on $\reals^2$ that supports 1D Lebesgue measure down the edge $I_{ij}$. 
We are interested in the (spectral) problem
\begin{align} \label{eq:ScalarExampleStrongForm}
	-\grad_\ddmes \cdot \grad_\ddmes u &= \omega^2 u
\end{align}
for $u\in\gradSob{\ddom}{\ddmes}$ and eigenvalues ($\omega^2$).

Before we begin our analysis we highlight how the setup just detailed relates to the wave propagation problems that we wish to consider in future, see figure \ref{fig:ScalarStrucDiagram}.
We interpret $\ddom$ as being the period cell of the cross-section of some waveguide that extends into 3D, whose waveguide axis is parallel to the $x_3$-axis.
The graph $\graph$ represents the (singular) structure within the period cell that is invariant down the waveguide, and the measure $\ddmes$ indicates that we are interested wave propagation on (the planes induced by) $\graph$ only.
In section \ref{sec:ScalarSystem} we will consider a similar system to this but with the addition a quasi-momentum parameter $\qm$ which reflects the fact that to obtain a problem on the period cell of the cross section ($\ddom$), we are required to apply a Gelfand problem to the full-space problem.
We will elaborate further on this in the relevant section.
\begin{figure}[h!]
	\centering
	\includegraphics[scale=1.0]{../Diagrams/Diagram_GraphInCrossSection.pdf}
	\caption{\label{fig:ScalarStrucDiagram} An illustration of the wider setting for $\ddom$ with singular structure. The 2D problem will arise from considering a structure that is periodic in the $\bracs{x_1,x_2}$-plane and translation invariant in the $x_3$ direction; and taking the (Fourier and) Gelfand transforms to arrive at a family of problems posed on the 2D-period (cross-section) period cell.}
\end{figure}

\subsection{Analysis of $\gradZero{\ddom}{\ddmes}$}
We first pursue an understanding of $\gradZero{\ddom}{\ddmes}$, as this will be crucial for helping us reduce the problem \eqref{eq:ScalarExampleStrongForm} to something more manageable.
Most of the arguments presented here are presented (albeit in a much more brief format) in \cite{zhikov2000extension}, however we aim to give a fuller explanation as we will later be using these arguments to motivate our work in chapter \ref{ch:VectorEqns}.
We begin by first considering what $\gradZero{\ddom}{\ddmes}$ looks like when $\graph$ consists of only a single edge $I$ parallel to the $x_1$-axis.

\begin{prop}[Gradients of Zero on a Segment Parallel to the $x_1$-axis] \label{prop:GradZeroParallelZhikov}
	Let $I$ be a segment in the $\bracs{x_1,x_2}$-plane parallel to the $x_1$-axis, and let $\lambda_I$ be the singular measure supported on $I$.
	Then 
	\begin{align*}
		\gradZero{\ddom}{\lambda_I} &= 
		\clbracs{
			\begin{pmatrix} 0 \\ f	\end{pmatrix}
			\ \vert \ f\in\ltwo{\ddom}{\lambda_I}
		}.
	\end{align*}
\end{prop}
\begin{proof}
	This result is quoted and illustrated in \cite{zhikov2000extension}.
	However as we will be wanting to utilise the ideas in this proof in our later work, w provide a fuller argument here. \newline

	Without loss of generality we assume $x_2=0$ on $I$, otherwise we apply a translation to the argument that follows.
	Additionally it suffices to show that the set on the right hand side includes all functions of the specified form when $f$ is smooth, as we can then apply a density argument to deduce the result for all $f\in\ltwo{\ddom}{\lambda_I}$. \newline
	
	So take some $f\in\smooth{\ddom}$, then the ``constant sequence" $\phi_n = \phi = x_2 f$ is such that
	\begin{align*}
		\integral{\ddom}{\abs{\phi}^2}{\lambda_I} &= \integral{I}{x_2^2\abs{f}^2}{\lambda_I} \\
		&= 0 \text{ as } x_2=0 \text{ on } I.
	\end{align*}
	Furthermore
	\begin{align*}
		\integral{\ddom}{\abs{\grad\phi - \bracs{0, f}^\top }^2}{\lambda_I}
		&= \integral{I}{\abs{ x_2\grad f + \bracs{0, f}^\top - \bracs{0,f}^\top }^2}{\lambda_I} \\
		&= \integral{I}{\abs{ x_2\grad f }^2}{\lambda_I} \\
		&= 0.
	\end{align*}
	Thus we have that $\phi_n\rightarrow0$ and $\grad\phi_n\rightarrow\bracs{0,f}^\top \toInfty{n}$, and so $\bracs{0,f}^\top\in\gradZero{\ddom}{\lambda_I}$. \newline
	
	We now prove that if $\bracs{f,0}^\top\in\gradZero{\ddom}{\lambda_I}$ then $f=0$.
	So suppose $\bracs{f,0}\in\gradZero{\ddom}{\lambda_I}$ and take an approximating sequence $\phi_n$ as in \eqref{eq:GradZeroSequenceDefinition}.
	Then as $\grad\phi_n\rightarrow\bracs{f,0}^\top$ in $\ltwo{\ddom}{\lambda_I}^2$ and
	\begin{align*}
		\integral{\ddom}{\abs{\grad\phi_n - \bracs{f,0}^\top}^2}{\lambda_I}
		&= \integral{I}{\abs{\partial_1\phi_n - f}^2}{\lambda_I} + \integral{I}{\abs{\partial_2\phi_n}^2}{\lambda_I},
	\end{align*}
	we have in particular that
	\begin{align*}
		\integral{I}{\abs{\partial_1\phi_n - f}^2}{\lambda_I} \rightarrow 0.
	\end{align*}
	We now perform a change of variables via the map $r:\interval{I}\rightarrow I$, via $r(t) = v_I + te_I$; where $v_I$ is one of the endpoints of $I$ and $e_I$ is the unit vector pointing away from this endpoint along the segment $I$.
	In particular we note that as $I$ is parallel to the $x_1$-axis that $e_I=e_1$, the canonical unit vector in the $x_1$-direction.
	Setting $\tilde{\phi}_n(t) := \phi_n\bracs{r(t)}$, we have by the chain rule $\diff{\tilde{\phi}_n}{t} = \partial_1\phi_n\bracs{r(t)}$, and so
	\begin{align*}
		0 &\leftarrow \integral{I}{\abs{\partial_1\phi_n - f}^2}{\lambda_I} 
		= \int_0^{\abs{I}}\abs{\diff{\tilde{\phi}_n}{t} - \tilde{f}}^2 \md t.
	\end{align*}
	Here we have set $\tilde{f} = f \circ r$.
	We can also see that
	\begin{align*}
		0 &\leftarrow \integral{I}{\abs{\phi_n}^2}{\lambda_I} 
		= \int_0^{\abs{I}} \abs{\tilde{\phi}^2} \md t.
	\end{align*}
	Thus $\tilde{\phi}_n\rightarrow 0$ and $\diff{\tilde{\phi}_n}{t}\rightarrow \tilde{f}$ in $\ltwo{\interval{I}}{t}$, hence $\tilde{f}$ is the distributional derivative (in the $\gradSob{\interval{I}}{t}$ sense) of the zero function.
	But in this classical Sobolev space we can conclude that $\tilde{f} = 0$, and thus $f = 0$, as we sought.
\end{proof}

We can interpret this result as follows, and have provided an illustration in figure \ref{fig:GradZeroEdge}.
The smooth gradient $\grad$ (from which $\grad_\ddmes$ is constructed) details the rate of change of the function it is applied to.
\begin{figure}[b]
	\centering
	\includegraphics[scale=1.0]{../Diagrams/Diagram_GradZeroEdge.pdf}
	\caption{\label{fig:GradZeroEdge} Illustration of how to interpret a ``gradient of zero" on an edge $I_{jk}$. As the singular measure $\lambda_{jk}$ only sees the component of the (smooth) gradient parallel to $I_{jk}$, anything perpendicular has no contribution.}
\end{figure}
The measure $\ddmes$ however can only ``see" along (or alternatively ``only cares about") what's going on along the segment $I$, as this is it's entire support.
As such $\ddmes$ can only see the change along the segment $I$, and hence we find that $\gradZero{\ddom}{\ddmes}$ consists of all the gradients that are directed perpendicular to $I$.
Furthermore any gradient orthogonal to $\gradZero{\ddom}{\ddmes}$ is directed parallel to $I$, so $\grad_\ddmes$ can only ``see" rates of change parallel to the segment $I$.
This interpretation is further reinforced by the following analysis, which characterises $\gradZero{\ddom}{\ddmes}$ when the segment $I$ is not (necessarily) parallel to the $x_1$-axis. \newline

\tstk{result on rotating gradients is kind of awkwardly placed, make it fit in better.}
\begin{lemma}[Gradients under Rotation] \label{lem:SmoothGradientsUnderRotation}
	Let $D\subset\reals^n$, $\phi\in C^1\bracs{D}$, and let $x=\bracs{x_j}$, $y=\bracs{y_j}$ be two co-ordinate systems for $D$, related by $x=Ry$ for a rotation $R\in\mathrm{SO}\bracs{n}$.
	Let $\grad_x$ and $\grad_y$ denote the gradient operator in the $x$ and $y$ co-ordinate systems accordingly.
	Then
	\begin{align*}
		\grad_x u &= R \grad_y u, \\
		\grad_y u &= R^{\top} \grad_x u.
	\end{align*}
\end{lemma}
\begin{proof}
	We use index notation to save explicitly writing summation symbols;
	\begin{align*}
		\bracs{\grad_x u}_j &= \pdiff{u}{x_j} = \pdiff{u}{y_k}\pdiff{y_k}{x_j} \\
		&= R_{jk}\pdiff{u}{y_k} = \bracs{\grad_y u}_j.
	\end{align*}
\end{proof}

\begin{prop}[Rotation of Edge Gradients] \label{prop:RotationOfEdgeGradients}
	Consider the case when $\graph$ consists of a single segment $I\subset\ddom$ with orthogonal co-ordinate system $y=\bracs{y_1,y_2}$, with $y_1$ parallel to $I$.
	Let $R$ be the orthogonal change of co-ordinates $x=Ry$ with $x=\bracs{x_1,x_2}$ the orthogonal co-ordinate system along the axes.
	Then
	\begin{align*}
		\gradZero{\ddom}{\lambda_I} 
		&= \clbracs{ R^{\top} \begin{pmatrix} 0 \\ f_2 \end{pmatrix} \ \vert \ f_2\in\ltwo{\ddom}{\lambda_I} }.
	\end{align*}
\end{prop}
\begin{proof}
	The proof centres around a the change of variables $x=Ry$, and each of the set inclusions that need to be shown adhere to the same style of argument - almost identical in fact save for the direction of the argument.
	As such, we only show one of the inclusions, as the argument for the other is so similar. \newline
	
	Set $\tilde{I} := RI$, and $\lambda_{\tilde{I}}\bracs{B} := \lambda_{I}\bracs{R^{\top}B}$; and we also adopt a subscript on the gradient operator to denote the co-ordinate system we are working in, see the notation in lemma \ref{lem:SmoothGradientsUnderRotation}.
	It is also clear that the rotated segment $\tilde{I}$ is parallel to the $x_1$-axis, so we know the form of $\gradZero{\ddom}{\lambda_{\tilde{I}}}$ by proposition \ref{prop:GradZeroParallelZhikov}. \newline
	
	Take any $z\in\gradZero{\ddom}{\lambda_I}$.
	Then we can find an approximating sequence $\phi_n\in\smooth{\ddom}$ as in \eqref{eq:GradZeroSequenceDefinition}, so we have that (as $n\rightarrow\infty$)
	\begin{align*}
		\integral{I}{\abs{\phi_n}^2}{\lambda_I} \rightarrow 0,
		&\quad \integral{I}{\abs{\grad_y\phi_n - z}^2}{\lambda_I} \rightarrow 0
	\end{align*}
	Performing the change of variables $x=Ry$ in the integrals implies
	\begin{align*}
		\integral{\tilde{I}}{\abs{\tilde{\phi}_n}^2}{\lambda_{\tilde{I}}} \rightarrow 0,
		&\quad \integral{\tilde{I}}{\abs{R^{\top}\grad_x\tilde{\phi}_n - \tilde{z}}^2}{\lambda_{\tilde{I}}} \rightarrow 0,
	\end{align*}
	where $\tilde{\phi}_n\bracs{x} = \phi_n{R^{\top}x}$ and $\tilde{z}\bracs{x} = z\bracs{R^{\top}x}$.
	The first convergence implies that $\tilde{\phi}_n\rightarrow 0$ in $\ltwo{\ddom}{\lambda_{\tilde{I}}}$.
	As for the second, some manipulations in the integral show us that
	\begin{align*}
		\integral{\tilde{I}}{\abs{\grad_x\tilde{\phi}_n - R\tilde{z}}^2}{\lambda_{\tilde{I}}} \rightarrow 0,
	\end{align*}
	so $\grad_x\tilde{\phi}_n\rightarrow\tilde{z}$ in $\ltwo{\ddom}{\lambda_{\tilde{I}}}^2$.
	Thus we conclude that $R\tilde{z}\in\gradZero{\ddom}{\lambda_{\tilde{I}}}$, so by proposition \ref{prop:GradZeroParallelZhikov}
	\begin{align*}
		R\tilde{z} &= \begin{pmatrix} 0 \\ \tilde{f}_2 \end{pmatrix}
	\end{align*}
	for some $\tilde{f}_2\in\ltwo{\ddom}{\lambda_{\tilde{I}}}$.
	Thus
	\begin{align*}
		z &= R^\top \begin{pmatrix} 0 \\ f_2 \end{pmatrix}
	\end{align*}
	for $f_2\in\ltwo{\ddom}{\lambda_I}$ (where $f_2\bracs{y} = \tilde{f}_2\bracs{Ry}$), and hence
	\begin{align*}
		\gradZero{\ddom}{\lambda_I} 
		&\subset \clbracs{ R^{\top} \begin{pmatrix} 0 \\ f_2 \end{pmatrix} \ \vert \ f_2\in\ltwo{\ddom}{\lambda_I} }.
	\end{align*}
	As mentioned earlier, the reverse inclusion is similar.
\end{proof}
We can colloquially write this result as
\begin{align*}
	\gradZero{\ddom}{\lambda_I} &= R^\top\gradZero{\ddom}{\lambda_{\tilde{I}}},
\end{align*}
or more formally as in the following corollary.
\begin{cory} \label{cory:Grad0SingleEdge}
	Assume the hypothesis of proposition \ref{prop:RotationOfEdgeGradients}, and denote by $e_I$ the unit vector parallel to the segment $I$.
	Then
	\begin{align*}
		\gradZero{\ddom}{\lambda_I} &= \clbracs{z\in\ltwo{\ddom}{\lambda_I} \ \vert \ z\vert_{I}\cdot e_I = 0}.
	\end{align*}
\end{cory}

\subsubsection{Graphs with Multiple Edges}
So far we have only described gradients of zero on single segments, that is when we have a one-edge graph embedded in the $\bracs{x_1,x_2}$-plane.
However we can build up an understanding of gradients of zero on more complex (finite) graphs using these results.
As such in this section we work with an embedded graph $\graph = \bracs{V,E}$ with supporting singular measure $\ddmes$; and we look to show that gradients of zero on the whole graph admit an edge-wise form that coincides with proposition \ref{prop:RotationOfEdgeGradients}.
Namely we wish to prove that
\begin{align*}
	\gradZero{\ddom}{\ddmes} &= \clbracs{g\in\ltwo{\ddom}{\ddmes} \ \vert \ g\vert_{I_{jk}}\cdot e_{jk}=0 \ \forall I_{jk}\in E} \\
	&= \clbracs{g\in\ltwo{\ddom}{\ddmes} \ \vert \ g\in\ltwo{\ddom}{\lambda_{jk}}^2 \ \forall I_{jk}\in E}.
\end{align*}
Here $e_{jk}$ denotes the unit vector (directed $v_j$ to $v_k$) along $I_{jk}$; and for convenience we denote the set on the right hand side of the equality by $B$.
The inclusion $\gradZero{\ddom}{\ddmes} \subset B$ follows quickly:

\begin{prop} \label{prop:Grad0IncB}
	For $B = \clbracs{g\in\ltwo{\ddom}{\ddmes} \ \vert \ g\vert_{I_{jk}}\cdot e_{jk}=0 \ \forall I_{jk}\in E}$, we have
	\begin{align*}
		\gradZero{\ddom}{\ddmes} \subset B
	\end{align*}
\end{prop}
\begin{proof}
	If $g\in\gradZero{\ddom}{\ddmes}$ then there exists a sequence of smooth functions $\phi_n$ such that $\phi_n\lconv{\ltwo{\ddom}{\ddmes}}0, \grad\phi_n\lconv{\ltwo{\ddom}{\ddmes}^2}g$.
	Thus due to the structure of $\ddmes$
	\begin{align*}
		\sum_{I_{jk}}\integral{I_{jk}}{\abs{\phi_n}^2}{\lambda_{jk}} &= \integral{\ddom}{\abs{\phi_n}^2}{\ddmes} \\
		&\rightarrow 0 \toInfty{n}.
	\end{align*}
	As every term in the sum is non-negative, each term must also be tending to 0.
	Thus 
	\begin{align*}
		\phi_n\lconv{\ltwo{\ddom}{\lambda_{jk}}}0 \ \forall I_{jk}\in E.
	\end{align*}
	Similarly
	\begin{align*}
		\sum_{I_{jk}}\integral{I_{jk}}{\abs{\grad\phi_n - g}^2}{\lambda_{jk}} &= \integral{\ddom}{\abs{\grad\phi_n - g}^2}{\ddmes} \\
		&\rightarrow 0 \toInfty{n},
	\end{align*}	
	and hence 
	\begin{align*}
		\grad\phi_n\lconv{\ltwo{\ddom}{\lambda_{jk}}^2} g \ \forall I_{jk}\in E.
	\end{align*}
	Hence $g\in B$.
\end{proof}

The reverse inclusion requires some preliminary results before it can be proven.
The first result sees us demonstrate that a gradient of zero on a (slightly shortened) edge of a graph is also a gradient of zero on the whole graph, if we extend it by zeros outside the original edge.

\begin{lemma}[Extension Lemma for Gradients] \label{lem:SegGradExtend}
	Let $I_{jk}^{n} = \clbracs{x\in I_{jk} \ \vert \ \mathrm{dist}\bracs{x, \partial I_{jk}}\geq\recip{n}}$.
	Suppose further that we have a function $g\in\ltwo{\ddom}{\ddmes}$ with $g=0$ on $\graph\setminus I_{jk}^{n}$ and $g\cdot e_{jk}=0$ on $I_{jk}^{n}$.
	Then 
	\begin{align*}
		g\in\gradZero{\ddom}{\ddmes}.
	\end{align*}
\end{lemma}
\begin{proof}
	As $g\cdot e_{jk}=0$ on $I_{jk}^{n}$ and $g=0$ on $I_{jk}\setminus I_{jk}^{n}$, we have that $g\cdot e_{jk}=0$ on $I_{jk}$ and hence $g\in\gradZero{\ddom}{\lambda_{jk}}$.
	So we can find a sequence of smooth functions $\phi_l$ as in \eqref{eq:GradZeroSequenceDefinition}.
	Now let $\chi_{jk}^{n}\in\smooth{\ddom}$ be the function such that
	\begin{align*}
		\chi_{jk}^{n}\bracs{x} &\in [0,1], \\
		\chi_{jk}^{n} = 1 &\text{ whenever } \mathrm{dist}\bracs{x, I_{jk}^n}\leq \recip{3n} \\
		\chi_{jk}^{n} = 0 &\text{ whenever } \mathrm{dist}\bracs{x, I_{jk}^n}\geq \frac{2}{3n}
	\end{align*}
	\begin{figure}
		\centering
		\includegraphics[scale=1.0]{../Diagrams/Diagram_ChiFunction.pdf}
		\caption{\label{fig:chiDiagram} The function $\chi_{jk}^n$ on the region surrounding the segment $I_{jk}^n$. Should another edge of $\graph$ lie in the region $\clbracs{x \ \vert \ \mathrm{dist}\bracs{x,I_{jk}^{n}}\leq\frac{2}{3n}}$, we can simply apply a rescaling to the argument of $\chi_{jk}^n$ to avoid this issue.}
	\end{figure}
	Note that since $\graph$ is finite, we can assume without loss of generality that the only edge of $\graph$ that lies in the support of $\chi_{jk}^n$ is $I_{jk}$, otherwise we apply a scaling to the argument of $\chi_{jk}^n$ to avoid this issue.
	Furthermore, $\abs{\grad\chi_{jk}^{n}}$ is bounded by a constant that depends on $n$ only.
	Now consider the sequence $\psi_l = \chi_{jk}^{n}\phi_l$.
	We have that
	\begin{align*}
		\integral{\ddom}{\abs{\psi_l}^2}{\ddmes} = \integral{I_{jk}}{\abs{\chi_{jk}^{n}\phi_l}^2}{\lambda_{jk}}
		\leq \integral{I_{jk}}{\abs{\phi_l}^2}{\lambda_{jk}} \rightarrow0 \toInfty{l},
	\end{align*}
	which is one of the desired convergence results for $\psi_l$.
	For the other convergence result we need, observe that
	\begin{align*}
		\integral{\ddom}{\abs{\phi_l\grad\chi_{jk}^{n}}^2}{\ddmes} &= \integral{I_{jk}}{\abs{\phi_l\grad\chi_{jk}^{n}}^2}{\lambda_{jk}} \\
		&\leq \sup_{I_{jk}}\bracs{\abs{\grad\chi_{jk}^{n}}^{2}}\integral{I_{jk}}{\abs{\phi_l}^2}{\ddmes} \\
		&\rightarrow 0 \toInfty{l}
	\end{align*}
	because $\abs{\grad\chi_{jk}^{n}}$ depends on $n$ only.
	Additionally
	\begin{align*}
		\integral{\ddom}{\abs{\chi_{jk}^n\grad\phi_l - g}^2}{\ddmes} &= \integral{I_{jk}}{\abs{\chi_{jk}^n\grad\phi_l - g}^2}{\lambda_{jk}} \\
		&= \integral{I_{jk}\setminus I_{jk}^n}{\abs{\chi_{jk}^n\grad\phi_l - g}^2}{\lambda_{jk}} + \integral{I_{jk}^n}{\abs{\chi_{jk}^n\grad\phi_l - g}^2}{\lambda_{jk}} \\
		&= \integral{I_{jk}\setminus I_{jk}^n}{\abs{\chi_{jk}^n\grad\phi_l}^2}{\lambda_{jk}} +  \integral{I_{jk}^n}{\abs{\grad\phi_l - g}^2}{\lambda_{jk}} \\
		&\leq \integral{I_{jk}\setminus I_{jk}^n}{\abs{\grad\phi_l}^2}{\lambda_{jk}} +  \integral{I_{jk}^n}{\abs{\grad\phi_l - g}^2}{\lambda_{jk}} \\
		&= \integral{I_{jk}\setminus I_{jk}^n}{\abs{\grad\phi_l - g}^2}{\lambda_{jk}} +  \integral{I_{jk}^n}{\abs{\grad\phi_l - g}^2}{\lambda_{jk}} \\
		&= \integral{I_{jk}}{\abs{\grad\phi_l - g}^2}{\lambda_{jk}} \rightarrow0 \toInfty{l},
	\end{align*}
	where we have made use of the fact that $g=0$ on $\graph\setminus I_{jk}^n$ and the various properties of $\chi_{jk}^n$.
	Armed with these inequalities, we have that
	\begin{align*}
		\integral{\ddom}{\abs{\grad\psi_l - g}^2}{\ddmes} 
		&= \integral{\ddom}{\abs{\chi_{jk}^n\grad\phi_l + \phi_l\grad\chi_{jk}^n - g}^2}{\ddmes} \\
		&\leq 2\integral{\ddom}{\abs{\phi_l\grad\chi_{jk}^n}^2}{\ddmes} + 2\integral{\ddom}{\abs{\chi_{jk}^n\grad\phi_l - g}^2}{\ddmes} \\
		&\rightarrow0 \toInfty{l}.
	\end{align*}
	Thus, $\psi_l$ is a sequence of smooth functions such that
	\begin{align*}
		\psi_l \lconv{\ltwo{\ddom}{\ddmes}} 0, &\quad
		\grad\psi_l \lconv{\ltwo{\ddom}{\ddmes}^2} g
	\end{align*}
	and hence, $g\in\gradZero{\ddom}{\ddmes}$.
\end{proof}

The other result we need is a convergence result for a particular form of function we will utilise later.
Let $\eta\in\smooth{\ddom}$ be the function with the properties
\begin{align*}
	\eta\bracs{x} &\in [0,1], \\
	\eta = 0 &\text{ whenever } \abs{x}\leq 1, \\
	\eta = 1 &\text{ whenever } \abs{x}\geq 2.
\end{align*}
Then for each $v_j\in V$ and $n\in\naturals$, we define
\begin{align*}
	\eta_j\bracs{x} = \eta\bracs{x-v_j}, &\quad \eta_j^n\bracs{x} = \eta_j\bracs{nx}
\end{align*}
which are clearly both smooth functions by composition.

\begin{lemma}[Convergence of $\eta_j^n$ in $\ltwo{\ddom}{\ddmes}$] \label{lem:etaConv}
	For any $v_j\in V$, 
	\begin{align*}
		\eta_j^n \rightarrow 1 \text{ in } \ltwo{\ddom}{\ddmes} \toInfty{n}.
	\end{align*}
\end{lemma}
\begin{proof}
	We can directly prove this convergence by estimating the integral from above:
	\begin{align*}
		\begin{split}
			\integral{\ddom}{\abs{\eta_j^n-1}^2}{\ddmes} &= \integral{\graph\setminus B_{2/n}\bracs{v_j}}{\abs{\eta_j^n-1}^2}{\ddmes} + \integral{\graph \cap \bracs{B_{2/n}\bracs{v_j} \setminus B_{1/n}\bracs{v_j}}}{\abs{\eta_j^n-1}^2}{\ddmes} \\ + &\integral{\graph\cap B_{1/n}\bracs{v_j}}{\abs{\eta_j^n-1}^2}{\ddmes} \\
			&= \integral{\graph\setminus B_{2/n}\bracs{v_j}}{0}{\ddmes} + \integral{\graph \cap \bracs{B_{2/n}\bracs{v_j} \setminus B_{1/n}\bracs{v_j}}}{\abs{\eta_j^n-1}^2}{\ddmes} \\ + &\integral{\graph\cap B_{1/n}\bracs{v_j}}{}{\ddmes} \\
			&\leq \integral{\graph \cap \bracs{B_{2/n}\bracs{v_j} \setminus B_{1/n}\bracs{v_j}}}{}{\ddmes} + \integral{\graph\cap B_{1/n}\bracs{v_j}}{}{\ddmes} \\
			&=\integral{\graph\cap B_{2/n}\bracs{v_j}}{}{\ddmes} = \ddmes\bracs{\graph\cap B_{2/n}\bracs{v_j}} \\
			&\leq \frac{4 \abs{E}}{n} \rightarrow0 \toInfty{n}.
		\end{split}
	\end{align*}
	The last line following because each edge of $\graph$ can intersect $B_{2/n}\bracs{v_j}$ on a segment of length at most $\frac{4}{n}$.
\end{proof}

We are now ready to prove that $B\subset\gradZero{\ddom}{\ddmes}$.
The idea for the proof is to use the fact that we can extend a gradient of zero on each edge by zero to obtain a gradient of zero on the whole graph; and then consider the sum of such functions in $\gradZero{\ddom}{\ddmes}$ to obtain the result we need.

\begin{prop} \label{prop:BIncGrad0}
	For $B = \clbracs{g\in\ltwo{\ddom}{\ddmes} \ \vert \ g\vert_{I_{jk}}\cdot e_{jk}=0 \ \forall I_{jk}\in E}$, we have
	\begin{align*}
		B \subset \gradZero{\ddom}{\ddmes}.
	\end{align*}
\end{prop}
\begin{proof}
	Take $g\in B$, and define a family of functions $g_n$ by
	\begin{align*}
		g_n\bracs{x} &= \recip{2}\sum_{j\in V}\sum_{j\sim k}\eta_j^n\bracs{x}\eta_k^n\bracs{x}g\vert_{I_{jk}}\bracs{x}
	\end{align*}
	where the notation $j\sim k$ means that there is an edge $(j,k)\in E$, and the sum is taken over such edges. \tstk{may not be needed if explained in QG chapter.}
	Recall that $\graph$ is assumed finite so there are no convergence issues with the double sum.
	Then for each $j,k$ with $j\sim k$, the function $\eta_j^n\eta_k^n g\vert_{I_{jk}}$ satisfies the hypothesis of lemma \ref{lem:SegGradExtend}, so $\eta_j^n\eta_k^n g\vert_{I_{jk}}\in\gradZero{\ddom}{\ddmes}$.
	Furthermore, as $\gradZero{\ddom}{\ddmes}$ is a linear subspace of $\ltwo{\ddom}{\ddmes}^{2}$, $g_n\in\gradZero{\ddom}{\ddmes}$ too, $\forall n\in\naturals$.
	By closure of $\gradZero{\ddom}{\ddmes}$ in $\ltwo{\ddom}{\ddmes}^2$; $g_n$ converges in $\gradZero{\ddom}{\ddmes}$ provided it converges at all, and so it remains to show that $g_n\lconv{\ltwo{\ddom}{\ddmes}^2} g \toInfty{n}$.
	However with the result of lemma \ref{lem:etaConv}, we have that $\eta_j^n\eta_k^n g\vert_{I_{jk}}\lconv{\ltwo{\ddom}{\ddmes}^2} g\vert_{I_{jk}}$ and hence
	\begin{align*}
		g_n \lconv{\ltwo{\ddom}{\ddmes}^2} &\recip{2}\sum_{j\in V}\sum_{j\sim k}g\vert_{I_{jk}} = g \toInfty{n},
	\end{align*}
	so $g\in\gradZero{\ddom}{\ddmes}$.
\end{proof}

This concludes our analysis of $\gradZero{\ddom}{\ddmes}$, as we now have an edge-wise characterisation for the functions in this set, as well as a form for the functions on each edge.
In the next section we consider the family of problems that arise from considering the operator that defines \eqref{eq:ScalarExampleStrongForm} in the infinite, periodic $\bracs{x_1,x_2}$-plane.

\section{Example in the Periodic $\bracs{x_1,x_2}$-plane} \label{sec:ScalarSystem}
In this section we will analyse a problem posed in an infinite plane with a finite period cell.
Physically this will represent our photonic fibre's cross sectional structure \tstk{earlier section discussion?}, which we extend to an infinite lattice so that we can employ tools like the Gelfand transform.
This will require us to study slightly altered versions of the Sobolev spaces introduced in section \ref{sec:ScalarSobSpaces}, as the Gelfand transform will break the problem in the infinite plane into a family of problems on the period cell, parametrised by 2 quasi-momenta.
We will then demonstrate how the analysis of section \ref{sec:ScalarExample} is employed to reduce the resulting problems on the period cell to a Quantum Graph problem, which can then be analysed or solved using the methods detailed in chapter \ref{ch:QuantumGraphs}. \newline

Embed a periodic graph into the $\bracs{x_1,x_2}$-plane, with period cell $\ddom = \sqbracs{0,1}^2$ and ``period graph" $\graph=(V,E)$.
Let \tstk{got to here!!!!!!!}
Consider the problem
\begin{align} \label{eq:WholeSpaceScalarProblem}
	\grad_\ddmes \cdot \grad_\ddmes u &= \omega^2 u	
\end{align}

%\section{Chapter Summary}
%The work in this section largely builds off that of \tstk{references!!}, before pursuing an example that will be relevant in the later sections.
%We have introduced the theory behind the kinds of mathematical problems and their associated spaces.
%One constructs the Sobolev spaces $\gradSob{D}{\nu}$ using closure of the set of $\smooth{D}$ functions and their gradients, the cost of which is uniqueness of the $\nu$-gradients, $\grad_\nu$.
%However one can characterise all the gradients of a given function $u\in\gradSob{D}{\nu}$ by finding the unique $\grad_\nu u$ which is perpendicular to the set of $\nu$-gradients of 0, $\gradZero{D}{\nu}$.
%As such understanding the structure of $\gradZero{D}{\nu}$ is central to finding solutions to problems posed in $\gradSob{D}{\nu}$. \newline
%
%The example provided in section \ref{sec:ScalarExample} demonstrates how this theory can be employed to reduce an abstract problem involving $\nu$-gradients to an equivalent problem which involves more familiar (or ``classical") objects.
%Starting from the problem \eqref{eq:ScalarExampleStrongForm}, one can obtain the system \eqref{eq:ScalarExampleLebesgueIntegrals} which can be approached numerically via schemes like finite elements, and the solution to the original problem can be recovered.
%Under further regularly assumptions, one can even obtain systems of ODEs which may prove admissible to an analytic approach and even yield exact solutions. \newline
%
%We will be looking to extend the concepts introduced here when we move towards considering systems of vector-valued functions, which are widely used in the description of physical systems.
%Many of the argumentative techniques that we have employed in this section will provide inspiration for the arguments we employ in chapter \ref{ch:3VectorEqns}.
