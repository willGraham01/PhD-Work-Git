\chapter{Quantum Graphs} \label{ch:QuantumGraphs}
In this chapter we shall introduce the concept of Quantum Graphs, their associated function spaces, and operators defined on these spaces.
Heavy references throughout to EKK, Kuchment, I guess Olaf \& Post too maybe?

\section{Notation, Definitions, and Conventions} \label{sec:QG-Notation}
\begin{definition}[Graph] \label{def:Graph}
	Let $\Lambda\subset\naturals$ and $V=\clbracs{v_j \ \vert \ j\in\Lambda}$ be a set of labels $v_j$ bijective to $\Lambda$ via the map $j\rightarrow v_j$.
	Let $E\subset V\times V$ be a set of \textit{unordered} pairs $\bracs{v_j,v_k}\in E$ where $j,k\in\Lambda$.
	If $\bracs{v_j,v_k}\in E$ then write $I_{jk} = \bracs{v_j,v_k}$, noting that $I_{jk}=I_{kj}$.
	Then $\graph=\bracs{V,E}$ is a \textit{graph} with vertex set $V$ and edge set $E$.
	Elements of the set $V$ are called \textit{vertices} of the graph $\graph$ and elements of $E$ are referred to as \textit{edges} of $\graph$. \newline
	We say $\graph$ is a \textit{finite graph} if $\Lambda$ (and consequently $V$ and $E$) are finite, and similarly say $\graph$ is an \textit{infinite graph} if $\Lambda$ is infinite.
\end{definition}
Definition \ref{def:Graph} is not as general as others that can be found in graph theory, however we work with this definition because it is sufficient for our purposes.
We highlight some features this definition imposes on our graphs that may be seen as ``non-standard" or ``restrictive" below.
\begin{itemize}
	\item Definition \ref{def:Graph} uses a notation that suggests that there is at most a single edge $I_{jk}$ connecting any pair of vertices (and similarly at most one loop $I_{jj}$), and in the definitions that follow this will come across as implying there can only be one edge with the same direction between two vertices.
	Contrary to this implication, we do not rule out the possibility of there being more than one (directed) edge between two vertices, even though it is suggested by the notation $I_{jk}$.
	To account for this one can introduce an additional superscript, along the lines of $I_{jk}^{l}$, for this possibility and then define certain equivalence relations on the sets $V$ and $E$ to make the notation we introduce later consistent.
	However we elect to save on the notational (and mathematical) load by not including this superscript simply because will not be interested in many systems that require this functionality, and where it is required the meaning of the notation we later introduce is implicit either way.
	Allowing for this possibility also doesn't change any of the results presented here, and if we do need to consider a graph with multiple edges between two vertices we will be explicit in how we are handling the specific situation.
	\item The labels $v_j$ are slightly unnecessary, as one can just work directly with the index set $\Lambda$.
	However we will be wanting labels for our vertices and edges when we come to consider quantum and embedded graphs, so we include them in this definition for consistency.
	\item It is also worth mentioning that although we have not stated it in definition \ref{def:Graph}, we will be assuming throughout this report that all of our graphs are connected.
	That is, for any pair of vertices $v_j, v_k\in V$ there exists a finite set of vertices $v_{n_m}, m=1,...,N\in\naturals$ such that $I_{jn_1}, I_{n_1 n_2}, ... , I_{n_m k}\in E$.
	In the case of our graphs being directed (as is the case with the remaining definitions that follow) for the purposes of connectedness we ignore any direction that is assigned to an edge. 
\end{itemize}

Having laid out this basis for the concept of a graph, we can now present some further definitions.
\begin{definition}[Directed Graph] \label{def:DirectedGraph}
	Let $\Lambda\subset\naturals$ and $V=\clbracs{v_j \ \vert \ j\in\Lambda}$ be a set of labels $v_j$ bijective to $\Lambda$ via the map $j\rightarrow v_j$.
	Let $E\subset V\times V$ be a set of \textit{ordered} pairs $\bracs{v_j,v_k}\in E$ where $j,k\in\Lambda$.
	Write $I_{jk} = \bracs{v_j,v_k}$ for the elements of $E$.
	Then $\graph=\bracs{V,E}$ is a \textit{directed graph} with vertex set $V$ and edge set $E$.
	Elements of the set $V$ are called vertices of the graph $\graph$ and elements of $E$ are referred to as edges of $\graph$.
	Each edge $I_{jk}$ where $j\neq k$ is referred to as the edge directed from $v_j$ to $v_k$, or just the edge from $v_j$ to $v_k$. \newline
	Again we may specify that $\graph$ is a finite graph if $\Lambda$ is finite, or an infinite graph if $\Lambda$ is infinite.
\end{definition}
Loops are still permitted by this definition, however the choice of a direction for these is essentially redundant.
\begin{definition}[Quantum Graph] \label{def:QuantumGraph}
	A \textit{quantum graph} is a directed graph $\graph = \bracs{V,E}$ where each edge $I_{jk}\in E$ is assigned a length $l_{jk}>0$ and associated interval $\sqbracs{0,l_{jk}}$.
\end{definition}
Note that there is no requirement for the edges (if they are present) $I_{jk}$ and $I_{kj}$ to have the same length, we shall see later why we wish to allow this. \newline

We require two further definitions that will enable us to link a structure in physical space to the more abstract Quantum graph.
\begin{definition}[Embedded Graph] \label{def:EmbeddedGraph}
	Set $d\geq2, D\subset\reals^d$ and $\Lambda\subset\naturals$.
	Let $V = \clbracs{\vec{v}_j \ \vert \ j\in\Lambda}$ be a set of distinct points in $D$, and $E\subset V\times V$ be a set of \textit{ordered} pairs of points $I_{jk} := \bracs{\vec{v}_j, \vec{v}_k}$.
	For each $I_{jk}\in E$ with $j\neq k$ let $\gamma_{jk}$ be a continuous curve in $D$ with endpoints $\vec{v}_j$ and $\vec{v}_k$, length $l_{jk}$ and smooth parametrisation $r_{jk}:\sqbracs{0,l_{jk}}\rightarrow\gamma_{jk}$ such that $r_{jk}(0) = \vec{v}_j, r_{jk}\bracs{l_{jk}} = \vec{v}_k$.
	For each $I_{jj}\in E$ let $\gamma_{jj}$ be a closed curve in $D$ passing through $\vec{v}_j$ and with smooth parametrisation $r_{jj}:\left[0,l_{jj}\right)\rightarrow\gamma_{jj}$ such that $r_{jj}(0) = \vec{v}_{j}, \lim_{t\rightarrow l_{jj}}r_{jj}(t) = \vec{v}_j$.
	Assume that all curves $\gamma_{jk}$ are non-intersecting.
	Then we call $\graph=\bracs{V, E, \clbracs{r_{jk}}}$ an \textit{embedded graph} in $D$, or a graph embedded in $D$. \newline
	Again, we say $\graph$ is finite (respectively infinite) if $\Lambda$ is finite (respectively infinite).
\end{definition}
At this point it would be prudent to some observations about definition \ref{def:EmbeddedGraph}; as well as some motivation and conventions.
\begin{itemize}
	\item Because the maps $r_{jk}$ are tied to the edges $I_{jk}$, for shorthand we will forgo including these when we introduce an embedded $\graph$, unless there is a clear need to specify these.
	Thus we shall henceforth specify embedded graphs by the shorthand $\graph=\bracs{V,E}$, meaning $\graph=\bracs{V, E, \clbracs{r_{jk}}}$.
	\item Am embedded graph $\graph = \bracs{V,E}$ is a framework for representing singular structures in physical space, by associating the vertices to points and the edges of a graph to curves connecting these points, we can think of the graph as occupying some physical volume/area.
	We can also effectively treat $\graph$ as a subset of $D$, and perform set operations to construct sub-graphs, or use set intersections to pull out select portions of a graph.
	For example, we may specify the sub-graph of $\graph$ composed of all the loops of $\graph$ by writing
	\begin{align*}
		S_{\graph} &:= \bigcup_{I_{jj}\in E} I_{jj}
	\end{align*}
	which should be taken to have the same meaning as the following;
	\begin{align*}
		V_S := \clbracs{\vec{v}_j\in V \ \vert \ I_{jj}\in E}, &\quad E_S := \clbracs{I_{jj} \ \vert \ I_{jj}\in E}, \\
		R_S := \clbracs{r_{jj} \ \vert \ I_{jj}\in E}, &\\
		S_{\graph} &:= \bracs{V_S, E_S, R_S}.
	\end{align*}
	Likewise we may also use $\graph$ as a set in the sense that
	\begin{align*}
		\graph &= \bigcup_{I_{jk}\in E} \gamma_{jk},
	\end{align*}
	so we could specify the subset of $D$ corresponding to the portion of the graph $\graph$ that occupies the square $\sqbracs{-\recip{2},\recip{2}}^2$ by writing
	\begin{align*}
		\graph \cap \sqbracs{-\recip{2},\recip{2}}^2.
	\end{align*}
	Of course we could define another embedded graph in this way; provided we don't cut any $\gamma_{jk}$ in two with the intersection, or if we do we specify what happens to such ``hanging edges".
	Essentially, we may treat $\graph$ as both a set in $D$ and in the sense of a graph as in definition \ref{def:DirectedGraph}.
	For this reason we will also drop the distinction between $I_{jk}$ and $\gamma_{jk}$, simply writing $I_{jk}$ for both objects.
	\item Any embedded graph clearly gives rise to a directed graph in the sense of definition \ref{def:DirectedGraph}, and also to an associated quantum graph as in definition \ref{def:QuantumGraph} (where the lengths of the edges $I_{jk}$ coincide with the lengths of the curves $\gamma_{jk}$).
	Likewise; by assigning some points, curves and parametrisations, we can obtain an embedded graph from directed or quantum graph.
	\item Loops (closed curves) are permitted but require a slightly different treatment to ``regular" edges, although this doesn't introduce major complexities in the theory that follows.
	In chapters \ref{ch:ScalarEqns} and \ref{ch:VectorEqns} we will not consider structures that contain loops; but apart from requiring some slightly more technical treatment they do not introduce major differences from the results that are presented for non-loop edges either.
\end{itemize}

Lastly, we need to provide a notion of periodicity for graphs.
\begin{definition}[Periodic Graph] \label{def:PeriodicGraph}
	Let $\graph=\bracs{V,E}$ be an infinite graph embedded in a domain $\reals^d$, $d\geq2$.
	Let $e_1, e_2, ..., e_d$ be the canonical basis for $\reals^d$.
	Suppose there exist real numbers $T_{j}\geq0, j\in\clbracs{1,...,d}$ such that for each $j\in\clbracs{1,...,d}$, the $\graph$ (treating it as a set of points in $\reals^d$) is invariant under the translation $T_j e_j$.
	Then $\graph$ is a \textit{periodic (embedded) graph} (in $\reals^d$). \newline
	
	Suppose further that the $T_j$ are minimal in the sense that for each $j$ there does not exist a $T_j'$ such that $T_j'<T_j$ and $\graph$ is invariant under $T_j' e_j$.
	Let $\mathcal{C}$ be the hyper-cuboid $\mathcal{C}:= \sqbracs{0,T_1}\times ... \times\sqbracs{0,T_d}$, then we define the \textit{period cell}, or \textit{period graph}, or \textit{unit cell}, of $\graph$ as the (embedded graph in $\mathcal{C}$)
	\begin{align*}
		\graph_{\mathrm{P}} &:= \graph \cap \mathcal{C}.
	\end{align*}
	To avoid cutting edges in two, we think of the hyper-cuboid $\mathcal{C}$ as being mapped onto the $d$-dimensional torus; with vertices that lie on the boundary of $\mathcal{C}$ being identified with each other, and curves ``leaving" a hyper-face of $\mathcal{C}$ connecting to the curve ``entering"  $\mathcal{C}$ from the opposite hyper-face.
	This is illustrated in figure \ref{fig:PeriodCellIllustration} in two dimensions, for clarity.
	
	Of course, because we can identify sub-graphs of embedded graphs by providing the sets which contain them, we can also say that $\mathcal{C}$ is the unit cell of $\graph$ too provided we remember how to deal with edges cut in two, as above.
\end{definition}
\begin{figure}[h!]
	\centering
	\begin{subfigure}[t]{0.45\textwidth}
		\centering
		\includegraphics[height=4.5cm]{Diagram_PeriodCellFullLattice.pdf}
		\caption{\label{fig:Diagram_PeriodCellFullLattice} A periodic graph in $\reals^2$, with the period cell marked.}
	\end{subfigure}
	~
	\begin{subfigure}[t]{0.45\textwidth}
		\centering
		\includegraphics[height=4.5cm]{Diagram_PeriodCellEdgeAssociation.pdf}
		\caption{\label{fig:Diagram_PeriodCellEdgeAssociation} The period graph of the graph in \ref{fig:Diagram_PeriodCellFullLattice}, illustrating how the edges of the period cell are associated. The cut-off edge leaving $v_4$ is attached to the beginning of the cut-off edge entering $v_2$, and similarly for the edge leaving $v_3$ and entering $v_1$.}
	\end{subfigure}
	\\
	\begin{subfigure}[b]{0.75\textwidth}
		\centering
		\includegraphics[scale=	1.0]{Diagram_PeriodCellOnTorus.pdf}
		\caption{\label{fig:Diagram_PeriodCellOnTorus} An illustration of the period graph as being thought of as occupying (or being embedded into) the 2D torus.}
	\end{subfigure}
	\caption{\label{fig:PeriodCellIllustration} Illustrating a periodic graph and the extraction of it's period graph, and how it can be thought of as embedded into the two-dimensional torus.}
\end{figure}

We want to be able to assign this meaning to the period cell of a periodic graph because when we take a Gelfand transform of a problem on a periodic graph $\graph$, we will end up with a family of problems $\graph_{\mathrm{P}}$.
Again we briefly discuss some potential concerns and conventions.
\begin{itemize}
	\item The need to associate the edges that are cut in two and vertices that lie on the boundary of $\mathcal{C}$ allows us to preserve some idea of edges with lengths connecting the vertices of $\graph_{\mathrm{P}}$, which in turn gives rise to an associated quantum graph.
	This associated quantum graph is what we will be posing our (new family) of problems on, post-Gelfand transform.
	It is for this reason it is helpful to think of the period cell as living on a torus; we can still have continuous curves in the torus (of the same lengths as the edges) and smooth parametrisations that map these curves to intervals in $\reals$, which is all we require to pose a quantum graph problem.
	\item Note that $\graph_{\mathrm{P}}$ (and hence it's associated quantum graph) can have loops even if $\graph$ did not, or rather did not appear to.
	Because of this we still provide results concerning quantum graphs with loops in chapter \ref{ch:ExampleSystems} even though our singular structures do not have loops themselves.
\end{itemize}
Finally, an assumption that we shall make from this point onwards in this work is that the edges or curves of an embedded graph are always the straight line segments that join the two vertices at either end.
Much like excluding the possibility of loops, allowing for curved edges is not a major issue (simply introducing some small complications in the expressions one gets when reducing variational problems to quantum graph ones) but we will not be interested in systems that exhibit curved edges, and so opt for the simplicity that making this assumption comes with.
Quantum graphs will be the objects that the theory of the remainder of this chapter section will describe, although they are not the starting point for the domains of our model (section \ref{sec:OurPhysicalSetup}). 
Rather we will begin with a periodic embedded graph, and then under our modelling assumptions in sections \ref{sec:ModellingAssumption1}-\ref{sec:ModellingAssumption3} we will obtain an equivalent problem on the period graph, which gives rise to a quantum graph to which we can apply the following theory.

\subsection{Connection and Summation Conventions} \label{sec:GraphSimDefinitions}
We now outline some notation that we shall use in the rest of the report.
For any quantum, directed, or embedded graph $\graph=\bracs{V,E}$ we will use the notation $j\conLeft k$ to mean $I_{jk}\in E$, which can be read as ``$j$ connects to $k$ with $j$ on the left".
Similarly we use $j\conRight k$ to mean $I_{kj}\in E$, read as ``$j$ connects to $k$ with $j$ on the right.
The use of the term right and left comes from the fact that the parametrisation $r_{jk}$ maps the left endpoint of $\sqbracs{0,l_{jk}}$ to $v_j$, and the right endpoint to $v_k$, and vice-versa for $r_{kj}$.
We may also use $j\conLeft k$ as indices in summations, if $v_j$ is fixed a priori then the expression $\sum_{j\conLeft k}$ is taken to mean the sum over all $v_k\in V$ such that $I_{jk}\in E$.
We assign the obvious similar meaning to $j\conRight k$ as a summation index, and note that (formally)
\begin{align*}
	\sum_{v_j\in V}\sum_{j\conLeft k} &= \sum_{v_j\in V}\sum_{j\conRight k} = \sum_{I_{jk}\in E},
\end{align*}
which provides us a succinct notation for summing over all the edges of $\graph$ by referencing their vertices.
We will mainly use this notation when dealing with integrals taken with respect to singular measures (section \ref{sec:GraphSigularMeasuresDef}) supported on $\graph$.

\subsection{Embedded Graphs and Singular Measures} \label{sec:GraphSigularMeasuresDef}
To complete in this section, we now briefly setup the framework that will link quantum graph problems (as in section \ref{sec:DEonQG}) and variational problems posed with respect to singular measures (which are the focus of chapters \ref{ch:ScalarEqns} and \ref{ch:VectorEqns}).
\begin{definition}[Singular Measure on an (Embedded) Graph]
Let $D\subset\reals^d, d\geq2$ and $\graph=\bracs{V,E}$ be a graph embedded into $D$ (note this encompasses period cells of periodic graphs).
On $D$ and for each $I_{jk}\in E$ define the (Borel) measure $\lambda_{jk}$ as the measure which supports 1D Lebesgue measure on the edge $I_{jk}$, so for some Borel set $B\subset D$ we have that 
\begin{align*}
	\lambda_{jk}\bracs{B} = \lambda_{1}\bracs{r_{jk}^{-1}\bracs{B \cap I_{jk}}}
\end{align*}
where $\lambda_1$ is the 1D-Lebesgue measure on $\reals$.
Then set $\nu$ to be the (Borel) measure on $D$ defined by
\begin{align*}
	\nu\bracs{B} = \sum_{v_j\in V}\sum_{j\conLeft k} \lambda_{jk}\bracs{B}.
\end{align*}
Then $\nu$ is the \textit{singular measure that supports $\graph$}; or alternatively the \textit{singular measure on $\graph$}, or the \textit{(singular) measure that supports the edges of $\graph$}.
\end{definition}
For a graph embedded into a 2D or 3D domain, the singular measure is illustrated in figure \ref{fig:Diagram_SingularMeasures}.
\begin{figure}[h!]
	\centering
	\begin{subfigure}[t]{0.45\textwidth}
		\centering
		\includegraphics[height=6cm]{Diagram_SingularMeasure2D.pdf}
		\caption{\label{fig:Diagram_SingularMeasure2D} For a graph embedded in $\reals^2$, the measure of any Borel set $B$ by each $\lambda_{jk}$ is indicated by the thickened and coloured lines.
		All other edge-measures give $B$ measure 0.}
	\end{subfigure}
	~
	\begin{subfigure}[t]{0.45\textwidth}
		\centering
		\includegraphics[height=6cm]{Diagram_SingularMeasure3D.pdf}
		\caption{\label{fig:Diagram_SingularMeasure3D} Illustrating the singular measure of a cube $B$ for a graph in 3D.
		The red portions of the edges indicate the contributions from each of the edge measures to the total measure $\nu\bracs{B}$.
		Note that the vertices of the graph are inflated for visual effect.}
	\end{subfigure}
	\caption{\label{fig:Diagram_SingularMeasures} Visual illustration of singular measures of a graph in (\ref{fig:Diagram_SingularMeasure2D}) 2D and (\ref{fig:Diagram_SingularMeasure3D}) 3D. }
\end{figure}

As discussed in section \ref{sec:ReportOverview}, a variational formulation is necessary for us to pose equations on singular domains.
When our structure corresponds to an embedded graph, we will be using the singular measure on that graph as the foundations for our variational formulation.
Intuitively the singular measure on a graph provides us with natural way to ``respect" the lower-dimensional structure of the graph, and thus integration (inherent to a variational problem) also ``respects" this structure.
The precise nuances of adopting these measures is the focus of chapters \ref{ch:ScalarEqns} and \ref{ch:VectorEqns}.

\section{Differential Equations on Quantum Graphs} \label{sec:DEonQG}
In this section we will be working with differential operators on function spaces that involve quantum graphs, in order to be consistent with several developments in the literature.
However we shall see that such operators and the functions in their domains are broken down in such a way that they can be thought of as a system of differential equations on intervals, coupled through (somewhat non-standard) boundary conditions.
As such we will begin this section by defining several function spaces that we wish to work on, then providing examples of the types of boundary conditions that we might want to consider, before finally providing a concrete example of a differential operator on a quantum graph. 
For the reader who wants a more complete picture of the theory of quantum graphs, they are directed to \cite{berkolaiko2013introduction}. \newline

Since quantum graphs come with lengths (and intervals) associated to their edges, we can define function spaces on them by combining function spaces on these intervals.
As such we define
\begin{subequations} \label{eq:GraphFuncSpaces}
	\begin{align}
		L^2\bracs{\graph} := \bigoplus_{I_{jk}\in E} \ltwo{\interval{l_{jk}}}{t},
		&\quad H^1\bracs{\graph} := \bigoplus_{I_{jk}\in E} \gradSob{\interval{l_{jk}}}{t}, \\
		H^2\bracs{\graph} := \bigoplus_{I_{jk}\in E} H^2_\mathrm{grad}\bracs{\interval{l_{jk}}, \md t}, &
	\end{align}
\end{subequations}
A function $u\in L^2\bracs{\graph}$ is then determined by it's form on each edge $I_{jk}$ (and similarly for functions and their distributional derivatives in $H^1\bracs{\graph}$).
Because we will mainly be working on the edges of our graphs, we define $u_{jk} = u\vert_{I_{jk}}$ to be the restriction of $u$ to the edge $I_{jk}$, extended by zero to the whole of $\graph$.
Because we associate the directed edge $I_{jk}$ to the interval $\sqbracs{0,l_{jk}}$, for a function $u\in L^2\bracs{\graph}$ we will sometimes use the shorthand $u_{jk}\bracs{v_k}$ to mean the value of the function $u_{jk}$ at the point in $\sqbracs{0,l_{jk}}$ corresponding to $v_k$, and similarly for $u_{jk}\bracs{v_j}$.
If $u$ is continuous at the vertex $v_j$, we will use the notation $u\bracs{v_j}$ for this value.
A further complication arising from the fact that our edges are directed is that it is necessary for us to adopt a notion of ``directional derivative" for functions $u_{jk}$ at the ends of the edges \cite{ershova2014isospectrality}, and so we adopt the following convention\footnote{Note that this is the opposite convention to that of \cite{ershova2014isospectrality}.};
\begin{align*}
	\diff{}{t}u_{jk}\bracs{v_j} &= -u'_{jk}\bracs{v_j}, \\
	\diff{}{t}u_{jk}\bracs{v_k} &= u'_{jk}\bracs{v_k}.
\end{align*}
Recall that the subscript $jk$ denotes that the edge $I_{jk}$ is directed from $v_j$ to $v_k$; so our convention is succinctly summarised as ``derivatives directed into a vertex are positive, whilst derivatives directed out of a vertex are negative". 
Lastly we provide another piece of notation that will ease the notational burden of the following work.
We write $j\con k$ to mean ``the vertex $v_j$ connects to $v_k$", or formally $j\con k$ if and only if $I_{jk}\in E$ or $I_{kj}\in E$.
In summations we may (for a fixed $v_j$) use $j\con k$ as a summation index for expressions like
\begin{align*}
	\sum_{j\con k} u_{jk}\bracs{v_j},
\end{align*} 
formally meaning the sum is taken over all $j\conLeft k$ and $j\conRight k$, with the appropriate swapping of indices on any edge-parts of functions in the summation:
\begin{align*}
	\sum_{j\con k} u_{jk}\bracs{v_j} &:= \sum_{j\conLeft k}u_{jk}\bracs{v_j} + \sum_{j\conRight k}u_{kj}\bracs{v_j}, \\
	\sum_{j\con k} \diff{u_{jk}}{t}\bracs{v_k} &:= \sum_{j\conLeft k} \diff{u_{jk}}{t}\bracs{v_k} + \sum_{j\conRight k} \diff{u_{kj}}{t}\bracs{v_k}
\end{align*}
The interpretations then follow from the interpretation of the $j\conLeft k$ and $j\conRight k$ sums.
This notation will mainly be used when we look to derive properties of functions $u\in L^2\bracs{\graph}$ (or similar function spaces) at the vertices of $\graph$, or obtaining boundary conditions from a variational problem (see chapters \ref{ch:ScalarEqns} and \ref{ch:VectorEqns}).
Note that (importantly) the point of evaluation of functions in these summations does not change, this notation is simply a shorthand to save us having to write out two cases in every sum we consider over the edges of $\graph$. \newline

This edge-wise breakdown of our function spaces allows us to define differential operators on $\graph$ by specifying the form of the operator on each edge $I_{jk}$ (by which we mean it's associated interval $\interval{l_{jk}}$).
However it is important to note that the spaces $L^2\bracs{\graph}$ and the other spaces in \eqref{eq:GraphFuncSpaces} do not come with an in-built appreciation for the connectivity of the graph itself; and it is not hard to see that for two graphs with the same number of edges and identical lengths, these spaces will be identical.
Thus to obtain a well-posed problem (strictly speaking, self-adjoint differential operator) on $\graph$, we require additional boundary conditions\footnote{Or matching conditions, or boundary data.} to obtain a unique solution to our problem.
For quantum graphs, these boundary conditions come at the vertices of the graph and we shall be referring to them as vertex conditions.
These conditions come in several types, and there is no requirement that every vertex in a graph has the same conditions imposed at it.
That being said, most of the systems that we will want to be considering will adhere to this, although this is largely due to how we arrive at such systems from our variational framework (see chapters \ref{ch:ScalarEqns} and \ref{ch:VectorEqns}).
The most intuitive vertex condition that we can impose at a given vertex $v_j$ is the requirement that the function $u$ be continuous at $v_j$.
Indeed the construction of the spaces in \eqref{eq:GraphFuncSpaces} does not place any requirement that there be a common value of $u$ at the vertices (as each $u_{jk}$ is an $L^2$-function on a disjoint interval).
If the condition of continuity is imposed at $v_j$ one can then also impose a Kirchoff-like condition on the (directional) derivatives of $u$ at the vertex,
\begin{align*}
	\sum_{j\con k}\diff{u_{jk}}{t}\bracs{v_j} &= \alpha_j u\bracs{v_j}.
\end{align*}
Here $\alpha_j\in\reals$ is a constant that is chosen for the vertex $v_j$, and the value $u\bracs{v_j}$ exists due to the condition of continuity at this vertex.
If continuity of $u$ is not imposed at a vertex, is it still possible to pose conditions that are Kirchoff-like, such as
\begin{align*}
	\sum_{j\con k}u_{jk}\bracs{v_j} &= \alpha_j, &\quad \alpha_j\in\reals, \\
	\sum_{j\con k}u_{jk}\bracs{v_j} &= \alpha_j \sum_{j\con k}\diff{u_{jk}}{t}\bracs{v_j}, &\quad \alpha_j\in\reals.
\end{align*}
These conditions will not be of interest to us, and the theory we present in this section is simply a selection of the more general work of \cite{ershova2014isospectrality}, \cite{ershova2016isospectrality} which deal with this additional generality. \newline

We now provide a simple example of a differential operator $\mathcal{A}$ on $\graph$, however it is not hard to see how the construction can be made general.
First we must provide a domain for $\mathcal{A}$ by deciding how much regularity we want in our functions, and the vertex conditions we want to impose;
\begin{align} \label{eq:ExampleOppDomainDef}
	\mathrm{dom}\mathcal{A} &= \clbracs{ u\in H^2\bracs{\graph} \ \vert \ u \text{ is continuous at all } v_j\in V, \ \sum_{j\con k}\diff{u_{jk}}{t}\bracs{v_j} = 0 \ \forall v_j\in V}.
\end{align}
Note that different vertex conditions can be imposed at different vertices by specifying them in the domain of the operator, however to avoid a cumbersome example we have taken identical conditions at each vertex.
The remaining ingredient for $\mathcal{A}$ is what it actually does to functions in it's domain, which is typically done by specifying the action on each edge of $\graph$ (hence the construction of the function spaces in \eqref{eq:GraphFuncSpaces});
\begin{align} \label{eq:ExampleOppEdgeDef}
	\mathcal{A} &= -\diff{}{t} \quad\text{on each } I_{jk}\in E.
\end{align}
Of course, by ``on each $I_{jk}\in E$" we mean ``on the interval $\sqbracs{0,l_{jk}}$ that we associate to $I_{jk}\in E$".
Then for a function $f\in L^2\bracs{\graph}$ we can pose the resolvent problem of finding $u\in\mathrm{dom}\mathcal{A}$ such that
\begin{align*}
	\mathcal{A}u &= f;
\end{align*}
or alternatively can consider the spectral problem of finding eigenpairs $\bracs{\lambda,u}\in\complex\times\mathrm{dom}\mathcal{A}$ such that
\begin{align*}
	\mathcal{A}u &= \lambda u.
\end{align*}
As the spaces in \eqref{eq:GraphFuncSpaces} break down into edge-wise components which are acted on individually by $\mathcal{A}$, and only linked through the vertex conditions, we can rewrite both of these problems as a set of ODEs on intervals coupled through vertex conditions;
\begin{align*}
	\mathcal{A}u = f \Leftrightarrow \
	& (i) \ -\diff{u_{jk}}{t} = f_{jk} \ \text{on } \sqbracs{0,l_{jk}}, \\
	& (ii) \ u \text{ is continuous at each } v_j\in V, \\
	& (iii) \ \sum_{j\con k}\diff{u_{jk}}{t}\bracs{v_j} = 0 \ \forall v_j\in V, \\
	\mathcal{A}u = \lambda u \Leftrightarrow \
	& (i) \ -\diff{u_{jk}}{t} = \lambda u_{jk} \ \text{on } \sqbracs{0,l_{jk}}, \\
	& (ii) \ u \text{ is continuous at each } v_j\in V, \\
	& (iii) \ \sum_{j\con k}\diff{u_{jk}}{t}\bracs{v_j} = 0 \ \forall v_j\in V.
\end{align*}
We will largely pose differential equations on quantum graphs by specifying the information on the right hand side of these equivalences, as this is where our theory in chapters \ref{ch:ScalarEqns} and \ref{ch:VectorEqns} will take us.
Either of the equivalent forms above will be referred to as a ``quantum graph problem" or a set of ``differential equations on a (quantum) graph" for the purposes of this work; and we will often coin the phrase ``spectrum of the (quantum graph) problem" to mean ``spectrum of the operator that defines the appropriate (quantum graph) problem".
The reason for us making this equivalence so explicit is because the operator-theoretic approach to quantum graph problems has yielded some useful tools for determining the spectrum of such operators, which we discuss in the following section.

\section{Spectral Problems and the M-Matrix} \label{sec:M-MatrixTheory}
Since we are primarily looking to consider wave-guidance problems in electromagnetism, we will often be concerned with determining the spectrum of a problem as the eigenvalues relate to the frequencies that can be guided by our fibre designs.
In particular we will be looking to determine whether such a spectrum exhibits band-gaps, and ideally a method to explore the spectrum numerically.
Because of our use of the Gelfand transform (section \ref{sec:ModellingAssumption2}) to simplify our original problem to a family of problems on a periodic domain, determining the spectrum of a problem will require solving several (eigenvalue) problems parametrised by the quasi-momentum $\qm$.
This prompts us to look for tools that enable us to determine the spectrum of one of these smaller problems both procedurally and efficiently, crucial if a numerical approach is to be used but also helpful if an analytic approach can be taken too.
In this section we will be discussing the tool that we are going to use; the $M$-matrix, and why it is appropriate for both a numerical and analytic approach to solving determining the spectra of quantum graph problems. \newline

To properly define the $M$-matrix we will need to introduce a few new concepts.
As discussed in section \ref{sec:GraphLitReview}, the $M$-matrix is in fact one particular case of the more general (Weyl-Titchmarsh) $M$-operator as it appears in the theory of boundary triples, and what we present here is simply the framework that we require without any of the generality of this theory.
Let $\mathcal{A}$ be (an operator that defines) a quantum graph problem on a quantum graph $\graph=\bracs{V,E}$.
Define the maps
\begin{align*}
	\dmap, \nmap: \mathrm{dom}\mathcal{A} \rightarrow \complex^{\abs{V}};
\end{align*}
where for a function $u\in\mathrm{dom}\mathcal{A}$, $\dmap$ sends $u$ to it's Dirichlet data at each of the vertices $v_j$ represented in a column vector in $\complex^{\abs{V}}$, and $\nmap$ sends $u$ to it's Neumann data at each of the vertices $v_j$ (again represented as a column vector).
Assuming we are using the labels $v_j, j\in\clbracs{1,...,\abs{V}}$ for the vertices of $\graph$, the $j$\textsuperscript{th} entry of the vectors $\dmap u$ and $\nmap u$ are given by
\begin{align*}
	\bracs{\dmap u}_j &= u\bracs{v_j}, \\
	\bracs{\nmap u}_j &= \sum_{j\con k}\diff{u_{jk}}{t}\bracs{v_j}.
\end{align*}
The $M$-matrix is then defined as the unique map
\begin{align*}
	M\bracs{\lambda}:\complex^{\abs{V}} &\rightarrow \complex^{\abs{V}},
\end{align*}
such that whenever $u\in\mathrm{ker}\bracs{\mathcal{A}-\lambda}$,
\begin{align*}
	M\bracs{\lambda}\dmap u &= \nmap u.
\end{align*}
Of particular relevance to our work is that eigenvalues of $\mathcal{A}$ occur at those $\lambda$ such that $\mathrm{det}\bracs{M\bracs{\lambda} -\lambda A}= 0$, where $A = \mathrm{diag}\bracs{\alpha_1, \alpha_2, ... , \alpha_{\abs{V}}}$ is the diagonal matrix of the coupling constants.
For ease we will tend to write $M_{\qm}\bracs{\lambda}v=0$ or $\det\bracs{M_{\qm}\bracs{\lambda}}=0$ for this problem; because the majority of our examples (chapter \ref{ch:ExampleSystems}) with have all coupling constants equal to 0.
Furthermore we can always relabel $\widetilde{M}_{\qm}\bracs{\lambda} = M_{\qm}\bracs{\lambda}-\lambda A$ if the need arises, which may be particularly useful if a numerical scheme is to be used. \tstk{also holomorphic, hermitian? so that's some simplifying stuff we should mention. ALSO coupling constants not equal to zero?}

Due to the fact that we will be taking a Gelfand transform, each problem in the family parametrised by $\qm$ will yield a different $M$-matrix, so we adopt the notation $M_{\qm}\bracs{\lambda}$ to provide distinction when we require it.
If the form for the $M$-matrix can be determined analytically (or even numerically) then determining the spectrum is now a case of solving the generalised eigenvalue problem
\begin{align} \label{eq:M-MatrixEvalProb}
	\bracs{M_{\qm}\bracs{\lambda}-\lambda A}v &= 0, \quad \lambda\in\complex, v\in\complex^{\abs{V}},
\end{align}
for each $\qm\in[-\pi,\pi)^2$.
Of course the approach to solving such a problem will be different if one is looking to pursue the spectrum analytically or numerically, and the discussion here will be developed when we come to consider some examples in chapter \ref{ch:ExampleSystems}.
Using the numerical approach, thinking of the problem as a generalised eigenvalue problem is much more suitable as one can use numerical schemes for generalised eigenvalue problems, such as those offered in \cite{guttel2017nonlinear}.
Of course there is the problem with this approach of how to deal with the need to ``sweep" across each value of $\qm$; for example one could choose to discretise $[-\pi,\pi)^2$ but then has to concede to solving one such eigenvalue problem per discrete value of $\qm$.
A final point on the numerical side is how easy to actually construct the $M$-matrix (for a given value of $\lambda$) in the first place, and how costly this is - although we will see in chapter \ref{ch:ExampleSystems} that in a select few cases we can proceed analytically to bypass this issue.
Taking an analytical approach, we shall see (chapter \ref{ch:ExampleSystems}) it is often better to consider $\mathrm{det}M_{\qm}\bracs{\lambda} = 0$ and attempt to derive conditions on $\lambda$ in terms of $\qm$ (which can then be explored numerically if so desired).

\section{Chapter Summary} \label{sec:QGSummary}
This chapter lays the foundations for our terminology and notation involving graphs, which we will be assuming throughout the rest of the report.
Although not as general as they could be made, this is no great loss to us as the quantum graph problems that arise from our physically-motivated systems will not exhibit every general feature.
This also allows us to be somewhat more suggestive with our notation, terminology and interpretation of embeddings of graphs, whilst also maintaining rigour. \newline

In terms of theory, sections \ref{sec:DEonQG} and \ref{sec:M-MatrixTheory} provide us with the objects and terminology we need to describe differential equations on graphs, and reduce spectral problems to the solution of matrix-eigenvalue problems.
Having the M-matrix as a tool is invaluable, not just because it has a rich general theory to draw upon \tstk{refs} but also because it opens the door for numerical approaches to determining the spectrum of quantum graph problems.
Much of this theory is well-established, and we do not go into the more abstract details of quantum graphs problems such as existence of solutions or being well-posed, as this is not the focus of the work of this report.
The interested reader is directed to, for example, \tstk{refs!!!}. \newline

In chapters \ref{ch:ScalarEqns} and \ref{ch:VectorEqns} we will largely focus on how to obtain a quantum graph problem from our physically motivated setup, but will use some of the notation and conventions from this chapter - in particular when defining quantum graph problems themselves.
Chapter \ref{ch:ExampleSystems} will explore some example quantum graph problems using the theory from the aforementioned chapters and this one, and will also build on the proposed ideas for numerically determining the spectrum of a quantum graph problem.