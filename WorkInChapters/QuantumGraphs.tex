\chapter{Quantum Graphs} \label{ch:QuantumGraphs}
In this chapter we shall introduce the concept of Quantum Graphs, their associated function spaces, and operators defined on these spaces.
Heavy references throughout to EKK, Kuchment, I guess Olaf \& Post too maybe?

\section{Introduction}
lit review part if it's needed, although we might have done this in the introduction chapter tbh.
If so we won't need to revisit the Kuchment nor Olaf \& Post references again, and can just go straight to the definitions and throw in the stuff by EKK etc.

\section{Notation, Definitions, and Conventions} \label{sec:QG-Notation}
It's really hard to define this shit.

\tstk{we only work with finite graphs!}
\begin{definition}[Graph] \label{def:Graph}
	Let $N\in\naturals$ and $V=\clbracs{v_j \ \vert \ j\in\clbracs{1,2,...,N}}$ be a set of labels $v_j$ bijective to $\clbracs{1,2,...,N}$ via the map $j\rightarrow v_j$.
	Let $E\subset V\times V$ be a finite set of \textit{unordered} pairs $\bracs{v_j,v_k}\in E$ where $j,k\in\clbracs{1,2,...,N}$.
	If $\bracs{v_j,v_k}\in E$ then write $I_{jk} = \bracs{v_j,v_k}$, note that $I_{jk}=I_{kj}$.
	Then $\graph=\bracs{V,E}$ is a (finite) graph with vertex set $V$ and edge set $E$.
	Elements of the set $V$ are called vertices of the graph $\graph$ and elements of $E$ are referred to as edges of $\graph$.	
\end{definition}
Definition \ref{def:Graph} is not as general as others that can be found in graph theory, however we work with this definition because it is sufficient for our later purposes so the loss of generality and certain features is not an issue.
In particular we highlight some features of this definition below.
\begin{itemize}
	\item The definition requires that there is only a single edge $I_{jk}$ connecting any pair of vertices.
	We will shortly define the concept of a directed graph where we allow two edges between any pair of vertices, having opposite ``directions".
	We do not go any further than this, even though it is possible by introducing certain equivalence relations on the sets $V$ and $E$, simply because we will not be interested in any systems that require this functionality. 
	\item A graph has a finite number of vertices and edges.
	This does not need to be true in general, and there is nothing wrong with removing the restriction that $V$ and $E$ be finite.
	However since we shall be wanting to use graphs to represent physical structures in some sense (\tstk{chapter ref}), we will not be needing this functionality.
	\item Loops (edges of the form $I_{jj}$) are permitted, but a given vertex can only have at most one loop.
	\item The labels $v_j$ are slightly unnecessary, as one can just work directly with the index set $\clbracs{1,2,...,N}$.
	However we will be wanting labels for our vertices and edges when we come to consider quantum and embedded graphs.
\end{itemize}

Having laid out this basis for the concept of a graph, we can now present some further definitions.
\begin{definition}[Directed Graph] \label{def:DirectedGraph}
	Let $N\in\naturals$ and $V=\clbracs{v_j \ \vert \ j\in\clbracs{1,2,...,N}}$ be a set of labels $v_j$ bijective to $\clbracs{1,2,...,N}$ via the map $j\rightarrow v_j$.
	Let $E\subset V\times V$ be a finite set of \textit{ordered} pairs $\bracs{v_j,v_k}\in E$ where $j,k\in\clbracs{1,2,...,N}$.
	Write $I_{jk} = \bracs{v_j,v_k}$ for the elements of $E$.
	Then $\graph=\bracs{V,E}$ is a (finite) directed graph with vertex set $V$ and edge set $E$.
	Elements of the set $V$ are called vertices of the graph $\graph$ and elements of $E$ are referred to as edges of $\graph$.
	Each edge $I_{jk}$ where $j\neq k$ is referred to as the edge directed from $v_j$ to $v_k$, or just the edge from $v_j$ to $v_k$.
\end{definition}
Loops are still permitted by this definition, however the choice of a direction for these is essentially redundant.
\begin{definition}[Quantum Graph] \label{def:QuantumGraph}
	A Quantum Graph is a directed graph $\graph = \bracs{V,E}$ where each edge $I_{jk}\in E$ is assigned a length $l_{jk}\geq0$ and associated interval $\interval{l_{jk}}$.
\end{definition}
Note that there is no requirement for the edges (if they are present) $I_{jk}$ and $I_{kj}$ to have the same length, we shall see later why we wish to allow this.
Loops are still permitted under this definition and will have an associated length $l_{jj}$.
Quantum graphs will be the objects that the theory of this section will describe, however they are not the starting point for our physical problems \tstk{chapter/section ref}.
We require one further definition that will enable us to link a structure in physical space to the more abstract Quantum graph.

\begin{definition}[Embedded Graph] \label{def:EmbeddedGraph}
	Set $d\geq2, D\subset\reals^d$ and $N\in\naturals$.
	Let $V = \clbracs{\vec{v}_j \ \vert \ j\in\clbracs{1,2,...,N}}$ be a set of distinct points in $D$, and $E\subset V\times V$ be a set of \textit{ordered} pairs of points $I_{jk} := \bracs{\vec{v}_j, \vec{v}_k}$.
	For each $I_{jk}\in E$ with $j\neq k$ let $\gamma_{jk}$ be a continuous curve in $D$ with endpoints $\vec{v}_j$ and $\vec{v}_k$, length $l_{jk}$ and smooth parametrisation $r_{jk}:\interval{l_{jk}}\rightarrow\gamma_{jk}$ such that $r_{jk}(0) = \vec{v}_j, r_{jk}\bracs{l_{jk}} = \vec{v}_k$.
	For each $I_{jj}\in E$ let $\gamma_{jj}$ be a closed curve in $D$ passing through $\vec{v}_j$ and with smooth parametrisation $r_{jj}:\left[0,l_{jj}\right)\rightarrow\gamma_{jj}$ such that $r_{jj}(0) = \vec{v}_{j}, \lim_{t\rightarrow l_{jj}}r_{jj}(t) = \vec{v}_j$.
	Assume that all curves $\gamma_{jk}$ are non-intersecting.
	Then we call $\graph=\bracs{V, E, \clbracs{r_{jk}}}$ an embedded graph in $D$, or a graph embedded in $D$.
\end{definition}
Again we make some observations; and provide some motivation and conventions for this definition.
\begin{itemize}
	\item Because the maps $r_{jk}$ are tied to the edges $I_{jk}$, for shorthand we will forgo including these when we introduce an embedded $\graph$, unless there is a need for a notational change.
	As such, we shall specify embedded graphs by the shorthand $\graph=\bracs{V,E}$, meaning $\graph=\bracs{V, E, \clbracs{r_{jk}}}$.
	\item Am embedded graph $\graph = \bracs{V,E}$ is a framework for representing singular structures in physical space, by associating the vertices to points and the edges of a graph to curves connecting these points, we can think of the graph as occupying some physical volume/area.
	We can also effectively treat $\graph$ as a subset of $D$, and perform set operations to construct sub-graphs, or use set intersections to pull out select portions of a graph.
	For example, we may specify the sub-graph of $\graph$ composed of all the loops of $\graph$ by writing
	\begin{align*}
		S_{\graph} &:= \bigcup_{I_{jj}\in E} I_{jj}
	\end{align*}
	which should be taken to have the same meaning as the following;
	\begin{align*}
		V_S := \clbracs{\vec{v}_j\in V \ \vert \ I_{jj}\in E}, &\quad E_S := \clbracs{I_{jj} \ \vert \ I_{jj}\in E}, \\
		R_S := \clbracs{r_{jj} \ \vert \ I_{jj}\in E}, &\\
		S_{\graph} &:= \bracs{V_S, E_S, R_S}.
	\end{align*}
	Likewise we may also use $\graph$ as a set in the sense that
	\begin{align*}
		\graph &= \bigcup_{I_{jk}\in E} \gamma_{jk},
	\end{align*}
	so we could specify the subset of $D$ corresponding to the portion of the graph $\graph$ that occupies the square $\sqbracs{-\recip{2},\recip{2}}^2$ by writing
	\begin{align*}
		\graph \cap \sqbracs{-\recip{2},\recip{2}}^2.
	\end{align*}
	Essentially, we may treat $\graph$ as both a set in $D$ and in the sense of a graph as in definition \ref{def:DirectedGraph}.
	\item Loops (closed curves) are permitted but require a slightly different treatment to ``regular" edges; and although they don't introduce major complexities in the theory that follows in this chapter, can introduce major complexities in the theory we wish to build on in chapters \ref{ch:ScalarEqns} and \ref{ch:VectorEqns}.
\end{itemize}

NEED to talk to Kirill about this - in particular how do we talk about periodic graphs? Also we might end up with loops in our equivalent QG despite not having them in our embedded graphs.
Also loops do nasty things to our gradients etc if they are in the embedded graphs, we don't consider embedded graphs with loops though.
We do consider QGs though that come from period-cells of periodic graphs, in which case we need to talk about how we associate edges and how we can get loops out of non-loopy period cells.
Essentially need a consistent framework to build off - the definition of embedded graph doesn't use any graph theory so is quite nice, but still need to define ``periodic embedded graph", unit cell, etc.
Once we've done this it should be fine - we only ever work on the period cell which is a graph embedded into (WLOG) $\sqbracs{0,1}^2$ and so our finite graph terminology is sufficient from then on.

Also how we will use derivs with $t$ but still evaluate at $v_j$'s!
\tstk{DEFINE SIM. Throughout this chapter I am going to use $j\conLeft k$ to mean ``$j$ connects to $k$ with $j$ on the left" and $j\conRight k$ to mean ``$j$ connects to $k$ with $j$ on the right. Have defined $j\con k$ for whatever symbol we want to use for ``$j$ connects to $k$, don't care about which side" - may need to go through other chapters to fix this notation and explain what it means in sums with $u_{jk}$, as the subscripts must not involve a direction any more.}

\section{Differential Equations on Quantum Graphs} \label{sec:DEonQG}
Strictly speaking in this section we will be defining differential operators on function spaces that involve quantum graphs, in order to be consistent with several developments in the literature \tstk{refs!!!}.
However we shall see that such operators and the functions in their domains are broken down in such a way that they can be thought of as a system of differential equations on intervals, coupled through (somewhat non-standard) boundary conditions.
As such we will begin this section by defining several function spaces that we wish to work on, then providing examples of the types of boundary conditions that we might want to consider, before finally providing a concrete example of a differential operator on a quantum graph. \newline

Since quantum graphs come with lengths (and intervals) associated to their edges, we can define function spaces on them by combining function spaces on these intervals.
As such we define
\begin{subequations} \label{eq:GraphFuncSpaces}
	\begin{align}
		L^2\bracs{\graph} := \bigoplus_{I_{jk}\in E} \ltwo{\interval{l_{jk}}}{t},
		&\quad H^1\bracs{\graph} := \bigoplus_{I_{jk}\in E} \gradSob{\interval{l_{jk}}}{t}, \\
		H^2\bracs{\graph} := \bigoplus_{I_{jk}\in E} H^2_\mathrm{grad}\bracs{\interval{l_{jk}}, \md t}, &
	\end{align}
\end{subequations}
A function $u\in L^2\bracs{\graph}$ is then determined by it's form on each edge $I_{jk}$ (and similarly for functions and their distributional derivatives in $H^1\bracs{\graph}$).
Because we will mainly be working on the edges of our graphs, we define $u_{jk} = u\vert_{I_{jk}}$ to be the restriction of $u$ to the edge $I_{jk}$, extended by zero to the whole of $\graph$.
Due to the fact that edges are directed, it is also necessary for us to adopt a notion of ``directional derivative" for the $u_{jk}$ at the ends of the edges (\tstk{EKK paper}), and so we adopt the following convention;
\begin{align*}
	\diff{}{t}u_{jk}\bracs{v_j} &= -u'_{jk}\bracs{v_j}, \\
	\diff{}{t}u_{jk}\bracs{v_k} &= u'_{jk}\bracs{v_k}.
\end{align*}
Recall that the subscript $jk$ denotes that the edge $I_{jk}$ is directed from $v_j$ to $v_k$; so our convention is succinctly summarised as ``derivatives directed into a vertex are positive, whilst derivatives directed out of a vertex are negative". \newline \tstk{this is opposite to EKK}

This edge-wise breakdown of our function spaces allows us to define differential operators on $\graph$ by specifying the form of the operator on each edge $I_{jk}$ (by which we mean it's associated interval $\interval{l_{jk}}$).
However it is important to note that the spaces $L^2\bracs{\graph}$ and the other spaces in \eqref{eq:GraphFuncSpaces} do not come with an in-built appreciation for the connectivity of the graph itself; and it is not hard to see that for two graphs with the same number of edges and identical lengths, these spaces will be identical.
Thus to obtain a well-posed problem (strictly speaking, self-adjoint differential operator) on $\graph$, we require additional boundary conditions\footnote{Or matching conditions, or boundary data.} to obtain a unique solution to our problem.
For quantum graphs, these boundary conditions come at the vertices of the graph and we shall be referring to them as vertex conditions.
These conditions come in several types, and there is no requirement that every vertex in a graph has the same conditions imposed at it.
That being said, most of the systems that we will want to be considering will adhere to this, although this is largely due to how we arrive at such systems from our variational framework (see chapters \ref{ch:ScalarEqns} and \ref{ch:VectorEqns}).
The most intuitive vertex condition that we can impose at a given vertex $v_j$ is the requirement that the function $u$ be continuous at $v_j$.
Indeed the construction of the spaces in \eqref{eq:GraphFuncSpaces} does not place any requirement that there be a common value of $u$ at the vertices (as each $u_{jk}$ is an $L^2$-function on a disjoint interval).
If the condition of continuity is imposed at $v_j$ one can then also impose a Kirchoff-like condition on the (directional) derivatives of $u$ at the vertex,
\begin{align*}
	\sum_{j\con k}\diff{u_{jk}}{t}\bracs{v_j} &= \alpha_j u\bracs{v_j}.
\end{align*}
Here $\alpha_j\in\reals$ is a constant that is chosen for the vertex $v_j$, and the value $u\bracs{v_j}$ exists due to the condition of continuity at this vertex.
If continuity of $u$ is not imposed at a vertex, is it still possible to pose conditions that are Kirchoff-like, such as
\begin{align*}
	\sum_{j\con k}u_{jk}\bracs{v_j} &= \alpha_j, &\quad \alpha_j\in\reals, \\
	\sum_{j\con k}u_{jk}\bracs{v_j} &= \alpha_j \sum_{j\con k}\diff{u_{jk}}{t}\bracs{v_j}, &\quad \alpha_j\in\reals.
\end{align*}
These conditions will not be of interest to us, and the theory we present in this section is simply a selection of the more general work of \tstk{references} which deals with this additional generality. \newline

We now provide a simple example of a differential operator $\mathcal{A}$ on $\graph$, however it is not hard to see how the construction can be made general.
First we must provide a domain for $\mathcal{A}$ by deciding how much regularity we want in our functions, and the vertex conditions we want to impose;
\begin{align*}
	\mathrm{dom}\mathcal{A} &= \clbracs{ u\in H^2\bracs{\graph} \ \vert \ u \text{ is continuous at all } v_j\in V, \ \sum_{j\con k}\diff{u_{jk}}{t}\bracs{v_j} = 0 \ \forall v_j\in V}.
\end{align*}
Note that different vertex conditions can be imposed at different vertices by specifying them in the domain of the operator, however to avoid a cumbersome example we have taken identical conditions at each vertex.
The remaining ingredient for $\mathcal{A}$ is what it actually does to functions in it's domain, which is typically done by specifying the action on each edge of $\graph$ (hence the construction of the function spaces in \eqref{eq:GraphFuncSpaces});
\begin{align*}
	\mathcal{A} &= -\diff{}{t} \quad\text{on each } I_{jk}\in E.
\end{align*}
Of course, by ``on each $I_{jk}\in E$" we mean ``on the interval $\interval{l_{jK}}$ that we associate to $I_{jk}\in E$".
Then for a function $f\in L^2\bracs{\graph}$ we can pose the resolvent problem of finding $u\in\mathrm{dom}\mathcal{A}$ such that
\begin{align*}
	\mathcal{A}u &= f;
\end{align*}
or alternatively can consider the spectral problem of finding eigenpairs $\bracs{\lambda,u}\in\complex\times u\in\mathrm{dom}\mathcal{A}$ such that
\begin{align*}
	\mathcal{A}u &= \lambda u.
\end{align*}
As the spaces in \eqref{eq:GraphFuncSpaces} break down into edge-wise components which are acted on individually by $\mathcal{A}$, and only linked through the vertex conditions, we can rewrite both of these problems as a set of ODEs on intervals coupled through vertex conditions;
\begin{align*}
	\mathcal{A}u = f \Leftrightarrow \
	& (i) \ -\diff{u_{jk}}{t} = f_{jk} \ \text{on } \interval{l_{jk}}, \\
	& (ii) \ u \text{ is continuous at each } v_j\in V, \\
	& (iii) \ \sum_{j\con k}\diff{u_{jk}}{t}\bracs{v_j} = 0 \ \forall v_j\in V, \\
	\mathcal{A}u = \lambda u \Leftrightarrow \
	& (i) \ -\diff{u_{jk}}{t} = \lambda u_{jk} \ \text{on } \interval{l_{jk}}, \\
	& (ii) \ u \text{ is continuous at each } v_j\in V, \\
	& (iii) \ \sum_{j\con k}\diff{u_{jk}}{t}\bracs{v_j} = 0 \ \forall v_j\in V.
\end{align*}
We will largely pose differential equations on quantum graphs by specifying the information on the right hand side of these equivalences, as this is where our theory in chapters \ref{ch:ScalarEqns} and \ref{ch:VectorEqns} will take us.
Either of the equivalent forms above will be referred to as a ``quantum graph problem" or a set of ``differential equations on a (quantum) graph" for the purposes of this work.
The reason for us making this equivalence so explicit is because the operator-theoretic approach to quantum graph problems has yielded some useful tools for determining the spectrum of such operators, which we discuss in the following section.

\section{Spectral Problems and the M-Matrix} \label{sec:M-MatrixTheory}
Why do we care so much about the spectral problem?
How do we propose to approach it?
Define M-matrix... YAYAY :(
Y is M-matrix good - computers and shit (NB should we talk about the numerical schemes or just hint at them in the relevant section... or tbh we don't actually use them yet as I've done everything by hand insofar, but might need this if we go the numerical route).

\section{Summary} \label{sec:QGSummary}