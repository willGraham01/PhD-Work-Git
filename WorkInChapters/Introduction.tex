\chapter{Introduction} \label{ch:Intro}
This chapter should essentially be a lit review, whilst also explaining how we are going to treat the physical systems we want to represent mathematically.

\section{Motivation} \label{sec:ProjectMotivation}
Optical fibres are the \textit{de facto} industry standard for large telecommunications systems, thanks to their ability to transmit information quickly and with far less signal loss than other methods (such as metal cables).
The technology has rapidly developed since the first optical fibres were fabricated in the 1970s \cite{knight2003photonic} and optical fibres in use today present a balance between several competing factors to deliver a reliable performance.
Factors such as (optical) loss are inherent, brought about by the materials needed to build the fibres; whilst other factors can be influenced by design (group-velocity dispersion) or the fabrication process (which can lead to imperfections and polarisation effects).
Despite the technological developments of the fibres, the underpinning physical processes remain unchanged --- all improvements to the technology have been incremental and largely centre around the manufacturing process.
The fibre will have a core made of a dielectric (non-conducting) material with a given refractive index and will be surrounded by a cladding; another dielectric material of a lower refractive index.
Typically the difference in refractive indices of the core and cladding is very small \tstk{silica fibres and doping, get some numbers}. 
By choosing a lower refractive index for the cladding material than the core, modes of light\footnote{A mode of light is a mono-frequency solution to the governing equations of electromagnetism in the fibre.} can be confined to the core of an optical fibre via the phenomenon of Total Internal Reflection (TIR) \tstk{reference, cba to explain as it's not important to the report}, allowing guided propagation of light over \tstk{actual distance?} hundreds of kilometres.\newline

Photonic crystal fibres (PCFs) are a departure from the setup of core surrounded by cladding \cite{russell2003photonic}; instead relying on the micro-structure of the photonic crystal to alter the optical properties of the fibre it forms.
This microstructure can cause the crystal to exhibit band-gaps; frequency ranges where there are no propagating modes of light in the crystal, despite the existence of propagating modes at lower (and/or higher) frequencies.
These band-gaps allow for light to be confined to core materials previously thought impossible (like air), or even when the core itself consists of vacuum.
This gives rise to the idea of a ``hollow-core fibre"; a PCF with air (or vacuum) as its core material, guiding light at frequencies determined by the band-gaps of the photonic crystal itself.
For completeness we should also mention that PCFs can also be ``solid core"; that is have a more conventional material like silica as the core material \cite{hou2008metallic}, or even be composed of metallic materials and still confine light \cite{luan2004allsolid}.
Crucially though, PCFs do not use TIR to guide light but rather exploit the fact that light at particular frequencies is confined to the cores simply because it is unable to propagate in the surrounding crystal, due to the optical properties bestowed on the crystal by its micro-structure.
\tstk{illustrative IMAGES of fibre differences! This is better than my sketches!}

In summary, PCFs present a cautiously optimistic improvement to the current industry standard.
The physical theory underlying the process by which they operate means that PCFs have the potential to replace optical fibres as the industry standard.
Whilst this is not a simple case of fabricating better fibres, as highlighted in \cite{knight2003photonic} there are several industry standards that will need to be met before PCFs are accepted and implemented over optical fibres.
However PCFs have an advantage over conventional optical fibres in that their applications are not limited to telecommunications.
Prominent alternative applications include (but are not limited to) non-linear optics (where they offer high optical intensities per unit power, making them highly efficient) and atom and particle guidance (dielectric particles can be guided by the dipole forces exerted by light).
Hence there has been much to motivate study of the optical properties of PCFs; and understand how the design (geometry of the fibre, fabrication material) affects these, resulting in the models based on the Maxwell equations and development of spectral convergence results for periodic problems.
The work in this report is motivated by this application to PCFs, albeit in a slightly convoluted manner.
In section \ref{sec:MathTreatmentPhysSystems} when we detail the systems that we are investigating; which will resemble the structure of PCFs, and also describe (and justify) the assumptions we will be using in our model.
This will bring us through to section \ref{sec:GraphLitReview}; where we will link the system that we have arrived at in section \ref{sec:MathTreatmentPhysSystems} to the PCFs we have described here, before setting the stage for a review of the relevant theory in chapters \ref{ch:QuantumGraphs} and \ref{ch:ScalarEqns}.

\section{Mathematical Treatment of Physical Systems} \label{sec:MathTreatmentPhysSystems}
\tstk{This section will explain how the problems we are considering relate to the physical setups we want to represent - the ``modelling assumptions" slide from BUC-XVI presentation again springs to mind.
Will need to talk about Gelfand \& Fourier transforms and their uses, as well as variational problems by Zhikov.
Kuchment and Olaf \& Post will most likely also crop up here, and we should probably also mention quantum graphs too.
NEED TO FIND THE KUCHMENT STUFF ON CONVERGENCE TO QG, AS CURRENTLY ONLY HAVE HIS BOOK ON QG THEORY.
May even be worth breaking this section into two - one for just the lit review and the other for how we are going about manipulating the physical setup to make it amiable to modelling, and how we will go about this.}

PCFs have been the subject of a number of mathematical models in recent years, and we will elaborate on our chosen approach to modelling these fibres in section \ref{sec:MathTreatmentPhysSystems}. \tstk{as well as this, maybe also reference some of David's papers \cite{birks2006approximate} and discuss use of numerical approaches too}
A typical starting point for established models is the time-harmonic system of Maxwell equations
\begin{align*}
	\grad\wedge\mathbf{E} &= i\omega\mu_{P}\mathbf{H}, &\quad \grad\wedge\mathbf{H} &= -i\omega\varepsilon_{P},\mathbf{E},
\end{align*} 
where $\mathbf{E}$ and $\mathbf{H}$ are the electric and magnetic-displacement fields respectively, and the medium in which the problem is posed has electric permittivity $\varepsilon_{P}$ and magnetic permeability $\mu_{P}$. 
Note that the reason for using the time-harmonic Maxwell system arises from consideration of the fibre structure and seeking ``Bloch wave" solutions via an ansatz of the form $\hat{\mathbf{E}}=\mathbf{E}e^{i\bracs{\wavenumber x_3-\omega t}}$ (and similarly for the magnetic field), where $x_3$ is the axis of the fibre and $\wavenumber$ the component of the wave-vector in $x_3$.
Further assumptions are also made on the fields, namely that the amplitude coefficients $\mathbf{E},\mathbf{H}$ are functions of the fibre cross-section axes variables $\bracs{x_1,x_2}$ only.
This assumption is well-grounded as these models are seeking waves that will propagate down the fibre, and expect the $\mathbf{E}$ and $\mathbf{H}$ fields to vary only in the plane perpendicular to the direction of propagation.
As for the domain of the problem which describes the micro structure and hence the fibre, models assume a unit cell (commonly taken to be the unit square) and periodically extend the domain from this.
The unit cell itself is taken to be composed of two materials with different material constants, which means $\varepsilon_{P}$ and $\mu_{P}$ become piecewise-constant functions of the $\bracs{x_1,x_2}$ variables.
This provides a general model for any kind of periodic structure, rather than the specific geometry we will restrict ourselves to in future.
Analysis of this model and the solution properties can be found in (for example) \cite{cooper2014bandgaps}.
The approach is to apply homogenisation theory due to the periodic structure of the domain, and introduce an appropriate representation of a periodic problem on the unit cell.
One can then prove convergence results for various aspects of the unit-cell problem and the original periodic problem (most notably the spectrum of the problem, which determines the frequencies $\omega$ of light which can propagate). \newline

\tstk{scaling discussion RE David \cite{birks2004scaling}}
The starting point our models also stems from the physical design of PCFs, however we make a number of modelling assumptions to bring us to a formulation different to that of the models described above.
In this section we will discuss the domains or setups that we will be concerned with, and how they relate to physical PCFs.
Of course we also require a system of governing equations to complete our model, however we do not go into detail about this in this section - the issue will be highlighted in section \ref{sec:ModellingAssumption2} and revisited in section \ref{sec:ReportOverview}.
For the time being it is sufficient to assume we have some underlying governing equations that describe wave propagation down a PCF; and we are concerned with determining whether the structure of the PCF admits frequency (equivalently spectral) band-gaps in these equations. \newline

We consider a photonic crystal fibre as an object with a lattice-like structure in the cross sectional (transverse) $\bracs{x_1,x_2}$ plane, which is extruded into the $x_3$ direction to produce a fibre whose structure is translation invariant down it's axis.
The cross-sectional lattice (or structure, as we will refer to it henceforth) will itself be a repeating pattern composed of a number of unit cells, with the thickness of the lattice ``struts" being of some finite (but small) length $l$.
This is schematically sketched in figure \ref{fig:Diagram_ThinStructurePeriodCell}, where for consistency later we have also given a length scale $r$ to lattice ``junction regions".
\begin{figure}[b!]
	\centering
	\begin{subfigure}[t]{0.45\textwidth}
		\centering
		\includegraphics[scale=1.0]{Diagram_ThinStructurePeriodCell}
		\caption{\label{fig:Diagram_ThinStructurePeriodCell} Schematic illustration of the unit cell of a PCF. The photonic crystal is a lattice whose struts have some scale $l$ relative to the cell size of 1. Lattice ``junctions" are also assigned a length scale $r$ for consistency later. The cross section of a PCF is then composed of a (finite) number of these cells stacked in the $\bracs{x_1,x_2}$ plane.}
	\end{subfigure}
	~
	\begin{subfigure}[t]{0.45\textwidth}
		\centering
		\includegraphics[scale=1.0]{Diagram_SingularStructurePeriodCell}
		\caption{\label{fig:Diagram_SingularStructurePeriodCell} The domain which we will consider in our models, obtained by sending both length scales $l$ and $r$ to zero. The nature of the resulting problem depends on the relative scaling between $l$ and $r$, as we shall discuss in section \ref{sec:GraphLitReview}.}
	\end{subfigure}
	\caption{\label{fig:ThinToSingularStructure} An illustration of how the unit cell of a physical PCF relates to that of the systems that we shall be considering. This will be elaborated on in section \ref{sec:GraphLitReview}.}
\end{figure}
From here one can simply impose governing equations on the PCF either in the cross section after imposing a Bloch-wave ansatz or use a numerical scheme such as the Finite Element Method to solve the chosen system of governing equations\footnote{In the context of PCFs, typically it will be Maxwell's equations that need to be solved. These are known to require some special treatment if techniques based on Finite Elements are to be used, but there is an extensive discourse on this in for example \cite{monk2003finite}.}.
Our concern however is mainly with determining whether a certain cross-sectional structure admits frequency band-gaps, and ideally maintaining an analytic approach to this problem for as long as possible, and so we make the following assumptions for our system.

\subsection{Fibres are Invariant along the Fibre Axis} \label{sec:ModellingAssumption1}
We first make the assumption that the PCF is translation invariant along it's fibre axis, aligned to the $x_3$-direction.
This is a standard modelling assumption for fibres in general, unless one wants to incorporate roughness into their model.
For us; determining propagation modes (equivalently band-gaps) is the priority rather than modelling fibre loss due to roughness, so we adopt this assumption.
As a bonus this also allows us to use a Fourier transform in $x_3$ to remove the $x_3$-spatial dependency from the governing equations we choose to adopt, replacing it with factors of the wave-number $\wavenumber$ of the wave propagating down the fibre.

\subsection{Infinite Cross-Sectional Lattice} \label{sec:ModellingAssumption2}
We now move onto assumptions specific to our model, beginning with the assumption that the cross-sectional PCF lattice is infinite and periodic in extent, with a unit cell contained in $\sqbracs{0,1}^2$.
Physically we can justify this assumption because the size of a lattice unit cell is much smaller than the (transverse) extent of the fibre itself.
Mathematically we make this assumption because it allows us to make use of several tools and results from analysis.
Specifically we want to use a Gelfand transform on our governing equations to move from a problem on an infinite-but-periodic domain to a family of problems on the unit cell of said domain.
This family of problems is parametrised by the ``quasi-momentum" $\qm\in[-\pi,\pi)^2$, whose role is to account for integer-wavelength ``mismatches" at the boundary of the period cell.
In this case, combined with the Fourier transform to remove the $x_3$-dependency, this takes us from a problem on full extent of the PCF cross-section to a family of problems on $\sqbracs{0,1}^2$.

\subsection{Singular Structure Lattice} \label{sec:ModellingAssumption3}
Finally, we assume that the lattice in the cross-section is composed of a singular structure, rather than a thin structure.
Physically this assumption can be interpreted as saying that the lattice thickness is much smaller than the unit cell size.
Mathematically, this means that our lattice can be thought of as a graph embedded into $\sqbracs{0,1}^2$; and colloquially we can describe the physical system that as that obtained by setting $l=r=0$ (sketched in \ref{fig:Diagram_SingularStructurePeriodCell}), however there are several things which should be highlighted here.
The foremost being that we now require some form of understanding as to how we can pose (differential) equations on our domain now that it is essentially a one-dimensional subset of the unit cell $\sqbracs{0,1}^2$.
This requires us to consider a variational formulation with respect to non-Lebesgue measures; and construct appropriate function spaces to make sense of the concepts of gradient, curl, and divergence.
This allows us to consider systems such as Maxwell's equations on these singular structures - we will revisit this point in further detail in chapters \ref{ch:ScalarEqns} and \ref{ch:VectorEqns}.
Secondly we should justify why this is helpful to us and how we can expect the results we obtain from this new setup to relate back to those for a thin-structure system.
This is the subject of section \ref{sec:GraphLitReview}, but for now it is enough to say that a framework for working with (differential) equations on graphs already exists and provides us with some further analytical tools that will aid us later (see chapter \ref{ch:QuantumGraphs}).
As for how we maintain relevance to the original thin-structure system; there exist convergence results which justify the use of a singular-structure model as a formal limit of a sequence of thin-structure problems with $l,r\rightarrow0$, although interestingly the relative scaling between these two scales as they tend to 0 can change the resulting ``limit" problem we obtain.
Again, we shall discuss this in section \ref{sec:GraphLitReview}.

\subsection{Resulting System} \label{sec:OurSystem}
Under the assumptions laid out in sections \ref{sec:ModellingAssumption1}-\ref{sec:ModellingAssumption3}, we arrive at the following formulation, which will be the starting point for the problems that we consider throughout this report.
Begin with a fibre invariant along it's axis (parallel to the $x_3$-direction), and having some lattice-like structure in it's cross-section.
We then assume that the cross-sectional lattice is periodic and infinite, occupying the whole $\bracs{x_1,x_2}$-plane; and that is is singular, rather than of a finite but thin thickness, so that we can model it as a graph embedded into the unit cell.
On this singular structure we then pose an appropriate variational problem (chapters \ref{ch:ScalarEqns} and \ref{ch:VectorEqns}), using a combination of Gelfand- and Fourier transforms to examine a family of equivalent problems on the unit cell of the lattice.
The result is a family of variational problems on the unit cell of (the graph describing the) singular lattice which we look to solve using the theory of chapter \ref{ch:QuantumGraphs}.

\section{Singular Structure Problems} \label{sec:GraphLitReview}
you promised a link between singular-structure and thin-structure systems, so you need to give it here (Olaf, Kuchment).
This section is also the lit-review part of chapter 2 (although we can move it across if it doesn't read well), so that needs to be done too!

\section{Overview of Research} \label{sec:ReportOverview}
Essentially a commentary on each section of the report, standard ending to the introductory chapter which is effectively an introduction-summary :L.

This section also needs to provide the link between the physical systems we described in \ref{sec:MathTreatmentPhysSystems}, the lit review in \ref{sec:GraphLitReview}, and chapters 2,3,4.