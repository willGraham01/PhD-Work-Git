\chapter{Introduction} \label{ch:Intro}

\section{Motivation} \label{sec:ProjectMotivation}
Optical fibres are the \textit{de facto} industry standard for large telecommunications systems, thanks to their ability to transmit information quickly and with far less signal loss than other methods (such as metal cables).
The technology has rapidly developed since the first optical fibres were fabricated in the 1970s \cite{knight2003photonic} and optical fibres in use today present a balance between several competing factors to deliver a reliable performance.
Factors such as (optical) loss are inherent, brought about by the materials needed to build the fibres; whilst other factors can be influenced by design (group-velocity dispersion) or the fabrication process (which can lead to imperfections and polarisation effects).
Despite the technological developments of the fibres, the underpinning physical processes remain unchanged --- all improvements to the technology have been incremental and largely centre around the manufacturing process.
The fibre will have a core made of a dielectric (non-conducting) material with a given refractive index and will be surrounded by a cladding, another dielectric material of a lower refractive index.
In practice this is normally achieved using a core material which is simply a doped version of the cladding material, with silica and doped silica are common choices, which leads to typical differences in refractive indices of the core and cladding of around $0.001$.
By choosing a lower refractive index for the cladding material than the core, modes of light\footnote{A mode of light is a mono-frequency solution to the governing equations of electromagnetism in the fibre.} can be confined to the core of an optical fibre via the phenomenon of Total Internal Reflection (TIR), allowing guided propagation of light over tens of kilometres (before a signal boost is required).
Wave guidance in fibres using TIR is known as weakly guiding, and the majority of modern optical fibres utilise this method of guidance. \newline

Photonic crystal fibres (PCFs) are a departure from the setup of core surrounded by cladding \cite{russell2003photonic}; instead relying on the micro-structure of the photonic crystal to alter the optical properties of the fibre it forms.
This microstructure can cause the crystal to exhibit band-gaps; frequency ranges where there are no propagating modes of light in the crystal, despite the existence of propagating modes at lower (and/or higher) frequencies.
These band-gaps allow for light to be confined to core materials previously thought impossible (like air), or even when the core itself consists of vacuum.
This gives rise to the idea of a ``hollow-core fibre"; a PCF with air (or vacuum) as its core material, guiding light at frequencies determined by the band-gaps of the photonic crystal itself.
For completeness we should also mention that PCFs can also be ``solid core"; that is have a more conventional material like silica as the core material \cite{hou2008metallic}, or even be composed of metallic materials and still confine light \cite{luan2004allsolid}.
Crucially though, PCFs do not use TIR to guide light but rather exploit the fact that light at particular frequencies is confined to the cores simply because it is unable to propagate in the surrounding crystal, due to the optical properties bestowed on the crystal by its micro-structure.
\tstk{illustrative IMAGES of fibre differences! This is better than my sketches!}

In summary, PCFs present a cautiously optimistic improvement to the current industry standard.
The physical theory underlying the process by which they operate means that PCFs have the potential to replace optical fibres as the industry standard.
Whilst this is not a simple case of fabricating better fibres, as highlighted in \cite{knight2003photonic} there are several industry standards that will need to be met before PCFs are accepted and implemented over optical fibres.
However PCFs have an advantage over conventional optical fibres in that their applications are not limited to telecommunications.
Prominent alternative applications include (but are not limited to) non-linear optics (where they offer high optical intensities per unit power, making them highly efficient) and atom and particle guidance (dielectric particles can be guided by the dipole forces exerted by light).
Hence there has been much to motivate study of the optical properties of PCFs; and understand how the design (geometry of the fibre, fabrication material) affects these, resulting in the models based on the Maxwell equations and development of spectral convergence results for periodic problems.
The work in this report is motivated by this application to PCFs, albeit in a slightly convoluted manner.
In section \ref{sec:OurPhysicalSetup} we detail the systems that we are investigating; which will resemble the structure of PCFs, and also describe (and justify) the assumptions we will be using in our model.
This will bring us through to section \ref{sec:GraphLitReview}; where we will link the system that we have arrived at in section \ref{sec:OurPhysicalSetup} to the PCFs we have described here, before setting the stage for a review of the relevant theory in chapters \ref{ch:QuantumGraphs} and \ref{ch:ScalarEqns}.

\section{Exisiting Models for PCFs} \label{sec:ExistingPCFModels}
PCFs have been the subject of a number of models in recent years. 
The most conceptually straightforward being the use of numerical techniques to solve Maxwell's equations
\begin{align*}
	\grad\wedge\mathbf{E} &= \pdiff{\bracs{\mu_{P}\mathbf{H}}}{t}, 
	&\quad \grad\wedge\mathbf{H} &= -\pdiff{\bracs{\varepsilon_{P}\mathbf{E}}}{t},	
\end{align*} 
to determine modes (single-frequency solutions) that are supported by the PCF.
Here $\mathbf{E}$ and $\mathbf{H}$ are the electric and magnetic-displacement fields respectively; with $\varepsilon_{P}$ and $\mu_{P}$ being material parameters (the electric permittivity and magnetic permeability respectively).
These modes can then be used to construct an approximate band-gap plot for the fibre, with the level of precision in the computational results coming at the cost of increasing the computing time.
Whilst this can provide detailed information about a fibre, it does not provide any insight into how the structure of the PCF has contributed or affected the resulting band-gap plot.
Furthermore numerical schemes (particularly those based off finite elements) are known to require special treatment when being used to solve Maxwell's equations, but this in itself has been studied in detail (see for example \cite{monk2003finite}).
To avoid the need for complex numerical solvers, models that make simplifying approximations have been proposed.
Works such as \tstk{citations, follow refs 9,10,\cite{laegsgaard2004gap} in \cite{birks2006approximate}} have seen some success in retaining the key features of band-gap plots; by considering the physical origins of band-gaps in fibres as arising from the resonant properties of the cores (treated as rods).
These largely consider the origins of band-gaps to stem from the resonant properties of these cores; in particular the spectral bands of a PCF are taken to correspond to modes that couple between the cores.
As a result only information about the modes of a single core, plus the relative core spacing and size, are needed to build an approximate band-gap plot.
These methods have made several insights into the origins of band-gap structures, but are limited by their approximations in the information they provide about the fibres themselves.
Intermediate approaches have also been proposed, for example in \cite{birks2006approximate} which proposes a model that considers the sizes of the regions separating the cores in the PCF, and provides alternative boundary conditions for when the unit cell of the fibre cross-section is hexagonal or circular.
The method eventually relies on a numerical scheme; but only to the point of root-finding for analytic expressions, and produces band-gap plots that are in good agreement with the more complete numerical models provided certain scaling between parameters is adhered to.
This makes such approaches useful when speed is more important than high accuracy, or when an intuitive picture of the band-structure is required. \newline

Alternative approaches to modelling band-gap structures use the time-harmonic system of Maxwell equations,
\begin{align*}
	\grad\wedge\mathbf{E} &= i\omega\mu_{P}\mathbf{H}, &\quad \grad\wedge\mathbf{H} &= -i\omega\varepsilon_{P}\mathbf{E},
\end{align*} 
as a starting point.
Time-harmonic Maxwell system arises from seeking ``Bloch wave" solutions in the fibre via an ansatz of the form $\hat{\mathbf{E}}=\mathbf{E}e^{i\bracs{\wavenumber x_3-\omega t}}$ (and similarly for the magnetic field), where $x_3$ is the axis of the fibre and $\wavenumber$ the component of the wave-vector in $x_3$.
Equivalently one can view this as taking a Fourier transform along the axis of the fibre.
Further assumptions are also made on the fields, namely that the amplitude coefficients $\mathbf{E},\mathbf{H}$ are functions of the fibre cross-sectional axes variables $\bracs{x_1,x_2}$ only.
This assumption is well-grounded as these models are seeking waves that will propagate down the fibre, and expect the $\mathbf{E}$ and $\mathbf{H}$ fields to vary only in the plane perpendicular to the direction of propagation.
As for the domain of the problem which describes the micro structure and hence the fibre, models assume a unit cell (commonly taken to be the unit square) and periodically extend the domain from this.
The unit cell itself is taken to be composed of two materials with different material constants, which means $\varepsilon_{P}$ and $\mu_{P}$ become piecewise-constant functions of the $\bracs{x_1,x_2}$ variables.
This provides a general model for any kind of periodic structure, rather than the specific geometry we will restrict ourselves to in future.
Analysis of this model and the solution properties can be found in (for example) \cite{cooper2014band}. \tstk{doesn't Kirill have a paper on this too?}
The approach is to apply homogenisation theory due to the periodic structure of the domain, and introduce an appropriate representation of a periodic problem on the unit cell.
One can then prove convergence results for various aspects of the unit-cell problem and the original periodic problem (most notably the spectrum of the problem, which determines the frequencies $\omega$ of light which can propagate). \newline
\tstk{scaling discussion RE David \cite{birks2004scaling}}

\section{Domain Structure for Our Model} \label{sec:OurPhysicalSetup}
In this section we will discuss the domains or setups that we will be concerned with, and how they relate to physical PCFs.
The purpose of our model is to provide insight into the band-gap structure of PCFs, by considering an appropriate ``limit problem" for the cross-sectional structure.
We will elaborate further on what the phrase ``limit problem" means in section \ref{sec:GraphLitReview}, but for now we just outline the domains (or the physical structures) that our model will consider.
These domains are motivated by the physical design of PCFs, however we make a number of modelling assumptions about fibre structure to bring us to a formulation different to that of the models described in section \ref{sec:ExistingPCFModels}.
Of course we also require a system of governing equations to complete our model, however we do not go into detail about this in this section - the issue will be highlighted in section \ref{sec:ModellingAssumption2} and revisited in section \ref{sec:ReportOverview}.
For the time being it is sufficient to assume we have some system of governing equations that describe wave propagation down a PCF; and we are concerned with determining whether the structure of the PCF admits frequency (equivalently spectral) band-gaps in these equations. \newline

We consider a PCF as an object with a lattice-like structure in the cross-sectional (transverse) $\bracs{x_1,x_2}$ plane, which is extruded into the $x_3$ direction to produce a fibre whose structure is translation invariant down it's axis.
The cross-sectional lattice (or structure, as we will refer to it henceforth) will itself be a repeating pattern composed of a number of unit cells, with the thickness of the lattice ``struts" being of some finite (but small) width $l$.
This is schematically sketched in figure \ref{fig:Diagram_ThinStructurePeriodCell}, where for consistency later we have also given a length scale $r$ to lattice ``junction regions"\footnote{There is some ambiguity in specifying the shape of a ``junction region"; but it is sufficient to simply think of these as regions in our domain where lattice struts overlap.
In figure \ref{fig:Diagram_ThinStructurePeriodCell} for example, we have taken the ``junction regions" as circular, however one could also take them as square (or as slightly smaller circles, even).
This is not a major issue as the convergence results of \cite{exner2005convergence} and \tstk{kuchment convergence stuff} rely only on how the area of these ``junction regions" scales with powers of $r$.}.
\begin{figure}[b!]
	\centering
	\begin{subfigure}[t]{0.45\textwidth}
		\centering
		\includegraphics[scale=1.0]{Diagram_ThinStructurePeriodCell}
		\caption{\label{fig:Diagram_ThinStructurePeriodCell} Schematic illustration of the unit cell of a PCF. The photonic crystal is a lattice whose struts have some scale $l$ relative to the cell size of 1. Lattice ``junctions" are also assigned a length scale $r$ for consistency later. The cross section of a PCF is then composed of a (finite) number of these cells stacked in the $\bracs{x_1,x_2}$ plane.}
	\end{subfigure}
	~
	\begin{subfigure}[t]{0.45\textwidth}
		\centering
		\includegraphics[scale=1.0]{Diagram_SingularStructurePeriodCell}
		\caption{\label{fig:Diagram_SingularStructurePeriodCell} The domain which we will consider in our models, obtained by sending both length scales $l$ and $r$ to zero. The nature of the resulting problem depends on the relative scaling between $l$ and $r$, as we shall discuss in section \ref{sec:GraphLitReview}.}
	\end{subfigure}
	\caption{\label{fig:ThinToSingularStructure} An illustration of how the unit cell of a physical PCF relates to that of the systems that we shall be considering. This will be elaborated on in section \ref{sec:GraphLitReview}.}
\end{figure}
Our concern is mainly with determining whether a certain cross-sectional structure admits frequency band-gaps, and ideally maintaining an analytic approach to this problem for as long as possible, and so we make the following assumptions for our system.

\subsection{Fibres are Invariant along the Fibre Axis} \label{sec:ModellingAssumption1}
We first make the assumption that the PCF is translation invariant along it's fibre axis, aligned to the $x_3$-direction.
This is a standard modelling assumption for fibres in general unless surface roughness of the fibre needs to be incorporated.
Since our objective is determining propagation modes (equivalently band-gaps) rather than modelling fibre loss due to roughness, we too adopt this assumption.
This enables us to use a Fourier transform in $x_3$ to remove the $x_3$-spatial dependency from the governing equations we choose to adopt, replacing it with factors of the wave-number $\wavenumber$ of the wave propagating down the fibre.
Furthermore we are now considering a model that on a 2D-domain rather than a 3D one (although if we are considering the Maxwell system, the fields $\mathbf{E}, \mathbf{H}$ would still be vector-valued).

\subsection{Infinite Cross-Sectional Lattice} \label{sec:ModellingAssumption2}
We now move onto assumptions rather specific to our model, beginning with the assumption that the cross-sectional PCF lattice is infinite and periodic in extent, with a unit cell contained in $\sqbracs{0,1}^2$.
Physically we can justify this assumption because the size of a lattice unit cell is much smaller than the (transverse) extent of the fibre itself.
Mathematically we make this assumption because it allows us to make use of several tools and results from analysis.
Specifically we want to use a Gelfand transform on our governing equations to move from a problem on an infinite-but-periodic domain to a family of problems on the unit cell of said domain.
Formally the Gelfand transform is used to provide a fibre representation of an operator defined on a periodic domain, and the theory surrounding it is much more extensive than what we require it for, but the following interpretation is sufficient for us.
For a function $u:\reals^{d_1}\rightarrow\complex^{d_2}$ defined on a periodic domain with unit cell $\sqbracs{0,1}^{d_1}$, the Gelfand transform can be thought of as mapping $u$ to the family of functions $\widehat{u}$ defined by
\begin{align*}
	\widehat{u}: &\sqbracs{0,1}^{d_2} \rightarrow \complex^{d_2}, \\
	\widehat{u}\bracs{x} &= \sum_{n\in\integers^{d_1}} u\bracs{x + n}e^{-i\qm\bracs{x + n}}, \\
	\qm &\in [-\pi,\pi)^{d_1},
\end{align*}
the parameter $\qm$ is called the quasi-momentum.
Now if we have some system of differential equations in $u$ on $\reals^{d_1}$, this transform effectively takes us from a problem involving $u$ on the entire domain $\reals^{d_1}$ to a family of slightly altered problems on the unit cell ($\sqbracs{0,1}^{d_1}$), for each $\qm$.
The extent of the alteration (to the equations in this context) is effectively applying a shift to any derivative we take, formally any partial derivative transforms as
\begin{align*}
	\pdiff{}{x_j} &\rightarrow \pdiff{}{x_j} - i\qm_j,
\end{align*}
the purpose of this shift being to account for integer-``mismatches" of $\widetilde{u}$ at the boundary of the period cell.
If we are looking at a spectral problem in $\reals^{d_1}$, we can recover the spectrum of the original problem by solving each spectral problem we obtain from the Gelfand transform, and taking the union of all the spectra over $\qm$. \newline

In our case; with our cross-section being two dimensional and infinite in extent, we can obtain a family of spectral problems on the (finite) unit cell and solve these instead (for each $\qm$) to recover the spectrum of our original problem.
Combined with the Fourier transform to remove the $x_3$-dependency, this takes us from a problem on an infinite 3D domain to a family of problems on $\sqbracs{0,1}^2$.
Having a finite domain is appealing from a numerical perspective for obvious practical reasons, but will also be useful to us analytically after considering the final assumption in section \ref{sec:ModellingAssumption3}.

\subsection{Singular Structure Lattice} \label{sec:ModellingAssumption3}
Finally, we assume that the lattice in the cross-section is composed of a singular structure, rather than a thin structure.
Physically this assumption can be interpreted as saying that the lattice thickness is much smaller than the unit cell size.
Mathematically this means that our lattice can be thought of as a graph embedded into $\sqbracs{0,1}^2$ (chapter \ref{ch:QuantumGraphs}); and colloquially we can describe the physical system that as that obtained by letting $l,r\rightarrow0$ (sketched in \ref{fig:Diagram_SingularStructurePeriodCell}), however there are several things which should be highlighted here.
The foremost being that we now require some form of understanding as to how we can pose (differential) equations on our domain now that it is essentially a one-dimensional subset of the unit cell $\sqbracs{0,1}^2$.
This requires us to consider a variational formulation with respect to non-Lebesgue measures; and construct appropriate function spaces to make sense of the concepts of gradient, curl, and divergence.
We will discuss the literature that is available to us in this field in section \ref{sec:VariationalProblemLitReview}, and will demonstrate how variational formulations allow us to consider wave-propagation problems on singular structures in further detail in chapters \ref{ch:ScalarEqns} and \ref{ch:VectorEqns}. \newline

Secondly we should justify why this is helpful to us and how we can expect the results we obtain from this new setup to relate back to those for a thin-structure PCF.
This is the subject of section \ref{sec:GraphLitReview}, but for now it is enough to say that a framework for working with (differential) equations on graphs already exists and provides us with some further analytical tools that will aid us later (see chapter \ref{ch:QuantumGraphs}).
As for how we maintain relevance to the original thin-structure system; there exist convergence results which justify the use of a singular-structure model as a formal limit of a sequence of thin-structure problems with $l,r\rightarrow0$, although interestingly the relative scaling between these two scales as they tend to 0 can change the resulting ``limit" problem we obtain.
Again, we shall discuss this in section \ref{sec:GraphLitReview}.

\subsection{Resulting System} \label{sec:OurSystem}
We can summarise the affects of the assumptions in sections \ref{sec:ModellingAssumption1}-\ref{sec:ModellingAssumption3} as follows.
We begin with a fibre invariant along it's axis (parallel to the $x_3$-direction), and having some lattice-like structure in it's cross-section.
We then assume that the cross-sectional lattice is periodic and infinite, occupying the whole $\bracs{x_1,x_2}$-plane; and that is is singular, rather than of a finite but thin thickness, so that we can model it as a graph embedded into the unit cell.
The link between the graph structure and the corresponding PCF cross-section is discussed in section \ref{sec:GraphLitReview}.
On this singular structure we will then pose an appropriate variational problem (section \ref{sec:VariationalProblemLitReview}), using a combination of Gelfand- and Fourier transforms to examine a family of equivalent problems on the unit cell of the lattice.
The result is a family of variational problems on the unit cell of (the graph describing the) singular lattice which we look to solve using the theory of chapter \ref{ch:QuantumGraphs}.
We will assume this domain structure throughout the remainder of the report, and will use it as our starting point when we consider examples in chapter \ref{ch:ExampleSystems}.
In chapters \ref{ch:QuantumGraphs}, \ref{ch:ScalarEqns}, \ref{ch:VectorEqns}, we will simply take a graph and variational formulation as our starting point for the theory that is presented in those sections, with the underlying relevance to wave-guidance problems in PCFs having been detailed here.

\section{Variational Problems for Singular Structures} \label{sec:VariationalProblemLitReview}
With the assumption in section \ref{sec:ModellingAssumption3} taking us to a domain which is essentially a 1D object embedded into a 2D plane, we need to address several issues that arise from this transition.
For clarity of terminology; we will refer to the domains we arrive at as singular-structure domains, or just singular-structures.
Whilst visualising the domain we arrive at is not taxing, determining what happens to the boundary conditions and governing equations from our original, thin-structure problem needs addressing.
In terms of the equations themselves; we need to justify what happens to concepts like gradients, curls and divergence now that we are working on a lower-dimensional structure.
This then relates to the problems involving the boundary conditions; for systems like Maxwell's equations one typically imposes conditions on the derivates of the normal- and tangential-field components at the domain boundary, but without a concept of gradient, and also a different (lower-dimensional) boundary, what form should these conditions take? \newline

The solution to these issues is to take a step back from the original equations, and consider it in a variational (or weak) form.
For example; rather than considering the problem of finding some function $u$ such that 
\begin{align*}
	-\grad\cdot\grad u &= f, \quad\text{ in } D\subset\reals^2, \\
	\pdiff{u}{n}\big\vert_{\partial D} &= 0,
\end{align*}
for an appropriate domain $D$ and forcing function $f$, we consider it's variational form
\begin{align*}
	\integral{D}{\grad u \cdot \grad \phi}{x} &= \integral{D}{f\phi}{x}, \quad\forall \phi\in V_{\mathrm{test}},
\end{align*}
for some appropriate set of test functions $V_{\mathrm{test}}$.
Formally integrating by parts in the variational formulation then produces the original second-order differential equation plus it's boundary conditions.
For thin-structure problems considering the variational form of a problem is a standard step in the derivation of numerical schemes like the Finite Element Method.
However for us it provides us with an idea as to how we can produce a similar problem on a singular-structure; change the measure that the integration in the variational problem is with respect to, so that it respects the singular-structure of our domain.
This doesn't come without costs; we now need to produce some appropriate spaces for our functions and their gradients, curls, and divergences to live in (not to mention still need to qualify what we mean by these concepts), and we also throw away several techniques - notably integration by parts.
However we can at least pose a consistent variational problem in this setting, and we will later demonstrate that these retain a connection to PCF models. \newline

In terms of work that has already been done regarding variational problems posed with respect to non-Lebesgue measures, there has been a lot of activity in the context of the equations of elasticity.
Notably the work in \cite{zhikov2000extension} lays a foundation for considering such problems in the context of elasticity, defining the appropriate function spaces and concepts of (symmetric) gradients.
This work will also serve largely as the motivation for the work in chapter \ref{ch:VectorEqns}, when we look to study our singular-structure variational problems (a specific choice of measure) and define the appropriate function spaces in a wave-propagation context.
Also presented in this work are some preliminary results on whether such variational problems are well-posed, including existence results for solutions.
The work in \cite{zhikov2002homogenization} further develops on this, by bringing techniques from homogenisation theory into the framework of these (elasticity-based) variational problems and discussing how they are adapted to be applied in these contexts. 
Outside of the context of elasticity; work has been done on singular-structure problems for scalar-valued functions, for example in \cite{cherednichenko2018elliptic} where operator-norm estimates are derived from elliptic equations on (scaled) periodic domains. 
In the context of electromagnetic wave-guidance problems, there have been similar developments for solution estimates for Maxwell's equations on periodic singular structures, such as in \cite{cherednichenko2018maxwell}. \tstk{more refs surely?}

\section{Quantum Graphs, and their relation to Thin-Structure Problems} \label{sec:GraphLitReview}
Quantum graphs have existed in some form (although not necessarily under that name) since the 1930s \cite{berkolaiko2013introduction} as surrogate models for various processes in chemistry, physics (including optical fibres) and mathematics (\tstk{kirill's elasticity paper with QGs?}).
However there has only recently been a push to develop a standard basis for the theory in the past few decades, with works produced before this time tending to focus on specific examples relevant to the application being considered.
At heart, quantum graph are graphs that are equipped with some concept of length or bulk; and quantum graph problems are simply a framework for (differential) operators on such graphs.
This is a departure from combinatorial graphs, where the vertices of the graph are the significant features and the edges merely represent some connections or processes.
Instead quantum graph problems focus largely on the lengths (and thus ``physical space") occupied by the edges, with the role of the vertices being reduced to that analogous to a boundary in more familiar differential equation settings.
A comprehensive introduction to the field can be found in \cite{berkolaiko2013introduction}, in which the main concepts and techniques for quantum graphs are gathered.
There have been several further advances in the theory of quantum graphs that will be relevant to our work though, notably surrounding analysis of spectral problems. 
It has been shown that by utilising the theory of boundary triples \tstk{refs!} and in particular an object called the Weyl-Titchmarsh M-function, spectral problems on quantum graphs can be drastically reduced in complexity \tstk{something like EKK \cite{ershova2014isospectrality}, and the other EKK? \cite{ershova2016isospectrality}}, or at least reframed in a manner more open to analysis.
This has opened up routes for exploration of resolvent and spectral properties of quantum graph problems, as well as producing tools and techniques that aid in the solution to such problems on a practical level.
Indeed with enough information about the spectral problem to be solved and the underlying quantum graph, it is possible to reduce the task of determining the spectrum of a (differential) operator to that of finding the eigenvalues of a finite matrix\footnote{This matrix itself being the particular form of the Weyl-Titchmarsh M-function in the context of quantum graphs}.
This last development is of particular importance to us; as it provides a tool that can take us to the solution of any (spectral) quantum graph problems that we come across, and opens us potential for numerical schemes should exact analysis prove difficult.
We will highlight this idea again in section \ref{sec:M-MatrixTheory}, when we have presented some of the theory of quantum graphs, and discuss it in more detail in section \ref{sec:NumericalMethodsDiscussion}.
For now; having provided some background on quantum graph problems we now look to highlight the link between them, the structures that arise out of our modelling assumptions, and how they retain relevance to models for PCFs. \newline

The modelling assumptions in sections \ref{sec:ModellingAssumption1}-\ref{sec:ModellingAssumption3} move us away from a structure that can be physically fabricated and into a singular-structure domain instead.
An important question to address at this point is whether the resulting singular-structure problem is still relevant to the original thin-structure problem.
It is not hard to believe that there will be some link between quantum graph problems and the singular-structure problems that we are going to propose, given that quantum graphs provide a framework for working on graphs with ``physical presence", which is precisely what our lattice-structures have been reduced to.
The exact nature of how one moves from the singular-structure (variational) problem to a quantum graph problem is covered in chapters \ref{ch:ScalarEqns} and \ref{ch:VectorEqns}, so we do not elaborate further here.
However it is important that we provide the link (hinted at in section \ref{sec:ModellingAssumption3}) between the original thin-structure problem modelling a PCF and quantum graph problems.
This link largely comes about due to the work of \cite{exner2005convergence} and \tstk{kuchment convergence stuff}.
These works are presented in a setting far more abstract than that which we require; in particular the setting of \cite{exner2005convergence} proves convergence results for graph-like manifolds, so we restrict ourselves to discussing the implications for work in our context.
Being somewhat blas\'e, we can summarise the link between thin-structure problems and quantum graphs by saying that the quantum graph problems are (effectively) limits of thin-structure problems (figure \ref{fig:ThinToSingularStructure}).
A more detailed explanation is as follows; suppose we have some connected, lattice-like domain $D$ such as that in figure \ref{fig:Diagram_ThinStructurePeriodCell}, and with two length scales $l$ and $r$ which capture the characteristic thickness/size of the lattice struts and size of the junction regions respectively.
For each $l,r>0$ assume that we pose a system of differential equations on our domain $D$\footnote{For clarity, these being the \textit{same} system of equations for each $\bracs{l,r}$ pair, only the size of the domain is changing}, and we are interested in computing the eigenvalues of this system.
Denote this problem by $\mathcal{P}_{l,r}$, and the resulting spectrum by $\clbracs{\sigma}_{l,r}$.
One can then ask the question of what happens to $\clbracs{\sigma}_{l,r}$ as $l,r\rightarrow0$; the answer being that $\clbracs{\sigma}_{l,r}$ converges\footnote{In an appropriate notion of convergence for sets.} to the spectrum of a quantum graph problem.
We denote this quantum graph problem by $\mathcal{P}_0$ and it's spectrum by $\clbracs{\sigma}_0$.
This means that in the limit as the thin-structure becomes increasingly fine, the spectrum of a problem on the thin structure approaches the spectrum given by a quantum graph problem.
By taking the assumptions in sections \ref{sec:ModellingAssumption1}-\ref{sec:ModellingAssumption3}, we are making the approximation that our fibre's microstructure is of a scale small enough to be approximated by this quantum graph problem $\mathcal{P}_0$. \newline

Interestingly the problem $\mathcal{P}_0$ and thus the spectrum $\clbracs{\sigma}_0$ depends on the relative scaling of $l$ and $r$ \cite{exner2005convergence} \tstk{sasha said that this is actually not vertex area - vol. area but rather something to do with the contact area of the edges and vertices.}.
As $l$ quantifies the size of the lattice struts, which in the limit $l\rightarrow0$ become graph edges, $l$ effectively determines how fast the edges are shrinking.
So for each $l>0$ we can define an ``edge area" $V_{\mathrm{edge}} \propto l^{\alpha}$ for some power $\alpha$, which decays to 0 as $l\rightarrow0$.
There is a similar concept of ``vertex area" $V_{\mathrm{vert}} \propto r^{\beta}$ which $r$ provides.
Then depending on the relative size of $V_{\mathrm{edge}}$ to $V_{\mathrm{vert}}$ as $l,r\rightarrow0$ simultaneously; the problem $\mathcal{P}_0$, or rather the vertex conditions in this problem (see section \ref{sec:DEonQG}) are different.
\begin{itemize}
	\item In the case when $V_{\mathrm{edge}} \ll V_{\mathrm{vert}}$, the vertices dominate in the limit problem.
	This case is also referred to as ``Dirichlet decoupling" - the result is a quantum graph problem where the boundary conditions at the vertices require that the solution take the value 0 at each of these points.
	This makes $\mathcal{P}_0$ merely a system of independent (that is, decoupled) ODEs on the edges of the graph.
	Intuitively we can think of this as saying that; with the edges decaying faster than the vertices, the vertices are large enough to prevent any interaction between any edges that share a common vertex.
	\item In the case when $V_{\mathrm{edge}} \gg V_{\mathrm{vert}}$, the edges dominate in the limit problem.
	This results in a quantum graph problem $\mathcal{P}_0$ where the boundary conditions at the vertices  are strict matching conditions.
	That is; the functions we are solving for are required to be continuous across at each vertex, and derivatives of functions at the vertex are required to sum to zero (the coupling constant $\alpha=0$ in a Kirchoff condition).
	Intuitively it can be thought that $V_{\mathrm{vert}}$ decaying faster than $V_{\mathrm{edge}}$ results in the vertices disappearing from the system, and with nothing to separate the edges, function values must match where there is overlap.
	\item In the case when $V_{\mathrm{edge}} \approx V_{\mathrm{vert}}$, we arrive at a quantum graph problem that falls between the other two cases.
	Namely the boundary conditions at the vertices still require continuity of the solution at the vertices, but conditions on the derivates are relaxed.
	Instead one can obtain Kirchoff-like conditions at the vertices for the derivates; and the coupling constants that appear in these conditions are related to the ratio of $V_{\mathrm{edge}}$ to $V_{\mathrm{vert}}$.
	In this case we think of the vertices and edges shrinking at a rate which keeps them of comparable size, so the effect of decoupling the solutions (due to the vertex area) is balanced out by the need for consistency across the vertices (the effect of the edge area).
\end{itemize}

This neatly ties together the variational problems that we will consider as a result of our assumptions in sections \ref{sec:ModellingAssumption1}-\ref{sec:ModellingAssumption3}, quantum graph problems, and their relevance to thin-structure PCF models.
The variational problems give provide us with equivalent quantum graph problems, which serve as approximations to PCF structures in the limit of small PCF cross-sectional microstructure.
Chapter \ref{ch:QuantumGraphs} elaborates on the ideas that have been presented in our review of the existing theory for quantum graphs, providing the formal statement of any definitions and results that we require for our work.
In chapters \ref{ch:ScalarEqns} and \ref{ch:VectorEqns} we will demonstrate how a variational problem on a singular structure is equivalent to a quantum graph problem that coincides with the $V_{\mathrm{edge}} \gg V_{\mathrm{vert}}$ case; whilst in chapter \ref{ch:ExampleSystems} we will briefly explain how to obtain a quantum graph problem that coincides with the $V_{\mathrm{edge}} \approx V_{\mathrm{vert}}$ case.

\section{Overview of Research} \label{sec:ReportOverview}
In this chapter we have provided some motivation for the kinds of problems we will be considering later, whilst also touching on several of the advantages and nuances of taking this approach.
How the topics discussed in this section fit together, including what we will be pursuing in this report, is illustrated in figure \ref{fig:Diagram_RelationBetweenMathematicalFields}.
\begin{figure}[b!]
	\centering
	\includegraphics[scale=0.75]{Diagram_RelationBetweenMathematicalFields.pdf}
	\caption{\label{fig:Diagram_RelationBetweenMathematicalFields} A visualisation of the topics covered in this chapter, how they link together and the research which we shall pursue.}
\end{figure}
Current models for PCFs fall broadly into three categories (section \ref{sec:ExistingPCFModels}):
\begin{itemize}
	\item Numerical schemes that look to solve systems of governing equations for a specific fibre geometries.
	These can provide accurate band-gap plots for the fibre geometry of interest, but do not provide any general insights into how the influence of the various PCF parameters.
	\item Analytical models that make several simplifying assumptions have had success in replicating broad features of PCF band-gap plots.
	However they are limited in the information they can provide about the fibres themselves.
	\item Intermediate models have been proposed that aim to strike a balance between these two approaches, which have seen success and do not suffer (to the same extent) the weaknesses of the other models.
\end{itemize}
The work in this report aims to study variational problems that can be seen as approximations to PCFs; looking to work analytically to provide insight into how the cross-sectional geometry of a fibre influences it's spectral band-gaps, and proposing numerical approaches when an analytic approach becomes unfeasible. \newline

We have highlighted that quantum graph problems naturally arise as a limit of thin-structure problems (section \ref{sec:GraphLitReview}); which solidifies the link between the singular-structure variational problems we will consider and thin-structure PCF models, through quantum graph problems.
By utilising developments in the theory of boundary triples and the existing theory for quantum graphs (section \ref{sec:GraphLitReview}) we can make the task of solving quantum graph problems easier still, and potentially open to rather general numerical schemes (section \ref{sec:NumericalMethodsDiscussion}).
The quantum graphs framework will also allow us to pursue analytical analysis of the problems we want to consider, helping us to strike a balance between modelling specific fibre geometries and providing more general insights into the effect of fibre geometry on the spectrum.
We will present the theory on quantum graphs that we require in chapter \ref{ch:QuantumGraphs}. \newline

The starting point for much of our work will be singular-structure variational problems, however this starting point is motivated by physical (thin-) structures of PCFs and use of the assumptions in section \ref{sec:ModellingAssumption1}-\ref{sec:ModellingAssumption3}.
It is helpful to consider these modelling assumptions though, as doing so raises several of the issues that will need to be addressed to maintain the relation between our work and physical PCFs.
In particular, we are required to adopt a variational framework for our model that respects the lower-dimensional singular-structure we obtain (section \ref{sec:VariationalProblemLitReview}.
In turn this requires some analysis of how we can interpret concepts like gradient, curl, and divergence, which is the subject of chapters \ref{ch:ScalarEqns} and \ref{ch:VectorEqns}.
We also look to demonstrate the equivalence between singular-structure problems and quantum graph problems in these chapters, and in chapter \ref{ch:ExampleSystems} will consider some examples that arise from problems in wave-guidance. 
Chapter \ref{ch:Conclusion} provides a summary of the work presented and discusses further avenues for exploration, and future work that could be undertaken.