\chapter{Introduction} \label{ch:Intro}

\section{Motivation} \label{sec:ProjectMotivation}
Optical fibres are the \textit{de facto} industry standard for large telecommunications systems, thanks to their ability to transmit information quickly and with far less signal loss than other methods (such as metal cables).
The technology has rapidly developed since the first optical fibres were fabricated in the 1970s \cite{knight2003photonic} and optical fibres in use today present a balance between several competing factors to deliver a reliable performance.
Factors such as (optical) loss are inherent, brought about by the materials needed to build the fibres; whilst other factors can be influenced by design (group-velocity dispersion) or the fabrication process (which can lead to imperfections and polarisation effects).
Despite the technological developments of the fibres, the underpinning physical processes remain unchanged --- all improvements to the technology have been incremental and largely centre around the manufacturing process.
The fibre will have a core made of a dielectric (non-conducting) material with a given refractive index and will be surrounded by a cladding, another dielectric material of a lower refractive index.
In practice this is normally achieved using a core material which is simply a doped version of the cladding material, with silica and doped silica are common choices, which leads to typical differences in refractive indices of the core and cladding of around $0.001$.
By choosing a lower refractive index for the cladding material than the core, modes of light\footnote{A mode of light is a mono-frequency solution to the governing equations of electromagnetism in the fibre.} can be confined to the core of an optical fibre via the phenomenon of Total Internal Reflection (TIR), allowing guided propagation of light over tens of kilometres (before a signal boost is required).
Wave guidance in fibres using TIR is known as weakly guiding, and the majority of modern optical fibres utilise this method of guidance. \newline

Photonic crystal fibres (PCFs) are a departure from the setup of core surrounded by cladding \cite{russell2003photonic}; instead relying on the micro-structure of the photonic crystal to alter the optical properties of the fibre it forms.
This microstructure can cause the crystal to exhibit band-gaps; frequency ranges where there are no propagating modes of light in the crystal, despite the existence of propagating modes at lower (and/or higher) frequencies.
These band-gaps allow for light to be confined to core materials previously thought impossible (like air), or even when the core itself consists of vacuum.
This gives rise to the idea of a ``hollow-core fibre"; a PCF with air (or vacuum) as its core material, guiding light at frequencies determined by the band-gaps of the photonic crystal itself.
For completeness we should also mention that PCFs can also be ``solid core"; that is have a more conventional material like silica as the core material \cite{hou2008metallic}, or even be composed of metallic materials and still confine light \cite{luan2004allsolid}.
Crucially though, PCFs do not use TIR to guide light but rather exploit the fact that light at particular frequencies is confined to the cores simply because it is unable to propagate in the surrounding crystal, due to the optical properties bestowed on the crystal by its micro-structure.
\tstk{illustrative IMAGES of fibre differences! This is better than my sketches!}

In summary, PCFs present a cautiously optimistic improvement to the current industry standard.
The physical theory underlying the process by which they operate means that PCFs have the potential to replace optical fibres as the industry standard.
Whilst this is not a simple case of fabricating better fibres, as highlighted in \cite{knight2003photonic} there are several industry standards that will need to be met before PCFs are accepted and implemented over optical fibres.
However PCFs have an advantage over conventional optical fibres in that their applications are not limited to telecommunications.
Prominent alternative applications include (but are not limited to) non-linear optics (where they offer high optical intensities per unit power, making them highly efficient) and atom and particle guidance (dielectric particles can be guided by the dipole forces exerted by light).
Hence there has been much to motivate study of the optical properties of PCFs; and understand how the design (geometry of the fibre, fabrication material) affects these, resulting in the models based on the Maxwell equations and development of spectral convergence results for periodic problems.
The work in this report is motivated by this application to PCFs, albeit in a slightly convoluted manner.
In section \ref{sec:OurPhysicalSetup} we detail the systems that we are investigating; which will resemble the structure of PCFs, and also describe (and justify) the assumptions we will be using in our model.
This will bring us through to section \ref{sec:GraphLitReview}; where we will link the system that we have arrived at in section \ref{sec:OurPhysicalSetup} to the PCFs we have described here, before setting the stage for a review of the relevant theory in chapters \ref{ch:QuantumGraphs} and \ref{ch:ScalarEqns}.

\section{Exisiting Models for PCFs} \label{sec:ExistingPCFModels}
PCFs have been the subject of a number of modelling techniques in recent years. 
The most conceptually straightforward being the use of numerical techniques to solve Maxwell's equations
\begin{align*}
	\grad\wedge\mathbf{E} &= \pdiff{\bracs{\mu_{P}\mathbf{H}}}{t}, 
	&\quad \grad\wedge\mathbf{H} &= -\pdiff{\bracs{\varepsilon_{P}\mathbf{E}}}{t},	
\end{align*} 
to determine modes (single-frequency solutions) that are supported by the PCF.
Here $\mathbf{E}$ and $\mathbf{H}$ are the electric and magnetic-displacement fields respectively; with $\varepsilon_{P}$ and $\mu_{P}$ being material parameters (the electric permittivity and magnetic permeability respectively).
These modes can then be used to construct an approximate band-gap plot for the fibre, with the level of precision in the computational results coming at the cost of increasing the computing time.
Whilst this can provide detailed information about a fibre, it does not provide any insight into how the structure of the PCF has contributed or affected the resulting band-gap plot.
Furthermore numerical schemes (particularly those based off finite elements) are known to require special treatment when being used to solve Maxwell's equations, but this in itself has been studied in detail (see for example \cite{monk2003finite}).
In our work we will largely stay clear of these kinds of solvers, primarily because they will not be directly applicable to the problems that we are looking to consider (section \ref{sec:OurPhysicalSetup}) and the aforementioned lack of insight into the dependence of band-gap spectra on the underlying fibre geometry. \newline

To avoid the need for complex numerical solvers, models that make simplifying approximations have been proposed.
Works such as \tstk{citations, follow refs 9,10,\cite{laegsgaard2004gap} in \cite{birks2006approximate}} have seen some success in retaining the key features of band-gap plots; by considering the physical origins of band-gaps in fibres as arising from the resonant properties of the cores.
In particular the spectral bands of a PCF are taken to correspond to modes that couple between the cores.
As a result only information about the modes of a single core, plus the relative core spacing and size, are needed to build an approximate band-gap plot.
These methods have made several insights into the origins of band-gap structures, but are limited by their approximations in the information they provide about the fibres themselves.
Intermediate approaches have also been proposed, for example in \cite{birks2006approximate} which proposes a model that considers the sizes of the regions separating the cores in the PCF and provides alternative boundary conditions for when the unit cell of the fibre cross-section is hexagonal or circular.
This model is still based on the physical origins of the band-gaps being due to resonances within the cores of the fibre however; and eventually relies on a numerical scheme.
But the numerical scheme is only needed to the extent of root-finding for analytic expressions, and produces band-gap plots that are in good agreement with the more fully-fledged numerical models, provided certain scaling between parameters is adhered to.
This makes such approaches useful when speed is more important than high accuracy, or when an intuitive picture of the band-structure is required. \newline

\tstk{this idea of band-gaps emerging from core resonances is not new to mathematics either}
The work in \cite{cooper2014band} demonstrates mathematically how one can achieve resonant phenomena in any structure that has an inclusion with a refractive index contrast.
This is done through consideration of the time-harmonic system of Maxwell equations\footnote{The time-harmonic Maxwell system arises from seeking ``Bloch wave" solutions in the fibre via an ansatz of the form $\hat{\mathbf{E}}=\mathbf{E}e^{i\bracs{\wavenumber x_3-\omega t}}$ (and similarly for the magnetic field), where $x_3$ is the axis of the fibre and $\wavenumber$ the component of the wave-vector in $x_3$.
Equivalently one can view this as taking a Fourier transform along the axis of the fibre.},
\begin{align*}
	\grad\wedge\mathbf{E} &= i\omega\mu_{P}\mathbf{H}, &\quad \grad\wedge\mathbf{H} &= -i\omega\varepsilon_{P}\mathbf{E}.
\end{align*}
\tstk{re-read cooper paper and provide exact details on the small parameter $\alpha$ and the size of the contrast in terms of the small param that emerges.}
\tstk{now the reconciliation of resonance discussion that we need to sort out, IE scaling discussion RE David \cite{birks2004scaling}}

\section{Introduction to the Problems we shall Consider} \label{sec:OurPhysicalSetup}
Having reviewed several existing models for PCFs in section \ref{sec:ExistingPCFModels}, we now set about introducing the problems that we will be analysing.
Whilst our work is motivated by the study of band-gap spectra of PCFs, the problems that we shall be treating will begin from a singular-structure setting (which we will elaborate on shortly).
At first this will appear to be removed from the more familiar thin-structure setting that the literature discussed in section \ref{sec:ExistingPCFModels} considers, where the cross-sectional structure of a PCF (or more generally any waveguide) is small but finite.
Furthermore our approach will also not come without complications in terms of how we pose mathematical problems on such singular-structure domains.
The purpose of this section will be to clarify use of the term ``singular-structure"; set out our research objectives, highlight the issues that our approach will raise, and how we propose to deal with them. \newline

We work in domains in $\reals^3$, assuming they all exhibit structure that is translation invariant in the $x_3$-direction (which represents the axis of the PCF, along which light propagates down the fibre).
The $\bracs{x_1,x_2}$-plane is referred to as the ``cross-section" or ``transverse plane" of the fibre, and is assumed to contain some periodic (micro-)structure with period cell $\sqbracs{0,1}^2$.
For the models discussed in section \ref{sec:ExistingPCFModels}, this microstructure typically consists of a series of circular inclusions in the cross-section (hence forming the cores along the fibre axis) and provides a thin-structure problem on a periodic domain when a suitable set of governing equations are posed.
Our starting point however is to take the cross-sectional structure as an embedded graph periodic in the two cross-sectional directions, giving rise to a so-called ``singular-structure" domain - the microstructure in our transverse plane does not have any thickness in the intuitive sense.
Contrastingly a thin-structure problem might have a periodic lattice in the cross-section, whose struts would have some finite thickness $l$ and lattice-junctions some characteristic size $r$, as illustrated in figure \ref{fig:Diagram_ThinStructurePeriodCell}.
\begin{figure}[b!]
	\centering
	\begin{subfigure}[t]{0.45\textwidth}
		\centering
		\includegraphics[scale=1.0]{Diagram_ThinStructurePeriodCell}
		\caption{\label{fig:Diagram_ThinStructurePeriodCell} Schematic illustration of the unit cell of a PCF. The photonic crystal is a lattice whose struts have some scale $l$ relative to the cell size of 1. Lattice ``junctions" are also assigned a length scale $r$ for consistency later. The cross section of a PCF is then composed of a (finite) number of these cells stacked in the $\bracs{x_1,x_2}$ plane.}
	\end{subfigure}
	~
	\begin{subfigure}[t]{0.45\textwidth}
		\centering
		\includegraphics[scale=1.0]{Diagram_SingularStructurePeriodCell}
		\caption{\label{fig:Diagram_SingularStructurePeriodCell} The domain which we will consider in our models, obtained by sending both length scales $l$ and $r$ to zero. The nature of the resulting problem depends on the relative scaling between $l$ and $r$, as we shall discuss in section \ref{sec:GraphLitReview}.}
	\end{subfigure}
	\caption{\label{fig:ThinToSingularStructure} An illustration of how the unit cell of a physical PCF relates to that of the systems that we shall be considering. This will be elaborated on in section \ref{sec:GraphLitReview}.}
\end{figure}
It is not unreasonable to think that one of our singular-structure domains might arise from taking the limit as these scales $l,r\rightarrow0$, as sketched in figure \ref{fig:Diagram_SingularStructurePeriodCell}.
Whilst geometrically this ``convergence" is quite easy to believe; it is less easy to deduce whether a mathematical problem can be posed on such a singular domain and still be consistent with this idea of it being a ``limit" in some sense of a family of thin-structure problems.
We will revisit this topic in more detail in section \ref{sec:GraphLitReview}. \newline

Because our problems will be posed on singular structure domains; we will require a formulation for problems on such domains that respects this lower-dimensional structure provided by the embedded graph, and is consistent with the idea of being a ``limit" problem as discussed in section \ref{sec:GraphLitReview}.
This will lead us to pose variational problems on our domains using non-Lebesgue measures (specifically, singular graph measures as defined in section \ref{sec:GraphSigularMeasuresDef}), the literature for which we discuss in section \ref{sec:VariationalProblemLitReview} and the theory we review in chapter \ref{ch:ScalarEqns} and develop ourselves in chapter \ref{ch:VectorEqns}.
A ``singular-structure problem" can thus be defined as referring to a singular-structure domain equipped with such a variational problem.
For the purposes of this section it is sufficient to think of these variational problems as providing a framework for posing differential equations (subject to defining appropriate function spaces) on singular-structure domains, and that we are able to look for eigenvalues and eigenfunctions of such equations.
In doing so, we are able to employ a couple of transforms to aid our analysis.
The most obvious being to apply a Fourier transform along the axis of the fibre, removing dependencies on $x_3$ and introducing it's Fourier variable $\wavenumber$, and effectively reducing our 3D problem to a 2D problem in the (periodic) cross-section.
This is a standard modelling assumption for wave-guidance models in general, unless surface roughness of the fibre needs to be incorporated.
Since our objective will be to determining propagation modes (equivalently band-gaps) rather than modelling fibre loss due to roughness, we too adopt this assumption for the simplifications it brings.
Of course, taking the Fourier transform can be thought of as looking for Bloch wave solutions, as discussed in section \ref{sec:ExistingPCFModels}. \newline

Given that we have also assumed our singular-structure to be composed of an embedded graph periodic in 2 dimensions, we can also employ a Gelfand transform to any singular-structure problems we consider.
This will enable us to move from a variational problem on an infinite-but-periodic singular-structure domain to a family of variational problems on the unit cell (assumed $\sqbracs{0,1}^2$) of said domain.
Formally the Gelfand transform provides a fibre representation of an operator defined on a periodic domain, and the theory surrounding it is much more extensive than what we require it for, but the following interpretation is sufficient for us.
For a function $u:\reals^{2}\rightarrow\complex^{d}$ for which we are solving our singular-structure problem, the Gelfand transform can be thought of as mapping $u$ to the family of functions $\widehat{u}$ defined by
\begin{align*}
	\widehat{u}: &\sqbracs{0,1}^{2} \rightarrow \complex^{d}, \\
	\widehat{u}\bracs{x} &= \sum_{n\in\integers^{2}} u\bracs{x + n}e^{-i\qm\bracs{x + n}}, \\
	\qm &\in [-\pi,\pi)^{2},
\end{align*}
the parameter $\qm$ is called the quasi-momentum.
This transform effectively takes us from a singular-structure problem involving $u$ on the original (infinite) domain to a family of slightly altered singular-structure problems on the period cell, with the family parametrised by $\qm$.
The extent of the alteration (to the variational formulation in this context) can be thought of as effectively applying a shift to any derivative we must take, formally any partial derivative transforms as
\begin{align*}
	\pdiff{}{x_j} &\rightarrow \pdiff{}{x_j} - i\qm_j.
\end{align*}
The reason for this shift being to account for integer-``mismatches" of $\widetilde{u}$ at the boundary of the period cell.
If we are considering spectral problems, the spectrum of the original problem (in the infinite domain) can be recovered from taking the union of the spectra of the transformed problems over $\qm$. \newline

Whilst the work that follows in this report will use the singular-structure problems as described above as a starting point and as the basis of the theory we develop and present; the explicit examples that we shall consider will be grounded in modelling wave-guidance in PCFs.
We will establish a link between our singular-structure problems and classical thin-structure problems in section \ref{sec:GraphLitReview}; and in chapters \ref{sec:ScalarEqns} and \ref{ch:VectorEqns} we focus our attention on two variational problems motivated by wave-guidance applications, completing this analysis with worked examples in chapter \ref{ch:ExampleSystems}.
In section \ref{sec:VariationalProblemLitReview} we will provide a review of the available literature on variational problems posed with respect to non-Lebesgue measures, as well as any work that has already been undertaken in this field.
We develop this theory in chapter \ref{ch:VectorEqns} so that we can examine wave-guidance problems, and also highlight how our variational problems give rise to ``quantum graph" problems that we introduce in section \ref{sec:GraphLitReview} and present the theory of in chapter \ref{ch:QuantumGraphs}.
Finally in section \ref{sec:ReportOverview} we reiterate the problems that we are studying and provide the objectives of our research; having provided a review of each of the areas of mathematics that we will need to aid us in sections \ref{sec:VariationalProblemLitReview}-\ref{sec:GraphLitReview}.

\section{Variational Problems for Singular Structures} \label{sec:VariationalProblemLitReview}
Because our singular-structure domain is essentially a 1D object embedded into a 2D plane; there are several oddities that must be addressed before we talk about posing equations or problems on these domains.
There is also the looming issue of ensuring that any problems we pose on our singular structure are consistent with thin-structure models of wave-guidance, however we delay this discussion until section \ref{sec:GraphLitReview}.
Of particular note is that the familiar concepts of gradient (and later curl and divergence) are no longer applicable to problems on our singular-structure domains; furthermore how can we define the ``boundary" of our singular-structure, and concepts like the ``normal derivative"?
This has significant impacts on the problems that we might want to consider; for systems like Maxwell's equations one typically imposes conditions on the derivates of the normal- and tangential-field components at the domain boundary. 
In this section we will provide the solution to these issues, and review some of the work that has been done in the field. \newline

The core of the problems outlined in the preceding paragraph emerge because when imagining a system of equations modelling some physical process, it is natural to think of differential equations complimented with boundary conditions.
Due to the fact that we are all familiar with differential equations on (what this report would label) thin-structure domains, these issues concerning gradients and boundary conditions never materialise.
However for singular-structure domains we can no longer afford to retain this idea; what we need to do instead is look for a problem that is formulated in such a way that the boundary conditions are naturally taken care of, and within our domain the original differential equations hold.
This leads us to consider variational (or weakly-formulated) problems; as an example rather than considering the problem of finding some function $u$ such that 
\begin{align*}
	-\grad\cdot\grad u &= f, \quad\text{ in } D\subset\reals^2, \\
	\pdiff{u}{n}\big\vert_{\partial D} &= 0,
\end{align*}
for an appropriate domain $D$ and forcing function $f$, we could consider it's variational form
\begin{align*}
	\integral{D}{\grad u \cdot \grad \phi}{x} &= \integral{D}{f\phi}{x}, \quad\forall \phi\in V_{\mathrm{test}},
\end{align*}
for some appropriate set of test functions $V_{\mathrm{test}}$.
Formally integrating by parts in the variational formulation\footnote{And using a suitable variant of the Fundamental Lemma of the Calculus of Variations.} then produces the original second-order differential equation plus it's boundary conditions.
For thin-structure problems considering the variational form is a standard step in the derivation of numerical schemes like the Finite Element Method.
For us it serves as an idea for how we can pose a set of ``differential equations" and ``boundary conditions" on a singular-structure; we just change the measure that the integration in the variational problem is performed with respect to, with this new measure respecting the singular-structure of our domain (see section \ref{sec:GraphSingularMeasuresDef}).
This doesn't come without costs; we now need to produce some appropriate spaces for the functions we are looking to solve for, and do similar things for their gradients, curls, and divergences.
Not to mention still need to qualify what we mean by these concepts (and indeed any concept involving differentiation), and we also throw away several techniques from familiar calculus - notably integration by parts.
However we can at least pose a consistent variational problem in this way, and we will later demonstrate that the subset of variational problems that use our chosen measures (section \ref{sec:GraphSingularMeasuresDef}) retain a connection to PCFs. \newline

In terms of work that has already been done regarding variational problems posed with respect to non-Lebesgue measures, there has been a lot of activity in the context of the equations of elasticity.
Notably the work in \cite{zhikov2000extension} lays a foundation for considering such problems in this context, defining the appropriate function spaces and concepts of (symmetric) gradients.
This work will also serve largely as the motivation for the work in chapter \ref{ch:VectorEqns}, when we look to study our singular-structure variational problems (a specific choice of measure) and define the appropriate function spaces in a wave-propagation context.
Also presented in this work are some preliminary results on whether such variational problems are well-posed, including existence results for solutions.
The work in \cite{zhikov2002homogenization} further develops the previously mentioned work, by bringing techniques from homogenisation theory into the framework of these (elasticity-based) variational problems and discussing how they are adapted to be applied in these contexts. 
Work on homogenisation of variational problems on periodic domains has also been carried out outside of the context of elasticity; such as for elliptic problems involving scalar-valued functions, for example in \cite{cherednichenko2018elliptic} where techniques from homogenisation theory are adapted to provide operator-norm estimates on solutions to such variational problems. 
In the context of electromagnetic wave-guidance problems, there have been similar developments for solution estimates for Maxwell's equations on periodic domains, such as in \cite{cherednichenko2018maxwell}.
Whilst these works do not specify the (Borel) measure that their variational problems use; these works are of particular importance when questions about existence of solutions and the effects of scaling the domains are asked.
That is they provide the answers to these questions, or at least the techniques to employ to find such answers, for a general setting which includes the subset of problems that we shall be considering.
We will provide a review of the theory that we need in chapter \ref{ch:ScalarEqns}, which largely comes from \cite{zhikov2000extension} and \cite{cherednichenko2018elliptic}.
In chapter \ref{ch:VectorEqns} we will look to extend the results of chapter \ref{ch:ScalarEqns} to variational problems involving vector-valued fields, bringing us closer to a classical electromagnetic-wave-guidance problem and the context of the work in \cite{cherednichenko2018maxwell}.
The only point to address now is how these variational problems that employ non-Lebesgue measures can be thought of as being a consistent ``limit" of thin-structure problems as the size of the structure tends to 0.
For this we will need the theory of quantum graphs, which we briefly introduce and review the literature for in section \ref{sec:GraphLitReview}.

\section{Quantum Graphs, and their relation to Thin-Structure Problems} \label{sec:GraphLitReview}
Quantum graphs have existed in some form (although not necessarily under that name) since the 1930s \cite{berkolaiko2013introduction} as surrogate models for various processes in chemistry, physics (including optical fibres) and mathematics.
However there has only recently been a push to develop a standard basis for the theory in the past few decades, with works produced before this time tending to focus on specific examples relevant to the application being considered.
At heart quantum graphs are graphs that are equipped with some concept of length or bulk; and quantum graph problems are simply a framework for (differential) operators on such graphs.
This is a departure from combinatorial graphs, where the vertices of the graph are the significant features and the edges merely represent some connections or processes.
Instead quantum graph problems focus largely on the lengths (and thus ``physical space") occupied by the edges, with the role of the vertices being reduced to that analogous to a boundary in more familiar differential equation settings.
A comprehensive introduction to the field can be found in \cite{berkolaiko2013introduction}, in which the main concepts and techniques for quantum graphs are gathered.
Further advances include development of homogenisation techniques for quantum graph problems that are periodic in one direction (which is the primary objective of \cite{cherednichenko2018unified}), and their use in demonstrating time-dispersive properties of solutions to quantum graph problems representing stratified materials \cite{cherednichenko2019time}.
Demonstration of this behaviour may yet be relevant to applications in PCFs, if quantum graph problems can be used as an approximate model for such fibres - we will review some of the literature on the more general topic of using quantum graphs as a ``limit" model of thin-structures shortly.
However there have been several advances surrounding analysis of spectral quantum graph problems, which we will need to review for our own research.  \newline

There has been a relatively recent increase in the interest of the properties of non-self-adjoint operators, which in turn has lead to the development of the theory of boundary triples.
Boundary triples have served many practical and analytical purposes when studying spectral problems (specifically, studying the spectra of operators).
Works such as \cite{ryzhov2007functional} are at the heart of this abstract study of non-self-adjoint operators, whilst others such as \cite{ryzhov2009spectral} focus their attention on this study of spectral problems.
On the more practical side; boundary triples, and in particular an associated object called the Weyl-Titchmarsh M-function, have been used as a tool to derive convergence estimates for differential problems.
The work in \cite{cherednichenko2017norm} derives operator-norm estimates for non-uniformly elliptic, periodic problems with rapidly varying coefficients by using a boundary triple related to the resolvent operator of said problem.
\cite{cherednichenko2018effective} follows up on this work, demonstrating how approaches using boundary triples and the M-function demonstrate that certain systems of PDEs give rise to time-dispersion and resonances. 
In terms of spectral problems on quantum graphs, the M-function can be represented as a finite square matrix \cite{ershova2014isospectrality} (which we henceforth refer to as the M-matrix) which means that spectral problems on quantum graphs can be drastically reduced in complexity, or at least reframed in a manner more open to analysis (see works such as \cite{ershova2016isospectrality}).
The fact that the M-function can be represented as a matrix is incredibly useful from a solution-based outlook; with enough information about the underlying graph and spectral problem at hand the task of determining the spectrum of a (differential) operator is reduced to that of finding the eigenvalues of the M-matrix.
This last development is of particular importance to us; as it provides a tool that can take us to the solution of any (spectral) quantum graph problems that we come across, and opens us potential for numerical schemes should exact analysis prove difficult.
We will highlight this idea again in section \ref{sec:M-MatrixTheory}, when we have presented some of the theory of quantum graphs, and discuss it in more detail in section \ref{sec:NumericalMethodsDiscussion}.
At present, having provided some background on quantum graph problems we now look to highlight the link between them; singular-structure problems of section \ref{sec:VariationalProblemLitReview}, and how they retain relevance to models for PCFs. \newline

Section \ref{sec:VariationalProblemLitReview} made it clear that the starting point for our models will be a singular-structure problem, and hinted at a link between these problems and thin-structure problems that are traditionally used to model wav-propagation.
We now look to address questions surrounding the nature of this link, in particular whether the singular-structure problems we consider are (or can be chosen so as to be) relevant to the original thin-structure problems for wave-guidance and hence PCFs.
The exact nature of how one moves from a singular-structure problem to a quantum graph problem is covered in chapters \ref{ch:ScalarEqns} and \ref{ch:VectorEqns}, so we do not elaborate further here.
However, it is not hard to believe that there will be connection between quantum graph problems and the singular-structure problems that we are going to propose, given that quantum graphs provide a framework for working on graphs with ``physical presence", which is precisely what our singular-structures are.
We do review the existing literature that provides a link between quantum graph problems and thin-structure problems, which largely comes about due to the work of \cite{kuchment2001convergence} and (developed independantly) \cite{exner2005convergence}.
These works are presented in a setting far more abstract than that which we require; in particular the setting of \cite{exner2005convergence} proves convergence results for graph-like manifolds, so we restrict ourselves to discussing the implications for work in our context.
Being somewhat blas\'e; we can summarise the link by saying that properties of the solutions to thin-structure problems in the limit of the thickness of the structure tending to zero coincide (in some formal sense of limit) with those of a quantum graph problem.
A more detailed explanation is as follows; suppose we have some domain $D$ (which may be periodic) such as that in figure \ref{fig:Diagram_ThinStructurePeriodCell}, and with two length scales $l$ and $r$ which capture the characteristic thickness/size of the struts (or ``tubes") and size of the junction regions (``tube connections") respectively.
For each $l,r>0$ assume that we pose a system of differential equations on our domain $D$\footnote{For clarity, these being the \emph{same} system of equations for each $\bracs{l,r}$ pair, only the size of the domain is changing.}, and we are interested in computing the eigenvalues of this system.
Denote this problem by $\mathcal{P}_{l,r}$, and the resulting spectrum by $\clbracs{\sigma}_{l,r}$.
One can then ask the question of what happens to $\clbracs{\sigma}_{l,r}$ as $l,r\rightarrow0$; the answer being that $\clbracs{\sigma}_{l,r}$ converges\footnote{In an appropriate notion of convergence for sets.} to the spectrum of a quantum graph problem.
We denote this quantum graph problem by $\mathcal{P}_0$ and it's spectrum by $\clbracs{\sigma}_0$.
Then we are able to argue that for a thin-structure problem with $l,r$ sufficiently small (compared to the size of the domain $D$, or if $D$ is periodic compared to the size of the unit cell of $D$), the properties of the thin-structure problems are sufficiently well-approximated by the solution to this ``limit" quantum graph problem $\mathcal{P}_0$. \newline

Interestingly the problem $\mathcal{P}_0$ and thus the spectrum $\clbracs{\sigma}_0$ depends on the relative scaling of $l$ and $r$ \cite{exner2005convergence}.
As $l$ quantifies the size of the struts, which in the limit $l\rightarrow0$ become graph edges, we can determine how fast (as a power of $l$) the edges are shrinking.
So for each $l>0$ we can define an ``edge area" $V_{\mathrm{edge}} \propto l^{\alpha}$ for some power $\alpha$, which decays to 0 as $l\rightarrow0$.
There is a similar concept of ``vertex area" $V_{\mathrm{vert}} \propto r^{\beta}$ which $r$ provides.
Then depending on the relative size\footnote{This is in fact not the complete picture, rather a consequence of the setting we are working in. The work in \cite{exner2013general} covers this more general setting and demonstrates the appropriate quantities to consider.} of $V_{\mathrm{edge}}$ to $V_{\mathrm{vert}}$ as $l,r\rightarrow0$ simultaneously; the problem $\mathcal{P}_0$, or rather the vertex conditions in this problem (see section \ref{sec:DEonQG}) are different.
\begin{itemize}
	\item In the case when $V_{\mathrm{edge}} \ll V_{\mathrm{vert}}$, the vertices dominate in the limit problem.
	This case is also referred to as ``Dirichlet decoupling" - the result is a quantum graph problem where the boundary conditions at the vertices require that the solution take the value 0 at each of these points.
	This makes $\mathcal{P}_0$ merely a system of independent (that is, decoupled) ODEs on the edges of the graph.
	Intuitively we can think of this as saying that; with the edges decaying faster than the vertices, the vertices are large enough to prevent any interaction between any edges that share a common vertex.
	\item In the case when $V_{\mathrm{edge}} \gg V_{\mathrm{vert}}$, the edges dominate in the limit problem.
	This results in a quantum graph problem $\mathcal{P}_0$ where the boundary conditions at the vertices  are strict matching conditions.
	That is; the functions we are solving for are required to be continuous across at each vertex, and derivatives of functions at the vertex are required to sum to zero (the coupling constant $\alpha=0$ in a Kirchoff condition).
	Intuitively it can be thought that $V_{\mathrm{vert}}$ decaying faster than $V_{\mathrm{edge}}$ results in the vertices disappearing from the system, and with nothing to separate the edges, function values must match where there is overlap.
	\item In the case when $V_{\mathrm{edge}} \approx V_{\mathrm{vert}}$, we arrive at a quantum graph problem that falls between the other two cases.
	Namely the boundary conditions at the vertices still require continuity of the solution at the vertices, but conditions on the derivates are relaxed.
	Instead one can obtain Kirchoff-like conditions at the vertices for the derivates; and the coupling constants that appear in these conditions are related to the ratio of $V_{\mathrm{edge}}$ to $V_{\mathrm{vert}}$.
	In this case we think of the vertices and edges shrinking at a rate which keeps them of comparable size, so the effect of decoupling the solutions (due to the vertex area) is balanced out by the need for consistency across the vertices (the effect of the edge area).
\end{itemize}

This neatly ties together the singular-structure problems introduced in section \ref{sec:VariationalProblemLitReview} that will be our starting point, quantum graph problems, and their relevance to thin-structure PCF models.
We shall demonstrate that singular-structure problems provide us with equivalent quantum graph problems, the relevance to PCFs coming being that for physical systems with thin microstructure the thin-structure problem we would consider is approximated by such a quantum graph problem.
Chapter \ref{ch:QuantumGraphs} elaborates on the ideas that have been presented in our review of the existing theory for quantum graphs, providing the formal statement of any definitions and results that we require for our work.
In chapters \ref{ch:ScalarEqns} and \ref{ch:VectorEqns} we will demonstrate how a variational problem on a singular structure is equivalent to a quantum graph problem that coincides with the $V_{\mathrm{edge}} \gg V_{\mathrm{vert}}$ case; whilst in chapter \ref{ch:ExampleSystems} we will briefly explain how to obtain a quantum graph problem that coincides with the $V_{\mathrm{edge}} \approx V_{\mathrm{vert}}$ case.

\section{Overview of Research} \label{sec:ReportOverview}
In this chapter we have provided some motivation for the singular-structure problems we will be considering later, whilst also touching on several of the advantages and nuances of taking this approach.
How the topics discussed in this chapter fit together, including what we will be pursuing in this report, is illustrated in figure \ref{fig:Diagram_RelationBetweenMathematicalFields}.
\begin{figure}[b!]
	\centering
	\includegraphics[scale=0.75]{Diagram_RelationBetweenMathematicalFields.pdf}
	\caption{\label{fig:Diagram_RelationBetweenMathematicalFields} A visualisation of the topics covered in this chapter, how they link together and the research which we shall pursue.}
\end{figure}
Current models for PCFs fall broadly into three categories (section \ref{sec:ExistingPCFModels}):
\begin{itemize}
	\item Numerical schemes that look to solve systems of governing equations for a specific fibre geometries.
	These can provide accurate band-gap plots for the fibre geometry of interest, but do not provide any general insights into how the influence of the various PCF parameters.
	\item Analytical models that make several simplifying assumptions have had success in replicating broad features of PCF band-gap plots.
	However they are limited in the information they can provide about the fibres themselves.
	\item Intermediate models have been proposed that aim to strike a balance between these two approaches, which have seen success and do not suffer (to the same extent) the weaknesses of the other models.
\end{itemize}
We have highlighted that quantum graph problems arise as a limit of thin-structure problems (section \ref{sec:GraphLitReview}); which solidifies the link between the singular-structure variational problems we will consider and thin-structure PCF models, through quantum graph problems.
By utilising developments in the theory of boundary triples and the existing theory for quantum graphs (section \ref{sec:GraphLitReview}) we can make the task of solving quantum graph problems easier still, and potentially open to rather general numerical schemes (section \ref{sec:NumericalMethodsDiscussion}).
The quantum graphs framework will also allow us to pursue analytical analysis of the problems we want to consider, helping us to strike a balance between modelling specific fibre geometries and providing more general insights into the effect of fibre geometry on the spectrum.
We will present the theory on quantum graphs that we require in chapter \ref{ch:QuantumGraphs}. \newline

The work in this report aims to study variational problems that can be seen as approximations to PCFs; looking to work analytically to provide insight into how the cross-sectional geometry of a fibre influences it's spectral band-gaps, and proposing numerical approaches when an analytic approach becomes unfeasible.
The starting point for our work will be singular-structure variational problems; however we will demonstrate in chapters \ref{ch:ScalarEqns} and \ref{ch:VectorEqns} how we can derive an equivalent quantum graph problem from our singular-structure problems, which then links us back to PCFs via the use as an approximation to thin-structure problems.
Since we are required to adopt a variational framework for our model that respects the lower-dimensional singular-structure (section \ref{sec:VariationalProblemLitReview}), we require some analysis of how we can interpret concepts like gradient, curl, and divergence, which is the subject of chapters \ref{ch:ScalarEqns} and \ref{ch:VectorEqns}.
In chapter \ref{ch:ExampleSystems} will consider some examples that arise from problems in wave-guidance, and finally chapter \ref{ch:Conclusion} provides a summary of the work presented and discusses further avenues that could be pursued.