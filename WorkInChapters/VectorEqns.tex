\chapter{Vector Equations} \label{ch:VectorEqns}
Chapter \ref{ch:ScalarEqns} introduced the theory of Sobolev spaces $\gradSob{D}{\nu}$ with respect to arbitrary (Borel) measures $\nu$, and how we can make sense of the notion of ``derivative" and pose problems in these spaces.
Also investigated was how (under suitable regularity assumptions) the variational problems that can be posed are reduced to Quantum Graph problems, when the measure $\nu$ supports a graph embedded into the domain $D$. \newline

In this chapter we look to develop this theory to handling vector-valued systems of equations, which necessitates developing theory for the associated curl- and divergence-operations, as well as the associated Sobolev spaces.
We begin with the definition and construction of these spaces, during which we will demonstrate parallels with the previously seen concepts of ``tangential-gradients".
In fact, many of the arguments in this chapter will be inspired or grounded in the approaches taken in chapter \ref{ch:ScalarEqns}.
After this we will discuss slightly altered versions of these spaces, which will arise from considering the wave-propagation problems of interest to us.
We will provide a worked example in this context, and once again demonstrate how we can obtain a Quantum Graphs problem from a variational formulation when the measure $\nu$ supports a graph in the $\bracs{x_1,x_2}$-plane.

\section{Vector Sobolev Spaces} \label{sec:VectorSobSpaces}
In this section we discuss the construction of Sobolev spaces of vector-valued functions that possess curls (in some sense) with respect to an arbitrary (Borel) measure $\nu$.
In later sections we will adapt the construction of these spaces to suit our physical wave-guide and photonic-fibre systems; by taking a Fourier transform along the axis of the wave-guide and also incorporating the quasi-momentum that arises from taking a Gelfand transform in the cross-section.
The reader is invited to bear in mind the arguments employed in section \ref{sec:ScalarSobSpaces} throughout this section. \newline

Let $D\subset\reals^3$ be our 3D-domain, and let $\nu$ be an arbitrary (Borel) measure on $D$.
Let $W=W\bracs{D,\md\nu}$ be the closure of the set of pairs $\bracs{\phi, \curl{}\phi}$ in $\ltwo{D}{\nu}^3\times\ltwo{D}{\nu}^3$ for $\phi\in\smooth{D}$.
Given that this is essentially the same construction as in section \ref{sec:ScalarSobSpaces}, it is unsurprising to learn that some work must be done before the element $c$ in the pair $\bracs{u,c}\in W$ can be thought of as a ``curl".
Indeed we actually have the same problem as in the gradient case; if both $\bracs{u,c_1}, \bracs{0, c_2}\in W$ then $\bracs{u,c_1+c_2}\in W$, and so we define the set of $\nu$-curls of zero as
\begin{align*}
	\curlZero{D}{\nu} &:= \clbracs{ c\in\ltwo{D}{\nu}^3 \ \vert \ \bracs{0,c}\in W }, \\
	&= \clbracs{ c\in\ltwo{D}{\nu}^3 \ \vert \ \exists\phi\in\bracs{\smooth{D}}^3 \text{ s.t. } \phi_n\lconv{\ltwo{D}{\nu}^3}0, \curl{}\phi_n\lconv{\ltwo{D}{\nu}^3}c }
\end{align*}
Fortunately we have the same methodology available to us as in the gradient case, to allow us address this issue.
That is, we observe that $\curlZero{D}{\nu}$ is a closed linear subspace of $\ltwo{D}{\nu}^3$ so we can decompose
\begin{align*}
	\ltwo{D}{\nu}^3 = \curlZero{D}{\nu}^\perp \oplus \curlZero{D}{\nu}.
\end{align*}
Hence every $u\in\ltwo{D}{\nu}^3$ has a unique tangential $\nu$-curl, $\curl{\nu} u$ such that every $\bracs{u,c}\in W$ can be written in the form $\bracs{u, \curl{\nu}u + \widetilde{c}}\in W$ for some $\widetilde{c}\in\curlZero{D}{\nu}$.
Thus we can define the Sobolev space of functions $u$ with $\nu$-curls by
\begin{align*}
	\curlSob{D}{\nu} := \clbracs{ \bracs{u,\curl{\nu}u}\in W },
\end{align*}
and although elements of this space are technically pairs, it is sufficient to specify only the first component $u$ to identify the pair.
We can also pose problems in this space using a variational formulation as was done in section \ref{sec:ScalarSobSpaces}, we write
\begin{align*}
	\curl{\nu}\bracs{A\curl{\nu}u} + u &= f, \quad u\in\curlSob{D}{\nu},
\end{align*}
and interpret this as the variational problem of finding $u\in\curlSob{D}{\nu}$ such that
\begin{align*}
	\integral{D}{ A\bracs{\curl{\nu}u}\cdot\overline{\bracs{\curl{}\phi}} + u\overline{\phi}}{\nu} &= \integral{D}{ f\overline{\phi} }{\nu}, \quad \forall\phi\in\bracs{\smooth{D}}^3.
\end{align*}
Such problems actually serve as the motivation for the notion of the tangential curl that is provided in the same way as tangential gradients were motivated.
The elliptic matrix $A$ simply requires us to write $A\curl{\nu}u$ for the element in $\curlZero{D}{\nu}^\perp$, the details again being analogous to those in section \ref{sec:ScalarSobSpaces} for gradients. \newline

The problem that we shall be focusing on throughout this chapter is the colloquially-named ``curl of the curl" equation 
\begin{align} \label{eq:curlOfCurlEqn}
	\curl{\nu}\bracs{A\curl{\nu}u} = f, &\quad u\in\curlSob{D}{\nu}, f\in\ltwo{D}{\nu}^3,
\end{align}
typically arising from the study of Maxwell's equations.
The vector field $u$ is interpreted as either the magnetic or electric field, and then matrix $A$ (if present) the inverse of the magnetic permeability or electric permittivity.
Maxwell's equations also include a condition that the divergence of the magnetic field is zero, as is that of the electric field in the presence of no external charges.
This divergence-free condition must be imposed alongside \eqref{eq:curlOfCurlEqn} for a complete physical description of the system, and in particular requires the function $f$ to be divergence-free too.
Due to the nature of our Sobolev spaces and mathematical tools, we can only interpret ``divergence-free" as meaning ``orthogonal to all gradients", as the following definition clarifies.
\begin{definition}[Divergence-Free]
	Define 
	\begin{align*}
		W_{\mathrm{grad}}\bracs{D,\nu} &:= \overline{\clbracs{ \bracs{\phi, \grad\phi} \ \vert \ \phi\in\smooth{D} }}
	\end{align*}
	where the closure is with respect to the norm of $\ltwo{D}{\nu}\times\ltwo{D}{\nu}^3$.
	A function $u\in\ltwo{D}{\nu}^3$ is $\nu$-divergence free if
	\begin{align*}
		\integral{D}{ u\cdot g }{\nu} &= 0, \quad \forall \bracs{z,g}\in W_{\mathrm{grad}}\bracs{D,\nu}.
	\end{align*}	 
\end{definition}

Of course due to the decomposition of $\ltwo{D}{\nu}^n$ by $\gradZero{D}{\nu}$ and it's orthogonal compliment, it is sufficient to test the conditions
\begin{align*}
	\integral{D}{ u\cdot\overline{\grad_{\nu}v} }{\nu} &= 0, \quad \forall v\in\gradSob{D}{\nu}, \\
	\integral{D}{ u\cdot\overline{g} }{\nu} &= 0, \quad \forall g\in\gradZero{D}{\nu},
\end{align*}
to show that $u$ is $\nu$-divergence free.

\section{Illustrative Examples} \label{sec:CurlExamples}
\tstk{everything is a curl of zero on segment example, and (hopefully, if you fix it!) the plane-example for the interpretation.}

Let $I$ be a segment in $\reals^3$ with $x_2=x_3=0$ on $I$, and let $\nu$ be the singular measure on $\reals^3$ that supports 1D Lebesgue measure on $I$.
In this setup we demonstrate that $\curlZero{\reals^3}{\nu}$ actually coincides with the whole of $\ltwo{\reals^3}{\nu}$, so the only tangential $\nu$-curl is the zero function!
To this end, let $f\in\smooth{\reals^3}$, $v=\bracs{0,0,f}^\top\in\ltwo{\reals^3}{\nu}^3$ and consider the ``sequence" of functions $\phi^{(n)}=\phi=\bracs{-x_2 f, 0, 0}^\top$.
Then we have that
\begin{align*}
	\integral{\reals^3}{\abs{\phi}^2}{\nu} &= \integral{I}{x_2^2\abs{f}^2}{\nu} \\
	&= 0, \quad \text{as } x_2=0 \text{ on } I.
\end{align*}
Furthermore,
\begin{align*}
	\integral{\reals^3}{\abs{\curl{}\phi - v}^2}{\nu} &= \integral{I}{\abs{0-0}^2}{\nu} + \integral{I}{\abs{-x_2\partial_3 f - 0}^2}{\nu} + \integral{I}{\abs{f + x_2\partial_2 f - f}^2}{\nu} \\
	&= \integral{I}{x_2^2\abs{f}^2}{\nu} + \integral{I}{x_2^2\abs{\partial_2 f}^2}{\nu}
	= 0.
\end{align*}
Thus the (constant) sequence $\bracs{\phi^{(n)}}_{n\in\naturals}$ is such that
\begin{align*}
	\phi^{(n)} \lconv{\ltwo{\reals^3}{\nu}^3} 0, &\quad \curl{}\phi^{(n)} \lconv{\ltwo{\reals^3}{\nu}^3} v,
\end{align*}
and thus $v\in\curlZero{\reals^3}{\nu}$.
A similar argument then applies in the cases when;
\begin{itemize}
	\item $v=\bracs{0, f, 0}^\top$, by choosing the constant sequence $\phi = \bracs{x_3 f, 0, 0}^\top$.
	\item $v=\bracs{f, 0, 0}^\top$, by choosing the constant sequence $\phi = \bracs{0, -x_3 f, 0}^\top$.
\end{itemize}
As such we conclude that if $f\in\smooth{\reals^3}$, then
\begin{align*}
	\begin{pmatrix}	f \\ 0 \\ 0	\end{pmatrix},
	\begin{pmatrix}	0 \\ f \\ 0	\end{pmatrix},
	\begin{pmatrix}	0 \\ 0 \\ f	\end{pmatrix}
	\in \curlZero{\reals^3}{\nu}.
\end{align*}
But then applying density arguments for $\smooth{\reals^3}$ in $\ltwo{\reals^3}{\nu}$ and the fact that $\curlZero{\reals^3}{\nu}$ is a closed linear subspace of $\ltwo{\reals^3}{\nu}^3$ by definition, we must conclude that
\begin{align*}
	\curlZero{\reals^3}{\nu} &= \mathrm{span}\clbracs{	
	\begin{pmatrix}	f \\ 0 \\ 0	\end{pmatrix},
	\begin{pmatrix}	0 \\ f \\ 0	\end{pmatrix},
	\begin{pmatrix}	0 \\ 0 \\ f	\end{pmatrix}
	\ \vert \ f\in\ltwo{\reals^3}{\nu}
	} \\
	&= \ltwo{\reals^3}{\nu}.
\end{align*}

\tstk{what about plane-measure case? Can't get that to work still - can in-plane directions are arbitrary l2 functions, but can't prove that orthogonal part to plane is zero.}

\section{The $\ktgrad$ operator} \label{sec:ktOperator}
As discussed in chapter \ref{ch:Intro}, the physical systems of interest to us are those involving wave-guidance .
In particular we will typically be seeking solutions in the form of a travelling wave down the axis of the wave-guide, and solving for the transverse components of the wave in the cross-section of the guide.
In such problems it is standard practice to take a Fourier transform in the direction along the wave-guide axis; in our case this coincides with the $x_3$ co-ordinate direction.
As such we introduce the Fourier variable $\wavenumber$ and accordingly adjust the $\grad$ operator to account for this transform, denoting by $\kgrad$ the operator that acts component-wise as
\begin{align*}
	\kgrad := \begin{pmatrix} \partial_1 \\ \partial_2 \\ i\wavenumber\end{pmatrix}.
\end{align*}
In addition to the translation invariance of our domain in the $x_3$-direction (section \ref{sec:PhysicalSystem}) \tstk{this needs to exist at some point!} we will also be taking a Gelfand transform in our periodic yet infinite, cross-section plane to obtain a family of problems on the period cell $\ddom$ parametrised by the quasi-momentum $\qm$.
Thus in the systems that we are considering it will be necessary for us to consider the operator $\ktgrad$ which component-wise is acts as
\begin{align*}
	\ktgrad := \kgrad + i\begin{pmatrix} \qm_1 \\ \qm_2 \\ 0 \end{pmatrix}.
\end{align*}
One can then construct the sets
\begin{subequations} \label{eq:WktDefinitions}
	\begin{align}
		W^{\wavenumber, \qm}_{\mathrm{curl}} &:= \overline{ \clbracs{ \bracs{\phi, \ktcurl{}\phi} \ \vert \ \phi\in\bracs{\smooth{\ddom}}^3 } }, \\
		W^{\wavenumber, \qm}_{\mathrm{grad}} &:= \overline{ \clbracs{ \bracs{\phi, \ktgrad\phi} \ \vert \ \phi\in\smooth{\ddom} } },
	\end{align}
\end{subequations}
and then follow the construction method to obtain the spaces $\ktcurlSob{\ddom}{\ddmes}$ and $\ktgradSob{\ddom}{\ddmes}$.
Since $\ktgrad$ is nothing more than the result of unitary transforms on the function spaces considered in section \ref{sec:VectorSobSpaces}, the theory is preserved, however we must still perform some explicit computations to characterise the objects that we will be working with.
Also note that the operator $\ktgrad$ replaces the operator $\grad$, and so we are in effect seeking the $\bracs{\wavenumber,\qm}$-tangential curl, gradient, and divergence-free conditions. 
We will adopt the notation $\kcurlZero{\ddom}{\ddmes}$ for the set of ``$\wavenumber$-curls of zero and $\kgradZero{\ddom}{\ddmes}$ for the set of $\wavenumber$-gradients of zero;
\begin{align*}
	\kcurlZero{\ddom}{\ddmes} &= \clbracs{ c \ \vert \ \bracs{0,c}\in W^{\wavenumber}_{\mathrm{curl}}}, \\
	\kgradZero{\ddom}{\ddmes} &= \clbracs{ z \ \vert \ \bracs{0,z}\in W^{\wavenumber}_{\mathrm{grad}}}.
\end{align*}
It will be shown in the relevant sections that these sets are invariant under changes to $\qm$ hence we only write the superscript $\wavenumber$ on these sets.
Throughout this section we will also assume our usual set-up; we have some period cell $\ddom$ in which is an embedded graph $\graph=\bracs{V,E}$, and we denote by $\ddmes$ the singular measure supported on the edges of $\graph$.

\subsection{$\kt$-gradients} \label{sec:ktGradients}
In this subsection we will work towards characterising $\bracs{\wavenumber,\qm}$-tangential gradients.
This will first involve us characterising the set of gradients of zero, $\kgradZero{\ddom}{\ddmes}$, however there will be several parallels with the methods and results of section \ref{sec:ScalarExample}. 
We begin by showing that the gradients of zero that we shall be dealing with are invariant under changes in $\qm$ (just like in the scalar case) and so throughout the notation $\kgradZero{\ddom}{\ddmes}$ will be sufficient, and we can always consider the $\qm=0$ case in our working.
\begin{prop}[$\wavenumber$-Gradients of Zero are Invariant Under Quasi-Momentum] \label{prop:kGradZeroInvarientUnderQM}
	Let $\qm\in[-\pi,\pi)^2$ and set 
	\begin{align*}
		\mathcal{G}^{\kt}_{\ddom, \mathrm{d}\ddmes} &= \clbracs{ z \ \vert \ \bracs{0,z,c}\in W^{\wavenumber, \qm}_{\mathrm{grad}}}, \\
		\kgradZero{\ddom}{\ddmes} &= \clbracs{ z \ \vert \ \bracs{0,z,c}\in W^{\wavenumber, 0}_{\mathrm{grad}}}.
	\end{align*}
	Then
	\begin{align*}
		\mathcal{G}^{\kt}_{\ddom, \mathrm{d}\ddmes} &= \kgradZero{\ddom}{\ddmes}.
	\end{align*}
\end{prop}
\begin{proof}
	This proof is essentially identical to proposition \ref{prop:GradZeroInvarientUnderQM}; note that the third component of $\ktgrad\phi$ doesn't depend on $\qm$, and then apply the argument of that proposition to the first and second components to obtain the result.
\end{proof}

Next we move onto characterising $\kgradZero{\ddom}{\ddmes}$, electing to use the same procedure as in section \ref{sec:ScalarExample} and find an edge-wise characterisation, then combine the results to provide a characterisation across a general graph.
To this end we note that the arguments in section \ref{sec:ScalarExample} all apply to the first and second components of $\ktgrad\phi$ (they are in fact identical); this includes the results regarding orientation-preserving orthogonal rotations in the $\bracs{x_1,x_2}$-plane.
Because of this, we only present those parts of the following proofs that concern the third components of $\kt$-gradients.
We begin by assuming that our underlying graph $\graph$ consists of a single edge parallel to the $x_2$-axis.
\begin{prop}[$\wavenumber$-Gradients of Zero on a Segment Parallel to the $x_2$-axis] \label{prop:kGradZeroParallel}
	Let $I$ be a segment in $\ddom$ parallel to the $x_2$-axis and $\lambda_I$ the singular measure supported on $I$.
	Then
	\begin{align*}
		\kgradZero{\ddom}{\lambda_I} = \clbracs{ 
		\begin{pmatrix} f \\ 0 \\ 0	\end{pmatrix} \ \vert \ f\in\ltwo{\ddom}{\lambda_I}
		}
	\end{align*}
\end{prop}
\begin{proof}
	Corollary \ref{cory:CharacterisationGradientsZero} deals with the first two components, so it remains for us to examine the third component.
	Suppose that $f\in\ltwo{\ddom}{\lambda_I}$ and that $\bracs{0,0,f}^\top\in\kgradZero{\ddom}{\lambda_I}$, then there exists a sequence of smooth functions $\phi_n\in\smooth{\ddom}$ such that
	\begin{align*}
		\phi_n \lconv{\ltwo{\ddom}{\lambda_I}} 0, &\quad \begin{pmatrix} \partial_1\phi_n \\ \partial_2\phi_n \\ i\wavenumber\phi_n \end{pmatrix} = \ktgrad\phi_n \lconv{\ltwo{\ddom}{\lambda_I}^3} \begin{pmatrix} 0 \\ 0 \\ f	\end{pmatrix}.
	\end{align*}
	Thus $\phi_n\rightarrow0$ but $i\wavenumber\phi_n\rightarrow f$, hence we must conclude that $f=0$.
\end{proof}

We expect this result as it means that after taking the Fourier transform, our formulation still accounts for (or ``can see") the direction of wave propagation.
Additionally, as the third component of $\ktgrad\phi$ is unaffected by rotations in the $\bracs{x_1,x_2}$-plane, we can utilise proposition \ref{prop:kGradZeroParallel} to state the following result.
\begin{cory}[Characterisation of $\kgradZero{\ddom}{\ddmes}$] \label{cory:CharacterisationkGradientsZero}
	Let $\graph=\bracs{V,E}$ be a finite graph in $\ddom$ and $\ddmes$ the singular measure supported on the edges of $graph$.
	For each $I_{jk}\in E$ let $\lambda_{jk}$ be the singular measure supported on $I_{jk}$, and $e_{jk}$ be the unit vector along $I_{jk}$.
	Then
	\begin{align*}
		\kgradZero{\ddom}{\ddmes} &= \clbracs{ 
		\begin{pmatrix} g_1 \\ g_2 \\ 0 \end{pmatrix}		
		\in\ltwo{\ddom}{\ddmes}^3 \ \vert \ \left.
		\begin{pmatrix} g_1 \\ g_2 \end{pmatrix}
		\right\rvert_{I_{jk}}\cdot e_{jk} = 0 \ \forall I_{jk}\in E } \\
		&= \clbracs{ g\in\ltwo{\ddom}{\ddmes}^3 \ \vert \ g\in\kgradZero{\ddom}{\lambda_{jk}} \ \forall I_{jk}\in E }
	\end{align*}
\end{cory}
\begin{proof}
	Proposition \ref{prop:kGradZeroParallel} provides us with the form for $\kgradZero{\ddom}{\lambda_I}$ when $I$ is parallel to the $x_2$-axis.
	By following the argument of section \ref{sec:ScalarExample}, we can obtain a characterisation for the case when the segment $I$ is orientated at an arbitrary angle to the $x_2$-axis; noting that in each case the third component of an element $g\in\kgradZero{\ddom}{\lambda_I}$ remains identically zero.
	As such we can deduce results analogous to those in section \ref{sec:ScalarExample}; specifically \ref{lem:SmoothGradientsUnderRotation}, \ref{prop:RotationOfEdgeGradients}, \ref{cory:Grad0SingleEdge}, \ref{prop:Grad0IncB}, \ref{lem:SegGradExtend}, and \ref{prop:BIncGrad0}, from which this result follows.
\end{proof}

Establishing this characterisation now allows us to determine the form for tangential $\kt$-gradients, however we should note that from now on it will be necessary for us to consider an arbitrary $\qm\in[-\pi,\pi)^2$ as tangential gradients will depend on the quasi-momentum.
Of course it is sufficient for us to determine an edge-wise form for the tangential gradient, and the arguments and results of section \ref{sec:ScalarSystem} apply to provide us with the forms for the first and second component of the tangential $\kt$-gradient.
As such all that remains for us is to explore the third component of the tangential gradient on an edge, and then extend the aforementioned arguments.
We now provide the remaining details for characterising the tangential $\kt$-gradient on a segment parallel to the $x_2$-axis, then simply state it's general form, referring the reader back to the arguments of section \ref{sec:ScalarSystem} which are easily adapted.
\begin{prop}[Tangential $\kt$-Gradient on a Segment Parallel to the $x_2$-axis] \label{prop:ktTangentialGradientSegment}
	Let $I$ be a segment in $\ddom$ parallel to the $x_2$-axis, and $\lambda_I$ the singular measure supported on $I$.
	Write $\gradSob{\interval{I}}{t}$ for the ``classical" Sobolev space on the interval $\interval{I}$ with respect to the Lebesgue measure; and let $r:\interval{I}\rightarrow I$ be the change of variables map $r(t) = v_I + te_I$, for either endpoint $v_I$ of $I$ and unit vector $e_I$ along $I$ directed away from $v_I$.
	Suppose $u\in\ktgradSob{\ddom}{\lambda_I}$ and let $\widetilde{u} = u\circ r\in\gradSob{\interval{I}}{t}$.
	Then
	\begin{align*}
		\ktgrad_{\lambda_I}u &= \begin{pmatrix} 0 \\ u' + i\qm_2 u \\ i\wavenumber u\end{pmatrix}
	\end{align*}
	where $u' = \widetilde{u}'\circ r^{-1}$.
\end{prop}
\begin{proof}
	Setting $\ktgrad_{\lambda_I}u = \bracs{v_1,v_2,v_3}^{\top}$, the condition of orthogonality to $\kgradZero{\ddom}{\lambda_I}$ requires that
	\begin{align*}
		0 &= \integral{\ddom}{\ktgrad_{\lambda_I}u \cdot \overline{g}}{\lambda_I} \quad \forall g\in\kgradZero{\ddom}{\lambda_I} \\
		&= \integral{I}{v_1 \overline{g}_1}{\lambda_{I}} \quad \forall g_1\in\ltwo{\ddom}{\lambda_I},
	\end{align*}
	so $v_1=0$.
	Turning our attention to the other components, take any sequence $\phi_n\in\smooth{\ddom}$ such that $\phi_n\rightarrow u, \ktgrad\phi_n\rightarrow \bracs{0,v_2,v_3}^\top$.
	These convergences imply that
	\begin{align*}
		\phi_n &\lconv{\ltwo{\ddom}{\lambda_I}} u, \\
		\partial_1\phi_n + i\qm_1 \phi_n \lconv{\ltwo{\ddom}{\lambda_I}} 0, \\
		\partial_2\phi_n + i\qm_2 \phi_n \lconv{\ltwo{\ddom}{\lambda_I}} v_2, \\
		i\wavenumber\phi_n \lconv{\ltwo{\ddom}{\lambda_I}} v_3.
	\end{align*}
	But $\phi_n \lconv{\ltwo{\ddom}{\lambda_I}} u$ implies that $v_3 = i\wavenumber u$.
	This leaves us determining the form for $v_2$, where we direct the reader to the process detailed in lemma \ref{lem:TangentialQMGradientSegmentParallel} which demonstrates how to show that $v_2 = u' + i\qm_2 u$, which completes the proof.
\end{proof}

The form for the tangential $\kt$-gradient when the segment $I$ is at an arbitrary angle in the $\bracs{x_1,x_2}$-plane follows by applying a rotation to the previous result; and gives us an analogue of corollary \ref{cory:TangentialQMGradientGraphs}.
\begin{cory}[$\kt$-Tangential Gradient on Graphs] \label{cory:ktTangentialGradientForm}
	Let $\graph=\bracs{V,E}$ be a graph embedded into $\ddom$ and let $\ddmes$ be the singular measure supported on $\graph$.
	Denote by $x=\bracs{x_1,x_2}$ the axes co-ordinate system.
	For each $I_{jk}\in E$, denote the restriction of $u$ to $I_{jk}$ by $u_{jk} = u\vert_{I_{jk}}$.
	Also equip each $I_{jk}$ with a local orthogonal co-ordinate system $y=\bracs{y_1,y_2}$ with $y_2$ parallel to $I_{jk}$, such that $x=R_{jk}y$ for some $R_{jk}\in\mathrm{SO}(2)$.
	If $u\in\ktgradSob{\ddom}{\ddmes}$ then
	\begin{align*}
		\left.\ktgrad_{\ddmes}u\right\vert_{I_{jk}} = \begin{pmatrix} R_{jk}^\top & 0 \\ 0 & 1 \end{pmatrix} \begin{pmatrix} 0 \\ u' + i\bracs{R_{jk}\qm}_2 u \\ iku \end{pmatrix},
	\end{align*}
	for every $I_{jk}\in E$.
\end{cory}
\begin{proof}
	It is sufficient to find an edge-wise characterisation for the tangential $\kt$-gradient, and we can apply a rotation to the result of proposition \ref{prop:ktTangentialGradientSegment} to obtain the form for the tangential $\kt$-gradient on any segment in $\ddom$.
\end{proof}

Finally we round off this section by stating a result that follows almost identical to it's analogue in theorem \ref{thm:CharGradSob}.
\begin{theorem}[Characterisation of $\ktgradSob{\ddom}{\ddmes}$] \label{thm:ktGradSobChar}
	We have that
	\begin{align*}
		u\in\ktgradSob{\ddom}{\ddmes} \quad \Leftrightarrow 
		&\quad\text{(i)} \ u\in\ktgradSob{\ddom}{\lambda_{jk}} \ \forall I_{jk}\in E, \text{ and}\\
		&\quad\text{(ii)} \ u \text{ is continuous at each } v_j\in V.
	\end{align*}
\end{theorem}
\begin{proof}
	This is identical to the proof of theorem \ref{thm:CharGradSob}, once the realisation is made that the additional third component in $\kt$-gradients does not affect the argument of the proof.
\end{proof}

\subsection{The set $\kcurlZero{\ddom}{\ddmes}$} \label{sec:ktCurlsZero}
Our primary focus here is to characterise $\wavenumber$-curls of zero, and our analysis follows in much of the same vein as in section \ref{sec:ScalarExample}.
First we seek an edge-wise understanding, then we derive an extension result (c.f. lemma \ref{lem:kCurlZeroExtensionLemma}) so that our edge-wise characterisation also provides us with a graph-wide form.
We will need the following result on rotations during this section:
\begin{lemma}[Curl under rotation] \label{lem:CurlUnderRotation}
	Let $x=\bracs{x_1,x_2,x_3}^\top, y=\bracs{y_1,y_2,y_3}^\top$ be right-handed orthonormal co-ordinate systems and suppose $R\in\mathrm{SO}(3)$ is such that $x=Ry$.
	Let $u$ be a vector field, and denote by $u_x =\bracs{u_{x,1},u_{x,2},u_{x,3}}^\top$ it's representation in the $x$ co-ordinate system, and $u_y =\bracs{u_{y,1},u_{y,2},u_{y,3}}^\top$ it's representation in the $y$ co-ordinate system.
	Denote by $\grad_x$ and $\grad_y$ the $\grad$ operator in the two co-ordinate systems, respectively.
	Then
	\begin{align*}
		\grad_y\wedge u_y\bracs{R^\top x} &= R \grad_x u_x\bracs{R^\top x}, \\
		\grad_x\wedge u_x\bracs{R^\top x} &= R\top\grad_y u_y\bracs{R\top x}.
	\end{align*}
\end{lemma}
\begin{proof}
	We use index notation and Einstein summation conventions throughout this proof.
	We also use $\epsilon_{jkm}$ to represent the Levi-Civita symbol for permutations of three elements, and $\delta_{jk}$ for the Kronecker delta.
	As $R\in\mathrm{SO}(3)$, $R^\top=R^{-1}$ and $\det R = 1$.
	Thus $R$ is the adjugate matrix of it's own transpose, that is $R = \mathrm{Adj}\bracs{R^\top}$ so we have that
	\begin{align*}
		R_{jk} &= \recip{2}\epsilon_{mnj}\epsilon_{pqk}R^\top_{mp}R^\top_{nq} \\
		\implies 2R_{jk}\epsilon_{kvw} &= \epsilon_{kvw}\epsilon_{mnj}\epsilon_{pqk}R^\top_{mp}R^\top_{nq} \\
		&= \epsilon_{mnj}R^\top_{mp}R^\top_{nq}\bracs{\delta_{pv}\delta_{qw} - \delta_{pw}\delta_{qv}} \\
		&= \epsilon_{mnj}\bracs{R^\top_{mv}R^\top_{nw} - R^\top_{nv}R^\top_{mw}}
		= 2\epsilon_{mnj}R^\top_{mv}R^\top_{nw} \\
		\implies R_{jk}\epsilon_{kml} &= \epsilon_{jkn}R_{mk}R_{ln}.
	\end{align*}
	We now use this identity to prove the result of the lemma;
	\begin{align*}
		\bracs{\grad_y\wedge u_y\bracs{R^\top x}}_j &= \epsilon_{jkl}\pdiff{u_{y,l}}{y_k}
		= \epsilon_{jkl}\pdiff{}{y_k}\bracs{R_{ml}u_{x,m}} \\
		&= \epsilon_{jkl}R_{ml}\pdiff{x_n}{y_k}\pdiff{u_{x,m}}{x_n} \\
		&= \epsilon_{jkl}R_{ml}R_{nk}\pdiff{u_{x,m}}{x_n} \\
		&= R_{jk}\epsilon_{jnm}\pdiff{u_{x,m}}{x_n} \\
		&= \bracs{ R\grad_x\wedge u_x\bracs{R^\top x} }_j.
	\end{align*}
	From which left-multiplication by $R^\top$ gives the other desired equality.
\end{proof}

Before we delve into analysis of $\kcurlZero{\ddom}{\ddmes}$, we first demonstrate that this set is invariant under changes in $\qm$.
\begin{prop}[Invariance of $\kcurlZero{\ddom}{\nu}$ under $\qm$]
	Let $\nu$ be a Borel measure on $\ddom$ and $\qm\in[-\pi,\pi)^2$.
	Set
	\begin{align*}
		\kcurlZero{\ddom}{\nu} &= \clbracs{ c \ \vert \ \bracs{0,c}\in W^{\wavenumber, 0}}, \\
		\mathcal{C}_{\ddom, \md\nu}^{\kt}(0) &= \clbracs{ c \ \vert \ \bracs{0,c}\in W^{\wavenumber, \qm} },
	\end{align*}
	for $W$ as defined in \eqref{eq:WktDefinitions}.
	Then
	\begin{align*}
		\kcurlZero{\ddom}{\nu} &= \mathcal{C}_{\ddom, \md\nu}^{\kt}(0).
	\end{align*}
\end{prop}
\begin{proof}
	The argument is essentially identical in both directions, so without loss of generality we only show one of the two required set inclusions.
	Let us take some $\qm\in[-\pi,\pi)^2$ and $c\in\kcurlZero{\ddom}{\nu}$.
	Then we can find an approximating sequence $\phi^{(n)}\in\bracs{\smooth{\ddom}}^3$ such that
	\begin{align*}
		\phi^{(n)} \lconv{\ltwo{\ddom}{\nu}^3} 0, &\quad \kcurl{}\phi^{(n)} \lconv{\ltwo{\ddom}{\nu}^3} c.
	\end{align*}
	The convergence $\phi^{(n)}\rightarrow0$ implies that 
	\begin{align*}
		i \begin{pmatrix} \qm_1 \\ \qm_2 \\ 0\end{pmatrix}\wedge\phi^{(n)}\lconv{\ltwo{\ddom}{\nu}^3} 0
	\end{align*}
	too, and thus
	\begin{align*}
		\ktcurl{}\phi^{(n)} &= \kcurl{}\phi^{(n)} + i\begin{pmatrix} \qm_1 \\ \qm_2 \\ 0\end{pmatrix}\wedge\phi^{(n)} \\
		&\lconv{\ltwo{\ddom}{\nu}^3} c + 0 = c.
	\end{align*}
	Hence, $c\in \mathcal{C}_{\ddom, \md\nu}^{\kt}(0)$.
\end{proof}

This result means that the following analysis of ``$\wavenumber$-curls of zero" is simplified, as we can always consider the $\qm=0$ case to obtain a general result for any $\qm\in[-\pi,\pi)^2$, as well as posthumously justifying the use of the term $\wavenumber$-curls of zero.
We begin our analysis by considering a single segment in the plane, parallel to the $x_2$-axis (if the reader is comparing the results of this section to that of chapter \ref{ch:ScalarEqns}, note that in that chapter single segments were taken parallel to the $x_1$-axis).
\begin{prop}[$\wavenumber$-Curls of Zero on a Segment Parallel to the $x_2$-axis] \label{prop:kCurlZeroParallel}
	Let $I$ be a segment in the $\bracs{x_1,x_2}$-plane parallel to the $x_1$-axis, and let $\lambda_I$ be the singular measure supported on $I$.
	Then
	\begin{align*}
		\kcurlZero{\ddom}{\lambda_I} &= 
		\clbracs{
			\begin{pmatrix} 0 \\ f_2 \\ f_3 \end{pmatrix}
			\ \vert \ f_2,f_3\in\ltwo{\ddom}{\lambda_I}					
		}
	\end{align*}
\end{prop}
\begin{proof}
	Without loss of generality we assume that $x_1=0$ on $I$, otherwise we apply a translation to the functions we present below.
	Throughout the proof let $f\in\smooth{\ddom}$, it will be sufficient for us to only work with smooth functions in this proof, as we can apply a density argument to the results we deduce to extend them to $\ltwo{\ddom}{\lambda_I}$ functions. \newline
	
	First suppose that the function $v=\bracs{f,0,0}^{\top}\in\kcurlZero{\ddom}{\lambda_I}$.
	Then there exists an approximating sequence $\phi^{(n)}\in\smooth{\ddom}^3$ such that
	\begin{align*}
		\phi^{(n)} \rightarrow 0, 
		&\quad \kcurl{}\phi^{(n)} \rightarrow v,
	\end{align*}
	in $\ltwo{\ddom}{\lambda_I}^3$.
	In particular the convergence $\phi^{(n)} \rightarrow 0$ implies that
	\begin{align*}
		\phi^{(n)}_j \rightarrow 0, &\quad\forall j=1,2,3, \text{ in } \ltwo{\ddom}{\lambda_I},
	\end{align*}
	which alongside the component-wise consequences of $\kcurl{}\phi^{(n)} \rightarrow v$ results in the convergences
	\begin{align*}
		\phi^{(n)}_3 \rightarrow 0, \quad \partial_2\phi^{(n)}_3\rightarrow f, \quad \partial_1\phi^{(n)}_3 \rightarrow 0,
	\end{align*}
	in $\ltwo{\ddom}{\lambda_I}$.
	Thus,
	\begin{align*}
		\phi^{(n)}_3 \lconv{\ltwo{\ddom}{\lambda_I}} 0, 
		&\quad \grad\phi^{(n)}_3 \lconv{\ltwo{\ddom}{\lambda_I}^2} \begin{pmatrix} 0 \\ f \end{pmatrix}
	\end{align*}
	so $\bracs{0,f}^{\top}\in\gradZero{\ddom}{\lambda_I}$, but by corollary \ref{cory:Grad0SingleEdge} we must conclude that $f=0$. \newline
	
	Now let $v=\bracs{0,f,0}^{\top}$, we will show by constructing an explicit approximating sequence that $v\in\kcurlZero{\ddom}{\lambda_I}$.
	Indeed we can even take $\phi^{(n)} = \phi = \bracs{0,x_1 f,0}^{\top}$ to be a constant sequence, giving
	\begin{align*}
		\kcurl{}\phi^{(n)} &= \begin{pmatrix} i\wavenumber x_1 f \\ 0 \\ x_1\partial_1 f + f	\end{pmatrix}.
	\end{align*}
	Then by explicit computation, 
	\begin{align*}
		\integral{\ddom}{\abs{\phi}^2}{\lambda_I} &= \integral{I}{x_1^2\abs{f}^2}{\lambda_I} = 0, \\
		\integral{\ddom}{\abs{\kcurl\phi - v}^2}{\lambda_I} &= \integral{I}{x_1^2\bracs{ \bracs{\partial_1 f}^2 - \wavenumber^2 f^2}}{\lambda_I} = 0.
	\end{align*}
	So trivially
	\begin{align*}
		\phi^{(n)} \rightarrow 0,
		&\quad \kcurl{}\phi^{(n)} \rightarrow v,
		\quad\text{in } \ltwo{\ddom}{\lambda_I}^3.
	\end{align*}
	Thus $v\in\kcurlZero{\ddom}{\lambda_I}$.
	Using the same approach, we can deduce that if $v=\bracs{0,0,f}^{\top}$ then $v\in\kcurlZero{\ddom}{\lambda_I}$ by considering the constant approximating sequence $\phi^{(n)} = \bracs{0,0,-x_1 f}^{\top}$.
	Application of a density argument extends these results from smooth functions $f$ to $\ltwo{\ddom}{\lambda_I}$ functions, and then as $\kcurlZero{\ddom}{\lambda_I}$ is a (closed) linear subspace of $\ltwo{\ddom}{\lambda_I}^3$ we have the desired result.
\end{proof}

\tstk{if you get the illustrative examples section working, this stuff should go there instead and this part can become just a recall}
Like with proposition \ref{prop:GradZeroParallelZhikov} we can provide some kind of geometric interpretation for this result, illustrated in figure \ref{fig:CurlZeroInterp}.
\begin{figure}[b!]
	\centering
	\includegraphics[scale=0.75]{Diagram_CurlZeroPlane.pdf}
	\caption{\label{fig:CurlZeroInterp} An illustration of $\wavenumber$-curls of zero. The curl is the axis of rotation of a small body in the vector field, whilst the measure we are using can only see things that happen in the plane/segment it supports. As such, unless the axis of rotation is normal to the plane, the measure $\lambda_I$ cannot ``see" it, and so it is a curl of zero.}
\end{figure}
First recall that although we are working on segments in two dimensions, prior to taking a Fourier transform these represented planes that extended into the $x_3$-direction.
With a segment parallel to the $x_2$-axis, the plane that it represents has the $x_1$-direction as it's normal.
Meanwhile the curl (hence $\wavenumber$-curl post Fourier transform) of a vector field is the axis of rotation a small body placed in the field would undergo.
Because our measure can only ``see" things that happen in the segment/plane it supports, any rotation that does not take place in the plane is irrelevant.
As only the component of the axis of rotation normal to the plane will induce rotation entirely in the plane, the set of $\wavenumber$-curls of zero consists of all the remaining parts.
In this case, the $x_1$-direction is normal to the plane/segment, so no $\wavenumber$-curl of zero has non-zero $x_1$-component.
Correspondingly the $x_2$- and $x_3$-components always induce rotation is that out-of-plane, hence all possible options for these components form the set $\kcurlZero{\ddom}{\lambda_I}$.
This example is generalised in the following results. \newline

\begin{lemma}[$\wavenumber$-Curls of Zero on a Segment] \label{lem:kCurlZeroAnySegment}
	Suppose $I$ is a segment in $\ddom$ with orthogonal co-ordinate system $y=\bracs{y_1,y_2}$ with $y_2$ parallel to $I$.
	Let $R\in\mathrm{SO}(2)$ be the change of co-ordinates $x=Ry$ where $x=\bracs{x_1,x_2}$ is the orthogonal co-ordinate system of the axes, and set
	\begin{align*}
		\widetilde{R} &= \begin{pmatrix} R & 0 \\ 0 & 1 \end{pmatrix} \in \mathrm{SO}(3).
	\end{align*}
	Also let $\widetilde{I} = RI$ be the image of $I$ under $R$ (which is parallel to the $x_2$-axis) and let $\lambda_{\widetilde{I}}(\cdot) = \lambda_I(R^{\top}\cdot)$ be the singular measure along $\widetilde{I}$.
	Then
	\begin{align*}
		\kcurlZero{\ddom}{\lambda_I} &= \clbracs{\widetilde{R}c \ \vert \ c\in\kcurlZero{\ddom}{\lambda_{\widetilde{I}}}} \\
		&= \clbracs{
		\widetilde{R} \begin{pmatrix} 0 \\ f_2 \\ f_3	\end{pmatrix} \ \vert \ v_2,v_3\in\ltwo{\ddom}{\lambda_I}
		} \\
		&= \clbracs{c\in\ltwo{\ddom}{\lambda_I}^3 \ \vert \ c\cdot n_I = 0 \text{ on } I},
	\end{align*}
	where $n_I$ is the unit normal to $I$.
\end{lemma}
\begin{proof}
	One can deduce the first set equality simply by employing the change of variables provided by $R$, and using similar arguments to proposition \ref{prop:RotationOfEdgeGradients}.
	The following set inequalities are simply alternative ways of writing the RHS of the first line, given the hypothesis of the lemma.
\end{proof}

Having established the form of $\wavenumber$-curls of zero on individual segments, we now look to develop a characterisation for $\wavenumber$-curls of zero on a general graph $\graph$, as we did when considering gradients.
Our goal will be the following theorem, which bears much resemblance to corollary \ref{cory:CharacterisationGradientsZero}:
\begin{theorem}[Characterisation of $\kcurlZero{\ddom}{\ddmes}$] \label{thm:kCurlZeroGraphChar}
	Let $\graph=\bracs{V,E}$ be an embedded graph in $\ddom$ with singular measure $\ddmes$.
	Then we have that
	\begin{align*}
		\kcurlZero{\ddom}{\ddmes} &= \clbracs{ c\in\ltwo{\ddom}{\ddmes}^3 \ \vert \ c\in\kcurlZero{\ddom}{\lambda_{jk}} \ \forall I_{jk}\in E}
	\end{align*}
\end{theorem}
For ease we set $B = \clbracs{ c\in\ltwo{\ddom}{\ddmes}^3 \ \vert \ c\in\kcurlZero{\ddom}{\lambda_{jk}} \ \forall I_{jk}\in E}$ in the analysis that follows, however the reader should notice several parallels with the methodology of section \ref{sec:GradZeroGraphAnalysis}; specifically propositions \ref{prop:Grad0IncB} and \ref{prop:BIncGrad0}, as well as lemma \ref{lem:SegGradExtend}.
\begin{lemma} \label{lem:kCurlZeroInB}
	\begin{align*}
		\kcurlZero{\ddom}{\ddmes} &\subset B.
	\end{align*}
\end{lemma}
\begin{proof}
	Suppose $c\in\kcurlZero{\ddom}{\ddmes}$, and take an approximating sequence $\phi^{(n)}$.
	Then for any $I_{jk}\in E$ we have that
	\begin{align*}
		\integral{\ddom}{\abs{\phi^{(n)}}^2}{\lambda_{jk}} 
		&\leq \integral{\ddom}{\abs{\phi^{(n)}}^2}{\ddmes} \rightarrow 0 \toInfty{n}, \\
		\integral{\ddom}{\abs{\kcurl\phi^{(n)} - c}^2}{\lambda_{jk}}
		&\leq \integral{\ddom}{\abs{\kcurl{}\phi^{(n)} - c}^2}{\ddmes} \rightarrow 0 \toInfty{n}.
	\end{align*}
	Thus
	\begin{align*}
		\phi^{(n)}\rightarrow 0, &\quad \kcurl{}\phi^{(n)}\rightarrow c \text{ in } \ltwo{\ddom}{\lambda_{jk}} \ \forall I_{jk}\in E,
	\end{align*}
	and so $c\in B$.
\end{proof}

\begin{lemma}[Extension Lemma for $\wavenumber$-curls of Zero] \label{lem:kCurlZeroExtensionLemma}
	For $n\in\naturals$, and an edge $I_{jk}\in E$, let $I_{jk}^n$ be as in \eqref{eq:ShortenedIntervalDef}.
	Suppose we have some $c\in\ltwo{\ddom}{\ddmes}^3$ with $c=0$ on $\ddom\setminus I_{jk}^n$ and $c\cdot n_{jk} = 0$ on $I_{jk}^n$, where $n_{jk}$ is the unit normal to $I_{jk}$.
	Then
	\begin{align*}
		c\in\kcurlZero{\ddom}{\ddmes}.
	\end{align*}
\end{lemma}
\begin{proof}
	As $c\cdot n_{jk} = 0$ on $I_{jk}^n$ and $c=0$ on $\ddom\setminus I_{jk}^n$, we have that $c\in\kcurlZero{\ddom}{\lambda_{jk}}$.
	Hence we can find an approximating sequence $\phi^{(l)}$ for $c$.
	Let $\chi_{jk}^{n}\in\smooth{\ddom}$ be the function defined in \eqref{eq:ChiDef}, and consider the sequence of functions $\psi^{(l)} := \chi_{jk}^n\phi^{(l)}$.
	By construction we have that
	\begin{align*}
		\integral{\ddom}{\abs{\psi^{(l)}}^2}{\ddmes}
		&= \integral{I_{jk}}{\abs{\chi_{jk}^n\phi^{(l)}}^2}{\lambda_{jk}}
		\leq \integral{I_{jk}}{\abs{\phi^{(l)}}^2}{\lambda_{jk}}
		\rightarrow 0 \toInfty{l}.
	\end{align*}
	Because
	\begin{align*}
		\kcurl{}\psi^{(l)} &= \grad^{(0)}\chi_{jk}^n\wedge\phi^{(l)} + \chi_{jk}^n\kcurl{}\phi^{(l)},
	\end{align*}
	(recall $\grad^{(0)}$ is just $\kgrad{}$ with $\wavenumber=0$) we also have that
	\begin{align*}
		\integral{\ddom}{\abs{\kcurl{}\psi^{(l)} - c}^2}{\ddmes}
		&= \integral{I_{jk}}{\abs{\grad^{(0)}\chi_{jk}^n\wedge\phi^{(l)} + \chi_{jk}^n\kcurl{}\phi^{(l)} - c}^2}{\lambda_{jk}} \\
		&\leq 2\integral{I_{jk}}{\abs{\chi_{jk}^n\kcurl{}\phi^{(l)} - c}^2}{\lambda_{jk}} \\
		&\quad + 2\sup\abs{\grad^{(0)}\chi_{jk}^n}^2\integral{I_{jk}}{\abs{\phi^{(l)}}^2}{\lambda_{jk}}.
	\end{align*}
	We notice that $\sup\abs{\grad^{(0)}\chi_{jk}^n}^2 = \sup\abs{\grad\chi_{jk}^n}^2$, which is bounded by a constant that does not depend on $l$, and so the second term is converging to $0 \toInfty{l}$.
	As for the first term,
	\begin{align*}
		\integral{I_{jk}}{\abs{\chi_{jk}^n\kcurl{}\phi^{(l)} - c}^2}{\lambda_{jk}}
		&= \integral{I_{jk}\setminus I_{jk}^n}{\abs{\chi_{jk}^n\kcurl{}\phi^{(l)}}^2}{\lambda_{jk}} \\
		&\quad + \integral{I_{jk}^n}{\abs{\kcurl{}\phi^{(l)} - c}^2}{\lambda_{jk}} \\
		&\leq \integral{I_{jk}\setminus I_{jk}^n}{\abs{\kcurl{}\phi^{(l)}}^2}{\lambda_{jk}} \\
		&\quad + \integral{I_{jk}^n}{\abs{\kcurl{}\phi^{(l)} - c}^2}{\lambda_{jk}} \\
		&= \integral{I_{jk}\setminus I_{jk}^n}{\abs{\kcurl{}\phi^{(l)} - c}^2}{\lambda_{jk}} \\
		&\quad + \integral{I_{jk}^n}{\abs{\kcurl{}\phi^{(l)} - c}^2}{\lambda_{jk}} \\
		&= \integral{I_{jk}}{\abs{\kcurl{}\phi^{(l)} - c}^2}{\lambda_{jk}} \rightarrow 0 \toInfty{l}.
	\end{align*}
	Hence we have provided a sequence $\psi^{(l)}$ such that
	\begin{align*}
		\psi^{(l)} \rightarrow 0, &\quad \kcurl{}\psi^{(l)} \rightarrow c \text{ in } \ltwo{\ddom}{\ddmes}^3,
	\end{align*}
	so $c\in\kcurlZero{\ddom}{\ddmes}$.
\end{proof}

\begin{lemma} \label{lem:BInkCurlZero}
	\begin{align*}
		B &\subset \kcurlZero{\ddom}{\ddmes}.
	\end{align*}
\end{lemma}
\begin{proof}
	Take $c\in B$, and define a family of functions $c_n$ by \tstk{this sim notation needs to be defined somewhere because it has ambiguous meaning in many places so far! I've settled that it appears to mean ``sum over all edges without repeating"}
	\begin{align*}
		c_n &= \sum_{j\sim k}\eta_j^n \eta_k^n c\vert_{I_{jk}},
	\end{align*}
	where $\eta_j^n, \eta_k^n$ are as defined in \eqref{eq:etaDef} and $c\vert_{I_{jk}}$ denotes the restriction of $c$ to $I_{jk}$, extended by zeros to $\ddom$.
	For each pair $j\sim k$, $\eta_j^n \eta_k^n c\vert_{I_{jk}}$ satisfies the hypothesis of lemma \ref{lem:kCurlZeroExtensionLemma} so is an element of $\kcurlZero{\ddom}{\ddmes}$.
	As $\kcurlZero{\ddom}{\ddmes}$ is a closed linear subspace of $\ltwo{\ddom}{\ddmes}^3$, $c_n\in\kcurlZero{\ddom}{\ddmes} \ \forall n\in\naturals$.
	By lemma \ref{lem:etaConv}, we conclude that $c_n$ converges to $c$ in $\ltwo{\ddom}{\ddmes}$ and as $\kcurlZero{\ddom}{\ddmes}$ is closed, $c\in\kcurlZero{\ddom}{\ddmes}$ as required.
\end{proof}

This concludes our analysis of $\wavenumber$-curls of zero.
We shall build on this in the following subsection, where we look to characterise tangential $\kt$-curls.

\subsection{Tangential $\kt$-curls} \label{sec:ktCurlsTangential}
With the set $\kcurlZero{\ddom}{\ddmes}$ understood, we can now examine tangential $\kt$-curls more closely in much the same way as for $\kt$-gradients.
Given the result of theorem \ref{thm:kCurlZeroGraphChar}, it will be sufficient for us to provide an edge-wise characterisation for tangential $\kt$-curls.
As has become standard practice, we begin with by considering a segment that is oriented parallel to one of the axes.
\begin{lemma}[Tangential $\kt$-curl on a Segment Parallel to the $x_2$-axis] \label{lem:ktTanCurlSegmentParallel}
	Let $I$ be a segment in the $\bracs{x_1,x_2}$-plane parallel to the $x_1$-axis, and let $\lambda_I$ be the singular measure supported on $I$.
	Write $\gradSob{\interval{I}}{t}$ for the ``classical" Sobolev space on the interval $\interval{I}$.
	Take $r:\interval{I}\rightarrow I$ as the change of variables map $r(t) = v_I + te_I$ where $v_I$ is either endpoint $v_I$ of $I$ and $e_I$ is the unit vector along $I$ directed away from $v_I$.
	Suppose $u\in\ktcurlSob{I}{\lambda_I}$.
	Then $\widetilde{u}_3 := u_3 \circ r \in\gradSob{\interval{I}}{t}$ and with $u_3' := \widetilde{u}_3' \circ r^{-1}$,
	\begin{align*}
		\ktcurl{\lambda_I}u &= 
		\begin{pmatrix} u_3' + i\qm_1 u_3 - i\wavenumber u_2 \\ 0 \\ 0 \end{pmatrix},
	\end{align*}
\end{lemma}
\begin{proof}
	Write $\ktcurl{\lambda_I}u = \bracs{v_1, v_2, v_3}^{\top}$ for some functions $v_1,v_2,v_3$ to be determined.
	As we know that $\ktcurl{\lambda_I}u \perp \kcurlZero{\ddom}{\lambda_I}$, given theorem \ref{thm:kCurlZeroGraphChar} we observe that
	\begin{align*}
		0 = \integral{I}{v_2\overline{f}_2 + v_3\overline{f}_3}{\lambda_I} \forall f_2,f_3\in\ltwo{\ddom}{\lambda_I}.
	\end{align*}
	Hence we conclude that $v_2=v_3=0$. \newline
	
	Now take an approximating sequence $\phi^{(l)}$ for $u$.
	In what follows we use an overhead tilde to denote composition with the change of variables map $r$.
	Then as $\phi^{(l)}\rightarrow u \toInfty{l}$,
	\begin{align*}
		\int_0^{\abs{I}}\abs{\widetilde{\phi}^{(l)}_j - \widetilde{u}_j}^2 \md t
		&= \integral{I}{\abs{\phi^{(l)}_j - u_j}^2}{\lambda_I}
		\rightarrow 0 \toInfty{l},
	\end{align*}
	for each $j\in\clbracs{1,2,3}$, and so $\widetilde{\phi}^{(l)}_j\rightarrow\widetilde{u}_j$ in $\ltwo{\interval{I}}{t}$.
	Because we also have that $\ktcurl{}\phi^{(l)}\rightarrow \ktcurl{\lambda_I}u \toInfty{l}$, we know that
	\begin{align*}
		\partial_2\phi^{(l)}_3 + i\qm_2\phi^{(l)}_3 - i\wavenumber\phi^{(l)}_3 \rightarrow v_1
	\end{align*}
	in $\ltwo{\ddom}{\lambda_I}$.
	Noting that 
	\begin{align*}
		\diff{\widetilde{\phi}^{(l)}_3}{t} = \partial_2\widetilde{\phi}^{(l)}_3,
	\end{align*}
	using the change of variables $r$ we can deduce that
	\begin{align*}
		\int_0^{\abs{I}}\abs{\widetilde{\phi}^{(l)}_3 + i\qm_2\widetilde{\phi}^{(l)}_3 - i\wavenumber\widetilde{\phi}^{(l)}_2 - \widetilde{v}_1}^2 \md t
		&= \integral{I}{\abs{\partial_2\phi^{(l)}_3 + i\qm_2\phi^{(l)}_3 - i\wavenumber\phi^{(l)}_3 - v_1}^2}{\lambda_I}
		\rightarrow 0 \toInfty{l}.
	\end{align*}
	So we have the following convergences in $\ltwo{\interval{I}}{t}$;
	\begin{align*}
		\widetilde{\phi}^{(l)}_2 \rightarrow \widetilde{u}_2,
		&\quad \widetilde{\phi}^{(l)}_3 \rightarrow \widetilde{u}_3, \\
		\widetilde{\phi}^{(l)}_3 + i\qm_2\widetilde{\phi}^{(l)}_3 - i\wavenumber\widetilde{\phi}^{(l)}_2 \rightarrow \widetilde{v}_1. &
	\end{align*}
	Using the Algebra of Limits we find that
	\begin{align*}
		\widetilde{\phi}^{(l)}_3 \rightarrow \widetilde{u}_3, 
		&\quad \widetilde{\phi}^{(l)}_3 \rightarrow i\wavenumber\widetilde{\phi}^{(l)}_2 - i\qm_2\widetilde{\phi}^{(l)}_3 + \widetilde{v}_1.
	\end{align*}
	Hence we have that $\widetilde{u}_3\in\gradSob{\interval{I}}{t}$ and identify $\widetilde{v}_1 = \widetilde{u}_3'$, the derivative in the $\gradSob{\interval{I}}{t}$ sense of $\widetilde{u}_3$.
	As the change of variables $r$ is invertible, we conclude that
	\begin{align*}
		\ktcurl{\lambda_I}u &= 
		\begin{pmatrix} u_3' + i\qm_1 u_3 - i\wavenumber u_2 \\ 0 \\ 0 \end{pmatrix},
	\end{align*}
	as we sought.
\end{proof}

Again we highlight that although the notation $u_3'$ is suggestive of derivative-like properties, it does not in fact convey any such things and is merely a useful shorthand notation.
We now present the form of the tangential $\kt$-curl on an arbitrary segment and for $\kt$-curls on a graph $\graph$.
\begin{lemma}[Tangential $\kt$-curl on a Segment] \label{ktTanCurlGeneralSegment}
	Suppose $I$ is a segment in $\ddom$ with orthogonal co-ordinate system $y=\bracs{y_1,y_2}$ with $y_2$ parallel to $I$.
	Let $R\in\mathrm{SO}(2)$ be the change of co-ordinates $x=Ry$ where $x=\bracs{x_1,x_2}$ is the orthogonal co-ordinate system of the axes, and set
	\begin{align*}
		\widetilde{R} &= \begin{pmatrix} R & 0 \\ 0 & 1 \end{pmatrix} \in \mathrm{SO}(3).
	\end{align*}
	Also let $\widetilde{I} = RI$ be the image of $I$ under $R$ (which is parallel to the $x_2$-axis) and let $r$ be the change of variables $r(t) = v_{\widetilde{I}} + te_{\widetilde{I}}$, for either endpoint $v_{\widetilde{I}}$ of $\widetilde{I}$ and unit vector $e_{\widetilde{I}}$ directed parallel to $\widetilde{I}$ away from $v_{\widetilde{I}}$.
	Take $u\in\ktcurlSob{\ddom}{\lambda_I}$.
	Then $\widetilde{u}_3 := u_3 \circ r\in\gradSob{\interval{\widetilde{I}}}{t}$ and
	\begin{align*}
		\ktcurl{\lambda_I}u &= \widetilde{R} \begin{pmatrix} u_3' + i\bracs{R\qm}_2u_3 - i\wavenumber u_2 \\ 0 \\ 0 \end{pmatrix}
	\end{align*}
	with $u_3 = \widetilde{u}_3' \circ r^{-1}$.
\end{lemma}
\begin{proof}
	One first uses the change of co-ordinates $\widetilde{R}$ (use of lemma  \ref{lem:CurlUnderRotation} is what introduces the pre-multiplication by $R$ of $\qm$) so that the framework now fits with lemma \ref{lem:ktTanCurlSegmentParallel}.
	Having done this the result given above follows.
\end{proof}

\begin{cory}[Tangential $\kt$-curls on Graphs] \label{cory:ktTanCurlsGraphs}
	Let $\graph=\bracs{V,E}$ be a graph embedded into $\ddom$ and $\ddmes$ be the singular measure supported on $\graph$.
	Let $x=\bracs{x_1,x_2}$ be the axes co-ordinate system.
	For each $I_{jk}\in E$ equip $I_{jk}$ with local co-ordinate system $y_{jk}=\bracs{y_1^{(jk)}, y_2^{(jk)}}$ with $y_2^{(jk)}$ parallel to $I_{jk}$.
	Let $R_{jk}\in\mathrm{SO}(2)$ be the change of co-ordinates $x=R_{jk}y_{jk}$ and set
	\begin{align*}
		\widetilde{R}_{jk} &= \begin{pmatrix} R_{jk} & 0 \\ 0 & 1 \end{pmatrix} \in \mathrm{SO}(3).
	\end{align*}
	Also let $\widetilde{I}_{jk} = R_{jk}I_{jk}$ be the image of $I_{jk}$ under $R_{jk}$ (which is parallel to the $x_2$-axis) and let $r_{jk}$ be the change of variables $r_{jk}(t) = \widetilde{v}_j + t\widetilde{e}_{jk}$; for $\widetilde{v}_j = R_{jk}v_j$ and unit vector $\widetilde{e}_{jk} = R_{jk}e_{jk}$.
	Take $u\in\ktcurlSob{\ddom}{\ddmes}$, and denote by $u_{l,(jk)}$ the restriction of $u_l$ to $I_{jk}$ (for $l\in\clbracs{1,2,3}$).
	Then we have that each $\widetilde{u}_{3,(jk)} = u_{3,(jk)}\circ r_{jk}\in\gradSob{\ddom}{\lambda_{jk}}$ and on each edge $I_{jk}$,
	\begin{align*}
		\bracs{\ktcurl{\ddmes}u}\vert_{I_{jk}} &= \widetilde{R}_{jk} \begin{pmatrix} u_{3,(jk)}' + i\bracs{R_{jk}\qm}_2 u_{3,(jk)} - i\wavenumber u_{2,(jk)} \\ 0 \\ 0 \end{pmatrix}
	\end{align*}
	where $u_{3,(jk)}' = \widetilde{u}_{3,(jk)}' \circ r_{jk}^{-1}$.
\end{cory}

\subsection{Additional Properties of $\ktcurlSob{\ddom}{\ddmes}$} \label{sec:ktcurlSobExtraProperties}
Much like $\gradSob{\ddom}{\ddmes}$, there is some additional structure within the space $\ktcurlSob{\ddom}{\ddmes}$.
This is mainly in the form of continuity conditions at the vertices of $\graph$ in the third components of vector fields $u\in\ktcurlSob{\ddom}{\ddmes}$.
Given that we have effectively defined a notion of curl, it should be unsurprising to learn that this comes with the requirement of some level of continuity in our fields.
\tstk{need to go over this section. Strictly speaking we have everything we need to use, which is the continuity condition on $u_3$ as our equations later imply that $u_1=0$ everywhere. That said, we still need one more set of boundary/vertex conditions, which will most likely have to do with $u_1$ as we get conditions for $u_2$ from the div-free stuff.}
So far we can establish; (NB might even just be non-$\kt$-grad spaces)
\begin{align*}
	u\in\ktcurlSob{\ddom}{\ddmes} \Rightarrow
	\begin{pmatrix} 0 \\ 0 \\ u_3 \end{pmatrix}\in\ktcurlSob{\ddom}{\ddmes} &\Leftrightarrow 
	u_3\in\ktgradSob{\ddom}{\ddmes} \\
	&\Leftrightarrow	\begin{cases} (i) u_3\in\ktgradSob{\ddom}{\lambda_{jk}} \ \forall I_{jk} \\ (ii) u_3 \text{ is continuous at every } v_j\in V	\end{cases}
\end{align*}

\subsection{The Divergence-Free Condition} \label{sec:ktDivFree}
Having established characterisations for both $\kgradZero{\ddom}{\ddmes}$ (corollary \ref{cory:CharacterisationkGradientsZero}) and the tangential gradients (corollary \ref{cory:ktTangentialGradientForm}), we can now examine what it means for a function $u\in\ktcurlSob{\ddom}{\ddmes}$ to be divergence-free.
As usual we will begin by considering a single edge $I$ and extending the results to the whole of $\graph$ via the edge-wise nature of $\ddmes$.
\begin{lemma}[Divergence-Free Functions on a Segment Parallel to the $x_2$-axis] \label{lem:DivFreeParallelSegment}
	Let $I$ be a segment in $\ddom$ parallel to the $x_2$-axis, and $\lambda_I$ be the singular measure supported on $I$.
	Denote by $\gradSob{\interval{I}}{\lambda_I}$ the ``classical" Sobolev space on $\interval{I}$ with respect to the Lebesgue measure; and let $r:\interval{I}\rightarrow I$ be the change of variables $r(t)= v_I + t e_I$, for some endpoint $v_I$ of $I$ and unit vector $e_I$ parallel to $I$ and directed away from $v_I$.
	Suppose $u=\bracs{u_1, u_2, u_3}^\top\in\curlSob{\ddom}{\lambda_I}$ is divergence-free.
	Then $\widetilde{u}_2 = u_2 \circ r$ and $\widetilde{u}_3 = u_3 \circ r$ satisfy
	\begin{align*}
		-\bracs{\diff{}{t} + i\qm_2}\widetilde{u}_2 &= i\wavenumber\widetilde{u}_3,
	\end{align*}
	where $\diff{\widetilde{u}_2}{t} = \widetilde{u}'_2$ is understood in the $\gradSob{\interval{I}}{t}$ sense.
\end{lemma}
\begin{proof}
	We simply check the requirement that $u$ be orthogonal to all gradients, first against gradients of zero and then against tangential $\kt$-gradients.
	We know that $g\in\kgradZero{\ddom}{\lambda_I}$ is such that $g=\bracs{g_1,0,0}^\top$ by proposition \ref{prop:kGradZeroParallel}, so
	\begin{align*}
		0 &= \integral{\ddom}{ u\cdot\overline{g} }{\lambda_I} \quad\forall g\in\kgradZero{\ddom}{\lambda_I}, \\
		\implies 0 &= \integral{I}{ u_1 \overline{g_1} }{\lambda_I} \quad \forall g_1\in\ltwo{\ddom}{\lambda_I}.
	\end{align*}
	Thus $u_1 = 0$.
	We now check orthogonality of $u$ against smooth $\kt$-gradients by letting $\phi\in\smooth{D}$, and considering the requirement that
	\begin{align*}
		0 &= \integral{\ddom}{ u\cdot\overline{\ktgrad\psi} }{\lambda_I}
		= \integral{\ddom}{u_2\bracs{\partial_2\overline{\psi} + i\qm_2\overline{\psi}} - i\wavenumber u_3\overline{\psi}}{\lambda_I} \\
		&= \int_0^{\abs{I}} \widetilde{u}_2\overline{\bracs{\widetilde{\psi}' + i\qm_2 \widetilde{\psi}}} - i\wavenumber\widetilde{u}_3\overline{\widetilde{\psi}} \ \md t \\
		&= \int_0^{\abs{I}} \overline{\widetilde{\psi}}\bracs{-\widetilde{u}'_2 - i\qm_2 \widetilde{u}_2 - i\wavenumber\widetilde{u}_3} \ \md t.
	\end{align*}
	Where we have used an overhead tilde to denote composition with the change of variables map $r$.
	This holds for each $\psi\in\smooth{\ddom}$ and hence for each $\widetilde{\psi}\in\smooth{\interval{I}}$, and so we conclude that
	\begin{align*}
		-\bracs{\diff{}{t} + i\qm_2}\widetilde{u}_2 &= i\wavenumber\widetilde{u}_3,
	\end{align*}
	which was the desired result.
\end{proof}

\begin{prop}[Divergence-Free Functions on a Segment] \label{prop:DivFreeGeneralSegment}
	Let $I$ be a segment in $\ddom$ with local co-ordinate system $y=\bracs{y_1,y_2}$ with $y_2$ parallel to $I$, and $\lambda_I$ the singular measure supported on $I$.
	Let $x=\bracs{x_1,x_2}$ be the axes co-ordinate system and $R\in\mathrm{SO}(2)$ such that $x=Ry$.
	Furthermore denote by $\gradSob{\interval{I}}{\lambda_I}$ the ``classical" Sobolev space on $\interval{I}$ with respect to the Lebesgue measure; and let $r:\interval{I}\rightarrow I$ be the change of variables $r(t)= v_I + t e_I$, for some endpoint $v_I$ of $I$ and unit vector $e_I$ parallel to $I$ and directed away from $v_I$.
	Suppose $u=\bracs{u_1, u_2, u_3}^\top\in\curlSob{\ddom}{\lambda_I}$ is divergence-free.
	Then $\widetilde{u}_2 = u_2 \circ r$ and $\widetilde{u}_3 = u_3 \circ r$ satisfy
	\begin{align*}
		-\bracs{\diff{}{t} + i\bracs{R\qm}_2}\widetilde{u}_2 &= i\wavenumber\widetilde{u}_3,
	\end{align*}
	where $\diff{\widetilde{u}_2}{t} = \widetilde{u}'_2$ is understood in the $\gradSob{\interval{I}}{t}$ sense.
\end{prop}
\begin{proof}
	One can simply apply rotation arguments to lemma \ref{lem:DivFreeParallelSegment}, or alternatively work directly from corollary \ref{cory:ktTangentialGradientForm}.
\end{proof}

Thus we arrive at the following classification for divergence-free functions on graphs.
\begin{cory}[Divergence-Free Functions on a Graph] \label{cory:DivFreeWholeGraph}
	Let $\graph=\bracs{V,E}$ be a graph embedded into $\ddom$ and $\ddmes$ the singular measure supported on the edges of $\graph$.
	For each $I_{jk}\in E$ let $\lambda_{jk}$ be the singular measure supported on $I_{jk}$, and denote $u^{(jk)} = u\vert_{I_{jk}}$ as the restriction of $u$ to $I_{jk}$.
	Let $r_{jk}:\interval{I_{jk}}\rightarrow I_{jk}$ be the change of variables map $r(t)=v_{j} + te_{jk}$ where $e_{jk}$ is the unit vector parallel to $I_{jk}$ directed from $v_j$ to $v_k$.
	Suppose $u\in\ktcurlSob{\ddom}{\ddmes}$, and set 
	\begin{align*}
		\widetilde{u}^{(jk)}_2 = u^{(jk)}_2 \circ r_{jk},
		&\quad \widetilde{u}^{(jk)}_3 = u^{(jk)}_3 \circ r_{jk},
	\end{align*}
	for each $I_{jk}\in E$.
	Then we have that
	\begin{align*}
		u \text{ is divergence free } \Leftrightarrow 
		& \ \text{(i)} \ -\bracs{\diff{}{t} + i\bracs{R_{jk}\qm}_2}\widetilde{u}^{(jk)}_2 = i\wavenumber\widetilde{u}^{(jk)}_3 \ \forall I_{jk}\in E, \\
		& \ \text{(ii)} \ \sum_{j\sim k}u_2^{(jk)}\bracs{v_j} = 0 \ \forall v_j\in V, \\
		& \ \text{(iii)} \ u \perp \kgradZero{\ddom}{\lambda_{jk}} \ \forall I_{jk}\in E.
	\end{align*}
	where derivatives are understood in the $\gradSob{\interval{I_{jk}}}{t}$ sense.
\end{cory}
\begin{proof}
	First we prove the right-directed implication ($\Rightarrow$).
	Given the result of corollary \ref{cory:CharacterisationkGradientsZero} it can clearly be seen that (iii) holds.
	As for $u$ being orthogonal to gradients, take some $v\in\ktgradSob{\ddom}{\lambda_{jk}}$ with $j\sim k$.
	Due to theorem \ref{thm:ktGradSobChar}, $v\in\ktgradSob{\ddom}{\ddmes}$ too and so
	\begin{align*}
		0 = \integral{\ddom}{u\cdot\overline{\ktgrad_\ddmes v}}{\ddmes}
		&= \integral{I_{jk}}{u\cdot\overline{\ktgrad_{\lambda_{jk}} v}}{\lambda_{jk}} \\
		&= \int_0^{\abs{I_{jk}}} \widetilde{u}_2^{(jk)}\overline{\widetilde{v}}' - \bracs{i\wavenumber\widetilde{u}_3^{(jk)} + i\bracs{R_{jk}\qm}_2\widetilde{u}_2^{(jk)}}\overline{\widetilde{v}} \ \md t \\
		&= \int_0^{\abs{I_{jk}}} \bracs{-\bracs{\diff{}{t} + i\bracs{R_{jk}\qm}_2}\widetilde{u}^{(jk)}_2 - i\wavenumber\widetilde{u}^{(jk)}_3}\overline{\widetilde{v}} \ \md t.
	\end{align*}
	This holds for every $v\in\ktgradSob{\ddom}{\lambda_{jk}}$ and thus we have condition (i).
	With this knowledge we now fix $v_j$ and take some $v\in\ktgradSob{\ddom}{\ddmes}$ with support that contains the vertex $v_j$ and no other vertices,
	\begin{align*}
		0 = \integral{\ddom}{u\cdot\overline{\ktgrad_\ddmes v}}{\ddmes}
		&= \sum_{j\sim k}\integral{I_{jk}}{u\cdot\overline{\ktgrad_{\lambda_{jk}} v}}{\lambda_{jk}} \\
		&= \sum_{j\sim k}\int_0^{\abs{I_{jk}}} \bracs{-\bracs{\diff{}{t} + i\bracs{R_{jk}\qm}_2}\widetilde{u}^{(jk)}_2 - i\wavenumber\widetilde{u}^{(jk)}_3}\overline{\widetilde{v}} \ \md t \\
		& \quad + \sum_{j\sim k}\sqbracs{\widetilde{u}_2^{(jk)}\overline{\widetilde{v}}}_0^{\abs{I_{jk}}} \\
		&= \sum_{j\sim k} - u_2^{(jk)}\bracs{v_j}\overline{v}\bracs{v_j} \\
	\end{align*}
	Thus we have condition (ii) and we are done with the right-directed implication. \newline
	
	For the left-directed implication ($\Leftarrow$), take some divergence-free $u\in\ktcurlSob{\ddom}{\ddmes}$.
	Corollary \ref{cory:CharacterisationkGradientsZero} easily gives us that $u\perp\kgradZero{\ddom}{\ddmes}$ due to the assumption (iii).
	As for demonstrating that $u$ is also orthogonal to tangential gradients, take some $v\in\ktgradSob{\ddom}{\ddmes}$.
	Then \tstk{sim notation!! Here there is no concept of directed edges, so it just means a sum over all the edges $I_{jk}$ WITHOUT repeating edges, and so the $jk$ scripts on $u$ don't indicate directions. This means we can perform the final manipulation in the sum, I draw some pictures to convince myself of this}
	\begin{align*}
		\integral{\ddom}{u\cdot\overline{\ktgrad_{\ddmes}v}}{\ddmes}
		&= \sum_{j\sim k}\integral{I_{jk}}{u^{(jk)}\cdot\overline{\ktgrad_{\lambda_{jk}}v}}{\lambda_{jk}} \\
		&= \sum_{j\sim k}\int_0^{\abs{I_{jk}}} \widetilde{u}_2^{(jk)}\overline{\widetilde{v}}' - i\bracs{\wavenumber\widetilde{u}_3^{(jk)} + \bracs{R_{jk}\qm}_2\widetilde{u}_2^{(jk)}}\overline{\widetilde{v}} \ \md t \\
		&= \sum_{j\sim k}\int_0^{\abs{I_{jk}}} \bracs{-\bracs{\diff{}{t} + i\bracs{R_{jk}\qm}_2}\widetilde{u}^{(jk)}_2 - i\wavenumber\widetilde{u}^{(jk)}_3}\overline{\widetilde{v}} \ \md t \\
		& \quad + \sum_{j\sim k}\sqbracs{\widetilde{u}_2^{(jk)}\overline{\widetilde{v}}}_0^{\abs{I_{jk}}} \\
		&= \sum_{j\sim k}0 + \sum_{j\sim k} u_2^{(jk)}\bracs{v_k}\overline{v}\bracs{v_k} - u_2^{(jk)}\bracs{v_j}\overline{v}\bracs{v_j} \\
		&= \sum_{v_j\in V}\sum_{j\sim k} u^{(jk)}\bracs{v_j}\overline{v}\bracs{v_j}
		= \sum_{v_j\in V}\overline{v}\bracs{v_j}\sum_{j\sim k} u^{(jk)}\bracs{v_j} \\
		& = 0.
	\end{align*}
	As so we have that $u$ is also orthogonal to tangential gradients, thus is divergence free.
	This completes the proof.
\end{proof}

\subsection{Summary}
Having introduced the operator $\ktgrad$ due to our interests in wave-guidance problems; we have now completed our analysis of $\ktcurlSob{\ddom}{\ddmes}$ and the divergence-free condition in the sense of $\kt$-gradients.
Our understanding of these is necessary for us to be able to reduce variational problems posed with respect to measures to quantum graph problems (section \ref{sec:CurlReductionToQG}).
Of important note is that the underlying arguments for understanding $\ktgradSob{\ddom}{\ddmes}$ are essentially identical to those in sections \ref{sec:GradZeroGraphAnalysis}; \ref{sec:ScalarSystem}, and \ref{sec:VertexContinuity}, being suitably adapted for the 3D context.
The path we take to understanding $\ktcurlSob{\ddom}{\ddmes}$ is also directed in much the same way as the arguments for $\gradSob{\ddom}{\ddmes}$ in these sections, and we even deduce a link between the two spaces \tstk{this will be in the continuity conditions section when that result is good}.
The divergence-free condition is also studied because we will ultimately be considering examples of wave propagation in electromagnetic contexts.
We will need the additional information from this condition to correctly formulate a variational problem, and from which obtain a quantum graph problem, that admits a solution.

\section{Reduction to Quantum Graphs} \label{sec:CurlReductionToQG}
The focus of this section will be analogous to that of section \ref{sec:ReduceToQMProblem}.
Having made use of the nature of our singular structure and the analysis of section \ref{sec:ktOperator}, we aim to reduce a variational problem posed in the appropriate Sobolev space to a more accessible quantum graphs problem.
Because we are focusing on electromagnetic wave-guidance, we shall be studying the spectral ``curl-of-the-curl" equation (see section \ref{sec:VectorSobSpaces}) subject to the divergence-free condition.
Formally, we define
\begin{align*}
	\ktcurlSobDivFree{\ddom}{\ddmes} 
	&:= \clbracs{ u\in\ktcurlSob{\ddom}{\ddmes} \ \vert \ u \text{ is } \ktgrad_\ddmes \text{-divergence free}}
\end{align*}
which is the subspace of $\ktcurlSob{\ddom}{\ddmes}$ that consists of all $\ktgrad_\ddmes$-divergence-free functions that have $\kt$-curls.
We then concern ourselves with the spectral problem
\begin{align*}
	\ktcurl{\ddmes}\bracs{\ktcurl{\ddmes}u} &= \omega^2 u, \quad u\in\ktcurlSobDivFree{\ddom}{\ddmes},
\end{align*}
which of course is shorthand for the variational problem
\begin{align*}
	\integral{\ddom}{\bracs{\ktcurl{\ddmes}u}\cdot\overline{\bracs{\ktcurl{}\phi}}}{\ddmes} 
	&= \omega^2\integral{\ddom}{u\cdot\overline{\phi}}{\ddmes}, \\
	\forall\phi\in\smooth{\ddom}^3, \quad u\in\ktcurlSobDivFree{\ddom}{\ddmes}. \labelthis\label{eq:CurlCurlEquationDivFree}
\end{align*}
The bulk of the work to get from this variational formulation to a quantum graphs problem has been done in the previous sections, and it is now a case of applying the knowledge we have of the objects in \eqref{eq:CurlCurlEquationDivFree}. \newline

We begin by recalling the form for the tangential $\kt$-curl given by corollary \ref{cory:ktTanCurlsGraphs}, and writing the integral in \eqref{eq:CurlCurlEquationDivFree} as a sum of integrals over the edges of $\graph$;
\begin{align*}
	\sum_{j\sim k}\integral{\ddom}{\bracs{\ktcurl{\lambda_{jk}}u}\cdot\overline{\bracs{\ktcurl{}\phi}}}{\lambda_{jk}} 
	&= \omega^2\sum_{j\sim k}\integral{\ddom}{u\cdot\overline{\phi}}{\lambda_{jk}}.
\end{align*}
Then as this must hold for all smooth functions $\phi\in\smooth{\ddom}^3$, it must also hold whenever the support of $\phi$ contains only (part of the interior of) a particular edge $I_{jk}$.
This would require
\begin{align*}
	\integral{\ddom}{\bracs{\ktcurl{\lambda_{jk}}u}\cdot\overline{\bracs{\ktcurl{}\phi}}}{\lambda_{jk}} 
	&= \omega^2\integral{\ddom}{u\cdot\overline{\phi}}{\lambda_{jk}},
\end{align*}
and given proposition \ref{prop:ktTangentialGradientSegment}, using our usual change of variables $r_{jk}$ and rotation $R_{jk}$, we can conclude that
\begin{align*}
	\int_0^{\abs{I_{jk}}} \bracs{\widetilde{u}_{3,jk}' - i\wavenumber\widetilde{u}_{2,jk} + i\bracs{R_{jk}\qm}_2\widetilde{u}_{3,jk} } \overline{\bracs{\widetilde{\phi}_3' - i\wavenumber\widetilde{\phi}_2 + i\bracs{R_{jk}\qm}_2\widetilde{\phi}_3 }} \ \md t \\
	= \omega^2 \int_0^{\abs{I_{jk}}} \widetilde{u}_{1,jk}\overline{\widetilde{\phi}}_1 + \widetilde{u}_{2,jk}\overline{\widetilde{\phi}}_2 + \widetilde{u}_{3,jk}\overline{\widetilde{\phi}}_3 \ \md t. \labelthis\label{eq:LongReductionEquation}
\end{align*}
Where we have used overhead tildes to denote compositions with $r_{jk}$, and $u_{l,jk}$ to denote the $l$\textsuperscript{th} component of $u$ along the edge $I_{jk}$, as before.
Given that this equation holds when we take pairs of the components $\phi_1,\phi_2,\phi_3$ to be zero, we have that the following all hold;
\begin{subequations} \label{eq:SplitVariationalEdgeEqns}
	\begin{align}
		0 &= \omega^2 \int_0^{\abs{I_{jk}}} \widetilde{u}_{1,jk}\overline{\widetilde{\phi}}_1 \ \md t, \\
		\int_0^{\abs{I_{jk}}} \wavenumber\overline{\widetilde{\phi}}_2\bracs{i\widetilde{u}_{3,jk}' + \wavenumber\widetilde{u}_{2,jk} - \bracs{R_{jk}\qm}_2\widetilde{u}_{3,jk}} \ \md t
		&= \omega^2 \int_0^{\abs{I_{jk}}} \widetilde{u}_{2,jk}\overline{\widetilde{\phi}}_2 \ \md t, \\
		\begin{split}
			\int_0^{\abs{I_{jk}}} -\bracs{R_{jk}\qm}_2\overline{\widetilde{\phi}}_3\bracs{i\widetilde{u}_{3,jk}' + \wavenumber\widetilde{u}_{2,jk} - \bracs{R_{jk}\qm}_2\widetilde{u}_{3,jk}} \quad & \\
			+ \overline{\widetilde{\phi}}_3'\bracs{i\widetilde{u}_{3,jk}' - i\wavenumber\widetilde{u}_{2,jk} + i\bracs{R_{jk}\qm}_2\widetilde{u}_{3,jk}} \ \md t
			&= \omega^2 \int_0^{\abs{I_{jk}}} \widetilde{u}_{3,jk}\overline{\widetilde{\phi}}_3 \ \md t.
		\end{split}
	\end{align}
\end{subequations}
Assuming sufficient regularity in $\widetilde{u}_3$ and $\widetilde{u}_2$, we can integrate by parts in the above expressions and observe that we can apply \tstk{should probably have a number for our version of the fundamental lemma of calculus of variations} to obtain three equations;
\begin{align*}
	\widetilde{u}_{1,jk} &= 0, \\
	i\wavenumber\bracs{\diff{}{t} + i\bracs{R_{jk}\qm}_2}\widetilde{u}_{3,jk} + \wavenumber^2\widetilde{u}_{2,jk} &= \omega^2 \widetilde{u}_{2,jk}, \\
	-\bracs{\diff{}{t} + i\bracs{R_{jk}\qm}_2}^2\widetilde{u}_{3,jk} + i\wavenumber\bracs{\diff{}{t} + i\bracs{R_{jk}\qm}_2}\widetilde{u}_{2,jk} &= \omega^2\widetilde{u}_{3,jk}.
\end{align*}
Whilst the equation for $\widetilde{u}_{1,jk}$ is trivial, this is expected because the rotation $R_{jk}$ transforms each edge $I_{jk}$ into a segment of equal length parallel to the $x_2$-axis.
Indeed if one were to solve for $\widetilde{u}_{2,jk}$ and work backwards through the transformations $r_{jk}$ and $R_{jk}$, then a non-zero component $u_1$ would be obtained (except in the obvious geometries).
The other two equations provide us with a system of ODEs on intervals, and we have two such equations for each edge $I_{jk}$ .
Alone these are not enough to determine eigenpairs $\bracs{\omega^2, u}$, however we can obtain more information from \eqref{eq:LongReductionEquation} by now considering the cases when our test functions $\phi$ have support containing at most one vertex, say the vertex $v_j$.
In such a case we can return to \eqref{eq:LongReductionEquation} and integrate by parts, obtaining \tstk{here $j$ is fixed but same sim notation is used - come up with a plan to deal with this!!!}
\begin{align*}
	0 &= \sum_{j\sim k} \sqbracs{ \bracs{ \bracs{\diff{}{t} + i\bracs{R_{jk}\qm}_2}\widetilde{u}_{3,jk} - i\wavenumber\widetilde{u}_{2,jk} }\overline{\widetilde{\phi}_3} }_0^{\abs{I_{jk}}}.
\end{align*}
Note that after integrating by parts, the resulting integral term is identically zero due to \eqref{eq:SplitVariationalEdgeEqns}.
As the support of $\phi$ (hence $\phi_3$) contains only the vertex $v_j$, and $phi$ is smooth so continuous at $v_j$, we can write this equation as
\begin{align*}
	0 &= \sum_{j\sim k} \bracs{\diff{}{t} + i\bracs{R_{jk}\qm}_2}u_{3,jk}\bracs{v_j} - i\wavenumber u_{2,jk}\bracs{v_j}.
\end{align*}
That is, for each vertex $v_j$ we obtain a boundary condition forcing the sum of the values of $u_3$, $u_3'$, and $u_2$ at $v_j$ to be zero. \newline

From the variational formulation, we have obtained two differential equations and one vertex/boundary condition, per vertex.
This is still not enough information to obtain eigenpairs $\bracs{\omega^2, u}$, so we require the additional information from the divergence-free condition and the structure of $\ktcurlSob{\ddom}{\ddmes}$.
Because we are working in $\ktcurlSobDivFree{\ddom}{\ddmes}$, which is a subspace of $\ktcurlSob{\ddom}{\ddmes}$, we know that the component $u_3$ of $u$ is continuous across the vertices of the graph by \tstk{this result needs to be written and referenced, it's the one which so far is just a bunch of IFF statements} and so obtain a further $n-1$ boundary conditions for each vertex of degree $n\geq2$.
The $\ktgrad$-divergence-free condition also provides us with a further differential equation and set of boundary conditions, as in corollary \ref{cory:DivFreeWholeGraph}.
Combining these, we arrive at the following system of equations on each interval $\interval{I_{jk}}$,
\begin{subequations} \label{eq:CurlEdgeEquations}
	\begin{align}
		\widetilde{u}_{1,jk} &= 0, \\
		i\wavenumber\bracs{\diff{}{t} + i\bracs{R_{jk}\qm}_2}\widetilde{u}_{3,jk} + \wavenumber^2\widetilde{u}_{2,jk} &= \omega^2 \widetilde{u}_{2,jk}, \label{eq:CurlEdgeEquations1} \\
		-\bracs{\diff{}{t} + i\bracs{R_{jk}\qm}_2}^2\widetilde{u}_{3,jk} + i\wavenumber\bracs{\diff{}{t} + i\bracs{R_{jk}\qm}_2}\widetilde{u}_{2,jk} &= \omega^2\widetilde{u}_{3,jk}, \label{eq:CurlEdgeEquations2} \\
		-\bracs{\diff{}{t} + i\bracs{R_{jk}\qm}_2}\widetilde{u}_{2,jk} &= i\wavenumber\widetilde{u}_{3,jk}. \label{eq:CurlEdgeEquations3}
	\end{align}
\end{subequations}
These are complimented by the boundary conditions
\begin{subequations} \label{eq:CurlVertexConditions}
	\begin{align}
		u_3 &\text{ is continuous at each } v_j\in V, \\
		0 &= \sum_{j\sim k} \bracs{\diff{}{t} + i\bracs{R_{jk}\qm}_2}u_{3,jk}\bracs{v_j} - i\wavenumber u_{2,jk}\bracs{v_j}, \\
		0 &= \sum_{j\sim k}u_{2,jk}\bracs{v_j}, \ \forall v_j\in V.
	\end{align}
\end{subequations}
Noting that any functions that satisfy two of \eqref{eq:CurlEdgeEquations1}-\eqref{eq:CurlEdgeEquations2} necessarily satisfy the remaining one, \eqref{eq:CurlEdgeEquations}-\eqref{eq:CurlVertexConditions} provide us with enough equations and boundary conditions to solve for the eigenpairs $\bracs{\omega^2, u}$.

\section{Chapter Summary} \label{sec:CurlSummary}
The theory presented in this chapter serves as justification and motivation for the problems that we will be considering in subsequent chapters and work.
We provide a construction of the Sobolev spaces $\curlSob{D}{\nu}$ for a domain $D$ and Borel measure $\nu$, which comes in much the same ilk as that in chapter \ref{ch:ScalarEqns}.
Some examples of the implications of this construction are discussed in section \ref{sec:CurlExamples}, however we do not develop the general theory any further.
Instead the focus of section \ref{sec:ktOperator} has been solely on the spaces we will be using in our wave-propagation problems (see chapter \tstk{chapter where we discuss the physical wave-guide system, probably intro}), and so we move onto the analysis of $\kt$-curls, -gradients, and $\ktgrad$-divergence free conditions on graphs. \newline

Construction and analysis of $\ktcurlSob{\ddom}{\ddmes}$ and $\ktcurlSobDivFree{\ddom}{\ddmes}$ enables us to make sense of variational problems such as \eqref{eq:CurlCurlEquationDivFree} and the analysis of section \ref{sec:ktOperator} provides us with a means to reformulate this problem as a more familiar and accessible system of differential equations.
The operator $\ktgrad$ is our primary concern because it appears in our wave-propagation problems after taking a Fourier transform in the direction of wave-propagation (introducing the variable $\wavenumber$ and removing the $x_3$-dependence) and a Gelfand transform in the periodic cross-section.
This then leaves us with a family of problems parametrised by $\qm$ on the unit graph $\graph$ (or unit cell $\ddom$), posed with respect to the singular measure $\ddmes$.
We end up reaching a characterisation for each of the spaces $\ktgradSob{\ddom}{\ddmes}$, $\ktcurlSob{\ddom}{\ddmes}$, and $\ktcurlSobDivFree{\ddom}{\ddmes}$ in terms of the behaviour on each edge of our graph $\graph$.
This decomposition into some kind of edge-wise behaviour (plus some continuity conditions at the vertices) is expected for $\ktgradSob{\ddom}{\ddmes}$ due to our analysis in chapter \ref{ch:ScalarEqns}.
Using this as motivation, we also follow a similar vein of investigation to deduce that elements of $\ktcurlSob{\ddom}{\ddmes}$ also display this edge-wise behaviour (section \ref{sec:ktOperator}) and some additional matching conditions (section \ref{sec:ktcurlSobExtraProperties}).
In particular we establish that any element $u\in\ktcurlSob{\ddom}{\ddmes}$ has a third component that is an element of $\ktgradSob{\ddom}{\ddmes}$, and that a converse also holds.
Due to the fact that the $\ktgrad$-divergence free condition ties to $\ktgradSob{\ddom}{\ddmes}$ and $\kgradZero{\ddom}{\ddmes}$, it is unsurprising to learn we have an edge-wise characterisation (plus some matching conditions) arising from this condition too.
These conditions together provide us with the tools to reformulate problems such as \eqref{eq:CurlCurlEquationDivFree} into quantum graph problems as in \eqref{eq:CurlEdgeEquations}-\eqref{eq:CurlVertexConditions}, which are more amiable to analytic (and numeric) than the variational formulations. \newline

In chapter \tstk{example systems and numerical results} we will tie together the work of this chapter and chapter \tstk{quantum graphs} and present some example wave-guidance problems, solved using the approaches described in those chapters.
We will not be developing the theory of the spaces $\ktcurlSob{\ddom}{\ddmes}$, $\ktgradSob{\ddom}{\ddmes}$, or $\ktcurlSobDivFree{\ddom}{\ddmes}$ any further than what has been done in this chapter; because for our purposes we now have all the requirements we need to justify and solve the class of problems that we wish to consider.
One can find further information on spaces such as $\curlSob{D}{\nu}$ in \tstk{references, probably Kirill \& Serena's paper} and how they may be used in a more analytic context.