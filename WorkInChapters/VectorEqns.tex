\chapter{Vector Equations} \label{ch:VectorEqns}
Chapter \ref{ch:ScalarEqns} introduced the theory of variational problems with respect to arbitrary (Borel) measures $\nu$ involving scalar-valued functions, and how we can make sense of the notion of ``gradient" and define appropriate Sobolev spaces to work in.
Also investigated was how (under suitable regularity assumptions) the variational problems that can be posed are reduced to quantum graph problems, when the measure $\nu$ supports a graph embedded into the domain $D$. 
In this chapter we look to extend this theory to handling vector-valued systems of equations, which necessitates developing theory for the associated curl- and divergence-operations, as well as the associated Sobolev spaces.
We begin with the definition and construction of these spaces, during which we will demonstrate parallels with the previously seen concepts of ``tangential-gradients".
In fact, many of the arguments in this chapter will be inspired or grounded in the approaches taken in chapter \ref{ch:ScalarEqns}.
After this we will discuss slightly altered versions of these spaces, which will arise from considering the wave-propagation problems of interest to us.
We will provide a worked example in this context, and once again demonstrate how we can obtain a quantum graphs problem from a variational formulation that uses a measure supporting a graph in the $\bracs{x_1,x_2}$-plane.
This will enable us to present some illustrative examples of this theory in action, in chapter \ref{ch:ExampleSystems}.

\section{Vector Sobolev Spaces} \label{sec:VectorSobSpaces}
In this section we discuss the construction of Sobolev spaces of vector-valued functions that possess curls (in some sense) with respect to an arbitrary (Borel) measure $\nu$.
In later sections we will adapt the construction of these spaces to suit our physical wave-guide and photonic-fibre systems (section \ref{sec:OurPhysicalSetup}); by taking into account a Fourier transform along the axis of the wave-guide and also incorporating the quasi-momentum that arises from taking a Gelfand transform in the cross-section.
The reader is invited to bear in mind the arguments employed in section \ref{sec:ScalarSobSpaces} throughout this section. \newline

Take $D\subset\reals^3$ as our 3D-domain, and let $\nu$ be an arbitrary (Borel) measure on $D$.
Set 
\begin{align} \label{eq:WCurlGeneral}
	W &= W\bracs{D,\md\nu} = \overline{ \clbracs{ \bracs{\phi, \curl{}\phi} \ \vert \ \phi\in\bracs{\smooth{D}}^3 } }
\end{align}
as the closure of the set of pairs $\clbracs{\bracs{\phi, \curl{}\phi}}$ in $\ltwo{D}{\nu}^3\times\ltwo{D}{\nu}^3$ for $\phi\in\bracs{\smooth{D}}^3$.
Given that this is essentially the same construction as in section \ref{sec:ScalarSobSpaces}, it is unsurprising to learn that some work must be done before the element $c$ in the pair $\bracs{u,c}\in W$ can be thought of as a ``curl".
Indeed we actually have the same problem as in the gradient case - if both $\bracs{u,c_1}, \bracs{0, c_2}\in W$ then $\bracs{u,c_1+c_2}\in W$.
Fortunately we also have the same methodology available to address this issue, so we begin by defining the set of curls of zero.
\begin{definition}[Set of Curls of Zero] \label{def:CurlZeroGeneral}
	Let $D$, $\nu$ and $W$ be as in \eqref{eq:WCurlGeneral}.
	The set of $\nu$-curls of zero is defined to be
	\begin{align*}
		\curlZero{D}{\nu} &:= \clbracs{ c\in\ltwo{D}{\nu}^3 \ \vert \ \bracs{0,c}\in W }, \\
		&= \clbracs{ c\in\ltwo{D}{\nu}^3 \ \vert \ \exists\phi\in\bracs{\smooth{D}}^3 \text{ s.t. } \phi_n\lconv{\ltwo{D}{\nu}^3}0, \curl{}\phi_n\lconv{\ltwo{D}{\nu}^3}c },
	\end{align*}
	with the construction of $W$ ensuring the expressions for the right hand side coincide.
\end{definition}

We observe that $\curlZero{D}{\nu}$ is a closed linear subspace of $\ltwo{D}{\nu}^3$ so we can decompose
\begin{align*}
	\ltwo{D}{\nu}^3 = \curlZero{D}{\nu}^\perp \oplus \curlZero{D}{\nu}.
\end{align*}
Hence every $u\in\ltwo{D}{\nu}^3$ has a unique tangential $\nu$-curl, $\curl{\nu} u$ such that every $\bracs{u,c}\in W$ can be written in the form $\bracs{u, \curl{\nu}u + \widetilde{c}}\in W$ for some $\widetilde{c}\in\curlZero{D}{\nu}$.
Thus we can define the Sobolev space of functions $u$ with $\nu$-curls.
\begin{definition}[Sobolev Space $\curlSob{D}{\nu}$]
	Let $D$, $\nu$, and $W$ be as in \eqref{eq:WCurlGeneral}.
	Then the Sobolev space of functions with $\nu$-curls is defined to be
	\begin{align*}
	\curlSob{D}{\nu} := \clbracs{ \bracs{u,\curl{\nu}u}\in W \ \vert \ \curl{\nu}u \perp \curlZero{D}{\nu} }.
\end{align*}
	Although elements of this space are technically pairs, it is sufficient to specify only the first component $u$ to identify the pair.
\end{definition}
We can also pose problems in this space using a variational formulation as was done in section \ref{sec:ScalarSobSpaces}.
\begin{convention}[Shorthand for Variational Problems] \label{conv:VariationalShorthandCurls}
We may write the variational problem of finding $u\in\curlSob{D}{\nu}$ such that
\begin{align*}
	\integral{D}{ A\bracs{\curl{\nu}u}\cdot\overline{\bracs{\curl{}\phi}} + u\overline{\phi}}{\nu} &= \integral{D}{ f\overline{\phi} }{\nu}, \quad \forall\phi\in\bracs{\smooth{D}}^3.
\end{align*}
in the shorthand
\begin{align*}
	\curl{\nu}\bracs{A\curl{\nu}u} + u &= f, \quad u\in\curlSob{D}{\nu}.
\end{align*}
\end{convention}
Such problems actually serve as the motivation for the notion of the tangential curl that is provided in the same way as tangential gradients were motivated.
The elliptic matrix $A$ simply requires us to write $A\curl{\nu}u$ for the element in $\curlZero{D}{\nu}^\perp$, the details again being analogous to those in section \ref{sec:ScalarSobSpaces} for gradients. \newline

The problem that we shall be focusing on throughout this chapter is the colloquially-named ``curl of the curl" equation 
\begin{align} \label{eq:curlOfCurlEqn}
	\curl{\nu}\bracs{A\curl{\nu}u} = \omega^2 u, &\quad u\in\curlSob{D}{\nu},
\end{align}
typically arising from the study of Maxwell's equations\footnote{The spectral problem is presented here because one is typically considering a time-harmonic solution to Maxwell's equations in the derivation of this equation. The resolvent problem is a perfectly valid mathematical system but does not tend to arise in applications.}.
The vector field $u$ is interpreted as either the magnetic or electric field, and the matrix $A$ (if present) the inverse of the magnetic permeability or electric permittivity.
Maxwell's equations also include a condition that the divergence of the magnetic field is zero, as is that of the electric field in the presence of no external charges.
This divergence-free condition must be imposed alongside \eqref{eq:curlOfCurlEqn} for a complete physical description of the system\footnote{And in particular requires the forcing function to be divergence-free too, if the resolvent problem is being considered.}.
Due to the nature of our Sobolev spaces and mathematical tools, we can only interpret ``divergence-free" as meaning ``orthogonal to all gradients", as the following definition clarifies.
\begin{definition}[$\nu$-divergence-Free] \label{def:DivFreeGeneral}
	Let $D\subset\reals^3$ and $\nu$ be a Borel measure on $D$.
	Also let
	\begin{align*}
		W_{\mathrm{grad}}\bracs{D,\nu} &:= \overline{\clbracs{ \bracs{\phi, \grad\phi} \ \vert \ \phi\in\smooth{D} }}
	\end{align*}
	where the closure is with respect to the norm of $\ltwo{D}{\nu}\times\ltwo{D}{\nu}^3$.
	A function $u\in\ltwo{D}{\nu}^3$ is $\nu$-divergence free if
	\begin{align*}
		\integral{D}{ u\cdot g }{\nu} &= 0, \quad \forall \bracs{z,g}\in W_{\mathrm{grad}}\bracs{D,\nu}.
	\end{align*}	 
\end{definition}
Of course due to the decomposition of $\ltwo{D}{\nu}^n$ by $\gradZero{D}{\nu}$ and it's orthogonal compliment, we have the following equivalence for functions $u\in\ltwo{D}{\nu}^3$:
\begin{align*}
	u \text{ is } \nu\text{-divergence-free } \quad\Leftrightarrow\quad
	\begin{cases}
		\integral{D}{ u\cdot\overline{\grad_{\nu}v} }{\nu} = 0, & \forall v\in\gradSob{D}{\nu}, \\
		\integral{D}{ u\cdot\overline{g} }{\nu} = 0, & \forall g\in\gradZero{D}{\nu}.
	\end{cases}
\end{align*}
In section \ref{sec:CurlExamples} we demonstrate that, unlike $\nu$-gradients, $\nu$-curls can display some unexpected behaviours for specific measures $\nu$.
Following this example section \ref{sec:ktOperator} will introduce modifications we need to make to the $\grad$ operator, to account for the fact that we will be taking Fourier and Gelfand transforms in our singular-structure problems.
This section will also provide characterisations of the (modified) Sobolev spaces and the divergence-free condition.
Finally, we conclude this chapter by demonstrating the the curl-of-the-curl equation \eqref{eq:curlOfCurlEqn} has an equivalent quantum graph formulation, which will be our starting point for the example systems we consider in chapter \ref{ch:ExampleSystems}.

\section{Example: Segment in $\reals^3$} \label{sec:CurlExamples}
In this section we provide an example of the spaces described in section \ref{sec:VectorSobSpaces}, for the case when we have a line segment in $\reals^3$ and the singular measure supporting it. 
This example will also help us provide an interpretation for tangential $\kt$-curls that we will consider in section \ref{sec:ktCurlsTangential}, and highlight how the vector-function spaces behave differently to their scalar-valued counterparts of chapter \ref{ch:ScalarEqns}. \newline

Let $I$ be a segment in $\reals^3$ with $x_2=x_3=0$ on $I$, and let $\nu$ be the singular measure on $\reals^3$ that supports 1D Lebesgue measure on $I$.
In this setup we demonstrate that $\curlZero{\reals^3}{\nu}$ actually coincides with the whole of $\ltwo{\reals^3}{\nu}$, so the only tangential $\nu$-curl is the zero function!
To this end, let $f\in\smooth{\reals^3}$, $v=\bracs{0,0,f}^\top\in\ltwo{\reals^3}{\nu}^3$ and consider the ``sequence" of functions $\phi^{(n)}=\phi=\bracs{-x_2 f, 0, 0}^\top$.
Then we have that
\begin{align*}
	\integral{\reals^3}{\abs{\phi}^2}{\nu} &= \integral{I}{x_2^2\abs{f}^2}{\nu} \\
	&= 0, \quad \text{as } x_2=0 \text{ on } I.
\end{align*}
Furthermore,
\begin{align*}
	\integral{\reals^3}{\abs{\curl{}\phi - v}^2}{\nu} &= \integral{I}{\abs{0-0}^2}{\nu} + \integral{I}{\abs{-x_2\partial_3 f - 0}^2}{\nu} + \integral{I}{\abs{f + x_2\partial_2 f - f}^2}{\nu} \\
	&= \integral{I}{x_2^2\abs{f}^2}{\nu} + \integral{I}{x_2^2\abs{\partial_2 f}^2}{\nu}
	= 0.
\end{align*}
Thus the (constant) sequence $\bracs{\phi^{(n)}}_{n\in\naturals}$ is such that
\begin{align*}
	\phi^{(n)} \lconv{\ltwo{\reals^3}{\nu}^3} 0, &\quad \curl{}\phi^{(n)} \lconv{\ltwo{\reals^3}{\nu}^3} v,
\end{align*}
and thus $v\in\curlZero{\reals^3}{\nu}$.
A similar argument then applies in the cases when;
\begin{itemize}
	\item $v=\bracs{0, f, 0}^\top$, by choosing the constant sequence $\phi = \bracs{x_3 f, 0, 0}^\top$.
	\item $v=\bracs{f, 0, 0}^\top$, by choosing the constant sequence $\phi = \bracs{0, -x_3 f, 0}^\top$.
\end{itemize}
As such we conclude that if $f\in\smooth{\reals^3}$, then
\begin{align*}
	\begin{pmatrix}	f \\ 0 \\ 0	\end{pmatrix},
	\begin{pmatrix}	0 \\ f \\ 0	\end{pmatrix},
	\begin{pmatrix}	0 \\ 0 \\ f	\end{pmatrix}
	\in \curlZero{\reals^3}{\nu}.
\end{align*}
But then applying density arguments for $\smooth{\reals^3}$ in $\ltwo{\reals^3}{\nu}$ and the fact that $\curlZero{\reals^3}{\nu}$ is a closed linear subspace of $\ltwo{\reals^3}{\nu}^3$ by definition, we must conclude that
\begin{align*}
	\curlZero{\reals^3}{\nu} &= \mathrm{span}\clbracs{	
	\begin{pmatrix}	f \\ 0 \\ 0	\end{pmatrix},
	\begin{pmatrix}	0 \\ f \\ 0	\end{pmatrix},
	\begin{pmatrix}	0 \\ 0 \\ f	\end{pmatrix}
	\ \vert \ f\in\ltwo{\reals^3}{\nu}
	} \\
	&= \ltwo{\reals^3}{\nu}.
\end{align*}
This leaves us with the conclusion that the space $\curlZero{\reals^3}{\nu} = \ltwo{\reals^3}{\nu}^3$, that is the only tangential curl is the zero function!
Fortunately for us this does not mean our singular-structure domains as described in section \ref{sec:OurPhysicalSetup} give only trivial solutions.
This is because we have planes in (a subset of) $\reals^3$ with each plane induced by an edge of the graph in the cross section, extended down the axis of the fibre. 
Upon taking the Fourier transform in the axial direction we then arrive at the familiar singular-structure problem in the cross-section of the fibre, crucially having a 1D structure embedded in a 2D domain, rather than in a 3D domain as is the case with this example.
As such we are required to consider a modified $\grad$ operator when constructing the spaces we will be using for our singular-structure problems; one which arises from taking the Fourier (and Gelfand) transforms of our problems on infinite, periodic domains.
This is the subject of the following section \ref{sec:ktOperator}.
We will also highlight how this example is consistent with the interpretation we obtain for the tangential $\kt$-curl in section \ref{sec:ktCurlsTangential}.

\section{The $\ktgrad$ operator} \label{sec:ktOperator}
We now move into a context that better suits the domain setups that we described in section \ref{sec:OurPhysicalSetup}.
Throughout this section we assume we have some period cell $\ddom$ in which is the period graph $\graph=\bracs{V,E}$ of some larger periodic embedded graph, and we denote by $\ddmes$ the singular measure of $\graph$.
Because we will be making use of Fourier and Gelfand transforms, it will be necessary for us to consider slightly modified gradient, curl, and divergence operators (similar to that which we required in section \ref{sec:sec:ScalarSystem}).
As such we introduce the Fourier variable $\wavenumber$ and accordingly adjust the $\grad$ operator to account for this transform, denoting by $\kgrad$ the operator that acts component-wise as
\begin{align*}
	\kgrad := \begin{pmatrix} \partial_1 \\ \partial_2 \\ i\wavenumber\end{pmatrix}.
\end{align*}
Then due to our Gelfand transform in the periodic cross-section, we have to make a further adjustment to $\kgrad$ due to the quasi-momentum $\qm$ and consider the operator $\ktgrad$ which component-wise is acts as
\begin{align*}
	\ktgrad := \kgrad + i\begin{pmatrix} \qm_1 \\ \qm_2 \\ 0 \end{pmatrix}.
\end{align*}
One can then construct the sets
\begin{subequations} \label{eq:WktDefinitions}
	\begin{align}
		W^{\wavenumber, \qm}_{\mathrm{curl}} &:= \overline{ \clbracs{ \bracs{\phi, \ktcurl{}\phi} \ \vert \ \phi\in\bracs{\smooth{\ddom}}^3 } }, \\
		W^{\wavenumber, \qm}_{\mathrm{grad}} &:= \overline{ \clbracs{ \bracs{\phi, \ktgrad\phi} \ \vert \ \phi\in\smooth{\ddom} } },
	\end{align}
\end{subequations}
with closures taken with respect to the respective product $\ltwo{\ddom}{\ddmes}$ norms.
One then defines the relevant sets of curls and gradients of zero, and thus the spaces $\ktcurlSob{\ddom}{\ddmes}$ and $\ktgradSob{\ddom}{\ddmes}$ as follows.
\begin{definition}[$\kt$-Sobolev Spaces] \label{def:ktSobSpaces}
	Let $\ddom$, $\graph$, $\ddmes$, $W^{\wavenumber, \qm}_{\mathrm{curl}}$, and $W^{\wavenumber, \qm}_{\mathrm{grad}}$ be as in \eqref{eq:WktDefinitions}.
	Denote the set of ``$\wavenumber$-curls of zero" by
	\begin{align*}
		\kcurlZero{\ddom}{\ddmes} &= \clbracs{ c \ \vert \ \bracs{0,c}\in W^{\wavenumber}_{\mathrm{curl}}},
	\end{align*}
	and the set of ``$\wavenumber$-gradients of zero" by
	\begin{align*}
		\kgradZero{\ddom}{\ddmes} &= \clbracs{ z \ \vert \ \bracs{0,z}\in W^{\wavenumber}_{\mathrm{grad}}}.
	\end{align*}
	It will be shown in the relevant sections that these sets are invariant under changes to $\qm$ hence we only write the superscript $\wavenumber$ on them.
	Then for a given $\kt$, one constructs the Sobolev spaces of $\kt$-curls and $\kt$-gradients (respectively) as
	\begin{align*}
		\ktcurlSob{\ddom}{\ddmes} &= \clbracs{ \bracs{u,\ktcurl{\ddmes}u}\in W^{\wavenumber, \qm}_{\mathrm{curl}} \ \vert \ \ktcurl{\ddmes}u \perp \kcurlZero{\ddom}{\ddmes} }, \\
		\ktgradSob{\ddom}{\ddmes} &= \clbracs{ \bracs{u,\ktgrad_{\ddmes}u}\in W^{\wavenumber, \qm}_{\mathrm{grad}} \ \vert \ \ktgrad_{\ddmes}u \perp \kgradZero{\ddom}{\ddmes} }.
	\end{align*}
\end{definition}
Since $\ktgrad$ is nothing more than the result of unitary transforms on the function spaces considered in section \ref{sec:VectorSobSpaces}; the theory concerning uniqueness of tangential curls and gradients is preserved, however we must still perform some explicit computations to characterise the objects that we will be working with.
Since the operator $\ktgrad$ replaces the operator $\grad$, and we are dealing with $\kt$-tangential curls and $\kt$-tangential gradients, we also have to use a ``$\kt$-divergence-free condition", as below.
\begin{definition}[$\kt$-divergence-free] \label{def:ktDivFreeCond}
	Let $\ddom$, $\graph$, $\ddmes$, and $W^{\wavenumber, \qm}_{\mathrm{grad}}$ be as in \eqref{eq:WktDefinitions}.
	A function $u\in\ltwo{\ddom}{\ddmes}^3$ is $\kt$-divergence-free if
	\begin{align*}
		\integral{\ddom}{ u\cdot g }{\ddmes} &= 0, \quad \forall \bracs{z,g}\in W^{\wavenumber, \qm}_{\mathrm{grad}}.
	\end{align*}
\end{definition}
Again due to the decomposition of $\ltwo{\ddom}{\ddmes}^3$ by $\gradZero{\ddom}{\ddmes}$ and it's orthogonal compliment, we have the following equivalence for functions $u\in\ltwo{\ddom}{\ddmes}^3$:
\begin{align*}
	u \text{ is } \kt\text{-divergence-free } \quad\Leftrightarrow\quad
	\begin{cases}
		\integral{\ddom}{ u\cdot\overline{\ktgrad_{\ddmes}v} }{\ddmes} = 0, & \forall v\in\ktgradSob{\ddom}{\ddmes}, \\
		\integral{\ddom}{ u\cdot\overline{g} }{\ddmes} = 0, & \forall g\in\kgradZero{\ddom}{\ddmes}.
	\end{cases}
\end{align*}
Finally, to save repeating the same assumptions in the results that follow, we adopt the following convention throughout the remainder of this chapter.
\begin{convention}[Local Co-ordinates for Edges $I_{jk}$] \label{conv:LocalEdgeCoords}
	Let $\graph=\bracs{V,E}$ be an embedded graph in $\ddom$.
	Then for each $I_{jk}\in E$, equip $I_{jk}$ with local co-ordinate system $y_{jk} = \bracs{y_{jk}^{(1)},y_{jk}^{(2)}}$ with $y_{jk}^{(2)}$ parallel to $I_{jk}$.
	Write $R_{jk}\in\mathrm{SO}(2)$ for the rotation such that $x=R_{jk}y_{jk}$ where $x=\bracs{x_1,x_2}$ is the axis co-ordinate system.
\end{convention}
In also recall that convention \ref{conv:StraightEdges} gives us a ready notation for the parametrisation $r_{jk}$ for each $I_{jk}$, the notation $e_{jk}$ for the unit vector directed along $I_{jk}$ from $v_j$ to $v_k$, and $\lambda_{jk}$ for the singular measure supported on $I_{jk}$.
As such $e_{jk}$ is the unit vector directed in $y^{(jk)}_2$, and we will use $n_{jk}$ to denote the unit normal to $I_{jk}$ directed along $y^{(jk)}_1$.

\subsection{$\kt$-gradients} \label{sec:ktGradients}
In this subsection we will work towards characterising $\bracs{\wavenumber,\qm}$-tangential gradients.
This will first involve us characterising the set of gradients of zero, $\kgradZero{\ddom}{\ddmes}$, however there will be several parallels with the methods and results of section \ref{sec:ScalarExample}. 
We begin by showing that the gradients of zero that we shall be dealing with are invariant under changes in $\qm$ (just like in the scalar case) and so throughout the notation $\kgradZero{\ddom}{\ddmes}$ will be sufficient, and we can always consider the $\qm=0$ case in our working.
\begin{prop}[$\wavenumber$-Gradients of Zero are Invariant Under Quasi-Momentum] \label{prop:kGradZeroInvarientUnderQM}
	Let $\qm\in[-\pi,\pi)^2$ and set 
	\begin{align*}
		\mathcal{G}^{\kt}_{\ddom, \mathrm{d}\ddmes} &= \clbracs{ z \ \vert \ \bracs{0,z}\in W^{\wavenumber, \qm}_{\mathrm{grad}}}, \\
		\kgradZero{\ddom}{\ddmes} &= \clbracs{ z \ \vert \ \bracs{0,z}\in W^{\wavenumber, 0}_{\mathrm{grad}}}.
	\end{align*}
	Then
	\begin{align*}
		\mathcal{G}^{\kt}_{\ddom, \mathrm{d}\ddmes} &= \kgradZero{\ddom}{\ddmes}.
	\end{align*}
\end{prop}
\begin{proof}
	This proof is essentially identical to proposition \ref{prop:GradZeroInvarientUnderQM}; note that the third component of $\ktgrad\phi$ doesn't depend on $\qm$, and then apply the argument of that proposition to the first and second components to obtain the result.
\end{proof}

Next we move onto characterising $\kgradZero{\ddom}{\ddmes}$, electing to use the same procedure as in section \ref{sec:ScalarExample} and find an edge-wise characterisation, then combine the results to provide a characterisation across a general graph.
To this end we note that the arguments in section \ref{sec:ScalarExample} all apply to the first and second components of $\ktgrad\phi$ (they are in fact identical); this includes the results regarding orientation-preserving orthogonal rotations in the $\bracs{x_1,x_2}$-plane.
Because of this, we only present those parts of the following proofs that concern the third components of $\kt$-gradients.
We begin by assuming that our underlying graph $\graph$ consists of a single edge parallel to the $x_2$-axis.
\begin{prop}[$\wavenumber$-Gradients of Zero on a Segment Parallel to the $x_2$-axis] \label{prop:kGradZeroParallel}
	Let $I$ be a segment in $\ddom$ parallel to the $x_2$-axis and $\lambda_I$ the singular measure supported on $I$.
	Then
	\begin{align*}
		\kgradZero{\ddom}{\lambda_I} = \clbracs{ 
		\begin{pmatrix} f \\ 0 \\ 0	\end{pmatrix} \ \vert \ f\in\ltwo{\ddom}{\lambda_I}
		}
	\end{align*}
\end{prop}
\begin{proof}
	Corollary \ref{cory:CharacterisationGradientsZero} deals with the first two components, so it remains for us to examine the third component.
	Suppose that $f\in\ltwo{\ddom}{\lambda_I}$ and that $\bracs{0,0,f}^\top\in\kgradZero{\ddom}{\lambda_I}$, then there exists a sequence of smooth functions $\phi_n\in\smooth{\ddom}$ such that
	\begin{align*}
		\phi_n \lconv{\ltwo{\ddom}{\lambda_I}} 0, &\quad \begin{pmatrix} \partial_1\phi_n \\ \partial_2\phi_n \\ i\wavenumber\phi_n \end{pmatrix} = \ktgrad\phi_n \lconv{\ltwo{\ddom}{\lambda_I}^3} \begin{pmatrix} 0 \\ 0 \\ f	\end{pmatrix}.
	\end{align*}
	Thus $\phi_n\rightarrow0$ but $i\wavenumber\phi_n\rightarrow f$, hence we must conclude that $f=0$.
\end{proof}

We expect this result as it means that after taking the Fourier transform, our formulation still accounts for (or ``can see") the direction of wave propagation.
Additionally, as the third component of $\ktgrad\phi$ is unaffected by rotations in the $\bracs{x_1,x_2}$-plane, we can utilise proposition \ref{prop:kGradZeroParallel} to state the following results.
\begin{cory}[$\wavenumber$-Gradients of Zero on a Segment] \label{cory:kGradZeroArbitraryAngle}
	Let $I$ be a segment in $\ddom$ with local orthogonal co-ordinate system $y=\bracs{y_1,y_2}$ with $y_2$ parallel to $I$, and let $\lambda_I$ be the singular measure supported on $I$.
	Suppose $R\in\mathrm{SO}\bracs{2}$ is the change of co-ordinates $x=Ry$ where $x=\bracs{x_1,x_2}$ is the axis co-ordinate system, and let $e_I$ be the unit vector directed along $I$ (so $e_I$ is parallel to $y_2$ and has length $1$).
	Then
	\begin{align*}
		\kgradZero{\ddom}{\lambda_I} &= \clbracs{ 
		\begin{pmatrix} R^{\top} & 0 \\ 0 & 1 \end{pmatrix}\begin{pmatrix} g_1 \\ 0 \\ 0 \end{pmatrix}				\ \vert \ g_1\in\ltwo{\ddom}{\lambda_I}
		} \\
		&= \clbracs{ 
		\begin{pmatrix} g_1 \\ g_2 \\ 0 \end{pmatrix} 
		\ \vert \ g_1,g_2\in\ltwo{\ddom}{\lambda_I}, \ \begin{pmatrix} g_1 \\ g_2 \\ 0 \end{pmatrix} \cdot e_I = 0
		}
	\end{align*}
\end{cory}
\begin{proof}
	This proof is an application of a rotation to proposition \ref{prop:kGradZeroParallel}, similar in detail to that of proposition \ref{prop:RotationOfEdgeGradients}.
\end{proof}

This brings us to the following characterisation of $\kgradZero{\ddom}{\ddmes}$.
\begin{cory}[Characterisation of $\kgradZero{\ddom}{\ddmes}$] \label{cory:CharacterisationkGradientsZero}
	Let $\graph=\bracs{V,E}$ be a finite graph embedded in $\ddom$ and $\ddmes$ the singular measure supported on the edges of $\graph$.
	Also recall convention \ref{conv:LocalEdgeCoords}.
	Then
	\begin{align*}
		\kgradZero{\ddom}{\ddmes} &= 
		\clbracs{ 
		g\in\ltwo{\ddom}{\ddmes}^3 \ \vert \
		g\vert_{I_{jk}} = \begin{pmatrix} R_{jk}^{\top} & 0 \\ 0 & 1 \end{pmatrix} \begin{pmatrix} g_{jk} \\ 0 \\ 0 \end{pmatrix}, \ g_{jk}\in\ltwo{\ddom}{\lambda_{jk}}
		} \\
		&=
		\clbracs{ 
		\begin{pmatrix} g_1 \\ g_2 \\ 0 \end{pmatrix}		
		\in\ltwo{\ddom}{\ddmes}^3 \ \vert \ \left.
		\begin{pmatrix} g_1 \\ g_2 \end{pmatrix}
		\right\rvert_{I_{jk}}\cdot e_{jk} = 0, \ \forall I_{jk}\in E 
		} \\
		&= \clbracs{ g\in\ltwo{\ddom}{\ddmes}^3 \ \vert \ g\in\kgradZero{\ddom}{\lambda_{jk}}, \ \forall I_{jk}\in E }.
	\end{align*}
\end{cory}
\begin{proof}
	Corollary \ref{cory:kGradZeroArbitraryAngle} provides us with the form for $\kgradZero{\ddom}{\lambda_I}$ on each $I_{jk}$.
	By following the argument of section \ref{sec:ScalarExample}, we can deduce results analogous to \ref{prop:RotationOfEdgeGradients}, \ref{cory:Grad0SingleEdge}, \ref{prop:Grad0IncB}, \ref{lem:SegGradExtend}, and \ref{prop:BIncGrad0}, from which this result follows.
\end{proof}

Establishing this characterisation now allows us to determine the form for tangential $\kt$-gradients, however we should note that from now on it will be necessary for us to account for the quasi-momentum $\qm\in[-\pi,\pi)^2$ as tangential gradients will depend on this.
Of course it is sufficient for us to determine an edge-wise form for the tangential gradient, and the arguments and results of section \ref{sec:ScalarSystem} apply to provide us with the forms for the first and second component of the tangential $\kt$-gradient.
As such all that remains for us is to explore the third component of the tangential gradient on an edge, and then extend the aforementioned arguments.
We now provide the remaining details for characterising the tangential $\kt$-gradient on a segment parallel to the $x_2$-axis, then simply state it's general form, referring the reader back to the arguments of section \ref{sec:ScalarSystem} which are easily adapted.
\begin{prop}[Tangential $\kt$-Gradient on a Segment Parallel to the $x_2$-axis] \label{prop:ktTangentialGradientSegment}
	Let $I$ be a segment in $\ddom$ parallel to the $x_2$-axis, and $\lambda_I$ the singular measure supported on $I$.
	Write $\gradSob{\interval{I}}{t}$ for the ``classical" Sobolev space on the interval $\interval{I}$ with respect to the Lebesgue measure; and let $r:\interval{I}\rightarrow I$ be the change of variables map $r(t) = v_I + te_I$, for the unit vector $e_I$ directed along $I$ directed and away from the appropriate endpoint $v_I$ of $I$.
	Suppose $u\in\ktgradSob{\ddom}{\lambda_I}$ and let $\widetilde{u} = u\circ r$. 
	Then $\widetilde{u}\in\gradSob{\interval{I}}{t}$ and 
	\begin{align*}
		\ktgrad_{\lambda_I}u &= \begin{pmatrix} 0 \\ u' + i\qm_2 u \\ i\wavenumber u\end{pmatrix}
	\end{align*}
	where $u' = \widetilde{u}'\circ r^{-1}$.
\end{prop}
\begin{proof}
	Setting $\ktgrad_{\lambda_I}u = \bracs{v_1,v_2,v_3}^{\top}$, the condition of orthogonality to $\kgradZero{\ddom}{\lambda_I}$ requires that
	\begin{align*}
		0 &= \integral{\ddom}{\ktgrad_{\lambda_I}u \cdot \overline{g}}{\lambda_I} \quad \forall g\in\kgradZero{\ddom}{\lambda_I} \\
		&= \integral{I}{v_1 \overline{g}_1}{\lambda_{I}} \quad \forall g_1\in\ltwo{\ddom}{\lambda_I},
	\end{align*}
	so $v_1=0$.
	Turning our attention to the other components, take any sequence $\phi_n\in\smooth{\ddom}$ such that $\phi_n\rightarrow u, \ktgrad\phi_n\rightarrow \bracs{0,v_2,v_3}^\top$.
	These convergences imply that
	\begin{align*}
		\phi_n \lconv{\ltwo{\ddom}{\lambda_I}} u, 
		&\quad \partial_1\phi_n + i\qm_1 \phi_n \lconv{\ltwo{\ddom}{\lambda_I}} 0, \\
		\partial_2\phi_n + i\qm_2 \phi_n &\lconv{\ltwo{\ddom}{\lambda_I}} v_2, 
		&\quad i\wavenumber\phi_n &\lconv{\ltwo{\ddom}{\lambda_I}} v_3.
	\end{align*}
	But $\phi_n \lconv{\ltwo{\ddom}{\lambda_I}} u$ implies that $v_3 = i\wavenumber u$.
	This leaves us determining the form for $v_2$, where we direct the reader to the process detailed in lemma \ref{lem:TangentialQMGradientSegmentParallel} which demonstrates how to show that $\widetilde{u}\in\gradSob{\interval{I}}{t}$ and hence $v_2 = u' + i\qm_2 u$, which completes the proof.
\end{proof}

The form for the tangential $\kt$-gradient when the segment $I$ is at an arbitrary angle in the $\bracs{x_1,x_2}$-plane follows by applying a rotation to the previous result, and from this we deduce an analogue of corollary \ref{cory:TangentialQMGradientGraphs}.
\begin{cory}[$\kt$-Tangential Gradient on Graphs] \label{cory:ktTangentialGradientForm}
	Let $\graph=\bracs{V,E}$ be a graph embedded into $\ddom$ and let $\ddmes$ be the singular measure supported on $\graph$.
	Recall convention \ref{conv:LocalEdgeCoords}.
	If $u\in\ktgradSob{\ddom}{\ddmes}$, then we have that
	\begin{align*}
		\widetilde{u}_{jk} = u_{jk} \circ r_{jk} \in \ktgradSob{\interval{I_{jk}}}{t}, &\ \forall I_{jk}\in E, \\
		\left.\ktgrad_{\ddmes}u\right\vert_{I_{jk}} = \begin{pmatrix} R_{jk}^{\top} & 0 \\ 0 & 1 \end{pmatrix} \begin{pmatrix} 0 \\ u_{jk}' + i\bracs{R_{jk}\qm}_2 u_{jk} \\ i\wavenumber u_{jk} \end{pmatrix}, &\ \forall I_{jk}\in E,
	\end{align*}
	where $u_{jk}' = \widetilde{u}_{jk}' \circ r_{jk}^{-1}$.
\end{cory}
\begin{proof}
	It is sufficient to find an edge-wise characterisation for the tangential $\kt$-gradient, and we can apply a rotation to the result of proposition \ref{prop:ktTangentialGradientSegment} to obtain the form for the tangential $\kt$-gradient on any edge $I_{jk}$.
	This form is
	\begin{align*}
		\ktgrad_{\lambda_{jk}}u_{jk} &= \begin{pmatrix} R_{jk}^{\top} & 0 \\ 0 & 1 \end{pmatrix} \begin{pmatrix} 0 \\ u_{jk}' + i\bracs{R_{jk}\qm}_2 u_{jk} \\ i\wavenumber u_{jk} \end{pmatrix},
	\end{align*}
	from which the result claimed above follows.
\end{proof}

Finally we round off this section by stating the analogue of \ref{thm:CharGradSob}, which provides us with some non-obvious additional structure for elements of $\ktgradSob{\ddom}{\ddmes}$.
\begin{theorem}[Characterisation of $\ktgradSob{\ddom}{\ddmes}$] \label{thm:ktGradSobChar}
	We have that
	\begin{align*}
		u\in\ktgradSob{\ddom}{\ddmes} \quad \Leftrightarrow 
		&\quad\text{(i)} \ u\in\ktgradSob{\ddom}{\lambda_{jk}} \ \forall I_{jk}\in E, \text{ and}\\
		&\quad\text{(ii)} \ u \text{ is continuous at each } v_j\in V.
	\end{align*}
\end{theorem}
\begin{proof}
	This is identical to the proof of theorem \ref{thm:CharGradSob}, once the realisation is made that the additional third component in $\kt$-gradients does not affect the argument of the proof.
\end{proof}

\subsection{The set $\kcurlZero{\ddom}{\ddmes}$} \label{sec:ktCurlsZero}
Our primary focus here is to characterise $\wavenumber$-curls of zero, and our analysis follows in much of the same vein as in section \ref{sec:ScalarExample}.
First we seek an edge-wise understanding, then we derive an extension result (c.f. lemma \ref{lem:kCurlZeroExtensionLemma}) so that our edge-wise characterisation also provides us with a graph-wide form.
We will need the following result on rotations during this section:
\begin{lemma}[Curl under rotation] \label{lem:CurlUnderRotation}
	Let $x=\bracs{x_1,x_2,x_3}^\top, y=\bracs{y_1,y_2,y_3}^\top$ be right-handed orthonormal co-ordinate systems and suppose $R\in\mathrm{SO}(3)$ is such that $x=Ry$.
	Let $u$ be a vector field, and denote by $u_x =\bracs{u_{x,1},u_{x,2},u_{x,3}}^\top$ it's representation in the $x$ co-ordinate system, and $u_y =\bracs{u_{y,1},u_{y,2},u_{y,3}}^\top$ it's representation in the $y$ co-ordinate system.
	Denote by $\grad_x$ and $\grad_y$ the $\grad$ operator in the two co-ordinate systems, respectively.
	Then
	\begin{align*}
		\curl{y} u_y\bracs{R^\top x} &= R \curl{x} u_x\bracs{R^\top x}, \\
		\curl{x} u_x\bracs{R^\top x} &= R^\top\curl{y} u_y\bracs{R^\top x}.
	\end{align*}
\end{lemma}
\begin{proof}
	We use index notation and Einstein summation conventions throughout this proof.
	We also use $\epsilon_{jkm}$ to represent the Levi-Civita symbol for permutations of three elements, and $\delta_{jk}$ for the Kronecker delta.
	As $R\in\mathrm{SO}(3)$, $R^\top=R^{-1}$ and $\det R = 1$.
	Thus $R$ is the adjugate matrix of it's own transpose, that is $R = \mathrm{Adj}\bracs{R^\top}$ so we have that
	\begin{align*}
		R_{jk} &= \recip{2}\epsilon_{mnj}\epsilon_{pqk}R^\top_{mp}R^\top_{nq} \\
		\implies 2R_{jk}\epsilon_{kvw} &= \epsilon_{kvw}\epsilon_{mnj}\epsilon_{pqk}R^\top_{mp}R^\top_{nq} \\
		&= \epsilon_{mnj}R^\top_{mp}R^\top_{nq}\bracs{\delta_{pv}\delta_{qw} - \delta_{pw}\delta_{qv}} \\
		&= \epsilon_{mnj}\bracs{R^\top_{mv}R^\top_{nw} - R^\top_{nv}R^\top_{mw}}
		= 2\epsilon_{mnj}R^\top_{mv}R^\top_{nw} \\
		\implies R_{jk}\epsilon_{kml} &= \epsilon_{jkn}R_{mk}R_{ln}.
	\end{align*}
	We now use this identity to prove the result of the lemma;
	\begin{align*}
		\bracs{\grad_y\wedge u_y\bracs{R^\top x}}_j &= \epsilon_{jkl}\pdiff{u_{y,l}}{y_k}
		= \epsilon_{jkl}\pdiff{}{y_k}\bracs{R_{ml}u_{x,m}} \\
		&= \epsilon_{jkl}R_{ml}\pdiff{x_n}{y_k}\pdiff{u_{x,m}}{x_n} \\
		&= \epsilon_{jkl}R_{ml}R_{nk}\pdiff{u_{x,m}}{x_n} \\
		&= R_{jk}\epsilon_{knm}\pdiff{u_{x,m}}{x_n} \\
		&= \bracs{ R\grad_x\wedge u_x\bracs{R^\top x} }_j.
	\end{align*}
	From which left-multiplication by $R^\top$ gives the other desired equality.
\end{proof}

Before we delve into analysis of $\kcurlZero{\ddom}{\ddmes}$, we first demonstrate that this set is invariant under changes in $\qm$.
\begin{prop}[Invariance of $\kcurlZero{\ddom}{\nu}$ under $\qm$]
	Let $\nu$ be a Borel measure on $\ddom$ and $\qm\in[-\pi,\pi)^2$.
	Set
	\begin{align*}
		\kcurlZero{\ddom}{\nu} &= \clbracs{ c \ \vert \ \bracs{0,c}\in W^{\wavenumber, 0}}, \\
		\mathcal{C}_{\ddom, \md\nu}^{\kt}(0) &= \clbracs{ c \ \vert \ \bracs{0,c}\in W^{\wavenumber, \qm} },
	\end{align*}
	for $W$ as defined in \eqref{eq:WktDefinitions}.
	Then
	\begin{align*}
		\kcurlZero{\ddom}{\nu} &= \mathcal{C}_{\ddom, \md\nu}^{\kt}(0).
	\end{align*}
\end{prop}
\begin{proof}
	The argument is essentially identical in both directions, so without loss of generality we only show one of the two required set inclusions.
	Let us take some $\qm\in[-\pi,\pi)^2$ and $c\in\kcurlZero{\ddom}{\nu}$.
	Then we can find an approximating sequence $\phi^{(n)}\in\bracs{\smooth{\ddom}}^3$ such that
	\begin{align*}
		\phi^{(n)} \lconv{\ltwo{\ddom}{\nu}^3} 0, &\quad \kcurl{}\phi^{(n)} \lconv{\ltwo{\ddom}{\nu}^3} c.
	\end{align*}
	The convergence $\phi^{(n)}\rightarrow0$ implies that 
	\begin{align*}
		i \begin{pmatrix} \qm_1 \\ \qm_2 \\ 0\end{pmatrix}\wedge\phi^{(n)}\lconv{\ltwo{\ddom}{\nu}^3} 0
	\end{align*}
	too, and thus
	\begin{align*}
		\ktcurl{}\phi^{(n)} &= \kcurl{}\phi^{(n)} + i\begin{pmatrix} \qm_1 \\ \qm_2 \\ 0\end{pmatrix}\wedge\phi^{(n)} \\
		&\lconv{\ltwo{\ddom}{\nu}^3} c + 0 = c.
	\end{align*}
	Hence, $c\in \mathcal{C}_{\ddom, \md\nu}^{\kt}(0)$.
\end{proof}

This result means that the following analysis of ``$\wavenumber$-curls of zero" is simplified, as we can always consider the $\qm=0$ case to obtain a general result for any $\qm\in[-\pi,\pi)^2$, as well as posthumously justifying the use of the term $\wavenumber$-curls of zero.
We begin our analysis by considering a single segment in the plane, parallel to the $x_2$-axis (if the reader is comparing the results of this section to that of chapter \ref{ch:ScalarEqns}, note that in that chapter single segments were taken parallel to the $x_1$-axis).
\begin{prop}[$\wavenumber$-Curls of Zero on a Segment Parallel to the $x_2$-axis] \label{prop:kCurlZeroParallel}
	Let $I$ be a segment in the $\bracs{x_1,x_2}$-plane parallel to the $x_1$-axis, and let $\lambda_I$ be the singular measure supported on $I$.
	Then
	\begin{align*}
		\kcurlZero{\ddom}{\lambda_I} &= 
		\clbracs{
			\begin{pmatrix} 0 \\ f_2 \\ f_3 \end{pmatrix}
			\ \vert \ f_2,f_3\in\ltwo{\ddom}{\lambda_I}					
		}
	\end{align*}
\end{prop}
\begin{proof}
	Without loss of generality we assume that $x_1=0$ on $I$, otherwise we apply a translation to the functions we present below.
	Throughout the proof let $f\in\smooth{\ddom}$, it will be sufficient for us to only work with smooth functions in this proof, as we can apply a density argument to the results we deduce to extend them to $\ltwo{\ddom}{\lambda_I}$ functions. \newline
	
	First suppose that the function $v=\bracs{f,0,0}^{\top}\in\kcurlZero{\ddom}{\lambda_I}$.
	Then there exists an approximating sequence $\phi^{(n)}\in\smooth{\ddom}^3$ such that
	\begin{align*}
		\phi^{(n)} \rightarrow 0, 
		&\quad \kcurl{}\phi^{(n)} \rightarrow v,
	\end{align*}
	in $\ltwo{\ddom}{\lambda_I}^3$.
	In particular the convergence $\phi^{(n)} \rightarrow 0$ implies that
	\begin{align*}
		\phi^{(n)}_j \rightarrow 0, &\quad\forall j=1,2,3, \text{ in } \ltwo{\ddom}{\lambda_I},
	\end{align*}
	which alongside the component-wise consequences of $\kcurl{}\phi^{(n)} \rightarrow v$ results in the convergences
	\begin{align*}
		\phi^{(n)}_3 \rightarrow 0, \quad \partial_2\phi^{(n)}_3\rightarrow f, \quad \partial_1\phi^{(n)}_3 \rightarrow 0,
	\end{align*}
	in $\ltwo{\ddom}{\lambda_I}$.
	Thus,
	\begin{align*}
		\phi^{(n)}_3 \lconv{\ltwo{\ddom}{\lambda_I}} 0, 
		&\quad \grad\phi^{(n)}_3 \lconv{\ltwo{\ddom}{\lambda_I}^2} \begin{pmatrix} 0 \\ f \end{pmatrix}
	\end{align*}
	so $\bracs{0,f}^{\top}\in\gradZero{\ddom}{\lambda_I}$, but by corollary \ref{cory:Grad0SingleEdge} we must conclude that $f=0$. \newline
	
	Now let $v=\bracs{0,f,0}^{\top}$, we will show by constructing an explicit approximating sequence that $v\in\kcurlZero{\ddom}{\lambda_I}$.
	Indeed we can even take $\phi^{(n)} = \phi = \bracs{0,x_1 f,0}^{\top}$ to be a constant sequence, giving
	\begin{align*}
		\kcurl{}\phi^{(n)} &= \begin{pmatrix} i\wavenumber x_1 f \\ 0 \\ x_1\partial_1 f + f	\end{pmatrix}.
	\end{align*}
	Then by explicit computation, 
	\begin{align*}
		\integral{\ddom}{\abs{\phi}^2}{\lambda_I} &= \integral{I}{x_1^2\abs{f}^2}{\lambda_I} = 0, \\
		\integral{\ddom}{\abs{\kcurl\phi - v}^2}{\lambda_I} &= \integral{I}{x_1^2\bracs{ \bracs{\partial_1 f}^2 - \wavenumber^2 f^2}}{\lambda_I} = 0.
	\end{align*}
	So trivially
	\begin{align*}
		\phi^{(n)} \rightarrow 0,
		&\quad \kcurl{}\phi^{(n)} \rightarrow v,
		\quad\text{in } \ltwo{\ddom}{\lambda_I}^3.
	\end{align*}
	Thus $v\in\kcurlZero{\ddom}{\lambda_I}$.
	Using the same approach, we can deduce that if $v=\bracs{0,0,f}^{\top}$ then $v\in\kcurlZero{\ddom}{\lambda_I}$ by considering the constant approximating sequence $\phi^{(n)} = \bracs{0,0,-x_1 f}^{\top}$.
	Application of a density argument extends these results from smooth functions $f$ to $\ltwo{\ddom}{\lambda_I}$ functions, and then as $\kcurlZero{\ddom}{\lambda_I}$ is a (closed) linear subspace of $\ltwo{\ddom}{\lambda_I}^3$ we have the desired result.
\end{proof}

Like with proposition \ref{prop:GradZeroParallelZhikov} we can provide some kind of geometric interpretation for this result, illustrated in figure \ref{fig:CurlZeroInterp}.
\begin{figure}[t]
	\centering
	\includegraphics[scale=0.75]{Diagram_CurlZeroPlane.pdf}
	\caption{\label{fig:CurlZeroInterp} An illustration of $\wavenumber$-curls of zero. The curl is the axis of rotation of a small body in the vector field, whilst the measure we are using can only see things that happen in the plane/segment it supports. As such, unless the axis of rotation is normal to the plane, the measure $\lambda_I$ cannot ``see" it, and so it is a curl of zero.}
\end{figure}
First recall that although we are working on segments in two dimensions, prior to taking a Fourier transform these represented planes that extended into the $x_3$-direction.
With a segment parallel to the $x_2$-axis, the plane that it represents has the $x_1$-direction as it's normal.
Meanwhile the curl (hence $\wavenumber$-curl post Fourier transform) of a vector field is the axis of rotation a small body placed in the field would undergo.
Because our measure can only ``see" things that happen in the segment/plane it supports, any rotation that does not take place in the plane is irrelevant.
As only the component of the axis of rotation normal to the plane will induce rotation entirely in the plane, the set of $\wavenumber$-curls of zero consists of all the remaining parts.
In this case, the $x_1$-direction is normal to the plane/segment, so no $\wavenumber$-curl of zero has non-zero $x_1$-component.
Correspondingly the $x_2$- and $x_3$-components always induce rotation is that out-of-plane, hence all possible options for these components form the set $\kcurlZero{\ddom}{\lambda_I}$.
This example is generalised in the following results. 
This explanation also captures why the example in section \ref{sec:CurlExamples} gave us the conclusion that all functions were curls of zero; in this case the supporting measure only sees the ``projection of the rotation" onto the segment, which is just a two-point set.
Even in 1D (that is, even for the singular measure on the segment) this corresponds to a set of zero measure, and so cannot be ``seen".
As a result, no rotations can be seen and everything is considered a curl of zero. \newline

\begin{lemma}[$\wavenumber$-Curls of Zero on a Segment] \label{lem:kCurlZeroAnySegment}
	Suppose $I$ is a segment in $\ddom$ with orthogonal co-ordinate system $y=\bracs{y_1,y_2}$ with $y_2$ parallel to $I$, with $\lambda_I$ being the singular measure that supports $I$. 
	Let $R\in\mathrm{SO}(2)$ be the change of co-ordinates $x=Ry$, where $x=\bracs{x_1,x_2}$ is the axis co-ordinate system.
	Finally, let $e_I$ be the unit vector directed along $I$ and $n_I$ the normal vector to $I$ (directed as $y_1$ is).
	Then
	\begin{align*}
		\kcurlZero{\ddom}{\lambda_I} &= 
		\clbracs{
		\begin{pmatrix} R & 0 \\ 0 & 1 \end{pmatrix}\begin{pmatrix} 0 \\ c_2 \\ c_3 \end{pmatrix} 
		\ \vert \ c_2,c_3\in\ltwo{\ddom}{\lambda_I}
		} \\
		&= \clbracs{
		\begin{pmatrix} c_1 \\ c_2 \\ c_3 \end{pmatrix}
		\ \vert \ \begin{pmatrix} c_1 \\ c_2 \end{pmatrix}\cdot n_I = 0, \ c_1,c_2,c_3\in\ltwo{\ddom}{\lambda_I}
		}.
	\end{align*}
\end{lemma}
\begin{proof}
	One can deduce the first set equality simply by employing the change of variables provided by $R$, lemma \ref{lem:CurlUnderRotation} and using similar arguments to proposition \ref{prop:RotationOfEdgeGradients}.
	The following set equalities are simply alternative ways of writing the RHS of the first line, given the hypothesis of the lemma.
\end{proof}

Having established the form of $\wavenumber$-curls of zero on individual segments, we now look to develop a characterisation for $\wavenumber$-curls of zero on a general graph $\graph$, as we did when considering gradients.
Our goal will be the following theorem, which bears much resemblance to corollary \ref{cory:CharacterisationGradientsZero}:
\begin{theorem}[Characterisation of $\kcurlZero{\ddom}{\ddmes}$] \label{thm:kCurlZeroGraphChar}
	Let $\graph=\bracs{V,E}$ be a finite graph embedded in $\ddom$ and $\ddmes$ the singular measure supported on the edges of $\graph$.
	Recall convention \ref{conv:LocalEdgeCoords}.
	Then we have that
	\begin{align*}
		\kcurlZero{\ddom}{\ddmes} &= 
		\clbracs{
		c\in\ltwo{\ddom}{\ddmes}^3 \ \vert \
		c\vert_{I_{jk}} = \begin{pmatrix} R_{jk} & 0 \\ 0 & 1 \end{pmatrix} \begin{pmatrix} 0 \\ c_2 \\ c_3 \end{pmatrix}, \ c_2,c_3\in\ltwo{\ddom}{\lambda_{jk}}^3, \ \forall I_{jk}\in E				
		}, \\
		&= \clbracs{
		\begin{pmatrix} c_1 \\ c_2 \\ c_3 \end{pmatrix} \in\ltwo{\ddom}{\ddmes}^3 \ \vert \
		\begin{pmatrix} c_1 \\ c_2 \end{pmatrix} \cdot n_{jk} = 0, \ \forall I_{jk}\in E	
		}, \\				
		&= \clbracs{ c\in\ltwo{\ddom}{\ddmes}^3 \ \vert \ c\in\kcurlZero{\ddom}{\lambda_{jk}}, \ \forall I_{jk}\in E}
	\end{align*}
\end{theorem}
For ease we set $B = \clbracs{ c\in\ltwo{\ddom}{\ddmes}^3 \ \vert \ c\in\kcurlZero{\ddom}{\lambda_{jk}}, \ \forall I_{jk}\in E}$ in the analysis that follows, however the reader should notice several parallels with the methodology of section \ref{sec:GradZeroGraphAnalysis}; specifically propositions \ref{prop:Grad0IncB} and \ref{prop:BIncGrad0}, as well as lemma \ref{lem:SegGradExtend}.
\begin{lemma} \label{lem:kCurlZeroInB}
	\begin{align*}
		\kcurlZero{\ddom}{\ddmes} &\subset B.
	\end{align*}
\end{lemma}
\begin{proof}
	Suppose $c\in\kcurlZero{\ddom}{\ddmes}$, and take an approximating sequence $\phi^{(n)}$.
	Then for any $I_{jk}\in E$ we have that
	\begin{align*}
		\integral{\ddom}{\abs{\phi^{(n)}}^2}{\lambda_{jk}} 
		&\leq \integral{\ddom}{\abs{\phi^{(n)}}^2}{\ddmes} \rightarrow 0 \toInfty{n}, \\
		\integral{\ddom}{\abs{\kcurl\phi^{(n)} - c}^2}{\lambda_{jk}}
		&\leq \integral{\ddom}{\abs{\kcurl{}\phi^{(n)} - c}^2}{\ddmes} \rightarrow 0 \toInfty{n}.
	\end{align*}
	Thus
	\begin{align*}
		\phi^{(n)}\rightarrow 0, &\quad \kcurl{}\phi^{(n)}\rightarrow c \text{ in } \ltwo{\ddom}{\lambda_{jk}} \ \forall I_{jk}\in E,
	\end{align*}
	and so $c\in B$.
\end{proof}

\begin{lemma}[Extension Lemma for $\wavenumber$-curls of Zero] \label{lem:kCurlZeroExtensionLemma}
	For $n\in\naturals$, and an edge $I_{jk}\in E$, let $I_{jk}^n$ be as in \eqref{eq:ShortenedIntervalDef}.
	Suppose we have some $c\in\ltwo{\ddom}{\ddmes}^3$ with $c=0$ on $\ddom\setminus I_{jk}^n$ and $c\cdot n_{jk} = 0$ on $I_{jk}^n$, where $n_{jk}$ is the unit normal to $I_{jk}$.
	Then
	\begin{align*}
		c\in\kcurlZero{\ddom}{\ddmes}.
	\end{align*}
\end{lemma}
\begin{proof}
	As $c\cdot n_{jk} = 0$ on $I_{jk}^n$ and $c=0$ on $\ddom\setminus I_{jk}^n$, we have that $c\in\kcurlZero{\ddom}{\lambda_{jk}}$.
	Hence we can find an approximating sequence $\phi^{(l)}$ for $c$.
	Let $\chi_{jk}^{n}\in\smooth{\ddom}$ be the function defined in definition \ref{def:ChiDef}, and consider the sequence of functions $\psi^{(l)} := \chi_{jk}^n\phi^{(l)}$.
	By construction we have that
	\begin{align*}
		\integral{\ddom}{\abs{\psi^{(l)}}^2}{\ddmes}
		&= \integral{I_{jk}}{\abs{\chi_{jk}^n\phi^{(l)}}^2}{\lambda_{jk}}
		\leq \integral{I_{jk}}{\abs{\phi^{(l)}}^2}{\lambda_{jk}}
		\rightarrow 0 \toInfty{l}.
	\end{align*}
	Because
	\begin{align*}
		\kcurl{}\psi^{(l)} &= \grad^{(0)}\chi_{jk}^n\wedge\phi^{(l)} + \chi_{jk}^n\kcurl{}\phi^{(l)},
	\end{align*}
	(recall $\grad^{(0)}$ is just $\kgrad{}$ with $\wavenumber=0$) we also have that
	\begin{align*}
		\integral{\ddom}{\abs{\kcurl{}\psi^{(l)} - c}^2}{\ddmes}
		&= \integral{I_{jk}}{\abs{\grad^{(0)}\chi_{jk}^n\wedge\phi^{(l)} + \chi_{jk}^n\kcurl{}\phi^{(l)} - c}^2}{\lambda_{jk}} \\
		&\leq 2\integral{I_{jk}}{\abs{\chi_{jk}^n\kcurl{}\phi^{(l)} - c}^2}{\lambda_{jk}} \\
		&\quad + 2\sup\abs{\grad^{(0)}\chi_{jk}^n}^2\integral{I_{jk}}{\abs{\phi^{(l)}}^2}{\lambda_{jk}}.
	\end{align*}
	We notice that $\sup\abs{\grad^{(0)}\chi_{jk}^n}^2 = \sup\abs{\grad\chi_{jk}^n}^2$, which is bounded by a constant that does not depend on $l$, and so the second term is converging to $0 \toInfty{l}$.
	As for the first term,
	\begin{align*}
		\integral{I_{jk}}{\abs{\chi_{jk}^n\kcurl{}\phi^{(l)} - c}^2}{\lambda_{jk}}
		&= \integral{I_{jk}\setminus I_{jk}^n}{\abs{\chi_{jk}^n\kcurl{}\phi^{(l)}}^2}{\lambda_{jk}} \\
		&\quad + \integral{I_{jk}^n}{\abs{\kcurl{}\phi^{(l)} - c}^2}{\lambda_{jk}} \\
		&\leq \integral{I_{jk}\setminus I_{jk}^n}{\abs{\kcurl{}\phi^{(l)}}^2}{\lambda_{jk}} \\
		&\quad + \integral{I_{jk}^n}{\abs{\kcurl{}\phi^{(l)} - c}^2}{\lambda_{jk}} \\
		&= \integral{I_{jk}\setminus I_{jk}^n}{\abs{\kcurl{}\phi^{(l)} - c}^2}{\lambda_{jk}} \\
		&\quad + \integral{I_{jk}^n}{\abs{\kcurl{}\phi^{(l)} - c}^2}{\lambda_{jk}} \\
		&= \integral{I_{jk}}{\abs{\kcurl{}\phi^{(l)} - c}^2}{\lambda_{jk}} \rightarrow 0 \toInfty{l}.
	\end{align*}
	Hence we have provided a sequence $\psi^{(l)}$ such that
	\begin{align*}
		\psi^{(l)} \rightarrow 0, &\quad \kcurl{}\psi^{(l)} \rightarrow c \text{ in } \ltwo{\ddom}{\ddmes}^3,
	\end{align*}
	so $c\in\kcurlZero{\ddom}{\ddmes}$.
\end{proof}

\begin{lemma} \label{lem:BInkCurlZero}
	\begin{align*}
		B &\subset \kcurlZero{\ddom}{\ddmes}.
	\end{align*}
\end{lemma}
\begin{proof}
	Take $c\in B$, and define a family of functions $c_n$ by
	\begin{align*}
		c_n &= \sum_{v_j\in V}\sum_{j\conLeft k}\eta_j^n \eta_k^n c\vert_{I_{jk}},
	\end{align*}
	where $\eta_j^n, \eta_k^n$ are as in definition \ref{def:etaDef}.
	For each pair $j\conLeft k$, $\eta_j^n \eta_k^n c\vert_{I_{jk}}$ satisfies the hypothesis of lemma \ref{lem:kCurlZeroExtensionLemma} so is an element of $\kcurlZero{\ddom}{\ddmes}$.
	As $\kcurlZero{\ddom}{\ddmes}$ is a closed linear subspace of $\ltwo{\ddom}{\ddmes}^3$, $c_n\in\kcurlZero{\ddom}{\ddmes} \ \forall n\in\naturals$.
	By lemma \ref{lem:etaConv}, we conclude that $c_n$ converges to $c$ in $\ltwo{\ddom}{\ddmes}$ and as $\kcurlZero{\ddom}{\ddmes}$ is closed, $c\in\kcurlZero{\ddom}{\ddmes}$ as required.
\end{proof}

This concludes our analysis of $\wavenumber$-curls of zero.
We shall build on this in the following subsection, where we look to characterise tangential $\kt$-curls.

\subsection{Tangential $\kt$-curls} \label{sec:ktCurlsTangential}
With the set $\kcurlZero{\ddom}{\ddmes}$ understood, we can now examine tangential $\kt$-curls in much the same way as for $\kt$-gradients.
Given the result of theorem \ref{thm:kCurlZeroGraphChar}, it will be sufficient for us to provide an edge-wise characterisation for tangential $\kt$-curls.
As has become standard practice, we begin with by considering a segment that is oriented parallel to one of the axes.
\begin{lemma}[Tangential $\kt$-curl on a Segment Parallel to the $x_2$-axis] \label{lem:ktTanCurlSegmentParallel}
	Let $I$ be a segment in the $\bracs{x_1,x_2}$-plane parallel to the $x_1$-axis, and let $\lambda_I$ be the singular measure supported on $I$.
	Write $\gradSob{\interval{I}}{t}$ for the ``classical" Sobolev space on the interval $\interval{I}$.
	Take $r:\interval{I}\rightarrow I$ as the change of variables map $r(t) = v_I + te_I$ where $v_I$ is the suitable endpoint of $I$ and $e_I$ is the unit vector along $I$ directed away from $v_I$.
	Suppose $u\in\ktcurlSob{I}{\lambda_I}$.
	Then $\widetilde{u}_3 := u_3 \circ r \in\gradSob{\interval{I}}{t}$ and with $u_3' := \widetilde{u}_3' \circ r^{-1}$,
	\begin{align*}
		\ktcurl{\lambda_I}u &= 
		\begin{pmatrix} u_3' + i\qm_1 u_3 - i\wavenumber u_2 \\ 0 \\ 0 \end{pmatrix},
	\end{align*}
\end{lemma}
\begin{proof}
	Write $\ktcurl{\lambda_I}u = \bracs{v_1, v_2, v_3}^{\top}$ for some functions $v_1,v_2,v_3$ to be determined.
	As we know that $\ktcurl{\lambda_I}u \perp \kcurlZero{\ddom}{\lambda_I}$, given theorem \ref{thm:kCurlZeroGraphChar} we observe that
	\begin{align*}
		0 = \integral{I}{v_2\overline{f}_2 + v_3\overline{f}_3}{\lambda_I} \quad\forall f_2,f_3\in\ltwo{\ddom}{\lambda_I}.
	\end{align*}
	Hence we conclude that $v_2=v_3=0$. \newline
	
	Now take an approximating sequence $\phi^{(l)}$ for $u$.
	In what follows we use an overhead tilde to denote composition with the change of variables map $r$.
	Then as $\phi^{(l)}\rightarrow u \toInfty{l}$,
	\begin{align*}
		\int_0^{\abs{I}}\abs{\widetilde{\phi}^{(l)}_j - \widetilde{u}_j}^2 \md t
		&= \integral{I}{\abs{\phi^{(l)}_j - u_j}^2}{\lambda_I}
		\rightarrow 0 \toInfty{l},
	\end{align*}
	for each $j\in\clbracs{1,2,3}$, and so $\widetilde{\phi}^{(l)}_j\rightarrow\widetilde{u}_j$ in $\ltwo{\interval{I}}{t}$.
	Because we also have that $\ktcurl{}\phi^{(l)}\rightarrow \ktcurl{\lambda_I}u \toInfty{l}$, we know that
	\begin{align*}
		\partial_2\phi^{(l)}_3 + i\qm_2\phi^{(l)}_3 - i\wavenumber\phi^{(l)}_2 \rightarrow v_1
	\end{align*}
	in $\ltwo{\ddom}{\lambda_I}$.
	Noting that 
	\begin{align*}
		\diff{\widetilde{\phi}^{(l)}_3}{t} = \partial_2\widetilde{\phi}^{(l)}_3,
	\end{align*}
	using the change of variables $r$ we can deduce that
	\begin{align*}
		\int_0^{\abs{I}}\abs{\bracs{\widetilde{\phi}^{(l)}_3}' + i\qm_2\widetilde{\phi}^{(l)}_3 - i\wavenumber\widetilde{\phi}^{(l)}_2 - \widetilde{v}_1}^2 \md t
		&= \integral{I}{\abs{\partial_2\phi^{(l)}_3 + i\qm_2\phi^{(l)}_3 - i\wavenumber\phi^{(l)}_2 - v_1}^2}{\lambda_I} \\
		&\rightarrow 0 \toInfty{l}.
	\end{align*}
	So we have the following convergences in $\ltwo{\interval{I}}{t}$;
	\begin{align*}
		\widetilde{\phi}^{(l)}_2 \rightarrow \widetilde{u}_2,
		&\quad \widetilde{\phi}^{(l)}_3 \rightarrow \widetilde{u}_3, \\
		\bracs{\widetilde{\phi}^{(l)}_3}' + i\qm_2\widetilde{\phi}^{(l)}_3 - i\wavenumber\widetilde{\phi}^{(l)}_2 \rightarrow \widetilde{v}_1. &
	\end{align*}
	Using the Algebra of Limits we find that
	\begin{align*}
		\widetilde{\phi}^{(l)}_3 \rightarrow \widetilde{u}_3, 
		&\quad \bracs{\widetilde{\phi}^{(l)}_3}' \rightarrow i\wavenumber\widetilde{u}^{(l)}_2 - i\qm_2\widetilde{u}^{(l)}_3 + \widetilde{v}_1.
	\end{align*}
	Hence we have that $\widetilde{u}_3\in\gradSob{\interval{I}}{t}$ and identify $i\wavenumber\widetilde{u}^{(l)}_2 - i\qm_2\widetilde{u}^{(l)}_3 + \widetilde{v}_1 = \widetilde{u}_3'$, the derivative in the $\gradSob{\interval{I}}{t}$ sense of $\widetilde{u}_3$.
	As the change of variables $r$ is invertible, we conclude that
	\begin{align*}
		\ktcurl{\lambda_I}u &= 
		\begin{pmatrix} u_3' + i\qm_2 u_3 - i\wavenumber u_2 \\ 0 \\ 0 \end{pmatrix},
	\end{align*}
	as we sought.
\end{proof}

Again we highlight that although the notation $u_3'$ is suggestive of derivative-like properties, it does not in fact convey any such things and is merely a useful shorthand notation.
We now present the form of the tangential $\kt$-curl on an arbitrary segment and for $\kt$-curls on a graph $\graph$.
\begin{lemma}[Tangential $\kt$-curl on a Segment] \label{ktTanCurlGeneralSegment}
	Let $I$ be a segment in $\ddom$ with local co-ordinate system $y=\bracs{y_1,y_2}$ with $y_2$ parallel to $I$, and let $\lambda_I$ be the singular measure supported on $I$.
	Suppose $R\in\mathrm{SO}(2)$ is the change of co-ordinates $x=Ry$ where $x=\bracs{x_1,x_2}$ is the axis co-ordinate system, and let $e_I$ be the unit vector directed along $I$ in the direction of $y_2$.
	Further suppose that $r:\interval{I}\rightarrow I$ is the change of variables map $r(t) = v_I + te_I$ for the suitable endpoint $v_I$ of $I$.
	Take $u\in\ktcurlSob{\ddom}{\lambda_I}$ and set $\widetilde{u} = u \circ r$, and
	\begin{align*}
		U &= R\begin{pmatrix} u_1 \\ u_2 \end{pmatrix}.
	\end{align*}
	Then we have that 
	\begin{align*}
		\widetilde{u}_3 &\in \ktgradSob{\interval{I}}{t}, \text{ and }\\
		\ktcurl{\lambda_I}u &= \begin{pmatrix} R & 0 \\ 0 & 1 \end{pmatrix} \begin{pmatrix} u_3' + i\bracs{R\qm}_2u_3 - i\wavenumber U_2 \\ 0 \\ 0 \end{pmatrix}
	\end{align*}
	with $u_3 = \widetilde{u}_3' \circ r^{-1}$.
\end{lemma}
\begin{proof}
	One uses the change of co-ordinates $R$ (use of lemma  \ref{lem:CurlUnderRotation} is what introduces the pre-multiplication by $R$ of $\qm$) so that the framework now fits with lemma \ref{lem:ktTanCurlSegmentParallel}, and having done this the result given above follows.
\end{proof}

\begin{theorem}[Tangential $\kt$-curls on Graphs] \label{thm:ktTanCurlsGraphs}
	Let $\graph=\bracs{V,E}$ be a graph embedded into $\ddom$ and $\ddmes$ the singular measure supported on the edges of $\graph$, and suppose $u\in\ktcurlSob{\ddom}{\ddmes}$.
	Recall convention \ref{conv:LocalEdgeCoords}.
	For each $I_{jk}\in E$, also define
	\begin{align*}
		U_{jk} := R_{jk}\begin{pmatrix} u_{1,jk} \\ u_{2,jk} \end{pmatrix}, \quad
		\widetilde{U}_{jk} := U_{jk} \circ r_{jk},
	\end{align*}
	Then we have that 
	\begin{align*}
		\widetilde{u}_{3,jk} := u_{3,jk}\circ r_{jk}\in\ktgradSob{\ddom}{\lambda_{jk}}, &\ \forall I_{jk}\in E, \\
		\bracs{\ktcurl{\ddmes}u}\vert_{I_{jk}} = \begin{pmatrix} R_{jk} & 0 \\ 0 & 1 \end{pmatrix} \begin{pmatrix} u_{3,jk}' + i\bracs{R_{jk}\qm}_2 u_{3,jk} - i\wavenumber U_{2,jk} \\ 0 \\ 0 \end{pmatrix}, 
		&\ \forall I_{jk}\in E,
	\end{align*}
	where $u_{3,jk}' = \widetilde{u}_{3,jk}' \circ r_{jk}^{-1}$.
\end{theorem}
\begin{proof}
	This is simply a repeated application of lemma \ref{lem:kCurlZeroAnySegment} on each of the edges $I_{jk}$, noting that $u\in\ktcurlSob{\ddom}{\ddmes}$ implies that $u\in\ktcurlSob{\ddom}{\lambda_{jk}}$ too.
\end{proof}

\subsection{Additional Properties of $\ktcurlSob{\ddom}{\ddmes}$} \label{sec:ktcurlSobExtraProperties}
Much like $\gradSob{\ddom}{\ddmes}$, there is some additional structure within the space $\ktcurlSob{\ddom}{\ddmes}$.
This is mainly in the form of continuity conditions at the vertices of $\graph$ in the third components of vector fields $u\in\ktcurlSob{\ddom}{\ddmes}$.
Given that we have effectively defined a notion of curl, it should be unsurprising to learn that this comes with the requirement of some level of continuity in our fields.
In fact we have the following proposition which immediately ties together $\ktgradSob{\ddom}{\ddmes}$ and the third component of the functions $u\in\ktcurlSob{\ddom}{\ddmes}$.

\begin{prop}[Continuity of 3\textsuperscript{rd} Components of $\ktcurlSob{\ddom}{\ddmes}$] \label{prop:Curl3rdComponentIFFInGradSpace}
It holds that
\begin{align*}
	\begin{pmatrix} 0 \\ 0 \\ u_3 \end{pmatrix}\in\ktcurlSob{\ddom}{\ddmes} &\Leftrightarrow 
	u_3\in\ktgradSob{\ddom}{\ddmes} \\
	&\Leftrightarrow	\begin{cases} (i) \ u_3\in\ktgradSob{\ddom}{\lambda_{jk}} \ \forall I_{jk} \\ (ii) \ u_3 \text{ is continuous at every } v_j\in V	\end{cases}
\end{align*}
for any $u_3\in\ktgradSob{\ddom}{\ddmes}$.
\end{prop}
\begin{proof}
	Note that theorem \ref{thm:ktGradSobChar} means it is sufficient to show the equivalence on the first line.
	Set $u = \bracs{0,0,u_3}^{\top}$.
	For the right directed implication, it is sufficient to note that if $u_3\in\ktgradSob{\ddom}{\ddmes}$ then we can find a smooth approximating sequence
	\begin{align*}
		\phi_n \lconv{\ltwo{\ddom}{\ddmes}} u_3, &\quad \ktgrad\phi_n \lconv{\ltwo{\ddom}{\ddmes}^3} \ktgrad_{\ddmes}u_3.
	\end{align*}
	This in turn implies that
	\begin{align*}
		\begin{pmatrix} 0 \\ 0 \\ \phi_n \end{pmatrix} \lconv{\ltwo{\ddom}{\ddmes}^3} u,
		&\quad \ktcurl{} \begin{pmatrix} 0 \\ 0 \\ \phi_n \end{pmatrix} \lconv{\ltwo{\ddom}{\ddmes}^3} \begin{pmatrix} \bracs{\ktgrad_{\ddmes}u_3}_2 \\ -\bracs{\ktgrad_{\ddmes}u_3}_1 \\ 0 \end{pmatrix},
	\end{align*}
	and hence $u\in\ktcurlSob{\ddom}{\ddmes}$. \newline
	
	Conversely, for the left-directed implication we can find a smooth approximating sequence $\psi^{(n)}$ such that
	\begin{align*}
		\phi^{(n)} \lconv{\ltwo{\ddom}{\ddmes}^3} u, 
		&\quad \ktcurl{}\psi^{(n)} \lconv{\ltwo{\ddom}{\ddmes}^3} \ktcurl{\ddmes}u.
	\end{align*}
	Then this implies that the sequences
	\begin{align*}
		\psi^{(n)}_3 \lconv{\ltwo{\ddom}{\ddmes}} u_3,
		&\quad \ktgrad\psi^{(n)}_3 \lconv{\ltwo{\ddom}{\ddmes}^3} \begin{pmatrix} -\bracs{\ktcurl{\ddmes}u}_2 \\ \bracs{\ktcurl{\ddmes}u}_1 \\ 0 \end{pmatrix},
	\end{align*}
	and hence that $u_3\in\ktgradSob{\ddom}{\ddmes}$.
\end{proof}
This provides us with continuity of the third component of elements of $\ktcurlSob{\ddom}{\ddmes}$ as it is a vector space (so $\bracs{u_1,u_2,u_3}^{\top}\in\ktcurlSob{\ddom}{\ddmes} \implies \bracs{0,0,u_3}^{\top}\in\ktcurlSob{\ddom}{\ddmes}$).
For the purposes of this report, this combined with the divergence-free condition (section \ref{sec:ktDivFree}) is all that we require for our analysis and the examples that we will consider in chapter \ref{ch:ExampleSystems}, because it provides us with enough vertex/boundary conditions in the equivalent quantum graph problem. \newline

We conjecture that there is more to be said on this topic, namely some further conditions exist that can be added to the right hand side of the implication to form a set of necessary and sufficient conditions.
The reason to believe that such conditions exist comes from the theory of ``classical" Sobolev spaces, namely that membership of $\curlSob{D}{\lambda_3}$ (for the 3D Lebesgue measure $\lambda_3$ and some domain $D\subset\reals^3$) and being divergence-free (in the Lebesgue sense) is sufficient for membership of $\gradSob{D}{\lambda_3}$.
Ideally we would provide this missing condition and state an analogue of theorem \ref{thm:ktGradSobChar} for the space $\ktcurlSob{\ddom}{\ddmes}$, which from a theoretical standpoint is a significant step in classifying the behaviour of this space.
Regrettably we do not yet have a candidate for such a condition, although we can hypothesise that these conditions (if they exist) likely involve some relation between the first and second components of the elements of $\ktcurlSob{\ddom}{\ddmes}$ on the approach to the vertices of $\graph$.
At present though, we can only state the following;
\begin{prop}[Necessary Conditions for Membership of $\ktcurlSob{\ddom}{\ddmes}$] \label{prop:NeccessaryConditionsForH1Curl}
	Suppose $u=\bracs{u_1,u_2,u_3}^{\top} \in\ktcurlSob{\ddom}{\ddmes}$.
	Then
	\begin{align*}
		u_3 &\in\ktgradSob{\ddom}{\ddmes}, \\
		\bracs{u_1,u_2,u_3} &\in\ktcurlSob{\ddom}{\lambda_{jk}} \quad \forall I_{jk}\in E
	\end{align*}
\end{prop}
In section \ref{sec:ktDivFree} we prove theorem \ref{thm:DivFreeWholeGraph}, which characterises divergence-free functions and provides another reason to believe that part of the description is missing from proposition \ref{prop:NeccessaryConditionsForH1Curl}.

\subsection{The Divergence-Free Condition} \label{sec:ktDivFree}
Having established characterisations for both $\kgradZero{\ddom}{\ddmes}$ (corollary \ref{cory:CharacterisationkGradientsZero}) and the tangential gradients (corollary \ref{cory:ktTangentialGradientForm}), we can now examine what it means for a function $u\in\ktcurlSob{\ddom}{\ddmes}$ to be divergence-free.
As usual we will begin by considering a single edge $I$ and extending the results to the whole of $\graph$ via the edge-wise nature of $\ddmes$.
\begin{lemma}[Divergence-Free Functions on a Segment Parallel to the $x_2$-axis] \label{lem:DivFreeParallelSegment}
	Let $I=[v_1,v_2]$ be a segment in $\ddom$ parallel to the $x_2$ axis, and $\lambda_I$ the singular measure supported on $I$.
	Let $e_I$ be the unit vector along $I$ directed from $v_1$ to $v_2$, and $r:\interval{I}\rightarrow I$ be the change of variables map $r(t) = v_1 + te_I$.
	Suppose that $u\in\ktcurlSob{\ddom}{\lambda_I}$.
	Then
	\begin{align*}
		u \text{ is divergence-free } &\quad\Leftrightarrow\quad
		\begin{cases}
		\mathrm{(i)} & u_1 = 0, \\
		\mathrm{(ii)} & \widetilde{u}_2 := u_2 \circ r \in \gradSob{\interval{I}}{t}, \\
		\mathrm{(iii)} & u_2' + i\qm_2 u_2 = -i\wavenumber u_3 \ \text{ on } I, \text{ where } u_2' = \widetilde{u}_2' \circ r^{-1}, \\
		\mathrm{(iv)} & u_2\bracs{v_1} = 0 = u_2\bracs{v_2}.
		\end{cases}
	\end{align*}
\end{lemma}
\begin{proof}
	For the right-directed implication, assume $u$ is divergence-free.
	The requirement that $u$ be orthogonal to all $\kt$-gradients of zero then implies that
	\begin{align*}
		 0 &= \integral{\ddom}{ u\cdot\overline{g} }{\lambda_I} \quad \forall g\in\kgradZero{\ddom}{\lambda_I}, \\
		 \implies 0 &= \integral{\ddom}{ u_1\overline{g_1} }{\lambda_I} \quad \forall g_1\in\ltwo{\ddom}{\lambda_I}.
	\end{align*}
	Thus $u_1 = 0$, giving condition (i).
	For the remaining conditions we will need to make use of the fact that as $u\in\ktcurlSob{\ddom}{\lambda_I}$ we can find a usual approximating sequence
	\begin{align*}
		\phi^{(n)} \lconv{\ltwo{\ddom}{\lambda_I}^3} u, 
		&\quad \ktcurl{}\phi^{(n)} \lconv{\ltwo{\ddom}{\lambda_I}^3} \ktcurl{\lambda_I}u.
	\end{align*}
	Then we consider the orthogonality of $u$ to tangential $\kt$-gradients.
	First for any $v\in\ktgradSob{\ddom}{\lambda_I}$ with support in the interior of $I$, we have that
	\begin{align*}
		0 &= \integral{\ddom}{ u\cdot\overline{\ktgrad_{\lambda_I}v} }{\lambda_I}
		= \lim_{n\rightarrow\infty}\integral{\ddom}{ \phi^{(n)}\cdot\overline{\ktgrad_{\lambda_I}v} }{\lambda_I} \\
		&= \lim_{n\rightarrow\infty}\integral{\ddom}{ \phi_2^{(n)}\bracs{\overline{v}' - i\qm_2\overline{v}} - i\wavenumber\phi_3^{(n)}\overline{v} }{\lambda_I} \\
		&= \lim_{n\rightarrow\infty}\int_0^{\abs{I}} -\overline{\widetilde{v}}\bracs{i\qm_2\widetilde{\phi}_2^{(n)} + i\wavenumber\widetilde{\phi}_3^{(n)}} + \widetilde{\phi}_2^{(n)}\overline{\widetilde{v}}' \ \md t \\
		&= \lim_{n\rightarrow\infty}\int_0^{\abs{I}} -\overline{\widetilde{v}}\bracs{i\qm_2\widetilde{\phi}_2^{(n)} + i\wavenumber\widetilde{\phi}_3^{(n)} + \bracs{\widetilde{\phi}_2^{(n)}}'} \ \md t \\
%		&= \int_0^{\abs{I}} -\overline{\widetilde{v}}\bracs{i\qm_2\widetilde{u}_2 + i\wavenumber\widetilde{u}_3 + \lim_{n\rightarrow\infty}\bracs{\widetilde{\phi}_2^{(n)}}'} \ \md t
	\end{align*}
	Where an overhead tilde is used to denote composition with the map $r$, for shorthand.
	As $v$ was arbitrary we must conclude that 
	\begin{align*}
		i\qm_2\widetilde{\phi}_2^{(n)} + i\wavenumber\widetilde{\phi}_3^{(n)} + \bracs{\widetilde{\phi}_2^{(n)}}' \lconv{\ltwo{\interval{I}}{t}} 0,
	\end{align*}
	but since
	\begin{align*}
		\widetilde{\phi}_2^{(n)} \lconv{\ltwo{\interval{I}}{t}} \widetilde{u}_2,
		&\quad \widetilde{\phi}_3^{(n)} \lconv{\ltwo{\interval{I}}{t}} \widetilde{u}_3,
	\end{align*}
	we have that $\bracs{\widetilde{\phi}_2^{(n)}}'$ converges in $\ltwo{\interval{I}}{t}$ too\footnote{Strictly speaking, we should first undo our change of variables $r$ to show that $\partial_2\phi_2^{(n)}$ converges in $\ltwo{\ddom}{\lambda_I}$ and then change variables via $r$ again to conclude that $\bracs{\widetilde{\phi}_2^{(n)}}'$ converges in $\ltwo{\interval{I}}{t}$.}.
	Thus $\widetilde{u}_2\in\gradSob{\interval{I}}{t}$, giving (ii).
	(iii) then follows immediately by noting that the final equality above now implies that
	\begin{align*}
		0 &= \widetilde{u}_2' + i\qm_2\widetilde{u}_2 + i\wavenumber\widetilde{u}_3
	\end{align*}
	holds on $\interval{I}$ and then ``undoing" the change of variables $r$.
	Finally we obtain (iv) by considering those $v\in\ktgradSob{\ddom}{\lambda_I}$ whose support contains precisely one of the endpoints $v_1$ or $v_2$; without loss of generality we illustrate the proof for the case of support containing $v_1$,
	Then
	\begin{align*}
		0 &= \integral{\ddom}{ u\cdot\overline{\ktgrad_{\lambda_I}v} }{\lambda_I} \\
		&= \int_0^{\abs{I}} -\overline{\widetilde{v}}\bracs{\widetilde{u}_2' + i\qm_2\widetilde{u}_2 + i\wavenumber\widetilde{u}_3} \ \md t + \sqbracs{\widetilde{u}_2\overline{\widetilde{v}}}_0^{\abs{I}} \\
		&= \widetilde{u}_2\bracs{0}\overline{\widetilde{v}}\bracs{0}
		= u_2\bracs{v_1}\overline{v}\bracs{v_1}.
	\end{align*}
	Then as $v$ was arbitrary we conclude that $u_2\bracs{v_1}=0$. \newline
	
	The left-directed implication follows by exploiting the ideas above and adapting the arguments to run in reverse.
	Clearly (i) will provide orthogonality of $u$ to any $g\in\kgradZero{\ddom}{\lambda_I}$.
	And given that (ii) ensures we have a weak derivative for $\widetilde{u}_2$, we are able to perform similar manipulations as in the previous argument to obtain
	\begin{align*}
		\integral{\ddom}{ u\cdot\overline{\ktgrad_{\lambda_I}v} }{\lambda_I}
		&= \integral{\ddom}{ -\overline{v}\bracs{u_2' + i\qm_2 u_2 + i\wavenumber u_3} }{\lambda_I} + \sqbracs{u_2\overline{v}}_{v_1}^{v_2}
		= 0,
	\end{align*}
	given (iii) and (iv).
\end{proof}
Note that condition (ii) is not sufficient to give us that $u_2\in\gradSob{\ddom}{\lambda_I}$, as we are always lacking information on the first derivative $\partial_1\phi_2^{(n)}$ and are not assured that it converges.
In the following proposition we generalise this result to any segment in the plane, which demonstrates that the fact the conditions (i)-(iv) in lemma \ref{lem:DivFreeParallelSegment} in general involve a specific combination of the functions $u_1$ and $u_2$.
The proof itself is almost identical, barring changes due to the orientation of the segment in the plane.

\begin{cory}[Divergence-Free Functions on a Segment] \label{cory:DivFreeGeneralSegment}
	Let $I=\sqbracs{v_1,v_2}$ be a segment in $\ddom$ with local co-ordinate system $y=\bracs{y_1,y_2}$ with $y_2$ parallel to $I$, and $\lambda_I$ the singular measure supported on $I$.
	Let $x=\bracs{x_1,x_2}$ be the axes co-ordinate system and $R\in\mathrm{SO}(2)$ such that $x=Ry$.
	Let $e_I$ be the unit vector along $I$ directed from $v_1$ to $v_2$, and $r:\interval{I}\rightarrow I$ be the change of variables map $r(t) = v_1 + te_I$.
	Suppose that $u\in\ktcurlSob{\ddom}{\lambda_I}$, and denote
	\begin{align*}
		U &= R\begin{pmatrix} u_1 \\ u_2 \end{pmatrix}.
	\end{align*}
	Then
	\begin{align*}
		u \text{ is divergence-free } &\quad\Leftrightarrow\quad
		\begin{cases}
		\mathrm{(i)} & U_1 = 0, \\
		\mathrm{(ii)} & \widetilde{U}_2 := U_2 \circ r \in \gradSob{\interval{I}}{t}, \\
		\mathrm{(iii)} & U_2' + i\bracs{R\qm}_2 U_2 = -i\wavenumber u_3 \ \text{ on } I, \text{ where } U_2' = \widetilde{U}_2' \circ r^{-1}, \\
		\mathrm{(iv)} & U_2\bracs{v_1} = 0 = U_2\bracs{v_2}.
		\end{cases}
	\end{align*}
\end{cory}
\begin{proof}
	This proof is essentially identical to that of lemma \ref{lem:DivFreeParallelSegment}, so we omit several technical details; and begin with the the right directed implication, so assume that $u$ is divergence-free.
	Then testing against $g\in\kgradZero{\ddom}{\lambda_I}$ gives that 
	\begin{align*}
		0 &= \integral{\ddom}{ u\cdot \overline{g}}{\lambda_I}
		= \integral{\ddom}{ \begin{pmatrix} u_1 \\ u_2 \\ u_3 \end{pmatrix} \cdot \begin{pmatrix} R^{\top} & 0 \\ 0 & 1 \end{pmatrix} \begin{pmatrix} \overline{g}_1 \\ 0 \\ 0 \end{pmatrix} }{\lambda_I} \\
		&= \integral{\ddom}{ R\begin{pmatrix} u_1 \\ u_2 \end{pmatrix} \cdot \begin{pmatrix} \overline{g}_1 \\ 0 \end{pmatrix} }{\lambda_I} \\
		\implies 0 &= \integral{\ddom}{ U_1 \overline{g}_1 }{\lambda_I} &\quad \forall g_1\in\ltwo{\ddom}{\lambda_I},
	\end{align*}
	and hence $U_1=0$.
	Note that the vector $U$ emerges because of the form for our gradients of zero on a rotated segment (corollary \ref{cory:kGradZeroArbitraryAngle}).
	We also see a similar situation arise when we consider orthogonality of $u$ to $v\in\ktgradSob{\ddom}{\lambda_I}$; but first we need to take one of our usual smooth approximating sequences
	\begin{align*}
	\phi^{(n)} \lconv{\ltwo{\ddom}{\lambda_I}^3} u, 
	&\quad \ktcurl{}\phi^{(n)} \lconv{\ltwo{\ddom}{\lambda_I}^3} \ktcurl{\lambda_I}u.
	\end{align*}
	and also define
	\begin{align*}
		\Phi^{(n)} &= R\begin{pmatrix} \phi_1^{(n)} \\ \phi_2^{(n)} \end{pmatrix}, 
	\end{align*}
	for ease.
	Then working from
	\begin{align*}
		0 &= \integral{\ddom}{ u\cdot\overline{\ktgrad_{\lambda_I}v} }{\lambda_I}
		= \integral{\ddom}{ -i\wavenumber\phi_3^{(n)}\overline{v} + \begin{pmatrix} \phi_1^{(n)} \\ \phi_2^{(n)} \end{pmatrix} \cdot R^{\top} \begin{pmatrix} 0 \\ \overline{v}' - i\bracs{R\qm}_2\overline{v} \end{pmatrix} }{\lambda_I}, \\
		\implies 0 &= \integral{\ddom}{ -i\wavenumber\phi_3^{(n)}\overline{v} + \Phi_2^{(n)}\bracs{\overline{v}' - i\bracs{R\qm}_2\overline{v}} }{\lambda_I},
	\end{align*}
	which is analogous to the expression we obtained in the proof of lemma \ref{lem:DivFreeParallelSegment}, and so we can perform the usual process of changing variables via $r$, integrating by parts and deducing that $\bracs{\Phi_2^{(n)}\circ r}'$ has the required convergence in $\ltwo{\interval{I}}{t}$, and thus $\widetilde{U}_2\in\gradSob{\interval{I}}{t}$ from which conditions (ii)-(iv) follow. \newline
	
	Having done this, the left-directed implication follows by reversing the steps of the argument, as it did in lemma \ref{lem:DivFreeParallelSegment}, completing the proof.
\end{proof}

Corollary \ref{cory:DivFreeGeneralSegment} essentially informs us that for a segment in the plane, the divergence-free condition places restrictions on the part of the field $u$ directed along and normal to the segment.
Namely the component of the field normal to the segment is required to be $0$, whilst the component parallel to the segment is required to have additional regularity and values at the ends of the segment.
We can use these results to obtain the following characterisation of $\kt$-divergence-free functions on a graph; note that the condition (iv) changes to reflect the fact that multiple edges may now meet at a vertex, and so we no longer simply have a Dirichlet condition at the ends of the segments that make up the graph.
\begin{theorem}[Divergence-Free Functions on a Graph] \label{thm:DivFreeWholeGraph}
	Let $\graph=\bracs{V,E}$ be a graph embedded into $\ddom$ and $\ddmes$ the singular measure supported on the edges of $\graph$, and suppose $u\in\ktcurlSob{\ddom}{\ddmes}$.
	Recall convention \ref{conv:LocalEdgeCoords}, and for each $I_{jk}\in E$ also define
	\begin{align*}
		U_{jk} := R_{jk}\begin{pmatrix} u_{1,jk} \\ u_{2,jk} \end{pmatrix}, \quad
		\widetilde{U}_{jk} := U_{jk} \circ r_{jk},
	\end{align*}
	Then we have that
	\begin{align*}
		u \text{ is divergence-free } &\quad\Leftrightarrow\quad
		\begin{cases}
		\mathrm{(i)} & U_{1,jk} = 0, \ \forall I_{jk}\in E\\
		\mathrm{(ii)} & \widetilde{U}_{2,jk} \in \gradSob{\interval{I_{jk}}}{t}, \ \forall I_{jk}\in E \\
		\mathrm{(iii)} & -\bracs{\diff{}{t} + i\bracs{R_{jk}\qm}_2}U_{2,jk} = i\wavenumber u_{3,jk} \text{ on } I_{jk}, \ \forall I_{jk}\in E\\
		\mathrm{(iv)} & \sum_{j\con l}U_{2,jl}\bracs{v_j} = 0, \ \forall v_j\in V,
		\end{cases}
	\end{align*}
	where we mean $\diff{U_{2,jk}}{t} = \widetilde{U}_{2,jk}' \circ r_{jk}^{-1}$.
\end{theorem}
\begin{proof}
	The proof of this theorem again borrows the ideas from lemma \ref{lem:DivFreeParallelSegment} and corollary \ref{cory:DivFreeGeneralSegment}.
	For the right directed implication, assume $u$ is divergence-free.
	Then considering the edge-wise characterisation of $\kgradZero{\ddom}{\ddmes}$ (corollary \ref{cory:CharacterisationkGradientsZero}) we obtain (i) by imposing orthogonality of $u$ against elements of this space.
	Likewise considering the edge-wise characterisation of $\kt$-tangential gradients (theorem \ref{thm:ktGradSobChar}); and considering $v$ with support that intersects precisely one $I_{jk}$, for each $I_{jk}\in E$, provides us with (ii) and (iii).
	Having deduced (ii) and (iii); (iv) follows by considering $\kt$-tangential-gradients whose support contains precisely one vertex $v_j$, splitting the integral into a sum over the edges of $\graph$, using the change of variables $r_{jk}$, integrating by parts, and employing (iii) to conclude that the boundary contributions (that is, the function values at the vertex $v_j$) must adhere to (iv). \newline
	
	For the left directed implication, one again follows these ideas in reverse and appeals to the characterisations in corollary \ref{cory:CharacterisationkGradientsZero} and theorem \ref{thm:ktGradSobChar}.
\end{proof}

\subsection{Summary}
Having introduced the operator $\ktgrad$ due to our interests in wave-guidance problems; we have now completed our analysis of $\ktgradSob{\ddom}{\ddmes}$, $\ktcurlSob{\ddom}{\ddmes}$ and the $\kt$-divergence-free condition.
Our understanding of these is necessary for us to be able to reduce variational problems posed with respect to measures to quantum graph problems (section \ref{sec:CurlReductionToQG}).
Of important note is that the underlying arguments for understanding $\ktgradSob{\ddom}{\ddmes}$ are essentially identical to those in sections \ref{sec:GradZeroGraphAnalysis}; \ref{sec:ScalarSystem}, and \ref{sec:VertexContinuity}, being suitably adapted for the 3D context.
The path we take to understanding $\ktcurlSob{\ddom}{\ddmes}$ is also directed in much the same way as the arguments for $\gradSob{\ddom}{\ddmes}$ in these sections, and we even deduce a link between the two spaces via proposition \ref{prop:Curl3rdComponentIFFInGradSpace}.
The divergence-free condition is also studied because we will ultimately be considering examples of wave propagation in electromagnetic contexts.
We will also need the additional information from this condition to correctly formulate a variational problem, and from which obtain a quantum graph problem, that admits a solution.

\section{Reduction to Quantum Graphs} \label{sec:CurlReductionToQG}
The focus of this section will be analogous to that of section \ref{sec:ReduceToQMProblem}.
Having made use of the nature of our singular structure and the analysis of section \ref{sec:ktOperator}, we aim to reduce a variational problem posed in the appropriate Sobolev space to a more accessible quantum graphs problem.
Because we are focusing on electromagnetic wave-guidance, we shall be studying the spectral ``curl-of-the-curl" equation (see section \ref{sec:VectorSobSpaces}) subject to the divergence-free condition.
Formally, we define
\begin{align*}
	\ktcurlSobDivFree{\ddom}{\ddmes} 
	&:= \clbracs{ u\in\ktcurlSob{\ddom}{\ddmes} \ \vert \ u \text{ is } \kt\text{-divergence free}}
\end{align*}
which is the subspace of $\ktcurlSob{\ddom}{\ddmes}$ that consists of all $\kt$-divergence-free functions that have $\kt$-curls.
We then concern ourselves with the spectral problem
\begin{align*}
	-\ktcurl{\ddmes}\bracs{\ktcurl{\ddmes}u} &= \omega^2 u, \quad u\in\ktcurlSobDivFree{\ddom}{\ddmes},
\end{align*}
which of course is shorthand for the variational problem
\begin{align*}
	\integral{\ddom}{\bracs{\ktcurl{\ddmes}u}\cdot\overline{\bracs{\ktcurl{}\phi}}}{\ddmes} 
	&= \omega^2\integral{\ddom}{u\cdot\overline{\phi}}{\ddmes}, \\
	\forall\phi\in\smooth{\ddom}^3, \quad u\in\ktcurlSobDivFree{\ddom}{\ddmes}. \labelthis\label{eq:CurlCurlEquationDivFree}
\end{align*}
The bulk of the work to get from this variational formulation to a quantum graphs problem has been done in the previous sections, and it is now a case of applying the knowledge we have of the objects in \eqref{eq:CurlCurlEquationDivFree}. \newline

Throughout this section we retain conventions \ref{conv:StraightEdges} and \ref{conv:LocalEdgeCoords}, whilst also adopting the following:
\begin{convention}[Rotated Components of Functions] \label{conv:RotatedComponents}
	When working with the edge-wise forms for functions and their $\kt$-curls (and divergences) in $\ktcurlSobDivFree{\ddom}{\ddmes}$, we will be making use of the functions $U_{jk}$ and $\Phi_{jk}$ defined by
	\begin{align*}
		U_{jk} &= R_{jk}\begin{pmatrix} u_{1,jk} \\ u_{2,jk} \end{pmatrix}, 
		&\quad
		\Phi_{jk} &= R_{jk}\begin{pmatrix} \phi_{1,jk} \\ \phi_{2,jk} \end{pmatrix},
	\end{align*}
	on each $I_{jk}$.
	In the calculations that follow we demonstrate that this results in a system involving $U_{1,jk}$, $U_{2,jk}$ and $u_{3,jk}$ only.
\end{convention}
\begin{convention}[Interpretation of Derivative-Like Notation] \label{conv:DerivLikeNotation}
	We use the notation $u_{3,jk}' = \widetilde{u}_{3,jk}' \circ r_{jk}^{-1}$, for the $\gradSob{\interval{I_{jk}}}{t}$-derivative $\widetilde{u}_{3,jk}'$ of $\widetilde{u}_{3,jk}$.
	We also adopt the following interpretation for the expression
	\begin{align*}
		\diff{u_{3,jk}}{t}\bracs{x} &= u_{3,jk}'\bracs{x} = \bracs{\widetilde{u}_{3,jk}' \circ r_{jk}^{-1}}\bracs{x}, \quad x\in I_{jk}.
	\end{align*}
\end{convention} 

We begin the process of deriving our equivalent quantum graph problem by recalling the form for the tangential $\kt$-curl given by theorem \ref{thm:ktTanCurlsGraphs}, and writing the integral in \eqref{eq:CurlCurlEquationDivFree} as a sum of integrals over the edges of $\graph$;
\begin{align*}
	\sum_{v_j\in V}\sum_{j\conLeft k}\integral{\ddom}{\bracs{\ktcurl{\lambda_{jk}}u}\cdot\overline{\bracs{\ktcurl{}\phi}}}{\lambda_{jk}} 
	&= \omega^2\sum_{v_j\in V}\sum_{j\conLeft k}\integral{\ddom}{u\cdot\overline{\phi}}{\lambda_{jk}}.
\end{align*}
Then as this must hold for all smooth functions $\phi\in\smooth{\ddom}^3$, it must also hold whenever the support of $\phi$ contains only (part of the interior of) a particular edge $I_{jk}$.
This would require
\begin{align*}
	\integral{\ddom}{\bracs{\ktcurl{\lambda_{jk}}u}\cdot\overline{\bracs{\ktcurl{}\phi}}}{\lambda_{jk}} 
	&= \omega^2\integral{\ddom}{u\cdot\overline{\phi}}{\lambda_{jk}},
\end{align*}
and given proposition \ref{prop:ktTangentialGradientSegment}, using our usual change of variables $r_{jk}$ and rotation matrix $R_{jk}$ for the edge $I_{jk}$, we can conclude that
\begin{align*}
	\int_0^{\abs{I_{jk}}} \bracs{\widetilde{u}_{3,jk}' - i\wavenumber\widetilde{U}_{2,jk} + i\bracs{R_{jk}\qm}_2\widetilde{u}_{3,jk} } \overline{\bracs{\widetilde{\phi}_3' - i\wavenumber\widetilde{\Phi}_2 + i\bracs{R_{jk}\qm}_2\widetilde{\phi}_3 }} \ \md t \\
	= \omega^2 \int_0^{\abs{I_{jk}}} \widetilde{U}_{1,jk}\overline{\widetilde{\Phi}}_1 + \widetilde{U}_{2,jk}\overline{\widetilde{\Phi}}_2 + \widetilde{u}_{3,jk}\overline{\widetilde{\phi}}_3 \ \md t, \labelthis\label{eq:LongReductionEquation}
\end{align*}
since $u_1\phi_1 + u_2\phi_2 = U_{jk} \cdot \Phi_{jk}$ on $I_{jk}$.
Given that this equation holds when we take pairs of the components $\Phi_1,\Phi_2,\phi_3$ to be zero, we have that the following all hold for all smooth $\Phi_1, \Phi_2, \phi_3$;
\begin{subequations} \label{eq:SplitVariationalEdgeEqns}
	\begin{align}
		0 &= \omega^2 \int_0^{\abs{I_{jk}}} \widetilde{U}_{1,jk}\overline{\widetilde{\Phi}}_1 \ \md t, \\
		\int_0^{\abs{I_{jk}}} \wavenumber\overline{\widetilde{\Phi}}_2\bracs{i\widetilde{u}_{3,jk}' + \wavenumber\widetilde{U}_{2,jk} - \bracs{R_{jk}\qm}_2\widetilde{u}_{3,jk}} \ \md t
		&= \omega^2 \int_0^{\abs{I_{jk}}} \widetilde{U}_{2,jk}\overline{\widetilde{\Phi}}_2 \ \md t, \\
		\begin{split}
			\int_0^{\abs{I_{jk}}} -\bracs{R_{jk}\qm}_2\overline{\widetilde{\phi}}_3\bracs{i\widetilde{u}_{3,jk}' + \wavenumber\widetilde{U}_{2,jk} - \bracs{R_{jk}\qm}_2\widetilde{u}_{3,jk}} \quad & \\
			+ \overline{\widetilde{\phi}}_3'\bracs{i\widetilde{u}_{3,jk}' - i\wavenumber\widetilde{U}_{2,jk} + i\bracs{R_{jk}\qm}_2\widetilde{u}_{3,jk}} \ \md t
			&= \omega^2 \int_0^{\abs{I_{jk}}} \widetilde{u}_{3,jk}\overline{\widetilde{\phi}}_3 \ \md t.
		\end{split}
	\end{align}
\end{subequations}
Assuming sufficient regularity in $\widetilde{u}_3$ and $\widetilde{U}_2$, we can integrate by parts in the above expressions and obtain three equations;
\begin{align*}
	\widetilde{U}_{1,jk} &= 0, \\
	i\wavenumber\bracs{\diff{}{t} + i\bracs{R_{jk}\qm}_2}\widetilde{u}_{3,jk} + \wavenumber^2\widetilde{U}_{2,jk} &= \omega^2 \widetilde{U}_{2,jk}, \\
	-\bracs{\diff{}{t} + i\bracs{R_{jk}\qm}_2}^2\widetilde{u}_{3,jk} + i\wavenumber\bracs{\diff{}{t} + i\bracs{R_{jk}\qm}_2}\widetilde{U}_{2,jk} &= \omega^2\widetilde{u}_{3,jk}.
\end{align*}
Notice that the first such equation is simply one of the conditions that $u$ be divergence free (see theorem \ref{thm:DivFreeWholeGraph}).
The other two equations provide us with a system of ODEs on intervals, and we have two such equations for each edge $I_{jk}$ .
Alone these are not enough to determine eigenpairs $\bracs{\omega^2, u}$, however we can obtain more information from \eqref{eq:LongReductionEquation} by now considering the cases when our test functions $\phi$ have support containing at most one vertex, say the vertex $v_j$.
In such a case we can return to \eqref{eq:LongReductionEquation} and integrate by parts, obtaining 
\begin{align*}
	0 &= \sum_{j\con k} \sqbracs{ \bracs{ \bracs{\diff{}{t} + i\bracs{R_{jk}\qm}_2}\widetilde{u}_{3,jk} - i\wavenumber\widetilde{U}_{2,jk} }\overline{\widetilde{\phi}_3} }_0^{\abs{I_{jk}}}.
\end{align*}
This arises because after integrating by parts, the resulting integral term is identically zero due to \eqref{eq:SplitVariationalEdgeEqns}.
As the support of $\phi$ (hence $\phi_3$) contains only the vertex $v_j$, and $\phi$ is smooth so continuous at $v_j$, we can write this equation as
\begin{align*}
	0 &= \sum_{j\con k} \bracs{\diff{}{t} + i\bracs{R_{jk}\qm}_2}u_{3,jk}\bracs{v_j} - i\wavenumber U_{2,jk}\bracs{v_j}.
\end{align*}
That is, for each vertex $v_j$ we obtain a boundary condition forcing the sum of the values of $u_3$, $u_3'$, and $U_2$ at $v_j$ to be zero. \newline

From the variational formulation, we have obtained two differential equations and one vertex/boundary condition, per vertex.
This is still not enough information to obtain eigenpairs $\bracs{\omega^2, u}$, so we require the additional information from the divergence-free condition and the structure of $\ktcurlSob{\ddom}{\ddmes}$.
Because we are working in $\ktcurlSobDivFree{\ddom}{\ddmes}$, which is a subspace of $\ktcurlSob{\ddom}{\ddmes}$, we know that the component $u_3$ of $u$ is continuous across the vertices of the graph by proposition \ref{prop:Curl3rdComponentIFFInGradSpace} and so obtain a further $n-1$ boundary conditions for each vertex of degree $n\geq2$.
The $\ktgrad$-divergence-free condition also provides us with a further differential equation and set of boundary conditions, as in theorem \ref{thm:DivFreeWholeGraph}.
Combining these, we arrive at the following system of equations on each interval $\interval{I_{jk}}$,
\begin{subequations} \label{eq:CurlEdgeEquations}
	\begin{align}
		\widetilde{U}_{1,jk} &= 0, \\
		i\wavenumber\bracs{\diff{}{t} + i\bracs{R_{jk}\qm}_2}\widetilde{u}_{3,jk} + \wavenumber^2\widetilde{U}_{2,jk} &= \omega^2 \widetilde{U}_{2,jk}, \label{eq:CurlEdgeEquations1} \\
		-\bracs{\diff{}{t} + i\bracs{R_{jk}\qm}_2}^2\widetilde{u}_{3,jk} + i\wavenumber\bracs{\diff{}{t} + i\bracs{R_{jk}\qm}_2}\widetilde{U}_{2,jk} &= \omega^2\widetilde{u}_{3,jk}, \label{eq:CurlEdgeEquations2} \\
		-\bracs{\diff{}{t} + i\bracs{R_{jk}\qm}_2}\widetilde{U}_{2,jk} &= i\wavenumber\widetilde{u}_{3,jk}. \label{eq:CurlEdgeEquations3}
	\end{align}
\end{subequations}
These are complimented by the boundary conditions
\begin{subequations} \label{eq:CurlVertexConditions}
	\begin{align}
		u_3 &\text{ is continuous at each } v_j\in V, \label{eq:CurlVertexConditionsCty}\\
		0 &= \sum_{j\con k} \bracs{\diff{}{t} + i\bracs{R_{jk}\qm}_2}u_{3,jk}\bracs{v_j} - i\wavenumber U_{2,jk}\bracs{v_j}, \quad\forall v_j\in V \label{eq:CurlVertexConditionsSumDeriv}\\
		0 &= \sum_{j\con k}U_{2,jk}\bracs{v_j}, \ \forall v_j\in V. \label{eq:CurlVertexConditionsSum2}
	\end{align}
\end{subequations}
Noting that any functions that satisfy two of \eqref{eq:CurlEdgeEquations1}-\eqref{eq:CurlEdgeEquations3} necessarily satisfy the remaining one, \eqref{eq:CurlEdgeEquations}-\eqref{eq:CurlVertexConditions} provide us with enough equations and boundary conditions to solve for the eigenpairs $\bracs{\omega^2, u}$.
Furthermore \eqref{eq:CurlEdgeEquations}-\eqref{eq:CurlVertexConditions} defines a quantum graph problem on the quantum graph associated with $\graph$, which is equivalent to our original variational problem.
In chapter \ref{ch:ExampleSystems} we will explore these equations on different geometries (underlying graphs), attempting to determine the eigenvalues $\omega^2$ and determine whether band-gap structure is exhibited.
To conclude this chapter, we provide a summary in section \ref{sec:CurlSummary}.

\section{Chapter Summary} \label{sec:CurlSummary}
The theory presented in this chapter serves as justification and motivation for the problems that we will be considering in subsequent chapters and work.
We provide a construction of the Sobolev spaces $\curlSob{D}{\nu}$ for a domain $D$ and Borel measure $\nu$, which comes in much the same ilk as that in chapter \ref{ch:ScalarEqns}.
Some examples of the implications of this construction are discussed in section \ref{sec:CurlExamples}, however we do not develop the general theory any further.
Instead the focus of section \ref{sec:ktOperator} has been solely on the domains we will be using in our wave-propagation problems (section \ref{sec:OurPhysicalSetup}), and so we move onto the analysis of $\kt$-curls, $\kt$-gradients, and the $\kt$-divergence-free condition. \newline

Construction and analysis of $\ktcurlSob{\ddom}{\ddmes}$ and $\ktcurlSobDivFree{\ddom}{\ddmes}$ enables us to make sense of variational problems such as \eqref{eq:CurlCurlEquationDivFree} and the analysis of section \ref{sec:ktOperator} provides us with a means to reformulate this problem as a more familiar and accessible system of differential equations.
The operator $\ktgrad$ is our primary concern because it appears in our wave-propagation problems after taking a Fourier transform in the direction of wave-propagation (introducing the variable $\wavenumber$ and removing the $x_3$-dependence) and a Gelfand transform in the periodic cross-section.
This then leaves us with a family of problems parametrised by $\qm$ on the unit graph $\graph$ (or unit cell $\ddom$), posed with respect to the singular measure $\ddmes$.
We end up reaching a characterisation for each of the spaces $\ktgradSob{\ddom}{\ddmes}$, $\ktcurlSob{\ddom}{\ddmes}$, and $\ktcurlSobDivFree{\ddom}{\ddmes}$ in terms of the behaviour on each edge of our graph $\graph$.
This decomposition into edge-wise behaviour (plus continuity conditions at the vertices) is expected for $\ktgradSob{\ddom}{\ddmes}$ due to our analysis in chapter \ref{ch:ScalarEqns}.
Using this as motivation, we also follow a similar vein of investigation to deduce that elements of $\ktcurlSob{\ddom}{\ddmes}$ also display this edge-wise behaviour (section \ref{sec:ktOperator}) and some additional matching conditions (section \ref{sec:ktcurlSobExtraProperties}).
In particular we establish that any element $u\in\ktcurlSob{\ddom}{\ddmes}$ has a third component that is an element of $\ktgradSob{\ddom}{\ddmes}$, and that a converse also holds.
Due to the fact that the $\ktgrad$-divergence free condition ties to $\ktgradSob{\ddom}{\ddmes}$ and $\kgradZero{\ddom}{\ddmes}$, it is unsurprising to learn we have an edge-wise characterisation (plus some matching conditions) arising from this condition too.
These conditions together provide us with the tools to reformulate problems such as \eqref{eq:CurlCurlEquationDivFree} into quantum graph problems as in \eqref{eq:CurlEdgeEquations}-\eqref{eq:CurlVertexConditions}, which are more amiable to analytic (and numeric) than the variational formulations. \newline

In chapter \ref{ch:ExampleSystems} we will tie together the work of this chapter and chapter \ref{ch:QuantumGraphs} and present some example wave-guidance problems, solved using the approaches described in those chapters.
In this report we will not be developing the theory of the spaces $\ktcurlSob{\ddom}{\ddmes}$, $\ktgradSob{\ddom}{\ddmes}$, or $\ktcurlSobDivFree{\ddom}{\ddmes}$ any further; because for our purposes we now have all the requirements we need to justify and solve the class of problems that we wish to consider.
We have however mentioned that there are some theoretical gaps that should be revisited in the future (section \ref{sec:ktcurlSobExtraProperties}).
Section \ref{sec:VariationalProblemLitReview} directs the interested reader to a few further works, notably on homogenisation and norm-resolvent estimates for vector-valued variational problems.