\chapter{Conclusion} \label{ch:Conclusion}
The conclusion chapter is great fun, and is the last thing that I'll write (even after the introduction)!
Split into 3 sections, one to summarise the existing literature, theory, and the motivation of the project (chapters \ref{Introduction}, \ref{ch:QuantumGraphs}, \ref{ch:ScalarEqns}), one for the work that has been done (\ref{ch:VectorEqns}, \ref{ch:ExampleSystems}) and a further section for the future direction and developments that should be explored.

\section{Summary of Theory and Motivation}
Mainly a summary of chapters \ref{Introduction}, \ref{ch:QuantumGraphs}, \ref{ch:ScalarEqns}, although we should lean more heavily on the physical interpretation stuff.
Important points:
\begin{itemize}
	\item How this represents a physical system
	\item Why we need to treat a variational approach
	\item How do we tie QG to variational-formulation
\end{itemize}

\section{Summary of Work and Examples}
A summary of chapters \ref{ch:VectorEqns}, \ref{ch:ExampleSystems}, leaning slightly towards the examples perhaps.
Important points:
\begin{itemize}
	\item Have developed the theory for treating vector equations
	\item Have explored some example systems
	\item Have encountered various problems numerically and analytically which need to be investigated
\end{itemize}

\section{Further Developments}
``We don't have to tie up loose ends so long as we organise them neatly."
\begin{itemize}
	\item There are still a few open questions from the introduction and linking process surrounding the thick vertex problem
	\item Solving the $M$-matrix problems is hard!
	\item Physical interpretation - Dielectric inclusions (more complex measures YAY)? Hexagonal unit cells?
\end{itemize}
