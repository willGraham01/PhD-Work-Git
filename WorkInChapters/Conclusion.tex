\chapter{Conclusion} \label{ch:Conclusion}
In this chapter we review the content of this report and speculate on the future direction of research.
Section \ref{sec:ConcTheory} will provide an overview of the existing theory that we have utilised, the motivation for our project and restate our research objectives.
We will then summarise the work we have done to build on this theory and towards these objectives in section \ref{sec:ConcWork}.
Finally, in section \ref{sec:ConcFuture} we shall discuss some of the loose ends or open questions that have been raised by our research or not yet addressed, and provide some direction for future work.

\section{Summary of Motivation, Research Objectives, and Existing Theory} \label{sec:ConcTheory}
Our research is motivated by applications to PCFs (section \ref{sec:ProjectMotivation}), specifically in investigating the nature how spectral band-gaps emerge due as a result of the geometries of the fibres.
Although our research centres on singular-structure problems as a starting point (section \ref{sec:OurPhysicalSetup}), there is a link through quantum graph problems back to familiar thin-structure problems that are commonly used model PCFs (section \ref{sec:GraphLitReview}).
This amounts to us being able to view our singular-structure problems in an intuitive manner - as the limit of thin-structure problems as the thickness of the structures tends to zero, although the manner in which this thickness tends to zero also influences the problems that we should be considering (section \ref{sec:GraphLitReview}).
As such, the research goals that we set out to achieve were:
\begin{enumerate}
	\item To demonstrate that the singular-structure problems we consider give rise to equivalent quantum graph problems, in turn linking our singular-structure problems to formal ``limits" of thin-structure problems and hence PCF models.
	\item To study singular-structure problems that can be seen as approximations to PCFs; deriving the equivalent quantum graph problems and providing insight into how the geometry of the fibre cross section influences the spectral band-gaps of the fibre.
	\item To propose numerical approaches to determining these band-gaps in the event that an analytic approach proves unfeasible.
\end{enumerate}

As we discussed in sections \ref{sec:VariationalProblemLitReview}, choosing to study singular-structure problems required us to rethink the concepts of gradient, curl and divergence.
This bought us to the theory of chapter \ref{ch:ScalarEqns}, in which we presented a framework for posing variational problems with respect to Borel measures and reviewed the existing theory on the matter.
We also provided a geometric insight into what the concept of gradient meant in the context of our singular-structure problems, and demonstrated how we could obtain an equivalent quantum graph problem from a variational problem.
These arguments formed the basis of our understanding and direction for our work in chapter \ref{ch:VectorEqns}. \newline

Quantum graph problems are comparatively well studied, there even being a comprehensive introductory text on the subject and a recent spike in research interest (section \ref{sec:GraphLitReview}).
As such we simply needed to borrow the relevant concepts and tools from the existing works in the area for our own purposes, which we did in chapter \ref{ch:QuantumGraphs}.
Of particular importance was the M-matrix and it's utility in solving spectral problems on quantum graphs; and we briefly touched on how the M-matrix opens these spectral problems up to numerical schemes in chapter \ref{ch:ExampleSystems}, a topic which we revisit in section \ref{sec:ConcFuture}.
We also use quantum graphs as the link between our singular-structure problems and thin-structure problems that describe PCFs.

\section{Summary of Work and Examples} \label{sec:ConcWork}
The work of chapter \ref{ch:VectorEqns} see us build on the existing work on variational problems by constructing the spaces $\ktgradSob{\ddom}{\ddmes}$, $\ktcurlSob{\ddom}{\ddmes}$ and $\ktcurlSobDivFree{\ddom}{\ddmes}$ and analysing the operator $\ktgrad$.
We provide an interpretation for the $\ddmes$-curl of a vector field (section \ref{sec:CurlExamples}) and also prove various properties about the elements of the aforementioned spaces.
In particular we deduce the form of the tangential $\kt$-curl and $\kt$-gradient, and characterise what it means to be divergence-free.
We also prove that the space $\ktcurlSob{\ddom}{\ddmes}$ has some inherent structural properties that are not obvious from it's construction (section \ref{sec:ktcurlSobExtraProperties}).
This analysis allows us to consider the singular-structure analogue of the ``curl-of-the-curl" equation, written in \eqref{eq:CurlCurlEquationDivFree} and determine the equivalent quantum graph problem in section \ref{sec:CurlReductionToQG}, forming the basis for our examples in chapter \ref{ch:ExampleSystems}.
This theory is essential if we are to address the first of our research objectives, as without knowledge of the form of objects like $\ktcurl{\ddmes}u$, we have no hope of obtaining an equivalent quantum graph problem from our singular-structure problems.
The fact that we can write problems like \eqref{eq:CurlCurlEquationDivFree} in the way they are presented is also satisfying in an intuitive sense; the ``equations" that we are studying have the same form as those which we would consider in the thin-structure setting, only now we have a different understanding of gradients (and curls, and divergences).
Our development of this theory will likely prove valuable if we are to move on from the ``curl-of-the-curl" equation to a full Maxwell problem involving coupled $\mathbf{E}$ and $\mathbf{H}$ fields, which we revisit in section \ref{sec:ConcFuture}. \newline

Having derived the equivalent quantum graph problem from our singular-structure problems, we spend chapter \ref{ch:ExampleSystems} looking over some examples and addressing the second and third objectives.
Our examples demonstrate how to construct the M-matrix, and how it can be used either analytically or numerically to solve spectral problems and hence reveal insights about band-gaps.
We take a mixture of numerical and analytic approaches in these examples, however stop short of a fully-fledged numerical scheme that begins from the M-matrix itself.
This is discussed in section \ref{sec:NumericalMethodsDiscussion} however, and will be revisited in section \ref{sec:ConcFuture}, which follows.
The examples also illustrate how it is possible to use the geometry of the cross-sectional structure to open band-gaps in the spectrum (section \ref{sec:ExampleGeneralLengths}).
However our investigation is limited to a single, simplified case and analytic progress proves hard, again highlighting that, at least for practical purposes, a numerical scheme that focuses on a physically relevant part of the spectrum may be more useful and applicable.
We provide a final example in section \ref{sec:ExampleThickVertex} that retains the geometry of the example in section \ref{sec:ExampleCrossInPlane}, but with a non-zero coupling constant at the central vertex.
The corresponding singular-structure problem corresponds to a different scaling limit (section \ref{sec:GraphLitReview}) than the previous examples, and in this case we demonstrate that band-gaps are opened simply by the presence of this coupling constant.
This in turn would suggest that thin-structures that adhere to this scaling between ``edge"- and ``vertex"-regions are more likely to give rise to band-gaps, however more work needs to be done beyond this example.
We make this suggestion because these examples demonstrate that the addition of a non-zero coupling constant can lead to the opening of band-gaps, however this may again be an artefact of the geometry of the problem rather than a general principle.
Another thing to be noted is that the M-matrix can be recycled from the first example, and the only difference in our solution method being that we consider a slightly different eigenvalue problem.
This is potentially useful for any numerical schemes - we only need construct (a function that evaluates) the M-matrix once for a given geometry (and set of governing equations).
If in addition we can produce results like proposition \ref{prop:M-MatrixEntries} for each quantum graph problem, there is the potential to further cut the complexity of such numerical constructions. \newline

Whilst our examples help us explore the second and third objectives, and do provide us with some intuition about what to expect, they do not provide us with any general insights yet.
This, alongside some of the considerations for a numerical scheme, are discussed in section \ref{sec:ConcFuture}.

\section{Further Developments} \label{sec:ConcFuture}
The work that has been carried out thus far makes progress towards the research objectives that were set out in section \ref{sec:ReportOverview}, but stops short of providing definitive answers in places.
In this section we examine some of these loose ends, and the direction of future work that could be undertaken to address them.
We cover issues surrounding a numerical scheme for solving our singular-structure problems in section \ref{sec:ConcFutureNumerical}; how we might look to qualify the dependence of the graph geometry on the spectrum of our problems in section \ref{sec:ConcFutureGeometry}, and discuss how we might make progress onto modelling electromagnetic wave-guidance through Maxwell's equations in section \ref{sec:ConcFutureMaxwell}.
Once we have explored these issues, we will conclude with section \ref{sec:ConcClosingRemarks}.

\subsection{Considerations for Numerical Schemes} \label{sec:ConcFutureNumerical}
In section \ref{sec:NumericalMethodsDiscussion} we discussed the possibility of using the M-matrix to explore the spectrum of quantum graph (hence our singular-structure) problems numerically.
Here we review what was said and elaborate on how such an approach might be developed, tested and analysed.
To make the discussion as general as possible; in this section we assume that we have some family of quantum graph problems $\mathcal{P}_{\qm}$, with spectral parameter $\lambda$ and spectra $\sigma\bracs{\mathcal{P}_{\qm}}$, each having an M-matrix $M_{\qm}\bracs{\lambda}$.
This family $\mathcal{P}_{\qm}$ is the result of taking a Gelfand transform of a periodic quantum graph problem $\mathcal{P}$ (with spectrum $\sigma\bracs{\mathcal{P}}$), that is equivalent to some singular-structure problem that we are concerned with.
Here we discuss a numerical scheme that is capable of being told the problems $\mathcal{P}_{\qm}$ and producing an approximation to the spectrum of $\mathcal{P}$, whose outline is along the lines of the following;
\begin{enumerate}
	\item Consider the spectral problem $\mathcal{P}_{\qm}$ as a generalised eigenvalue problem
	\begin{align*}
		M_{\qm}\bracs{\lambda} v &= 0,
	\end{align*}
	involving the M-matrix.
	\item Determine a method for constructing $M_{\qm}\bracs{\lambda}$.
	\item Solve the generalised eigenvalue problem $\mathcal{P}_{\qm}$ for $\sigma\bracs{\mathcal{P}_{\qm}}$, and hence construct an approximation to the spectrum of $\mathcal{P}$.
\end{enumerate}
Broadly speaking; the issues surround how to construct the M-matrix efficiently and quickly, and the optimal method of determining the spectrum of $\mathcal{P}_{\qm}$ and hence $\mathcal{P}$.
We discuss each of these below, along with how they might be addressed. \newline

\subsubsection{Construction of the M-Matrix} \label{sec:ConcFutureConstructM}
Any numerical scheme will require a method for constructing the M-matrix, because the solver for the generalised eigenvalue problem will need this ability.
Naively we can construct the M-matrix simply by following the constructive proof of proposition \ref{prop:M-MatrixEntries} numerically, for each value of $\lambda$ that we are required to evaluate $M_{\qm}$ at.
This involves solving each edge-ODE in $\mathcal{P}_{\qm}$ numerically, and once for each column of the M-matrix (although this is a large overestimate and can be reduced - see section \ref{sec:NumericalMethodsDiscussion}), reading off the approximate Neumann data for the edge solution, and then summing the appropriate combination of derivatives at the vertices.
Needless to say this will be an expensive process for a large (in the sense of number of edges) graph, given that it needs to be done each time the M-matrix needs to be evaluated.
Having said this we should also note that this method is relatively simple to program, and provided there was sufficient care in solving the edge-ODEs of $\mathcal{P}_{\qm}$ would provide access to the M-matrix. \newline

We can make constructing the M-matrix cheaper (in computational terms) by employing one of the tactics in chapter \ref{ch:ExampleSystems} - working analytically to a suitable point and then proceeding numerically when the expressions become too complex.
The obvious candidate for a stopping point for each $\mathcal{P}_{\qm}$ would be the analogue of proposition \ref{prop:M-MatrixEntries}, as this bypasses the need to solve each edge-ODE every time the M-matrix needs to be evaluated (and even provides the M-matrix as a function of $\lambda=\omega^2$ and the quasi-momentum $\qm$).
In fact this result reduces the construction of the M-matrix to a case of looking up geometric properties of the underlying graph and evaluating trigonometric functions.
Of course the exchange we make for this simpler construction of $M_{\qm}$ is that we loose generality in our numerical scheme; proposition \ref{prop:M-MatrixEntries} only holds for the specific set of equations we chose to examine in chapter \ref{ch:ExampleSystems}, and so we would have to prove an analogue of proposition \ref{prop:M-MatrixEntries} for each quantum graph problem we want to consider. \newline

At present, computer code is being developed to construct the M-matrix for the set of equations \eqref{eq:QGEquation} using proposition \ref{prop:M-MatrixEntries}, and we will also be looking to write code for the more general approach that relies on solving the edge-ODEs directly.
However the bottom line of this issue is that future work needs to be done looking into the computational cost and accuracy of both approaches.
For the purely numerical approach the direction of research is fairly clear; we should look at existing theory in this area that surrounds the ODE solvers that we would be employing, and couple this with the algorithm that we end up proposing to construct the M-matrix.
The alternative approach requires slightly different treatment, as although in theory one will obtain exact expressions for the elements of the M-matrix, we do not yet know how easy it will be to obtain an analogue of proposition \ref{prop:M-MatrixEntries} when the edge-ODEs of $\mathcal{P}_{\qm}$ change.
Indeed it may not even be possible to obtain such a result, or the end-user of the numerical scheme may not want to spend time deriving it.
Assuming we have such a result however, we can do some basic analysis on the computational cost of assembling the M-matrix using this approach and then compare this with the alternative approach.
Although we expect the latter (analytic entries) approach to be more accurate and faster in all situations, the extent of these gains might be considered too small to warrant the derivation of an analogue of proposition \ref{prop:M-MatrixEntries}.

\subsubsection{Determination of the Spectrum of $\mathcal{P}$} \label{sec:ConcFutureGetSpectrum}
The other consideration for our numerical scheme are the nuances that come with solving the generalised eigenvalue problems, and how we construct an approximation to $\sigma\bracs{\mathcal{P}}$ from the values we get for $\sigma\bracs{\mathcal{P}_{\qm}}$.
The foremost problems with the latter is that we cannot take the union of each of the spectra $\sigma\bracs{\mathcal{P}_{\qm}}$ over the $\qm$ as we did analytically - we will be forced to (at best) use a fine mesh of discrete $\qm$ values to build up an approximation to $\sigma\bracs{\mathcal{P}}$.
As such one glaring issue is whether we can quantify how fine a mesh in $\qm$ is required, or whether there are certain values of $\qm$ that are of significance to the spectrum (values that correspond to eigenvalues found at the ends of band-gaps, for example).
There may also be symmetries that we can exploit on a problem-by-problem basis; like in the example in section \ref{sec:ExampleCrossInPlane} where there was symmetry in the components of $\qm$ and hence we could set $\qm_2=0$ to determine the spectrum.
However if we expect some kind of stability of $\sigma\bracs{\mathcal{P}_{\qm}}$ with respect to $\qm$ (that is, we can show that small changes in $\qm$ correspond to some kind of small changes in $\sigma\bracs{\mathcal{P}_{\qm}}$) then the idea of meshing $\qm$ and solving a finite number of the $\mathcal{P}_{\qm}$ isn't a terrible one.
Given the lack of immediate alternative suggestions, this kind of analysis is the way to go on this front. \newline

The former problem mentioned above, the nuances that come with solving the generalised eigenvalue problems, are another concern.
In particular we have to deal with the complication that there may be (and in our case will almost always be) an infinite number of solutions $\bracs{\lambda,v}$ to $M_{\qm}\bracs{\lambda} v = 0$.
Some knowledge of the form of the M-matrix (like proposition \ref{prop:M-MatrixEntries}) is helpful in this regard, for example we know that if the M-matrix is periodic in $\lambda$ then it is sufficient to determine all the unique eigenvalues over one period and from there can construct the remainder.
However even if the M-matrix is periodic the presence of coupling constants on the vertices can make this potential advantage redundant, as can be seen for the M-matrix of section \ref{sec:ExampleThickVertex}.
If a numerical scheme is being used solely from a fabrication/design perspective, then there is always the option to restrict the solver to finding eigenvalues $\lambda$ within a given range of interest, such as the operating frequencies of a PCF in the electromagnetic setting.
We should also not forget that, regardless of the computational power available to us, we can never determine the full spectrum computationally either (we require some analytic techniques for this) and so this compromise is likely one of the best we can provide.
This in turn raises a further question - given a particular range of interest for $\lambda$, how can be we be sure to find every eigenvalue in $\sigma\bracs{\mathcal{P}_{\qm}}$ that lies in this range?
An examination of existing solver methods for generalised eigenvalue problems would go some way to answering this question, alongside whether we can deduce (analytically) anything about the distribution of the eigenvalues for a given $\mathcal{P}_{\qm}$.

\subsection{Effect of Geometry on Spectra} \label{sec:ConcFutureGeometry}
One of our objectives was to attempt to describe how the underlying geometry of the singular-structure affects the resulting spectrum and hand-gaps that the structure exhibits, and some progress has been made with this through the theory of chapter \ref{ch:VectorEqns}.
Namely we see that the $\qm$ undergoes rotations dependant on the orientations of the underlying graph and the quantum graph problem that we obtain has solutions dependant on the lengths of the singular-structure edges.
And although we have not provided the details; it is also known that the relative scaling of the vertex- and edge-regions in the thin-structures we are approximating gives rise to different quantum graph problems and hence different spectra, as we illustrated in the examples of section \ref{sec:ExampleCrossInPlane} and \ref{sec:ExampleThickVertex}.
By affecting the coupling constants (through the scaling of the micro-structure), quasi-momentum and incorporating the edge-lengths, the spectrum and hence band-gaps of the structure are also changed.
This being said, we have not drawn any general insights into how these effects change the resulting spectra, only provided examples in chapter \ref{ch:ExampleSystems} to demonstrate that they do.
We can make one conjecture on this topic though; that a graph with zero coupling constants and (geometric) symmetry in the periodic directions will not give rise to band-gaps.
This idea is reinforced by the examples of section \ref{sec:ExampleCrossInPlane} and \ref{sec:ExampleGeneralLengths}, as well as other examples with geometric symmetries that we have looked at but didn't include in this report. \newline

Another consideration that goes beyond what we have done in this report is looking at geometries whose peroid-cells are not rectangular in shape.
This direction of future work is motivated more by the desire to produce a model that is relevant to physical PCFs and wave-guidance problems, rather than mathematical interest or completeness.
In particular PCFs are typically fabricated with hexagonal lattice-structures, which for us would result in a hexagonal unit cell for our infinite, periodic singular-structure.
The affect is largely felt by the Gelfand transform, as we no longer have a set of orthogonal vectors that describe the translation-invariance of the singular-structure.
We expect the analysis we have carried out in chapter \ref{ch:VectorEqns} to still be of relevance in this slightly altered case; in particular we have demonstrated that curls and gradients of zero are invariant with respect to the quasi-momentum, and so we expect that these objects will not change in these new period cell shapes.
Regardless, there is the option to investigate any changes that the shape of the period cell itself induces in the problems that we obtain, and the work in chapter \ref{ch:VectorEqns} can be used as a basis for this research.

\subsection{Generalisations to our Singular-Structure Systems} \label{sec:ConcFutureMaxwell}
We focused our attention on the ``curl-of-the-curl" equation \eqref{eq:CurlCurlEquationDivFree} in chapters \ref{ch:VectorEqns} and \ref{ch:ExampleSystems} because it arises from the time-harmonic Maxwell system which describes electromagnetic wave propagation.
However the ``curl-of-the-curl" equation can only be derived if we assume a time-harmonic solution to the full Maxwell system, namely assume that the $\mathbf{E}$ and $\mathbf{H}$ fields are of the form
\begin{align*}
	\mathbf{E}\bracs{x_1,x_2,x_3} = \widehat{\mathbf{E}}\bracs{x_1,x_2,x_3}e^{-i\omega t}, 
	&\quad  \mathbf{H}\bracs{x_1,x_2,x_3} = \widehat{\mathbf{H}}\bracs{x_1,x_2,x_3}e^{-i\omega t}.
\end{align*}
Substituting this ansatz into the system of Maxwell equations \eqref{eq:MaxwellSystem} and taking the curl of either equation and substituting the result into the other then provides the curl of the curl equation.
Whilst there is probably little harm in using the ``curl-of-the-curl" equation as our starting point for out singular-structure problems; for the purposes of providing a complete description of electromagnetic wave-guidance on singular-structures it would be good to derive an equivalent quantum graph problem for the system of Maxwell equations.
One could then use the quantum graph problem derived in chapter \ref{ch:VectorEqns} as a check for consistency in the system that was obtained.
No major obstacles are expected by from leaving the dependence on time in the system, however there will need to be some care in how to setup the various function spaces that we are working with.
However because the time co-ordinate is essentially separate from the spatial co-ordinates, we should be able to recycle the work in chapter \ref{ch:VectorEqns} here. \newline

Another possible generalisation that could be made to our systems concerns the ``empty space" (the cores if we adopt the language of PCFs) in our cross-sectional structure.
Currently our singular-structure problems effectively ignore this part of our domain, which lends itself to the description that there is no field $u$ in that part of the domain (or we don't care what it is).
In the context of electromagnetism this would likely mean that this ``empty space" was actually filled with a metallic material, and so we wouldn't expect a field in this region.
In reality this may not be the case (although metallic mode confinement in PCFs has been demonstrated, see \cite{hou2008metallic}), fibres are typically fabricated using two dielectric materials (one of which may be vacuum for PCFs).
As such we would need to consider the system of Maxwell equations \eqref{eq:MaxwellSystem} both on the singular-structure and in the remainder of the domain, and the electric permittivity $\eps_{P}$ and magnetic permeability $\mu_{P}$ would no longer be constant across the whole domain.
This opens up several avenues for exploration - foremost being how to correctly formulate Maxwell's equations in this instance.
We would be required to keep the variational approach we have adopted to deal with the singular-structure correctly, and so the natural avenue of investigation would be to adapt the measure that we pose our variational problem with respect to.
The foremost candidate being a ``Lebesgue plus singular graph" measure, namely a Borel measure
\begin{align*}
	\dddmes\bracs{B} &= \lambda_{2}\bracs{B} + \ddmes\bracs{B}, \quad B\in\mathcal{B}_{\ddom},
\end{align*}
where we now account for the singular structure using our singular measure on the underlying graph $\ddmes$ as before, but also add the standard 2D Lebesgue measure $\lambda_2$ so that we no longer discard the larger regions.
Again using a variational problem posed with respect to $\dddmes$ will allow us to deal with the issue of boundary conditions between the singular structure and surrounding dielectric, but whether we can determine a method to solve such problems (as we did by borrowing theory from quantum graphs) remains open.
There would also be similar questions raised about whether such a variational problem could still be thought of as the limit of some thin-structure problem, like with the singular-structure problems we have considered through this report thus far.

\section{Closing Remarks} \label{sec:ConcClosingRemarks}
The work in this report has made progress towards addressing the research objectives that it set out to achieve, as laid out in chapter \ref{ch:Intro}.
We look to consider singular-structure domains as setup in section \ref{sec:OurPhysicalSetup}, motivated by existing theory for variational problems (section \ref{sec:VariationalProblemLitReview}, chapter \ref{ch:ScalarEqns}) and seeking a consistent framework for our approximation to physical waveguides.
We develop this theory of variational problems in the context of our singular-structure problems in chapter \ref{ch:VectorEqns}; describing what the concepts of gradient, curl and divergence-free mean and hence building appropriate function spaces for our problems.
Furthermore, a link to quantum graph problems from our singular-structure problems is established (section \ref{sec:CurlReductionToQG}), meaning we can bring in existing theory from quantum graph problems (section \ref{sec:GraphLitReview}, chapter \ref{ch:QuantumGraphs}) to aid in the solution to our original singular-structure problems, and open up the potential for numerical approaches to be used (section \ref{sec:NumericalMethodsDiscussion}).
The link to quantum-graph problems also solidifies our singular-structure problems as formal limits of more familiar thin-structure models for waveguides (section \ref{sec:GraphLitReview}), addressing the first of the research objectives.
The examples of chapter \ref{ch:ExampleSystems} serves the purpose of bringing together the theory of the previous chapters and highlighting important considerations for solving our singular-structure problems.
We are able to demonstrate analytically that certain geometries (underlying graphs) give rise to band-gap spectra whilst others do not, and discuss how a numerical scheme that is designed to solve such problems might be employed (section \ref{sec:NumericalMethodsDiscussion}).
These examples also go some way to addressing research objectives two and three, although as we highlight in section \ref{sec:ConcFuture} there are further considerations and questions that need answering.
Section \ref{sec:ConcFutureMaxwell} also highlights that we have not yet fully addressed the first research objective, at least in the context of PCFs (electromagnetic wave-guidance) which is another direction we should pursue. \newline

In summary, the work presented in this report serves as a basis for future work on the objectives in chapter \ref{ch:Intro}.
The existing theory gathered, and that which has been developed, will allow further investigation into (limits of) more descriptive wave-guidance problems - addressing objective one.
The examples that have been analysed provide some underlying intuition about what we should expect the answers to objective two should be, and highlight the issues that a numerical scheme sought by objective three will need to consider.
We have discussed the direction of future work that will be taken with each of these objectives in mind throughout section \ref{sec:ConcFuture}, and will look to pursue these directions in the immediate future.