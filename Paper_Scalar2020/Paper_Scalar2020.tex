\documentclass[10pt]{article}

\usepackage{url}
%\usepackage[margin=2.5cm]{geometry} % See geometry.pdf to learn the layout options. There are lots.
\usepackage{geometry}
\geometry{a4paper} %or letterpaper or a5paper or ...
\usepackage{fullpage}

%for figures and graphics
\usepackage{graphicx}
\usepackage{subcaption} %allows subfigures
\usepackage[bottom]{footmisc} %footnotes go below figures
%\usepackage{parskip} %adds line space between paragraphs by default
\usepackage{enumerate}

\DeclareGraphicsRule{.tif}{png}{.png}{`convert #1 `dirname #1`/`basename #1 .tif`.png}
\graphicspath{{../Diagrams/Diagram_PDFs/} {../Diagrams/Numerical_Results/}}

%\input imports all commands from the target files
%The idea behind this file is that it will be used to store all the maths-related macros that I concoct; so that I can import all the commands by \input{this file} in the preamble of any file that I want to use them in.
%This should make the top-level files look a lot cleaner, and the preamble much shorter!

\usepackage{amssymb}
\usepackage{amsmath}
%\usepackage{mathtools}

%theorems and lemma etc setup using amsthm
\usepackage{amsthm}
\theoremstyle{definition}
\newtheorem{definition}{Definition}[section]
\theoremstyle{plain}
\newtheorem{theorem}{Theorem}[section]
\theoremstyle{plain}
\newtheorem{lemma}[theorem]{Lemma}
\theoremstyle{plain}
\newtheorem{prop}[theorem]{Proposition}
\theoremstyle{plain}
\newtheorem{cory}[theorem]{Corollary}
\theoremstyle{definition}
\newtheorem{convention}[theorem]{Convention}
\theoremstyle{definition}
\newtheorem{assumption}[theorem]{Assumption}
\theoremstyle{definition}
\newtheorem{conjecture}[theorem]{Conjecture}

\allowdisplaybreaks %allows equations in the same align environment to split over multiple pages.

%tstk always need be there
\newcommand{\tstk}[1]{\textbf{#1} \newline}

%this adds extra functionality to pmatrix, vmatrix, bmatrix etc by allowing you to pass an optional argument in [FACTOR] to multiply the default spacing between elements by FACTOR
\makeatletter
\renewcommand*\env@matrix[1][\arraystretch]{%
  \edef\arraystretch{#1}%
  \hskip -\arraycolsep
  \let\@ifnextchar\new@ifnextchar
  \array{*\c@MaxMatrixCols c}
  }
\makeatother

%begin the macros via newcommand. Try to group them up reasonably!

%notation and variable use throughout the file
\renewcommand{\vec}[1]{\mathbf{#1}}				%vectors are bold, not overline arrow
\newcommand{\recip}[1]{\frac{1}{#1}}			%fast reciprocal as a fraction
\newcommand{\interval}[1]{\sqbracs{0,\abs{#1}}}	%creates the closed interval from 0 to the length of the input #1, denoted by absolute value
\newcommand{\eps}{\varepsilon}					%pretty epsilons
\newcommand{\charFunc}[1]{\mathcal{I}_{#1}}%{\mathbb{1}_{#1}}		%characteristic function of a set

\newcommand{\dddom}{\widetilde{\Omega}}			%3D domain notation
\newcommand{\ddom}{\Omega}						%2D domain notation
\newcommand{\dddmes}{\widetilde{\mu}}			%3D measure
\newcommand{\ddmes}{\mu}						%2D measure

\newcommand{\graph}{\mathbb{G}}					%graph variable
\newcommand{\vertSet}{\mathcal{V}}					%set of vertices rather than big V
\newcommand{\edgeSet}{\mathcal{E}}					%set of edges rather than large E, \graph = (V,E)
\newcommand{\wavenumber}{\kappa}				%fourier variable or wavenumber, not to confuse with jk subscripts!
\newcommand{\qm}{\theta}						%quasi-momentum parameter
\newcommand{\kt}{\bracs{\wavenumber, \qm}}		%(k, theta) pair
\newcommand{\dmap}{\Gamma_0}					%Dirichlet map
\newcommand{\nmap}{\Gamma_1}					%Neumann map
\newcommand{\effFreq}{\Lambda}					%Effective frequency sqrt(w^2-\wavenumber^2)
\newcommand{\conLeft}{\stackrel{\rightarrow}{\smash{\sim}\rule{0pt}{0.4ex}}} %j connects to k, j left
\newcommand{\conRight}{\stackrel{\leftarrow}{\smash{\sim}\rule{0pt}{0.4ex}}} %j connects to k, j right
\newcommand{\con}{\sim}							%j connects to k, indifferent of direction

%standard sets
\newcommand{\naturals}{\mathbb{N}}			%natural numbers
\newcommand{\integers}{\mathbb{Z}}			%integers
\newcommand{\rationals}{\mathbb{Q}}			%rational numbers
\newcommand{\reals}{\mathbb{R}}				%real numbers
\newcommand{\complex}{\mathbb{C}}			%complex numbers

%other notations
\newcommand{\rmi}{\mathrm{i}}				%imaginary unit i (RoMan font i)
\newcommand{\e}{\mathrm{e}}					%Euler number, e (Roman font e)

%brackets and norms
\newcommand{\bracs}[1]{\left( #1 \right)}				%encloses input in brackets
\newcommand{\sqbracs}[1]{\left[ #1 \right]}				%encloses input in square brackets
\newcommand{\clbracs}[1]{\left\{ #1 \right\}}			%encloses input in curly bracers
\newcommand{\abs}[1]{\left\lvert #1 \right\rvert}					%absolute value
\newcommand{\norm}[1]{\lvert\lvert #1 \rvert\rvert}		%norm 
\newcommand{\setVert}{\ \middle\vert \ }					%vertical bar for the middle of sets

%function sets
\newcommand{\smooth}[1]{C^{\infty}\bracs{#1}}							%smooth functions
\newcommand{\ltwo}[2]{L^{2}\bracs{#1,\mathrm{d}#2}}						%general L^2 space
\newcommand{\gradSob}[2]{H^1_\mathrm{grad}\bracs{#1, \mathrm{d}#2}}		%gradient Sobolev space
\newcommand{\gradSobQM}[2]{H^1_{\qm, \mathrm{grad}}\bracs{#1, \mathrm{d}#2}} %gradient + i\qm Sobolev space
\newcommand{\ktgradSob}[2]{H^1_{\wavenumber,\qm,\mathrm{grad}}\bracs{#1, \mathrm{d}#2}}	%k,\qm-gradient Sobolev space
\newcommand{\curlSob}[2]{H^1_\mathrm{curl}\bracs{#1, \mathrm{d}#2}}		%curl Sobolev space
\newcommand{\tcurlSob}[2]{H^1_{\qm, \mathrm{curl}}\bracs{#1, \mathrm{d}#2}}		%curl + i\qm Sobolev space
\newcommand{\kcurlSob}[2]{H^1_{\wavenumber,\mathrm{curl}}\bracs{#1, \mathrm{d}#2}}	%k-curl Sobolev space
\newcommand{\ktcurlSob}[2]{H^1_{\wavenumber,\qm,\mathrm{curl}}\bracs{#1, \mathrm{d}#2}}	%k,\qm-curl Sobolev space
\newcommand{\ktcurlSobDivFree}[2]{\mathcal{H}^{\kt}\bracs{#1, \mathrm{d}#2}}					%k,\qm-curl, divergence-free Sobolev space
\newcommand{\supp}{\mathrm{supp}}										%support of a function

%grad and curl sets
\newcommand{\gradZero}[2]{\mathcal{G}_{ #1, \mathrm{d}#2}\bracs{0}}		%gradients of zero for domain #1 with measure #2
\newcommand{\kgradZero}[2]{\mathcal{G}_{ #1, \mathrm{d}#2}^{(\wavenumber)}\bracs{0}}	%k-gradients of zero for domain #1 with measure #2
\newcommand{\curlZero}[2]{\mathcal{C}_{ #1, \mathrm{d}#2}\bracs{0}}	%curls of zero for domain #1 with measure #2
\newcommand{\kcurlZero}[2]{\mathcal{C}_{ #1, \mathrm{d}#2}^{(\wavenumber)}\bracs{0}}	%k-curls of zero for domain #1 with measure #2

%derivatives and grad-like symbols
\newcommand{\diff}[2]{\dfrac{\mathrm{d}#1}{\mathrm{d}#2}}			%complete derivative d#1/d#2
\newcommand{\pdiff}[2]{\dfrac{\partial #1}{\partial #2}}			%partial derivative p#1/p#2
\newcommand{\ddiff}[2]{\dfrac{\mathrm{d}^2 #1}{\mathrm{d} {#2}^2}}	%2nd deriv
\newcommand{\pddiff}[2]{\dfrac{\partial^2 #1}{\partial {#2}^2}}		%2nd partial derivative
\newcommand{\grad}{\nabla}											%grad operator
\newcommand{\tgrad}{\nabla^{\qm}}									%grad operator with qm superscript
\newcommand{\kgrad}{\grad^{(\wavenumber)}}							%grad with wavenumber superscript
\newcommand{\ktgrad}{\grad^{\kt}}				%grad with wavenumber, qm superscript
\newcommand{\curl}[1]{\grad_{#1}\wedge}							%curl with subscript #1
\newcommand{\kcurl}[1]{\grad_{#1}^{(\wavenumber)}\wedge}			%k-curl with measure subscript #1
\newcommand{\ktcurl}[1]{\grad_{#1}^{\kt}\wedge}		%k,theta-curl with measure subscript #1
\newcommand{\laplacian}{\Delta}						%laplacian operator, can have subscripts attached

%displaying integrals
\newcommand{\integral}[3]{\int_{#1}#2 \ \mathrm{d}#3}			%integral, domain #1, integrand #2, measure #3
\newcommand{\md}{\mathrm{d}}									%differential d

%convergence
%\newcommand{\lconv}[1]{\xrightarrow{#1}}							%convergence with #1 above the rightarrow - requires mathtools
\newcommand{\lconv}[1]{\overset{#1}{\longrightarrow}}				%convergence with #1 above the rightarrow
\newcommand{\toInfty}[1]{ \ \text{as} \ #1 \rightarrow\infty}		%writes out "as #1 tends to infty" %maths commands, variables, and other packages

%labelling hacks
\newcommand\labelthis{\addtocounter{equation}{1}\tag{\theequation}}

%reminders of things to fill in
%\newcommand{\tstk}[1]{\textbf{#1}\newline}

%changes the title of the bibliography page to ``References"
\renewcommand{\refname}{References} %NB: For reports, this is \bibname rather than \refname

\title{Wave propagation in two-dimensional singular photonic crystals via spectral analysis of quantum graph problems}  % Declares the document's title.
\author{K. Cherednichenko \& W. Graham} 	% Declares the author's name.
\date{\today}      % commenting out this command produces today's date.

%-------------------------------------------------------------------------
%DOCUMENT STARTS

\begin{document}


%REPORT BEGINS

\maketitle

%\input chapters one at a time; the references should all match up and can even cross-reference; however you won't get prompts for references across files when editing.
%NB: Can use \include for speed, but then the directory fills up with useless .aux files.
%To save time, comment out sections that aren't being edited. References may disappear but the file will still compile!

\begin{abstract}

\tstk{this might need a second look?}
We develop a mathematical framework for the study of wave propagation in two-dimensional photonic crystals whose cross-sections are well approximated by ``graph-like" structures. 
We study spectral properties of such structures for various types of coupling conditions at the junctions between the graph edges, in the case of transverse polarisation, when either the electric or magnetic field is aligned with the crystal.

\tstk{note sure where acknowledgements go, but SAMBa requires me to include one:
William Graham is supported by a scholarship from the EPSRC Centre for Doctoral Training in Statistical Applied Mathematics at Bath (SAMBa), under the project EP/L015684/1.}

\end{abstract}

\section{Introduction} \label{sec:Intro}

This section includes our literature review and motivation.
The purpose of the paper is to present the equations that were derived for the curl-curl system as what can be interpreted as an "effective" or "limit" problem for a fine/thin-structure material.
This is subject to future advances in the field that draw parallel to the scalar-case developments, of course.

\chapter{Quantum Graphs} \label{ch:QuantumGraphs}
In this chapter we shall introduce the concept of Quantum Graphs, their associated function spaces, and operators defined on these spaces.
Heavy references throughout to EKK, Kuchment, I guess Olaf \& Post too maybe?

\section{Introduction, Notation and Conventions}
It's really hard to define this shit.

\tstk{we only work with finite graphs!}
\begin{definition}[Graph] \label{def:Graph}
	Let $N\in\naturals$ and $V=\clbracs{v_j \ \vert \ j\in\clbracs{1,2,...,N}}$ be a set of labels $v_j$ bijective to $\clbracs{1,2,...,N}$ via the map $j\rightarrow v_j$.
	Let $E\subset V\times V$ be a finite set of \textit{unordered} pairs $\bracs{v_j,v_k}\in E$ where $j,k\in\clbracs{1,2,...,N}$.
	If $\bracs{v_j,v_k}\in E$ then write $I_{jk} = \bracs{v_j,v_k}$, note that $I_{jk}=I_{kj}$.
	Then $\graph=\bracs{V,E}$ is a (finite) graph with vertex set $V$ and edge set $E$.
	Elements of the set $V$ are called vertices of the graph $\graph$ and elements of $E$ are referred to as edges of $\graph$.	
\end{definition}
Definition \ref{def:Graph} is not as general as others that can be found in graph theory, however we work with this definition because it is sufficient for our later purposes so the loss of generality and certain features is not an issue.
In particular we highlight some features of this definition below.
\begin{itemize}
	\item The definition requires that there is only a single edge $I_{jk}$ connecting any pair of vertices.
	We will shortly define the concept of a directed graph where we allow two edges between any pair of vertices, having opposite ``directions".
	We do not go any further than this, even though it is possible by introducing certain equivalence relations on the sets $V$ and $E$, simply because we will not be interested in any systems that require this functionality. 
	\item A graph has a finite number of vertices and edges.
	This does not need to be true in general, and there is nothing wrong with removing the restriction that $V$ and $E$ be finite.
	However since we shall be wanting to use graphs to represent physical structures in some sense (\tstk{chapter ref}), we will not be needing this functionality.
	\item Loops (edges of the form $I_{jj}$) are permitted, but a given vertex can only have at most one loop.
	\item The labels $v_j$ are slightly unnecessary, as one can just work directly with the index set $\clbracs{1,2,...,N}$.
	However we will be wanting labels for our vertices and edges when we come to consider quantum and embedded graphs.
\end{itemize}

Having laid out this basis for the concept of a graph, we can now present some further definitions.
\begin{definition}[Directed Graph] \label{def:DirectedGraph}
	Let $N\in\naturals$ and $V=\clbracs{v_j \ \vert \ j\in\clbracs{1,2,...,N}}$ be a set of labels $v_j$ bijective to $\clbracs{1,2,...,N}$ via the map $j\rightarrow v_j$.
	Let $E\subset V\times V$ be a finite set of \textit{ordered} pairs $\bracs{v_j,v_k}\in E$ where $j,k\in\clbracs{1,2,...,N}$.
	Write $I_{jk} = \bracs{v_j,v_k}$ for the elements of $E$.
	Then $\graph=\bracs{V,E}$ is a (finite) directed graph with vertex set $V$ and edge set $E$.
	Elements of the set $V$ are called vertices of the graph $\graph$ and elements of $E$ are referred to as edges of $\graph$.
	Each edge $I_{jk}$ where $j\neq k$ is referred to as the edge directed from $v_j$ to $v_k$, or just the edge from $v_j$ to $v_k$.
\end{definition}
Loops are still permitted by this definition, however the choice of a direction for these is essentially redundant.
\begin{definition}[Quantum Graph] \label{def:QuantumGraph}
	A Quantum Graph is a directed graph $\graph = \bracs{V,E}$ where each edge $I_{jk}\in E$ is assigned a length $l_{jk}\geq0$ and associated interval $\interval{l_{jk}}$.
\end{definition}
Note that there is no requirement for the edges (if they are present) $I_{jk}$ and $I_{kj}$ to have the same length, we shall see later why we wish to allow this.
Loops are still permitted under this definition and will have an associated length $l_{jj}$.
Quantum graphs will be the objects that the theory of this section will describe, however they are not the starting point for our physical problems \tstk{chapter/section ref}.
We require one further definition that will enable us to link a structure in physical space to the more abstract Quantum graph.

\begin{definition}[Embedded Graph] \label{def:EmbeddedGraph}
	Set $d\geq2, D\subset\reals^d$ and $N\in\naturals$.
	Let $V = \clbracs{\vec{v}_j \ \vert \ j\in\clbracs{1,2,...,N}}$ be a set of distinct points in $D$, and $E\subset V\times V$ be a set of \textit{ordered} pairs of points $I_{jk} := \bracs{\vec{v}_j, \vec{v}_k}$.
	For each $I_{jk}\in E$ with $j\neq k$ let $\gamma_{jk}$ be a continuous curve in $D$ with endpoints $\vec{v}_j$ and $\vec{v}_k$, length $l_{jk}$ and smooth parametrisation $r_{jk}:\interval{l_{jk}}\rightarrow\gamma_{jk}$ such that $r_{jk}(0) = \vec{v}_j, r_{jk}\bracs{l_{jk}} = \vec{v}_k$.
	For each $I_{jj}\in E$ let $\gamma_{jj}$ be a closed curve in $D$ passing through $\vec{v}_j$ and with smooth parametrisation $r_{jj}:\left[0,l_{jj}\right)\rightarrow\gamma_{jj}$ such that $r_{jj}(0) = \vec{v}_{j}, \lim_{t\rightarrow l_{jj}}r_{jj}(t) = \vec{v}_j$.
	Assume that all curves $\gamma_{jk}$ are non-intersecting.
	Then we call $\graph=\bracs{V, E, \clbracs{r_{jk}}}$ an embedded graph in $D$, or a graph embedded in $D$.
\end{definition}
Again we make some observations; and provide some motivation and conventions for this definition.
\begin{itemize}
	\item Because the maps $r_{jk}$ are tied to the edges $I_{jk}$, for shorthand we will forgo including these when we introduce an embedded $\graph$, unless there is a need for a notational change.
	As such, we shall specify embedded graphs by the shorthand $\graph=\bracs{V,E}$, meaning $\graph=\bracs{V, E, \clbracs{r_{jk}}}$.
	\item Am embedded graph $\graph = \bracs{V,E}$ is a framework for representing singular structures in physical space, by associating the vertices to points and the edges of a graph to curves connecting these points, we can think of the graph as occupying some physical volume/area.
	We can also effectively treat $\graph$ as a subset of $D$, and perform set operations to construct sub-graphs, or use set intersections to pull out select portions of a graph.
	For example, we may specify the sub-graph of $\graph$ composed of all the loops of $\graph$ by writing
	\begin{align*}
		S_{\graph} &:= \bigcup_{I_{jj}\in E} I_{jj}
	\end{align*}
	which should be taken to have the same meaning as the following;
	\begin{align*}
		V_S := \clbracs{\vec{v}_j\in V \ \vert \ I_{jj}\in E}, &\quad E_S := \clbracs{I_{jj} \ \vert \ I_{jj}\in E}, \\
		R_S := \clbracs{r_{jj} \ \vert \ I_{jj}\in E}, &\\
		S_{\graph} &:= \bracs{V_S, E_S, R_S}.
	\end{align*}
	Likewise we may also use $\graph$ as a set in the sense that
	\begin{align*}
		\graph &= \bigcup_{I_{jk}\in E} \gamma_{jk},
	\end{align*}
	so we could specify the subset of $D$ corresponding to the portion of the graph $\graph$ that occupies the square $\sqbracs{-\recip{2},\recip{2}}^2$ by writing
	\begin{align*}
		\graph \cap \sqbracs{-\recip{2},\recip{2}}^2.
	\end{align*}
	Essentially, we may treat $\graph$ as both a set in $D$ and in the sense of a graph as in definition \ref{def:DirectedGraph}.
	\item Loops (closed curves) are permitted but require a slightly different treatment to ``regular" edges; and although they don't introduce major complexities in the theory that follows in this chapter, can introduce major complexities in the theory we wish to build on in chapters \ref{ch:ScalarEqns} and \ref{ch:VectorEqns}.
\end{itemize}

NEED to talk to Kirill about this - in particular how do we talk about periodic graphs? Also we might end up with loops in our equivalent QG despite not having them in our embedded graphs.
Also loops do nasty things to our gradients etc if they are in the embedded graphs, we don't consider embedded graphs with loops though.
We do consider QGs though that come from period-cells of periodic graphs, in which case we need to talk about how we associate edges and how we can get loops out of non-loopy period cells.
Essentially need a consistent framework to build off - the definition of embedded graph doesn't use any graph theory so is quite nice, but still need to define ``periodic embedded graph", unit cell, etc.
Once we've done this it should be fine - we only ever work on the period cell which is a graph embedded into (WLOG) $\sqbracs{0,1}^2$ and so our finite graph terminology is sufficient from then on.

\section{Differential Equations on Quantum Graphs}
Since quantum graphs come with lengths (and intervals) associated to their edges, we can define function spaces on them by combining function spaces on these intervals.
As such we define
\begin{align*}
	L^2\bracs{\graph} := \bigoplus_{I_{jk}\in E} \ltwo{\interval{l_{jk}}}{t},
	&\quad H^1\bracs{\graph} := \bigoplus_{I_{jk}\in E} \gradSob{\interval{l_{jk}}}{t}.
\end{align*}
A function $u\in L^2\bracs{\graph}$ is then determined by it's form on each edge $I_{jk}$ (and similarly for functions and their distributional derivatives in $H^1\bracs{\graph}$).
Because we will mainly be working on the edges of our graphs, we define $u_{jk} = u\vert_{I_{jk}}$ to be the restriction of $u$ to the edge $I_{jk}$, extended by zero to the whole of $\graph$.
Due to the fact that edges are directed, it is also necessary for us to adopt a notion of ``directional derivative" for the $u_{jk}$ at the ends of the edges.
Specifically we must distinguish between a derivative directed into a vertex, and a derivative directed out from a vertex.

\section{Spectral Problems and the M-Matrix}

\section{Summary}

\section{Derivation and Formulation of Quantum Graph Problem} \label{sec:SystemDerivation}

In this section we provide an overview of how we obtain the system \eqref{eq:QGFullSystem} from \eqref{eq:PeriodCellLaplaceStrongForm}, which will setup our discussion revolving around the methods that can be employed for solving \eqref{eq:QGFullSystem} in section \ref{sec:Discussion}.
To reiterate what was said in section \ref{sec:QuantumGraphs}, assumption \ref{ass:MeasTheoryProblemSetup} is adopted throughout this section and the rest of this work. \newline

Precise definition and analysis of the ``Sobolev spaces" used here can be found in the appendix (\ref{app:MeasureTheory}), although we provide a short intuitive idea of the object $\tgrad_{\ddmes}u$ here.
The central idea behind understanding $\tgrad_{\ddmes}u$ is that the singular measure $\ddmes$ only supports the edges of $\graph$, and so cannot ``see" any changes in the function $u$ ``across" (or in the direction perpendicular) to the edge $I_{jk}$.
So at any point $x\in I_{jk}$, the ``gradient" $\tgrad_{\ddmes}u$ encapsulates the rate of change of the function $u$ \emph{only} in the direction along the edge $I_{jk}$.
This is true for each edge $I_{jk}$, and so $\tgrad_{\ddmes}u$ has the form of a one-dimensional derivative along each edge $I_{jk}$.
As a result, it is not inaccurate to think of $\tgrad_{\ddmes}u(x) = u_{jk}'(x)e_{jk}$ for $x\in I_{jk}$, where $u_{jk}' = \pdiff{u_{jk}}{e_{jk}}$.
This also means that $\tgrad_{\ddmes}u$ can be characterised by its form on each edge of the graph $\graph$, which is crucial for deriving the set of ``edge ODEs" \eqref{eq:QGEdgeODEs} and gives meaning to the $\diff{}{t}$ operator which appears in those equations.
The coupling constants attached to the vertices of the graph, as well as the connectivity of the graph itself, then dictates that these ``edge-wise" components $u_{jk}$ and $u'_{jk}$ adhere to certain matching conditions at the vertices.
These conditions transpire to be continuity of the function $u$ at all vertices of the graph, which one might expect given that we claim to be working with (some measure-theoretic notion of) differentiable functions and Sobolev spaces.
We also find that the ``gradient" of the function $u$ at the vertices vanishes at the vertices $v_j\in\vertSet$, however the incoming directional derivatives along the edges connecting to $v_j$ need not have limit zero as they approach $v_j$.
In summary, the functions $u\in\gradSobQM{\ddom}{\dddmes}$ and their gradients $\tgrad_{\dddmes}u$ can be thought of as possessing the following properties (precise statements can be found in appendices \ref{app:muAnalysis}-\ref{app:SumMeasureAnalysis}):
\begin{itemize}
	\item The function $u$ is continuous at all vertices $v_j\in\vertSet$.
	\item On each edge $I_{jk}\in\edgeSet$; $\tgrad_{\dddmes}u = \tgrad_{\lambda_{jk}}u$, with $\tgrad_{\lambda_{jk}}u$ in turn amounting to the directional derivative of $u_{jk}$ along the edge $I_{jk}$.
	\item At each vertex $v_j\in\vertSet$, we have $\tgrad_{\dddmes}u=0$, however $\lim_{x\rightarrow v_j}\tgrad_{\lambda_{jk}}u$ need not be zero.
\end{itemize}
The reality is not quite as simple as described above, particularly for understanding the nature of the gradients at the vertices, but this idea is the motivation for the line of argument presented in appendix \tstk{ref} and suffices for the derivation of \eqref{eq:QGFullSystem} that we present below. \newline

In this section we will write our usual ``singular measure plus vertex point masses" as
\begin{align*}
	\dddmes &= \ddmes + \nu, \quad
	& \nu = \sum_{v_j\in\vertSet}\alpha_j \delta_j,
\end{align*}
where $\ddmes$ is the singular measure supporting one of our usual graphs $\graph$; and for each $v_j\in\vertSet$, $\delta_j$ is a point-mass measure centred on the vertex $v_j$, and $\alpha_j$ is the coupling constant attached to the vertex $v_j$.
A function $u\in\gradSobQM{\ddom}{\dddmes}$ solves \eqref{eq:PeriodCellLaplaceStrongForm} if and only if
\begin{align} \label{eq:PeriodCellLaplaceWeakForm}
	\integral{\ddom}{\tgrad_{\dddmes}u\cdot\overline{\tgrad_{\dddmes}\phi}}{\dddmes} &= \omega^2\integral{\ddom}{u\overline{\phi}}{\dddmes}, \quad\forall \phi\in\smooth{\ddom}.
\end{align}
We first note that since we require this expression to hold for all smooth functions $\phi$, then for each $I_{jk}\in \edgeSet$ it must in hold for all those smooth functions $\psi$ whose support intersects the interior of the edge $I_{jk}$ and no other parts of $\graph$.
Combined with the fact that $\dddmes$ is just a sum of the edge measures (and point masses at the vertices), and that $\tgrad_{\dddmes}u=\tgrad_{\lambda_{jk}}u$ on the edge $I_{jk}$, in this case we can reduce \eqref{eq:PeriodCellLaplaceWeakForm} to
\begin{align*}
	0 &= \integral{\ddom}{\tgrad_\ddmes u \cdot \overline{\tgrad\psi} - \omega^2 u\overline{\psi}}{\ddmes} \\
	&= \integral{I_{jk}}{ \tgrad_{\lambda_{jk}}u \cdot \overline{\tgrad\psi} - \omega^2 u_{jk}\overline{\psi} }{\lambda_{jk}} \\
	&= \integral{I_{jk}}{ \bracs{u_{jk}' + i\bracs{R_{jk}\qm}_1 u_{jk}}\bracs{\overline{\psi}' - i\bracs{R_{jk}\qm}_1 \overline{\psi} } - \omega^2 u_{jk}\overline{\psi} }{\lambda_{jk}}.
\end{align*}
Now using $r_{jk}$ as a change of variables and denoting $\tilde{u} = u \circ r_{jk}$ and $\varphi = \psi\circ r_{jk}$ we arrive at
\begin{align*}
	0 &= \int_{0}^{\abs{I_{jk}}} \bracs{\tilde{u}_{jk}' + i\bracs{R_{jk}\qm}_1 \tilde{u}_{jk}}\bracs{\overline{\varphi}' - i\bracs{R_{jk}\qm}_1 \overline{\varphi} } - \omega^2 \tilde{u}_{jk}\overline{\varphi} \ \md t .
\end{align*}
This holds for all smooth $\varphi$ with support contained in the interior of $\interval{I_{jk}}$, and can be thought of as the weak form of the equation
\begin{align*}
	-\bracs{\diff{}{t} + i\qm_{jk}}^2 \tilde{u}_{jk} &= \omega^2 \tilde{u}_{jk}, \quad t\in\interval{I_{jk}},
\end{align*}
where $\qm_{jk} = \bracs{R_{jk}\qm}_1$.
Under the assumption that $\tilde{u}_{jk}$ has enough regularity to be differentiated again, this is precisely what we have arrived at.
Since we can repeat this process for each $I_{jk}$, we essentially have an ODE that must be satisfied on each edge of the graph $\graph$. \newline

Now we turn our attention to the vertices to derive the vertex conditions we will need to complete our ODEs on the edges.
Fix a specific $v_j\in \vertSet$, and consider functions $\psi\in\smooth{\ddom}$ whose support only intersects $\graph$ in some neighbourhood of $v_j$ that only contains edges which connect to $v_j$ (imagine the support as a ball centred on $v_j$, for example).
Then we can work from \eqref{eq:PeriodCellLaplaceWeakForm} to obtain
\begin{align*}
	0 &= \sum_{j\con k} \integral{I_{jk}}{ \tgrad_\ddmes u \cdot \overline{\tgrad\psi} - \omega^2 u\overline{\psi} }{\lambda_{jk}} 
	+ \integral{\ddom}{\tgrad_{\dddmes}u\cdot\overline{\tgrad_{\dddmes}\psi}-\omega^2 u\overline{\psi}}{\nu} \\
	&= \sum_{j\con k} \int_{0}^{\abs{I_{jk}}} \bracs{\tilde{u}_{jk}' + i\bracs{R_{jk}\qm}_1 \tilde{u}_{jk}}\bracs{\overline{\varphi}' - i\bracs{R_{jk}\qm}_1 \overline{\varphi} } - \omega^2 \tilde{u}_{jk}\overline{\varphi} \ \md t \\
	&\quad + \alpha_j\sqbracs{ \tgrad_{\dddmes}u\cdot\overline{\tgrad_{\dddmes}\psi}-\omega^2 u\overline{\psi} }\vert_{v_j} \\
	&= \sum_{j\con k} \int_{0}^{\abs{I_{jk}}} \bracs{\tilde{u}_{jk}' + i\bracs{R_{jk}\qm}_1 \tilde{u}_{jk}}\bracs{\overline{\varphi}' - i\bracs{R_{jk}\qm}_1 \overline{\varphi} } - \omega^2 \tilde{u}_{jk}\overline{\varphi} \ \md t \\
	&\quad - \alpha_j \omega^2 u\bracs{v_j}\overline{\psi}\bracs{v_j}.
\end{align*}
Here, we have again used $r_{jk}$ as a change of variables (and the same notation for the transforms of $u$ and $\psi$), and the results of the appendix \ref{app:SumMeasureAnalysis} inform us that $\tgrad_{\dddmes}$ vanishes at the vertex $v_j$.
Under the assumption that $\tilde{u}_{jk}$ can be differentiated again, we can integrate by parts in each integral to obtain
\begin{align*}
	\alpha_j\omega^2 u\bracs{v_j}\overline{\psi}\bracs{v_j} &= - \sum_{j\con k} \int_{0}^{\abs{I_{jk}}} \bracs{ \bracs{\diff{}{t} + i\qm_{jk}}^2 \tilde{u}_{jk} +\omega^2 \tilde{u}_{jk} }\overline{\varphi} \ \md t \\
	&\quad + \sum_{j\con k}\overline{\varphi}\bracs{v_j}\bracs{\diff{}{t} + i\qm_{jk}}\tilde{u}_{jk}\bracs{v_j} \\
	&= \overline{\varphi}\bracs{v_j}\sum_{j\con k}\bracs{\diff{}{t} + i\qm_{jk}}\tilde{u}_{jk}\bracs{v_j}.
\end{align*}
Given that this holds for every smooth $\varphi$, and that $\overline{\varphi}\bracs{v_j}=\overline{\psi}\bracs{v_j}$, we arrive at the condition that
\begin{align*}
	\alpha_j\omega^2 u\bracs{v_j} &= \sum_{j\con k}\bracs{\diff{}{t} + i\qm_{jk}}\tilde{u}_{jk}\bracs{v_j}, \quad \forall v_j \in \vertSet.
\end{align*}
Repeating the argument for each $v_j\in \vertSet$ then provides us with a condition of this form at each vertex; a Robin-like condition on the derivatives of the edge-wise components of $u$, similar to those introduced in section \ref{sec:QuantumGraphs}.
Given the result of theorem \ref{thm:CharOfGradSob} tells us that functions $u\in\gradSobQM{\ddom}{\dddmes}$ are also continuous at each vertex $v_j$, then we have thus derived the following problem:
\begin{align*}
	-\bracs{\diff{}{t} + i\qm_{jk}}^2 \tilde{u}_{jk} = \omega^2 \tilde{u}_{jk}, &\quad t\in\interval{I_{jk}}, \quad \forall I_{jk}\in \edgeSet, \\
	u \text{ is continuous at each } &v_j \in \vertSet, \\
	\sum_{j\con k}\bracs{\diff{}{t} + i\qm_{jk}}\tilde{u}_{jk}\bracs{v_j} &= \omega^2\alpha_{j}u\bracs{v_j},  \quad \forall v_j \in \vertSet,
\end{align*}
that is, we have arrived at \eqref{eq:QGFullSystem}.
Solving for the eigenvalues $\omega^2$ will net us the eigenvalues of our original problem \eqref{eq:PeriodCellLaplaceStrongForm} and taking the union of the eigenvalues over $\qm$ will provide the spectrum of \eqref{eq:WholeSpaceLaplaceEqn}.
As will be made clear in the discussion that follows, the quantum graph problem \eqref{eq:QGFullSystem} is much easier to handle both analytically and numerically.

\section{General formula for the $M$-Matrix of a finite period graph} \label{sec:Discussion}
In this section we want to discuss some of the important applications of our work.
We should touch on:
\begin{itemize}
	\item ``decorations" - adding a graph to a periodic 1D structure by attaching at a vertex (or better: taking a graph, breaking an edge in two and identifying either sides of the ``break" as the edge of the period cell).
	This ties into the setups of EG \cite{cherednichenko2019time} (might be wrong Kirill paper, all the bibTex refs have similar names, check the PDFs and ensure citation points to the right work)
	\item M-matrix work to reduce the QG problem into a nicer, more accessible state.
	Potentially mention we have code to solve this, or do not depending on data plans!
	Also, avoid explicit computation and discussion of numerical ideas unless we want to include them - the examples section which follows is where to put these things.
	\item Also might want to discuss our thoughts about graph symmetry affecting the M-matrix properties (real), stability results, etc, which we can formulate into (at least) conjectures.
\end{itemize}
Although, we have one ``example" of a decoration in the Examples section, and plan to also have an example for bullet point number 2 in there as well. So we might only talk about 3 here, and then in our examples sections employ these two other links.

\tstk{prop \ref{prop:M-MatrixEntries} needs contextual text around it}
\begin{prop}[$M$-matrix entries] \label{prop:M-MatrixEntries}
	Let $\graph=\bracs{\vertSet,\edgeSet}$ be an embedded graph on which the problem \eqref{eq:QGFullSystem} is posed.
	Suppose that $\dmap u = e_k$ where $e_k$ is the $k$\textsuperscript{th} canonical unit vector in $\complex^{\abs{\vertSet}}$.
	Then the $j$\textsuperscript{th} entry of $\nmap u$, and hence the $jk$\textsuperscript{th} entry in the $M$-matrix, is given by
	\begin{align*}
		\bracs{\nmap u}_j &= 
		\begin{cases}
			\!\begin{aligned}
				&0,
			\end{aligned}			
			& j \not\con k, \\
			\!\begin{aligned}
				&-\sum_{j\conLeft k} \omega e^{i\qm_{jk}l_{jk}} \csc\bracs{l_{jk}\omega} 
				\\ &\quad - \sum_{j\conRight k} \omega e^{-i\qm_{kj}l_{kj}} \csc\bracs{l_{kj}\omega},
			\end{aligned}
			& j\neq k, \ j\con k, \\
			\!\begin{aligned}
				&\sum_{j\con l} \omega\cot\bracs{l_{jl}\omega}
				\\ &\quad + 2\omega\sum_{j\conLeft j} \cot\bracs{l_{jj}\omega} - \cos\bracs{\qm_{jj}l_{jj}}\csc\bracs{l_{jj}\omega},
			\end{aligned}
			& j=k.
		\end{cases}
	\end{align*}
	Note the choice of $j\conLeft j$ in the contributions from loops is simply a convention, $j\conRight j$ is equivalent here.
	Also recall the convention for summing over $j\con k$:
	\begin{align*}
		\sum_{j\con k} \omega^2\cot\bracs{l_{jk}\omega} &= \sum_{j\conLeft k} \omega\cot\bracs{l_{jk}\omega}	+ \sum_{j\conRight k} \omega\cot\bracs{l_{kj}\omega}
	\end{align*}
\end{prop}
\begin{proof}
	The proof is an explicit computation, and follows the same idea as in \cite{ershova2014isospectrality} with adjustments for the fact that there are $\qm$ terms floating around. \newline
	
	We first determine the form of the edge solution $u_{jk}$ via solution of \eqref{eq:QGEdgeODEs} up to two complex constants per edge,
	\begin{align} \label{eq:EdgeEqnGeneralSolution}
		u_{jk} &= e^{-i\qm_{jk}t}\bracs{ C_{+}^{(jk)}e^{-i\omega t} + C_{-}^{(jk)}e^{i\omega t} }.
	\end{align}
	Since the $M$-matrix maps $\complex^{\abs{\vertSet}}\rightarrow\complex^{\abs{\vertSet}}$, it is sufficient to determine the $M$-matrix's action on the canonical basis of $\complex^{\abs{\vertSet}}$.
	So for each fixed $k\in\clbracs{1,...,\abs{\vertSet}}$ we set $\dmap u = e_k$.
	This provides us with sufficient Dirichlet data to solve \eqref{eq:QGEdgeODEs} on each edge and eliminate the constants $C_{+}^{(jk)}$, $C_{-}^{(jk)}$ in \eqref{eq:EdgeEqnGeneralSolution}, obtaining
	\begin{align*}
		j\not\con k &\implies
		\begin{cases}
			u_{jk}(t) = 0, \\
			u_{kj}(t) = 0,
		\end{cases} \\
		j\neq k, \ j\con k &\implies
		\begin{cases}
			u_{jk}(t) = e^{-i\qm_{jk}\bracs{t-l_{jk}}}\csc\bracs{\omega l_{jk}}\sin\bracs{\omega t}, \\
			u_{kj}(t) = -e^{-i\qm_{kj}\bracs{t-l_{kj}}}\csc\bracs{\omega l_{kj}}\sin\bracs{\omega t},
		\end{cases} \\
		j = k &\implies 
		\begin{cases}
			u_{jj}(t) = e^{-i\qm_{jj}t} \bracs{ e^{-i\omega t} + \sqbracs{e^{-i\qm_{jj}l_{jj}}-e^{-i\omega l_{jj}}}\csc\bracs{\omega l_{jj}}\sin\bracs{\omega t}  },
		\end{cases}
	\end{align*}
	This in turn enables us to explicitly differentiate the expressions for $u_{jk}$, and read off the values of $\bracs{\diff{}{t}+i\qm_{jk}}u_{jk}$ at the vertices.
	In the case $j\not\con k$, we obviously get zero contribution from the edges $I_{jk}$ and $I_{kj}$.
	The case $j\neq k, \ j\con k$, yields contributions from the edges $I_{jk}$ and $I_{kj}$ as follows:
	\begin{align*}
		\bracs{\diff{}{t}+i\qm_{jk}}u_{jk}\bracs{v_j} &= -\omega e^{-i\qm_{jk}l_{jk}}\csc\bracs{\omega l_{jk}}, \\
		\bracs{\diff{}{t}+i\qm_{jk}}u_{jk}\bracs{v_k} &= \omega\cot\bracs{\omega l_{jk}}, \\
		\bracs{\diff{}{t}+i\qm_{kj}}u_{kj}\bracs{v_j} &= -\omega e^{-i\qm_{kj}l_{kj}}\csc\bracs{\omega l_{kj}}, \\
		\bracs{\diff{}{t}+i\qm_{kj}}u_{kj}\bracs{v_k} &= \omega\cot\bracs{\omega l_{kj}}.
	\end{align*}
	And finally, when considering the case $j=k$ the contribution to $\bracs{\nmap u}_j$ from loops $I_{jj}$ in the graph also requires us to compute
	\begin{align*}
		\lim_{t\rightarrow0}-\bracs{u_{jj}'+i\qm_{jj}u_{jj}}(t) + \lim_{t\rightarrow l_{jj}}\bracs{u_{jj}'+i\qm_{jj}u_{jj}} & (t) \\
		&= 2\omega\bracs{ \cot\bracs{\omega l_{jj}} - \cos\bracs{\qm_{jj}l_{jj}}\csc\bracs{\omega l_{jj}}}.	
	\end{align*}
	We then use the formula
	\begin{align*}
		\bracs{\nmap u}_j &= \sum_{j\con l} \bracs{\diff{}{t}+i\qm_{jl}}u_{jl}\bracs{v_j},
	\end{align*}
	and our previous computations to compute each entry $\bracs{\nmap u}_j$, and obtain the desired result.
\end{proof}
Importantly this result demonstrates that the $M$-matrix can be thought of as a function of $\omega$ parametrised by $\qm$, hence will denote it by $M_{\qm}\bracs{\omega}$.

\section{Examples} \label{sec:Examples}

\subsection{A Periodic Cross in the Plane}
\tstk{linking first sentence}
Our first example is a two-dimensional graph whose period cell represents a lattice-like structure in $\reals^2$.
In this example we will use the $M$-matrix to determine an equation describing those $\omega$ which constitute the spectrum analytically, and complement this with some numerical computations to plot the resulting spectrum and density of states.

Consider the periodic graph defined as follows; for each $\bracs{n,m}\in\integers^2$ define
\begin{align*}
	& v_1^{\bracs{n,m}} = \bracs{\recip{2},0} + \bracs{n,m}, 
	v_2^{\bracs{n,m}} = \bracs{0,\recip{2}} + \bracs{n,m},
	v_3^{\bracs{n,m}} = \bracs{\recip{2},\recip{2}} + \bracs{n,m}, \\
	& I_{13}^{\bracs{n,m}} = \sqbracs{v_1^{\bracs{n,m}}, v_3^{\bracs{n,m}}},
	I_{23}^{\bracs{n,m}} = \sqbracs{v_2^{\bracs{n,m}}, v_3^{\bracs{n,m}}}, \\
	& I_{31}^{\bracs{n,m}} = \sqbracs{v_3^{\bracs{n,m}}, v_1^{\bracs{n+1,m}}},
	I_{32}^{\bracs{n,m}} = \sqbracs{v_3^{\bracs{n,m}}, v_2^{\bracs{n,m+1}}}.
\end{align*}
With 
\begin{align*}
	\vertSet^* = \clbracs{v_j^{\bracs{n,m}} \ \vert \ j\in\clbracs{1,2,3}, \bracs{n,m}\in\integers^2},
	\qquad \edgeSet^* = \clbracs{I_{jk}^{\bracs{n,m}} \ \vert \ j,k\in\clbracs{1,2,3}, \bracs{n,m}\in\integers^2},
\end{align*}
and setting coupling constants
\begin{align*}
	\alpha_3^{\bracs{n,m}} = \alpha \in\reals, 
	\qquad \alpha_j^{\bracs{n,m}} = 0, \quad j\in\clbracs{1,2,4,5},
\end{align*}
$\graph^* = \bracs{\vertSet^*,\edgeSet^*}$ is an embedded, periodic graph in $\reals^2$.
Its period graph occupies $\sqbracs{0,1}^2$ and can be visualised in figure \ref{fig:Diagram_TFRGraph}; consisting of 5 vertices and 4 edges.
\begin{figure}[t]
	\centering
	\begin{subfigure}[t]{0.45\textwidth}
		\centering
		\includegraphics[height=4.5cm]{Diagram_TFRGraph.pdf}
		\caption{\label{fig:Diagram_TFRGraph} The period graph that we are considering. All edges have length $\recip{2}$, and the quasi-momentum on horizontal edges is $-\qm_1$ and on vertical edges is $-\qm_2$.}
	\end{subfigure}
	~
	\begin{subfigure}[t]{0.45\textwidth}
		\centering
		\includegraphics[height=4.5cm]{Diagram_TFRQuantumGraph.pdf}
		\caption{\label{fig:Diagram_TFRQuantumGraph} The quantum graph that appears in our example in section \ref{ssec:ExampleCrossInPlane}. Due to the identification of vertices on the boundary of the period graph, we are effectively dealing with a 3-vertex quantum graph.}
	\end{subfigure}
	\caption{\label{fig:5VertexCross} (\ref{fig:Diagram_TFRGraph}) The period cell of the graph $\graph^*$. (\ref{fig:Diagram_TFRQuantumGraph}) The equivalent quantum graph on which we pose \eqref{eq:QGFullSystem}, retaining the lengths $l_{jk}$ and appropriate $\qm_{jk}$.}
\end{figure}
We next associate vertices that lie on the boundary of the period cell, matching $v_2$ with $v_4$ and $v_1$ with $v_5$, to obtain the quantum graph $\graph=\bracs{\vertSet,\edgeSet}$ with $\vertSet=\clbracs{v_1,v_2,v_3}$, $\edgeSet=\clbracs{I_{13},I_{23},I_{31},I_{32}}$, and lengths
\begin{align*}
	l_{13} = l_{23} = l_{31} = l_{32} = \recip{2}.
\end{align*}
Given that all the edges of $\graph^*$ are parallel to the co-ordinate axes, it is easy to compute the values of $\qm_{jk}$ for each $I_{jk}\in E$ and a given $\qm=\bracs{\qm_1,\qm_2}\in[-\pi,\pi)^2$:
\begin{align*}
	\qm_{13} = \qm_{31} = -\qm_2, &\quad \qm_{23} = \qm_{32} = -\qm_1.
\end{align*}
\tstk{$\kt$-results for curl-curl...}

\subsection{A Periodic ``Star" in the Plane} \label{ssec:9VertexStarGraph}
This is the example that Kirill said we should look at - required me to numerically solve for the DR.
\begin{figure}[t]
	\centering
	\begin{subfigure}[t]{0.45\textwidth}
		\centering
		\includegraphics[height=4.5cm]{Diagram_9VertexStarGraph.pdf}
		\caption{\label{fig:Diagram_9VertexStarGraph} The period graph considered in section \ref{ssec:9VertexStarGraph}. Diagonal edges have length $\recip{\sqrt{2}}$ with $\qm_{jk}$ being a linear combination of $\qm_1$ and $\qm_2$. Horizontal and vertical edges have length $\recip{2}$ and $\qm_{jk}$ only involves one of the components $\qm_1$ or $\qm_2$.}
	\end{subfigure}
	~
	\begin{subfigure}[t]{0.45\textwidth}
		\centering
		\includegraphics[height=4.5cm]{Diagram_9VertexStarQuantumGraph.pdf}
		\caption{\label{fig:Diagram_9VertexStarQuantumGraph} The quantum graph that appears in our example in section \ref{ssec:9VertexStarGraph}. Due to the identification of vertices on the boundary of the period graph, we are effectively dealing with a 4-vertex quantum graph.}
	\end{subfigure}
	\caption{\label{fig:9VertexStarGraph} (\ref{fig:Diagram_9VertexStarGraph}) The period cell of the graph considered in section \ref{ssec:9VertexStarGraph}. (\ref{fig:Diagram_9VertexStarQuantumGraph}) The equivalent quantum graph on which we pose \eqref{eq:QGFullSystem}, retaining the lengths $l_{jk}$ and appropriate $\qm_{jk}$.}
\end{figure}
For this graph, we find that
\begin{align*}
	H_{\qm}^{(1)} &=
	\begin{pmatrix}[1]
		\Lambda\cos\dfrac{\Lambda}{2}\sin\dfrac{\Lambda}{\sqrt{2}} &
		0 &
		-\Lambda\sin\dfrac{\Lambda}{\sqrt{2}}\cos\dfrac{\qm_1}{2} &
		0 \\
		0 &
		\Lambda\cos\dfrac{\Lambda}{2}\sin\dfrac{\Lambda}{\sqrt{2}} &
		-\Lambda\sin\dfrac{\Lambda}{\sqrt{2}}\cos\dfrac{\qm_2}{2} &
		0 \\
		-\Lambda\sin\dfrac{\Lambda}{\sqrt{2}}\cos\dfrac{\qm_1}{2} &
		-\Lambda\sin\dfrac{\Lambda}{\sqrt{2}}\cos\dfrac{\qm_2}{2} &
		2\Lambda\bracs{ \cos\dfrac{\Lambda}{2}\sin\dfrac{\Lambda}{\sqrt{2}} + \sin\dfrac{\Lambda}{2}\cos\dfrac{\Lambda}{\sqrt{2}} } &
		-2\Lambda\sin\dfrac{\Lambda}{2}\cos\dfrac{\qm_1}{2}\cos\dfrac{\qm_2}{2} \\
		0 &
		0 &
		-2\Lambda\sin\dfrac{\Lambda}{2}\cos\dfrac{\qm_1}{2}\cos\dfrac{\qm_2}{2} &
		2\Lambda\sin\dfrac{\Lambda}{2}\cos\dfrac{\Lambda}{\sqrt{2}}
	\end{pmatrix}, \\
	H_{\qm}^{(2)} &= 2\csc\dfrac{\Lambda}{2}\csc\dfrac{\Lambda}{\sqrt{2}}, \\
	A &= \mathrm{diag}\bracs{0,0,\alpha,0}.
\end{align*}
Note that we have labelled $v_5$ as $V_3$ in our quantum graph... hence the $\alpha$ in position $(3,3)$ in $A$.
Solving the usual $\det\bracs{H_{\qm}^{(1)} - \bracs{ H_{\qm}^{(2)} }^{-1}\omega^2 A}=0$ then provides us with
\begin{align*}
	0 &= \Lambda^3\sin\dfrac{\Lambda}{2}\cos\dfrac{\Lambda}{2}\sin^2\dfrac{\Lambda}{sqrt{2}},
\end{align*}
or
\begin{align*}
	0 &= -2\Lambda\sin\Lambda\cos^2\dfrac{\qm_1}{2}\cos^2\dfrac{\qm_2}{2} \\
	&\quad -\Lambda\sin\bracs{\Lambda\sqrt{2}}\bracs{\cos^2\dfrac{\qm_1}{2} + \cos^2\dfrac{\qm_2}{2}} \\
	&\quad + \Lambda\bracs{ \cos\Lambda\sin\bracs{\Lambda\sqrt{2}} + \sin\Lambda\cos\bracs{\Lambda\sqrt{2}} + \sin\Lambda + \sin\bracs{\Lambda\sqrt{2}} } \\
	&\quad - \frac{\alpha\omega^2}{4}\sin\Lambda\sin\bracs{\Lambda\sqrt{2}}.
\end{align*}
Here was can do the trick of solving for roots of a 2-d polynomial
\begin{align*}
	A v_1 v_2 + B (v_1 + v_2) + C = 0,
\end{align*}
with $v_1 = \cos^2\dfrac{\qm_1}{2}$, $v_2 = \cos^2\dfrac{\qm_2}{2}$ and $A,B,C$ being $\qm$-invariant.
This way, we can determine whether given $\omega, \wavenumber$ are spectral points by determining whether this multivariate polynomial has a root in $\sqbracs{0,1}^2$.

\subsection{A Periodic Diamond-Like Structure in the Plane} \label{ssec:5VertexDiamondGraph}
This is another example, where we have kept the same number of vertices but made our analysis more complicated by the addition of more edges.
\begin{figure}[t]
	\centering
	\begin{subfigure}[t]{0.45\textwidth}
		\centering
		\includegraphics[height=4.5cm]{Diagram_5VertexDiamondGraph.pdf}
		\caption{\label{fig:Diagram_5VertexDiamondGraph} The period graph considered in section \ref{ssec:9VertexStarGraph}. Diagonal edges have length $\recip{\sqrt{2}}$ with $\qm_{jk}$ being a linear combination of $\qm_1$ and $\qm_2$. Horizontal and vertical edges have length $\recip{2}$ and $\qm_{jk}$ only involves one of the components $\qm_1$ or $\qm_2$.}
	\end{subfigure}
	~
	\begin{subfigure}[t]{0.45\textwidth}
		\centering
		\includegraphics[height=4.5cm]{Diagram_5VertexDiamondQuantumGraph.pdf}
		\caption{\label{fig:Diagram_5VertexDiamondQuantumGraph} The quantum graph that appears in our example in section \ref{ssec:9VertexStarGraph}. Due to the identification of vertices on the boundary of the period graph, we are effectively dealing with a 4-vertex quantum graph.}
	\end{subfigure}
	\caption{\label{fig:5VertexDiamondGraph} (\ref{fig:Diagram_5VertexDiamondGraph}) The period cell of the graph considered in section \ref{ssec:5VertexDiamondGraph}. (\ref{fig:Diagram_5VertexDiamondQuantumGraph}) The equivalent quantum graph on which we pose \eqref{eq:QGFullSystem}, retaining the lengths $l_{jk}$ and appropriate $\qm_{jk}$.}
\end{figure}

$M$-matrix for this problem is
\begin{align*}
	H_{\qm}^{(1)} &= 
	\begin{pmatrix}[2]
		\Lambda\bracs{ \cos\dfrac{\Lambda}{2}\sin\dfrac{\Lambda}{\sqrt{2}} + 2\sin\dfrac{\Lambda}{2}\cos\dfrac{\Lambda}{\sqrt{2}} } &
		-2\Lambda\sin\dfrac{\Lambda}{2}\cos\dfrac{\qm_1}{2}\cos\dfrac{\qm_2}{2} &
		-\Lambda\sin\dfrac{\Lambda}{\sqrt{2}}\cos\dfrac{\qm_2}{2} \\
		-2\Lambda\sin\dfrac{\Lambda}{2}\cos\dfrac{\qm_1}{2}\cos\dfrac{\qm_2}{2} &
		\Lambda\bracs{ \cos\dfrac{\Lambda}{2}\sin\dfrac{\Lambda}{\sqrt{2}} + 2\sin\dfrac{\Lambda}{2}\cos\dfrac{\Lambda}{\sqrt{2}} } &
		-\Lambda\sin\dfrac{\Lambda}{\sqrt{2}}\cos\dfrac{\qm_1}{2} \\
		-\Lambda\sin\dfrac{\Lambda}{\sqrt{2}}\cos\dfrac{\qm_2}{2} &
		-\Lambda\sin\dfrac{\Lambda}{\sqrt{2}}\cos\dfrac{\qm_1}{2} &
		2\Lambda\cos\dfrac{\Lambda}{2}\sin\dfrac{\Lambda}{\sqrt{2}}
	\end{pmatrix}, \\
	H_{\qm}^{(2)} &= 2\csc\dfrac{\Lambda}{2}\csc\dfrac{\Lambda}{\sqrt{2}}, \\
	\bracs{ H_{\qm}^{(2)} }^{-1} &= \recip{2}\sin\dfrac{\Lambda}{2}\sin\dfrac{\Lambda}{\sqrt{2}}, \\
	A &= \mathrm{diag}\bracs{0, 0, \alpha}.
\end{align*}
Then we just solve $\det\bracs{H_{\qm}^{(1)} - \bracs{ H_{\qm}^{(2)} }^{-1}\omega^2 A}=0$ to get the eigenvalues - it's not particularly nice, ending up at
\begin{align*}
	0 = \Lambda^2\sin\dfrac{\Lambda}{\sqrt{2}},
\end{align*}
or
\begin{align*}
	0 &= 2\bracs{\alpha\omega^2\sin^2\dfrac{\Lambda}{2} - 2\Lambda\sin\dfrac{\Lambda}{\sqrt{2}} - 4\Lambda\sin\dfrac{\Lambda}{2}\cos\dfrac{\Lambda}{2} }\sin\dfrac{\Lambda}{2}\cos^2\dfrac{\qm_1}{2}\cos^2\dfrac{\qm_2}{2} \\
	&\quad - \Lambda\bracs{ 2\sin\dfrac{\Lambda}{2}\cos\dfrac{\Lambda}{\sqrt{2}} + \cos\dfrac{\Lambda}{2}\sin\dfrac{\Lambda}{\sqrt{2}} }\sin\dfrac{\Lambda}{\sqrt{2}} \bracs{\cos^2\dfrac{\qm_1}{2 + }\cos^2\dfrac{\qm_2}{2}} \\
	&\quad + \bracs{4\Lambda\cos\dfrac{\Lambda}{2} - \alpha\omega^2\sin\dfrac{\Lambda}{2}} \bracs{ \recip{2}\cos^2\dfrac{\Lambda}{2}\sin^2\dfrac{\Lambda}{\sqrt{2}} + 2\sin\dfrac{\Lambda}{2}\cos\dfrac{\Lambda}{2}\sin\dfrac{\Lambda}{\sqrt{2}}\cos\dfrac{\Lambda}{\sqrt{2}} + 2\sin^2\dfrac{\Lambda}{2}\cos^2\dfrac{\Lambda}{\sqrt{2}} }.
\end{align*}
\tstk{validate these before doing numerics, Will! Written in this form so that we can do the polynomial root-search trick in terms of $\cos^2\dfrac{\qm_1}{2}$ and $\cos^2\dfrac{\qm_2}{2}$ rather than trawling through values of $\omega, \wavenumber$ instead.}

\chapter{Conclusion} \label{ch:Conclusion}
In this chapter we review the content of this report and speculate on the future direction of research.
Section \ref{sec:ConcTheory} will provide an overview of the existing theory that we have utilised, the motivation for our project and restate our research objectives.
We will then summarise the work we have done to build on this theory and towards these objectives in section \ref{sec:ConcWork}.
Finally, in section \ref{sec:ConcFuture} we shall discuss some of the loose ends or open questions that have been raised by our research or not yet addressed, and provide some direction for future work.

\section{Summary of Motivation, Research Objectives, and Existing Theory} \label{sec:ConcTheory}
Our research is motivated by applications to PCFs (section \ref{sec:ProjectMotivation}), specifically in investigating the nature how spectral band-gaps emerge due as a result of the geometries of the fibres.
Although our research centres on singular-structure problems as a starting point (section \ref{sec:OurPhysicalSetup}), there is a link through quantum graph problems back to familiar thin-structure problems that are commonly used model PCFs (section \ref{sec:GraphLitReview}).
This amounts to us being able to view our singular-structure problems in an intuitive manner - as the limit of thin-structure problems as the thickness of the structures tends to zero, although the manner in which this thickness tends to zero also influences the problems that we should be considering (section \ref{sec:GraphLitReview}).
As such, the research goals that we set out to achieve were:
\begin{enumerate}
	\item To demonstrate that the singular-structure problems we consider give rise to equivalent quantum graph problems, in turn linking our singular-structure problems to formal ``limits" of thin-structure problems and hence PCF models.
	\item To study singular-structure problems that can be seen as approximations to PCFs; deriving the equivalent quantum graph problems and providing insight into how the geometry of the fibre cross section influences the spectral band-gaps of the fibre.
	\item To propose numerical approaches to determining these band-gaps in the event that an analytic approach proves unfeasible.
\end{enumerate}

As we discussed in sections \ref{sec:VariationalProblemLitReview}, choosing to study singular-structure problems required us to rethink the concepts of gradient, curl and divergence.
This bought us to the theory of chapter \ref{ch:ScalarEqns}, in which we presented a framework for posing variational problems with respect to Borel measures and reviewed the existing theory on the matter.
We also provided a geometric insight into what the concept of gradient meant in the context of our singular-structure problems, and demonstrated how we could obtain an equivalent quantum graph problem from a variational problem.
These arguments formed the basis of our understanding and direction for our work in chapter \ref{ch:VectorEqns}. \newline

Quantum graph problems are comparatively well studied, there even being a comprehensive introductory text on the subject and a recent spike in research interest (section \ref{sec:GraphLitReview}).
As such we simply needed to borrow the relevant concepts and tools from the existing works in the area for our own purposes, which we did in chapter \ref{ch:QuantumGraphs}.
Of particular importance was the M-matrix and it's utility in solving spectral problems on quantum graphs; and we briefly touched on how the M-matrix opens these spectral problems up to numerical schemes in chapter \ref{ch:ExampleSystems}, a topic which we revisit in section \ref{sec:ConcFuture}.
We also use quantum graphs as the link between our singular-structure problems and thin-structure problems that describe PCFs.

\section{Summary of Work and Examples} \label{sec:ConcWork}
The work of chapter \ref{ch:VectorEqns} saw us build on the existing work on variational problems by constructing the spaces $\ktgradSob{\ddom}{\ddmes}$, $\ktcurlSob{\ddom}{\ddmes}$ and $\ktcurlSobDivFree{\ddom}{\ddmes}$ and analysing the operator $\ktgrad$.
We provide an interpretation for the $\ddmes$-curl of a vector field (section \ref{sec:CurlExamples}) and also prove various properties about the elements of the aforementioned spaces.
In particular we deduce the form of the tangential $\kt$-curl and $\kt$-gradient, and characterise what it means to be divergence-free.
We also prove that the space $\ktcurlSob{\ddom}{\ddmes}$ has some inherent structural properties that are not obvious from it's construction (section \ref{sec:ktcurlSobExtraProperties}); and further conjecture that there are some additional features of this space which we have not yet uncovered.
This analysis allows us to consider the singular-structure analogue of the ``curl-of-the-curl" equation, written in \eqref{eq:CurlCurlEquationDivFree} and determine the equivalent quantum graph problem in section \ref{sec:CurlReductionToQG}, forming the basis for our examples in chapter \ref{ch:ExampleSystems}.
This theory is essential if we are to address the first of our research objectives, as without knowledge of the form of objects like $\ktcurl{\ddmes}u$, we have no hope of obtaining an equivalent quantum graph problem from our singular-structure problems.
The fact that we can write problems like \eqref{eq:CurlCurlEquationDivFree} in the way they are presented is also satisfying in an intuitive sense; the ``equations" that we are studying have the same form as those which we would consider in the thin-structure setting, only now we have a different understanding of gradients (and curls, and divergences).
Our development of this theory will likely prove valuable if we are to move on from the ``curl-of-the-curl" equation to a full Maxwell problem involving coupled $\mathbf{E}$ and $\mathbf{H}$ fields, which we revisit in section \ref{sec:ConcFuture}. \newline

Having derived the equivalent quantum graph problem from our singular-structure problems, we spend chapter \ref{ch:ExampleSystems} looking over some examples and addressing the second and third objectives.
Our examples demonstrate how to construct the M-matrix, and how it can be used either analytically or numerically to solve spectral problems and hence reveal insights about band-gaps.
We take a mixture of numerical and analytic approaches in these examples, however stop short of a fully-fledged numerical scheme that begins from the M-matrix itself.
This is discussed in section \ref{sec:NumericalMethodsDiscussion} however, and will be revisited in section \ref{sec:ConcFuture}, which follows.
The examples also illustrate how it is possible to use the geometry of the cross-sectional structure to open band-gaps in the spectrum (section \ref{sec:ExampleGeneralLengths}).
However our investigation is limited to a single, simplified case and analytic progress proves hard, again highlighting that (at least for practical purposes) a numerical scheme that focuses on a physically relevant part of the spectrum may be more useful and applicable.
We provide a final example in section \ref{sec:ExampleThickVertex} that retains the geometry of the example in section \ref{sec:ExampleCrossInPlane}, but with a non-zero coupling constant at the central vertex.
The corresponding singular-structure problem corresponds to a different scaling limit (of thin-structure problems, section \ref{sec:GraphLitReview}) than the previous examples, and in this case we demonstrate that band-gaps are opened simply by the presence of this coupling constant.
This in turn would suggest that thin-structures that adhere to this scaling between ``edge"- and ``vertex"-regions are more likely to give rise to band-gaps, however more work needs to be done beyond this example.
We make this suggestion because the example demonstrates that the addition of a non-zero coupling constant can lead to the opening of band-gaps, however this may again be an artefact of the geometry of the problem rather than a general principle.
Another thing to be noted is that the M-matrix can be recycled from the first example, and the only difference in our solution method being that we consider a slightly different generalised eigenvalue problem.
This is potentially useful for any numerical schemes - we only need construct (a function that evaluates) the M-matrix once for a given geometry (and set of governing equations).
If in addition we can produce results like proposition \ref{prop:M-MatrixEntries} for each quantum graph problem, there is the potential to further cut the complexity of such numerical constructions. \newline

Whilst our examples help us explore the second and third objectives, and do provide us with some intuition about what to expect, they do not provide us with any general insights yet.
This, alongside some of the considerations for a numerical scheme, are discussed in section \ref{sec:ConcFuture}.

\section{Further Developments} \label{sec:ConcFuture}
The work that has been carried out thus far makes progress towards the research objectives that were set out in section \ref{sec:ReportOverview}, but stops short of providing definitive answers in places.
In this section we examine some of these loose ends, and the direction of future work that could be undertaken to address them.
We cover issues surrounding a numerical scheme for solving our singular-structure problems in section \ref{sec:ConcFutureNumerical}; how we might look to qualify the dependence of the graph geometry on the spectrum of our problems in section \ref{sec:ConcFutureGeometry}, and discuss how we might make progress onto modelling electromagnetic wave-guidance through Maxwell's equations in section \ref{sec:ConcFutureMaxwell}.
Once we have explored these issues, we will conclude with section \ref{sec:ConcClosingRemarks}.

\subsection{Considerations for Numerical Schemes} \label{sec:ConcFutureNumerical}
In section \ref{sec:NumericalMethodsDiscussion} we discussed the possibility of using the M-matrix to explore the spectrum of quantum graph (hence our singular-structure) problems numerically.
Here we review what was said and elaborate on how such an approach might be developed, tested and analysed.
To make the discussion as general as possible; in this section we assume that we have some family of quantum graph problems $\mathcal{P}_{\qm}$, with spectral parameter $\lambda$ and spectra $\sigma\bracs{\mathcal{P}_{\qm}}$, each having an M-matrix $M_{\qm}\bracs{\lambda}$.
This family $\mathcal{P}_{\qm}$ is the result of taking a Gelfand transform of a periodic quantum graph problem $\mathcal{P}$ (with spectrum $\sigma\bracs{\mathcal{P}}$), that is equivalent to some singular-structure problem that we are concerned with.
Here we discuss a numerical scheme that is capable of being told the problems $\mathcal{P}_{\qm}$ and producing an approximation to $\sigma\bracs{\mathcal{P}}$, whose outline is along the lines of the following;
\begin{enumerate}
	\item Consider the spectral problem $\mathcal{P}_{\qm}$ as a generalised eigenvalue problem
	\begin{align*}
		M_{\qm}\bracs{\lambda} v &= 0,
	\end{align*}
	involving the M-matrix.
	\item Determine a method for constructing $M_{\qm}\bracs{\lambda}$.
	\item Solve the generalised eigenvalue problem $\mathcal{P}_{\qm}$ for $\sigma\bracs{\mathcal{P}_{\qm}}$, and hence construct an approximation to $\sigma\bracs{\mathcal{P}}$.
\end{enumerate}
Broadly speaking; the issues surround how to construct the M-matrix efficiently and quickly, and the optimal method of determining $\sigma\bracs{\mathcal{P}_{\qm}}$ and hence $\sigma\bracs{\mathcal{P}}$.
We discuss each of these below, along with how they might be addressed. \newline

\subsubsection{Construction of the M-Matrix} \label{sec:ConcFutureConstructM}
Any numerical scheme will require a method for constructing the M-matrix, because the solver for the generalised eigenvalue problem will need the ability to evaluate the M-matrix.
Na\"ively we can construct the M-matrix by following the constructive proof of proposition \ref{prop:M-MatrixEntries} numerically, for each value of $\lambda$ that we are required to evaluate $M_{\qm}$ at.
This involves solving each edge-ODE in $\mathcal{P}_{\qm}$ numerically, and once for each column of the M-matrix (although this is a large overestimate and can be reduced - see section \ref{sec:NumericalMethodsDiscussion}), reading off the approximate Neumann data for the edge solution, and then summing the appropriate combination of derivatives at the vertices.
Needless to say this will be an expensive process for a large (in the sense of number of edges) graph, given that it needs to be done each time the M-matrix needs to be evaluated.
Having said this we should also note that this method is relatively simple to program and, provided there was sufficient care in solving the edge-ODEs of $\mathcal{P}_{\qm}$, would provide access to the M-matrix. \newline

We can make constructing the M-matrix cheaper (in computational terms) by employing one of the tactics in chapter \ref{ch:ExampleSystems} - working analytically to a suitable point and then proceeding numerically when the expressions become too complex.
The obvious candidate for a stopping point for each $\mathcal{P}_{\qm}$ would be the analogue of proposition \ref{prop:M-MatrixEntries}, as this bypasses the need to solve each edge-ODE every time the M-matrix needs to be evaluated (and even provides the M-matrix as a function of $\lambda=\omega^2$ and the quasi-momentum $\qm$).
In fact this result reduces the construction of the M-matrix to a case of looking up geometric properties of the underlying graph and evaluating trigonometric functions.
Of course the exchange we make for this simpler construction of $M_{\qm}$ is that we loose generality in our numerical scheme; proposition \ref{prop:M-MatrixEntries} only holds for the specific set of equations we chose to examine in chapter \ref{ch:ExampleSystems}, and so we would have to prove an analogue of proposition \ref{prop:M-MatrixEntries} for each quantum graph problem we want to consider. \newline

At present, computer code is being developed to construct the M-matrix for the set of equations \eqref{eq:QGEquation} using proposition \ref{prop:M-MatrixEntries}, and we will also be looking to write code for the more general approach that relies on solving the edge-ODEs directly.
However the bottom line of this issue is that future work needs to be done looking into the computational cost and accuracy of both approaches.
For the purely numerical approach the direction of research is fairly clear; we should look at existing theory in this area that surrounds the ODE solvers that we would be employing, and couple this with the algorithm that we end up proposing to construct the M-matrix.
The alternative approach requires slightly different treatment, as although in theory one will obtain exact expressions for the elements of the M-matrix, we do not yet know how easy it will be to obtain an analogue of proposition \ref{prop:M-MatrixEntries} when the edge-ODEs of $\mathcal{P}_{\qm}$ change.
Indeed it may not even be possible to obtain such a result, or the end-user of the numerical scheme may not want to spend time deriving it.
Assuming we have such a result however, we can do some basic analysis on the computational cost of assembling the M-matrix using this approach and then compare this with the alternative approach.
Although we expect the latter (analytic entries) approach to be more accurate and faster in all situations, the extent of these gains might be considered too small to warrant the derivation of an analogue of proposition \ref{prop:M-MatrixEntries}.

\subsubsection{Determination of $\sigma\bracs{\mathcal{P}}$} \label{sec:ConcFutureGetSpectrum}
The other consideration for our numerical scheme are the nuances that come with solving the generalised eigenvalue problems, and how we construct an approximation to $\sigma\bracs{\mathcal{P}}$ from the values we get for $\sigma\bracs{\mathcal{P}_{\qm}}$.
The foremost problems with the latter is that we cannot take the union of each of the spectra $\sigma\bracs{\mathcal{P}_{\qm}}$ over $\qm$ as we did analytically - we will be forced to (at best) use a fine mesh of discrete $\qm$ values to build up an approximation to $\sigma\bracs{\mathcal{P}}$.
One glaring issue is whether we can quantify how fine a mesh in $\qm$ is required, or whether there are certain values of $\qm$ that are of significance to the spectrum (values that correspond to eigenvalues found at the ends of band-gaps, for example).
There may also be symmetries that we can exploit on a problem-by-problem basis; like in the example in section \ref{sec:ExampleCrossInPlane} where there was symmetry in the components of $\qm$ and hence we could set $\qm_2=0$ to determine the spectrum.
However if we expect some kind of stability of $\sigma\bracs{\mathcal{P}_{\qm}}$ with respect to $\qm$ (that is, we can show that small changes in $\qm$ correspond to some kind of small changes in $\sigma\bracs{\mathcal{P}_{\qm}}$) then the idea of meshing $\qm$ and solving a finite number of the $\mathcal{P}_{\qm}$ isn't a terrible one.
Given the lack of immediate alternative suggestions, this kind of analysis is the direction to take for future work on this issue. \newline

The former problem mentioned above, the nuances that come with solving the generalised eigenvalue problems, are another concern.
In particular we have to deal with the complication that there may be (and in our case seems to always be) an infinite number of solutions $\bracs{\lambda,v}$ to $M_{\qm}\bracs{\lambda} v = 0$.
Some knowledge of the form of the M-matrix (like proposition \ref{prop:M-MatrixEntries}) is helpful in this regard, for example we know that if the M-matrix is periodic in $\lambda$ then it is sufficient to determine all the unique eigenvalues over one period and from there can construct the remainder.
However even if the M-matrix is periodic the presence of coupling constants on the vertices can make this potential advantage redundant, as can be seen for the M-matrix of section \ref{sec:ExampleThickVertex}.
If a numerical scheme is being used solely from a fabrication/design perspective, then there is always the option to restrict the solver to finding eigenvalues $\lambda$ within a given range of interest, such as the operating frequencies of a PCF in the electromagnetic setting.
We should also not forget that, regardless of the computational power available to us, we can never determine the full spectrum computationally either (we require some analytic techniques for this) and so this compromise is likely one of the best we can provide.
This in turn raises a further question - given a particular range of interest for $\lambda$, how can be we be sure to find every eigenvalue in $\sigma\bracs{\mathcal{P}_{\qm}}$ that lies in this range?
An examination of existing solver methods for generalised eigenvalue problems would go some way to answering this question, alongside whether we can deduce (analytically) anything about the distribution of the eigenvalues for a given $\mathcal{P}_{\qm}$.

\subsection{Effect of Geometry on Spectra} \label{sec:ConcFutureGeometry}
One of our objectives was to attempt to describe how the underlying geometry of the singular-structure affects the resulting spectrum and hand-gaps that the structure exhibits, and some progress has been made with this through the theory of chapter \ref{ch:VectorEqns}.
Namely we see that $\qm$ undergoes rotations dependant on the orientations of the underlying graph, and the quantum graph problem that we obtain has solutions dependant on the lengths of the singular-structure edges.
And although we have not provided the details it is also known that the relative scaling of the vertex- and edge-regions in the thin-structures we are approximating gives rise to different quantum graph problems and hence different spectra, as we illustrated in the examples of section \ref{sec:ExampleCrossInPlane} and \ref{sec:ExampleThickVertex}.
By affecting the coupling constants (through the scaling of the thin-structure), quasi-momentum and incorporating the edge-lengths, the spectrum and hence band-gaps of the singular-structure problem are also changed.
This being said, we have not drawn any general insights into how these effects change the resulting spectra, only provided examples in chapter \ref{ch:ExampleSystems} to demonstrate that they do.
We can make one conjecture on this topic though; that a graph with zero coupling constants and (geometric) symmetry in the periodic directions will not give rise to band-gaps.
This idea is reinforced by the examples of section \ref{sec:ExampleCrossInPlane} and \ref{sec:ExampleGeneralLengths}, as well as other examples with geometric symmetries that we have examined but don't include in this report. \newline

Another consideration that goes beyond what we have done in this report is looking at geometries whose peroid-cells are not rectangular in shape.
This direction of future work is motivated more by the desire to produce a model that is relevant to physical PCFs and wave-guidance problems, rather than mathematical interest or completeness.
In particular PCFs are typically fabricated with hexagonal lattice-structures, which for us would result in a hexagonal unit cell for our infinite, periodic singular-structure.
The affect is largely felt by the Gelfand transform as we no longer have a set of orthogonal vectors that describe the translation-invariance of the singular-structure, which affects how we deal with ``mismatches" of periodic solutions at the boundary of the period cell.
We expect the analysis we have carried out in chapter \ref{ch:VectorEqns} to still be of relevance in this slightly altered case; in particular we have demonstrated that curls and gradients of zero are invariant with respect to the quasi-momentum, and so we expect that these objects will not change in these new period cell shapes.
Regardless, there is the option to investigate any changes that the shape of the period cell itself induces in the problems that we obtain, and the work in chapter \ref{ch:VectorEqns} can be used as a basis for this research.

\subsection{Generalisations to our Singular-Structure Systems} \label{sec:ConcFutureMaxwell}
We focused our attention on the ``curl-of-the-curl" equation \eqref{eq:CurlCurlEquationDivFree} in chapters \ref{ch:VectorEqns} and \ref{ch:ExampleSystems} because it arises from the time-harmonic Maxwell system which describes electromagnetic wave propagation.
However the ``curl-of-the-curl" equation can only be derived if we assume a time-harmonic solution to the full Maxwell system, namely assume that the $\mathbf{E}$ and $\mathbf{H}$ fields are of the form
\begin{align*}
	\mathbf{E}\bracs{x_1,x_2,x_3} = \widehat{\mathbf{E}}\bracs{x_1,x_2,x_3}e^{-i\omega t}, 
	&\quad  \mathbf{H}\bracs{x_1,x_2,x_3} = \widehat{\mathbf{H}}\bracs{x_1,x_2,x_3}e^{-i\omega t}.
\end{align*}
Substituting this ansatz into the system of Maxwell equations \eqref{eq:MaxwellSystem} and taking the curl of either equation and substituting the result into the other then provides the curl of the curl equation.
Whilst there is probably little harm in using the ``curl-of-the-curl" equation as our starting point for out singular-structure problems; for the purposes of providing a complete description of electromagnetic wave-guidance on singular-structures it would be good to derive an equivalent quantum graph problem for the system of Maxwell equations.
One could then use the quantum graph problem derived in chapter \ref{ch:VectorEqns} as a check for consistency in the system that was obtained.
No major obstacles are expected from leaving the time-dependence in the system, however there will need to be some care in how to setup the various function spaces that we are working with.
However because the time co-ordinate is essentially separate from the spatial co-ordinates, we should again be able to recycle the work in chapter \ref{ch:VectorEqns} here. \newline

Another possible generalisation that could be made to our systems concerns the ``empty space" (the cores if we adopt the language of PCFs) in our cross-sectional structure.
Currently our singular-structure problems effectively ignore this part of our domain, which lends itself to the description that there is no field $u$ in that part of the domain (or we don't care what it is).
In the context of electromagnetism this would likely mean that this ``empty space" was actually filled with a metallic material, and so we wouldn't expect an $\mathbf{E}$- or $\mathbf{H}$-field in this region.
In reality this may not be the case (although metallic mode confinement in PCFs has been demonstrated, see \cite{hou2008metallic}) as fibres are typically fabricated using two dielectric materials (one of which may be vacuum for PCFs).
As such we would need to consider the system of Maxwell equations \eqref{eq:MaxwellSystem} both on the singular-structure and in the remainder of the domain, and the electric permittivity $\eps_{P}$ and magnetic permeability $\mu_{P}$ would no longer be constant across the whole domain.
This opens up several avenues for exploration - foremost being how to correctly formulate Maxwell's equations in this instance.
We would be required to keep the variational approach we have adopted to deal with the singular-structure correctly, and so the natural avenue of investigation would be to adapt the measure that we pose our variational problem with respect to.
The foremost candidate being a ``Lebesgue plus singular graph" measure, namely a Borel measure
\begin{align*}
	\dddmes\bracs{B} &= \lambda_{2}\bracs{B} + \ddmes\bracs{B}, \quad B\in\mathcal{B}_{\ddom},
\end{align*}
where we now account for the singular structure using our singular measure on the underlying graph $\ddmes$ as before, but also add the standard 2D Lebesgue measure $\lambda_2$ so that we no longer discard the larger regions.
Using a variational problem posed with respect to $\dddmes$ will allow us to deal with the issue of boundary conditions between the singular structure and surrounding dielectric, but whether we can determine a method to solve such problems (as we did by borrowing theory from quantum graphs) remains open.
There would also be similar questions raised about whether such a variational problem could still be thought of as the limit of some thin-structure problem, like with the singular-structure problems we have considered through this report thus far.

\section{Closing Remarks} \label{sec:ConcClosingRemarks}
The work in this report has made progress towards addressing the research objectives that it set out to achieve, as laid out in chapter \ref{ch:Intro}.
We look to consider singular-structure domains as setup in section \ref{sec:OurPhysicalSetup}, motivated by existing theory for variational problems (section \ref{sec:VariationalProblemLitReview}, chapter \ref{ch:ScalarEqns}) and seeking a consistent framework for our approximation to physical waveguides.
We develop this theory of variational problems in the context of our singular-structure problems in chapter \ref{ch:VectorEqns}; describing what the concepts of gradient, curl and divergence-free mean and hence building appropriate function spaces for our problems.
A link to quantum graph problems from our singular-structure problems is established (section \ref{sec:CurlReductionToQG}), meaning we can bring in existing theory from quantum graph problems (section \ref{sec:GraphLitReview}, chapter \ref{ch:QuantumGraphs}) to aid in the solution to our original singular-structure problems, and open up the potential for numerical approaches to be used (section \ref{sec:NumericalMethodsDiscussion}).
The link to quantum-graph problems also solidifies our singular-structure problems as formal limits of more familiar thin-structure models for waveguides (section \ref{sec:GraphLitReview}), addressing the first of the research objectives.
The examples of chapter \ref{ch:ExampleSystems} serves the purpose of bringing together the theory of the previous chapters and highlighting important considerations for solving our singular-structure problems.
We are able to demonstrate analytically that certain geometries (underlying graphs) give rise to band-gap spectra whilst others do not, and discuss how a numerical scheme that is designed to solve such problems might be employed (section \ref{sec:NumericalMethodsDiscussion}).
These examples also go some way to addressing research objectives two and three, although as we highlight in section \ref{sec:ConcFuture} there are further considerations and questions that need answering.
Section \ref{sec:ConcFutureMaxwell} also highlights that we have not yet fully addressed the first research objective, at least in the context of PCFs (electromagnetic wave-guidance) which is another direction we should pursue. \newline

In summary, the work presented in this report serves as a strong foundation for future work on the objectives in chapter \ref{ch:Intro}.
The existing theory gathered, and that which has been developed, will allow further investigation into (limits of) more descriptive wave-guidance problems - addressing objective one.
The examples that have been analysed provide some underlying intuition about what we should expect the answers to objective two should be, and highlight the issues that a numerical scheme sought by objective three will need to consider.
We have discussed the direction of future work that will be taken with each of these objectives in mind throughout section \ref{sec:ConcFuture}, and will look to pursue these directions in the immediate future.

\newpage %have a new page to start the bibliography
\bibliographystyle{acm}
\bibliography{../BibFiles/BTandSS.bib,../BibFiles/MiscRefs.bib,../BibFiles/PCFsAndEM.bib,../BibFiles/QuantumGraphs.bib}

%\newpage %new page to clear space between bibliography and the appendix, as \section doesn't call \newpage

\appendix %this informs LaTeX that the appendix starts here, so section numbering should now be alphabetical rather than numerical

\section{Appendix: Singular Measures} \label{app:SingularMeasures}
Embedded graphs (section \ref{ssec:EmbeddedGraphs}) allow us to introduce the measure that links quantum graphs to the measure-theoretic formulation we introduced in section \ref{ssec:OurSystem}.
For an embedded graph $\graph = \bracs{\vertSet,\edgeSet}$ and for each $I_{jk}\in \edgeSet$, define the (Borel) measure $\lambda_{jk}$ as the measure that supports the one-dimensional Lebesgue measure on $I_{jk}$:
\begin{align*}
	\lambda_{jk}\bracs{B} = \lambda_{1}\bracs{r_{jk}^{-1}\bracs{B \cap I_{jk}}},
	&\quad\text{for all Borel } B,
\end{align*}
where $\lambda_1$ is the Lebesgue measure on $\reals$, and $r_{jk}$ is the parametrisation of the edge $I_{jk}$ (see \eqref{eq:GeneralCurveParam}).
Then set $\ddmes$ to be the (Borel) measure defined by
\begin{align*}
	\ddmes\bracs{B} = \sum_{v_j\in \vertSet}\sum_{j\conLeft k} \lambda_{jk}\bracs{B}.
\end{align*}
We refer to $\ddmes$ as the ``singular measure that supports $\graph$"; or alternatively the ``singular measure on $\graph$", or the ``(singular) measure that supports the edges of $\graph$".
For a graph embedded into a 2D domain, the singular measure $\ddmes$ is illustrated in figure \ref{fig:Diagram_SingularMeasure2D}.
\begin{figure}[b!]
	\centering
	\includegraphics[scale=0.85]{Diagram_SingularMeasure2D.pdf}
	\caption{\label{fig:Diagram_SingularMeasure2D} For a graph embedded in $\reals^2$, the $\ddmes$-measure of any Borel set $B$ is obtained from summing the contributions of each $\lambda_{jk}$, as indicated by the thickened lines.
	Sets that do not intersect $\graph$ have zero measure.}
\end{figure}

Now for each $v_j\in\vertSet$ let $\delta_j$ be a point-mass measure centred on $v_j$, namely
\begin{align*}
	\delta_j\bracs{B} &= \begin{cases} 1, & v_j\in B, \\ 0, & v_j\not\in B, \end{cases}
\end{align*}
and let
\begin{align*}
	\nu\bracs{B} = \sum_{v_j\in\vertSet}\alpha_j\delta_j\bracs{B},
	\quad\text{for all Borel } B,
\end{align*}
where each $\alpha_j$ is the coupling constant at the vertex $v_j$.
Finally, consider the measure $\dddmes = \ddmes + \nu$.
The measures $\dddmes$, $\ddmes$, and $\nu$ will be key to our measure-theoretic formulations, enabling us to establish notions of derivatives on a domains with no interior (or no ``area" in the Lebesgue-sense).

\section{Appendix: Key Measure Theoretic Concepts} \label{app:MeasureTheory}
In this section we address the question of how one understands the equations \eqref{eq:WholeSpaceLaplaceEqn} and \eqref{eq:PeriodCellLaplaceStrongForm}, by introducing the relevant differential operators and function spaces.
Attempting to pose a boundary-value problem on a singular-structure, by drawing analogy to the ``ingredients" of a boundary-value problem on a thin structure, runs into problems.
These are due to the singular structure lacking a domain interior from the perspective of the space it is embedded in, so the notion of boundary values ceases to make sense.
This issue is resolved by not abandoning what we believe are the ingredients of a boundary-value problem, but rather by reworking our concepts of integration and differentiation so that they respect the fact that we are looking at a problem on the singular structure itself.
As we will be working with each of the measures $\dddmes, \ddmes$, and $\nu$ above individually before combining our knowledge of each, what we present here is stated in terms of a generic measure $\rho$.
Our approach below is motivated by \cite{zhikov2000extension} and \cite{zhikov2002homogenization}.

First, let us introduce the notation for this section.
Suppose $\rho$ is a Borel measure on 
\begin{align*}
	\ddom:= \left[0,T_1\right)\times\left[0,T_2\right), 
\end{align*}
which can be thought of as the period cell of an embedded graph $\graph$.
For 
\begin{align*}
	\qm=\bracs{\qm_1,\qm_2}\in\left[-\frac{\pi}{T_1},\frac{\pi}{T_1}\right)\times\left[-\frac{\pi}{T_2},\frac{\pi}{T_2}\right)
\end{align*}
define the ``shifted" gradient operator $\tgrad$ on smooth functions $\phi\in\smooth{\ddom}$ by
\begin{align*}
	\tgrad\phi &= \begin{pmatrix} \partial_1\phi + \rmi\qm_1\phi \\ \partial_2\phi + \rmi\qm_2\phi \end{pmatrix}.
\end{align*}
Recall that we pose \eqref{eq:WholeSpaceLaplaceEqn} on a periodic, embedded graph $\graph$ in $\reals^2$, and the operator $\tgrad$ arises after using a Gelfand transform, to move us to a family of problems on the period cell of $\graph$.

Denote the set of smooth functions on $\ddom$ by $\smooth{\ddom}$, and let
\begin{align*} %\label{eq:WSetDefinition}
	W = W^{\qm}\bracs{\ddom,\rho} &:= \overline{\clbracs{\bracs{\phi,\tgrad\phi} \setVert \phi\in\smooth{\ddom}}} \quad \text{in} \ \ltwo{\ddom}{\rho}\times\ltwo{D}{\rho}^2.
\end{align*}
The idea here is to construct an analogy of a Sobolev space for the measure $\rho$, and hence obtain a concept of (weak) derivative.
Notice that if $\bracs{u,g_1},\bracs{0,g_2}\in W$ then clearly $\bracs{u, g_1+g_2}\in W$ too.
We define the ``set of $\rho$-gradients of zero" as
\begin{align} \label{eq:GradZeroDef}
	\gradZero{\ddom}{\rho} &= \clbracs{ g\in\ltwo{\ddom}{\rho}^2 \setVert \bracs{0,g}\in W}, \\
	&= \clbracs{ g\in\ltwo{\ddom}{\rho}^2 \setVert \exists\phi_n\in\smooth{\ddom} \text{ such that } \phi_n\lconv{\ltwo{\ddom}{\rho}} 0, \tgrad\phi_n\lconv{\ltwo{\ddom}{\rho}^2} g }
\end{align}
which is a closed, linear subspace of $\ltwo{D}{\rho}^2$. 
It can be shown that $\gradZero{D}{\rho}$ does not depend on the value of $\qm$, which is why the notation lacks a $\qm$ symbol:
\begin{prop} \label{prop:GradZeroInvarientUnderQM}
	For any fixed $\qm\in[-\pi,\pi)^2$, and with
	\begin{align*}
		W^0 &= \overline{\clbracs{\bracs{\phi,\grad\phi} \setVert \phi\in\smooth{\ddom}}} \quad \mathrm{in} \ \ltwo{\ddom}{\rho}\times\ltwo{\ddom}{\rho}^2,
	\end{align*}	
	we have
	\begin{align*}
		\mathcal{G}^0_{\ddom, \md \rho} := \clbracs{ g \ \vert \ \bracs{0,g}\in W^0} &= 
		\clbracs{ g \ \vert \ \bracs{0,g}\in W} =: \mathcal{G}^{\qm}_{\ddom, \md \rho}.
	\end{align*}
\end{prop}
\begin{proof}
	Given that $\tgrad\phi = \grad\phi + \rmi\qm\phi$ for smooth $\phi$, if $g\in\mathcal{G}^0_{\ddom, \md \rho}$ there exists a sequence of smooth functions $\phi_n$ such that 
	\begin{align*}
		\phi_n\lconv{\ltwo{\ddom}{\rho}}0, &\quad \grad\phi_n\lconv{\ltwo{\ddom}{\rho}^2}g.
	\end{align*}
	But then clearly
	\begin{align*}
		\phi_n\lconv{\ltwo{\ddom}{\rho}}0, &\quad
		\tgrad\phi_n = \grad\phi_n + \rmi\qm\phi_n \lconv{\ltwo{\ddom}{\rho}^2} g + 0 = g,
	\end{align*}
	so $g\in\mathcal{G}^\qm_{\ddom, \md \rho}$.
	The proof of the opposite inclusion is similar.
\end{proof}
The illustration of the non-uniqueness of gradients earlier employed the fact that we can always add an element of $\gradZero{\ddom}{\rho}$ to the second member of a pair $\bracs{u,z}\in W$ and produce another element of $W$.

For each $u\in\ltwo{\ddom}{\rho}$ there exists a $\tgrad_\rho u\in\gradZero{\ddom}{\rho}^{\perp}$ such that any pair $\bracs{u,z}\in W$ can be written as $\bracs{u,\tgrad_\rho u + g}$ where $g\in\gradZero{\ddom}{\rho}$.
We call $\tgrad_\rho u$ the \emph{($\rho$-)tangential gradient} of $u$ and it is unique in this sense (see \cite[Section~9]{zhikov2000extension}); which allows us to construct the ``Sobolev space"
\begin{align*}
	\gradSobQM{\ddom}{\rho} &= \clbracs{ \bracs{u, \tgrad_\rho u}\in W \ \vert \ \tgrad_\rho u \in \gradZero{\ddom}{\rho}^{\perp} }.
\end{align*}
Since $\grad_\rho u$ is unique, we will use the shorthand $u\in\gradSobQM{\ddom}{\rho}$ to refer to the pair $\bracs{u, \grad_\rho u}$.
We can now precisely state what we mean when we write equation \eqref{eq:PeriodCellLaplaceStrongForm}; a pair $u\in\gradSobQM{\ddom}{\dddmes}$ solves \eqref{eq:PeriodCellLaplaceStrongForm} if and only if (see \eqref{eq:PeriodCellLaplaceWeakForm})
\begin{align*}
	\integral{\ddom}{\tgrad_{\dddmes}u\cdot\overline{\tgrad_{\dddmes}\phi}}{\dddmes} &= \omega^2\integral{\ddom}{u\overline{\phi}}{\dddmes}, \quad\forall \phi\in\smooth{\ddom},
\end{align*}
and one has a similar interpretation for \eqref{eq:WholeSpaceLaplaceEqn}.
We will continue to use \eqref{eq:PeriodCellLaplaceStrongForm} as shorthand for \eqref{eq:PeriodCellLaplaceWeakForm} to save on notational clutter and maintain readability, however the steps of our derivation of \eqref{eq:QGFullSystem} from \eqref{eq:PeriodCellLaplaceStrongForm} will require us to work with \eqref{eq:PeriodCellLaplaceWeakForm} directly.

In order to demonstrate how \eqref{eq:PeriodCellLaplaceStrongForm} reduces to the system \eqref{eq:QGFullSystem}, we must understand the properties of the functions (and their gradients) in the space $\gradSobQM{\ddom}{\ddmes}$, which is the focus of sections \ref{app:muAnalysis} through \ref{app:SumMeasureAnalysis}.
This begins by examining $\gradSobQM{\ddom}{\ddmes}$, which will then determine the behaviour of functions on the singular-structure.
The space $\gradSobQM{\ddom}{\nu}$ is examined next, before concluding with section \ref{app:SumMeasureAnalysis} in which we describe the functions in $\gradSobQM{\ddom}{\dddmes}$.

\tstk{this is still written as an ``overview" section, the full proofs are not included. You should probably put them in Will, or have a final section of the appendix which is just the details of the proofs, and leave this one with some narrative.}

\section{Appendix: Analysis of the Behaviour of Sobolev Functions on the (Straight) Edges of an Embedded Graph} \label{app:muAnalysis}
Crucial to our understanding of the functions (and their gradients) in $\gradSobQM{\ddom}{\ddmes}$ will be understanding the corresponding ``gradients of zero".
We work towards obtaining this understanding progressively; first we look to understand the set $\gradZero{\ddom}{\lambda_{jk}}$ when the edge $I_{jk}$ is assumed to be parallel to the $x_1$-axis, then employ a rotation argument to understand $\gradZero{\ddom}{\lambda_{jk}}$ for a general edge that is at an angle to the $x_1$-axis.
Given that the singular measure $\ddmes$ is just the sum of the individual singular measures supporting each edge, we can then prove that elements of $\gradZero{\ddom}{\ddmes}$ display the same behaviour as elements of $\gradZero{\ddom}{\lambda_{jk}}$ when restricted to the edge $I_{jk}$.
This argument also provides us with a clear geometric interpretation for what a ($\lambda_{jk}$)-gradient of zero" is, and the edge-wise characterisation we obtain for elements of $\gradZero{\ddom}{\ddmes}$ will carry through into how we describe functions in $\gradSobQM{\ddom}{\ddmes}$.
Throughout this section we take assumption \ref{ass:MeasTheoryProblemSetup} as given.

\subsection{Gradients of Zero} \label{appS:muGradZero}
We begin by describing $\gradZero{\ddom}{\lambda_{jk}}$ for when the edge $I_{jk}$ is parallel to the $x_1$-axis.
\begin{prop}[Gradients of Zero on a Segment Parallel to the $x_1$-axis] \label{prop:GradZeroParallelZhikov}
	Let $I$ be a segment in the $\bracs{x_1,x_2}$-plane parallel to the $x_1$-axis, and let $\lambda_I$ be the singular measure supported on $I$.
	Then 
	\begin{align*}
		\gradZero{\ddom}{\lambda_I} &= 
		\clbracs{
			\begin{pmatrix} 0 \\ f	\end{pmatrix}
			\ \vert \ f\in\ltwo{\ddom}{\lambda_I}
		}.
	\end{align*}
\end{prop}
\begin{proof}
	By combining proposition \ref{prop:GradZeroInvarientUnderQM} with a result in \cite{zhikov2000extension} \tstk{precise lemma/page citation}, this result follows, however we present a brief summary of the argument.
	Without loss of generality we assume $x_2=0$ on $I$.
	Additionally it suffices to show that the set on the right hand side includes all functions of the specified form when $f$ is smooth, as we can then apply a density argument. \newline
	
	So take some $f\in\smooth{\ddom}$, then the ``constant sequence" $\phi_n = \phi = x_2 f$ is such that
	\begin{align*}
		\phi_n\lconv{\ltwo{\ddom}{\lambda_I}}0, 
		&\quad \grad\phi_n\lconv{\ltwo{\ddom}{\lambda_I}^2} \begin{pmatrix} 0 \\ f \end{pmatrix}
		&\quad \toInfty{n}
	\end{align*}	 
	and so $\bracs{0,f}^\top\in\gradZero{\ddom}{\lambda_I}$. \newline
	
	We now prove that if $\bracs{f,0}^\top\in\gradZero{\ddom}{\lambda_I}$ then $f=0$.
	So suppose $\bracs{f,0}\in\gradZero{\ddom}{\lambda_I}$ and take an approximating sequence $\phi_n$ as in \eqref{eq:GradZeroDef}.
	Performing a change of variables via the map $r:\interval{I}\rightarrow I$ (as described for an edge $I_{jk}$ in convention \ref{ass:MeasTheoryProblemSetup}), and setting $\tilde{\phi}_n(t) := \phi_n\bracs{r(t)}$, we have
	\begin{align*}
		\tilde{\phi}_n\lconv{\ltwo{\interval{I}}{t}} 0, 
		&\quad \diff{\tilde{\phi}_n}{t}\lconv{\ltwo{\interval{I}}{t}} \tilde{f}
		&\quad \toInfty{n}.
	\end{align*}
	Hence $\tilde{f}$ is the distributional derivative (in the $\gradSob{\interval{I}}{t}$ sense) of the zero function, so we can conclude that $\tilde{f} = 0$, and thus $f = 0$, as we sought.
\end{proof}

Proposition \ref{prop:GradZeroParallelZhikov} provides the following interpretation for ``gradients of zero".
The measure $\lambda_I$ however can only ``see" along the segment $I$, as this is it's entire support.
As such $\lambda_I$ can only see the change in a function in the direction along the segment $I$, hence we find that $\gradZero{\ddom}{\lambda_I}$ consists of all the components of gradients that are directed perpendicular to $I$.
The following proposition reinforces this interpretation, although the argument is simply to invoke the result of proposition \ref{prop:GradZeroParallelZhikov} after applying the obvious rotation.
\begin{prop}[Rotation of Edge Gradients of Zero] \label{prop:RotationOfEdgeGradients}
	Consider the case when $\graph$ consists of a single edge (or segment) $I\subset\ddom$ with orthogonal co-ordinate system $y=\bracs{y_1,y_2}$, with $y_1$ parallel to $I$.
	Let $R$ be the orthogonal change of co-ordinates $x=Ry$ with $x=\bracs{x_1,x_2}$ the orthogonal co-ordinate system along the axes.
	Then
	\begin{align*}
		\gradZero{\ddom}{\lambda_I} 
		&= \clbracs{ R^{\top} \begin{pmatrix} 0 \\ f_2 \end{pmatrix} \ \vert \ f_2\in\ltwo{\ddom}{\lambda_I} }.
	\end{align*}
\end{prop}
With these two results, there is the following corollary which further reinforces the interpretation of $\gradZero{\ddom}{\lambda_I}$ given earlier.
\begin{cory} \label{cory:Grad0SingleEdge}
	Assume the hypothesis of proposition \ref{prop:RotationOfEdgeGradients}, and denote by $e_I$ the unit vector parallel to the segment $I$.
	Then
	\begin{align*}
		\gradZero{\ddom}{\lambda_I} &= \clbracs{z\in\ltwo{\ddom}{\lambda_I} \ \vert \ z\vert_{I}\cdot e_I = 0}.
	\end{align*}
\end{cory}

Using our understanding of gradients of zero on single edges, we can build up an understanding of gradients of zero on embedded graphs consisting of multiple edges.
This turns out to be the following characterisation; where each function in $\gradZero{\ddom}{\ddmes}$ behaves as a function in $\gradZero{\ddom}{\lambda_{jk}}$ when restricted to the edge $I_{jk}$.
\begin{prop} \label{prop:GradZeroGraph}
	Given convention \ref{ass:MeasTheoryProblemSetup}, we have that
	\begin{align*}
		\gradZero{\ddom}{\ddmes} &= \clbracs{g\in\ltwo{\ddom}{\ddmes}^2 \ \vert \ g\vert_{I_{jk}}\cdot e_{jk}=0 \ \forall I_{jk}\in \edgeSet} \\
		&= \clbracs{g\in\ltwo{\ddom}{\ddmes}^2 \ \vert \ g\in\gradZero{\ddom}{\lambda_{jk}} \ \forall I_{jk}\in \edgeSet}. \labelthis\label{eq:GradZeroSetRHS}
	\end{align*}
\end{prop}
\begin{proof}
	For the full proof, see appendix \tstk{appendix reference}.
	Showing that elements of $\gradZero{\ddom}{\ddmes}$ are contained in the set on the right-hand-side of \eqref{eq:GradZeroSetRHS} is straightforward due to the definition of $\ddmes$ and the fact that
	\begin{align} \label{eq:GraphMeasNormEdgeBreakdown}
		\norm{\cdot}_{\ltwo{\ddom}{\ddmes}}^2 = \sum_{j\con k}\norm{\cdot}_{\ltwo{\ddom}{\lambda_{jk}}}^2,
	\end{align}
	so if one has convergence in the $\ltwo{\ddom}{\ddmes}$-norm then we have convergence in each $\ltwo{\ddom}{\lambda_{jk}}$-norm.
	The opposite set inclusion involves a number of technical steps, and the full argument is given in the appendix.
	The gist of the argument involves showing that if $g$ is a member of the set on the right-had-side of \eqref{eq:GradZeroSetRHS}, then we can demonstrate that $g_{jk}\in\gradZero{\ddom}{\ddmes}$ (recall $g_{jk}$ is $g$ restricted to $I_{jk}$ then extended by zero to $\ddom$) given that $g_{jk}\in\gradZero{\ddom}{\lambda_{jk}}$.
	Then we can use the idea that $g = \sum_{j\con k} g_{jk}$ and that $\gradZero{\ddom}{\ddmes}$ is a closed, linear subspace to obtain membership of $g$ in $\gradZero{\ddom}{\ddmes}$; although in practice the expression for the sum is made more complex by the need to ensure good behaviour near the vertices of the graph.
\end{proof}

\subsection{The ``Non-Classical" Sobolev Space} \label{appS:SobSpacesTheory}
Establishing an understanding of $\gradZero{\ddom}{\ddmes}$  affords us greater insight into the tangential gradient $\tgrad_\ddmes u$ of functions $u\in\gradSobQM{\ddom}{\ddmes}$.
Given that we know that $\tgrad_\ddmes u \perp \gradZero{\ddom}{\ddmes}$, and we have an edge-wise characterisation of $\gradZero{\ddom}{\ddmes}$, it will not be surprising to learn that we also obtain an edge-wise ``form" for the tangential gradient, as given in the following proposition.
\begin{prop} \label{prop:GraphTangGrad}
	For each $I_{jk}\in \edgeSet$ write $\gradSob{\interval{I_{jk}}}{t}$ for the (``classical") Sobolev space on the interval $\interval{I_{jk}}$ with respect to the Lebesgue measure, and let $\tilde{u}_{jk} = u_{jk} \circ r_{jk}$.
	Then for $u\in\gradSobQM{\ddom}{\ddmes}$ we have that $\tilde{u}_{jk}\in\gradSob{\interval{I_{jk}}}{t}$ for each $I_{jk}\in \edgeSet$, and that
	\begin{align*}
		\bracs{ \tgrad_\ddmes u }_{jk} 
		&= R_{jk}^\top \begin{pmatrix} u_{jk}' + i\bracs{R_{jk}\qm}_1 u_{jk} \\ 0	\end{pmatrix}
	\end{align*}
	where $u_{jk}' = \bracs{ \tilde{u}_{jk}' } \circ r_{jk}^{-1}$.
\end{prop}
Note that the prime notation on $u_{jk}'$ does \emph{not} imply the existence of any kind of ``classical" derivative for $u_{jk}$ or $u$, it is just a helpful piece of notation to remind us that $u_{jk}$ does have some regularity after composition with $r_{jk}$.
\begin{proof}
	The proof proceeds in much the same way as how we sought to understand elements of $\gradZero{\ddom}{\ddmes}$, and full details can be found in section \ref{sec:tstk}.
	Any tangential gradient must be orthogonal to elements $g_{jk}\in\gradZero{\ddom}{\ddmes}$ where $g\in\gradZero{\ddom}{\lambda_{jk}}$, and this must hold for each edge $I_{jk}$.
	The plan is again to first consider an edge aligned parallel to the $x_1$-axis, then apply a rotation before appealing to the edge-wise decomposition of our measure. \newline
	
	As just mentioned, first consider an edge $I_{jk}$ parallel to the $x_1$-axis. 
	Let $\tgrad_\ddmes u = \bracs{v_1, v_2}^\top$ denote the components of $\tgrad_\ddmes u$; we can see that $v_2=0$ immediately due to the result of proposition \ref{prop:GradZeroGraph} and the requirement that $\tgrad_\ddmes u$ be orthogonal $g_{jk}$ for every member of $g\in\gradZero{\ddom}{\lambda_{jk}}$.
	This leaves the form of $v_1$ to be determined.
	Since $u\in\gradSobQM{\ddom}{\ddmes}$ there existences a sequence of smooth functions $\phi_n$ which converge to $u$ in $\ltwo{\ddom}{\ddmes}$ and whose gradients $\tgrad\phi_n$ converge to $\tgrad_\ddmes u$.
	Clearly any such sequence also converges to $u_{jk}$ in $\ltwo{\ddom}{\lambda_{jk}}$ as well (see \eqref{eq:GraphMeasNormEdgeBreakdown}) and $\partial_1\phi_n$ converges to $v_1\vert_{jk}$ in $\ltwo{\ddom}{\lambda_{jk}}$.
	Considering the composition $\tilde{\phi}_n = \phi_n \circ r_{jk}$ we find that
	\begin{align*}
		\tilde{\phi}_n \lconv{\ltwo{\interval{I_{jk}}}{t}} \tilde{u}_{jk},
		&\quad \diff{\tilde{\phi}_n}{t} \lconv{\ltwo{\interval{I_{jk}}}{t}} v_1\vert_{jk} - i\qm_1 \tilde{u}_{jk},
	\end{align*}
	from which we can deduce that $\tilde{u}_{jk}' = v_1 - i\qm_1\tilde{u}$, and hence obtain the result for an edge parallel to the $x_1$-axis ($R_{jk}$ being the identity).
	Any edges that are not parallel to the $x_1$-axis can now be rotated under $R_{jk}$ to bring them into an appropriate framework, the argument repeated and then unpacked to obtain the quoted form on each edge.
	Then since this holds for each edge in any graph, the result of in the proposition is obtained.
\end{proof}

Whilst proposition \ref{prop:GraphTangGrad} arrives at an expected conclusion (given proposition \ref{prop:GradZeroGraph}) for the form of the tangential gradient, $\gradSobQM{\ddom}{\ddmes}$ has some additional structure that is not obvious from this study.
In particular the behaviour of functions near the vertices of $\graph$ has been ignored up until this point, due to the fact that it does not warrant investigation when dealing with gradients of zero and hence the tangential gradients.
It is also not unreasonable to expect some special behaviour of the functions $u\in\gradSobQM{\ddom}{\ddmes}$ at the vertices, otherwise there will be no resemblance of the connectivity of $\graph$ in our function space.
We can deduce that functions $u\in\gradSobQM{\ddom}{\ddmes}$ actually possess continuity at the vertices $v_j\in \vertSet$ of $\graph$ (for any $\qm$), as is proven in \tstk{zhikov ref, with page/thm number} for when $\qm=0$:
\begin{theorem} \label{thm:CharOfGradSob}
	Assume convention \ref{ass:MeasTheoryProblemSetup}.
	Then we have that
	\begin{align*}
		u\in\gradSobQM{\ddom}{\ddmes} \quad\Leftrightarrow\quad 
		& (i) u\in\gradSobQM{\ddom}{\lambda_{jk}} \ \forall I_{jk}\in \edgeSet, \\
		& (ii) u \text{ is continuous at each } v_j\in \vertSet.
	\end{align*}
\end{theorem}
For consistency with this work, we give a full proof in the appendix so that the interested reader has all the arguments associated with this work in one place.
A sketch of the key ideas is below;
\begin{proof}
	The right-directed ($\Rightarrow$) implication is essentially a result of \eqref{eq:GraphMeasNormEdgeBreakdown}, as (i) follows from this almost immediately.
	(ii) is then obtained by showing that any sequence $\phi_n$ of smooth functions approximating $u$ and $\tgrad_\ddmes u$ (as in the definition of $\gradSobQM{\ddom}{\ddmes}$) is actually Cauchy in the uniform norm.
	As such it must also converge to a continuous function by completeness of this norm, and the limit must be $u_{jk}$.
	In particular it must also converge uniformly on the ``junction" surrounding each vertex $v_j$, and thus $u$ must be continuous at $v_j$ in particular. \newline
	
	The reverse implication is essentially a repeat of the argument for extending gradients of zero from one edge to the whole graph, except now we are doing similar steps for tangential gradients instead.
	Take smooth sequences approximating each $u_{jk}\in\gradSobQM{\ddom}{\lambda_{jk}}$, and sum them in an appropriate way to obtain convergence in $\ltwo{\ddom}{\ddmes}^2$ from the individual convergences in $\ltwo{\ddom}{\lambda_{jk}}^2$.
	Continuity at each vertex is required to control the behaviour of the sequence that is constructed near the vertices - namely we need to ensure there is a small ball around each vertex where the value of each $u_{jk} - u\bracs{v_j}$ is uniformly bounded across those edges $j\con k$.
\end{proof}

\section{Appendix: Analysis of the Sobolev Spaces Associated with the Measure $\nu$} \label{app:VertexAnalysis}
Understanding the behaviour of Sobolev functions on the edges of a graph is the bulk of the information which we need to determine the edge-wise action of the equivalent quantum graph problem that we plan to derive.
Our analysis of $\gradSobQM{\ddom}{\ddmes}$ even provides us with Sobolev functions that are continuous across the vertices.
However we are not encapsulating the effect of introducing coupling constants $\alpha_j$ to the vertices in $\graph$ yet - and to do so we must examine the ``vertex part" $\nu$ of our measure $\dddmes$ and the Sobolev spaces associated with it.
Thankfully, our analysis of the space $\gradSobQM{\ddom}{\nu}$ is short, on account of the fact that $\gradSobQM{\ddom}{\nu}$ is actually isomorphic \tstk{check with Kirill, as this normally implies we have a norm and I never introduced a norm on our Sob Spaces... I guess bijective is safer?} to $\complex^N$, where $N=\abs{\vertSet}$ is the number of vertices in the graph $\graph$.

\subsection{Analysis of $\nu$-Gradients of Zero} \label{appS:VertexGradZero}
If we fix $N = \abs{\vertSet}$ as the number of vertices, we will see that $\gradZero{\ddom}{\nu} \cong \complex^{2N}$, and thus that any $\bracs{u,\grad_{\nu}u}\in\gradSob{\ddom}{\nu}$ is such that $\grad_{\nu}u=0$.
\begin{definition}[$d$, $\varphi_c$, and $g^j$] \label{def:UsefulObjects}
	Let 
	\begin{align*}
		d=\recip{2}\min\clbracs{\norm{v_j-v_k}_2 \ \vert \ v_j,v_k\in\vertSet}
	\end{align*}
	be half the minimum distance between any two vertices in the graph (note that this may occur between two vertices that do not share a single edge).
	$d$ exists since the graph $\graph$ is assumed finite.
	For $c\in\complex$, let $\varphi_c:\reals^2\rightarrow\complex$ be the smooth function such that
	\begin{align*}
		\varphi_c\bracs{0} = 0, &\quad \grad\varphi_c\bracs{0} = c, \\
		\supp\bracs{\varphi_c} &\subset B_{d}\bracs{0},
	\end{align*}
	where $B_{d}\bracs{0}$ denotes the ball of radius $d$ centred at the origin.
	Finally, for each $v_j\in\vertSet$ define
	\begin{align*}
		g^j_1\bracs{x} &=
		\begin{cases}
			\begin{pmatrix} 1 \\ 0 \end{pmatrix} & x=v_j, \\
			0 & x\neq v_j. \\
		\end{cases}
		&\quad
		g^j_2\bracs{x} &=
		\begin{cases}
			\begin{pmatrix} 0 \\ 1 \end{pmatrix} & x=v_j, \\
			0 & x\neq v_j. \\
		\end{cases}
	\end{align*}
\end{definition}

Notice that we have the following result:
\begin{lemma}
	The space $\ltwo{\ddom}{\nu}$ is isomorphic to $\complex^{2N}$.
	Moreover, the collection 
	\begin{align*}
		\clbracs{g_1^j, g_2^j \ \vert \ v_j\in\vertSet}
	\end{align*}
	forms a basis of $\ltwo{\ddom}{\nu}^2$.
\end{lemma}
\begin{proof}
	It is sufficient to notice that any $f\in\ltwo{\ddom}{\nu}^2$ is entirely determined by the values it takes at the vertices $v_j$.
	Each of these values is a $\complex^2$-vector, and thus we may define the function
	\begin{align*}
		\iota:\ltwo{\ddom}{\nu}^2 \rightarrow\complex^{2N}, &\quad
		\iota\bracs{f} = \begin{pmatrix} f\bracs{v_1} \\ f\bracs{v_2} \\ \vdots \\ f\bracs{v_N} \end{pmatrix},
	\end{align*}
	where we vertically concatenate the collection of two-vectors $f\bracs{v_j}, v_j\in\vertSet$.
	Clearly $\iota$ is a bijection, and additionally for $f,g\in\ltwo{\ddom}{\nu}$ we have that
	\begin{align*}
		\integral{\ddom}{f\cdot \overline{g}}{\nu} &= \sum_{v_j\in\vertSet} f\bracs{v_j}\cdot\overline{g\bracs{v_j}} \\
		&= \iota\bracs{f}\cdot\overline{\iota\bracs{g}},
	\end{align*}
	so $\iota$ is an isometry.
	Moreover, the preimage of the canonical basis $\clbracs{e_k \ \vert \ k\in\clbracs{1,...,2N}}$ under $\iota$ is the collection $\clbracs{g_1^j, g_2^j \ \vert \ v_j\in\vertSet}$, and hence $\clbracs{g_1^j, g_2^j \ \vert \ v_j\in\vertSet}$ forms a basis of $\ltwo{\ddom}{\nu}^2$.
\end{proof}

We now characterise the set of $\nu$-gradients of zero, $\gradZero{\ddom}{\nu}$, which will turn out to be the entire space $\ltwo{\ddom}{\nu}^2$.
\begin{prop}[Characterisation of $\gradZero{\ddom}{\nu}$] \label{prop:CharPointMassGradZero}
	We have that $\gradZero{\ddom}{\nu} = \ltwo{\ddom}{\nu}^2$.
\end{prop}
\begin{proof}
	Since $\gradZero{\ddom}{\nu}$ is a closed, linear subspace of $\ltwo{\ddom}{\nu}^2$ by definition, it is sufficient to show that $\gradZero{\ddom}{\nu}$ contains the basis $\clbracs{g_1^j, g_2^j \ \vert \ v_j\in\vertSet}$.
	We demonstrate inclusion of the elements $g^j_1$ (as that of $g^j_2$ is similar, with the obvious alternative choice of $c$ in what follows).
	Take $c=\bracs{1,0}^{\top}$, fix $v_j\in\vertSet$, and set
	\begin{align*}
		\phi\bracs{x} &= \varphi_c\bracs{x-v_j},
	\end{align*}
	where $\varphi_c$ is as in definition \ref{def:UsefulObjects}.
	Note that $\phi$ is smooth by composition of smooth functions, $\supp\bracs{\phi}\subset B_{d}\bracs{v_j}$, and that
	\begin{align*}
		\grad\phi\bracs{x} &= \bracs{\grad\varphi_c}\bracs{x-v_j}.
	\end{align*}
	Then we have that
	\begin{align*}
		\integral{\ddom}{\abs{\phi}^2}{\nu} &= \sum_{v_l\in\vertSet} \alpha_l\abs{\phi\bracs{v_l}}^2 \\
		&= \alpha_j\abs{\varphi_c\bracs{0}}^2 + \sum_{v_l\neq v_j}\alpha_l\abs{\phi\bracs{v_l}}^2 \\
		&= 0 + \sum_{v_l\neq v_j} \alpha_l \times 0 = 0,
	\end{align*}
	and
	\begin{align*}
		\integral{\ddom}{\abs{\grad\phi - g^j_1}^2}{\nu} &= \sum_{v_l\in\vertSet} \alpha_l\abs{\grad\phi\bracs{v_l} - g^j_1\bracs{v_l}}^2 \\
		&= \alpha_j\abs{\grad\varphi_c\bracs{0} - \bracs{1,0}^{\top}}^2 + \sum_{v_l\neq v_j} \alpha_l\abs{\grad\varphi_c\bracs{v_l} - 0}^2 \\
		&= 0 + \sum_{v_l\neq v_j} \alpha_l\abs{0 - 0}^2 = 0.
	\end{align*}
	Hence, the constant sequence of smooth functions $\phi$ is such that
	\begin{align*}
		\phi\lconv{\ltwo{\ddom}{\nu}}0, \quad \grad\phi\lconv{\ltwo{\ddom}{\nu}^2} g^j_1,
	\end{align*}
	and hence $g^j_1\in\gradZero{\ddom}{\nu}$.
\end{proof}

\subsection{The Sobolev Space $\gradSobQM{\ddom}{\nu}$} \label{appS:VertexSobSpace}
Given that we now know that $\gradZero{\ddom}{\nu}$ encompasses the whole of $\ltwo{\ddom}{\nu}^2$, we must conclude that our ``Sobolev functions" have zero derivative ($\nu$-)almost-everywhere.
Indeed, proposition \ref{prop:CharPointMassGradZero} gives us the following corollary:
\begin{cory}[Characterisation of $\gradSobQM{\ddom}{\nu}$] \label{eq:CharPointMassSpace}
	We have that
	\begin{align*}
		\bracs{u,\tgrad_{\nu}u}\in\gradSobQM{\ddom}{\nu} \quad\Leftrightarrow\quad 
		& \ \text{(i)} \ u\in\ltwo{\ddom}{\nu}, \\
		& \ \text{(ii)} \ \tgrad_{\nu}u = 0 \ \nu\text{-almost everywhere}.
	\end{align*}
\end{cory}
\begin{proof}
	($\Rightarrow$) For the right-directed implication; $\tgrad_{\nu}u\in\ltwo{\ddom}{\nu}^2$ is an element of $\ltwo{\ddom}{\nu}^2$ by definition and is orthogonal to $\gradZero{\ddom}{\nu}$, but by proposition \ref{prop:CharPointMassGradZero} we know that $\gradZero{\ddom}{\nu}=\ltwo{\ddom}{\nu}^2$, we must conclude that $\grad_{\nu}u = 0$. \newline
	($\Leftarrow$) For the left-directed implication, take smooth ``bump" functions $\psi_j$ (for each $v_j\in\vertSet$) with the properties
	\begin{align*}
		\psi_j\bracs{v_j} = 1, &\quad \grad\psi_j\bracs{v_j} = -i\qm, \\
		\supp\bracs{\psi_j} &\subset B_{d}\bracs{v_j}.
	\end{align*}
	Then consider the smooth function
	\begin{align*}
		\phi\bracs{x} &:= \sum_{v_j\in\vertSet} u\bracs{v_j}\psi_j\bracs{x}, \\
		\implies \grad\phi\bracs{x} &= \sum_{v_j\in\vertSet} u\bracs{v_j}\grad\psi_j\bracs{x}.
	\end{align*}
	Then we have that
	\begin{align*}
		\integral{\ddom}{\abs{\phi - u}^2}{\nu} &= \sum_{v_j\in\vertSet} \alpha_j\abs{\phi\bracs{v_j} - u\bracs{v_j}}^2 \\
		&= \sum_{v_j\in\vertSet} \alpha_j\abs{ \sum_{v_l\in\vertSet}u\bracs{v_l}\psi_l\bracs{v_j} - u\bracs{v_j} }^2
		&= \sum_{v_j\in\vertSet} \alpha_j\abs{u\bracs{v_j}}^2\abs{\psi_j\bracs{v_j}-1}^2 \\
		&= \sum_{v_j\in\vertSet} \alpha_j\abs{u\bracs{v_j}}^2 \times 0 = 0,
	\end{align*}
	and
	\begin{align*}
		\integral{\ddom}{\abs{\tgrad\phi - 0}^2}{\nu} 
		&= \sum_{v_j\in\vertSet} \alpha_j\abs{ \sum_{v_l\in\vertSet} u\bracs{v_l}\grad\psi_l\bracs{v_j} + i\qm\psi_l\bracs{v_j} }^2 \\
		&= \sum_{v_j\in\vertSet} \alpha_j\abs{u\bracs{v_j}}^2 \abs{ \grad\psi_j\bracs{v_j} + i\qm\psi_j\bracs{v_j} }^2 \\
		&= \sum_{v_j\in\vertSet} \alpha_j\abs{u\bracs{v_j}}^2 \abs{ i\qm - i\qm } = 0.
	\end{align*}
	Thus, the constant sequence $\phi$ is such that
	\begin{align*}
		\phi \lconv{\ltwo{\ddom}{\nu}} u, \quad \tgrad\phi \lconv{\ltwo{\ddom}{\nu}^2} 0,
	\end{align*}
	and thus $\bracs{u,0}\in\gradSob{\ddom}{\nu}$.
\end{proof}
Corollary \ref{conj:ThickVertexSpaceCharacterisation} means that the space $\gradSob{\ddom}{\nu}$ is essentially isomorphic to $\complex^N$, namely functions in this space are entirely determined by their values at the vertices, and their gradients are always zero.
This matches our intuitive expectations, as the notion of a gradient (or rate of change) at an isolated point being non-zero implies that there is a small neighbourhood around the point in which we can observe the function values changing, but in the case of a point-mass measure this is not the case.

\section{Appendix: Sobolev Spaces associated with the measure $\dddmes$} \label{app:SumMeasureAnalysis}
Now that the analysis of sections \ref{app:3DMuAnalysis} and \ref{app:3DVertexAnalysis} is complete, we turn out attention to the notion of $\kt$-gradients, curls, and divergence free with respect to the measure $\dddmes$.
Needless to say, our reason for previously considering the two measures $\ddmes$ and $\nu$ individually is because we expect a link back to the properties that these measures bestowed upon functions.
Indeed, our first result informs us that gradients of zero with respect to $\dddmes$ are formed from combinations of gradients of zero with respect to the measures $\ddmes$ and $\nu$.
\begin{prop} \label{prop:3DThickVertexGradZeroCharacterisation}
	Let $g\in\ltwo{\ddom}{\dddmes}^3$ and let 
	\begin{align*}
		g_{\ddmes}(x) = \begin{cases} g(x) & x\neq v_j \ \forall v_j\in\vertSet, \\ 0 & x=v_j, \ v_j\in\vertSet, \end{cases} 
		&\qquad
		g_{\nu}(x) = \begin{cases} 0 & x\neq v_j \ \forall v_j\in\vertSet, \\ g(x) & x=v_j, \ v_j\in\vertSet. \end{cases}
	\end{align*}		
	Then we have that
	\begin{align*}
		g\in\gradZero{\ddom}{\dddmes} \quad\Leftrightarrow\quad 
		& g_{\ddmes}\in\gradZero{\ddom}{\ddmes} \text{ and } g_{\nu}\in\gradZero{\ddom}{\nu}.
	\end{align*}
\end{prop}
\begin{proof}
	\tstk{this is simply a restatement of the result E1 from the scalar paper - there is literally no difference to the argument when we carry around an additional component that is always 0 anyway.
	I can put the whole proof in, but I just don't think it's necessary here.}
\end{proof}

Knowing the correspondence between the various gradients of zero with respect to the measures $\dddmes$, $\ddmes$, and $\nu$, we can deduce the following about the $\kt$-tangential gradients of functions in $\ktgradSob{\ddom}{\dddmes}$.
\begin{theorem} \label{thm:3DThickVertexTangGradImplication}
	\begin{align*}
		\bracs{u, \ktgrad_{\dddmes}u}\in\ktgradSob{\ddom}{\dddmes} \quad\Rightarrow\quad
		& \ \mathrm{(i)} \ \bracs{u, \ktgrad_{\dddmes}u}\in\ktgradSob{\ddom}{\ddmes}, \\
		& \ \mathrm{(ii)} \ \bracs{u, \ktgrad_{\dddmes}u}\in\ktgradSob{\ddom}{\nu}. \\
	\end{align*}
\end{theorem}
\begin{proof}
	Taking an approximating sequence $\phi_n$ such that
	\begin{align*}
		\phi_n \lconv{\ltwo{\ddom}{\dddmes}} u, \quad \ktgrad\phi_n\lconv{\ltwo{\ddom}{\dddmes}^2}\grad_{\dddmes}u,
	\end{align*}
	then using the fact that
	\begin{align*}
		\norm{\cdot}_{\ltwo{\ddom}{\dddmes}}^2 &= \norm{\cdot}_{\ltwo{\ddom}{\ddmes}}^2 + \norm{\cdot}_{\ltwo{\ddom}{\nu}}^2,
	\end{align*}
	we infer that $\phi_n$ also converges in $\ltwo{\ddom}{\ddmes}$ and $\ltwo{\ddom}{\nu}$, as do its gradients in $\ltwo{\ddom}{\ddmes}^2$ and $\ltwo{\ddom}{\nu}^2$.
	Furthermore, since $\ktgrad_{\dddmes}u \perp \gradZero{\ddom}{\dddmes}$ and given proposition \ref{prop:3DThickVertexGradZeroCharacterisation}, we have that $\ktgrad_{\dddmes}u$ is orthogonal to $\gradZero{\ddom}{\ddmes}$ (in $\ltwo{\ddom}{\ddmes}^2$) and to $\gradZero{\ddom}{\nu}$ (in $\ltwo{\ddom}{\nu}^2$), and we are done. 
\end{proof}
From our analysis of $\ktgradSob{\ddom}{\ddmes}$ we know that the conditions (i) and (ii) in theorem \ref{thm:3DThickVertexTangGradImplication} are sufficient for the following to hold;
\begin{align*}
		\text{(a)} \ & \bracs{u, \ktgrad_{\dddmes}u}\in\gradSob{\ddom}{\lambda_{jk}} \ \forall I_{jk}\in\edgeSet, \\
		\text{(b)} \ & u \text{ is continuous across the vertices} \ v_j\in\vertSet, \\
		\text{(c)} \ & \ktgrad_{\dddmes}u\vert_{v_j} = 0 \ \forall v_j\in\vertSet.
\end{align*}
If the converse to theorem \ref{thm:CharOfGradSob} is true, as implied from results in \cite{zhikov2002homogenization}, then (i)-(ii) are necessary and sufficient for (a)-(c).
Understanding the space $\ktgradSob{\ddom}{\dddmes}$ through the properties of $\ktgradSob{\ddom}{\ddmes}$ and $\ktgradSob{\ddom}{\nu}$ will allow us to understand the $\kt$-divergence-free condition with respect to $\dddmes$.
However, before proceeding we also deduce results similar to proposition \ref{prop:3DThickVertexGradZeroCharacterisation} and theorem \ref{thm:3DThickVertexTangGradImplication}, but concerning curls with respect to $\dddmes$.

\begin{prop} \label{prop:ThickVertexCurlZeroCharacterisation}
	Let $c\in\ltwo{\ddom}{\dddmes}^3$ and set
	\begin{align*}
		c_{\ddmes}(x) = \begin{cases} c(x) & x\neq v_j \ \forall v_j\in\vertSet, \\ 0 & x=v_j, \ v_j\in\vertSet, \end{cases} 
		&\qquad
		c_{\nu}(x) = \begin{cases} 0 & x\neq v_j \ \forall v_j\in\vertSet, \\ c(x) & x=v_j, \ v_j\in\vertSet. \end{cases}
	\end{align*}		
	Then we have that
	\begin{align*}
		c\in\curlZero{\ddom}{\dddmes} \quad\Leftrightarrow\quad 
		& c_{\ddmes}\in\curlZero{\ddom}{\ddmes} \text{ and } c_{\nu}\in\curlZero{\ddom}{\nu}.
	\end{align*}
\end{prop}
\begin{proof}
	For the right-directed implication ($\Rightarrow$), it is sufficient to notice that 
	\begin{align*}
		\norm{\cdot}_{\ltwo{\ddom}{\dddmes}}^2 &= \norm{\cdot}_{\ltwo{\ddom}{\ddmes}}^2 + \norm{\cdot}_{\ltwo{\ddom}{\nu}}^2,
	\end{align*}
	so any approximating sequence for $c$ that converges in $\ltwo{\ddom}{\dddmes}^3$ also converges in $\ltwo{\ddom}{\ddmes}^3$ to $c_{\ddmes}$ and in $\ltwo{\ddom}{\nu}^3$ to $c_{\nu}$.
	
	For the left-directed implication ($\Leftarrow$), it is sufficient for us to demonstrate the implication holds for the case when $c_{\nu}=0$, and the case that $c_{\ddmes}=0$ with $c_{\nu}\neq0$ at precisely one vertex $v$.
	Having shown the implication in these cases, linearity of $\curlZero{\ddom}{\dddmes}$ will then complete the implication.
	As such, first consider the case when $c_{\nu}=0$.
	Notice that the conclusion of lemma \ref{lem:CurlZeroExtensionLemma} can be strengthened to membership of $\curlZero{\ddom}{\dddmes}$, as the approximating sequence $\psi_l$ that is constructed satisfies $\psi_l\bracs{v_j}=0, \grad^{(0)}\wedge\psi_l\bracs{v_j}=0$. \tstk{we should make this observation in appendix C too)}
	With this, the argument of proposition \ref{prop:lem:BInCurlZero} can be recycled to conclude that $c_{\ddmes}\in\curlZero{\ddom}{\dddmes}$, and hence $c\in\curlZero{\ddom}{\dddmes}$ too.
	
	Next, consider the case when $c_{\ddmes}=0$, and when $c_{\nu}=0$ at all vertices except $v\in\vertSet$, with $v=\bracs{v_1, v_2}\in\ddom$ and with $c_{\nu}(v) = \bracs{c_1, c_2, c_3}^\top$.
	For each $n\in\naturals$ consider the smooth function $\phi_n:\reals\rightarrow\sqbracs{-1,1}$ as illustrated in figure \ref{fig:Diagram_SmoothFunctionBoundedGrad1D}, with the properties
	\begin{align*}
		\phi_n(t) &= 0, &\quad t\not\in B_{\frac{2}{n}}(0), \\
		\abs{\phi_n(t)} &\leq \recip{n} &\quad t\in B_{\frac{2}{n}}(0), \\
		\phi_n(0) &= 0, \\
		\phi'_n(0) &= 1.
	\end{align*}
	\begin{figure}[b]
		\centering
		\includegraphics[scale=1.0]{Diagram_SmoothFunctionBoundedGrad1D.pdf}
		\caption{\label{fig:Diagram_SmoothFunctionBoundedGrad1DAltAxisLabels} The profile of the functions $\phi_n$ \tstk{$x$-axis label should be $t$, and so should argument of $y$-axis!}.}
	\end{figure}
	Since $\abs{\phi_n(t)} \leq \recip{n}$ when $\abs{t}\leq\recip{n}$, $\phi_n$ can be chosen so that exists a constant $K$ independent of $n$ such that $\abs{\grad\phi_n} \leq K$ when $\recip{n} \leq \abs{x-v} \leq \frac{2}{n}$.
	Define the functions $\Phi^n\in\smooth{\ddom}$ (and compute their curls) as follows;
	\begin{align*}
		\Phi^n(x) = 
		\begin{pmatrix} 
			0 \\ 
			c_3\phi_n\bracs{x_1 - v_1} \\ 
			c_1\phi_n\bracs{x_2-v_2} + c_2\phi_n\bracs{v_1-x_1} 
		\end{pmatrix},
		&\qquad
		\grad^{(0)}\wedge\Phi_n(x) =
		\begin{pmatrix}
			c_1\phi'_n\bracs{x_2-v_2} \\
			c_2\phi'_n\bracs{v_1-x_1} \\
			c_3\phi'_n\bracs{x_1-v_1}
		\end{pmatrix}.
	\end{align*}
	Then we have the following:
	\begin{align*}
		\integral{\ddom}{ \abs{ \Phi^n }^2 }{\dddmes}
		&\leq \bracs{c_3^2 + \bracs{c_1 + c_2}^2} \bracs{ \integral{B_{\frac{2}{n}}(v)}{ \recip{n^2} }{\ddmes}
		+ \alpha_v\abs{\phi_n(0)}^2 } \\
		&= \frac{2\mathrm{deg}(v)}{n^3}\bracs{\abs{c_3}^2 + \abs{c_1 + c_2}^2} \rightarrow 0 \toInfty{n}, \\
		\integral{\ddom}{ \abs{ \grad^{(0)}\wedge\Phi^n - c }^2 }{\dddmes}
		&= \integral{\ddom}{ \abs{ \grad^{(0)}\wedge\Phi^n}^2 }{\ddmes}
		+ \integral{\ddom}{ \abs{ \grad^{(0)}\wedge\Phi^n - c_{\nu} }^2 }{\nu} \\
		&= \abs{c_1}^2\integral{\ddom}{ \abs{\phi'_n\bracs{x_2-v_2}}^2 }{\ddmes}
		+ \abs{c_2}^2\integral{\ddom}{ \abs{\phi'_n\bracs{v_1-x_1}}^2 }{\ddmes} \\
		&\quad + \abs{c_3}^2\integral{\ddom}{ \abs{\phi'_n\bracs{x_1-v_1}}^2 }{\ddmes}
		+ \alpha_v\abs{c(v)}^2\abs{\phi'_n(0)-1}^2 \\
		&\leq \abs{c(v)}^2 \bracs{ K^2\integral{B_{\frac{2}{n}}(v)}{ }{\ddmes}
		+ \alpha_v\abs{\phi'_n(0)-1}^2 } \\
		&= \frac{2\mathrm{deg}(v)}{n}\abs{c(v)}^2 K^2 \rightarrow 0 \toInfty{n},
	\end{align*}
	where $\mathrm{\deg}(v)$ is the degree of the vertex $v$.
	We thus conclude that $c\in\curlZero{\ddom}{\dddmes}$, and given the linearity of $\curlZero{\ddom}{\dddmes}$, the proof is complete.
\end{proof}

\begin{theorem} \label{thm:ThickVertexTangCurlImplication}
	\begin{align*}
		\bracs{u, \ktcurl{\dddmes}u}\in\ktcurlSob{\ddom}{\dddmes}
		\quad\Rightarrow\quad
		\mathrm{(i)} &\ \bracs{u, \ktcurl{\dddmes}u}\in\ktcurlSob{\ddom}{\ddmes} \\
		\mathrm{(ii)} &\ \bracs{u, \ktcurl{\dddmes}u}\in\ktcurlSob{\ddom}{\nu}
	\end{align*}
\end{theorem}
\begin{proof}
	Taking an approximating sequence $\phi_n$ such that
	\begin{align*}
		\phi_n \lconv{\ltwo{\ddom}{\dddmes}} u, \quad \ktcurl{}\phi_n\lconv{\ltwo{\ddom}{\dddmes}^3}\ktcurl{\dddmes}u,
	\end{align*}
	then using the fact that
	\begin{align*}
		\norm{\cdot}_{\ltwo{\ddom}{\dddmes}}^2 &= \norm{\cdot}_{\ltwo{\ddom}{\ddmes}}^2 + \norm{\cdot}_{\ltwo{\ddom}{\nu}}^2,
	\end{align*}
	we infer that $\phi_n$ also converges in $\ltwo{\ddom}{\ddmes}$ and $\ltwo{\ddom}{\nu}$, as do its $\kt$-curls in $\ltwo{\ddom}{\ddmes}^3$ and $\ltwo{\ddom}{\nu}^3$.
	Furthermore, since $\ktcurl{\dddmes}u \perp \curlZero{\ddom}{\dddmes}$ and given proposition \ref{prop:ThickVertexCurlZeroCharacterisation}, we have that $\ktcurl{\dddmes}u$ is orthogonal to $\curlZero{\ddom}{\ddmes}$ (in $\ltwo{\ddom}{\ddmes}^3$) and to $\curlZero{\ddom}{\nu}$ (in $\ltwo{\ddom}{\nu}^3$), and we are done. 
\end{proof}

Theorem \ref{thm:ThickVertexTangCurlImplication} provides us with a description of the space $\ktcurlSob{\ddom}{\dddmes}$.
When combined with the $\kt$-divergence-free condition, we will have enough information to derive the system \eqref{eq:QGFullSystem} from \eqref{eq:PeriodCellCurlCurlStrongForm}.
Of course, given that we have proposition \ref{prop:3DThickVertexGradZeroCharacterisation} and theorem \ref{thm:3DThickVertexTangGradImplication}, we can look to understand the divergence-free condition.

\begin{prop} \label{prop:ThickVertexDivFree}
	Let $\graph=\bracs{\vertSet, \edgeSet}$ be a graph embedded into $\ddom$.
	Give each $I_{jk}\in\edgeSet$ local coordinate system $y_{jk}=\bracs{y_{1, jk},y_{2, jk}}$ with $y_{2,jk}$ parallel to $I_{jk}$, and let $R_{jk}\in\mathrm{SO}(2)$ be the change of coordinates $x=R_{jk}y_{jk}$ where $x=\bracs{x_1, x_2}$ is the axis coordinate system.
	Write $r_{jk}$ for the usual parametrisation of the edge $I_{jk}$, and suppose $u=\bracs{u_1, u_2, u_3}^\top\in\ktcurlSob{\ddom}{\dddmes}$, and write $u_{jk}=\bracs{u_{1,jk}, u_{2,jk}, u_{3,jk}}^\top$ for the restriction of $u$ to the edge $I_{jk}$.
	Also define
	\begin{align*}
		U_{jk}(x) = R_{jk}\begin{pmatrix} u_{1, jk}(x) \\ u_{2, jk}(x) \end{pmatrix},
		\qquad \qm_{jk} = \bracs{ R_{jk}\qm }_2,
		\qquad \widetilde{U}_{jk} = U_{jk} \circ r_{jk},
	\end{align*}
	for each edge $I_{jk}$.
	Then
	\begin{align*}
		u \text{ is divergence free } \quad\Leftrightarrow\quad
		\begin{cases}
			\mathrm{(i)} & U_{1, jk} = 0, \ \forall I_{jk}\in\edgeSet, \\
			\mathrm{(ii)} & \widetilde{U}_{2,jk} \in\gradSob{\ddom}{\lambda_I}, \ \forall I_{jk}\in\edgeSet, \\
			\mathrm{(iii)} & U_{2,jk}' + \rmi\qm_{jk} U_{2,jk} + \rmi\wavenumber U_{3,jk} = 0 \text{ on } I_{jk}, \ \forall I_{jk}\in\edgeSet, \\
			\mathrm{(iv)} & \sum_{j\conRight k} U_{2,jk}\bracs{v_j} - \sum_{j\conLeft k} U_{2,jk}\bracs{v_j} = 0, \ \forall v_j\in\vertSet, \\
			\mathrm{(v)} & u\bracs{v_j}=0, \ \forall v_j\in\vertSet,
		\end{cases}
	\end{align*}
	where $U_{2,jk}' = \widetilde{U}_{2,jk}'\circ r_{jk}^{-1}$.
\end{prop}
\begin{proof}
	This is a direct consequence of proposition \ref{prop:3DThickVertexGradZeroCharacterisation} and theorem \ref{thm:3DThickVertexTangGradImplication}.
\end{proof}

With theorem \ref{thm:ThickVertexTangCurlImplication} and proposition \ref{prop:ThickVertexDivFree}, we have the necessary properties of $\kt$-divergence-free functions $u\in\ktcurlSob{\ddom}{\dddmes}$ to derive the system \eqref{eq:QGFullSystem}.

%need further two appendices - one if we really want a complete walkthrough of the derivation of the QG problem, and another for the full-blown, formal proof recitals.

\end{document}