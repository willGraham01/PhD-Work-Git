\section{Derivation of quantum graph problem} \label{sec:SystemDerivation}
In this section we provide an overview of a system of the form \eqref{eq:QGFullSystem} is obtained from \eqref{eq:PeriodCellLaplaceStrongForm}, which will setup our discussion revolving around the methods we employ for solving \eqref{eq:QGFullSystem} in section \ref{sec:Discussion}.
To reiterate what was said at the end of section \ref{sec:QuantumGraphs}, assumption \ref{ass:MeasTheoryProblemSetup} is adopted throughout this section and the rest of this work. 

Precise definition and analysis of the ``Sobolev spaces" used here can be found in appendix (\ref{app:MeasureTheory}), although we provide a short intuitive idea of the object $\tgrad_{\ddmes}u$ here.
The central idea behind understanding $\tgrad_{\ddmes}u$ is that the singular measure $\ddmes$ only supports the edges of $\graph$, and so cannot ``see" any changes in the function $u$ ``across" (in the direction perpendicular to) the edge $I_{jk}$.
So at any point $x\in I_{jk}$, the ``gradient" $\tgrad_{\ddmes}u$ encapsulates the rate of change of the function $u$ \emph{only} in the direction along the edge $I_{jk}$.
As a result, it is not inaccurate to think of $\tgrad_{\ddmes}u(x) = u_{jk}'(x)e_{jk}$ for $x\in I_{jk}$, where $u_{jk}' = \pdiff{u_{jk}}{e_{jk}}$.
This also means that $\tgrad_{\ddmes}u$ can be characterised by its form on each edge of the graph $\graph$, which is crucial for deriving the set of ``edge ODEs" \eqref{eq:QGEdgeODEs} and providing a meaning to the $\diff{}{x}$ operator that appears in those equations.
As discussed in section \ref{ssec:FunctionSpaces}, the coupling constants attached to the vertices of the graph as well as the connectivity of the graph itself then dictate that these ``edge-wise" components $u_{jk}$ and $u'_{jk}$ adhere to certain matching conditions at the vertices.
%These conditions transpire to be continuity of the function $u$ at all vertices of the graph, which one might expect given that we claim to be working with (some measure-theoretic notion of) differentiable functions and Sobolev spaces.
%We also find that the ``gradient" of the function $u$ at the vertices vanishes at the vertices $v_j\in\vertSet$, however the incoming directional derivatives along the edges connecting to $v_j$ need not have limit zero as they approach $v_j$.
The functions $u\in\gradSobQM{\ddom}{\dddmes}$ and their gradients $\tgrad_{\dddmes}u$ can be thought of as possessing the following properties (precise statements can be found in appendices \ref{app:muAnalysis}-\ref{app:SumMeasureAnalysis}):
\begin{enumerate}[(a)]
	\item The function $u$ is continuous at all vertices $v_j\in\vertSet$.
	\item On each edge $I_{jk}\in\edgeSet$; $\tgrad_{\dddmes}u = \tgrad_{\lambda_{jk}}u$, with $\tgrad_{\lambda_{jk}}u$ in turn amounting to the directional derivative of $u_{jk}$ along the edge $I_{jk}$.
	\item At each vertex $v_j\in\vertSet$, we have $\tgrad_{\dddmes}u=0$, however $\lim_{x\rightarrow v_j}\tgrad_{\lambda_{jk}}u$ need not be zero.
\end{enumerate}

We can now provide a conceptual argument for how a system of the form \eqref{eq:QGFullSystem} arises from \eqref{eq:PeriodCellLaplaceStrongForm}.
We write $\dddmes$ as ``the sum of a singular measure supporting a graph and point mass measures supported at the vertices", that is
\begin{align*}
	\dddmes &= \ddmes + \nu, \qquad
	\nu = \sum_{v_j\in\vertSet}\alpha_j \delta_j,
\end{align*}
where $\ddmes$ is the singular measure supporting an embedded, periodic graph $\graph$; and for each $v_j\in\vertSet$, $\delta_j$ is a point-mass measure centred on the vertex $v_j$, and $\alpha_j$ is the coupling constant attached to the vertex $v_j$.
A function $u\in\gradSobQM{\ddom}{\dddmes}$ is said to be a solution to \eqref{eq:PeriodCellLaplaceStrongForm} if
\begin{align} \label{eq:PeriodCellLaplaceWeakForm}
	\integral{\ddom}{\tgrad_{\dddmes}u\cdot\overline{\tgrad_{\dddmes}\phi}}{\dddmes} &= \omega^2\integral{\ddom}{u\overline{\phi}}{\dddmes}, \quad\forall \phi\in\smooth{\ddom},
\end{align}
where for $a=\bracs{a_1,a_2,a_3}^\top, b=\bracs{b_1,b_2,b_3}^\top\in\complex^3$, we use the notation $a\cdot b = \sum_{j=1}^{3}a_j b_j$.
We first note that whenever the equality in \eqref{eq:PeriodCellLaplaceWeakForm} holds for all smooth functions $\phi$, it also holds for all smooth functions $\psi$ whose support intersects the interior of a single edge $I_{jk}\in\edgeSet$ and no other parts of $\graph$.
Combined with the fact that $\dddmes$ is a sum of the edge measures and point masses at the vertices, and that $\tgrad_{\dddmes}u=\tgrad_{\lambda_{jk}}u$ on the edge $I_{jk}$, the identity \eqref{eq:PeriodCellLaplaceWeakForm} implies
\begin{align*}
	0 &= \integral{\ddom}{ \bracs{\tgrad_\ddmes u \cdot \overline{\tgrad\psi} - \omega^2 u\overline{\psi}} }{\ddmes}
	= \integral{I_{jk}}{ \bracs{\tgrad_{\lambda_{jk}}u \cdot \overline{\tgrad\psi} - \omega^2 u_{jk}\overline{\psi}} }{\lambda_{jk}} \\
	&= \integral{I_{jk}}{ \clbracs{ \bracs{u_{jk}' + \rmi\bracs{R_{jk}\qm}_1 u_{jk}}\bracs{\overline{\psi}' - \rmi\bracs{R_{jk}\qm}_1 \overline{\psi} } - \omega^2 u_{jk}\overline{\psi} } }{\lambda_{jk}},
\end{align*}
where $R_{jk}$ is the rotation introduced in assumption \ref{ass:MeasTheoryProblemSetup}.
Now using the mapping $r_{jk}$ given by \eqref{eq:EdgeParameterisation} as a change of variables and denoting $\tilde{u} = u \circ r_{jk}$ and $\varphi = \psi\circ r_{jk}$ we arrive at
\begin{align*}
	0 &= \int_{0}^{\abs{I_{jk}}} \bracs{\tilde{u}_{jk}' + \rmi\bracs{R_{jk}\qm}_1 \tilde{u}_{jk}}\bracs{\overline{\varphi}' - \rmi\bracs{R_{jk}\qm}_1 \overline{\varphi} } - \omega^2 \tilde{u}_{jk}\overline{\varphi} \ \md x .
\end{align*}
This holds for all smooth $\varphi$ with support contained in the interior of $\interval{I_{jk}}$, and can be thought of as the weak form of the equation
\begin{align*}
	-\bracs{\diff{}{x} + \rmi\qm_{jk}}^2 \tilde{u}_{jk} &= \omega^2 \tilde{u}_{jk}, \quad x\in\interval{I_{jk}},
\end{align*}
where $\qm_{jk} := \bracs{R_{jk}\qm}_1$.

Now we turn our attention to the derivation of vertex conditions.
Fix a vertex $v_j\in \vertSet$, and consider functions $\psi\in\smooth{\ddom}$ whose support only intersects $\graph$ in a neighbourhood of $v_j$ that only contains edges which connect to $v_j$ (which can be, for example, a ball of sufficiently small radius centred on $v_j$).
Then we can work from \eqref{eq:PeriodCellLaplaceWeakForm} to obtain
\begin{align*}
	0 &= \sum_{k: \ k\con j} \integral{I_{jk}}{ \bracs{ \tgrad_\ddmes u \cdot \overline{\tgrad\psi} - \omega^2 u\overline{\psi} } }{\lambda_{jk}} 
	+ \integral{\ddom}{ \bracs{ \tgrad_{\dddmes}u\cdot\overline{\tgrad_{\dddmes}\psi}-\omega^2 u\overline{\psi} } }{\nu} \\
	&= \sum_{k: \ k\con j} \int_{0}^{\abs{I_{jk}}} \clbracs{ \bracs{\tilde{u}_{jk}' + \rmi\bracs{R_{jk}\qm}_1 \tilde{u}_{jk}}\bracs{\overline{\varphi}' - \rmi\bracs{R_{jk}\qm}_1 \overline{\varphi} } - \omega^2 \tilde{u}_{jk}\overline{\varphi} } \ \md x
	+ \alpha_j\left.\bracs{ \tgrad_{\dddmes}u\cdot\overline{\tgrad_{\dddmes}\psi}-\omega^2 u\overline{\psi} }\right\vert_{v_j} \\
	&= \sum_{k: \ k\con j} \int_{0}^{\abs{I_{jk}}} \clbracs{ \bracs{\tilde{u}_{jk}' + \rmi\qm_{jk} \tilde{u}_{jk}}\bracs{\overline{\varphi}' - \rmi\qm_{jk} \overline{\varphi} } - \omega^2 \tilde{u}_{jk}\overline{\varphi} } \ \md t
	 - \alpha_j \omega^2 u\bracs{v_j}\overline{\psi}\bracs{v_j}.
\end{align*}
Here, we have again used $r_{jk}$ as a change of variables (and the same notation for the transforms of $u$ and $\psi$), and the results of the appendix \ref{app:SumMeasureAnalysis} inform us that $\tgrad_{\dddmes}$ vanishes at the vertex $v_j$.
Under the assumption that $\tilde{u}_{jk}$ can be differentiated again, this implies
\begin{align*}
	\alpha_j\omega^2 u\bracs{v_j}\overline{\psi}\bracs{v_j} 
	&= - \sum_{k: \ k\con j} \int_{0}^{\abs{I_{jk}}} \bracs{ \bracs{\diff{}{x} + \rmi\qm_{jk}}^2 \tilde{u}_{jk} +\omega^2 \tilde{u}_{jk} }\overline{\varphi} \ \md x
	+ \sum_{k: \ k\con j}\overline{\varphi}\bracs{v_j}\bracs{\pdiff{}{n} + \rmi\qm_{jk}}\tilde{u}_{jk}\bracs{v_j} \\
	&= \overline{\varphi}\bracs{v_j}\sum_{k: \ k\con j}\bracs{\pdiff{}{n} + \rmi\qm_{jk}}\tilde{u}_{jk}\bracs{v_j}. \labelthis\label{eq:DerivationVertexConditionWeak}
\end{align*}
Here we use the notation for the ``shifted, signed derivative",
\begin{align*}
	\bracs{\pdiff{}{n}+ \rmi\qm}u_{jk}\bracs{v_j} &= -\bracs{u_{jk}' + \rmi\qm u_{jk}}\bracs{v_j} = -\lim_{x\rightarrow0} \bracs{ u_{jk}'(x)+\rmi\qm u_{jk}(x) }, \\
	\bracs{\pdiff{}{n}+ \rmi\qm}u_{jk}\bracs{v_k} &= \bracs{u_{jk}' + \rmi\qm u_{jk}}\bracs{v_k} = \lim_{x\rightarrow l_{jk}} \bracs{ u_{jk}'(x)+\rmi\qm u_{jk}(x) },
\end{align*}
as is appropriate after use of the Gelfand transform.
Given that \eqref{eq:DerivationVertexConditionWeak} holds for every smooth $\varphi$, and that $\overline{\varphi}\bracs{v_j}=\overline{\psi}\bracs{v_j}\neq 0$, we arrive at the condition that
\begin{align*}
	\alpha_j\omega^2 u\bracs{v_j} &= \sum_{j\con k}\bracs{\pdiff{}{n} + \rmi\qm_{jk}}\tilde{u}_{jk}\bracs{v_j}, \quad \forall v_j \in \vertSet.
\end{align*}
Repeating the argument for each $v_j\in \vertSet$ then provides us with a condition of this form at each vertex.
One should note the presence of $\omega^2$ in this equation, so this is not a standard Robin condition on the derivatives of the edge-wise components of $u$, but rather indicates that our problem belongs to the class of problems with generalised resolvents, as mentioned in section \ref{sec:QuantumGraphs}.
The result of theorem \ref{thm:CharOfGradSob} tells us that functions $u\in\gradSobQM{\ddom}{\dddmes}$ are also continuous at each vertex $v_j$, and thus the following problem (precisely \eqref{eq:QGFullSystem}) has been derived:
\begin{align*}
	-\bracs{\diff{}{x} + \rmi\qm_{jk}}^2 \tilde{u}_{jk} = \omega^2 \tilde{u}_{jk}, &\quad x\in\interval{I_{jk}}, \quad \forall I_{jk}\in \edgeSet, \\
	u \text{ is continuous at each } &v_j \in \vertSet, \\
	\sum_{k: \ k\con j}\bracs{\pdiff{}{n} + \rmi\qm_{jk}}\tilde{u}_{jk}\bracs{v_j} &= \omega^2\alpha_{j}u\bracs{v_j},  \quad \forall v_j \in \vertSet.
\end{align*}
Solving for the eigenvalues $\omega^2$ will net us the eigenvalues of our original problem \eqref{eq:PeriodCellLaplaceStrongForm}, and taking the union of the eigenvalues over $\qm$ will provide the spectrum of \eqref{eq:WholeSpaceLaplaceEqn}.
As will be made clear in the discussion that follows, the quantum graph problem \eqref{eq:QGFullSystem} is much easier to handle both analytically and numerically thanks to the utility of the $M$-matrix.