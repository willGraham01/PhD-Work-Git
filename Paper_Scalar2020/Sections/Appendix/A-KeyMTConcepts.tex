\section{Appendix: Key Measure Theoretic Concepts} \label{app:MeasureTheory}
In this section we address the question of how one understands the equations \eqref{eq:WholeSpaceLaplaceEqn} and \eqref{eq:PeriodCellLaplaceStrongForm}, by introducing the relevant differential operators and function spaces.
Attempting to pose a boundary-value problem on a singular-structure, by drawing analogy to the ``ingredients" of a boundary-value problem on a thin structure, runs into problems.
These are due to the singular structure lacking a domain interior from the perspective of the space it is embedded in, so the notion of boundary values ceases to make sense.
This issue is resolved by not abandoning what we believe are the ingredients of a boundary-value problem, but rather by reworking our concepts of integration and differentiation so that they respect the fact that we are looking at a problem on the singular structure itself.
As we will be working with each of the measures $\dddmes, \ddmes$, and $\nu$ above individually before combining our knowledge of each, what we present here is stated in terms of a generic measure $\rho$.
Our approach below is motivated by \cite{zhikov2000extension} and \cite{zhikov2002homogenization}.

First, let us introduce the notation for this section.
Suppose $\rho$ is a Borel measure on 
\begin{align*}
	\ddom:= \left[0,T_1\right)\times\left[0,T_2\right), 
\end{align*}
which can be thought of as the period cell of an embedded graph $\graph$.
For 
\begin{align*}
	\qm=\bracs{\qm_1,\qm_2}\in\left[-\frac{\pi}{T_1},\frac{\pi}{T_1}\right)\times\left[-\frac{\pi}{T_2},\frac{\pi}{T_2}\right)
\end{align*}
define the ``shifted" gradient operator $\tgrad$ on smooth functions $\phi\in\smooth{\ddom}$ by
\begin{align*}
	\tgrad\phi &= \begin{pmatrix} \partial_1\phi + \rmi\qm_1\phi \\ \partial_2\phi + \rmi\qm_2\phi \end{pmatrix}.
\end{align*}
Recall that we pose \eqref{eq:WholeSpaceLaplaceEqn} on a periodic, embedded graph $\graph$ in $\reals^2$, and the operator $\tgrad$ arises after using a Gelfand transform, to move us to a family of problems on the period cell of $\graph$.

Denote the set of smooth functions on $\ddom$ by $\smooth{\ddom}$, and let
\begin{align*} %\label{eq:WSetDefinition}
	W = W^{\qm}\bracs{\ddom,\rho} &:= \overline{\clbracs{\bracs{\phi,\tgrad\phi} \setVert \phi\in\smooth{\ddom}}} \quad \text{in} \ \ltwo{\ddom}{\rho}\times\ltwo{D}{\rho}^2.
\end{align*}
The idea here is to construct an analogy of a Sobolev space for the measure $\rho$, and hence obtain a concept of (weak) derivative.
Notice that if $\bracs{u,g_1},\bracs{0,g_2}\in W$ then clearly $\bracs{u, g_1+g_2}\in W$ too.
We define the ``set of $\rho$-gradients of zero" as
\begin{align} \label{eq:GradZeroDef}
	\gradZero{\ddom}{\rho} &= \clbracs{ g\in\ltwo{\ddom}{\rho}^2 \setVert \bracs{0,g}\in W}, \\
	&= \clbracs{ g\in\ltwo{\ddom}{\rho}^2 \setVert \exists\phi_n\in\smooth{\ddom} \text{ such that } \phi_n\lconv{\ltwo{\ddom}{\rho}} 0, \tgrad\phi_n\lconv{\ltwo{\ddom}{\rho}^2} g }
\end{align}
which is a closed, linear subspace of $\ltwo{D}{\rho}^2$. 
It can be shown that $\gradZero{D}{\rho}$ does not depend on the value of $\qm$, which is why the notation lacks a $\qm$ symbol:
\begin{prop} \label{prop:GradZeroInvarientUnderQM}
	For any fixed $\qm\in[-\pi,\pi)^2$, and with
	\begin{align*}
		W^0 &= \overline{\clbracs{\bracs{\phi,\grad\phi} \setVert \phi\in\smooth{\ddom}}} \quad \mathrm{in} \ \ltwo{\ddom}{\rho}\times\ltwo{\ddom}{\rho}^2,
	\end{align*}	
	we have
	\begin{align*}
		\mathcal{G}^0_{\ddom, \md \rho} := \clbracs{ g \ \vert \ \bracs{0,g}\in W^0} &= 
		\clbracs{ g \ \vert \ \bracs{0,g}\in W} =: \mathcal{G}^{\qm}_{\ddom, \md \rho}.
	\end{align*}
\end{prop}
\begin{proof}
	Given that $\tgrad\phi = \grad\phi + \rmi\qm\phi$ for smooth $\phi$, if $g\in\mathcal{G}^0_{\ddom, \md \rho}$ there exists a sequence of smooth functions $\phi_n$ such that 
	\begin{align*}
		\phi_n\lconv{\ltwo{\ddom}{\rho}}0, &\quad \grad\phi_n\lconv{\ltwo{\ddom}{\rho}^2}g.
	\end{align*}
	But then clearly
	\begin{align*}
		\phi_n\lconv{\ltwo{\ddom}{\rho}}0, &\quad
		\tgrad\phi_n = \grad\phi_n + \rmi\qm\phi_n \lconv{\ltwo{\ddom}{\rho}^2} g + 0 = g,
	\end{align*}
	so $g\in\mathcal{G}^\qm_{\ddom, \md \rho}$.
	The proof of the opposite inclusion is similar.
\end{proof}
The illustration of the non-uniqueness of gradients earlier employed the fact that we can always add an element of $\gradZero{\ddom}{\rho}$ to the second member of a pair $\bracs{u,z}\in W$ and produce another element of $W$.

For each $u\in\ltwo{\ddom}{\rho}$ there exists a $\tgrad_\rho u\in\gradZero{\ddom}{\rho}^{\perp}$ such that any pair $\bracs{u,z}\in W$ can be written as $\bracs{u,\tgrad_\rho u + g}$ where $g\in\gradZero{\ddom}{\rho}$.
We call $\tgrad_\rho u$ the \emph{($\rho$-)tangential gradient} of $u$ and it is unique in this sense (see \cite[Section~9]{zhikov2000extension}); which allows us to construct the ``Sobolev space"
\begin{align*}
	\gradSobQM{\ddom}{\rho} &= \clbracs{ \bracs{u, \tgrad_\rho u}\in W \ \vert \ \tgrad_\rho u \in \gradZero{\ddom}{\rho}^{\perp} }.
\end{align*}
Since $\grad_\rho u$ is unique, we will use the shorthand $u\in\gradSobQM{\ddom}{\rho}$ to refer to the pair $\bracs{u, \grad_\rho u}$.
We can now precisely state what we mean when we write equation \eqref{eq:PeriodCellLaplaceStrongForm}; a pair $u\in\gradSobQM{\ddom}{\dddmes}$ solves \eqref{eq:PeriodCellLaplaceStrongForm} if and only if (see \eqref{eq:PeriodCellLaplaceWeakForm})
\begin{align*}
	\integral{\ddom}{\tgrad_{\dddmes}u\cdot\overline{\tgrad_{\dddmes}\phi}}{\dddmes} &= \omega^2\integral{\ddom}{u\overline{\phi}}{\dddmes}, \quad\forall \phi\in\smooth{\ddom},
\end{align*}
and one has a similar interpretation for \eqref{eq:WholeSpaceLaplaceEqn}.
We will continue to use \eqref{eq:PeriodCellLaplaceStrongForm} as shorthand for \eqref{eq:PeriodCellLaplaceWeakForm} to save on notational clutter and maintain readability, however the steps of our derivation of \eqref{eq:QGFullSystem} from \eqref{eq:PeriodCellLaplaceStrongForm} will require us to work with \eqref{eq:PeriodCellLaplaceWeakForm} directly.

In order to demonstrate how \eqref{eq:PeriodCellLaplaceStrongForm} reduces to the system \eqref{eq:QGFullSystem}, we must understand the properties of the functions (and their gradients) in the space $\gradSobQM{\ddom}{\ddmes}$, which is the focus of sections \ref{app:muAnalysis} through \ref{app:SumMeasureAnalysis}.
This begins by examining $\gradSobQM{\ddom}{\ddmes}$, which will then determine the behaviour of functions on the singular-structure.
The space $\gradSobQM{\ddom}{\nu}$ is examined next, before concluding with section \ref{app:SumMeasureAnalysis} in which we describe the functions in $\gradSobQM{\ddom}{\dddmes}$.