\section{Appendix: Key Measure Theoretic Concepts} \label{app:MeasureTheory}
In this section we address the question of how one understands the equations \eqref{eq:WholeSpaceLaplaceEqn} and \eqref{eq:PeriodCellLaplaceStrongForm}, by outlining the appropriate differential operators and function spaces that are required.
Attempting to pose a boundary-value problem on a singular-structure, by drawing analogy to the ``ingredients" of a boundary-value problem on a thin structure, runs into problems.
These are due to the singular structure lacking a domain interior from the perspective of the space it is embedded in, and thus the notion of boundary values ceases to make sense, and this problem persists even after taking a Gelfand transform to bring us to a family of problems on the period cell of the graph.
This issue is resolved by not abandoning what we believe are the ingredients of a boundary-value problem, but rather by reworking our concepts of integration and differentiation so that they respect the fact that we are looking at a problem on the singular-structure itself, despite the fact that it is embedded in a higher-dimensional space.
As we will be working with each of the measures $\dddmes, \ddmes$, and $\nu$ above individually before combining our knowledge of each, what we present here is stated in terms of a measure $\eta$ which can be thought of as representing any of the three aforementioned measures.
Our defintions and the more general properties of the spaces presented here are motivated by the work of \cite{zhikov2000extension} and \cite{zhikov2002homogenization}. \newline

First, let us introduce the notation for this section.
Let $\eta$ be any of the (Borel) measures $\dddmes,\ddmes,\nu$ mentioned above, on a domain $\ddom\subset\reals^2$ whose closure is $\sqbracs{0,T_1}\times\sqbracs{0,T_2}$ ($\ddom$ should be thought of as the period cell of an embedded graph $\graph$).
For $\qm=\bracs{\qm_1,\qm_2}\in\left[-\frac{\pi}{T_1},\frac{\pi}{T_1}\right)\times\left[-\frac{\pi}{T_2},\frac{\pi}{T_2}\right)$ define the shifted gradient operator $\tgrad$ on smooth functions $\phi\in\smooth{\ddom}$ by
\begin{align*}
	\tgrad\phi &= \begin{pmatrix} \partial_1\phi + i\qm_1\phi \\ \partial_2\phi + i\qm_2\phi \end{pmatrix}.
\end{align*}
Recall that we pose \eqref{eq:WholeSpaceLaplaceEqn} on a periodic, embedded graph $\graph$ in $\reals^2$, and the operator $\tgrad$ arises after using a Gelfand transform to move us to a family of problems on the period cell of $\graph$. \newline

Denote the set of smooth functions on $D$ by $\smooth{D}$, and then let $W=W\bracs{D,\mathrm{d}\eta}$ be the closure of the set of pairs $\bracs{\phi,\tgrad\phi}$ in $\ltwo{D}{\eta}\times\ltwo{D}{\eta}^2$, where $\phi\in\smooth{D}$.
That is
\begin{align} \label{eq:WSetDefinition}
	W = W^{\qm}\bracs{D,\eta} &= \overline{\clbracs{\bracs{\phi,\tgrad\phi} \ \vert \ \phi\in\smooth{D}}} \quad \text{in} \ \ltwo{D}{\eta}\times\ltwo{D}{\eta}^2.
\end{align}
The idea here is to construct an analogy of a Sobolev space for the measure $\eta$, and hence obtain a concept of (weak) derivative.
This is the reason for us taking the closure of the set of pairs $\bracs{\phi, \tgrad\phi}$ in \eqref{eq:WSetDefinition}, one hopes that if a sequence of smooth functions converges in $L^2$ to $u$, then their gradients also converge in $L^2$ to a function that represents the gradient of $u$ in a sense that is consistent with the measure $\eta$.
Regrettably this does not turn out to be as simple a process as we would hope; we can immediately notice that if $\bracs{u,g_1},\bracs{0,g_2}\in W$ then clearly $\bracs{u, g_1+g_2}\in W$ too.
This means that if $\bracs{u,g}\in W$ we cannot call $g$ ``the gradient" of $u$, which would imply that $W$ isn't quite the set of functions we should be working with.
However $W$ does contain the functions we should be working with, we just need to make sense of this ``non-uniqueness of gradients" issue. \newline

We define the ``set of $\eta$-gradients of zero" as
\begin{align} \label{eq:GradZeroDef}
	\gradZero{D}{\eta} &= \clbracs{ g \ \vert \ \bracs{0,g}\in W}, \\
	&= \clbracs{ g \ \vert \ \exists\phi_n\in\smooth{D} \text{ s.t. } \phi_n\lconv{\ltwo{D}{\eta}}0, \tgrad\phi_n\lconv{\ltwo{D}{\eta}^2}g }
\end{align}
which is a closed, linear subspace of $\ltwo{D}{\eta}^2$. 
It can be shown that $\gradZero{D}{\eta}$ does not depend on the value of $\qm$, which is why the notation lacks a $\qm$ symbol:
\begin{prop}[Gradients of Zero are Invariant Under Quasi-Momentum] \label{prop:GradZeroInvarientUnderQM}
	For any fixed $\qm\in[-\pi,\pi)^2$, and with
	\begin{align*}
		W^0 &= \overline{\clbracs{\bracs{\phi,\grad\phi} \ \vert \ \phi\in\smooth{D}}} \quad \text{in} \ \ltwo{D}{\eta}\times\ltwo{D}{\eta}^2,
	\end{align*}	
	we have that
	\begin{align*}
		\mathcal{G}^0_{D, \md \eta} := \clbracs{ g \ \vert \ \bracs{0,g}\in W^0} &= 
		\clbracs{ g \ \vert \ \bracs{0,g}\in W} =: \mathcal{G}^{\qm}_{D, \md \eta}.
	\end{align*}
\end{prop}
\begin{proof}
	Given that $\tgrad\phi = \grad\phi + i\qm\phi$ for smooth $\phi$, if $g\in\mathcal{G}^0_{D, \md \eta}$ there exists a sequence of smooth functions $\phi_n$ such that 
	\begin{align*}
		\phi_n\lconv{\ltwo{D}{\eta}}0, &\grad\phi_n\lconv{\ltwo{D}{\eta}^2}g.
	\end{align*}
	But then clearly
	\begin{align*}
		\phi_n\lconv{\ltwo{D}{\eta}}0, &\quad
		\tgrad\phi_n = \grad\phi_n + i\qm\phi_n \lconv{\ltwo{D}{\eta}^2} g + 0 = g,
	\end{align*}
	so $g\in\mathcal{G}^\qm_{D, \md \eta}$.
	The proof of the opposite inclusion is similar.
\end{proof}
The illustration of the non-uniqueness of gradients earlier employed the fact that we can always add an element of $\gradZero{D}{\eta}$ to the second member of a pair $\bracs{u,z}\in W$ and produce another element of $W$.
Colloquially, we can always ``add a gradient of zero" to an existing ``gradient", which will produce another ``gradient".
So whilst a function $u$ might not have a unique gradient, it may at least have only one ``\emph{important}" gradient. \newline

For each $u\in\ltwo{D}{\eta}$ there exists a $\tgrad_\eta u\in\gradZero{D}{\eta}^{\perp}$ such that any pair $\bracs{u,z}\in W$ can be written as $\bracs{u,\tgrad_\eta u + g}$ where $g\in\gradZero{D}{\eta}$.
We call $\tgrad_\eta u$ the \emph{($\eta$-)tangential derivative} of $u$ and it is unique in this sense (see \cite[Section~9]{zhikov2000extension}); which allows us to construct the ``Sobolev space"
\begin{align*}
	\gradSobQM{D}{\eta} &= \clbracs{ \bracs{u, \tgrad_\eta u}\in W \ \vert \ \tgrad_\eta u \in \gradZero{D}{\eta}^{\perp} }.
\end{align*}
Since $\grad_\eta u$ is unique, we will use the shorthand $u\in\gradSobQM{D}{\eta}$ to refer to the pair $\bracs{u, \grad_\eta u}$.
We can now precisely state what we mean when we write equation \eqref{eq:PeriodCellLaplaceStrongForm}; a pair $u\in\gradSobQM{\ddom}{\dddmes}$ solves \eqref{eq:PeriodCellLaplaceStrongForm} if and only if (see \eqref{eq:PeriodCellLaplaceWeakForm})
\begin{align*}
	\integral{\ddom}{\tgrad_{\dddmes}u\cdot\overline{\tgrad_{\dddmes}\phi}}{\dddmes} &= \omega^2\integral{\ddom}{u\overline{\phi}}{\dddmes}, \quad\forall \phi\in\smooth{\ddom},
\end{align*}
and one has a similar interpretation for \eqref{eq:WholeSpaceLaplaceEqn}.
We will continue to use \eqref{eq:PeriodCellLaplaceStrongForm} as shorthand for \eqref{eq:PeriodCellLaplaceWeakForm} to save on notational clutter and maintain readability, however the steps of our derivation of \eqref{eq:QGFullSystem} from \eqref{eq:PeriodCellLaplaceStrongForm} will require us to work with \eqref{eq:PeriodCellLaplaceWeakForm} directly. \newline

The space $\gradSobQM{D}{\eta}$ gives us a concept of (weak) derivative to work with, however this is as far as the general theory takes us.
In order to demonstrate how \eqref{eq:PeriodCellLaplaceStrongForm} reduces to the system \eqref{eq:QGFullSystem}, we must understand the properties of the functions (and their gradients) that live in the space $\gradSobQM{\ddom}{\ddmes}$.
This means that there is work to be done towards understanding the form of the tangential derivatives $\tgrad_{\dddmes}u$, which is the focus of sections \ref{app:muAnalysis} through \ref{app:SumMeasureAnalysis}.
We will begin with an examination of the space $\gradSobQM{\ddom}{\ddmes}$ on (the period graph of) an embedded graph $\graph\subset\sqbracs{0,1}^2$, which will inform us about the behaviour of functions on the edges of our singular-structure.
Then we shall proceed to analyse the space $\gradSobQM{\ddom}{\nu}$, before combining our knowledge in section \ref{app:SumMeasureAnalysis} to understand the properties of functions in $\gradSobQM{\ddom}{\dddmes}$.
As the arguments are fairly long and cumbersome in terms of readability, we present a narrative overview (with appropriate detail) in sections \ref{app:muAnalysis} through \ref{app:SumMeasureAnalysis}, and relegate longer proofs to section \tstk{create final appendix section for the long, full version of all the long-winded proofs}

%
%%old conf report stuff
%
%With this in mind, consider the following variational problem; find a pair $\bracs{u,z}\in W$ such that
%\begin{align} \label{eq:TangentialGradientVariationalMotivation}
%	\integral{D}{z \cdot \grad\overline{\phi} - u\overline{\phi}}{\nu} &= 0 \quad \forall\phi\in\smooth{D}.
%\end{align}
%Setting aside questions of existence and uniqueness of solutions to this problem for the purposes of illustration, take an element $g\in\gradZero{D}{\nu}$, and an approximating sequence $\phi_n$ as in the definition \eqref{eq:GradZeroSequenceDefinition}.
%Substituting $\phi_n$ into \eqref{eq:TangentialGradientVariationalMotivation} and employing a density-argument, we can see that
%\begin{align*}
%	\integral{D}{z \cdot g}{\nu} &= 0 \quad \forall g\in\gradZero{D}{\nu}.
%\end{align*}
%Namely that the member $z$ of the solution pair is orthogonal (in the $\ltwo{D}{\nu}$-norm) to $\gradZero{D}{\nu}$.
%Stepping back, as $\gradZero{D}{\nu}$ is a closed linear subspace of $\ltwo{D}{\nu}^N$ we can decompose $\ltwo{D}{\nu}^N$ as
%\begin{align*}
%	\ltwo{D}{\nu}^N &= \gradZero{D}{\nu}^{\perp} \oplus \gradZero{D}{\nu}.
%\end{align*}
%But we have just seen that the member of the solution pair $z\in\gradZero{D}{\nu}^{\perp}$, hence $z$ is the \textit{unique} element of $\gradZero{D}{\nu}^{\perp}$ such that every pair $\bracs{u,\tilde{z}}\in W$ can be written as $\bracs{u,z+g}\in W$ for some $g\in\gradZero{D}{\nu}$.
%Thus whilst the concept of a unique gradient does not make sense in this setting, once we consider variational problems we can obtain the concept of the \textit{tangential} gradient $z$, which is unique.
%We therefore adopt the notation $z := \grad_\nu u$ for the second member of the solution pair $\bracs{u,z}\in W$, and can now define the (non-classical) ``Sobolev space" $\gradSob{D}{\nu}$.
%\begin{definition}[Sobolev Space $\gradSob{D}{\nu}$] \label{def:GradSobGeneral}
%	For $D$, $\nu$, and $W$ as in \eqref{eq:WSetDefinition} we define
%	\begin{align*}
%		\gradSob{D}{\nu} &:= \clbracs{\bracs{u,\grad_\nu u}\in W \ \vert \ \grad_\nu u \perp \gradZero{D}{\nu}}.
%	\end{align*}
%	to be the \emph{(non-classical) Sobolev space (on $D$) with respect to $\nu$}.
%\end{definition}
%
%The definition of $\gradSob{D}{\nu}$ now means that we have an appropriate function space to pose variational problems such as \eqref{eq:TangentialGradientVariationalMotivation} in; this problem in particular is that of finding $\bracs{u,\grad_\nu u}\in\gradSob{D}{\nu}$ such that
%\begin{align} \label{eq:GeneralVariationProblem}
%	\integral{D}{\grad_\nu u \cdot \grad\overline{\phi} - u\overline{\phi}}{\nu} &= 0 \quad \forall\phi\in\smooth{D}.
%\end{align}
%In general, we use the following conventions for shorthand when dealing with variational problems.
%\begin{convention}[Shorthand for Elements of $\gradSob{D}{\nu}$] \label{conv:ShorthandSobSpaceElements}
%	As $\grad_\nu u$ is unique for each $u$ it is sufficient to specify only the element $u$ to identify the pair $\bracs{u,\grad_\nu u}\in\gradSob{D}{\nu}$, so henceforth we adopt the shorthand $u\in\gradSob{D}{\nu}$ when referring to this element.
%\end{convention}
%\begin{convention}[Shorthand for Variational Problems] \label{conv:ShorthandGradVariationalProblems}
%	In general for $f\in\ltwo{D}{\nu}$ and an elliptic matrix $A(x)$, we say that $u\in\gradSob{D}{\nu}$ is a solution to the (elliptic) equation
%	\begin{align} \label{eq:GeneralScalarStrongForm}
%		-\grad_\nu \cdot \bracs{A(x)\grad_\nu u(x)} &= f(x), \quad x\in D
%	\end{align}
%	to mean that $u\in\gradSob{D}{\nu}$ solves the variational problem
%	\begin{align} \label{eq:GeneralScalarWeakForm}
%		\integral{D}{A\grad_\nu u \cdot \grad\overline{\phi}}{\nu} &= \integral{D}{f\overline{\phi}}{\nu}, \quad \forall \phi\in\smooth{D}.
%	\end{align}
%	In the case when $A$ is the identity we shall write $\grad_{\nu}^2 u = \grad_{\nu}\cdot\grad_{\nu} u$.
%\end{convention}
%Of course \eqref{eq:GeneralScalarStrongForm} is just notation and has no meaning without \eqref{eq:GeneralScalarWeakForm}; and the convention \ref{conv:ShorthandGradVariationalProblems} is made because it is nice to envisage \eqref{eq:GeneralScalarWeakForm} as somehow being the weak form of \eqref{eq:GeneralScalarStrongForm} for analogy with classical Sobolev spaces and variational problems, however there is no formal link in this context because of the lack of the ``integration by parts" technique.
%Existence and uniqueness of the solution pair $\bracs{u,\grad_\nu u}$ is guaranteed though, by appealing to the Riesz Representation theorem and the bilinear form defined by \eqref{eq:GeneralScalarWeakForm}.
%One can (see \cite{zhikov2000extension}) establish that the $\grad_\nu u$ in the solution pair coincides with the unique gradient of $u$ such that $A\grad_\nu u \perp \gradZero{D}{\nu}$, highlighting that understanding $\gradZero{D}{\nu}$ is crucial to determining solutions to \eqref{eq:GeneralScalarStrongForm}.
%For completeness, we also note that the spectral problem (when $f$ is replaced by $\lambda u$ for some eigenvalue $\lambda\in\complex$) is written as
%\begin{align*}
%	-\grad_\nu \cdot \bracs{A(x)\grad_\nu u(x)} &= \lambda u(x), \quad x\in D
%\end{align*}
%which is interpreted as 
%\begin{align*}
%	\integral{D}{A\grad_\nu u \cdot \grad\overline{\phi}}{\nu} &= \lambda\integral{D}{u\overline{\phi}}{\nu} \quad \forall \phi\in\smooth{D}.
%\end{align*}
%In which case solutions (if they exist) are triplets $\bracs{\lambda, u, \grad_\nu u}$, although it suffices to specify $\bracs{\lambda, u}$ only. \newline
%
%To conclude this section we present some general results that we will need later in our analysis.
%The first is a simple result which makes proving membership of $\gradSob{D}{\nu}$ easier, and is presented in \cite{zhikov2002homogenization} as lemma 5.5.
%\begin{prop}[Sufficient Condition for Membership of $\gradSob{D}{\nu}$] \label{prop:Lemma5-5}
%	Suppose we have a sequence $u_n\in\gradSob{D}{\nu}$ such that $u_n\rightarrow u$ in $\ltwo{D}{\nu}$ for some $u\in\ltwo{D}{\nu}$, and that the sequence of tangential gradients $\grad_\nu u_n$ is bounded in $\ltwo{D}{\nu}^N$.
%	Then 
%	\begin{align*}
%		u\in\gradSob{D}{\nu};
%	\end{align*}
%	that is, $\grad_{\nu} u$ exists and the pair $\bracs{u,\grad_{\nu} u}\in\gradSob{D}{\nu}$).
%\end{prop}
%\begin{proof}
%	As $\grad_\nu u_n$ is bounded in $\ltwo{D}{\nu}^N$ it has a convergent subsequence that we denote by $\grad_\nu u_{n_k}$, and this sequence has some limit $v\in\ltwo{D}{\nu}^N$.
%	Note that since $\grad_\nu u_{n_k}\in\gradZero{D}{\nu}^\perp$ for all $k\in\naturals$, the limit $v\in\gradZero{D}{\nu}^\perp$ too.
%	Furthermore the corresponding subsequence $u_{n_k}$ of $u_n$ still converges in $\ltwo{D}{\nu}$ to $u$, being a subsequence of a convergent sequence.
%	Now for each $k\in\naturals$ there exists an sequence of smooth functions $\phi_{k,l}$ such that
%	\begin{align*}
%		\phi_{k,l}\lconv{\ltwo{D}{\nu}} u_{n_k}, &\quad \grad\phi_{k,l}\lconv{\ltwo{D}{\nu}^n}\grad u_{n_k}, \toInfty{l}.
%	\end{align*}
%	We can therefore apply a diagonal argument to find a sequence of smooth functions $\phi_{k, l_k}$ such that
%	\begin{align*}
%		\phi_{k,l_k}\lconv{\ltwo{D}{\nu}} u, &\quad \grad\phi_{k,l_k}\lconv{\ltwo{D}{\nu}^N}v, \toInfty{k}.
%	\end{align*}
%	Since $v\in\gradZero{D}{\nu}^\perp$, we conclude that $v = \grad_\nu u$ and thus
%	\begin{align*}
%		\bracs{u, \grad_\nu u}\in \gradSob{D}{\nu}.
%	\end{align*}
%\end{proof}
%
%The second result that we present is essentially the equivalent of the product rule for tangential gradients.
%\begin{prop}[Product Rule for Tangential Gradients] \label{prop:ProductRuleGradients}
%	Suppose $u,v\in\gradSob{D}{\nu}$.
%	Then we also have that
%	\begin{align*}
%		\bracs{uv, u\grad_\nu v + v\grad_\nu u}\in\gradSob{D}{\nu}.
%	\end{align*}
%\end{prop}
%\begin{proof}
%	Take smooth sequences $\phi_l, \psi_l$ such that
%	\begin{align*}
%		\phi_l \lconv{\ltwo{D}{\nu}} u, &\quad \grad\phi_l \lconv{\ltwo{D}{\nu}^N} \grad_\nu u, \\
%		\psi_l \lconv{\ltwo{D}{\nu}} v, &\quad \grad\psi_l \lconv{\ltwo{D}{\nu}^N} \grad_\nu v.
%	\end{align*}
%	Note that because of these convergences, the sequences 
%	\begin{align*}
%		\norm{\phi_l}_{\ltwo{D}{\nu}}, \norm{\psi_l}_{\ltwo{D}{\nu}}, \norm{\grad\phi_l}_{\ltwo{D}{\nu}^N}, \norm{\grad\psi_l}_{\ltwo{D}{\nu}^N}.
%	\end{align*}
%	are all bounded in $\reals$.
%	We now examine the sequence formed from term-wise products, $\phi_l\psi_l$.
%	Note that this remains a sequence of smooth functions, and we have that $\phi_l\psi_l \rightarrow uv \toInfty{l}$ by the Algebra of Limits in $\ltwo{D}{\nu}$.
%	Furthermore,
%	\begin{align*}
%		\recip{4}\integral{D}{\abs{ \grad\bracs{\phi_l\psi_l} - u\grad_\nu v - v\grad_\nu u }^2}{\nu}
%		&\leq \recip{2}\integral{D}{\abs{ \phi_l\grad\psi_l - u\grad_\nu v }^2}{\nu} \\
%		& \ + \recip{2}\integral{D}{\abs{ \psi_l\grad\phi_l - v\grad_\nu u }^2}{\nu} \\
%		&\leq \norm{\phi_l - u}_{\ltwo{D}{\nu}}^2\norm{\grad\psi_l}_{\ltwo{D}{\nu}^N}^2 \\
%		& \ + \norm{u}_{\ltwo{D}{\nu}}^2\norm{\grad\psi_l - \grad_\nu v}_{\ltwo{D}{\nu}^N}^2 \\
%		& \ + \norm{\psi_l - v}_{\ltwo{D}{\nu}}^2\norm{\grad\phi_l}_{\ltwo{D}{\nu}^N}^2 \\
%		& \ + \norm{v}_{\ltwo{D}{\nu}}^2\norm{\grad\phi_l - \grad_\nu u}_{\ltwo{D}{\nu}^N}^2 \\
%		&\rightarrow 0 \toInfty{l}.
%	\end{align*}
%	Finally we note that as $\grad_\nu u, \grad_\nu v \in\gradZero{D}{\nu}^\perp$, the function $u\grad_\nu v + v\grad_\nu u\in\gradZero{D}{\nu}^\perp$ too.
%	Since
%	\begin{align*}
%		\phi_l\psi_l \lconv{\ltwo{D}{\nu}} uv, &\quad \grad\bracs{\phi_l\psi_l} \lconv{\ltwo{D}{\nu}^N} u\grad_\nu v + v\grad_\nu u,
%	\end{align*}
%	we conclude that $\grad_\nu\bracs{uv} = u\grad_\nu v + v\grad_\nu u$ and
%	\begin{align*}
%		\bracs{uv, u\grad_\nu v + v\grad_\nu u}\in\gradSob{D}{\nu}.
%	\end{align*}
%\end{proof}