\section{Appendix: Analysis of the Sobolev Spaces Associated with the Measure $\nu$} \label{app:VertexAnalysis}
Understanding the behaviour of Sobolev functions on the edges of a graph is the bulk of the information which we need to determine the edge-wise action of the equivalent quantum graph problem that we plan to derive.
Our analysis of $\gradSobQM{\ddom}{\ddmes}$ even provides us with Sobolev functions that are continuous across the vertices.
However we are not encapsulating the effect of introducing coupling constants $\alpha_j$ to the vertices in $\graph$ yet - and to do so we must examine the ``vertex part" $\nu$ of our measure $\dddmes$ and the Sobolev spaces associated with it.
Thankfully, our analysis of the space $\gradSobQM{\ddom}{\nu}$ is short, on account of the fact that $\gradSobQM{\ddom}{\nu}$ is actually isomorphic \tstk{check with Kirill, as this normally implies we have a norm and I never introduced a norm on our Sob Spaces... I guess bijective is safer?} to $\complex^N$, where $N=\abs{\vertSet}$ is the number of vertices in the graph $\graph$.

\subsection{Analysis of $\nu$-Gradients of Zero} \label{appS:VertexGradZero}
If we fix $N = \abs{\vertSet}$ as the number of vertices, we will see that $\gradZero{\ddom}{\nu} \cong \complex^{2N}$, and thus that any $\bracs{u,\grad_{\nu}u}\in\gradSob{\ddom}{\nu}$ is such that $\grad_{\nu}u=0$.
\begin{definition}[$d$, $\varphi_c$, and $g^j$] \label{def:UsefulObjects}
	Let 
	\begin{align*}
		d=\recip{2}\min\clbracs{\norm{v_j-v_k}_2 \ \vert \ v_j,v_k\in\vertSet}
	\end{align*}
	be half the minimum distance between any two vertices in the graph (note that this may occur between two vertices that do not share a single edge).
	$d$ exists since the graph $\graph$ is assumed finite.
	For $c\in\complex$, let $\varphi_c:\reals^2\rightarrow\complex$ be the smooth function such that
	\begin{align*}
		\varphi_c\bracs{0} = 0, &\quad \grad\varphi_c\bracs{0} = c, \\
		\supp\bracs{\varphi_c} &\subset B_{d}\bracs{0},
	\end{align*}
	where $B_{d}\bracs{0}$ denotes the ball of radius $d$ centred at the origin.
	Finally, for each $v_j\in\vertSet$ define
	\begin{align*}
		g^j_1\bracs{x} &=
		\begin{cases}
			\begin{pmatrix} 1 \\ 0 \end{pmatrix} & x=v_j, \\
			0 & x\neq v_j. \\
		\end{cases}
		&\quad
		g^j_2\bracs{x} &=
		\begin{cases}
			\begin{pmatrix} 0 \\ 1 \end{pmatrix} & x=v_j, \\
			0 & x\neq v_j. \\
		\end{cases}
	\end{align*}
\end{definition}

Notice that we have the following result:
\begin{lemma}
	The space $\ltwo{\ddom}{\nu}$ is isomorphic to $\complex^{2N}$.
	Moreover, the collection 
	\begin{align*}
		\clbracs{g_1^j, g_2^j \ \vert \ v_j\in\vertSet}
	\end{align*}
	forms a basis of $\ltwo{\ddom}{\nu}^2$.
\end{lemma}
\begin{proof}
	It is sufficient to notice that any $f\in\ltwo{\ddom}{\nu}^2$ is entirely determined by the values it takes at the vertices $v_j$.
	Each of these values is a $\complex^2$-vector, and thus we may define the function
	\begin{align*}
		\iota:\ltwo{\ddom}{\nu}^2 \rightarrow\complex^{2N}, &\quad
		\iota\bracs{f} = \begin{pmatrix} f\bracs{v_1} \\ f\bracs{v_2} \\ \vdots \\ f\bracs{v_N} \end{pmatrix},
	\end{align*}
	where we vertically concatenate the collection of two-vectors $f\bracs{v_j}, v_j\in\vertSet$.
	Clearly $\iota$ is a bijection, and additionally for $f,g\in\ltwo{\ddom}{\nu}$ we have that
	\begin{align*}
		\integral{\ddom}{f\cdot \overline{g}}{\nu} &= \sum_{v_j\in\vertSet} f\bracs{v_j}\cdot\overline{g\bracs{v_j}} \\
		&= \iota\bracs{f}\cdot\overline{\iota\bracs{g}},
	\end{align*}
	so $\iota$ is an isometry.
	Moreover, the preimage of the canonical basis $\clbracs{e_k \ \vert \ k\in\clbracs{1,...,2N}}$ under $\iota$ is the collection $\clbracs{g_1^j, g_2^j \ \vert \ v_j\in\vertSet}$, and hence $\clbracs{g_1^j, g_2^j \ \vert \ v_j\in\vertSet}$ forms a basis of $\ltwo{\ddom}{\nu}^2$.
\end{proof}

We now characterise the set of $\nu$-gradients of zero, $\gradZero{\ddom}{\nu}$, which will turn out to be the entire space $\ltwo{\ddom}{\nu}^2$.
\begin{prop}[Characterisation of $\gradZero{\ddom}{\nu}$] \label{prop:CharPointMassGradZero}
	We have that $\gradZero{\ddom}{\nu} = \ltwo{\ddom}{\nu}^2$.
\end{prop}
\begin{proof}
	Since $\gradZero{\ddom}{\nu}$ is a closed, linear subspace of $\ltwo{\ddom}{\nu}^2$ by definition, it is sufficient to show that $\gradZero{\ddom}{\nu}$ contains the basis $\clbracs{g_1^j, g_2^j \ \vert \ v_j\in\vertSet}$.
	We demonstrate inclusion of the elements $g^j_1$ (as that of $g^j_2$ is similar, with the obvious alternative choice of $c$ in what follows).
	Take $c=\bracs{1,0}^{\top}$, fix $v_j\in\vertSet$, and set
	\begin{align*}
		\phi\bracs{x} &= \varphi_c\bracs{x-v_j},
	\end{align*}
	where $\varphi_c$ is as in definition \ref{def:UsefulObjects}.
	Note that $\phi$ is smooth by composition of smooth functions, $\supp\bracs{\phi}\subset B_{d}\bracs{v_j}$, and that
	\begin{align*}
		\grad\phi\bracs{x} &= \bracs{\grad\varphi_c}\bracs{x-v_j}.
	\end{align*}
	Then we have that
	\begin{align*}
		\integral{\ddom}{\abs{\phi}^2}{\nu} &= \sum_{v_l\in\vertSet} \alpha_l\abs{\phi\bracs{v_l}}^2 \\
		&= \alpha_j\abs{\varphi_c\bracs{0}}^2 + \sum_{v_l\neq v_j}\alpha_l\abs{\phi\bracs{v_l}}^2 \\
		&= 0 + \sum_{v_l\neq v_j} \alpha_l \times 0 = 0,
	\end{align*}
	and
	\begin{align*}
		\integral{\ddom}{\abs{\grad\phi - g^j_1}^2}{\nu} &= \sum_{v_l\in\vertSet} \alpha_l\abs{\grad\phi\bracs{v_l} - g^j_1\bracs{v_l}}^2 \\
		&= \alpha_j\abs{\grad\varphi_c\bracs{0} - \bracs{1,0}^{\top}}^2 + \sum_{v_l\neq v_j} \alpha_l\abs{\grad\varphi_c\bracs{v_l} - 0}^2 \\
		&= 0 + \sum_{v_l\neq v_j} \alpha_l\abs{0 - 0}^2 = 0.
	\end{align*}
	Hence, the constant sequence of smooth functions $\phi$ is such that
	\begin{align*}
		\phi\lconv{\ltwo{\ddom}{\nu}}0, \quad \grad\phi\lconv{\ltwo{\ddom}{\nu}^2} g^j_1,
	\end{align*}
	and hence $g^j_1\in\gradZero{\ddom}{\nu}$.
\end{proof}

\subsection{The Sobolev Space $\gradSobQM{\ddom}{\nu}$} \label{appS:VertexSobSpace}
Given that we now know that $\gradZero{\ddom}{\nu}$ encompasses the whole of $\ltwo{\ddom}{\nu}^2$, we must conclude that our ``Sobolev functions" have zero derivative ($\nu$-)almost-everywhere.
Indeed, proposition \ref{prop:CharPointMassGradZero} gives us the following corollary:
\begin{cory}[Characterisation of $\gradSobQM{\ddom}{\nu}$] \label{eq:CharPointMassSpace}
	We have that
	\begin{align*}
		\bracs{u,\tgrad_{\nu}u}\in\gradSobQM{\ddom}{\nu} \quad\Leftrightarrow\quad 
		& \ \text{(i)} \ u\in\ltwo{\ddom}{\nu}, \\
		& \ \text{(ii)} \ \tgrad_{\nu}u = 0 \ \nu\text{-almost everywhere}.
	\end{align*}
\end{cory}
\begin{proof}
	($\Rightarrow$) For the right-directed implication; $\tgrad_{\nu}u\in\ltwo{\ddom}{\nu}^2$ is an element of $\ltwo{\ddom}{\nu}^2$ by definition and is orthogonal to $\gradZero{\ddom}{\nu}$, but by proposition \ref{prop:CharPointMassGradZero} we know that $\gradZero{\ddom}{\nu}=\ltwo{\ddom}{\nu}^2$, we must conclude that $\grad_{\nu}u = 0$. \newline
	($\Leftarrow$) For the left-directed implication, take smooth ``bump" functions $\psi_j$ (for each $v_j\in\vertSet$) with the properties
	\begin{align*}
		\psi_j\bracs{v_j} = 1, &\quad \grad\psi_j\bracs{v_j} = -i\qm, \\
		\supp\bracs{\psi_j} &\subset B_{d}\bracs{v_j}.
	\end{align*}
	Then consider the smooth function
	\begin{align*}
		\phi\bracs{x} &:= \sum_{v_j\in\vertSet} u\bracs{v_j}\psi_j\bracs{x}, \\
		\implies \grad\phi\bracs{x} &= \sum_{v_j\in\vertSet} u\bracs{v_j}\grad\psi_j\bracs{x}.
	\end{align*}
	Then we have that
	\begin{align*}
		\integral{\ddom}{\abs{\phi - u}^2}{\nu} &= \sum_{v_j\in\vertSet} \alpha_j\abs{\phi\bracs{v_j} - u\bracs{v_j}}^2 \\
		&= \sum_{v_j\in\vertSet} \alpha_j\abs{ \sum_{v_l\in\vertSet}u\bracs{v_l}\psi_l\bracs{v_j} - u\bracs{v_j} }^2
		&= \sum_{v_j\in\vertSet} \alpha_j\abs{u\bracs{v_j}}^2\abs{\psi_j\bracs{v_j}-1}^2 \\
		&= \sum_{v_j\in\vertSet} \alpha_j\abs{u\bracs{v_j}}^2 \times 0 = 0,
	\end{align*}
	and
	\begin{align*}
		\integral{\ddom}{\abs{\tgrad\phi - 0}^2}{\nu} 
		&= \sum_{v_j\in\vertSet} \alpha_j\abs{ \sum_{v_l\in\vertSet} u\bracs{v_l}\grad\psi_l\bracs{v_j} + i\qm\psi_l\bracs{v_j} }^2 \\
		&= \sum_{v_j\in\vertSet} \alpha_j\abs{u\bracs{v_j}}^2 \abs{ \grad\psi_j\bracs{v_j} + i\qm\psi_j\bracs{v_j} }^2 \\
		&= \sum_{v_j\in\vertSet} \alpha_j\abs{u\bracs{v_j}}^2 \abs{ i\qm - i\qm } = 0.
	\end{align*}
	Thus, the constant sequence $\phi$ is such that
	\begin{align*}
		\phi \lconv{\ltwo{\ddom}{\nu}} u, \quad \tgrad\phi \lconv{\ltwo{\ddom}{\nu}^2} 0,
	\end{align*}
	and thus $\bracs{u,0}\in\gradSob{\ddom}{\nu}$.
\end{proof}
Corollary \ref{conj:ThickVertexSpaceCharacterisation} means that the space $\gradSob{\ddom}{\nu}$ is essentially isomorphic to $\complex^N$, namely functions in this space are entirely determined by their values at the vertices, and their gradients are always zero.
This matches our intuitive expectations, as the notion of a gradient (or rate of change) at an isolated point being non-zero implies that there is a small neighbourhood around the point in which we can observe the function values changing, but in the case of a point-mass measure this is not the case.