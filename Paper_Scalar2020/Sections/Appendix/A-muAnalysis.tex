\section{Appendix: Sobolev functions on the edges of an embedded graph} \label{app:muAnalysis}
In what follows, we describe the set $\gradZero{\ddom}{\lambda_{jk}}$ when the edge $I_{jk}$ is parallel to the $x_1$-axis, then employ a rotation argument to understand $\gradZero{\ddom}{\lambda_{jk}}$ for a general edge that is at an angle to the $x_1$-axis.
Given that the singular measure $\ddmes$ is just the sum of the individual singular measures supporting each edge, we can then prove that elements of $\gradZero{\ddom}{\ddmes}$ display the same behaviour as elements of $\gradZero{\ddom}{\lambda_{jk}}$ when restricted to the edge $I_{jk}$.
Throughout this section, Assumption \ref{ass:MeasTheoryProblemSetup} is adopted.

\subsection{Gradients of Zero} \label{appS:muGradZero}
\begin{prop}[Gradients of Zero on a Segment Parallel to the $x_1$-axis] \label{prop:GradZeroParallelZhikov}
	Let $I$ be a segment in the $\bracs{x_1,x_2}$-plane parallel to the $x_1$-axis, and let $\lambda_I$ be the singular measure supported on $I$.
	Then 
	\begin{align*}
		\gradZero{\ddom}{\lambda_I} &= 
		\clbracs{ \bracs{0,f}^\top	\setVert f\in\ltwo{\ddom}{\lambda_I}
		}.
	\end{align*}
\end{prop}
\begin{proof}
	This is a version of the argument in \cite[Section~3.1]{zhikov2000extension}, accomodating the dependence on $\qm$ in accordance with proposition \ref{prop:GradZeroInvarientUnderQM}.
\end{proof}

\begin{prop}\label{prop:RotationOfEdgeGradients}
	Suppose $\graph$ consists of a single edge (or segment) $I\subset\ddom$ with orthogonal co-ordinate system $y=\bracs{y_1,y_2}$, with $y_1$ parallel to $I$.
	Let $R$ be the orthogonal change of co-ordinates $x=Ry$ with $x=\bracs{x_1,x_2}$ the orthogonal co-ordinate system along the axes.
	Then
	\begin{align*}
		\gradZero{\ddom}{\lambda_I} 
		&= \clbracs{ R^{\top} \bracs{0,f_2}^\top \setVert f_2\in\ltwo{\ddom}{\lambda_I} } \\
		&= \clbracs{ g\in\ltwo{\ddom}{\ddmes}^2 \setVert g\vert_{I}\cdot e_{I} =0 },
	\end{align*}
where $e_I$ is the unit vector parallel to the segment $I$.
\end{prop}

The following characterisation of $\gradZero{\ddom}{\ddmes}$ follows immediately.
\begin{prop} \label{prop:GradZeroGraph}
	One has
	\begin{align} \label{eq:GradZeroSetRHS}
		\gradZero{\ddom}{\ddmes} &= \clbracs{g\in\ltwo{\ddom}{\ddmes}^2 \ \vert \ g\vert_{I_{jk}}\cdot e_{jk}=0 \ \forall I_{jk}\in \edgeSet}. 
	\end{align}
\end{prop}
\textit{(Sketch of proof)}:
	Showing that elements of $\gradZero{\ddom}{\ddmes}$ are contained in the set on the right-hand-side of \eqref{eq:GradZeroSetRHS} is straightforward due to the definition of $\ddmes$ and the fact that
	\begin{align} \label{eq:GraphMeasNormEdgeBreakdown}
		\norm{\cdot}_{\ltwo{\ddom}{\ddmes}}^2 = \sum_{j\con k}\norm{\cdot}_{\ltwo{\ddom}{\lambda_{jk}}}^2,
	\end{align}
	so if one has convergence in the $\ltwo{\ddom}{\ddmes}$-norm then we have convergence in each $\ltwo{\ddom}{\lambda_{jk}}$-norm.
	The opposite set inclusion involves a number of technical steps, and is given in full in section \ref{appS:ProofOfGradZeroChar}
	The gist of the argument involves demonstrating that if $g$ is a member of the set on the right-had-side of \eqref{eq:GradZeroSetRHS}, then $g_{jk}\in\gradZero{\ddom}{\ddmes}$.
	Then by writing $g = \sum_{j\con k} g_{jk}$ and noticing that $\gradZero{\ddom}{\ddmes}$ is a closed, linear subspace, membership of $g$ in $\gradZero{\ddom}{\ddmes}$ is obtained.

\subsubsection{Proof of Proposition \ref{prop:GradZeroGraph}} \label{appS:ProofOfGradZeroChar}
Denote 
\begin{align*}
	B = \clbracs{g\in\ltwo{\ddom}{\ddmes}^2 \setVert g\vert_{I_{jk}}\cdot e_{jk}=0 \ \forall I_{jk}\in \edgeSet}.
\end{align*}
We introduce two families of smooth functions that we shall make repeated use of, and examine some limits in $\ltwo{\ddom}{\ddmes}$ that involve them.
Let $\eta\in\smooth{\ddom}$ be a function with the properties $\eta\bracs{x}\in [0,1]$, $\eta = 0$ whenever $\abs{x}\leq 1$, $\eta = 1$ whenever $\abs{x}\geq 2$.
For $\graph=\bracs{\vertSet,\edgeSet}$ an embedded graph, for each $v_j\in \vertSet$ and $n\in\naturals$ we define
\begin{align} \label{eq:etaDef}
	\eta_j\bracs{x} = \eta\bracs{x-v_j}, &\quad \eta_j^n\bracs{x} = \eta_j\bracs{nx}
\end{align}
which are smooth functions by composition.
We use the functions $\eta_j^n$ when looking to extend functions defined on one edge of $\graph$ to the whole of $\graph$, and when dealing with matching conditions at the vertices of $\graph$.
Notice that
\begin{align*}
	\eta_j^n \rightarrow 1 \text{ in } \ltwo{\ddom}{\ddmes} \toInfty{n},
\end{align*}
since
\begin{align*}
	\integral{\ddom}{\abs{\eta_j^n-1}^2}{\ddmes} 
	&\leq \integral{B_{2/n}\bracs{v_j} \setminus B_{1/n}\bracs{v_j}}{}{\ddmes} 
	= \ddmes\bracs{B_{2/n}\bracs{v_j}}
	\leq \frac{4 \abs{\edgeSet}}{n}.
\end{align*}

Furthermore, we define a family of functions that will help us to deal with the graph edges.
Let $I_{jk}\in \edgeSet$ be an edge of the embedded graph $\graph=\bracs{\vertSet,\edgeSet}$, and take $\eps>0$.
Define the set 
\begin{align} \label{eq:ShortenedIntervalDef}
	I_{jk}^{\eps} := \clbracs{ x\in I_{jk} \setVert \mathrm{dist}\bracs{x, \partial I_{jk}}\leq \recip{\eps}},
\end{align}
and then let $\chi_{jk}^\eps\in\smooth{\ddom}$ be the function such that $\chi_{jk}^\eps\bracs{x}\in [0,1]$, $\chi_{jk}^\eps = 1$ whenever $\mathrm{dist}\bracs{x, I_{jk}^\eps}\leq \recip{3\eps}$, $\chi_{jk}^\eps = 0$ whenever $\mathrm{dist}\bracs{x, I_{jk}^\eps}\geq \frac{2}{3\eps}$.
Note that since $\graph$ is finite, we can assume without loss of generality that the only edge of $\graph$ that intersects $\supp(\chi_{jk}^\eps)$ is $I_{jk}$, otherwise we apply a suitable scaling.
We also check the convergence of $\chi_{jk}^\eps$ in $\ltwo{\ddom}{\ddmes}$ as $\eps\rightarrow\infty$.
Denote by $\charFunc{jk}$ the characteristic function of the edge $I_{jk}$.
Then $\chi_{jk}^\eps \rightarrow \charFunc{jk}$ in $\ltwo{\ddom}{\ddmes}$, and in $\ltwo{\ddom}{\lambda_{jk}}$, as $\eps\rightarrow\infty$.
To see this, notice that
\begin{align*}
	\integral{\ddom}{\abs{ \chi_{jk}^\eps - \charFunc{jk} }^2}{\ddmes}
	&= \integral{I_{jk}\cap\clbracs{\chi_{jk}^\eps = 0}}{}{\lambda_{jk}}
	+ \integral{I_{jk}\cap\clbracs{0\leq\chi_{jk}^\eps \leq 1}}{\abs{ \chi_{jk}^\eps - 1 }^2}{\lambda_{jk}}
	\leq \recip{3\eps} + \recip{3\eps} = \frac{2}{3\eps}.
\end{align*}
Furthermore, we remark that $\lvert\grad\chi_{jk}^\eps\rvert \leq c\eps$ for some $c\geq0$.
This is a consequence of the construction of $\chi_{jk}^\eps$ as a smooth function with bounded gradient that increases from 0 to 1 over a unit interval.
The constant $c$ arises in case we have to apply a scaling to the argument of $\chi_{jk}^\eps$, due to the proximity of other edges.

We can now begin to prove proposition \ref{prop:GradZeroGraph}; with the inclusion $\gradZero{\ddom}{\ddmes} \subset B$ following quickly.
\begin{prop} \label{prop:Grad0IncB}
	For $B = \clbracs{g\in\ltwo{\ddom}{\ddmes} \ \vert \ g\vert_{I_{jk}}\cdot e_{jk}=0 \ \forall I_{jk}\in E}$, we have $\gradZero{\ddom}{\ddmes} \subset B$.
\end{prop}
\begin{proof}
	This is a direct consequence of each $\lambda_{jk}$ being a restriction of $\ddmes$, as by \eqref{eq:GraphMeasNormEdgeBreakdown} any sequence of smooth functions converging in $\ltwo{\ddom}{\ddmes}$ necessarily converges in $\ltwo{\ddom}{\lambda_{jk}}$ for each $I_{jk}\in\edgeSet$.
\end{proof}

The converse requires some preliminary results.
First, we show that a gradient of zero on a ``shortened" edge of the graph is also a gradient of zero on the whole graph, if we extend it by zero.
\begin{lemma} \label{lem:SegGradExtend}
	For $n\in\naturals$, let $I_{jk}^n$ be as in \eqref{eq:ShortenedIntervalDef}.
	Suppose that we have a function $g\in\ltwo{\ddom}{\ddmes}$ with $g=0$ on $\graph\setminus I_{jk}^{n}$ and $g\cdot e_{jk}=0$ on $I_{jk}^{n}$.
	Then $g\in\gradZero{\ddom}{\ddmes}$.
\end{lemma}
\begin{proof}
	As $g\cdot e_{jk}=0$ on $I_{jk}^{n}$ and $g=0$ on $I_{jk}\setminus I_{jk}^{n}$, we have that $g\cdot e_{jk}=0$ on $I_{jk}$ and hence $g\in\gradZero{\ddom}{\lambda_{jk}}$, so we can find a sequence of smooth functions $\phi_l$ as in \eqref{eq:GradZeroDef}.
	Now consider the sequence $\psi_l = \chi_{jk}^{n}\phi_l$; we have that
	\begin{align*}
		\integral{\ddom}{\abs{\psi_l}^2}{\ddmes} = \integral{I_{jk}}{\abs{\chi_{jk}^{n}\phi_l}^2}{\lambda_{jk}}
		\leq \integral{I_{jk}}{\abs{\phi_l}^2}{\lambda_{jk}} \rightarrow0 \toInfty{l},
	\end{align*}
	which is one of the desired convergence results for $\psi_l$.
	For the other convergence result we need, observe that
	\begin{align*}
		\integral{\ddom}{\abs{\phi_l\grad\chi_{jk}^{n}}^2}{\ddmes} 
		&= \integral{I_{jk}}{\abs{\phi_l\grad\chi_{jk}^{n}}^2}{\lambda_{jk}}
		\leq \sup_{I_{jk}}\bracs{\abs{\grad\chi_{jk}^{n}}^{2}}\integral{I_{jk}}{\abs{\phi_l}^2}{\ddmes}
		\rightarrow 0 \toInfty{l}
	\end{align*}
	because $\abs{\grad\chi_{jk}^{n}}$ depends on $n$ only.
	Additionally
	\begin{align*}
		\integral{\ddom}{\abs{\chi_{jk}^n\grad\phi_l - g}^2}{\ddmes} &= \integral{I_{jk}}{\abs{\chi_{jk}^n\grad\phi_l - g}^2}{\lambda_{jk}}
		= \integral{I_{jk}\setminus I_{jk}^n}{\abs{\chi_{jk}^n\grad\phi_l}^2}{\lambda_{jk}}
		+  \integral{I_{jk}^n}{\abs{\grad\phi_l - g}^2}{\lambda_{jk}} \\
		&\leq \integral{I_{jk}\setminus I_{jk}^n}{\abs{\grad\phi_l}^2}{\lambda_{jk}} +  \integral{I_{jk}^n}{\abs{\grad\phi_l - g}^2}{\lambda_{jk}} \\
		&= \integral{I_{jk}}{\abs{\grad\phi_l - g}^2}{\lambda_{jk}} \rightarrow0 \toInfty{l},
	\end{align*}
	where we have made use of the fact that $g=0$ on $\graph\setminus I_{jk}^n$ and the various properties of $\chi_{jk}^n$.
	Armed with these estimates, we have that
	\begin{align*}
		\integral{\ddom}{\abs{\grad\psi_l - g}^2}{\ddmes} 
		&= \integral{\ddom}{\abs{\chi_{jk}^n\grad\phi_l + \phi_l\grad\chi_{jk}^n - g}^2}{\ddmes} \\
		&\leq 2\integral{\ddom}{\abs{\phi_l\grad\chi_{jk}^n}^2}{\ddmes} + 2\integral{\ddom}{\abs{\chi_{jk}^n\grad\phi_l - g}^2}{\ddmes}
		\rightarrow0 \toInfty{l}.
	\end{align*}
	Thus, $\psi_l$ is a sequence of smooth functions such that
	\begin{align*}
		\psi_l \lconv{\ltwo{\ddom}{\ddmes}} 0, &\quad
		\grad\psi_l \lconv{\ltwo{\ddom}{\ddmes}^2} g
	\end{align*}
	and hence, $g\in\gradZero{\ddom}{\ddmes}$.
\end{proof}

We are now in a position to prove the aforementioned converse inclusion.
\begin{prop} \label{prop:BIncGrad0}
	We have $B \subset \gradZero{\ddom}{\ddmes}$.
\end{prop}
\begin{proof}
	Take $g\in B$, and define a family of functions $g_n$ by
	\begin{align*}
		g_n\bracs{x} &= \sum_{j\in V}\sum_{j\conLeft k}\eta_j^n\bracs{x}\eta_k^n\bracs{x}g\vert_{I_{jk}}\bracs{x}.
	\end{align*}
	Recall that $\graph$ is assumed finite so the sum converges.
	For each $j,k$ with $j\conLeft k$, the function $\eta_j^n\eta_k^n g\vert_{I_{jk}}$ satisfies the hypothesis of lemma \ref{lem:SegGradExtend}, so $\eta_j^n\eta_k^n g\vert_{I_{jk}}\in\gradZero{\ddom}{\ddmes}$.
	Furthermore, as $\gradZero{\ddom}{\ddmes}$ is a linear subspace of $\ltwo{\ddom}{\ddmes}^{2}$, $g_n\in\gradZero{\ddom}{\ddmes}$ too, $\forall n\in\naturals$.
	By closure of $\gradZero{\ddom}{\ddmes}$ in $\ltwo{\ddom}{\ddmes}^2$; $g_n$ converges in $\gradZero{\ddom}{\ddmes}$ provided it converges at all, and so it remains to show that $g_n\lconv{\ltwo{\ddom}{\ddmes}^2} g \toInfty{n}$.
	However with the result of lemma \ref{lem:etaConv}, we have that $\eta_j^n\eta_k^n g\vert_{I_{jk}}\lconv{\ltwo{\ddom}{\ddmes}^2} g\vert_{I_{jk}}$ and hence
	\begin{align*}
		g_n \lconv{\ltwo{\ddom}{\ddmes}^2} &\sum_{j\in V}\sum_{j\conLeft k}g\vert_{I_{jk}} = g \toInfty{n},
	\end{align*}
	so $g\in\gradZero{\ddom}{\ddmes}$.
\end{proof}
Propositions \ref{prop:BIncGrad0} and \ref{prop:Grad0IncB} complete the proof of proposition \ref{prop:GradZeroGraph} when combined with proposition \ref{prop:RotationOfEdgeGradients} and corollary \ref{cory:Grad0SingleEdge}.

\subsection{The ``Non-Classical" Sobolev Space} \label{appS:SobSpacesTheory}
Establishing an understanding of $\gradZero{\ddom}{\ddmes}$  affords us greater insight into the tangential gradient $\tgrad_\ddmes u$ of functions $u\in\gradSobQM{\ddom}{\ddmes}$, and enables us to obtain an edge-wise ``form" for the tangential gradient, as follows:
\begin{prop} \label{prop:GraphTangGrad}
	For each $I_{jk}\in \edgeSet$ write $\gradSob{\interval{I_{jk}}}{t}$ for the (``classical") Sobolev space on the interval $\interval{I_{jk}}$ with respect to the (1 dimensional) Lebesgue measure, and let $\tilde{u}_{jk} = u_{jk} \circ r_{jk}$.
	Then for $u\in\gradSobQM{\ddom}{\ddmes}$ we have that $\tilde{u}_{jk}\in\gradSob{\interval{I_{jk}}}{t}$ for each $I_{jk}\in \edgeSet$, and that
	\begin{align*}
		\bracs{ \tgrad_\ddmes u }_{jk} 
		&= R_{jk}^\top \begin{pmatrix} u_{jk}' + \rmi\bracs{R_{jk}\qm}_1 u_{jk} \\ 0	\end{pmatrix}
	\end{align*}
	where $u_{jk}' = \tilde{u}_{jk}' \circ r_{jk}^{-1}$.
\end{prop}
Note that the prime notation on $u_{jk}'$ does \emph{not} imply the existence of any kind of ``classical" derivative for $u_{jk}$ or $u$, it is just a helpful piece of notation to remind us that $u_{jk}$ does have some regularity after composition with $r_{jk}$.
\begin{proof}
	The proof proceeds in much the same way as how we sought to understand elements of $\gradZero{\ddom}{\ddmes}$.
	Any tangential gradient must be orthogonal to elements $g_{jk}\in\gradZero{\ddom}{\ddmes}$ where $g\in\gradZero{\ddom}{\lambda_{jk}}$ (by proposition \ref{prop:GradZeroGraph}), and this must hold for each edge $I_{jk}$.
	The plan is again to first consider an edge aligned parallel to the $x_1$-axis, then apply a rotation before appealing to the edge-wise decomposition of our measure. \newline
	
	As just mentioned, first consider an edge $I_{jk}$ parallel to the $x_1$-axis. 
	Let $\tgrad_\ddmes u = \bracs{v_1, v_2}^\top$ denote the components of $\tgrad_\ddmes u$; we can see that $v_2\vert_{I_{jk}}=0$ immediately due to the result of proposition \ref{prop:GradZeroGraph} and thus the requirement that $\tgrad_\ddmes u$ be orthogonal to every member of $g\in\gradZero{\ddom}{\lambda_{jk}}$.
	This leaves the form of $v_1\vert_{I_{jk}}$ to be determined.
	Since $u\in\gradSobQM{\ddom}{\ddmes}$ there exists a sequence of smooth functions $\phi_n$ which converges to $u$ in $\ltwo{\ddom}{\ddmes}$, and whose gradients $\tgrad\phi_n$ converge to $\tgrad_\ddmes u$.
	Clearly any such sequence also converges to $u_{jk}$ in $\ltwo{\ddom}{\lambda_{jk}}$ as well (see \eqref{eq:GraphMeasNormEdgeBreakdown}), and $\partial_1\phi_n$ converges to $v_1\vert_{I_{jk}}-\rmi\qm_{1}u_{jk}$ in $\ltwo{\ddom}{\lambda_{jk}}$.
	Considering the composition $\tilde{\phi}_n = \phi_n \circ r_{jk}$ we find that
	\begin{align*}
		\tilde{\phi}_n \lconv{\ltwo{\interval{I_{jk}}}{t}} \tilde{u}_{jk},
		&\quad \diff{\tilde{\phi}_n}{t} \lconv{\ltwo{\interval{I_{jk}}}{t}} \tilde{v}_1\vert_{jk} - \rmi\qm_1 \tilde{u}_{jk},
	\end{align*}
	from which we can deduce that $\tilde{u}_{jk}' = \tilde{v}_1\vert_{I_{jk}} - i\qm_1\tilde{u}$, and hence obtain the result for an edge parallel to the $x_1$-axis ($R_{jk}$ being the identity).
	Then to deal with edges that are not parallel to the $x_1$-axis, we can first apply a rotation (using $R_{jk}$) to lie in a co-ordinate system with $I_{jk}$ parallel to an axis, apply the above argument, and then rotate back --- the process of which introducing the various $R_{jk}$ pre-multipliers.
	Given proposition \ref{prop:GradZeroGraph}, we are done.
\end{proof}

Insofar, we have not considered the behaviour of functions $u\in\gradSobQM{\ddom}{\ddmes}$ near the vertices of the graph, and it is reasonable to expect some special behaviour at the vertices, otherwise there will be no reflection of the connectivity of $\graph$.
This is addressed in the following theorem --- we can deduce that $u$ is continuous at the vertices $v_j\in \vertSet$ of $\graph$ (for any $\qm$).
Notably, the converse is claimed in \cite[Section~4]{zhikov2002homogenization} for when $\qm=0$ on a single-junction-like graph structure.
\begin{theorem} \label{thm:CharOfGradSob}
	Suppose that $u \in\gradSobQM{\ddom}{\ddmes}$.
	Then $\bracs{u, \tgrad_{\lambda_{jk}}u}\in\gradSobQM{\ddom}{\lambda_{jk}}$ for all $I_{jk}\in \edgeSet$, and $u$ is continuous at each $v_j\in \vertSet$.
	where $\tgrad_{\lambda_{jk}}u = \tgrad_{\ddmes}u\vert_{I_{jk}}$ on each edge $I_{jk}$.
\end{theorem}
\textit{(Sketch of proof)}: 
	The implication is essentially a result of \eqref{eq:GraphMeasNormEdgeBreakdown}, as $\bracs{u, \tgrad_{\lambda_{jk}}u}\in\gradSobQM{\ddom}{\lambda_{jk}}$ follows from this immediately.
	Continuity at the vertices is then obtained by showing that any sequence $\phi_n$ of smooth functions approximating $u$ and $\tgrad_\ddmes u$ (as in the definition of $\gradSobQM{\ddom}{\ddmes}$) is actually Cauchy in the uniform norm.
	As such it must also converge to a continuous function by completeness of this norm, and the limit must be $u_{jk}$.
	In particular it must also converge uniformly on the ``junction" surrounding each vertex $v_j$, and thus $u$ must be continuous at $v_j$ in particular.
\begin{proof}
	Suppose $u\in\gradSob{\ddom}{\ddmes}$, then we can find a sequence of smooth functions $\phi_l$ such that
	\begin{align*}
		\phi_l \lconv{\ltwo{\ddom}{\ddmes}} u, 
		&\quad \grad\phi_l \lconv{\ltwo{\ddom}{\ddmes}^2} \grad_\ddmes u.
	\end{align*}
	But for each $I_{jk}\in \edgeSet$,
	\begin{align*}
		\integral{\ddom}{\abs{ \phi_l - u }^2}{\lambda_{jk}}
		&\leq \integral{\ddom}{\abs{ \phi_l - u }^2}{\ddmes} \rightarrow 0, \\
		\integral{\ddom}{\abs{ \grad\phi_l - \grad_\ddmes u }^2}{\lambda_{jk}}
		&\leq \integral{\ddom}{\abs{ \grad\phi_l - \grad_\ddmes u }^2}{\ddmes} \rightarrow 0 \toInfty{l}.
	\end{align*}
	Thus $u\in\gradSob{\ddom}{\lambda_{jk}}$ (with $\grad_{\lambda_{jk}}u = \grad_\ddmes u$).
	We now retain this sequence $\phi_l$, and consider one $v_j\in \vertSet$ and it's connecting edges $I_{jk}$ where $j\sim k$.
	For each such $k$, to reduce the notational load denote composition with $r_{jk}$ (see assumption \ref{ass:MeasTheoryProblemSetup}) by an overhead tilde.
	Since we have shown $u\in\gradSob{\ddom}{\lambda_{jk}}$, we conclude that 
	\begin{align*}
		\widetilde{\phi}_l \lconv{\ltwo{\interval{I_{jk}}}{t}} \widetilde{u}, 
		&\quad \widetilde{\phi}'_l \lconv{\ltwo{\interval{I_{jk}}}{t}} \widetilde{u}',
	\end{align*}
	and hence $u\in\gradSob{\interval{I_{jk}}}{t}$.
	As the embedding 
	\begin{align*}
		W^{1,2}\bracs{\interval{I_{jk}}, \md t} = \gradSob{\interval{I_{jk}}}{t} \ &\hookrightarrow \ C^{0,\recip{2}}\interval{I_{jk}}
	\end{align*}
	is compact, we can conclude that $\phi_l$ is a Cauchy sequence in the $C^{0,\recip{2}}$-norm, and hence is also Cauchy in the uniform norm,
	\begin{align*}
		\norm{\widetilde{\phi}_l}_{\mathrm{sup}_{jk}} := \sup_{\interval{I_{jk}}}\abs{\phi_l}.
	\end{align*}
	As the space of continuous functions is complete with respect to this norm we can conclude that $\widetilde{\phi}_l$ converges (uniformly) on this interval; and this limit must be $\widetilde{u}$, which is itself continuous on $\interval{I_{jk}}$ as it is the uniform limit of continuous functions.
	By ``undoing" the change of variables under $r_{jk}$, we can also conclude that
	\begin{align*}
		\sup_{I_{jk}}\abs{\phi_l - u}\rightarrow0 \toInfty{l},
	\end{align*}
	that is $\phi_l$ converges uniformly to $u$ on $I_{jk}$.
	We now claim that $\phi_l$ converges uniformly to $u$ on
	\begin{align*}
		J\bracs{v_j} &:= \bigcup_{j\sim k}I_{jk}.
	\end{align*}
	Indeed, one has 
	\begin{align*}
		\sup_{J\bracs{v_j}}\abs{\phi_l - u} 
		&= \sup_{j\sim k}\sup_{I_{jk}}\abs{\phi_l - u}
		\rightarrow0 \toInfty{l},
	\end{align*}
	due to the uniform convergence on each edge.
	Thus $u$ is also the uniform limit of continuous functions on $J\bracs{v_j}$, and so it continuous here, which in particular includes the vertex $v_j$ itself, which completes the proof.
\end{proof}