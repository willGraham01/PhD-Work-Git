\section{Appendix: Singular Measures} \label{app:SingularMeasures}
Embedded graphs (section \ref{ssec:EmbeddedGraphs}) allow us to introduce the measure which links quantum graph problems to the measure-theoretic problems we first introduced in section \ref{ssec:OurSystem}
For an embedded graph $\graph = \bracs{\vertSet,\edgeSet}$ and for each $I_{jk}\in \edgeSet$, define the (Borel) measure $\lambda_{jk}$ as the measure which supports the one-dimensional Lebesgue measure on the edge $I_{jk}$.
So for each Borel set $B$ we have
\begin{align*}
	\lambda_{jk}\bracs{B} = \lambda_{1}\bracs{r_{jk}^{-1}\bracs{B \cap I_{jk}}}
\end{align*}
where $\lambda_1$ is the Lebesgue measure on $\reals$, and $r_{jk}$ is the parametrisation of the edge $I_{jk}$ (see \eqref{eq:GeneralCurveParam}).
Then set $\ddmes$ to be the (Borel) measure defined by
\begin{align*}
	\ddmes\bracs{B} = \sum_{v_j\in \vertSet}\sum_{j\conLeft k} \lambda_{jk}\bracs{B}.
\end{align*}
Then $\ddmes$ is the ``singular measure that supports $\graph$"; or alternatively the ``singular measure on $\graph$", or the ``(singular) measure that supports the edges of $\graph$".
For a graph embedded into a 2D domain, the singular measure $\ddmes$ is illustrated in figure \ref{fig:Diagram_SingularMeasure2D}.
\begin{figure}[b!]
	\centering
	\includegraphics[scale=0.85]{Diagram_SingularMeasure2D.pdf}
	\caption{\label{fig:Diagram_SingularMeasure2D} For a graph embedded in $\reals^2$, the $\ddmes$-measure of any Borel set $B$ is obtained from summing the contributions of each $\lambda_{jk}$, as indicated by the thickened and coloured lines.
	Sets that do not intersect $\graph$ have zero measure.}
\end{figure} \newline

Now for each $v_j\in\vertSet$ let $\delta_j$ be a point-mass measure centred on $v_j$, namely
\begin{align*}
	\delta_j\bracs{B} &= \begin{cases} 1 & v_j\in B, \\ 0 & v_j\not\in B, \end{cases}
\end{align*}
and let
\begin{align*}
	\nu\bracs{B} &= \sum_{v_j\in\vertSet}\alpha_j\delta_j\bracs{B},
\end{align*}
for Borel sets $B$, where each $\alpha_j$ is the coupling constant at the vertex $v_j$.
Finally, form the measure $\dddmes = \ddmes + \nu$.
The measures $\dddmes$, $\ddmes$, and $\nu$ will be the key components in our measure-theoretic formulations, enabling us to establish ideas of derivatives whilst working on a domain which has no ``area" in the Lebesgue-sense.