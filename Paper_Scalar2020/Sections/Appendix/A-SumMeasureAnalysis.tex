\section{Appendix: On the Sobolev Space $\gradSobQM{\ddom}{\dddmes}$} \label{app:SumMeasureAnalysis}
Now that the analysis of sections \ref{app:muAnalysis} and \ref{app:VertexAnalysis} is complete, we turn out attention to $\gradSobQM{\ddom}{\dddmes}$.
Needless to say, our reason for considering the spaces in this order is because we expect that functions that live in $\gradSobQM{\ddom}{\dddmes}$ are combinations of functions that live in $\gradSobQM{\ddom}{\ddmes}$ and $\gradSobQM{\ddom}{\nu}$; in a similar vein of reasoning to how we demonstrated that $\gradSobQM{\ddom}{\ddmes}$ is constructed by considering the various spaces $\gradSobQM{\ddom}{\lambda_{jk}}$.
The results that we require are given below, however we conjecture that the converse implication also holds, given the analogies with \ref{thm:CharOfGradSob}.
\begin{prop} \label{prop:ThickVertexGradZeroIFF}
	Let $g\in\ltwo{\ddom}{\dddmes}^2$ and let 
	\begin{align*}
		g_{\ddmes}(x) = \begin{cases} g(x) & x\neq v_j \ \forall v_j\in\vertSet, \\ 0 & x=v_j, \ v_j\in\vertSet, \end{cases} 
		&\qquad
		g_{\nu}(x) = \begin{cases} 0 & x\neq v_j \ \forall v_j\in\vertSet, \\ g(x) & x=v_j, \ v_j\in\vertSet. \end{cases}
	\end{align*}		
	Then we have that
	\begin{align*}
		g\in\gradZero{\ddom}{\dddmes} \quad\Leftrightarrow\quad 
		& g_{\ddmes}\in\gradZero{\ddom}{\ddmes} \text{ and } g_{\nu}\in\gradZero{\ddom}{\nu}.
	\end{align*}
\end{prop}
\begin{proof}
	For the right-directed implication ($\Rightarrow$), it is sufficient to notice that 
	\begin{align*}
		\norm{\cdot}_{\ltwo{\ddom}{\dddmes}}^2 &= \norm{\cdot}_{\ltwo{\ddom}{\ddmes}}^2 + \norm{\cdot}_{\ltwo{\ddom}{\nu}}^2,
	\end{align*}
	so any approximating sequence for $g$ that converges in $\ltwo{\ddom}{\dddmes}^2$ also converges in $\ltwo{\ddom}{\ddmes}^2$ to $g_{\ddmes}$ and in $\ltwo{\ddom}{\nu}^2$ to $g_{\nu}$.
	
	For the left-directed implication ($\Leftarrow$), it is sufficient for us to demonstrate the implication holds for the case when $g_{\nu}=0$, and the case that $g_{\ddmes}=0$ with $g_{\nu}\neq0$ at precisely one vertex $v$.
	Having shown the implication in these cases, linearity of $\gradZero{\ddom}{\dddmes}$ will then complete the implication.
	As such, first consider the case when $g_{\nu}=0$.
	Notice that the conclusion of lemma \ref{lem:SegGradExtend} can be strengthened to membership of $\gradZero{\ddom}{\dddmes}$, as the approximating sequence $\psi_l$ that is constructed satisfies $\psi_l\bracs{v_j}=0, \grad\psi_l\bracs{v_j}=0$. \tstk{we should make this observation in appendix C too, probably)}
	With this, the argument of proposition \ref{prop:BIncGrad0} can be recycled to conclude that $g_{\ddmes}\in\gradZero{\ddom}{\dddmes}$, and hence $g\in\gradZero{\ddom}{\dddmes}$ too.
	
	Next, consider the case when $g_{\ddmes}=0$, and when $g_{\nu}=0$ at all vertices except $v\in\vertSet$, with $v=\bracs{v_1, v_2}\in\ddom$ and with $g_{\nu}(v) = \bracs{g_1, g_2}^\top$.
	Consider smooth functions $\phi_n$ for $n\in\naturals$ where
	\begin{align*}
		\phi_n(x) &= \bracs{x_1 - v_1} g_1 + \bracs{x_2 - v_2}g_2, &\quad x\in B_{\recip{n}}\bracs{v}, \\
		\phi_n(x) &= 0, &\quad \abs{x - v} \geq \frac{2}{n},
	\end{align*}
	schematically illustrated in figure \ref{fig:Diagram_SmoothFunctionBoundedGrad1D}.
	\begin{figure}[b]
		\centering
		\includegraphics[scale=1.0]{Diagram_SmoothFunctionBoundedGrad1D.pdf}
		\caption{\label{fig:Diagram_SmoothFunctionBoundedGrad1D} The profile of the functions $\phi_n$ across the vertex $v$. One can take a function with this profile on $\reals$ and ``place" it across the vertex $v$ parallel to $g(v)$, smoothing to 0 as perpendicular distance from $v$ increases, giving rise to the $\phi_n$.}
	\end{figure}
	Since $\abs{\phi_n(x)} \leq \recip{n}\abs{g_{\nu}\bracs{v}}$ when $\abs{x-v}=\recip{n}$, $\phi_n$ can be chosen so that exists a constant $c$ independent of $n$ such that $\abs{\grad\phi_n} \leq c\abs{g_{\nu}\bracs{v}}$ when $\recip{n} \leq \abs{x-v} \leq \frac{2}{n}$.
	In addition, $\phi_n(v)=0$ and $\grad\phi_n=g(v)$ for all $n\in\naturals$.
	Then
	\begin{align*}
		\integral{\ddom}{\abs{\phi_n}^2}{\dddmes} 
		&= \integral{I}{ \abs{\phi_n}^2 }{\ddmes} + \alpha_v\abs{\phi_n\bracs{v}}^2
		\leq \integral{B_{\frac{2}{n}}(v)}{ \recip{n^2}\abs{g_{\nu}(v)}^2 }{\ddmes} + 0 \\
		&= \frac{2\abs{g_{\nu}(v)}^2 \mathrm{deg}(v)}{n^3} \rightarrow 0, \\
		\integral{\ddom}{ \abs{\grad\phi_n - g}^2 }{\ddmes}
		&= \integral{B_{\frac{2}{n}}(v)}{ \abs{\grad\phi_n}^2 }{\ddmes} + \alpha_v\abs{\grad\phi_n(v)-g(v)}^2
		\leq c\abs{g(v)}^2\integral{B_{\frac{2}{n}}(v)}{ }{\ddmes} \\
		&\leq \frac{2c\abs{g(v)}^2 \mathrm{deg}(v)}{n} \rightarrow 0,
	\end{align*}
	where $\mathrm{deg}(v)$ denotes the degree of the vertex $v$.
	We thus conclude that $g\in\gradZero{\ddom}{\dddmes}$.
	Given the linearity of $\gradZero{\ddom}{\dddmes}$ mentioned previously, the proof is now complete.
\end{proof}

Knowing the correspondence between the various gradients of zero with respect to the measures $\dddmes$, $\ddmes$, and $\nu$, we can deduce the following about the tangential gradients of functions in $\gradSobQM{\ddom}{\dddmes}$.
\begin{theorem} \label{thm:ThickVertexSpaceCharacterisation}
	\begin{align*}
		\bracs{u, \tgrad_{\dddmes}u}\in\gradSob{\ddom}{\dddmes} \quad\Rightarrow\quad
		& \ \text{(i)} \ \bracs{u, \tgrad_{\dddmes}u}\in\gradSob{\ddom}{\ddmes}, \\
		& \ \text{(ii)} \ \bracs{u, \tgrad_{\dddmes}u}\in\gradSob{\ddom}{\nu}. \\
	\end{align*}
\end{theorem}
\begin{proof}
	Taking an approximating sequence $\phi_n$ such that
	\begin{align*}
		\phi_n \lconv{\ltwo{\ddom}{\dddmes}} u, \quad \tgrad\phi_n\lconv{\ltwo{\ddom}{\dddmes}^2}\grad_{\dddmes}u,
	\end{align*}
	then using the fact that
	\begin{align*}
		\norm{\cdot}_{\ltwo{\ddom}{\dddmes}}^2 &= \norm{\cdot}_{\ltwo{\ddom}{\ddmes}}^2 + \norm{\cdot}_{\ltwo{\ddom}{\nu}}^2,
	\end{align*}
	we infer that $\phi_n$ also converges in $\ltwo{\ddom}{\ddmes}$ and $\ltwo{\ddom}{\nu}$, as do its gradients in $\ltwo{\ddom}{\ddmes}^2$ and $\ltwo{\ddom}{\nu}^2$.
	Furthermore, since $\tgrad_{\dddmes}u \perp \gradZero{\ddom}{\dddmes}$ and given proposition \ref{prop:ThickVertexGradZeroIFF}, we have that $\tgrad_{\dddmes}u$ is orthogonal to $\gradZero{\ddom}{\ddmes}$ (in $\ltwo{\ddom}{\ddmes}^2$) and to $\gradZero{\ddom}{\nu}$ (in $\ltwo{\ddom}{\nu}^2$), and we are done. 
\end{proof}
From our analysis of $\gradSob{\ddom}{\ddmes}$ we know that the conditions (i) and (ii) in theorem \ref{thm:ThickVertexSpaceCharacterisation} are sufficient for the following to hold;
\begin{align*}
		\text{(a)} \ & \bracs{u, \tgrad_{\dddmes}u}\in\gradSob{\ddom}{\lambda_{jk}} \ \forall I_{jk}\in\edgeSet, \\
		\text{(b)} \ & u \text{ is continuous across the vertices} \ v_j\in\vertSet, \\
		\text{(c)} \ & \tgrad_{\dddmes}u\vert_{v_j} = 0 \ \forall v_j\in\vertSet.
\end{align*}
If the converse to theorem \ref{thm:CharOfGradSob} is true, as implied from results in \cite{zhikov2002homogenization}, then (i)-(ii) are necessary and sufficient for (a)-(c).
Since we pose the (measure-theoretic) variational problem in the space $\gradSob{\ddom}{\dddmes}$, and now that we know that every function in that space also lives in $\gradSob{\ddom}{\ddmes}$ and $\gradSob{\ddom}{\nu}$, we have the properties we need to derive the edge-ODEs and vertex conditions (our quantum graph problem).