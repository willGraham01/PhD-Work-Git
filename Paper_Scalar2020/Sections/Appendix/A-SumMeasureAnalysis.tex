\section{Appendix: On the Sobolev Space $\gradSobQM{\ddom}{\dddmes}$} \label{app:SumMeasureAnalysis}
Now that the analysis of sections \ref{app:muAnalysis} and \ref{app:VertexAnalysis} is complete, we turn out attention to $\gradSobQM{\ddom}{\dddmes}$.
Needless to say, our reason for considering the spaces in this order is because we expect that we can ``build" $\gradSobQM{\ddom}{\dddmes}$ from a combination of functions that live in $\gradSobQM{\ddom}{\ddmes}$ and $\gradSobQM{\ddom}{\nu}$; in a similar manner to how we demonstrated that $\gradSobQM{\ddom}{\ddmes}$ is constructed by considering the various spaces $\gradSobQM{\ddom}{\lambda_{jk}}$.
The characterisation that we require is given below \tstk{currently conjecture, pending other direction of argument. But we actually only need the direction that we can already prove, at least for the derivation!}.
\begin{conjecture} \label{conj:ThickVertexSpaceCharacterisation}
	\begin{align*}
		\bracs{u, \grad_{\dddmes}u}\in\gradSob{\ddom}{\dddmes} \quad\Leftrightarrow\quad
		& \ \text{(i)} \ \bracs{u, \grad_{\dddmes}u}\in\gradSob{\ddom}{\ddmes}, \\
		& \ \text{(ii)} \ \bracs{u, \grad_{\dddmes}u}\in\gradSob{\ddom}{\nu}. \\
		\\
		\quad\Leftrightarrow\quad
		& \ \text{(a)} \ \bracs{u, \grad_{\dddmes}u}\in\gradSob{\ddom}{\lambda_{jk}} \ \forall I_{jk}\in\edgeSet, \\
		& \ \text{(b)} \ u \text{ is continuous across the vertices} \ v_j\in\vertSet, \\
		& \ \text{(c)} \ \grad_{\dddmes}u\vert_{v_j} = 0 \ \forall v_j\in\vertSet.
	\end{align*}
\end{conjecture}
Note that from our analysis of $\gradSob{\ddom}{\ddmes}$ we know that (i) is necessary and sufficient for (a) and (b).
Our analysis of $\gradSob{\ddom}{\nu}$ also demonstrates that (ii) is necessary and sufficient for (c), so it suffices to show that membership of $\gradSob{\ddom}{\dddmes}$ is equivalent to either (i)-(ii) or (a)-(c).
\begin{proof}
	($\Rightarrow$) We will demonstrate that conditions (i)-(ii) hold.
	Clearly, if we have smooth functions such that
	\begin{align*}
		\phi_n \lconv{\ltwo{\ddom}{\dddmes}} u, \quad \grad\phi_n\lconv{\ltwo{\ddom}{\dddmes}^2}\grad_{\dddmes}u,
	\end{align*}
	then using the fact that
	\begin{align*}
		\norm{\cdot}_{\ltwo{\ddom}{\dddmes}}^2 &= \norm{\cdot}_{\ltwo{\ddom}{\ddmes}}^2 + \norm{\cdot}_{\ltwo{\ddom}{\nu}}^2,
	\end{align*}
	we have that $\phi_n$ also converges in $\ltwo{\ddom}{\ddmes}$ and $\ltwo{\ddom}{\nu}$, as do its gradients in $\ltwo{\ddom}{\ddmes}^2$ and $\ltwo{\ddom}{\nu}^2$. \newline
	
	($\Leftarrow$) \tstk{Still need to refine!}
\end{proof}
Whilst this is only one-half of conjecture \ref{conj:ThickVertexSpaceCharacterisation}, it is the only direction we need to employ in our derivation of the quantum-graph problem.
Since we pose the (measure-theoretic) variational problem in the space $\gradSob{\ddom}{\dddmes}$, we then know that every function in that space also lives in $\gradSob{\ddom}{\ddmes}$ and $\gradSob{\ddom}{\nu}$, which gives us the properties we need to derive the edge-ODEs and vertex conditions.
That said, it would be ideal from an analytical perspective to show that the reverse implication holds (to complete conjecture \ref{conj:ThickVertexSpaceCharacterisation}) so that our intuition is validated; or alternatively a counterexample would be instructive in showing why our intuition isn't correct.