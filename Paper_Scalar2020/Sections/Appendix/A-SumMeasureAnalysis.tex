\section{Appendix: On the Sobolev Space $\gradSobQM{\ddom}{\dddmes}$} \label{app:SumMeasureAnalysis}
Now that the analysis of sections \ref{app:muAnalysis} and \ref{app:VertexAnalysis} is complete, we turn out attention to $\gradSobQM{\ddom}{\dddmes}$.
Needless to say, our reason for considering the spaces in this order is because we expect that functions that live in $\gradSobQM{\ddom}{\dddmes}$ are combinations of functions that live in $\gradSobQM{\ddom}{\ddmes}$ and $\gradSobQM{\ddom}{\nu}$; in a similar vein of reasoning to how we demonstrated that $\gradSobQM{\ddom}{\ddmes}$ is constructed by considering the various spaces $\gradSobQM{\ddom}{\lambda_{jk}}$.
The implication that we require is given below, however we conjecture that the converse implication also holds, given the analogies with \ref{thm:CharOfGradSob}.
\begin{theorem} \label{thm:ThickVertexSpaceCharacterisation}
	\begin{align*}
		\bracs{u, \grad_{\dddmes}u}\in\gradSob{\ddom}{\dddmes} \quad\Rightarrow\quad
		& \ \text{(i)} \ \bracs{u, \grad_{\dddmes}u}\in\gradSob{\ddom}{\ddmes}, \\
		& \ \text{(ii)} \ \bracs{u, \grad_{\dddmes}u}\in\gradSob{\ddom}{\nu}. \\
		\\
		\quad\Rightarrow\quad
		& \ \text{(a)} \ \bracs{u, \grad_{\dddmes}u}\in\gradSob{\ddom}{\lambda_{jk}} \ \forall I_{jk}\in\edgeSet, \\
		& \ \text{(b)} \ u \text{ is continuous across the vertices} \ v_j\in\vertSet, \\
		& \ \text{(c)} \ \grad_{\dddmes}u\vert_{v_j} = 0 \ \forall v_j\in\vertSet.
	\end{align*}
\end{theorem}
Note that from our analysis of $\gradSob{\ddom}{\ddmes}$ we know that (i) is sufficient for (a) and (b).
Our analysis of $\gradSob{\ddom}{\nu}$ also demonstrates that (ii) is necessary and sufficient for (c).
If the converse to theorem \ref{thm:CharOfGradSob} is true, as implied from results in \cite{zhikov2002homogenization}, then (i)-(ii) are necessary and sufficient for (a)-(c).
\begin{proof}
	Clearly, if we have smooth functions such that
	\begin{align*}
		\phi_n \lconv{\ltwo{\ddom}{\dddmes}} u, \quad \grad\phi_n\lconv{\ltwo{\ddom}{\dddmes}^2}\grad_{\dddmes}u,
	\end{align*}
	then using the fact that
	\begin{align*}
		\norm{\cdot}_{\ltwo{\ddom}{\dddmes}}^2 &= \norm{\cdot}_{\ltwo{\ddom}{\ddmes}}^2 + \norm{\cdot}_{\ltwo{\ddom}{\nu}}^2,
	\end{align*}
	we have that $\phi_n$ also converges in $\ltwo{\ddom}{\ddmes}$ and $\ltwo{\ddom}{\nu}$, as do its gradients in $\ltwo{\ddom}{\ddmes}^2$ and $\ltwo{\ddom}{\nu}^2$.
\end{proof}
Since we pose the (measure-theoretic) variational problem in the space $\gradSob{\ddom}{\dddmes}$, we then know that every function in that space also lives in $\gradSob{\ddom}{\ddmes}$ and $\gradSob{\ddom}{\nu}$, which gives us the properties we need to derive the edge-ODEs and vertex conditions (our quantum graph problem).