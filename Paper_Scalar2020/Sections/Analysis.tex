\section{Analysis and Derivation of System} \label{sec:SystemAndAnalysis}

Here the intention is to explain how we arrive at the system of equations for a singular structure problem from our measure theoretic formulation.
We recall theory from the previous sections of the paper to do this, and then proceed to discuss our numerical schemes and demonstrate band-gap structures and other conjectures we have made. \newline

Having provided definitions for the ``Sobolev spaces" we wish to use in section \ref{sec:Definitions}, we now provide an overview of the arguments that one employs to derive the quantum graph problem \tstk{qg problem we get reference} from the measure-theoretic \ref{MT system we start from}.
Provided in this chapter are the crucial results that the derivation leans heavily on, and commentary describing the direction of the arguments involved in proving each result.
Because the proofs themselves are quite verbose, we present these in an appendix \tstk{appendix ref} in the interest of readability. \newline

For the remainder of this section we assume the following convention:
\begin{convention} \label{conv:MeasTheoryProblemSetup}
	Let $\graph=\bracs{V,E}$ be the period graph of an embedded graph in $\reals^2$ with period cell $\ddom$, so $\graph\subset\ddom$.
	We restrict ourselves to considering straight-edges between vertices, so each $I_{jk}\in E$ is the line segment joining the vertices at either end, with lengths $l_{jk} = \norm{v_j-v_k}_2$.
	Let $r_{jk}$ and $e_{jk}$ be as given in \eqref{eq:EdgeParameterisation}, and let $n_{jk}$ be the unit normal to $I_{jk}$ such that $y_{jk} = \bracs{e_{jk}, n_{jk}}$ can be obtained by an orthonormal rotation $R_{jk}\in\mathrm{SO}(2)$ of the (canonical) axis vectors $x = \bracs{x_1, x_2}$, formally by $x = R_{jk}y_{jk}$.
\end{convention}
We begin with a more complete description of the functions belonging to $\gradZero{\ddom}{\lambda_{jk}}$ and $\gradZero{\ddom}{\ddmes}$, which in turn will help us understand the form of the functions $u\in\gradSobQM{\ddom}{\ddmes}$ through an understanding of their tangential gradient.

\subsection{On $\gradZero{\ddom}{\lambda_{jk}}$ and $\gradZero{\ddom}{\ddmes}$} \label{ssec:GradsOfZeroTheory}

\subsection{On $\gradSobQM{\ddom}{\ddmes}$} \label{ssec:SobSpacesTheory}

\subsection{Derivation and Formulation of Quantum Graph Problem} \label{ssec:Derivation}

\tstk{this needs to be moved to the right place in the text, in particular after the QG problem is derived - maybe even into the examples section?}
\begin{prop}[$M$-matrix entries] \label{prop:M-MatrixEntries}
	Let $\graph=\bracs{V,E}$ be an embedded graph on which the problem \eqref{eq:QGEquation}-\eqref{eq:QGVertexConditions} is posed.
	For each $I_{jk}\in E$ let $\qm_{jk} = \bracs{R_{jk}\qm}_2$ and $l_{jk} = \abs{I_{jk}}$.
	Suppose that $\dmap u = e_k$ where $e_k$ is the $k$\textsuperscript{th} canonical unit vector in $\reals^{\abs{V}}$.
	Then the $j$\textsuperscript{th} entry of $\nmap u$, and hence the $jk$\textsuperscript{th} entry in the $M$-matrix, is given by
	\begin{align*}
		\bracs{\nmap u}_j &= 
		\begin{cases}
			\!\begin{aligned}
				&0,
			\end{aligned}			
			& j \not\con k, \\
			\!\begin{aligned}
				&-\sum_{j\conLeft k} \effFreq e^{i\qm_{jk}l_{jk}} \csc\bracs{l_{jk}\effFreq} 
				\\ &\quad - \sum_{j\conRight k} \effFreq e^{-i\qm_{kj}l_{kj}} \csc\bracs{l_{kj}\effFreq},
			\end{aligned}
			& j\neq k, \ j\con k, \\
			\!\begin{aligned}
				&\sum_{j\con l} \effFreq\cot\bracs{l_{jl}\effFreq}
				\\ &\quad + 2\effFreq\sum_{j\conLeft j} \cot\bracs{l_{jj}\effFreq} - \cos\bracs{\qm_{jj}l_{jj}}\csc\bracs{l_{jj}\effFreq},
			\end{aligned}
			& j=k.
		\end{cases}
	\end{align*}
	Note the choice of $j\conLeft j$ in the contributions from loops is simply a convention, $j\conRight j$ is equivalent here.
	Also recall the convention for summing over $j\con k$:
	\begin{align*}
		\sum_{j\con k} \effFreq\cot\bracs{l_{jk}\effFreq} &= \sum_{j\conLeft k} \effFreq\cot\bracs{l_{jk}\effFreq}	+ \sum_{j\conRight k} \effFreq\cot\bracs{l_{kj}\effFreq}
	\end{align*}
\end{prop}
\begin{proof}
	The proof is an explicit computation, and follows the same idea as in \cite{ershova2014isospectrality} with adjustments for the fact that there are $\wavenumber$ and $\qm$ terms floating around.
	For each $k$, setting $\dmap u = e_k$ provides us with sufficient Dirichlet data at each vertex to eliminate the constants $C^{(jk)}_{\pm}$ in \eqref{eq:CurlEdgeEqnGenSol}.
	This in turn enables us to explicitly write the solutions $u_{3,jk}$, differentiate them, and read off their values at any relevant vertices - hence if one wishes, the form of the eigenfunctions can also be recovered via this method.
	This then provides us with the value of each term in the sum in $\bracs{\nmap u}_j$, for each $j$.
\end{proof}
Importantly this result demonstrates that the $M$-matrix can be thought of as a function of $\effFreq$ parametrised by $\qm$, hence will denote it by $M_{\qm}\bracs{\effFreq}$.