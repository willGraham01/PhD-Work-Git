\section{Analysis and Derivation of System} \label{sec:SystemAndAnalysis}

Here the intention is to explain how we arrive at the system of equations for a singular structure problem from our measure theoretic formulation.
We recall theory from the previous sections of the paper to do this, and then proceed to discuss our numerical schemes and demonstrate band-gap structures and other conjectures we have made. \newline

Having provided definitions for the ``Sobolev spaces" we wish to use in section \ref{sec:Definitions}, we now provide an overview of the arguments that one employs to derive the quantum graph problem \tstk{qg problem we get reference} from the measure-theoretic \ref{MT system we start from}.
Provided in this chapter are the crucial results that the derivation leans heavily on, and commentary describing the direction of the arguments involved in proving each result.
Because the proofs themselves are quite verbose, we present these in an appendix \tstk{appendix ref} in the interest of readability. \newline

For the remainder of this section we assume the following convention:
\begin{convention} \label{conv:MeasTheoryProblemSetup}
	Let $\graph=\bracs{V,E}$ be the period graph of an embedded graph in $\reals^2$ with period cell $\ddom$, so $\graph\subset\ddom$.
	We restrict ourselves to considering straight-edges between vertices, so each $I_{jk}\in E$ is the line segment joining the vertices at either end, with lengths $l_{jk} = \norm{v_j-v_k}_2$.
	Let $r_{jk}$ and $e_{jk}$ be as given in \eqref{eq:EdgeParameterisation}, and let $n_{jk}$ be the unit normal to $I_{jk}$ such that $y_{jk} = \bracs{e_{jk}, n_{jk}}$ can be obtained by an orthonormal rotation $R_{jk}\in\mathrm{SO}(2)$ of the (canonical) axis vectors $x = \bracs{x_1, x_2}$, formally by $x = R_{jk}y_{jk}$.
\end{convention}
We begin with a more complete description of the functions belonging to $\gradZero{\ddom}{\lambda_{jk}}$ and $\gradZero{\ddom}{\ddmes}$, which in turn will help us understand the form of the functions $u\in\gradSobQM{\ddom}{\ddmes}$ through an understanding of their tangential gradient.

\subsection{On $\gradZero{\ddom}{\lambda_{jk}}$ and $\gradZero{\ddom}{\ddmes}$} \label{ssec:GradsOfZeroTheory}
Crucial to our understanding of the functions (and their gradients) in $\gradSobQM{\ddom}{\ddmes}$ will be understanding the corresponding ``gradients of zero".
\tstk{argument is based of Zhikov's stuff, and is in appendix in full}
We work towards obtaining this understanding progressively; first we look to understand the set $\gradZero{\ddom}{\lambda_{jk}}$ when the edge $I_{jk}$ is assumed to be parallel to the $x_1$-axis, then employ a rotation argument to understand $\gradZero{\ddom}{\lambda_{jk}}$ for a general edge that is at an angle to the $x_1$-axis.
Given that the singular measure $\ddmes$ is just the sum of the individual singular measures supporting each edge, we can then prove that elements of $\gradZero{\ddom}{\ddmes}$ display the same behaviour as elements of $\gradZero{\ddom}{\lambda_{jk}}$ when restricted to the edge $I_{jk}$.
This argument also provides us with a clear geometric interpretation for what a ``gradient of zero" is, and the edge-wise characterisation we obtain for elements of $\gradZero{\ddom}{\ddmes}$ will carry through into how we describe functions in $\gradSobQM{\ddom}{\ddmes}$.
\tstk{should include here some of the key properties of gradients of zero sets, like being a closed linear subspace and density of smooth functions...? Unless I did this in the definitions chapter?}

We begin by describing $\gradZero{\ddom}{\lambda_{jk}}$ for when the edge $I_{jk}$ is parallel to the $x_1$-axis.
\begin{prop}[Gradients of Zero on a Segment Parallel to the $x_1$-axis] \label{prop:GradZeroParallelZhikov}
	Let $I$ be a segment in the $\bracs{x_1,x_2}$-plane parallel to the $x_1$-axis, and let $\lambda_I$ be the singular measure supported on $I$.
	Then 
	\begin{align*}
		\gradZero{\ddom}{\lambda_I} &= 
		\clbracs{
			\begin{pmatrix} 0 \\ f	\end{pmatrix}
			\ \vert \ f\in\ltwo{\ddom}{\lambda_I}
		}.
	\end{align*}
\end{prop}
\begin{proof}
	By combining proposition \ref{prop:GradZeroInvarientUnderQM} with a result in \cite{zhikov2000extension} \tstk{precise lemma/page citation}, this result follows, however we present a brief summary of the argument.
	Without loss of generality we assume $x_2=0$ on $I$.
	Additionally it suffices to show that the set on the right hand side includes all functions of the specified form when $f$ is smooth, as we can then apply a density argument. \newline
	
	So take some $f\in\smooth{\ddom}$, then the ``constant sequence" $\phi_n = \phi = x_2 f$ is such that
	\begin{align*}
		\phi_n\lconv{\ltwo{\ddom}{\lambda_I}}0, 
		&\quad \grad\phi_n\lconv{\ltwo{\ddom}{\lambda_I}^2} \begin{pmatrix} 0 \\ f \end{pmatrix}
		&\quad \toInfty{n}
	\end{align*}	 
	and so $\bracs{0,f}^\top\in\gradZero{\ddom}{\lambda_I}$. \newline
	
	We now prove that if $\bracs{f,0}^\top\in\gradZero{\ddom}{\lambda_I}$ then $f=0$.
	So suppose $\bracs{f,0}\in\gradZero{\ddom}{\lambda_I}$ and take an approximating sequence $\phi_n$ as in \eqref{eq:GradZeroDef}.
	Performing a change of variables via the map $r:\interval{I}\rightarrow I$ (as described for an edge $I_{jk}$ in convention \ref{conv:MeasTheoryProblemSetup}), and setting $\tilde{\phi}_n(t) := \phi_n\bracs{r(t)}$, we have
	\begin{align*}
		\tilde{\phi}_n\lconv{\ltwo{\interval{I}}{t}} 0, 
		&\quad \diff{\tilde{\phi}_n}{t}\lconv{\ltwo{\interval{I}}{t}} \tilde{f}
		&\quad \toInfty{n}.
	\end{align*}
	Hence $\tilde{f}$ is the distributional derivative (in the $\gradSob{\interval{I}}{t}$ sense) of the zero function, so we can conclude that $\tilde{f} = 0$, and thus $f = 0$, as we sought.
\end{proof}

Proposition \ref{prop:GradZeroParallelZhikov} provides the following interpretation for ``gradients of zero".
The measure $\lambda_I$ however can only ``see" along the segment $I$, as this is it's entire support.
As such $\lambda_I$ can only see the change in a function in the direction along the segment $I$, hence we find that $\gradZero{\ddom}{\lambda_I}$ consists of all the components of gradients that are directed perpendicular to $I$.
The following proposition reinforces this interpretation, although the argument is simply to invoke the result of proposition \ref{prop:GradZeroParallelZhikov} after applying the obvious rotation.
\begin{prop}[Rotation of Edge Gradients of Zero] \label{prop:RotationOfEdgeGradients}
	Consider the case when $\graph$ consists of a single edge (or segment) $I\subset\ddom$ with orthogonal co-ordinate system $y=\bracs{y_1,y_2}$, with $y_1$ parallel to $I$.
	Let $R$ be the orthogonal change of co-ordinates $x=Ry$ with $x=\bracs{x_1,x_2}$ the orthogonal co-ordinate system along the axes.
	Then
	\begin{align*}
		\gradZero{\ddom}{\lambda_I} 
		&= \clbracs{ R^{\top} \begin{pmatrix} 0 \\ f_2 \end{pmatrix} \ \vert \ f_2\in\ltwo{\ddom}{\lambda_I} }.
	\end{align*}
\end{prop}
With these two results, there is the following corollary which further reinforces the interpretation of $\gradZero{\ddom}{\lambda_I}$ given earlier.
\begin{cory} \label{cory:Grad0SingleEdge}
	Assume the hypothesis of proposition \ref{prop:RotationOfEdgeGradients}, and denote by $e_I$ the unit vector parallel to the segment $I$.
	Then
	\begin{align*}
		\gradZero{\ddom}{\lambda_I} &= \clbracs{z\in\ltwo{\ddom}{\lambda_I} \ \vert \ z\vert_{I}\cdot e_I = 0}.
	\end{align*}
\end{cory}

Using our understanding of gradients of zero on single segments, we can build up an understanding of gradients of zero on embedded graphs consisting of multiple edges.
This turns out to be the following characterisation; where each function in $\gradZero{\ddom}{\ddmes}$ behaves as a function in $\gradZero{\ddom}{\lambda_{jk}}$ when restricted to the edge $I_{jk}$.
\begin{prop} \label{prop:GradZeroGraph}
	Given convention \ref{conv:MeasTheoryProblemSetup}, we have that
	\begin{align*}
		\gradZero{\ddom}{\ddmes} &= \clbracs{g\in\ltwo{\ddom}{\ddmes}^2 \ \vert \ g\vert_{I_{jk}}\cdot e_{jk}=0 \ \forall I_{jk}\in E} \\
		&= \clbracs{g\in\ltwo{\ddom}{\ddmes}^2 \ \vert \ g\in\gradZero{\ddom}{\lambda_{jk}} \ \forall I_{jk}\in E}. \labelthis\label{eq:GradZeroSetRHS}
	\end{align*}
\end{prop}
\begin{proof}
	For the full proof, see appendix \tstk{appendix reference}.
	Showing that elements of $\gradZero{\ddom}{\ddmes}$ are contained in the set on the right-hand-side of \eqref{eq:GradZeroSetRHS} is straightforward due to the definition of $\ddmes$ and the fact that
	\begin{align*}
		\norm{\cdot}_{\ltwo{\ddom}{\ddmes}}^2 = \sum_{j\con k}\norm{\cdot}_{\ltwo{\ddom}{\lambda_{jk}}}^2,
	\end{align*}
	so if one has convergence in the $\ltwo{\ddom}{\ddmes}$-norm then we have convergence in each $\ltwo{\ddom}{\lambda_{jk}}$-norm.
	The opposite set inclusion involves a number of technical steps, and the full argument is given in the appendix.
	The gist of the argument involves showing that if $g$ is a member of the set on the right-had-side of \eqref{eq:GradZeroSetRHS}, then we can demonstrate that $g_{jk}\in\gradZero{\ddom}{\ddmes}$ (recall $g_{jk}$ is $g$ restricted to $I_{jk}$ then extended by zero to $\ddom$) given that $g_{jk}\in\gradZero{\ddom}{\lambda_{jk}}$.
	Then we can use the idea that $g = \sum_{j\con k} g_{jk}$ and that $\gradZero{\ddom}{\ddmes}$ is a closed, linear subspace to obtain membership of $g$ in $\gradZero{\ddom}{\ddmes}$; although in practice the expression for the sum is made more complex by the need to ensure good behaviour near the vertices of the graph.
\end{proof}

\subsection{On $\gradSobQM{\ddom}{\ddmes}$} \label{ssec:SobSpacesTheory}

\subsection{Derivation and Formulation of Quantum Graph Problem} \label{ssec:Derivation}

\tstk{prop \ref{prop:M-MatrixEntries} needs to be moved to the right place in the text, in particular after the QG problem is derived - maybe even into the examples section?}
\begin{prop}[$M$-matrix entries] \label{prop:M-MatrixEntries}
	Let $\graph=\bracs{V,E}$ be an embedded graph on which the problem \eqref{eq:QGEquation}-\eqref{eq:QGVertexConditions} is posed.
	For each $I_{jk}\in E$ let $\qm_{jk} = \bracs{R_{jk}\qm}_2$ and $l_{jk} = \abs{I_{jk}}$.
	Suppose that $\dmap u = e_k$ where $e_k$ is the $k$\textsuperscript{th} canonical unit vector in $\reals^{\abs{V}}$.
	Then the $j$\textsuperscript{th} entry of $\nmap u$, and hence the $jk$\textsuperscript{th} entry in the $M$-matrix, is given by
	\begin{align*}
		\bracs{\nmap u}_j &= 
		\begin{cases}
			\!\begin{aligned}
				&0,
			\end{aligned}			
			& j \not\con k, \\
			\!\begin{aligned}
				&-\sum_{j\conLeft k} \effFreq e^{i\qm_{jk}l_{jk}} \csc\bracs{l_{jk}\effFreq} 
				\\ &\quad - \sum_{j\conRight k} \effFreq e^{-i\qm_{kj}l_{kj}} \csc\bracs{l_{kj}\effFreq},
			\end{aligned}
			& j\neq k, \ j\con k, \\
			\!\begin{aligned}
				&\sum_{j\con l} \effFreq\cot\bracs{l_{jl}\effFreq}
				\\ &\quad + 2\effFreq\sum_{j\conLeft j} \cot\bracs{l_{jj}\effFreq} - \cos\bracs{\qm_{jj}l_{jj}}\csc\bracs{l_{jj}\effFreq},
			\end{aligned}
			& j=k.
		\end{cases}
	\end{align*}
	Note the choice of $j\conLeft j$ in the contributions from loops is simply a convention, $j\conRight j$ is equivalent here.
	Also recall the convention for summing over $j\con k$:
	\begin{align*}
		\sum_{j\con k} \effFreq\cot\bracs{l_{jk}\effFreq} &= \sum_{j\conLeft k} \effFreq\cot\bracs{l_{jk}\effFreq}	+ \sum_{j\conRight k} \effFreq\cot\bracs{l_{kj}\effFreq}
	\end{align*}
\end{prop}
\begin{proof}
	The proof is an explicit computation, and follows the same idea as in \cite{ershova2014isospectrality} with adjustments for the fact that there are $\wavenumber$ and $\qm$ terms floating around.
	For each $k$, setting $\dmap u = e_k$ provides us with sufficient Dirichlet data at each vertex to eliminate the constants $C^{(jk)}_{\pm}$ in \eqref{eq:CurlEdgeEqnGenSol}.
	This in turn enables us to explicitly write the solutions $u_{3,jk}$, differentiate them, and read off their values at any relevant vertices - hence if one wishes, the form of the eigenfunctions can also be recovered via this method.
	This then provides us with the value of each term in the sum in $\bracs{\nmap u}_j$, for each $j$.
\end{proof}
Importantly this result demonstrates that the $M$-matrix can be thought of as a function of $\effFreq$ parametrised by $\qm$, hence will denote it by $M_{\qm}\bracs{\effFreq}$.