\section{Effective System and Analysis} \label{sec:SystemAndAnalysis}

Here the intention is to explain how we arrive at the system of equations for a singular structure problem from our measure theoretic formulation.
We recall theory from the previous sections of the paper to do this, and then proceed to discuss our numerical schemes and demonstrate band-gap structures and other conjectures we have made.

\tstk{this needs to be moved to the right place in the text, in particular after the QG problem is derived}
\begin{prop}[$M$-matrix entries] \label{prop:M-MatrixEntries}
	Let $\graph=\bracs{V,E}$ be an embedded graph on which the problem \eqref{eq:QGEquation}-\eqref{eq:QGVertexConditions} is posed.
	For each $I_{jk}\in E$ let $\qm_{jk} = \bracs{R_{jk}\qm}_2$ and $l_{jk} = \abs{I_{jk}}$.
	Suppose that $\dmap u = e_k$ where $e_k$ is the $k$\textsuperscript{th} canonical unit vector in $\reals^{\abs{V}}$.
	Then the $j$\textsuperscript{th} entry of $\nmap u$, and hence the $jk$\textsuperscript{th} entry in the $M$-matrix, is given by
	\begin{align*}
		\bracs{\nmap u}_j &= 
		\begin{cases}
			\!\begin{aligned}
				&0,
			\end{aligned}			
			& j \not\con k, \\
			\!\begin{aligned}
				&-\sum_{j\conLeft k} \effFreq e^{i\qm_{jk}l_{jk}} \csc\bracs{l_{jk}\effFreq} 
				\\ &\quad - \sum_{j\conRight k} \effFreq e^{-i\qm_{kj}l_{kj}} \csc\bracs{l_{kj}\effFreq},
			\end{aligned}
			& j\neq k, \ j\con k, \\
			\!\begin{aligned}
				&\sum_{j\con l} \effFreq\cot\bracs{l_{jl}\effFreq}
				\\ &\quad + 2\effFreq\sum_{j\conLeft j} \cot\bracs{l_{jj}\effFreq} - \cos\bracs{\qm_{jj}l_{jj}}\csc\bracs{l_{jj}\effFreq},
			\end{aligned}
			& j=k.
		\end{cases}
	\end{align*}
	Note the choice of $j\conLeft j$ in the contributions from loops is simply a convention, $j\conRight j$ is equivalent here.
	Also recall the convention for summing over $j\con k$:
	\begin{align*}
		\sum_{j\con k} \effFreq\cot\bracs{l_{jk}\effFreq} &= \sum_{j\conLeft k} \effFreq\cot\bracs{l_{jk}\effFreq}	+ \sum_{j\conRight k} \effFreq\cot\bracs{l_{kj}\effFreq}
	\end{align*}
\end{prop}
\begin{proof}
	The proof is an explicit computation, and follows the same idea as in \cite{ershova2014isospectrality} with adjustments for the fact that there are $\wavenumber$ and $\qm$ terms floating around.
	For each $k$, setting $\dmap u = e_k$ provides us with sufficient Dirichlet data at each vertex to eliminate the constants $C^{(jk)}_{\pm}$ in \eqref{eq:CurlEdgeEqnGenSol}.
	This in turn enables us to explicitly write the solutions $u_{3,jk}$, differentiate them, and read off their values at any relevant vertices - hence if one wishes, the form of the eigenfunctions can also be recovered via this method.
	This then provides us with the value of each term in the sum in $\bracs{\nmap u}_j$, for each $j$.
\end{proof}
Importantly this result demonstrates that the $M$-matrix can be thought of as a function of $\effFreq$ parametrised by $\qm$, hence will denote it by $M_{\qm}\bracs{\effFreq}$.