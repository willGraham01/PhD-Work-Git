\section{Definitions and Theory} \label{sec:DefsAndTheory}

In this section we will (lightly) review the definitions and theory that we will need to obtain our effective equations on a graph.
We need to be prudent in terms of length - don't include all the lengthy proofs and technical details, nor every definition.
We'll also need to talk about both quantum/metric graphs and the Sob-spaces stuff here, as we don't intend to introduce anything new in the next section (only to recall theory here and explain how our equations are derived).

\subsection{Operator of Interest} \label{ssec:ktCurls}
In this section we should introduce our kt-curl operator (or kt-nabla to be more specific) and explain why it is the object of interest to us.
This may require us to go through the motion of explaining how we transition from a problem in $\reals^3$ to a problem on $[0,1]^2$ via Gelfand and Fourier transforms, if we did not mention this in the introductory chapter.

This should then transition nicely into our light overview of the theory we developed concerning this operator, and the quirks of the spaces of functions that we want to work with.

\subsection{Function Spaces} \label{ssec:FunctionSpaces}
This is the natural follow on from introducing the $\ktgrad$ operator in the previous subsection.
We outline the definitions that we are working with and will then explore the main results of our theory concerning gradients (and tangential gradients), curls, divergence-free functions and the like.
By the end of this subsection we should at least have included results that illustrate what each of the ``tangential" gradients/curls look like, and what properties divergence-free functions have.

\subsection{Quantum/Metric Graphs} \label{ssec:QuantumGraphs}
This section will need a brief overview of the theory we are borrowing from Quantum Graphs, and previous works on them.
Mainly this is just how to associate each interval to an edge, the notation that we are going to use, and how to setup a quantum graph problem so that when we reach one in the later sections, we can say ``ta-da"!