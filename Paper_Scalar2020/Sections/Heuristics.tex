\section{Definitions and Theory} \label{sec:Heuristics}

In this section we will (lightly) review the definitions and theory that we will need to obtain our effective equations on a graph.
We need to be prudent in terms of length - don't include all the lengthy proofs and technical details, nor every definition.
We'll also need to talk about both quantum/metric graphs and the Sob-spaces stuff here, as we don't intend to introduce anything new in the next section (only to recall theory here and explain how our equations are derived).

\subsection{Quantum/Metric Graphs} \label{ssec:QuantumGraphs}
This section will need a brief overview of the theory we are borrowing from Quantum Graphs, and previous works on them.
Mainly this is just how to associate each interval to an edge, the notation that we are going to use, and how to setup a quantum graph problem so that when we reach one in the later sections, we can say ``ta-da"!

We begin by briefly outlining the concept of a quantum graph, introducing the function spaces associated with them, and defining differential operators on them.
Let $\graph=\bracs{V,E}$ be a directed graph with vertices $v_j\in V$ and edges $I_{jk}=\bracs{v_j, v_k}\in E$, where $I_{jk}$ is directed from vertex $v_j$ (``on the left") to vertex $v_k$ (``on the right").
Strictly we should use the notation $I_{jk}^l$ for the edges, as in general there may be $l$ such connections from $v_j$ to $v_k$, however this fine detail will not play a significant role in our analysis and so we drop this additional superscript to avoid notational clutter.
Each edge $I_{jk}$ is assigned a length $l_{jk}>0$ and interval $\sqbracs{0,l_{jk}}$, this interval we also denote by $I_{jk}$ (for reasons that will become apparent shortly).
We also assign each vertex $v_j$ a ``coupling constant" $\alpha_j\in\reals$ \tstk{should only be $<0$ I think, due to our normal derivative choice}.
Then the graph $\graph$ is called a quantum graph. \newline

By providing each edge with a length and hence interval, we can construct functions on quantum graphs by describing their behaviour on each edge of the graph, and hence can produce function spaces such as
\begin{subequations} \label{eq:GraphFuncSpaces}
	\begin{align}
		L^2\bracs{\graph} := \bigoplus_{I_{jk}\in E} \ltwo{I_{jk}}{t},
		&\quad H^1\bracs{\graph} := \bigoplus_{I_{jk}\in E} \gradSob{I_{jk}}{t}, \\
		H^2\bracs{\graph} := \bigoplus_{I_{jk}\in E} H^2_\mathrm{grad}\bracs{I_{jk}, \md t}. &
	\end{align}
\end{subequations}
Because we will mainly be working on the edges of our graphs, we define $u_{jk} = u\vert_{I_{jk}}$ to be the restriction of a function $u$ defined on $\graph$ to the edge $I_{jk}$, extended by zero.
As the directed edge $I_{jk}$ is associated to the interval $\sqbracs{0,l_{jk}}$, we will also use the shorthand
\begin{align*}
	u_{jk}\bracs{v_j} &= u_{jk}\bracs{0}, &\quad
	u_{jk}\bracs{v_k} = u_{jk}\bracs{l_{jk}}, \\
	u_{kj}\bracs{v_j} &= u_{kj}\bracs{l_{kj}}, &\quad
	u_{kj}\bracs{v_k} = u_{kj}\bracs{0}.
\end{align*}
If $u$ is continuous at a vertex $v_j$, we will use the notation $u\bracs{v_j}$ for this value.
This will be convenient later, when we work with quantum graphs that have been obtained from graphs embedded in $\reals^2$.
We also require a notion of ``normal derivative" for our functions when we approach the endpoints of an edge in the graph $\graph$.
To this end, we adopt the following convention;
\begin{align*}
	\diff{}{t}u_{jk}\bracs{v_j} &= -u_{jk}'\bracs{v_j} = -\lim_{t\rightarrow0}u_{jk}'(t), \\
	\diff{}{t}u_{jk}\bracs{v_k} &= u_{jk}'\bracs{v_k} = \lim_{t\rightarrow l_{jk}}u_{jk}'(t),
\end{align*}
for suitably differentiable functions $u$ on $\graph$. 
In effect, we are using an ``exterior facing" normal derivative for our edges. 
Before we discuss differential operators on quantum graphs we introduce a small amount of notation to make the equations we present more readable.
We use the notation $j\conLeft k$ to mean ``the vertex $v_j$ connects to $v_k$ with $v_j$ on the left", and $j\conRight k$ to mean ``the vertex $v_j$ connects to $v_k$ with $v_j$ on the right".
Lastly, we use $j\con k$ to mean ``the vertex $v_j$ connects to $v_k$" in cases where we don't care about the direction of the connection.
Formally we have
\begin{align*}
	j\conLeft k &\Leftrightarrow I_{jk}\in E, \quad
	j\conRight k \Leftrightarrow I_{kj}\in E, \\
	j\con k &\Leftrightarrow I_{jk}\in E \text{ or } I_{kj}\in E.
\end{align*}
We can then utilise this notation as shorthand in summations involving edges or vertices in the following manner; for a fixed $v_j\in V$
\begin{align*}
	\sum_{j\conLeft k} = \sum_{\substack{k \\ I_{jk}\in E}}, \quad
	\sum_{j\conLeft k} = \sum_{\substack{k \\ I_{kj}\in E}}, \quad
	\sum_{j\con k} = \sum_{j\conLeft k} + \sum_{j\conRight k}.
\end{align*}

Differential operators (or differential equations) on quantum graphs can then be defined by specifying their domains and their action on each edge of the graph $\graph$.
To obtain a well-posed problem (strictly speaking, a self-adjoint operator \tstk{refs!}) on a graph however will require specification of boundary conditions at the vertices, which are essentially the boundaries of our edges $I_{jk}$.
Specifying these ``boundary conditions" or ``vertex conditions" also imbues the problem with an appreciation for the connectivity of the graph (which the function spaces in \ref{eq:GraphFuncSpaces} do not convey on their own).
Whilst there are several choices one can make for the vertex conditions, we will only be interested in imposing continuity of the solution at each vertex, and a Kirchoff-like condition on the derivatives of the solution.
By way of example, let us demonstrate how we define the operator $-\ddiff{}{t}$ on $\graph$.
The domain of $\mathcal{A}$ shall be
\begin{align} \label{eq:ExampleOppDomainDef}
	\mathrm{dom}\mathcal{A} &= \clbracs{ u\in H^2\bracs{\graph} \ \vert \ \text{At each } v_j\in V, \ u \text{ is continuous and } \sum_{j\con k}\diff{u_{jk}}{t}\bracs{v_j} = \alpha_j u\bracs{v_j} },
\end{align}
whilst it's action on each edge is
\begin{align} \label{eq:ExampleOppEdgeDef}
	\mathcal{A} &= -\ddiff{}{t} \quad\text{on each } I_{jk}\in E.
\end{align}
Of course, by ``on each $I_{jk}\in E$" we mean ``on the interval $\sqbracs{0,l_{jk}}$ that we associate to $I_{jk}\in E$".
Then for a function $f\in L^2\bracs{\graph}$ we can pose the resolvent problem of finding $u\in\mathrm{dom}\mathcal{A}$ such that
\begin{align*}
	\mathcal{A}u &= f;
\end{align*}
or alternatively can consider the spectral problem of finding eigenpairs $\bracs{\lambda,u}\in\complex\times\mathrm{dom}\mathcal{A}$ such that
\begin{align*}
	\mathcal{A}u &= \lambda u.
\end{align*}
As the spaces in \eqref{eq:GraphFuncSpaces} break down into edge-wise components which are acted on individually by $\mathcal{A}$, and only linked through the vertex conditions, we can rewrite both of these problems as a set of ODEs on intervals coupled through vertex conditions;
\begin{align*}
	\mathcal{A}u = f \quad\Leftrightarrow\quad &
	\begin{cases}
		-\ddiff{u_{jk}}{t} = f_{jk} \ \text{on } I_{jk}, \\
		u \text{ is continuous at each } v_j\in V, \\
		\sum_{j\con k}\diff{u_{jk}}{t}\bracs{v_j} = \alpha_j u\bracs{v_j} \ \forall v_j\in V
	\end{cases} \\
	\mathcal{A}u = \lambda u \quad\Leftrightarrow\quad &
	\begin{cases}
		-\ddiff{u_{jk}}{t} = \lambda u_{jk} \ \text{on } I_{jk}, \\
		u \text{ is continuous at each } v_j\in V, \\
		\sum_{j\con k}\diff{u_{jk}}{t}\bracs{v_j} = \alpha_j u\bracs{v_j} \ \forall v_j\in V.
	\end{cases}
\end{align*}

\subsection{Embedded Graphs and Singular Measures} \label{ssec:EmbeddedGraphs}
Quantum graphs and differential equations on them will be what we aim to translate our abstract, measure-theoretic problems into (\tstk{section reference}), but the quantum graphs themselves will arise from embedded graphs in $\reals^2$.
We say a (directed) graph $\graph = \bracs{V,E}$ is embedded into $\reals^2$ if each vertex $v_j$ is associated to a point which we also label $v_j\in\reals^2$; and each edge $I_{jk}$ associated to a curve $\gamma_{jk}\subset\reals^2$ with (arc-)length $l_{jk}$, such that there is a continuous map
\begin{align*}
	r: \sqbracs{0,l_{jk}} \rightarrow \gamma_{jk}, \quad r(0) = v_j, \quad r\bracs{l_{jk}} = v_k.
\end{align*}
Again, we will drop the distinction between $I_{jk}$ and $\gamma_{jk}$, simply using $I_{jk}$ for both.
It is clear from this definition how an embedded graph gives rise to a corresponding quantum graph, or how a quantum graph could be used to construct an embedded graph.
The choice of $\reals^2$ is simply to cover the situations we will be considering in this work, there are more general definitions depending on the choice of space to ``embed" the graph into.
For a unit vector $x\in\reals^2$, an embedded graphs is said to be $T$-periodic in the direction $x$ if it is invariant under the translation $Tx$ applied to it's vertices and edges.
By convention, $T$ refers to the minimal such scalar $T>0$ for which this happens.
If $\graph$ is periodic in the axial directions $e_1, e_2$ (with scalars $T_1, T_2$ respectively), then we can define the ``period cell" or ``unit cell" of $\graph$ in the obvious manner.
Take the intersection of the graph $\graph$ with the region $\mathcal{P} = \sqbracs{0,T_1}\times\sqbracs{0,T_2}$ and match the left boundary to the right, and top boundary to the bottom - essentially view the graph in the region $\mathcal{P}$ as living on the torus, as in figure \ref{fig:PeriodCellIllustration}.
\begin{figure}[b!]
	\centering
	\begin{subfigure}[t]{0.45\textwidth}
		\centering
		\includegraphics[height=4.5cm]{Diagram_PeriodCellFullLattice.pdf}
		\caption{\label{fig:Diagram_PeriodCellFullLattice} A periodic graph embedded into $\reals^2$, with the period cell marked.}
	\end{subfigure}
	~
	\begin{subfigure}[t]{0.45\textwidth}
		\centering
		\includegraphics[height=4.5cm]{Diagram_PeriodCellEdgeAssociation.pdf}
		\caption{\label{fig:Diagram_PeriodCellEdgeAssociation} The period cell of the graph in \ref{fig:Diagram_PeriodCellFullLattice}, illustrating how the edges of the period cell are associated.}
	\end{subfigure}
	\\
	\begin{subfigure}[b]{0.75\textwidth}
		\centering
		\includegraphics[scale=	1.0]{Diagram_PeriodCellOnTorus.pdf}
		\caption{\label{fig:Diagram_PeriodCellOnTorus} An illustration of the period cell being situated on a torus.}
	\end{subfigure}
	\caption{\label{fig:PeriodCellIllustration} Illustrating a periodic cell of a periodic embedded graph.}
\end{figure} 
Obviously the period cell of an embedded periodic graph is itself an embedded graph. \newline

In the work that follows we restrict ourselves to considering straight-edges between vertices, so the curves $\gamma_{jk} = \sqbracs{v_j, v_k}$ are simply the line segments joining the vertices at either end, with lengths $l_{jk} = \norm{v_j-v_k}_2$.
As such, for each edge $I_{jk}$ in our embedded graphs, we can use the parametrisation
\begin{align} \label{eq:EdgeParameterisation}
	r_{jk}:\sqbracs{0, l_{jk}} \rightarrow I_{jk},
	&\quad r_{jk}\bracs{t} = v_j + te_{jk},
\end{align}
where $e_{jk}$ is the unit vector parallel to $I_{jk}$ and directed from $v_j$ to $v_k$ (consistent with the direction of the edge $I_{jk}$ itself).
Note that we also have $r_{jk}'(t) = e_{jk}$.

\subsubsection{Singular Measures} \label{sssec:SingularMeasures}
Whilst discussing embedded graphs we should introduce the object which will link quantum graph problems (as in section \ref{ssec:QuantumGraphs}) and the measure-theoretic problems that we will take as our start point \tstk{ref when it exists}.
For an embedded graph $\graph = \bracs{V,E}$ and for each $I_{jk}\in E$, define the (Borel) measure $\lambda_{jk}$ as the measure which supports 1D Lebesgue measure on the edge $I_{jk}$.
So for each Borel set $B$ we have that 
\begin{align*}
	\lambda_{jk}\bracs{B} = \lambda_{1}\bracs{r_{jk}^{-1}\bracs{B \cap \sqbracs{0,l_{jk}}}}
\end{align*}
where $\lambda_1$ is the 1D-Lebesgue measure on $\reals$, and $r_{jk}$ is the parametrisation of the edge $I_{jk}$ (see \eqref{eq:EdgeParameterisation}).
Then set $\ddmes$ to be the (Borel) measure defined by
\begin{align*}
	\ddmes\bracs{B} = \sum_{v_j\in V}\sum_{j\conLeft k} \lambda_{jk}\bracs{B}.
\end{align*}
Then $\ddmes$ is the ``singular measure that supports $\graph$"; or alternatively the ``singular measure on $\graph$", or the ``(singular) measure that supports the edges of $\graph$".
For a graph embedded into a 2D domain, the singular measure is illustrated in figure \ref{fig:Diagram_SingularMeasures2D}.
\begin{figure}[t!]
	\centering
	\includegraphics[scale=0.85]{Diagram_SingularMeasure2D.pdf}
	\caption{\label{fig:Diagram_SingularMeasure2D} For a graph embedded in $\reals^2$, the $\ddmes$-measure of any Borel set $B$ is obtained from summing the contributions of each $\lambda_{jk}$, as indicated by the thickened and coloured lines.
	Sets that do not intersect $\graph$ have zero measure.}
\end{figure} \newline

The singular measure of a graph will be the key component in our measure-theoretic formulations, which enables us to establish ideas of derivatives whilst working on a domain which has no ``area" in the Lebesgue-sense.

\subsection{Operator of Interest} \label{ssec:ktCurls}
In this section we should introduce our kt-curl operator (or kt-nabla to be more specific) and explain why it is the object of interest to us.
This may require us to go through the motion of explaining how we transition from a problem in $\reals^3$ to a problem on $[0,1]^2$ via Gelfand and Fourier transforms, if we did not mention this in the introductory chapter.

This should then transition nicely into our light overview of the theory we developed concerning this operator, and the quirks of the spaces of functions that we want to work with.

\subsection{Function Spaces} \label{ssec:FunctionSpaces}
This is the natural follow on from introducing the $\ktgrad$ operator in the previous subsection.
We outline the definitions that we are working with and will then explore the main results of our theory concerning gradients (and tangential gradients), curls, divergence-free functions and the like.
By the end of this subsection we should at least have included results that illustrate what each of the ``tangential" gradients/curls look like, and what properties divergence-free functions have.

