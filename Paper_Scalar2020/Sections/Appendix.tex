\section{Appendix} \label{sec:Appendix}
\tstk{change appendix numbering system from $X.Y$ Appendix, to Appendix: A,B,C etc. So Appendix: Singular Measures should read as Appendix A: Singular Measures}

\subsubsection{Appendix: Singular Measures} \label{ssec:SingularMeasures}
Embedded graphs (section \ref{ssec:EmbeddedGraphs} allow us to introduce the measure which links quantum graph problems to the measure-theoretic problems we first introduced in section \ref{ssec:OurSystem}
For an embedded graph $\graph = \bracs{\vertSet,\edgeSet}$ and for each $I_{jk}\in \edgeSet$, define the (Borel) measure $\lambda_{jk}$ as the measure which supports the one-dimensional Lebesgue measure on the edge $I_{jk}$.
So for each Borel set $B$ we have
\begin{align*}
	\lambda_{jk}\bracs{B} = \lambda_{1}\bracs{r_{jk}^{-1}\bracs{B \cap I_{jk}}}
\end{align*}
where $\lambda_1$ is the Lebesgue measure on $\reals$, and $r_{jk}$ is the parametrisation of the edge $I_{jk}$ (see \eqref{eq:GeneralCurveParam}).
Then set $\ddmes$ to be the (Borel) measure defined by
\begin{align*}
	\ddmes\bracs{B} = \sum_{v_j\in \vertSet}\sum_{j\conLeft k} \lambda_{jk}\bracs{B}.
\end{align*}
Then $\ddmes$ is the ``singular measure that supports $\graph$"; or alternatively the ``singular measure on $\graph$", or the ``(singular) measure that supports the edges of $\graph$". \tstk{didn't include the mass-on-the-vertices (coupling constant / vertex order) here, which will be needed for when we consider the scaled-vertex example.}
For a graph embedded into a 2D domain, the singular measure is illustrated in figure \ref{fig:Diagram_SingularMeasure2D}.
\begin{figure}[b!]
	\centering
	\includegraphics[scale=0.85]{Diagram_SingularMeasure2D.pdf}
	\caption{\label{fig:Diagram_SingularMeasure2D} For a graph embedded in $\reals^2$, the $\ddmes$-measure of any Borel set $B$ is obtained from summing the contributions of each $\lambda_{jk}$, as indicated by the thickened and coloured lines.
	Sets that do not intersect $\graph$ have zero measure.}
\end{figure} \newline
The singular measure of a graph will be the key component in our measure-theoretic formulations, which enables us to establish ideas of derivatives whilst working on a domain which has no ``area" in the Lebesgue-sense.

\subsection{Appendix: Key Measure Theoretic Concepts} \label{ssec:MeasureTheory}
In this section we address the question of how one understands the equations \eqref{eq:WholeSpaceLaplaceEqn} and \eqref{eq:PeriodCellLaplaceStrongForm}, by outlining the appropriate differential operators and function spaces that are required.
Attempting to pose a boundary-value problem on a singular-structure, by drawing analogy to the ``ingredients" of a boundary-value problem on a thin structure, runs into problems.
These are due to the singular structure lacking a domain interior from the perspective of the space it is embedded in, and thus the notion of boundary values ceases to make sense, and this problem persists even after taking a Gelfand transform to bring us to a family of problems on the period cell of the graph.
This issue is resolved by not abandoning what we believe are the ingredients of a boundary-value problem, but rather by reworking our concepts of integration and differentiation so that they respect the fact that we are looking at a problem on the singular-structure itself, despite the fact that it is embedded in a higher-dimensional space. \tstk{zhikov mention?}

\tstk{appendix subsection here}
Let $\graph=\bracs{\vertSet,\edgeSet}$ be a finite graph embedded (as detailed in section \ref{sec:QuantumGraphs}) into the region $\ddom=\sqbracs{0,T_1}\times\sqbracs{0,T_2}\subset\reals^2$, which will serve as our singular structure.
For $\qm=\bracs{\qm_1,\qm_2}\in\left[-\frac{\pi}{T_1},\frac{\pi}{T_1}\right)\times\left[-\frac{\pi}{T_2},\frac{\pi}{T_2}\right)$ define the shifted gradient operator $\tgrad$ on smooth functions $\phi\in\smooth{\ddom}$ by
\begin{align*}
	\tgrad\phi &= \begin{pmatrix} \partial_1\phi + i\qm_1\phi \\ \partial_2\phi + i\qm_2\phi \end{pmatrix}.
\end{align*}
Recall that we pose \eqref{eq:WholeSpaceLaplaceEqn} on a periodic, embedded graph $\graph$ in $\reals^2$, and the operator $\tgrad$ arises after using a Gelfand transform to move us to a family of problems on the period cell of $\graph$.
Let $\ddmes$ be the singular measure supported on $\graph$, and for each $I_{jk}\in \edgeSet$, let $\lambda_{jk}$ be the singular measure supporting the edge $I_{jk}$, as in section \ref{ssec:SingularMeasures}.
Then one can construct the set
\begin{align} \label{eq:WSetDefintion}
	W^{\qm}_\graph &= \overline{ \clbracs{ \bracs{\phi, \tgrad\phi} \ \vert \ \phi\in\smooth{\ddom} } },
\end{align}
where the closure is taken in the $\ltwo{\ddom}{\ddmes}\times\ltwo{\ddom}{\ddmes}^2$ norm.
$W^{\qm}_{jk}$ is similarly constructed from the closure of set of pairs $\bracs{\phi, \tgrad\phi}$ in the $\ltwo{\ddom}{\lambda_{jk}}\times\ltwo{\ddom}{\lambda_{jk}}^2$ norm, for each $I_{jk}\in\edgeSet$.
The idea here is to construct an analogy of a Sobolev space for functions defined on our singular-structure, and hence obtain a concept of (weak) derivative.
This is the reason for us taking the closure of the set of pairs $\bracs{\phi, \tgrad\phi}$ in \eqref{eq:WSetDefintion}, one hopes that if a sequence of smooth functions converges in $L^2$ to $u$, then their gradients also converge in $L^2$ to a function that represents the gradient of $u$ in a sense that is consistent with the measure $\ddom$.
Regrettably this does not turn out to be as simple a process as we would hope; we can immediately notice that if $\bracs{u,g_1},\bracs{0,g_2}\in W^{\qm}_\graph$ then clearly $\bracs{u, g_1+g_2}\in W^{\qm}_\graph$ too.
This means that if $\bracs{u,g}\in W^{\qm}_\graph$ we cannot call $g$ ``the gradient" of $u$, which would imply that $W^{\qm}_\graph$ isn't quite the set of functions we want to work with when it comes to singular-structure problems.
However, $W^{\qm}_\graph$ does contain all the functions we should be working with, we just need to deal make sense of this ``non-uniqueness of gradients" issue. \newline

We define the ``set of $\ddmes$-gradients of zero" as
\begin{align} \label{eq:GradZeroDef}
	\gradZero{\ddom}{\ddmes} &= \clbracs{ g \ \vert \ \bracs{0,g}\in W^{\qm}_\graph }, \\
	&= \clbracs{ g \ \vert \ \exists\phi_n\in\smooth{\ddom} \text{ s.t. } \phi_n\lconv{\ltwo{\ddom}{\ddmes}}0, \tgrad\phi_n\lconv{\ltwo{\ddom}{\ddmes}^2}g }
\end{align}
which is a closed, linear subspace of $\ltwo{\ddom}{\ddmes}^2$. 
\tstk{should include here some of the key properties of gradients of zero sets, like being a closed linear subspace and density of smooth functions. See Zhikov's appendix in his papers, these properties are needed for when we do the graph-wide extension theorem}
It can be shown that $\gradZero{\ddom}{\ddmes}$ does not depend on the value of $\qm$, which is why the notation lacks a $\qm$ symbol:
\begin{prop}[Gradients of Zero are Invariant Under Quasi-Momentum] \label{prop:GradZeroInvarientUnderQM}
	For any fixed $\qm\in[-\pi,\pi)^2$, we have that
	\begin{align*}
		\mathcal{G}^0_{\ddom, \md \ddmes} := \clbracs{ g \ \vert \ \bracs{0,g}\in W^0_\graph} &= 
		\clbracs{ g \ \vert \ bracs{0,g}\in W^{\qm}_\graph } =: \mathcal{G}^{\qm}_{\ddom, \md \ddmes}.
	\end{align*}
\end{prop}
\begin{proof}
	Given that $\tgrad\phi = \grad\phi + i\qm\phi$ for smooth $\phi$, if $g\in\mathcal{G}^0_{\ddom, \md \ddmes}$ there exists a sequence of smooth functions $\phi_n$ such that 
	\begin{align*}
		\phi_n\lconv{\ltwo{\ddom}{\ddmes}}0, &\grad\phi_n\lconv{\ltwo{\ddom}{\ddmes}^2}g.
	\end{align*}
	But then clearly
	\begin{align*}
		\phi_n\lconv{\ltwo{\ddom}{\ddmes}}0, &\tgrad\phi_n = \grad\phi_n + i\qm\phi_n \lconv{\ltwo{\ddom}{\ddmes}^2} g + 0 = g,
	\end{align*}
	so $g\in\mathcal{G}^\qm_{\ddom, \md \ddmes}$.
	The proof of the opposite inclusion is similar.
\end{proof}

For each $u\in\ltwo{\ddom}{\ddmes}$ there exists a $\grad_\ddmes u\in\gradZero{\ddom}{\ddmes}^{\perp}$ such that any pair $\bracs{u,z}\in W^{\qm}_\graph$ can be written as $\bracs{u,\grad_\ddmes u + g}$ where $g\in\gradZero{\ddom}{\ddmes}$.
We call $\grad_\ddmes u$ the \emph{($\ddom$-)tangential derivative} of $u$ and it is unique in this sense (see \tstk{ref to Zhikov}); which allows us to construct the ``Sobolev space"
\begin{align*}
	\gradSobQM{\ddom}{\ddmes} &= \clbracs{ \bracs{u, \grad_\ddmes u}\in W^{\qm}_\graph \ \vert \ \grad_\ddmes u \in \gradZero{\ddom}{\ddmes}^{\perp} }.
\end{align*}
Since $\grad_\ddmes u$ is unique, we will use the shorthand $u\in\gradSobQM{\ddom}{\ddmes}$ to refer to the pair $\bracs{u, \grad_\ddmes u}$. \tstk{worth a note about how tang gradient isn't always the gradient we want when solving equations with non-identity elliptic $A$, again a Zhikov reference. For us though, this definition is convenient because we use $A=I$ throughout.}
We can follow a similar process to define the sets $\gradZero{\ddom}{\lambda_{jk}}$ and $\gradSobQM{\ddom}{\lambda_{jk}}$. 
These constructions give us concepts of (weak) derivative to work with; however there is still work to be done towards understanding the form of these tangential derivatives, which will be key to the derivation of our system \eqref{eq:QGFullSystem} in section \ref{sec:SystemDerivation}.
This is the focus of sections \ref{ssec:GradsOfZeroTheory} through \ref{ssec:SobSpacesTheory}. \newline

Before we continue however, we can now precisely state what we mean when we write equation \eqref{eq:PeriodCellLaplaceStrongForm}; a pair $u\in\gradSobQM{\ddom}{\ddmes}$ solves \eqref{eq:PeriodCellLaplaceStrongForm} if and only if (see \eqref{eq:PeriodCellLaplaceWeakForm})
\begin{align*}
	\integral{\ddom}{\tgrad_{\ddmes}u\cdot\overline{\tgrad_{\ddmes}\phi}}{\ddmes} &= \omega^2\integral{\ddom}{u\overline{\phi}}{\ddmes}, \quad\forall \phi\in\smooth{\ddom},
\end{align*}
and one has a similar interpretation for \eqref{eq:WholeSpaceLaplaceEqn}.
We will continue to use \eqref{eq:PeriodCellLaplaceStrongForm} as shorthand for \eqref{eq:PeriodCellLaplaceWeakForm} to save on notational clutter and maintain readability, however naturally the steps of our derivation of \eqref{eq:QGFullSystem} from \eqref{eq:PeriodCellLaplaceStrongForm} requires us to work with \eqref{eq:PeriodCellLaplaceWeakForm} directly.

\tstk{this is still written as an ``overview" section, the full proofs are not included. You should probably put them in Will, or have a final section of the appendix which is just the details of the proofs, and leave this one with some narrative.}

\subsection{Appendix subsection: Analysis of Gradients of Zero} \label{ssec:GradsOfZeroTheory}
Crucial to our understanding of the functions (and their gradients) in $\gradSobQM{\ddom}{\ddmes}$ will be understanding the corresponding ``gradients of zero".
\tstk{argument is based of Zhikov's stuff, and is in appendix in full}
We work towards obtaining this understanding progressively; first we look to understand the set $\gradZero{\ddom}{\lambda_{jk}}$ when the edge $I_{jk}$ is assumed to be parallel to the $x_1$-axis, then employ a rotation argument to understand $\gradZero{\ddom}{\lambda_{jk}}$ for a general edge that is at an angle to the $x_1$-axis.
Given that the singular measure $\ddmes$ is just the sum of the individual singular measures supporting each edge, we can then prove that elements of $\gradZero{\ddom}{\ddmes}$ display the same behaviour as elements of $\gradZero{\ddom}{\lambda_{jk}}$ when restricted to the edge $I_{jk}$.
This argument also provides us with a clear geometric interpretation for what a ``gradient of zero" is, and the edge-wise characterisation we obtain for elements of $\gradZero{\ddom}{\ddmes}$ will carry through into how we describe functions in $\gradSobQM{\ddom}{\ddmes}$.

We begin by describing $\gradZero{\ddom}{\lambda_{jk}}$ for when the edge $I_{jk}$ is parallel to the $x_1$-axis.
\begin{prop}[Gradients of Zero on a Segment Parallel to the $x_1$-axis] \label{prop:GradZeroParallelZhikov}
	Let $I$ be a segment in the $\bracs{x_1,x_2}$-plane parallel to the $x_1$-axis, and let $\lambda_I$ be the singular measure supported on $I$.
	Then 
	\begin{align*}
		\gradZero{\ddom}{\lambda_I} &= 
		\clbracs{
			\begin{pmatrix} 0 \\ f	\end{pmatrix}
			\ \vert \ f\in\ltwo{\ddom}{\lambda_I}
		}.
	\end{align*}
\end{prop}
\begin{proof}
	By combining proposition \ref{prop:GradZeroInvarientUnderQM} with a result in \cite{zhikov2000extension} \tstk{precise lemma/page citation}, this result follows, however we present a brief summary of the argument.
	Without loss of generality we assume $x_2=0$ on $I$.
	Additionally it suffices to show that the set on the right hand side includes all functions of the specified form when $f$ is smooth, as we can then apply a density argument. \newline
	
	So take some $f\in\smooth{\ddom}$, then the ``constant sequence" $\phi_n = \phi = x_2 f$ is such that
	\begin{align*}
		\phi_n\lconv{\ltwo{\ddom}{\lambda_I}}0, 
		&\quad \grad\phi_n\lconv{\ltwo{\ddom}{\lambda_I}^2} \begin{pmatrix} 0 \\ f \end{pmatrix}
		&\quad \toInfty{n}
	\end{align*}	 
	and so $\bracs{0,f}^\top\in\gradZero{\ddom}{\lambda_I}$. \newline
	
	We now prove that if $\bracs{f,0}^\top\in\gradZero{\ddom}{\lambda_I}$ then $f=0$.
	So suppose $\bracs{f,0}\in\gradZero{\ddom}{\lambda_I}$ and take an approximating sequence $\phi_n$ as in \eqref{eq:GradZeroDef}.
	Performing a change of variables via the map $r:\interval{I}\rightarrow I$ (as described for an edge $I_{jk}$ in convention \ref{ass:MeasTheoryProblemSetup}), and setting $\tilde{\phi}_n(t) := \phi_n\bracs{r(t)}$, we have
	\begin{align*}
		\tilde{\phi}_n\lconv{\ltwo{\interval{I}}{t}} 0, 
		&\quad \diff{\tilde{\phi}_n}{t}\lconv{\ltwo{\interval{I}}{t}} \tilde{f}
		&\quad \toInfty{n}.
	\end{align*}
	Hence $\tilde{f}$ is the distributional derivative (in the $\gradSob{\interval{I}}{t}$ sense) of the zero function, so we can conclude that $\tilde{f} = 0$, and thus $f = 0$, as we sought.
\end{proof}

Proposition \ref{prop:GradZeroParallelZhikov} provides the following interpretation for ``gradients of zero".
The measure $\lambda_I$ however can only ``see" along the segment $I$, as this is it's entire support.
As such $\lambda_I$ can only see the change in a function in the direction along the segment $I$, hence we find that $\gradZero{\ddom}{\lambda_I}$ consists of all the components of gradients that are directed perpendicular to $I$.
The following proposition reinforces this interpretation, although the argument is simply to invoke the result of proposition \ref{prop:GradZeroParallelZhikov} after applying the obvious rotation.
\begin{prop}[Rotation of Edge Gradients of Zero] \label{prop:RotationOfEdgeGradients}
	Consider the case when $\graph$ consists of a single edge (or segment) $I\subset\ddom$ with orthogonal co-ordinate system $y=\bracs{y_1,y_2}$, with $y_1$ parallel to $I$.
	Let $R$ be the orthogonal change of co-ordinates $x=Ry$ with $x=\bracs{x_1,x_2}$ the orthogonal co-ordinate system along the axes.
	Then
	\begin{align*}
		\gradZero{\ddom}{\lambda_I} 
		&= \clbracs{ R^{\top} \begin{pmatrix} 0 \\ f_2 \end{pmatrix} \ \vert \ f_2\in\ltwo{\ddom}{\lambda_I} }.
	\end{align*}
\end{prop}
With these two results, there is the following corollary which further reinforces the interpretation of $\gradZero{\ddom}{\lambda_I}$ given earlier.
\begin{cory} \label{cory:Grad0SingleEdge}
	Assume the hypothesis of proposition \ref{prop:RotationOfEdgeGradients}, and denote by $e_I$ the unit vector parallel to the segment $I$.
	Then
	\begin{align*}
		\gradZero{\ddom}{\lambda_I} &= \clbracs{z\in\ltwo{\ddom}{\lambda_I} \ \vert \ z\vert_{I}\cdot e_I = 0}.
	\end{align*}
\end{cory}

Using our understanding of gradients of zero on single segments, we can build up an understanding of gradients of zero on embedded graphs consisting of multiple edges.
This turns out to be the following characterisation; where each function in $\gradZero{\ddom}{\ddmes}$ behaves as a function in $\gradZero{\ddom}{\lambda_{jk}}$ when restricted to the edge $I_{jk}$.
\begin{prop} \label{prop:GradZeroGraph}
	Given convention \ref{ass:MeasTheoryProblemSetup}, we have that
	\begin{align*}
		\gradZero{\ddom}{\ddmes} &= \clbracs{g\in\ltwo{\ddom}{\ddmes}^2 \ \vert \ g\vert_{I_{jk}}\cdot e_{jk}=0 \ \forall I_{jk}\in \edgeSet} \\
		&= \clbracs{g\in\ltwo{\ddom}{\ddmes}^2 \ \vert \ g\in\gradZero{\ddom}{\lambda_{jk}} \ \forall I_{jk}\in \edgeSet}. \labelthis\label{eq:GradZeroSetRHS}
	\end{align*}
\end{prop}
\begin{proof}
	For the full proof, see appendix \tstk{appendix reference}.
	Showing that elements of $\gradZero{\ddom}{\ddmes}$ are contained in the set on the right-hand-side of \eqref{eq:GradZeroSetRHS} is straightforward due to the definition of $\ddmes$ and the fact that
	\begin{align} \label{eq:GraphMeasNormEdgeBreakdown}
		\norm{\cdot}_{\ltwo{\ddom}{\ddmes}}^2 = \sum_{j\con k}\norm{\cdot}_{\ltwo{\ddom}{\lambda_{jk}}}^2,
	\end{align}
	so if one has convergence in the $\ltwo{\ddom}{\ddmes}$-norm then we have convergence in each $\ltwo{\ddom}{\lambda_{jk}}$-norm.
	The opposite set inclusion involves a number of technical steps, and the full argument is given in the appendix.
	The gist of the argument involves showing that if $g$ is a member of the set on the right-had-side of \eqref{eq:GradZeroSetRHS}, then we can demonstrate that $g_{jk}\in\gradZero{\ddom}{\ddmes}$ (recall $g_{jk}$ is $g$ restricted to $I_{jk}$ then extended by zero to $\ddom$) given that $g_{jk}\in\gradZero{\ddom}{\lambda_{jk}}$.
	Then we can use the idea that $g = \sum_{j\con k} g_{jk}$ and that $\gradZero{\ddom}{\ddmes}$ is a closed, linear subspace to obtain membership of $g$ in $\gradZero{\ddom}{\ddmes}$; although in practice the expression for the sum is made more complex by the need to ensure good behaviour near the vertices of the graph.
\end{proof}

\subsection{Appendix subsection: The ``Non-Classical" Sobolev Spaces} \label{ssec:SobSpacesTheory}
Establishing an understanding of $\gradZero{\ddom}{\ddmes}$  affords us greater insight into the tangential gradient $\tgrad_\ddmes u$ of functions $u\in\gradSobQM{\ddom}{\ddmes}$.
Given that we know that $\tgrad_\ddmes u \perp \gradZero{\ddom}{\ddmes}$, and we have an edge-wise characterisation of $\gradZero{\ddom}{\ddmes}$, it will not be surprising to learn that we also obtain an edge-wise ``form" for the tangential gradient, as given in the following proposition.
\begin{prop} \label{prop:GraphTangGrad}
	Assume convention \ref{ass:MeasTheoryProblemSetup}.
	For each $I_{jk}\in \edgeSet$ write $\gradSob{\interval{I_{jk}}}{t}$ for the (``classical") Sobolev space on the interval $\interval{I_{jk}}$ with respect to the Lebesgue measure, and let $\tilde{u}_{jk} = u_{jk} \circ r_{jk}$.
	Then for $u\in\gradSobQM{\ddom}{\ddmes}$ we have that $\tilde{u}_{jk}\in\gradSob{\interval{I_{jk}}}{t}$ for each $I_{jk}\in \edgeSet$, and that
	\begin{align*}
		\bracs{ \tgrad_\ddmes u }_{jk} 
		&= R_{jk}^\top \begin{pmatrix} u_{jk}' + i\bracs{R_{jk}\qm}_1 u_{jk} \\ 0	\end{pmatrix}
	\end{align*}
	where $u_{jk}' = \bracs{ \tilde{u}_{jk}' } \circ r_{jk}^{-1}$.
\end{prop}
Note that the prime notation on $u_{jk}'$ does \emph{not} imply the existence of any kind of ``classical" derivative for $u_{jk}$ or $u$, it is just a helpful piece of notation to remind us that $u_{jk}$ does have some regularity after composition with $r_{jk}$.
\begin{proof}
	The proof proceeds in much the same way as how we sought to understand elements of $\gradZero{\ddom}{\ddmes}$, and full details can be found in the appendix.
	Any tangential gradient must be orthogonal to elements $g_{jk}\in\gradZero{\ddom}{\ddmes}$ where $g\in\gradZero{\ddom}{\lambda_{jk}}$, and this must hold for each edge $I_{jk}$.
	The plan is again to first consider an edge aligned parallel to the $x_1$-axis, then apply a rotation before appealing to the edge-wise decomposition of our measure. \newline
	
	As just mentioned, first consider an edge $I_{jk}$ parallel to the $x_1$-axis. 
	Let $\tgrad_\ddmes u = \bracs{v_1, v_2}^\top$ denote the components of $\tgrad_\ddmes u$; we can see that $v_2=0$ immediately due to the result of proposition \ref{prop:GradZeroGraph} and the requirement that $\tgrad_\ddmes u$ be orthogonal $g_{jk}$ for every member of $g\in\gradZero{\ddom}{\lambda_{jk}}$.
	This leaves the form of $v_1$ to be determined.
	Since $u\in\gradSobQM{\ddom}{\ddmes}$ there existences a sequence of smooth functions $\phi_n$ which converge to $u$ in $\ltwo{\ddom}{\ddmes}$ and whose gradients $\tgrad\phi_n$ converge to $\tgrad_\ddmes u$.
	Clearly any such sequence also converges to $u_{jk}$ in $\ltwo{\ddom}{\lambda_{jk}}$ as well (see \eqref{eq:GraphMeasNormEdgeBreakdown}) and $\partial_1\phi_n$ converges to $v_1\vert_{jk}$ in $\ltwo{\ddom}{\lambda_{jk}}$.
	Considering the composition $\tilde{\phi}_n = \phi_n \circ r_{jk}$ we find that
	\begin{align*}
		\tilde{\phi}_n \lconv{\ltwo{\interval{I_{jk}}}{t}} \tilde{u}_{jk},
		&\quad \diff{\tilde{\phi}_n}{t} \lconv{\ltwo{\interval{I_{jk}}}{t}} v_1\vert_{jk} - i\qm_1 \tilde{u}_{jk},
	\end{align*}
	from which we can deduce that $\tilde{u}_{jk}' = v_1 - i\qm_1\tilde{u}$, and hence obtain the result for an edge parallel to the $x_1$-axis ($R_{jk}$ being the identity).
	Any edges that are not parallel to the $x_1$-axis can now be rotated under $R_{jk}$ to bring them into an appropriate framework, the argument repeated and then unpacked to obtain the quoted form on each edge.
	Then since this holds for each edge in any graph, the result of in the proposition is obtained.
\end{proof}

Whilst proposition \ref{prop:GraphTangGrad} arrives at an expected conclusion (given proposition \ref{prop:GradZeroGraph}) for the form of the tangential gradient, $\gradSobQM{\ddom}{\ddmes}$ has some additional structure that is not obvious from this study.
In particular the behaviour of functions near the vertices of $\graph$ has been ignored up until this point, due to the fact that it does not warrant investigation when dealing with gradients of zero and hence the tangential gradients.
It is also not unreasonable to expect some special behaviour of the functions $u\in\gradSobQM{\ddom}{\ddmes}$ at the vertices, otherwise there will be no resemblance of the connectivity of $\graph$ in our function space.
We can deduce that functions $u\in\gradSobQM{\ddom}{\ddmes}$ actually possess continuity at the vertices $v_j\in \vertSet$ of $\graph$ (for any $\qm$), as is proven in \tstk{zhikov ref, with page/thm number} for when $\qm=0$:
\begin{theorem} \label{thm:CharOfGradSob}
	Assume convention \ref{ass:MeasTheoryProblemSetup}.
	Then we have that
	\begin{align*}
		u\in\gradSobQM{\ddom}{\ddmes} \quad\Leftrightarrow\quad 
		& (i) u\in\gradSobQM{\ddom}{\lambda_{jk}} \ \forall I_{jk}\in \edgeSet, \\
		& (ii) u \text{ is continuous at each } v_j\in \vertSet.
	\end{align*}
\end{theorem}
For consistency with this work, we give a full proof in the appendix so that the interested reader has all the arguments associated with this work in one place.
A sketch of the key ideas is below;
\begin{proof}
	The right-directed ($\Rightarrow$) implication is essentially a result of \eqref{eq:GraphMeasNormEdgeBreakdown}, as (i) follows from this almost immediately.
	(ii) is then obtained by showing that any sequence $\phi_n$ of smooth functions approximating $u$ and $\tgrad_\ddmes u$ (as in the definition of $\gradSobQM{\ddom}{\ddmes}$) is actually Cauchy in the uniform norm.
	As such it must also converge to a continuous function by completeness of this norm, and the limit must be $u_{jk}$.
	In particular it must also converge uniformly on the ``junction" surrounding each vertex $v_j$, and thus $u$ must be continuous at $v_j$ in particular. \newline
	
	The reverse implication is essentially a repeat of the argument for extending gradients of zero from one edge to the whole graph, except now we are doing similar steps for tangential gradients instead.
	Take smooth sequences approximating each $u_{jk}\in\gradSobQM{\ddom}{\lambda_{jk}}$, and sum them in an appropriate way to obtain convergence in $\ltwo{\ddom}{\ddmes}^2$ from the individual convergences in $\ltwo{\ddom}{\lambda_{jk}}^2$.
	Continuity at each vertex is required to control the behaviour of the sequence that is constructed near the vertices - namely we need to ensure there is a small ball around each vertex where the value of each $u_{jk} - u\bracs{v_j}$ is uniformly bounded across those edges $j\con k$.
\end{proof}

\tstk{another subsection for the derivation of the QG problem from the measure-theoretic one. Possibly merge \ref{ssec:SobSpacesTheory} with this new section too.}