\section{Introduction} \label{sec:Intro}

This section includes our literature review and motivation.
The purpose of the paper is to present the equations that were derived for the TFR/curl-curl system as what can be interpreted as an "effective" or "limit" problem for a fine/thin-structure material.
This is subject to future advances in the field that draw parallel to the scalar-case developments, of course.

\subsection{Lit Review} \label{ssec:LitReview}

\subsection{Problem Formulation} \label{ssec:OurSystem}
We shall now contextualise and outline the problem that we are interested in solving in this work.
The reader should see section \ref{sec:QuantumGraphs} and the appendix (sections \ref{ssec:MeasureTheory} through \ref{ssec:SobSpacesTheory}) for a precise description of the objects that are mentioned in the following discussion. \newline

The problem we consider in this work is motivated by the desire to study wave propagation on a periodic, singular structure in $\reals^2$, for applications in studying how the structure of a photonic crystal (or other appropriate waveguide material in non-electromagnetic contexts) impacts the waves that can propagate through the material (see section \ref{ssec:LitReview}).
We represent this singular structure by a graph $\graph$ in $\reals^2$ which we will assume to be periodic in the sense that there exists a region (period cell) $\ddom\subset\reals^2$ and vectors $p_1, p_2$ such that for any $z\in\integers$ the part of $\graph$ contained in the shifted regions $\ddom+p_1$ and $\ddom+p_2$ coincides with the part of $\graph$ contained in $\ddom$.
We call this part of $\graph$ in $\ddom$ as the \emph{period graph} of $\graph$, and denote it by $\graph_{\mathcal{P}}=\bracs{\vertSet, \edgeSet}$, where $\vertSet$ is a finite set of vertices and $\edgeSet$ a finite set of edges.
We define the Laplacian, $-\laplacian_{\ddmes}$, on our singular structure and consider the ``spectral problem" of $-\laplacian_{\ddmes}$ on $\graph$ in $\reals^2$
\begin{align} \label{eq:WholeSpaceLaplaceEqn}
	-\laplacian_{\ddmes} u &= \omega^2 u.
\end{align}
Here $\ddmes$ is a singular measure that supports $\graph$, the details of which are available in the appendix (section \ref{ssec:MeasureTheory}).
It is convenient to replace \eqref{eq:WholeSpaceLaplaceEqn} with a family of problems on $\ddom$ parameterised by $\qm$ (the ``quasi-momentum") which varies over the dual-cell of $\ddom$.
This provides us with a family of problems involving ``$\qm$-shifted" operators $\tgrad_{\mu}$,
\begin{align} \label{eq:PeriodCellLaplaceStrongForm}
	-\bracs{\tgrad_{\ddmes}}^2 u &= \omega^2 u, \quad\text{in } \ddom,
\end{align}
where $\tgrad_{\ddmes} = \grad_{\ddmes} + i\qm$ acts component-wise.
The link between \eqref{eq:WholeSpaceLaplaceEqn} and \eqref{eq:PeriodCellLaplaceStrongForm} is established by means of a version of the so-called Gelfand transform \tstk{pos reference}
\begin{align*}
	\hat{u}\bracs{x} &= \sum_{z\in\integers}u\bracs{x+zp_1+zp_2}e^{-i\qm\bracs{x+zp_1+zp_2}},
\end{align*}
\eqref{eq:PeriodCellLaplaceStrongForm} can be made more explicit by writing it in the following form:
\begin{subequations} \label{eq:QGFullSystem}
	\begin{align}
		-\bracs{\diff{}{t} + i\qm_{jk}}^2 \tilde{u}_{jk} = \omega^2 \tilde{u}_{jk}, &\quad t\in\interval{I_{jk}}, \quad \forall I_{jk}\in \edgeSet, \label{eq:QGEdgeODEs} \\
		u \text{ is continuous at each } &v_j \in \vertSet, \label{eq:QGVertCty}\\
		\sum_{j\con k}\bracs{\diff{}{t} + i\qm_{jk}}\tilde{u}_{jk}\bracs{v_j} &= \alpha_{j}u_{jk}\bracs{v_j} &\quad \forall v_j \in \vertSet. \label{eq:QGDerivCondition}
	\end{align}
\end{subequations}
Subscripts $jk$ in \eqref{eq:QGFullSystem} denote restrictions to the edges of $\graph_{\mathcal{P}}$.
Our main result is the analysis of the spectrum of \eqref{eq:WholeSpaceLaplaceEqn}, equivalently \eqref{eq:QGFullSystem}, in terms of the geometry of the structure and the coupling constants $\alpha_j\in\complex$.
The details of how \eqref{eq:QGFullSystem} can be obtained from \eqref{eq:PeriodCellLaplaceStrongForm} are included in the appendix (\tstk{section ref}).
\tstk{here we mention that we can derive the equations for curved edges which gives $\qm_{jk}$ some $t$-dependence, but choose not to consider this case, that would open a whole other can of worms that would require a different treatment.}