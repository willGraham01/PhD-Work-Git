\tstk{SECTION REDO FOR CORRECTED M-MATRIX SIGN, AND $\alpha<0$ THROUGHOUT. REDO FIGURES AND NUMERICS AS APPROPRIATE}

\section{Dispersion relations for concrete graph topologies} \label{sec:Examples}
Throughout this section, we are concerned with determining the spectrum of the problem \eqref{eq:WholeSpaceLaplaceEqn}, via the spectra of \eqref{eq:QGFullSystem}, on graphs with a period cell contained in the unit square, and thus for $\qm\in\left[-\pi,\pi\right)^2$.
The examples we consider in this section are chosen to highlight the methodology discussed in section \ref{sec:Discussion}, illustrate how the properties of the spectrum vary on the coupling constants at the vertices as well as the geometry of the singular-structure itself, and elaborate on the (non-)effect on the spectrum due to the choice of embedding a quantum graph. 

\subsection{One-Dimensional Loop} \label{ssec:Example1DLoop}
\tstk{edit this section to emphasise ``splitting edges" method of removing loops? Quite a nice section to demonstrate this in. Also fix notation - define the $H$-matrices, etc.}
We begin with the simplest example: a ``chain" of vertices that is periodic in one direction, to provide a conceptual demonstration of how one takes the period graph of a physical singular-structure and employs proposition \ref{prop:M-MatrixEntries} to construct the $M$-matrix and extract the spectral information.
We will also highlight a useful trick involving ``splitting" edges of a graph via the use of ``dummy vertices", which can aid analysis of the problem.

Consider the graph $\graph$ periodic in one direction (without loss of generality assumed parallel to the $x_1$-axis with $x_2=0$) in $\reals\times\sqbracs{0,1}$, with vertices $v_j = \bracs{j + \recip{2}, 0}^\top$ and edges $I_{j\bracs{j+1}}$ for $j\in\integers$.
Since $\graph$ is only periodic in the $x_1$-direction, the period cell lies in $S^1\times\sqbracs{0,1}$ rather than the usual 2D-torus for a graph periodic in two directions in $\reals^2$, and we only have a scalar quasi-momentum $\qm\in\left[-\pi,\pi\right)$ (one can set $\qm_2=0$ in \eqref{eq:QGFullSystem} when constructing the $\qm_{jk}$ to account for the lack of periodicity in $x_2$).
Assume the coupling constants at each vertex are identical, taking the value $\alpha<0$.
Then the quantum graph which corresponds to the period graph of $\graph$ is simply a single vertex $v$ with a looping edge $I$ of length 1, with quasi-momentum $\qm$ on $I$ and coupling constant $\alpha$ at $v$. 
One can apply proposition \ref{prop:M-MatrixEntries} at this point to compute the (one-dimensional) $M$-matrix, however for illustrative purposes (and consistency with the examples that follow), we elect to introduce an artificial vertex to break the loop $I$ into two edges, producing a new quantum graph $\graph_{\mathcal{P}}=\bracs{\vertSet, \edgeSet}$ with
\begin{align*}
	\vertSet = \clbracs{ v_1 , v_2 }, \quad \edgeSet = \clbracs{ I_{12}, I_{21} },
	&\qquad \abs{I_{12}} = \abs{I_{21}} = \recip{2}, \quad \qm_{12} = \qm_{21} = \qm, 
	&\qquad \alpha_1 = \alpha, \quad &\alpha_2 = 0.
\end{align*}
We illustrate the process of moving from $\graph$ to $\graph_{\mathcal{P}}$ in figure \ref{fig:Diagram_1DExample}.
Note that the artificial vertex having coupling constant equal to zero ensures that the incoming edge solutions and their derivatives match at $v_2$, ensuring that introducing $v_2$ does not alter the behaviour of solutions along the ``cut loop".
Studying the $M$-matrix of $\graph_{\mathcal{P}}$ to determine the eigenvalues $\omega^2$ is now equivalent to studying the spectrum of the original problem.
\begin{figure}[h!]
	\centering
	\begin{subfigure}[t]{0.3\textwidth}
		\centering
		\includegraphics[scale=2]{Diagram_1DLineGraph.pdf}
		\caption{\label{fig:Diagram_1DLineGraph} The graph $\graph$, periodic in one dimension, consisting of integer-spaced vertices.}
	\end{subfigure}
	~
	\begin{subfigure}[t]{0.3\textwidth}
		\centering
		\includegraphics[scale=2]{Diagram_1DLineQuantumGraph.pdf}
		\caption{\label{fig:Diagram_1DLineQuantumGraph} The corresponding quantum graph of the period cell of $\graph$, containing one (looping) edge of length 1}
	\end{subfigure}
	~
	\begin{subfigure}[t]{0.3\textwidth}
		\centering
		\includegraphics[scale=2]{Diagram_1DLineComputationGraph.pdf}	
		\caption{\label{fig:Diagram_1DLineComputationGraph} The equivalent graph $\graph_{\mathcal{P}}$ that we study. Note that the dummy vertex $v_2$ has coupling constant 0.}
	\end{subfigure}
	\caption{\label{fig:Diagram_1DExample} The graphs $\graph$ and $\graph_{\mathcal{P}}$.}
\end{figure} \newline

One can construct the $M$-matrix for $\graph_{\mathcal{P}}$ using proposition \ref{prop:M-MatrixEntries}, obtaining \tstk{signs are wrong!}
\begin{align*}
	M_{\qm}\bracs{\omega^2} &= 
	\begin{pmatrix}[1.75]
		2\omega\cot\dfrac{\omega}{2} & -2\omega\csc\dfrac{\omega}{2}\cos\dfrac{\qm}{2} \\
		-2\omega\csc\dfrac{\omega}{2}\cos\frac{\qm}{2} & 2\omega\cot\dfrac{\omega}{2}
	\end{pmatrix}.
\end{align*}
The matrix of coupling constants is $A = \mathrm{diag}\bracs{\alpha, 0}$, so we work with the matrix
\begin{align*}
	\tilde{M}_\qm\bracs{\omega^2} &= 
	\begin{pmatrix}[1.75]
		2\omega\cot\dfrac{\omega}{2} - \alpha\omega^2 & -2\omega\csc\dfrac{\omega}{2}\cos\dfrac{\qm}{2} \\
		-2\omega\csc\dfrac{\omega}{2}\cos\dfrac{\qm}{2} & 2\omega\cot\dfrac{\omega}{2}		
	\end{pmatrix},
\end{align*}
and solve either $\tilde{M}_\qm\bracs{\omega^2} w = 0$ for eigenpairs $\bracs{\omega^2, w}$, or $\det\tilde{M}_\qm \bracs{\omega^2} = 0$ for the eigenvalues $\omega^2$.
With $\tilde{M}_\qm$ being a relatively small matrix, taking the latter approach is feasible analytically, and one can obtain the following equation for $\omega$ for a given $\qm$:
\begin{align*}
	\cos\qm &= \cos\omega - \frac{\alpha\omega}{2}\sin\omega.
\end{align*}
We would then need to determine, for each $\qm\in\left[-\pi, \pi\right)$, the set of solutions for $\omega$ and then take the union to provide the spectrum of our original problem.
This procedure is similar to that in example \ref{ssec:ExampleCrossInPlane}, which we will study in greater depth, and as one might expect produces a qualitatively similar spectrum.

\subsection{``Decorated" Graph with Dependencies Arising from the Embedding} \label{ssec:EmbeddingDependentExample}
We next provide an explicit example to complement the discussion that concluded section \ref{ssec:MMatrix} as well as demonstrate that adding additional edges to a graph can be used to open gaps in the resulting spectrum.
Because of the need to distinguish between two objects that bear similar names, we will prefix the term ``quantum graph" with ``(embedded)" when we are discussing an quantum graph that has been equipped with an embedding, and will use the prefix ``(abstract)" when we wish to refer to a quantum graph that has yet to be assigned an embedding.

Consider the (embedded) graph $\graph$ periodic in one direction (without loss of generality assumed parallel to the $x_1$-axis with $x_2=0$) in $\reals\times\sqbracs{0,1}$, with vertices
\begin{align*}
	v_1^m = \bracs{m + \recip{2}, \recip{2}}^\top, 
	&\quad v_2^m = \bracs{m + \recip{2}\bracs{1+\cos\beta}, \recip{2}\bracs{1+\sin\beta}}^\top,
\end{align*}
for a fixed angle $\beta\in\bracs{0,\pi}$, and edges $I_{1}^{m} = \sqbracs{v_1^m, v_1^{m+1}}$ and $I_{12}^m=\sqbracs{v_1^m, v_2^m}$ for $m\in\integers$.
Place a single coupling constant with value $\alpha$ at each $v_1^m$, and assume that $v_2^m$ has coupling constant zero for each $m$.
Then the (abstract) quantum graph which corresponds to the (embedded) period graph of $\graph$ consists of two vertices and two edges, with one of the edges being a loop.
However it is important to note that the (abstract) quantum graph does not contain any reference to the angle $\beta$ at which the edge connecting the two vertices is orientated at --- this is entirely an artefact of our decision to embed this graph into $\reals\times\sqbracs{0,1}$.
We again elect to introduce an artificial vertex to obtain an equivalent (embedded) quantum graph $\graph_{\mathcal{P}}$ which describes the (embedded) period graph of $\graph$:
\begin{align*}
	&\graph_{\mathcal{P}} = \bracs{\vertSet, \edgeSet}, \quad
	\vertSet = \clbracs{ v_1, v_2, v_3 }, \quad
	\edgeSet = \clbracs{ I_{12}, I_{13}, I_{31} }, \\
	&v_1 = \bracs{\recip{2},\recip{2}}, \quad
	v_2 = \bracs{\recip{2}\bracs{1+\cos\beta}, \recip{2}\bracs{1+\sin\beta}}, \quad
	v_3 = \bracs{1, \recip{2}}.
\end{align*}
The process of moving from $\graph$ to $\graph_{\mathcal{P}}$ is illustrated in figure \ref{fig:Diagram_1DAngledEdgeExample}.
\begin{figure}[t]
	\centering
	\begin{subfigure}[t]{0.3\textwidth}
		\centering
		\includegraphics[scale=1.85]{Diagram_1DAngledEdge-Embedded.pdf}
		\caption{\label{fig:Diagram_1DAngledEdge-Embedded} The graph $\graph$, periodic in one dimension, consisting of integer-spaced vertices with an edge ``hanging" at an angle $\beta$.}
	\end{subfigure}
	~
	\begin{subfigure}[t]{0.3\textwidth}
		\centering
		\includegraphics[scale=1.85]{Diagram_1DAngledEdge-Quantum.pdf}
		\caption{\label{fig:Diagram_1DAngledEdge-Quantum} The corresponding quantum graph of the period cell of $\graph$, containing one (looping) edge of length 1 and another edge connecting the two vertices.}
	\end{subfigure}
	~
	\begin{subfigure}[t]{0.3\textwidth}
		\centering
		\includegraphics[scale=1.85]{Diagram_1DAngledEdge-Computation.pdf}	
		\caption{\label{fig:Diagram_1DAngledEdge-Computation} The equivalent graph $\graph_{\mathcal{P}}$ that we study. Note that the dummy vertex $v_3$ has coupling constant 0.}
	\end{subfigure}
	\caption{\label{fig:Diagram_1DAngledEdgeExample} The graphs $\graph$ and $\graph_{\mathcal{P}}$.}
\end{figure}

Since $\graph_{\mathcal{P}}$ is periodic in the $x_1$-direction with period 1, we have a scalar quasi-momentum $\qm\in\left[-\pi,\pi\right)$ with
\begin{align*}
	\qm_{13} = \qm_{31} = \qm, \quad \qm_{12} = \qm\cos\beta,
\end{align*}
and still have a coupling constant $\alpha$ at $v_1$. 
One can apply proposition \ref{prop:M-MatrixEntries} at this point to compute the $M$-matrix corresponding to the problem \eqref{eq:QGFullSystem} on $\graph_{\mathcal{P}}$, obtaining
\begin{align} \label{eq:EmbeddedGraphMMatrix}
	M_\qm\bracs{\omega^2} &=
	\begin{pmatrix}[1.75]
		-3\omega\cot\dfrac{\omega}{2} & \omega \exp\bracs{\dfrac{\rmi\qm\cos\beta}{2}}\csc\dfrac{\omega}{2} & 2\omega\csc\dfrac{\omega}{2}\cos\dfrac{\qm}{2} \\
		\omega \exp\bracs{-\dfrac{\rmi\qm\cos\beta}{2}}\csc\dfrac{\omega}{2} & -\omega\cot\dfrac{\omega}{2} & 0 \\
		2\omega\csc\dfrac{\omega}{2}\cos\frac{\qm}{2} & 0 & -2\omega\cot\dfrac{\omega}{2}
	\end{pmatrix}.
\end{align}
Here it can be seen that the angle $\beta$ has entered into the form of the $M$-matrix due to our choice of embedding, in particular the manner in which we chose to embed the vertex $v_2$ and edge $I_{12}$.
However we will see that the spectrum of \eqref{eq:QGFullSystem} on $\graph$ is independent of $\beta$, as one would expect from examining its (abstract) periodic quantum graph in the alternative manner described in section \ref{ssec:MMatrix}.

Setting $H^{(2)} = \omega\csc\omega$, and noting that the matrix of coupling constants is $A=\mathrm{diag}\bracs{\alpha, 0, 0}$, we solve \eqref{eq:QGDetSolveCondition} to obtain
\begin{align} \label{eq:EmbeddedGraphDetSolveCondition}
	0 = 2\cos\frac{\omega}{2}\sqbracs{ \cos\qm + \recip{2} - \frac{3}{2}\cos\omega - \frac{\alpha\omega}{2}\sin\omega }.
\end{align}
The factor in front of the brackets in \eqref{eq:EmbeddedGraphDetSolveCondition} attains the value 0 at $\omega=\bracs{2n-1}\pi, \ n\in\naturals$ (independently of the value of $\qm$), whilst the rest of the spectrum is those $\omega$ for which
\begin{align*}
	-\recip{2} \leq \frac{3}{2}\cos\omega + \frac{\alpha\omega}{2}\sin\omega & \leq \frac{3}{2}, \\
	\Leftrightarrow \abs{ 3\cos\omega + \alpha\omega\sin\omega - 1 } &\leq 2. 
\end{align*}
Poles of $H^{(2)}$ occur at $\omega= 2n\pi$, which also solve \eqref{eq:EmbeddedGraphDetSolveCondition} for $\qm=0$, however one can discern that these are not part of the spectrum upon examination of the eigenvalue branches.
The spectrum of \eqref{eq:WholeSpaceLaplaceEqn} therefore consists of those $\omega$ such that
\begin{align*}
	\abs{ 3\cos\omega + \alpha\omega\sin\omega - 1 } \leq 2,
	\quad\text{ and }\omega\neq 2n\pi, n\in\naturals \\
	\text{or} \ \omega = (2n-1)\pi, \quad n\in\naturals.
\end{align*}
As was expected, the resulting spectrum does not depend on $\beta$ despite the fact that the $M$-matrix for each operator on $\graph_{\mathcal{P}}$ does contain a $\beta$-dependency.
It is also due to the presence of the ``decoration" $v_2$ and edge $I_{12}$ that the spectrum differs from that of the example in section \ref{ssec:Example1DLoop}, although this difference will only depend on the length of the additional edge $I_{12}$ (and any coupling constant placed at this vertex).

\subsection{Cross in the Periodic Plane} \label{ssec:ExampleCrossInPlane}
Our final example is a two-dimensional graph whose period cell represents a lattice-like structure in $\reals^2$.
In this example we will use the $M$-matrix to determine an equation describing those $\omega$ which constitute the spectrum analytically, and complement this with some numerical computations to plot the resulting spectrum and density of states.

Consider the periodic graph defined as follows; for each $\bracs{n,m}\in\integers^2$ define
\begin{align*}
	& v_1^{\bracs{n,m}} = \bracs{\recip{2},0} + \bracs{n,m}, 
	v_2^{\bracs{n,m}} = \bracs{0,\recip{2}} + \bracs{n,m},
	v_3^{\bracs{n,m}} = \bracs{\recip{2},\recip{2}} + \bracs{n,m}, \\
	& I_{13}^{\bracs{n,m}} = \sqbracs{v_1^{\bracs{n,m}}, v_3^{\bracs{n,m}}},
	I_{23}^{\bracs{n,m}} = \sqbracs{v_2^{\bracs{n,m}}, v_3^{\bracs{n,m}}}, \\
	& I_{31}^{\bracs{n,m}} = \sqbracs{v_3^{\bracs{n,m}}, v_1^{\bracs{n+1,m}}},
	I_{32}^{\bracs{n,m}} = \sqbracs{v_3^{\bracs{n,m}}, v_2^{\bracs{n,m+1}}}.
\end{align*}
With 
\begin{align*}
	\vertSet^* = \clbracs{v_j^{\bracs{n,m}} \ \vert \ j\in\clbracs{1,2,3}, \bracs{n,m}\in\integers^2},
	\qquad \edgeSet^* = \clbracs{I_{jk}^{\bracs{n,m}} \ \vert \ j,k\in\clbracs{1,2,3}, \bracs{n,m}\in\integers^2},
\end{align*}
and setting coupling constants
\begin{align*}
	\alpha_3^{\bracs{n,m}} = \alpha \in\reals, 
	\qquad \alpha_j^{\bracs{n,m}} = 0, \quad j\in\clbracs{1,2,4,5},
\end{align*}
$\graph^* = \bracs{\vertSet^*,\edgeSet^*}$ is an embedded, periodic graph in $\reals^2$.
Its period graph occupies $\sqbracs{0,1}^2$ and can be visualised in figure \ref{fig:Diagram_TFRGraph}; consisting of 5 vertices and 4 edges.
\begin{figure}[b!]
	\centering
	\begin{subfigure}[t]{0.45\textwidth}
		\centering
		\includegraphics[height=4.5cm]{Diagram_TFRGraph.pdf}
		\caption{\label{fig:Diagram_TFRGraph} The period graph that we are considering. All edges have length $\recip{2}$, and the quasi-momentum on horizontal edges is $-\qm_1$ and on vertical edges is $-\qm_2$.}
	\end{subfigure}
	~
	\begin{subfigure}[t]{0.45\textwidth}
		\centering
		\includegraphics[height=4.5cm]{Diagram_TFRQuantumGraph.pdf}
		\caption{\label{fig:Diagram_TFRQuantumGraph} The quantum graph that appears in our example in section \ref{ssec:ExampleCrossInPlane}. Due to the identification of vertices on the boundary of the period graph, we are effectively dealing with a 3-vertex quantum graph.}
	\end{subfigure}
	\caption{\label{fig:5VertexCross} (\ref{fig:Diagram_TFRGraph}) The period cell of the graph $\graph^*$. (\ref{fig:Diagram_TFRQuantumGraph}) The equivalent quantum graph on which we pose \eqref{eq:QGFullSystem}, retaining the lengths $l_{jk}$ and appropriate $\qm_{jk}$.}
\end{figure}
We next associate vertices that lie on the boundary of the period cell, matching $v_2$ with $v_4$ and $v_1$ with $v_5$, to obtain the quantum graph $\graph=\bracs{\vertSet,\edgeSet}$ with $\vertSet=\clbracs{v_1,v_2,v_3}$, $\edgeSet=\clbracs{I_{13},I_{23},I_{31},I_{32}}$, and lengths
\begin{align*}
	l_{13} = l_{23} = l_{31} = l_{32} = \recip{2}.
\end{align*}
Given that all the edges of $\graph^*$ are parallel to the co-ordinate axes, it is easy to compute the values of $\qm_{jk}$ for each $I_{jk}\in E$ and a given $\qm=\bracs{\qm_1,\qm_2}\in[-\pi,\pi)^2$:
\begin{align*}
	\qm_{13} = \qm_{31} = -\qm_2, &\quad \qm_{23} = \qm_{32} = -\qm_1.
\end{align*}

To determine the spectrum of the problem \eqref{eq:QGFullSystem}, we write down the $M$-matrix via \ref{prop:M-MatrixEntries}, as follows: \tstk{wrong signs!}
\begin{align*}
	M_{\qm}\bracs{\omega^2} &=
	\begin{pmatrix}[1.75]
		2\omega\cot\bracs{\dfrac{\omega}{2}} & 0 & -2\omega\csc\bracs{\dfrac{\omega}{2}}\cos\bracs{\dfrac{\qm_2}{2}} \\
		0 & 2\omega\cot\bracs{\dfrac{\omega}{2}} & -2\omega\csc\bracs{\dfrac{\omega}{2}}\cos\bracs{\dfrac{\qm_1}{2}} \\
		-2\omega\csc\bracs{\dfrac{\omega}{2}}\cos\bracs{\dfrac{\qm_2}{2}} & -2\omega\csc\bracs{\dfrac{\omega}{2}}\cos\bracs{\dfrac{\qm_1}{2}} & 4\omega\cot\bracs{\dfrac{\omega}{2}}
	\end{pmatrix}.
\end{align*}
Now we employ the decomposition provided by corollary \ref{cory:M-MatrixEntriesNoPoles}, and set
\begin{align*}
	H_\qm^{(1)}\bracs{\omega^2} &=
	\begin{pmatrix}[1.75]
		2\cos\bracs{\dfrac{\omega}{2}} & 0 & -2\cos\bracs{\dfrac{\qm_2}{2}} \\
		0 & 2\cos\bracs{\dfrac{\omega}{2}} & -2\cos\bracs{\dfrac{\qm_1}{2}} \\
		-2\cos\bracs{\dfrac{\qm_2}{2}} & -2\cos\bracs{\dfrac{\qm_1}{2}} & 4\cos\bracs{\dfrac{\omega}{2}}
	\end{pmatrix}, \\
	H^{(2)}\bracs{\omega^2} &= \omega\csc\bracs{\frac{\omega}{2}}, \\
	\widetilde{M}_\qm\bracs{\omega^2} &= H_\qm^{(1)}\bracs{\omega^2} - \bracs{H^{(2)}\bracs{\omega^2}}^{-1}A,
\end{align*}
note that we have chosen to take the additional factor of $\omega$ out of $H_\qm^{(1)}$ and into $H^{(2)}$ due to $\graph$ having an odd number of vertices.
We now proceed analytically by solving \eqref{eq:QGDetSolveCondition}, where $A=\mathrm{diag}\bracs{0,0,\alpha}$ is the matrix of coupling constants for $\graph$, obtaining
\begin{align} \label{eq:ExampleThickVertexSolution}
	0 = 2\cos\bracs{\frac{\omega}{2}}
	\bracs{ \cos\bracs{\omega} - \frac{\alpha\omega}{4}\sin\bracs{\omega} - \cos\bracs{\frac{\qm_1+\qm_2}{2}}\cos\bracs{\frac{\qm_1-\qm_2}{2}} }.
\end{align}
For ease, we define
\begin{align*}
	\Xi\bracs{\omega} := \cos\omega - \frac{\alpha\omega}{4}\sin\omega.
\end{align*}
Clearly, $\omega=\bracs{2n-1}\pi$ is a solution to \eqref{eq:ExampleThickVertexSolution} for every $\qm\in\left[-\pi,\pi\right), n\in\naturals$ due to the factor of $2\cos\bracs{\frac{\omega}{2}}$.
This leaves the remaining solutions $\omega$ to be determined by
\begin{align*}
	\Xi\bracs{\omega} &= \cos\bracs{\frac{\qm_1+\qm_2}{2}}\cos\bracs{\frac{\qm_1-\qm_2}{2}},
\end{align*}
for each $\qm\in\left[-\pi,\pi\right)$.
However we have separated $\qm$ and $\omega$ in this expression, and the left-hand side of \eqref{eq:ExampleThickVertexSolution} attains every value in the interval $\sqbracs{-1,1}$ over the range $\qm\in[-\pi,\pi)^2$, we simply have to determine $\omega$ for which
\begin{align} \label{eq:ExampleThickVertexDispExpr}
	-1 \leq \Xi\bracs{\omega} \leq 1.
\end{align}
We now have to consider the possibility that some solutions that have been found coincide with poles of $H^{(2)}\bracs{\omega^2}$, which occur when $\omega = 2n\pi, n\in\naturals_0$.
Indeed we have that $\Xi\bracs{2n\pi}=1$, so $2n\pi$ is a solution to \eqref{eq:ExampleThickVertexSolution} when $\qm=0$.
At this point it is necessary for us to look at the eigenvalues $\widetilde{\beta}_j\bracs{\omega^2}$ (see section \ref{ssec:ApproachConsiderations}) of $\tilde{M}_0$.
Whilst we could numerically explore the relevant branch $\widetilde{\beta}_{j,0}$ as discussed in section \ref{ssec:ApproachConsiderations}, in this example we can compute them as
\begin{align*}
	\widetilde{\beta}_{1,0}\bracs{\omega^2} &= 2\cos\bracs{\frac{\omega}{2}}, \\
	\widetilde{\beta}_{2,0}\bracs{\omega^2} &= 3\cos\bracs{\frac{\omega}{2}} - \frac{\alpha\omega}{2}\sin\bracs{\frac{\omega}{2}} - \recip{2}\sqrt{ 32 + \bracs{ 2\cos\bracs{\frac{\omega}{2}} - \alpha\omega\sin\bracs{\frac{\omega}{2}} }^2 }, \\
	\widetilde{\beta}_{3,0}\bracs{\omega^2} &= 3\cos\bracs{\frac{\omega}{2}} - \frac{\alpha\omega}{2}\sin\bracs{\frac{\omega}{2}} + \recip{2}\sqrt{ 32 + \bracs{ 2\cos\bracs{\frac{\omega}{2}} - \alpha\omega\sin\bracs{\frac{\omega}{2}} }^2 }.
\end{align*}
The first branch is never zero at $\omega=2n\pi$, whilst
\begin{align*}
	\widetilde{\beta}_{2,0}\bracs{2n\pi} &= 3\bracs{\bracs{-1}^n - 1}, \\
	\widetilde{\beta}_{3,0}\bracs{2n\pi} &= 3\bracs{\bracs{-1}^n + 1},
\end{align*}
so when $n$ is odd, $\widetilde{\beta}_{3,0}\bracs{2n\pi} = 0$, and when $n$ is even, $\widetilde{\beta}_{2,0}\bracs{2n\pi} = 0$.
An examination of the limits $\lim_{\omega\rightarrow2n\pi}H^{(2)}\widetilde{\beta}_{2,0}$ and $\lim_{\omega\rightarrow2n\pi}H^{(2)}\widetilde{\beta}_{3,0}$ then implies that only $\omega=0$ is part of the spectrum of \eqref{eq:QGFullSystem}.
And so we conclude that the spectrum of \eqref{eq:QGFullSystem} on $\graph$ is composed of those $\omega$ such that:
\begin{subequations} \label{eq:CrossInPlane_ScalarDR}
	\begin{align}
		&\omega = \bracs{2n-1}\pi, \quad n\in\naturals, \\
		\text{or } &\omega = 0, \\
		\text{or } & -1 \leq \cos\omega - \frac{\alpha\omega}{4}\sin\omega \leq 1, \ \text{and} \ \omega\neq 2n\pi, n\in\naturals.
	\end{align}
\end{subequations}
One can visualise the curve $\Xi$ and corresponding spectrum in \ref{fig:CrossInPlane_ScalarDR}.
\begin{figure}[b]
	\centering
	\begin{subfigure}[t]{0.45\textwidth}
		\centering
		\includegraphics[scale=0.5]{CrossInPlane_ScalarDR_alpha-1-00.pdf}
		\caption{\label{fig:CrossInPlane_ScalarDR_alpha-1-00} The function $\Xi$ for $\alpha=-1$.}
	\end{subfigure}
	~
	\begin{subfigure}[t]{0.45\textwidth}
		\centering
		\includegraphics[scale=0.5]{CrossInPlane_ScalarDR_alpha-4-00.pdf}
		\caption{\label{fig:CrossInPlane_ScalarDR_alpha-4-00} The function $\Xi$ for $\alpha=-4$.}
	\end{subfigure}
	\caption{\label{fig:CrossInPlane_ScalarDR} The function $\Xi$ for $\alpha=-1$ and $\alpha=-4$. Red regions indicate those $\omega$ correspond to eigenvalues $\omega^2$, occurring when $\Xi$ takes values between $-1$ and $1$.}
\end{figure}

For any $\alpha<0$, the curve $\Xi$ has the same shape as shown in figure \ref{fig:CrossInPlane_ScalarDR}, which can be used to demonstrate that the spectrum comprises a union of pairwise disjoint ``spectral bands" $I_n\subset\sqbracs{(n-1)\pi, n\pi}$.
The band $I_n$ (for any $n\in\naturals$) has left-endpoint $(n-1)\pi$, although whether this endpoint is part of the spectrum depends on whether $n$ is odd or even in accordance with \eqref{eq:CrossInPlane_ScalarDR}, and a right-endpoint strictly less than $n\pi$.
It is worth remarking that this behaviour is reversed for $\alpha\geq 0$ and bands with $n>1$ (them having right-endpoint $n\pi$ and left-endpoint strictly greater than $(n-1)\pi$), and for $\alpha\geq2$ there is a gap between an isolated eigenvalue at $0$ and the beginning of the band $I_1$ - however $\alpha\geq0$ breaks the constraints of models for physical applications \tstk{should link back to where (if ANYWHERE) we mentioned thin-structure limit stuff?}.
This is consistent with similar applications of singular-structures as approximations to thin-material structures, like the work of \cite{cherednichenko2019time}, in which the natural physical constraints on the parameters in the problem prevent the opening of a band-gap at the bottom of the spectrum.
To round off the analysis, quantities such as the integrated density of states (IDoS) and density of states (DoS) can also be estimated from \eqref{eq:CrossInPlane_ScalarDR}, which demonstrating that the spectrum ``concentrates" at the center of each spectral band.
Examples are shown in figure \ref{fig:CrossInPlane_ScalarDoS}.
\begin{figure}[h]
	\begin{subfigure}[t]{0.45\textwidth}
		\centering
		\includegraphics[scale=0.5]{CrossInPlane_ScalarDoS_alpha-1-00.pdf}
		\caption{\label{fig:CrossInPlane_ScalarDoS_alpha-1-00} The (relative) integrated density of states (IDoS), density of states (DoS) and spectrum for the system with $\alpha=-1$.}
	\end{subfigure}
	~
	\begin{subfigure}[t]{0.45\textwidth}
		\centering
		\includegraphics[scale=0.5]{CrossInPlane_ScalarDoS_alpha-4-00.pdf}
		\caption{\label{fig:CrossInPlane_ScalarDoS_alpha-4-00} The (relative) integrated density of states (IDoS), density of states (DoS) and spectrum for the system with $\alpha=-4$.}
	\end{subfigure}	
	\caption{\label{fig:CrossInPlane_ScalarDoS} The (relative) IDoS, DoS, and spectrum for the graph topology in section \ref{ssec:ExampleCrossInPlane}.
	The relative IDoS at the value $x$ is defined as the IDoS at $x$ minus $\left\lfloor\frac{x}{\pi}\right\rfloor\bracs{2\pi}^2$.}
\end{figure}