\section{Closing Remarks} \label{sec:Conclusion}
To conclude, we summarise the work presented and discuss directions of possible future work, motivated by what we have seen.
Through the work of section \ref{sec:SystemDerivation} and the analysis in the appendices \ref{app:SingularMeasures}-\ref{app:SumMeasureAnalysis}, a link between the singular structure problem \eqref{eq:WholeSpaceLaplaceEqn} and the field of quantum graphs has been established.
This in turn has allowed us to solve the otherwise intractable problem \eqref{eq:WholeSpaceLaplaceEqn}, through the study of the problem \eqref{eq:QGFullSystem} and the use of the $M$-matrix.
Section \ref{sec:Discussion} provides an explicit expression for the $M$-matrix of \eqref{eq:QGFullSystem} on any graph topology, and outlines a number of considerations one might wish to take on board when using the $M$-matrix to analyse the spectrum.
We build on this discussion with the examples of section \ref{sec:Examples}, highlighting how the $M$-matrix provides an explicit description of the spectrum.
In what follows, we expand on two topics previously mentioned; open questions concerning generalised resolvents and the application to the theory in section \ref{ssec:MMatrix}, and the use of singular structure problems in advancing results on the connections between quantum graphs and thin structure problems.

As was highlighted in sections \ref{ssec:DiffOpsOnGraphs} and \ref{ssec:MMatrixConsequences}, the problem \eqref{eq:QGFullSystem} belongs to the class of problems with generalised resolvents.
In section \ref{ssec:MMatrix} we introduced the $M$-matrix in a more familiar setting (with no $\omega^2$-dependence in the vertex conditions) and remarked that the analysis of the spectrum of \eqref{eq:QGFullSystem} can be carried by replacing the matrix $B$ in section \ref{ssec:MMatrix} with $\omega^2 B$ in section \ref{ssec:MMatrixConsequences}.
Justification for doing so lies in observing that introducing explicit $\omega^2$-dependence will not affect the (structure of) the arguments in the supporting theory \tstk{refs to Ryzhov here?}, and consequentially is expected to give rise to the aforementioned alteration to $B$.
However, a formal argument to justify methodology based on the $M$-matrix and boundary triples (in the context of generalised resolvents) has not been carried out in the literature.
As such it remains open to investigation, but would closely follow and resemble arguments that are already available (and form the basis for the theory presented in section \ref{ssec:MMatrix}).
This observation (and justification) is made in other works that analyse problems with generalised resolvents via use of the theory of boundary triples and the $M$-matrix --- see for example \cite[page 1846]{cherednichenko2018effective} concerning the results of \tstk{The paper \cite[page 1846]{cherednichenko2018effective} specifically refers to those arguments found in Ryzhov, V.: Weyl-Titchmarsh function of an abstract boundary value problem, operator colligations, and linear systems with boundary control. Complex Anal. Oper. Theory 3(1), 289–322 (2009)}.
If one has reservations about this ``gap" in the theory, an alternative to analysing a problem with generalised resolvents directly is explored in \cite[Section 6]{cherednichenko2017norm}.
One could look to transform a problem with $\omega^2$-dependent $\delta$-type vertex condition (like \eqref{eq:QGFullSystem}) into a problem with $\omega^2$-independent $\delta'$-type vertex conditions.
This comes at the cost of having to determine the appropriate (unitary) transform to apply to \eqref{eq:QGFullSystem}; but the theory of section \ref{ssec:MMatrix} would apply to the transformed problem, could be used to analyse the spectrum, and then the inverse transform applied if desired.

Our final thoughts concern the connection between thin structures, singular structures, and quantum graphs mentioned in section \ref{ssec:PhysMot}.
The operator in \eqref{eq:WholeSpaceLaplaceEqn} is motivated by those in \eqref{eq:NonDimensionalWaveEqn} --- all that has been done is replacing the derivatives with their (measure theoretic) singular structure equivalents.
Additionally, the derivation of \eqref{eq:QGFullSystem} (section \ref{sec:SystemDerivation}) and the supporting analysis (appendices \ref{app:MeasureTheory}-\ref{app:SumMeasureAnalysis}) provides a route from our singular structure problem to a quantum graph.
This has provided us with a means of ``translating" the 2D gradient operator in \eqref{eq:NonDimensionalWaveEqn} into its 1D analogue in the resulting quantum graph problem.
Importantly, this method of analysis admits generalisation to other differential operators acting on vector-valued functions, such as curl or divergence in 3D.
In turn, this would provide candidates for  the ``approximate" quantum graphs of these differential operators on thin structures, opening routes for estabishing links between thin structures and quantum graphs in the spirit of the analysis in \cite{exner2005convergence, kuchment2001convergence}.
\tstk{kind of what to put a ``dibs" phrase here, like ``this work is currently being carried out for equations involving the curl operator" so that people don't try and do it before I write my thesis up!}