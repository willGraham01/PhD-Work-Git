\section{Closing Remarks} \label{sec:Conclusion}

Concluding remarks, directions of future work.
The usual outro sequence.


We want to elaborate on the following things:
\begin{itemize}
	\item Conjecture that $\det\mathfrak{M}_{\qm}$ has the form $\det\mathfrak{M}_{\qm} = \bracs{H^{(2)}}^{\abs{\vertSet}-1} F\bracs{\omega^2, \qm}$ for some function $F$. This has implications for the structure of the spectrum, being ``morally" equivalent to stating that the spectrum is absolutely continuous, or has singular-continuous part depending on the properties of $F$.
	This needs to be contextualised with what's currently available on the structure of periodic quantum graphs, so a good place to start looking is Kuchment \& Berlenko's book, section 4.3.
	\item Generalised resolvants and extensions - the formal arguments to justify our methodology involving the $M$-matrix and boundary triples (and operator extensions) have yet to be done in the case of generalised resolvants, IE with $\omega^2$ in the BCs. However, we expect that a direct approach, whilst novel, will follow essentially the same arguments as in work by Rhyzov. Alternatively, we can also appeal to the idea that we can always transform our problem with $\omega^2$-dependant BCs into a problem with $\omega^2$-independant BCs (although we'll obtain $\delta'$-type vertex conditions), use existing results (EKK say?) and then transform back, expecting the result of this process to be what we take for our method involving the $M$-matrix. Kirill has a couple of papers where this is touched upon, which might be a good place to start.
	\item Reiterate the thin-structure $\rightarrow$ singular-structure $\rightarrow$ quantum graphs links again, and hint towards vector equations etc.
\end{itemize}