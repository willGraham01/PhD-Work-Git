\section{General formula for the $M$-Matrix of a finite period graph} \label{sec:Discussion}
Having obtained the quantum graph problem \eqref{eq:QGFullSystem}, we now turn our attention to determination of the eigenvalues $\omega^2$.
The appeal of transitioning from \eqref{eq:WholeSpaceLaplaceEqn} to \eqref{eq:QGFullSystem} was so that we could make use of the $M$-matrix (introduced in section \ref{ssec:MMatrix}) as a tool for determining the spectrum.
In this section we contextualise the theory introduced in section \ref{sec:QuantumGraphs}, and in particular section \ref{ssec:MMatrix}, explaining how it is employed in our approach to determining the spectrum of \eqref{eq:WholeSpaceLaplaceEqn}.
In doing so, we will also provide a general formula for the $M$-matrix of a finite period graph on which \eqref{eq:QGFullSystem} is posed; and we will follow up on this in section \ref{sec:Examples} where we provide some explicit examples of quantum graph problems that can be solved by employing the $M$-matrix in the manner discussed below. \newline

\subsection{General formula for the $M$-Matrix} \label{ssec:MMatrixResult}
\begin{prop}[$M$-matrix entries] \label{prop:M-MatrixEntries}
	Let $\graph=\bracs{\vertSet,\edgeSet}$ be an embedded graph on which the problem \eqref{eq:QGFullSystem} is posed.
	Suppose that $\dmap u = e_k$ where $e_k$ is the $k$\textsuperscript{th} canonical unit vector in $\complex^{\abs{\vertSet}}$.
	Then the $j$\textsuperscript{th} entry of $\nmap u$, and hence the $jk$\textsuperscript{th} entry in the $M$-matrix, is given by
	\begin{align*}
		\bracs{\nmap u}_j &= 
		\begin{cases}
			\!\begin{aligned}
				&0,
			\end{aligned}			
			& j \not\con k, \\
			\!\begin{aligned}
				&-\sum_{j\conLeft k} \omega e^{i\qm_{jk}l_{jk}} \csc\bracs{l_{jk}\omega} 
				\\ &\quad - \sum_{j\conRight k} \omega e^{-i\qm_{kj}l_{kj}} \csc\bracs{l_{kj}\omega},
			\end{aligned}
			& j\neq k, \ j\con k, \\
			\!\begin{aligned}
				&\sum_{j\con l} \omega\cot\bracs{l_{jl}\omega}
				\\ &\quad + 2\omega\sum_{j\conLeft j} \cot\bracs{l_{jj}\omega} - \cos\bracs{\qm_{jj}l_{jj}}\csc\bracs{l_{jj}\omega},
			\end{aligned}
			& j=k.
		\end{cases}
	\end{align*}
\end{prop}
Note the choice of $j\conLeft j$ in the contributions from loops is simply a convention, $j\conRight j$ is equivalent here.
Also recall the convention for summing over $j\con k$:
\begin{align*}
	\sum_{j\con k} \omega\cot\bracs{l_{jk}\omega} &= \sum_{j\conLeft k} \omega\cot\bracs{l_{jk}\omega}	+ \sum_{j\conRight k} \omega\cot\bracs{l_{kj}\omega}
\end{align*}
\begin{proof}
	The proof is an explicit computation, and follows the same idea as in \cite{ershova2014isospectrality} with adjustments for the fact that there are $\qm$ terms floating around. \newline
	
	We first determine the form of the edge solution $u_{jk}$ via solution of \eqref{eq:QGEdgeODEs} up to two complex constants per edge,
	\begin{align} \label{eq:EdgeEqnGeneralSolution}
		u_{jk} &= e^{-i\qm_{jk}t}\bracs{ C_{+}^{(jk)}e^{-i\omega t} + C_{-}^{(jk)}e^{i\omega t} }.
	\end{align}
	Since the $M$-matrix maps $\complex^{\abs{\vertSet}}\rightarrow\complex^{\abs{\vertSet}}$, it is sufficient to determine the $M$-matrix's action on the canonical basis of $\complex^{\abs{\vertSet}}$.
	So for each fixed $k\in\clbracs{1,...,\abs{\vertSet}}$ we set $\dmap u = e_k$.
	This provides us with sufficient Dirichlet data to solve \eqref{eq:QGEdgeODEs} on each edge and eliminate the constants $C_{+}^{(jk)}$, $C_{-}^{(jk)}$ in \eqref{eq:EdgeEqnGeneralSolution}, obtaining
	\begin{align*}
		j\not\con k &\implies
		\begin{cases}
			u_{jk}(t) = 0, \\
			u_{kj}(t) = 0,
		\end{cases} \\
		j\neq k, \ j\con k &\implies
		\begin{cases}
			u_{jk}(t) = e^{-i\qm_{jk}\bracs{t-l_{jk}}}\csc\bracs{\omega l_{jk}}\sin\bracs{\omega t}, \\
			u_{kj}(t) = -e^{-i\qm_{kj}\bracs{t-l_{kj}}}\csc\bracs{\omega l_{kj}}\sin\bracs{\omega t},
		\end{cases} \\
		j = k &\implies 
		\begin{cases}
			u_{jj}(t) = e^{-i\qm_{jj}t} \bracs{ e^{-i\omega t} + \sqbracs{e^{-i\qm_{jj}l_{jj}}-e^{-i\omega l_{jj}}}\csc\bracs{\omega l_{jj}}\sin\bracs{\omega t}  },
		\end{cases}
	\end{align*}
	This in turn enables us to explicitly differentiate the expressions for $u_{jk}$, and read off the values of $\bracs{\diff{}{t}+i\qm_{jk}}u_{jk}$ at the vertices.
	In the case $j\not\con k$, we obviously get zero contribution from the edges $I_{jk}$ and $I_{kj}$.
	The case $j\neq k, \ j\con k$, yields contributions from the edges $I_{jk}$ and $I_{kj}$ as follows:
	\begin{align*}
		\bracs{\diff{}{t}+i\qm_{jk}}u_{jk}\bracs{v_j} &= -\omega e^{-i\qm_{jk}l_{jk}}\csc\bracs{\omega l_{jk}}, \\
		\bracs{\diff{}{t}+i\qm_{jk}}u_{jk}\bracs{v_k} &= \omega\cot\bracs{\omega l_{jk}}, \\
		\bracs{\diff{}{t}+i\qm_{kj}}u_{kj}\bracs{v_j} &= -\omega e^{-i\qm_{kj}l_{kj}}\csc\bracs{\omega l_{kj}}, \\
		\bracs{\diff{}{t}+i\qm_{kj}}u_{kj}\bracs{v_k} &= \omega\cot\bracs{\omega l_{kj}}.
	\end{align*}
	And finally, when considering the case $j=k$ the contribution to $\bracs{\nmap u}_j$ from loops $I_{jj}$ in the graph also requires us to compute
	\begin{align*}
		\lim_{t\rightarrow0}-\bracs{u_{jj}'+i\qm_{jj}u_{jj}}(t) + \lim_{t\rightarrow l_{jj}}\bracs{u_{jj}'+i\qm_{jj}u_{jj}} & (t) \\
		&= 2\omega\bracs{ \cot\bracs{\omega l_{jj}} - \cos\bracs{\qm_{jj}l_{jj}}\csc\bracs{\omega l_{jj}}}.	
	\end{align*}
	We then use the formula
	\begin{align*}
		\bracs{\nmap u}_j &= \sum_{j\con l} \bracs{\diff{}{t}+i\qm_{jl}}u_{jl}\bracs{v_j},
	\end{align*}
	and our previous computations to compute each entry $\bracs{\nmap u}_j$, and obtain the desired result.
\end{proof}

As mentioned in section \ref{ssec:MMatrix}, determination of the eigenvalues $\omega^2$ of a quantum graph problem can be done by determining those $\omega^2$ such that
\begin{align*}
	\det\bracs{ M\bracs{\omega} - A } = 0,
\end{align*}
where $A$ is the vector of vertex coupling constants and $M$ is the $M$-matrix of the quantum graph.
In the case of \eqref{eq:QGFullSystem}, specifically \eqref{eq:QGDerivCondition}, one will notice that there is a factor of $\omega^2$ multiplying the coupling constants $\alpha_j$ as they appear in the right-hand-side of the equation.
Furthermore, proposition \ref{prop:M-MatrixEntries} also demonstrates that the $M$-matrix (in the context of \eqref{eq:QGFullSystem}) can be thought of as a function of $\omega$ parametrised by $\qm$, and so we shall denote it by $M_{\qm}$ henceforth.
The dependence of $M_\qm$ on $\qm$ is due to our decision to specify our singular structure (comprising the domain which \eqref{eq:WholeSpaceLaplaceEqn} was posed) as an embedded, periodic metric graph and then apply the Gelfand transform - as touched on in section \ref{ssec:MMatrix}.
As such, to determine the spectrum of \eqref{eq:QGFullSystem} for each $\qm$, and thus the spectrum of \eqref{eq:WholeSpaceLaplaceEqn} it is necessary to solve
\begin{align} \label{eq:QGDetSolveCondition}
	\det\bracs{M_\qm\bracs{\omega} - \omega^2 A} &= 0,
\end{align}
for $\omega$, over each value of $\qm$.

\subsection{Consequences of Proposition \ref{prop:M-MatrixEntries}} \label{ssec:MMatrixConsequences}
Proposition \ref{prop:M-MatrixEntries} effectively reduces the potentially intractable problem \eqref{eq:WholeSpaceLaplaceEqn} to a more accessible (family of) matrix-eigenvalue problem(s) for $M_\qm$.
In this subsection, we will remark on some of the consequences of proposition \ref{prop:M-MatrixEntries} and discuss some of the ways that it can be exploited for use in examining physically-motivated graph topologies. \newline

The task of solving \eqref{eq:QGDetSolveCondition} for $\omega$ can be accomplished in two ways.
One can simply compute an expression for the determinant in terms of $\omega$, set this equal to zero, and determine those $\omega$ that satisfy the equation.
Of course the determinant also depends on $\qm$, and so it is still necessary to solve for $\omega$ for each $\qm$.
However the task can be simplified so long as \eqref{eq:QGDetSolveCondition} can have $\omega$ and $\qm$ separated, that is be written equivalently as
\begin{align*}
	F\bracs{\omega} &= G\bracs{\qm}
\end{align*}
for some (continuous) functions $F$ and $G$.
In this case, one can simply compute the range of $G$, and solve the inequality
\begin{align*}
	\min\clbracs{G(\qm)} \leq F\bracs{\omega} \leq \max\clbracs{G(\qm)},
\end{align*}
netting the spectrum of \eqref{eq:WholeSpaceLaplaceEqn} in one inequality.
This is an approach that we employ in our examples in section \ref{sec:Examples}.
The alternative approach is to realise \eqref{eq:QGDetSolveCondition} as a (generalised) eigenvalue problem,
\begin{align} \label{eq:QGGenEvalSolve}
	\bracs{ M_\qm\bracs{\omega}-\omega^2 A }w &= 0,
\end{align}
where one is solving for the eigenpairs $\bracs{\omega, w}$.
This formulation of \eqref{eq:QGDetSolveCondition} is much better suited to situations in which a numerical approach is to be used, as working with the determinant directly is prone to instabilities and errors.
However provided that one has an appropriate numerical scheme, \tstk{could \cite{guttel2017nonlinear} Guttel, Manchester paper here} a computer can compute (an approximation to) the spectrum of \eqref{eq:QGFullSystem} and thus, by varying $\qm$ over a suitable mesh, the spectrum of \eqref{eq:WholeSpaceLaplaceEqn}.
Of course, a numerical approach will never recover the full spectrum of \eqref{eq:WholeSpaceLaplaceEqn}, but is more appropriate when considering singular-structures as an approximation to physically-motivated thin-structures.
There are of course other complexities to consider if one wishes to solve \eqref{eq:QGDetSolveCondition} numerically, which we highlight later in section \ref{ssec:NumericalApproachConsiderations}. \newline

Given that we have two potential approaches to tackling \eqref{eq:QGDetSolveCondition}, the next question to consider is when one approach is more amiable than the other.
This is answered predominantly by noticing that proposition \ref{prop:M-MatrixEntries} tells us that the complexity of the entry of $\bracs{M_\qm}_{jk}$ depends on the number of edges between the vertices $v_j$ and $v_k$, whilst the diagonal entries have a complexity that depends on the degree of the vertex $v_j$.
Furthermore, the number of vertices in the graph determines the dimensions of $M_\qm$, and the sparsity of $M_\qm$ depends on the number of pairs of vertices that are not (directly) connected by an edge.
This leads to the colloquial rule that ``graphs with a small number of edges lend themselves to the first approach (via the determinant), whilst graphs with a large number of edges are easier to handle with approach two (as a generalised eigenvalue problem)".
Indeed, as the entries of $M_\qm$ grow more complex and $M_\qm$ becomes less sparse, computing the determinant in \eqref{eq:QGDetSolveCondition} becomes much harder analytically (and of course, even more susceptible to instabilities numerically).
Of course, even though it is the number of edges in the graph which predominantly dictates how ``nice" $M_\qm$ is to work with analytically, making a graph larger (that is, adding additional vertices and/or edges to an existing graph) will always make $M_\qm$ harder to work with.
Even for relatively simple geometries like those in section \ref{sec:Examples}, the computations one has to go through to obtain expressions for the eigenvalues $\omega^2$ are fairly involved.

\subsection{Considerations for a Numerical Approach to Solving \eqref{eq:QGDetSolveCondition}} \label{ssec:NumericalApproachConsiderations}
In this section we briefly discuss some considerations for solving \eqref{eq:QGDetSolveCondition} as a generalised eigenvalue problem \eqref{eq:QGGenEvalSolve} for a fixed $\qm$, and suggestions as to how one can then efficiently recover the spectrum of \eqref{eq:WholeSpaceLaplaceEqn}.
We take as a baseline that one has an appropriate numerical scheme for handling generalised eigenvalue problems to hand \tstk{Guttel \cite{guttel2017nonlinear} would also be good here}, and so do not delve into the details of how such an algorithm would operate.
Instead we highlight some potentially useful ideas for handling $M_\qm$ in these contexts, and invite further insight into the utility of these ideas from the reader.
It is of course worth mentioning that $M_\qm$ is Hermitian (as can be seen from proposition \ref{prop:M-MatrixEntries}) so any numerical scheme gains the benefits of $M_\qm$ having this property. \newline

Whilst proposition \ref{prop:M-MatrixEntries} provides an explicit form for the entries of the $M$-matrix, these entries aren't particularly friendly when attempting to work with $M_\qm$ numerically.
Each (non-zero) entry involves at least one reciprocal trigonometric function with poles in $\omega$, which need to be dealt with if $M_\qm$ is to be used in a numerical scheme that requires evaluation of $M_\qm\bracs{\omega}$.
This can be dealt with by pulling a scalar prefactor out of $M_\qm$ containing all the poles in the entries of $M_\qm$, as follows:
\begin{cory}[$M$-matrix entries without poles] \label{cory:M-MatrixEntriesNoPoles}
	Let $\tilde{M}_\qm\bracs{\omega}$ have entries defined by
	\begin{align*}
		\bracs{\tilde{M}_\qm}_{jk} &= 
		\begin{cases}
			\!\begin{aligned}
				&0,
			\end{aligned}			
			& j \not\con k, \\
			\!\begin{aligned}
				&-\sum_{j\conLeft k} \bracs{ \omega e^{i\qm_{jk}l_{jk}} \prod_{\substack{ l\con m \\ \bracs{l,m} \neq \bracs{j,k} }}\sin\bracs{l_{lm}\omega} }
				\\ &\quad - \sum_{j\conRight k} \bracs{ \omega e^{-i\qm_{kj}l_{kj}} \prod_{\substack{l\con m \\ \bracs{l,m} \neq \bracs{j,k} }}\sin\bracs{l_{ml}\omega} },
			\end{aligned}
			& j\neq k, \ j\con k, \\
			\!\begin{aligned}
				&\sum_{j\con l} \bracs{ \omega\cos\bracs{l_{jl}\omega}\prod_{\substack{ m\con n \\ \bracs{m,n}\neq\bracs{j,l} }}\sin\bracs{l_{mn}\omega} }
				\\ &\quad + 2\omega\sum_{j\conLeft j} \bracs{ \sqbracs{ \prod_{\substack{l\con m \\ \bracs{l,m}\neq\bracs{j,j} }}\sin\bracs{l_{lm}\omega} }\sqbracs{ \cos\bracs{l_{jj}\omega} - \cos\bracs{\qm_{jj}l_{jj}} } },
			\end{aligned}
			& j=k.
		\end{cases}
	\end{align*}
	And set
	\begin{align*}
		P_{M_\qm} &= \prod_{j\con k}\csc\bracs{l_{jk}\omega}.
	\end{align*}
	Then
	\begin{align*}
		M_\qm\bracs{\omega} &= P_{M_\qm}\tilde{M}_\qm\bracs{\omega}.
	\end{align*}
\end{cory}
The product notation should be understood analogously to the summation notation over $j\con k$ introduced in section \ref{sec:QuantumGraphs}.
Corollary \ref{cory:M-MatrixEntriesNoPoles} is just the result of pulling out the scalar factor $P_{M_\qm}$ from $M_\qm$.
Since $P_{M_\qm}$ contains all the poles of the entries of $M_\qm$, the resulting matrix $\tilde{M}_\qm$ has no poles, and even has it's $jk^{\mathrm{th}}$ entry bounded (for all $\omega$) by the number of (direct) connections $j\con k$.
This also means that one can then solve the generalised eigenvalue problem
\begin{align} \label{eq:QGGenEvalSolveNoPoles}
	\bracs{ \tilde{M}_\qm - \omega^2 P_{M_\qm}^{-1} A}w = 0
\end{align}
for $\omega$ instead of \eqref{eq:QGGenEvalSolve}.
Writing
\begin{align*}
	P_{M_\qm}^{-1} &= \prod_{j\con k}\sin\bracs{l_{jk}\omega}
\end{align*}
avoids introducing poles to the mix, and it is also worth noting that $\abs{P_{M_\qm}^{-1}}\leq 1$.
Then to ensure that the result obtained is consistent with \eqref{eq:QGGenEvalSolve}, one would have to exclude those $\omega$ found which correspond to a pole of $P_{M_\qm}$.
However the set of poles is easily computed from the set of all lengths $l_{jk}$, with poles occurring at those
\begin{align*}
	\omega &= \frac{n\pi}{l_{jk}}, \quad j\conLeft k, \ n\in\naturals_{0}.
\end{align*}
As such, one can solve the generalised eigenvalue problem involving the bounded matrix $\tilde{M}_\qm$, and then exclude any $\omega$ that correspond to one of the (known) poles of $P_{M_\qm}$. \newline

The other immediate discussion point to consider when taking a numerical approach is how to efficiently solve \eqref{eq:QGGenEvalSolve} for $\omega$ for each $\qm$ (or for each value in a suitable mesh of $\qm$ values).
This must be done to obtain the spectrum of \eqref{eq:WholeSpaceLaplaceEqn}, since our Gelfand transform requires us to take the union of the spectra of \eqref{eq:QGFullSystem} over $\qm$.
One suggestion that comes to mind is to investigate whether the solutions $\omega$ to \eqref{eq:QGGenEvalSolve} (or \eqref{eq:QGGenEvalSolveNoPoles}) are stable with respect to changes in the $\qm$, that is whether one can obtain a result like (one of) the following:
\begin{itemize}
	\item For every $\eps>0$, there exists some $\delta>0$ such that if $\norm{\qm^{(1)}-\qm^{(2)}}<\delta$, then the solutions $\omega^{(1)}$ and $\omega^{(2)}$ to \eqref{eq:QGGenEvalSolve} (with $\qm$ taking the values $\qm^{(1)}$ and $\qm^{(2)}$ respectively) are such that $\norm{\omega^{(1)}-\omega^{(2)}}<\eps$.
	\item There exists some $c\geq 0$ such that; if $\omega^{(1)}$ and $\omega^{(2)}$ are solutions to \eqref{eq:QGGenEvalSolve} with $\qm$ taking the values $\qm^{(1)}$ and $\qm^{(2)}$ respectively, then
	\begin{align*}
		\frac{ \norm{\qm^{(1)}-\qm^{(2)}} }{ \norm{\qm^{(1)}} } \leq c \frac{ \norm{\omega^{(1)}-\omega^{(2)}} }{ \norm{\omega^{(1)}} }
	\end{align*}
\end{itemize}
Having either of these confirmed would mean that one could adopt a more informed approach to solving \eqref{eq:QGGenEvalSolve} on a $\qm$-mesh: after solving at $\qm^{(1)}$ and obtaining a solution $\omega^{(1)}$, and knowing the next mesh-point $\qm^{(2)}$ to be solved at, the value $\omega^{(1)}$ would provide a good starting guess to an algorithm determining the solution of \eqref{eq:QGGenEvalSolve} at $\qm^{(2)}$.
By providing an informed ``initial guess" in this way, a numerical method that relies on an accurate guess should require less time to converge to a solution and thus require less time to reconstruct the spectrum of \eqref{eq:WholeSpaceLaplaceEqn}.
Of course this suggestion also assumes that one is using a numerical method that relies on stepping through a $\qm$-mesh in the first place, and if other methods that do not rely on such an approach can be used, they would not benefit in the same way as suggested here.
\tstk{Final sentence here like ``the authors would be interested to hear of any known theory or developments in this area"? Or just leave the point as it is and start the next section?}