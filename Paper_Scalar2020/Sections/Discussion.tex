\section{General formula for the $M$-Matrix of a finite period graph} \label{sec:Discussion}
In this section we want to discuss some of the important applications of our work.
We should touch on:
\begin{itemize}
	\item ``decorations" - adding a graph to a periodic 1D structure by attaching at a vertex (or better: taking a graph, breaking an edge in two and identifying either sides of the ``break" as the edge of the period cell).
	This ties into the setups of EG \cite{cherednichenko2019time} (might be wrong Kirill paper, all the bibTex refs have similar names, check the PDFs and ensure citation points to the right work)
	\item M-matrix work to reduce the QG problem into a nicer, more accessible state.
	Potentially mention we have code to solve this, or do not depending on data plans!
	Also, avoid explicit computation and discussion of numerical ideas unless we want to include them - the examples section which follows is where to put these things.
	\item Also might want to discuss our thoughts about graph symmetry affecting the M-matrix properties (real), stability results, etc, which we can formulate into (at least) conjectures.
\end{itemize}

\tstk{prop \ref{prop:M-MatrixEntries} needs contextual text around it}
\begin{prop}[$M$-matrix entries] \label{prop:M-MatrixEntries}
	Let $\graph=\bracs{\vertSet,\edgeSet}$ be an embedded graph on which the problem \eqref{eq:QGFullSystem} is posed.
	Suppose that $\dmap u = e_k$ where $e_k$ is the $k$\textsuperscript{th} canonical unit vector in $\complex^{\abs{\vertSet}}$.
	Then the $j$\textsuperscript{th} entry of $\nmap u$, and hence the $jk$\textsuperscript{th} entry in the $M$-matrix, is given by
	\begin{align*}
		\bracs{\nmap u}_j &= 
		\begin{cases}
			\!\begin{aligned}
				&0,
			\end{aligned}			
			& j \not\con k, \\
			\!\begin{aligned}
				&-\sum_{j\conLeft k} \omega e^{i\qm_{jk}l_{jk}} \csc\bracs{l_{jk}\omega} 
				\\ &\quad - \sum_{j\conRight k} \omega e^{-i\qm_{kj}l_{kj}} \csc\bracs{l_{kj}\omega},
			\end{aligned}
			& j\neq k, \ j\con k, \\
			\!\begin{aligned}
				&\sum_{j\con l} \omega\cot\bracs{l_{jl}\omega}
				\\ &\quad + 2\omega\sum_{j\conLeft j} \cot\bracs{l_{jj}\omega} - \cos\bracs{\qm_{jj}l_{jj}}\csc\bracs{l_{jj}\omega},
			\end{aligned}
			& j=k.
		\end{cases}
	\end{align*}
	Note the choice of $j\conLeft j$ in the contributions from loops is simply a convention, $j\conRight j$ is equivalent here.
	Also recall the convention for summing over $j\con k$:
	\begin{align*}
		\sum_{j\con k} \omega^2\cot\bracs{l_{jk}\omega} &= \sum_{j\conLeft k} \omega\cot\bracs{l_{jk}\omega}	+ \sum_{j\conRight k} \omega\cot\bracs{l_{kj}\omega}
	\end{align*}
\end{prop}
\begin{proof}
	The proof is an explicit computation, and follows the same idea as in \cite{ershova2014isospectrality} with adjustments for the fact that there are $\qm$ terms floating around. \newline
	
	We first determine the form of the edge solution $u_{jk}$ via solution of \eqref{eq:QGEdgeODEs} up to two complex constants per edge,
	\begin{align} \label{eq:EdgeEqnGeneralSolution}
		u_{jk} &= e^{-i\qm_{jk}t}\bracs{ C_{+}^{(jk)}e^{-i\omega t} + C_{-}^{(jk)}e^{i\omega t} }.
	\end{align}
	Since the $M$-matrix maps $\complex^{\abs{\vertSet}}\rightarrow\complex^{\abs{\vertSet}}$, it is sufficient to determine the $M$-matrix's action on the canonical basis of $\complex^{\abs{\vertSet}}$.
	So for each fixed $k\in\clbracs{1,...,\abs{\vertSet}}$ we set $\dmap u = e_k$.
	This provides us with sufficient Dirichlet data to solve \eqref{eq:QGEdgeODEs} on each edge and eliminate the constants $C_{+}^{(jk)}$, $C_{-}^{(jk)}$ in \eqref{eq:EdgeEqnGeneralSolution}, obtaining
	\begin{align*}
		j\not\con k &\implies
		\begin{cases}
			u_{jk}(t) = 0, \\
			u_{kj}(t) = 0,
		\end{cases} \\
		j\neq k, \ j\con k &\implies
		\begin{cases}
			u_{jk}(t) = e^{-i\qm_{jk}\bracs{t-l_{jk}}}\csc\bracs{\omega l_{jk}}\sin\bracs{\omega t}, \\
			u_{kj}(t) = -e^{-i\qm_{kj}\bracs{t-l_{kj}}}\csc\bracs{\omega l_{kj}}\sin\bracs{\omega t},
		\end{cases} \\
		j = k &\implies 
		\begin{cases}
			u_{jj}(t) = e^{-i\qm_{jj}t} \bracs{ e^{-i\omega t} + \sqbracs{e^{-i\qm_{jj}l_{jj}}-e^{-i\omega l_{jj}}}\csc\bracs{\omega l_{jj}}\sin\bracs{\omega t}  },
		\end{cases}
	\end{align*}
	This in turn enables us to explicitly differentiate the expressions for $u_{jk}$, and read off the values of $\bracs{\diff{}{t}+i\qm_{jk}}u_{jk}$ at the vertices.
	In the case $j\not\con k$, we obviously get zero contribution from the edges $I_{jk}$ and $I_{kj}$.
	The case $j\neq k, \ j\con k$, yields contributions from the edges $I_{jk}$ and $I_{kj}$ as follows:
	\begin{align*}
		\bracs{\diff{}{t}+i\qm_{jk}}u_{jk}\bracs{v_j} &= -\omega e^{-i\qm_{jk}l_{jk}}\csc\bracs{\omega l_{jk}}, \\
		\bracs{\diff{}{t}+i\qm_{jk}}u_{jk}\bracs{v_k} &= \omega\cot\bracs{\omega l_{jk}}, \\
		\bracs{\diff{}{t}+i\qm_{kj}}u_{kj}\bracs{v_j} &= -\omega e^{-i\qm_{kj}l_{kj}}\csc\bracs{\omega l_{kj}}, \\
		\bracs{\diff{}{t}+i\qm_{kj}}u_{kj}\bracs{v_k} &= \omega\cot\bracs{\omega l_{kj}}.
	\end{align*}
	And finally, when considering the case $j=k$ the contribution to $\bracs{\nmap u}_j$ from loops $I_{jj}$ in the graph also requires us to compute
	\begin{align*}
		\lim_{t\rightarrow0}-\bracs{u_{jj}'+i\qm_{jj}u_{jj}}(t) + \lim_{t\rightarrow l_{jj}}\bracs{u_{jj}'+i\qm_{jj}u_{jj}} & (t) \\
		&= 2\omega\bracs{ \cot\bracs{\omega l_{jj}} - \cos\bracs{\qm_{jj}l_{jj}}\csc\bracs{\omega l_{jj}}}.	
	\end{align*}
	We then use the formula
	\begin{align*}
		\bracs{\nmap u}_j &= \sum_{j\con l} \bracs{\diff{}{t}+i\qm_{jl}}u_{jl}\bracs{v_j},
	\end{align*}
	and our previous computations to compute each entry $\bracs{\nmap u}_j$, and obtain the desired result.
\end{proof}
Importantly this result demonstrates that the $M$-matrix can be thought of as a function of $\omega$ parametrised by $\qm$, hence will denote it by $M_{\qm}\bracs{\omega}$.