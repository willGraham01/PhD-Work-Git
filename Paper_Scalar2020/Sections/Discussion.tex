\section{Discussion} \label{sec:Discussion}
In this section we want to discuss some of the important applications of our work.
We should touch on:
\begin{itemize}
	\item ``decorators" - adding a graph to a periodic 1D structure by attaching at a vertex (or better: taking a graph, breaking an edge in two and identifying either sides of the ``break" as the edge of the period cell).
	This ties into the setups of EG \cite{cherednichenko2019time} (might be wrong Kirill paper, all the bibTex refs have similar names, check the PDFs and ensure citation points to the right work)
	\item M-matrix work to reduce the QG problem into a nicer, more accessible state.
	Potentially mention we have code to solve this, or do not depending on data plans!
	Also, avoid explicit computation and discussion of numerical ideas unless we want to include them - the examples section which follows is where to put these things.
	\item Also might want to discuss our thoughts about graph symmetry affecting the M-matrix properties (real), stability results, etc, which we can formulate into (at least) conjectures.
\end{itemize}

\tstk{prop \ref{prop:M-MatrixEntries} needs to be moved to the right place in the text, in particular after the QG problem is derived - maybe even into the examples section?}
\begin{prop}[$M$-matrix entries] \label{prop:M-MatrixEntries}
	Let $\graph=\bracs{\vertSet,\edgeSet}$ be an embedded graph on which the problem \eqref{eq:QGFullSystem} is posed.
	For each $I_{jk}\in \edgeSet$ let $\qm_{jk} = \bracs{R_{jk}\qm}_1$ and $l_{jk} = \abs{I_{jk}}$.
	Suppose that $\dmap u = e_k$ where $e_k$ is the $k$\textsuperscript{th} canonical unit vector in $\reals^{\abs{V}}$.
	Then the $j$\textsuperscript{th} entry of $\nmap u$, and hence the $jk$\textsuperscript{th} entry in the $M$-matrix, is given by
	\begin{align*}
		\bracs{\nmap u}_j &= 
		\begin{cases}
			\!\begin{aligned}
				&0,
			\end{aligned}			
			& j \not\con k, \\
			\!\begin{aligned}
				&-\sum_{j\conLeft k} \omega e^{i\qm_{jk}l_{jk}} \csc\bracs{l_{jk}\omega} 
				\\ &\quad - \sum_{j\conRight k} \omega e^{-i\qm_{kj}l_{kj}} \csc\bracs{l_{kj}\omega},
			\end{aligned}
			& j\neq k, \ j\con k, \\
			\!\begin{aligned}
				&\sum_{j\con l} \omega\cot\bracs{l_{jl}\omega}
				\\ &\quad + 2\omega\sum_{j\conLeft j} \cot\bracs{l_{jj}\omega} - \cos\bracs{\qm_{jj}l_{jj}}\csc\bracs{l_{jj}\omega},
			\end{aligned}
			& j=k.
		\end{cases}
	\end{align*}
	Note the choice of $j\conLeft j$ in the contributions from loops is simply a convention, $j\conRight j$ is equivalent here.
	Also recall the convention for summing over $j\con k$:
	\begin{align*}
		\sum_{j\con k} \omega^2\cot\bracs{l_{jk}\omega} &= \sum_{j\conLeft k} \omega\cot\bracs{l_{jk}\omega}	+ \sum_{j\conRight k} \omega\cot\bracs{l_{kj}\omega}
	\end{align*}
\end{prop}
\begin{proof}
	\tstk{the proof needs contextualising too}
	The proof is an explicit computation, and follows the same idea as in \cite{ershova2014isospectrality} with adjustments for the fact that there are $\qm$ terms floating around.
	For each $k$, setting $\dmap u = e_k$ provides us with sufficient Dirichlet data at each vertex to eliminate determine the form of the edge solution $u_{jk}$.
	This in turn enables us to explicitly differentiate $u_{jk}$, and read off the values of $u_{jk}$ and $u_{jk}'$ at any relevant vertices - hence if one wishes, the form of the eigenfunctions can also be recovered via this method.
	This then provides us with the value of each term in the sum in $\bracs{\nmap u}_j$, for each $j$.
\end{proof}
Importantly this result demonstrates that the $M$-matrix can be thought of as a function of $\omega$ parametrised by $\qm$, hence will denote it by $M_{\qm}\bracs{\omega}$.