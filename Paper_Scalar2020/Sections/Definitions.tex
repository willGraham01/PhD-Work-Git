\section{Definitions and Key Concepts} \label{sec:Definitions}
Before we present a derivation of the system \tstk{ode system ref} from the measure-theoretic problem \tstk{ref}, we first need to formally introduce the objects involved in each.
To this end, in section \ref{ssec:QuantumGraphs} we cover the theory we need to borrow from the field of Quantum Graphs, and also setup our notation for graphs throughout this work.
We also define the concept of an embedded graph (section \ref{sssec:EmbeddedGraphs}) and a ``singular measure" (section \ref{sssec:SingularMeasures}) as they will be used in this work, which will be our link between the two problems \tstk{ode system ref} and \tstk{var formulation ref}.
Finally, the definition of the function spaces we plan to use in our variational problem \tstk{ref!} is given in section \ref{ssec:MeasureTheory}.

\subsection{Quantum/Metric Graphs} \label{ssec:QuantumGraphs}
In this section we outline the concept of a quantum graph, as well as introduce the function spaces associated with them, and defining differential operators on them.
Let $\graph=\bracs{V,E}$ be a directed graph with vertices $v_j\in V$ and edges $I_{jk}=\bracs{v_j, v_k}\in E$, where $I_{jk}$ is directed from vertex $v_j$ (``on the left") to vertex $v_k$ (``on the right").
Strictly we should use the notation $I_{jk}^l$ for the edges, as in general there may be $l$ such connections from $v_j$ to $v_k$, however this fine detail will not play a significant role in our analysis and so we drop this additional superscript to avoid notational clutter.
Each edge $I_{jk}$ is assigned a length $l_{jk}>0$ and interval $\sqbracs{0,l_{jk}}$, this interval we also denote by $I_{jk}$ (for reasons that will become apparent shortly).
We also assign each vertex $v_j$ a ``coupling constant" $\alpha_j\in\reals$ \tstk{should only be $<0$ I think, due to our normal derivative choice}.
Then the graph $\graph$ is called a quantum graph. \newline

By providing each edge with a length and hence interval, we can construct functions on quantum graphs by describing their behaviour on each edge of the graph, and hence can produce function spaces such as
\begin{subequations} \label{eq:GraphFuncSpaces}
	\begin{align}
		L^2\bracs{\graph} := \bigoplus_{I_{jk}\in E} \ltwo{I_{jk}}{t},
		&\quad H^1\bracs{\graph} := \bigoplus_{I_{jk}\in E} \gradSob{I_{jk}}{t}, \\
		H^2\bracs{\graph} := \bigoplus_{I_{jk}\in E} H^2_\mathrm{grad}\bracs{I_{jk}, \md t}. &
	\end{align}
\end{subequations}
Because we will mainly be working on the edges of our graphs, we define $u_{jk} = u\vert_{I_{jk}}$ to be the restriction of a function $u$ defined on $\graph$ to the edge $I_{jk}$, extended by zero.
As the directed edge $I_{jk}$ is associated to the interval $\sqbracs{0,l_{jk}}$, we will also use the shorthand
\begin{align*}
	u_{jk}\bracs{v_j} &= u_{jk}\bracs{0}, &\quad
	u_{jk}\bracs{v_k} = u_{jk}\bracs{l_{jk}}, \\
	u_{kj}\bracs{v_j} &= u_{kj}\bracs{l_{kj}}, &\quad
	u_{kj}\bracs{v_k} = u_{kj}\bracs{0}.
\end{align*}
If $u$ is continuous at a vertex $v_j$, we will use the notation $u\bracs{v_j}$ for this value.
This will be convenient later, when we work with quantum graphs that have been obtained from graphs embedded in $\reals^2$.
We also require a notion of ``normal derivative" for our functions when we approach the endpoints of an edge in the graph $\graph$.
To this end, we adopt the following convention;
\begin{align*}
	\diff{}{t}u_{jk}\bracs{v_j} &= -u_{jk}'\bracs{v_j} = -\lim_{t\rightarrow0}u_{jk}'(t), \\
	\diff{}{t}u_{jk}\bracs{v_k} &= u_{jk}'\bracs{v_k} = \lim_{t\rightarrow l_{jk}}u_{jk}'(t),
\end{align*}
for suitably differentiable functions $u$ on $\graph$. 
In effect, we are using an ``exterior facing" normal derivative for our edges. 
Before we discuss differential operators on quantum graphs we introduce a small amount of notation to make the equations we present more readable.
We use the notation $j\conLeft k$ to mean ``the vertex $v_j$ connects to $v_k$ with $v_j$ on the left", and $j\conRight k$ to mean ``the vertex $v_j$ connects to $v_k$ with $v_j$ on the right".
Lastly, we use $j\con k$ to mean ``the vertex $v_j$ connects to $v_k$" in cases where we don't care about the direction of the connection.
Formally we have
\begin{align*}
	j\conLeft k &\Leftrightarrow I_{jk}\in E, \quad
	j\conRight k \Leftrightarrow I_{kj}\in E, \\
	j\con k &\Leftrightarrow I_{jk}\in E \text{ or } I_{kj}\in E.
\end{align*}
We can then utilise this notation as shorthand in summations involving edges or vertices in the following manner; for a fixed $v_j\in V$
\begin{align*}
	\sum_{j\conLeft k} = \sum_{\substack{k \\ I_{jk}\in E}}, \quad
	\sum_{j\conRight k} = \sum_{\substack{k \\ I_{kj}\in E}}.
\end{align*}
The use of $j\con k$ as a summation index requires more careful treatment, as there may only be edges directed one-way between vertices $v_j$ and $v_k$, and the order of our indices needs to be reversed when the sum involves $u_{jk}$ and the like.
Formally we have
\begin{align*}
	\sum_{j\con k} = \sum_{j\conLeft k} + \sum_{j\conRight k},
\end{align*}
which we will use in context as below:
\begin{align*}
	\sum_{j\con k}u_{jk}\bracs{v_j} &= \sum_{j\conLeft k}u_{jk}\bracs{v_j} + \sum_{j\conRight }u_{kj}\bracs{v_j}.
\end{align*}
Essentially any $jk$ index (referencing an edge) appearing on the left hand side is reversed as appropriate in the sums on the right hand side, but indices referencing vertices (like the $v_j$) remain the same in all sums. \newline

Differential operators (or differential equations) on quantum graphs can then be defined by specifying their domains and their action on each edge of the graph $\graph$.
To obtain a well-posed problem (strictly speaking, a self-adjoint operator \tstk{refs!}) on a graph however will require specification of boundary conditions at the vertices, which are essentially the boundaries of our edges $I_{jk}$.
Specifying these ``boundary conditions" or ``vertex conditions" also imbues the problem with an appreciation for the connectivity of the graph (which the function spaces in \ref{eq:GraphFuncSpaces} do not convey on their own).
Whilst there are several choices one can make for the vertex conditions, we will only be interested in imposing continuity of the solution at each vertex, and a Kirchoff-like condition on the derivatives of the solution.
By way of example, let us demonstrate how we define the operator $-\ddiff{}{t}$ on $\graph$.
The domain of $\mathcal{A}$ shall be
\begin{align} \label{eq:ExampleOppDomainDef}
	\mathrm{dom}\mathcal{A} &= \clbracs{ u\in H^2\bracs{\graph} \ \vert \ \text{At each } v_j\in V, \ u \text{ is continuous and } \sum_{j\con k}\diff{u_{jk}}{t}\bracs{v_j} = \alpha_j u\bracs{v_j} },
\end{align}
whilst it's action on each edge is
\begin{align} \label{eq:ExampleOppEdgeDef}
	\mathcal{A} &= -\ddiff{}{t} \quad\text{on each } I_{jk}\in E.
\end{align}
Of course, by ``on each $I_{jk}\in E$" we mean ``on the interval $\sqbracs{0,l_{jk}}$ that we associate to $I_{jk}\in E$".
Then for a function $f\in L^2\bracs{\graph}$ we can pose the resolvent problem of finding $u\in\mathrm{dom}\mathcal{A}$ such that
\begin{align*}
	\mathcal{A}u &= f;
\end{align*}
or alternatively can consider the spectral problem of finding eigenpairs $\bracs{\lambda,u}\in\complex\times\mathrm{dom}\mathcal{A}$ such that
\begin{align*}
	\mathcal{A}u &= \lambda u.
\end{align*}
As the spaces in \eqref{eq:GraphFuncSpaces} break down into edge-wise components which are acted on individually by $\mathcal{A}$, and only linked through the vertex conditions, we can rewrite both of these problems as a set of ODEs on intervals coupled through vertex conditions;
\begin{align*}
	\mathcal{A}u = f \quad\Leftrightarrow\quad &
	\begin{cases}
		-\ddiff{u_{jk}}{t} = f_{jk} \ \text{on } I_{jk}, \\
		u \text{ is continuous at each } v_j\in V, \\
		\sum_{j\con k}\diff{u_{jk}}{t}\bracs{v_j} = \alpha_j u\bracs{v_j} \ \forall v_j\in V
	\end{cases} \\
	\mathcal{A}u = \lambda u \quad\Leftrightarrow\quad &
	\begin{cases}
		-\ddiff{u_{jk}}{t} = \lambda u_{jk} \ \text{on } I_{jk}, \\
		u \text{ is continuous at each } v_j\in V, \\
		\sum_{j\con k}\diff{u_{jk}}{t}\bracs{v_j} = \alpha_j u\bracs{v_j} \ \forall v_j\in V.
	\end{cases}
\end{align*}

\subsubsection{The $M$-Matrix} \label{sssec:MMatrix}
When solving spectral problems on quantum graphs, recent developments involving an object known as the $M$-matrix will provide us with a useful tool in determining the spectrum, as we discuss in \tstk{section reference}.
The $M$-matrix is in fact one particular case of the more general (Weyl-Titchmarsh) $M$-operator as it appears in the theory of boundary triples (more information can be found \tstk{refs}), and what we present here is simply the framework that we require without any of the generality of this theory.
Let $\mathcal{A}$ be the operator in equations \eqref{eq:ExampleOppDomainDef}-\eqref{eq:ExampleOppEdgeDef}, and define the maps
\begin{align*}
	\dmap, \nmap: \mathrm{dom}\mathcal{A} \rightarrow \complex^{\abs{V}};
\end{align*}
where for a function $u\in\mathrm{dom}\mathcal{A}$, $\dmap$ sends $u$ to it's Dirichlet data at each of the vertices $v_j$ represented in a column vector in $\complex^{\abs{V}}$, and $\nmap$ sends $u$ to it's Neumann data at each of the vertices $v_j$ (again represented as a column vector).
Assuming we are using the labels $v_j, j\in\clbracs{1,...,\abs{V}}$ for the vertices of $\graph$, the $j$\textsuperscript{th} entry of the vectors $\dmap u$ and $\nmap u$ when considering $\mathcal{A}$ as in \eqref{eq:ExampleOppDomainDef}-\eqref{eq:ExampleOppEdgeDef} are given by
\begin{align*}
	\bracs{\dmap u}_j &= u\bracs{v_j}, \\
	\bracs{\nmap u}_j &= \sum_{j\con k}\diff{u_{jk}}{t}\bracs{v_j}.
\end{align*}
The $M$-matrix is then the unique map
\begin{align*}
	M\bracs{\lambda}:\complex^{\abs{V}} &\rightarrow \complex^{\abs{V}},
\end{align*}
such that whenever $u\in\mathrm{ker}\bracs{\mathcal{A}-\lambda}$,
\begin{align*}
	M\bracs{\lambda}\dmap u &= \nmap u.
\end{align*}
Of particular relevance to our work is that eigenvalues of $\mathcal{A}$ occur at those $\lambda$ such that 
\begin{align*}
	\mathrm{det}\bracs{M\bracs{\lambda} - A}= 0,
\end{align*}
where $A = \mathrm{diag}\bracs{\alpha_1, \alpha_2, ... , \alpha_{\abs{V}}}$ is the diagonal matrix of the vertex coupling constants.
One can see that this effectively reduces a system of ODEs to a matrix eigenvalue problem, at least as far as determining the spectrum is concerned.
We will exploit this fact in section \tstk{ref!}.

\subsubsection{Embedded Graphs and Singular Measures} \label{sssec:EmbeddedGraphs}
Quantum graphs and differential equations on them will be what we aim to translate our abstract, measure-theoretic problems into (\tstk{section reference}), but the quantum graphs themselves will arise from embedded graphs in $\reals^2$.
We say a (directed) graph $\graph = \bracs{V,E}$ is embedded into $\reals^2$ if each vertex $v_j$ is associated to a point which we also label $v_j\in\reals^2$; and each edge $I_{jk}$ associated to a curve $\gamma_{jk}\subset\reals^2$ with (arc-)length $l_{jk}$, such that there is a continuous map
\begin{align*}
	r: \sqbracs{0,l_{jk}} \rightarrow \gamma_{jk}, \quad r(0) = v_j, \quad r\bracs{l_{jk}} = v_k.
\end{align*}
Again, we will drop the distinction between $I_{jk}$ and $\gamma_{jk}$, simply using $I_{jk}$ for both.
It is clear from this definition how an embedded graph gives rise to a corresponding quantum graph, or how a quantum graph could be used to construct an embedded graph.
The choice of $\reals^2$ is simply to cover the situations we will be considering in this work, there are more general definitions depending on the choice of space to ``embed" the graph into.
For a unit vector $x\in\reals^2$, an embedded graphs is said to be $T$-periodic in the direction $x$ if it is invariant under the translation $Tx$ applied to it's vertices and edges.
By convention, $T$ refers to the minimal such scalar $T>0$ for which this happens.
If $\graph$ is periodic in the axial directions $e_1, e_2$ (with scalars $T_1, T_2$ respectively), then we can define the ``period cell" or ``unit cell" $\mathcal{P}$ of $\graph$ in the obvious manner: take the intersection of the graph $\graph$ with the region $\mathcal{P} = \sqbracs{0,T_1}\times\sqbracs{0,T_2}$ and match the left boundary to the right, and top boundary to the bottom.
That is, view the graph in the region $\mathcal{P}$ as living on the torus, as in figure \ref{fig:PeriodCellIllustration}.
\begin{figure}[b!]
	\centering
	\begin{subfigure}[t]{0.45\textwidth}
		\centering
		\includegraphics[height=4.5cm]{Diagram_PeriodCellFullLattice.pdf}
		\caption{\label{fig:Diagram_PeriodCellFullLattice} A periodic graph embedded into $\reals^2$, with the period cell marked.}
	\end{subfigure}
	~
	\begin{subfigure}[t]{0.45\textwidth}
		\centering
		\includegraphics[height=4.5cm]{Diagram_PeriodCellEdgeAssociation.pdf}
		\caption{\label{fig:Diagram_PeriodCellEdgeAssociation} The period cell of the graph in \ref{fig:Diagram_PeriodCellFullLattice}, illustrating how the edges of the period cell are associated.}
	\end{subfigure}
	\\
	\begin{subfigure}[b]{0.75\textwidth}
		\centering
		\includegraphics[scale=	1.0]{Diagram_PeriodCellOnTorus.pdf}
		\caption{\label{fig:Diagram_PeriodCellOnTorus} An illustration of the period cell being situated on a torus.}
	\end{subfigure}
	\caption{\label{fig:PeriodCellIllustration} Illustrating a periodic cell of a periodic embedded graph.}
\end{figure} 
Obviously the graph $\graph_{\mathcal{P}}$ contained in the period cell $\mathcal{P}$ of an embedded periodic graph $\graph$ is itself an embedded graph (after association of the boundaries of the period cell). 
We will refer to $\graph_{\mathcal{P}}$ as the ``period graph" or ``unit graph" of $\graph$, and write $\graph_{\mathcal{P}} = \graph \cap \mathcal{P}$ as shorthand. \newline

In the work that follows we restrict ourselves to considering straight-edges between vertices, so the curves $\gamma_{jk} = \sqbracs{v_j, v_k}$ are simply the line segments joining the vertices at either end, with lengths $l_{jk} = \norm{v_j-v_k}_2$.
As such, for each edge $I_{jk}$ in our embedded graphs, we can use the parametrisation
\begin{align} \label{eq:EdgeParameterisation}
	r_{jk}:\sqbracs{0, l_{jk}} \rightarrow I_{jk},
	&\quad r_{jk}\bracs{t} = v_j + te_{jk},
\end{align}
where $e_{jk}$ is the unit vector parallel to $I_{jk}$ and directed from $v_j$ to $v_k$ (consistent with the direction of the edge $I_{jk}$ itself).
Note that we also have $r_{jk}'(t) = e_{jk}$.

\subsubsection{Singular Measures} \label{sssec:SingularMeasures}
Whilst discussing embedded graphs we should introduce the object which will link quantum graph problems (as in section \ref{ssec:QuantumGraphs}) and the measure-theoretic problems that we will take as our start point \tstk{ref when it exists}.
For an embedded graph $\graph = \bracs{V,E}$ and for each $I_{jk}\in E$, define the (Borel) measure $\lambda_{jk}$ as the measure which supports 1D Lebesgue measure on the edge $I_{jk}$.
So for each Borel set $B$ we have that 
\begin{align*}
	\lambda_{jk}\bracs{B} = \lambda_{1}\bracs{r_{jk}^{-1}\bracs{B \cap I_{jk}}}
\end{align*}
where $\lambda_1$ is the 1D-Lebesgue measure on $\reals$, and $r_{jk}$ is the parametrisation of the edge $I_{jk}$ (see \eqref{eq:EdgeParameterisation}).
Then set $\ddmes$ to be the (Borel) measure defined by
\begin{align*}
	\ddmes\bracs{B} = \sum_{v_j\in V}\sum_{j\conLeft k} \lambda_{jk}\bracs{B}.
\end{align*}
Then $\ddmes$ is the ``singular measure that supports $\graph$"; or alternatively the ``singular measure on $\graph$", or the ``(singular) measure that supports the edges of $\graph$". \tstk{didn't include the mass-on-the-vertices (coupling constant / vertex order) here, which will be needed for when we consider the scaled-vertex example.}
For a graph embedded into a 2D domain, the singular measure is illustrated in figure \ref{fig:Diagram_SingularMeasure2D}.
\begin{figure}[t!]
	\centering
	\includegraphics[scale=0.85]{Diagram_SingularMeasure2D.pdf}
	\caption{\label{fig:Diagram_SingularMeasure2D} For a graph embedded in $\reals^2$, the $\ddmes$-measure of any Borel set $B$ is obtained from summing the contributions of each $\lambda_{jk}$, as indicated by the thickened and coloured lines.
	Sets that do not intersect $\graph$ have zero measure.}
\end{figure} \newline

The singular measure of a graph will be the key component in our measure-theoretic formulations, which enables us to establish ideas of derivatives whilst working on a domain which has no ``area" in the Lebesgue-sense.

\subsection{Measure Theoretic Concepts} \label{ssec:MeasureTheory}
In this section we address the question of how one can pose ``differential equations" on singular structures, by outlining the appropriate differential operators and function spaces that are required.
As touched on in section \tstk{sec ref}, attempting to pose some kind of differential system on a singular-structure by drawing analogy to the ``ingredients" of a differential equation on a thin-structure runs into problems, due to the singular-structure lacking a domain interior.
With no interior, imposing any kind of boundary data would be imposing a condition on the entire domain.
One could ``bypass" explicitly writing boundary conditions by using a weak/variational formulation rather than a strong formulation, but since the Lebesgue measure of the singular-structure is zero this results in a formulation involving integrals which are identically zero.
The solution to these problems is not to abandon what we believe are the ingredients of a differential system, but rather rework our concepts of integration and differentiation so that they respect the fact that we are looking at a problem on the singular-structure itself, despite the fact that the structure lives in a higher-dimensional set. \tstk{zhikov mention?}

Let $\graph=\bracs{V,E}$ be a finite graph embedded into the region $\ddom=\sqbracs{0,T_1}\times\sqbracs{0,T_2}\subset\reals^2$, which will serve as our singular structure.
For $\qm=\bracs{\qm_1,\qm_2}\in\left[-\frac{\pi}{T_1},\frac{\pi}{T_1}\right)\times\left[-\frac{\pi}{T_2},\frac{\pi}{T_2}\right)$ define the shifted gradient operator $\tgrad$ on smooth functions $\phi\in\smooth{\ddom}$ by
\begin{align*}
	\tgrad\phi &= \begin{pmatrix} \partial_1\phi + i\qm_1\phi \\ \partial_2\phi + i\qm_2\phi \end{pmatrix}.
\end{align*}
In sections \tstk{section ref} we will consider $\graph$ as the period cell of some periodic, embedded graph in $\reals^2$, and the operator $\tgrad$ arises after using a Gelfand transform to aid us in solving differential equations posed on such a graph.
As such one can think of the parameter $\qm$ as the quasi-momentum, however in this section it has little effect beyond being a scalar-shift to our derivatives.
Let $\ddmes$ be the singular measure supported on $\graph$, and for each $I_{jk}\in E$ let $\lambda_{jk}$ be the singular measure supporting the edge $I_{jk}$, as in section \ref{sssec:SingularMeasures}.
Then one can construct the sets $W^{\qm}_\graph$ as the closure of the set of pairs $\bracs{\phi, \tgrad\phi}$ in the $\ltwo{\ddom}{\ddmes}\times\ltwo{\ddom}{\ddmes}^2$ norm, and $W^{\qm}_{jk}$ similarly as the closure of set of pairs $\bracs{\phi, \tgrad\phi}$ in the $\ltwo{\ddom}{\lambda_{jk}}\times\ltwo{\ddom}{\lambda_{jk}}^2$ norm.
The idea here being to construct an analogy of a ``Sobolev space" for functions defined on our singular-structure, and hence obtain a concept of ``(weak) derivative".
With this in mind, for a pair $\bracs{u,g}\in W^{\qm}_\graph$ it is tempting to call $g$ the ``gradient" of $u$ in this new sense; however this implies that $u$ has only one gradient in this sense, and this is not the case.
Indeed if $\bracs{u,g_1},\bracs{0,g_2}\in W^{\qm}_\graph$ then clearly $\bracs{u, g_1+g_2}\in W^{\qm}_\graph$ too.
This means that $W^{\qm}_\graph$ isn't quite the set of functions we want when it comes to singular-structure problems, however we can obtain a usable set of functions from $W^{\qm}_\graph$. \newline

We define the ``set of $\ddmes$-gradients of zero" as
\begin{align} \label{eq:GradZeroDef}
	\gradZero{\ddom}{\ddmes} &= \clbracs{ g \ \vert \ \bracs{0,g}\in W^{\qm}_\graph }, \\
	&= \clbracs{ g \ \vert \ \exists\phi_n\in\smooth{\ddom} \text{ s.t. } \phi_n\lconv{\ltwo{\ddom}{\ddmes}}0, \tgrad\phi_n\lconv{\ltwo{\ddom}{\ddmes}^2}g }
\end{align}
which is a closed, linear subspace of $\ltwo{\ddom}{\ddmes}^2$.
It can be shown that $\gradZero{\ddom}{\ddmes}$ does not depend on the value of $\qm$, which is why the notation lacks a $\qm$ symbol:
\begin{prop}[Gradients of Zero are Invariant Under Quasi-Momentum] \label{prop:GradZeroInvarientUnderQM}
	For any fixed $\qm\in[-\pi,\pi)^2$, let 
	\begin{align*}
		\mathcal{G}^0_{\ddom, \md \ddmes} &= \clbracs{ g \ \vert \ \bracs{0,g}\in W^0_\graph}, \\
		\mathcal{G}^{\qm}_{\ddom, \md \ddmes} &= \clbracs{ g \ \vert \ bracs{0,g}\in W^{\qm}_\graph }.
	\end{align*}
	Then $\mathcal{G}^0_{\ddom, \md \ddmes} = \mathcal{G}^{\qm}_{\ddom, \md \ddmes}$.
\end{prop}
\begin{proof}
	Given that $\tgrad\phi = \grad\phi + i\qm\phi$ for smooth $\phi$, if $g\in\mathcal{G}^0_{\ddom, \md \ddmes}$ there exists a sequence of smooth functions $\phi_n$ such that 
	\begin{align*}
		\phi_n\lconv{\ltwo{\ddom}{\ddmes}}0, &\grad\phi_n\lconv{\ltwo{\ddom}{\ddmes}^2}g.
	\end{align*}
	But then clearly
	\begin{align*}
		\phi_n\lconv{\ltwo{\ddom}{\ddmes}}0, &\tgrad\phi_n = \grad\phi_n + i\qm\phi_n \lconv{\ltwo{\ddom}{\ddmes}^2} g + 0 = g,
	\end{align*}
	so $g\in\mathcal{G}^\qm_{\ddom, \md \ddmes}$.
	The reverse direction is similar.
\end{proof}

With $\gradZero{\ddom}{\ddmes}$ being a closed, linear subspace of $\ltwo{\ddom}{\ddmes}^2$, one has the orthogonal decomposition
\begin{align*}
	\ltwo{\ddom}{\ddmes}^2 &= \gradZero{\ddom}{\ddmes}^{\perp} \oplus \gradZero{\ddom}{\ddmes}.
\end{align*}
So for each $u\in\ltwo{\ddom}{\ddmes}$ there exists a $\grad_\ddmes u\in\gradZero{\ddom}{\ddmes}^{\perp}$ such that any pair $\bracs{u,z}\in W^{\qm}_\graph$ can be written as $\bracs{u,\grad_\ddmes u + g}$ where $g\in\gradZero{\ddom}{\ddmes}$.
We call $\grad_\ddmes u$ the ``tangential derivative" of $u$ and it is unique in this sense; which allows us to construct the ``Sobolev space"
\begin{align*}
	\gradSobQM{\ddom}{\ddmes} &= \clbracs{ \bracs{u, \grad_\ddmes u}\in W^{\qm}_\graph \ \vert \ \grad_\ddmes u \in \gradZero{\ddom}{\ddmes}^{\perp} }.
\end{align*}
Since $\grad_\ddmes u$ is unique, we will use the shorthand $u\in\gradSobQM{\ddom}{\ddmes}$ to refer to the pair $\bracs{u, \grad_\ddmes u}$. \tstk{worth a note about how tang gradient isn't always the gradient we want when solving equations with non-identity elliptic $A$}
We can follow a similar process to define the sets $\gradZero{\ddom}{\lambda_{jk}}$ and $\gradSobQM{\ddom}{\lambda_{jk}}$. \newline

This construction gives us a concept of ``derivative" to work with; however there is still work to be done towards understanding the nature of these tangential derivatives, which will be key to the derivation of our system \tstk{system reference} from \tstk{m theory system}.
We explore this in section \tstk{section ref!}.
