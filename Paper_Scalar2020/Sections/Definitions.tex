\section{Definitions and Key Concepts} \label{sec:Definitions}
Before we present a derivation of the system \tstk{ode system ref} from the measure-theoretic problem \tstk{ref}, we first need to formally introduce the objects involved in each.
To this end, in section \ref{ssec:QuantumGraphs} we cover the theory we need to borrow from the field of Quantum Graphs, and also setup our notation for graphs throughout this work.
We also define the concept of an embedded graph (section \ref{sssec:EmbeddedGraphs}) and a ``singular measure" (section \ref{sssec:SingularMeasures}) as they will be used in this work, which will be our link between the two problems \tstk{ode system ref} and \tstk{var formulation ref}.
Finally, the definition of the function spaces we plan to use in our variational problem \tstk{ref!} is given in section \ref{ssec:MeasureTheory}.

\subsection{Quantum/Metric Graphs} \label{ssec:QuantumGraphs}
In this section we outline the concept of a quantum graph, as well as introduce the function spaces associated with them, and defining differential operators on them.
Let $\graph=\bracs{\vertSet,\edgeSet}$ be a directed graph with vertices $v_j\in \vertSet$ and edges $I_{jk}=\bracs{v_j, v_k}\in \edgeSet$, where $I_{jk}$ is directed from vertex $v_j$ (``on the left") to vertex $v_k$ (``on the right").
Strictly we should use the notation $I_{jk}^l$ for the edges; with $l$ being a script which runs from $1$ to the number of edges with left endpoint $v_j$ and right endpoint $v_k$, as in general there may be multiple such connections from $v_j$ to $v_k$.
However we drop this superscript to avoid notational clutter, and because it's inclusion is not necessary to understand the concepts we introduce here.
Each edge $I_{jk}$ is assigned a length $l_{jk}>0$ and interval $\sqbracs{0,l_{jk}}$, which we also denote by $I_{jk}$ (for reasons that will become apparent shortly).
We also assign each vertex $v_j$ a ``coupling constant" $\alpha_j\in\complex$ (although in our examples we will only be considering $\alpha_j\in\reals$).
A graph $\graph$ with the metric structure described above is called a \emph{quantum graph}, or \emph{metric graph}. \tstk{cite Kuchment by page/chapter, also check Kuchment book whether we need to include the operator in the definition too!}

By providing each edge with a length and hence associating it with an interval, we can construct functions on quantum graphs by describing their behaviour on the edges, and hence can produce the function spaces
\begin{subequations} \label{eq:GraphFuncSpaces}
	\begin{align}
		L^2\bracs{\graph} := \bigoplus_{I_{jk}\in \edgeSet} \ltwo{I_{jk}}{t},
		&\quad H^1\bracs{\graph} := \bigoplus_{I_{jk}\in \edgeSet} \gradSob{I_{jk}}{t}, \\
		H^2\bracs{\graph} := \bigoplus_{I_{jk}\in \edgeSet} H^2_\mathrm{grad}\bracs{I_{jk}, \md t}. &
	\end{align}
\end{subequations}
We define $u_{jk} = u\vert_{I_{jk}}$ to be the restriction of a function $u$ defined on $\graph$ to the edge $I_{jk}$, extended by zero.
As the directed edge $I_{jk}$ is associated to the interval $\sqbracs{0,l_{jk}}$, we will also use the shorthand
\begin{align*}
	u_{jk}\bracs{v_j} &= u_{jk}\bracs{0}, &\quad
	u_{jk}\bracs{v_k} = u_{jk}\bracs{l_{jk}}, \\
	u_{kj}\bracs{v_j} &= u_{kj}\bracs{l_{kj}}, &\quad
	u_{kj}\bracs{v_k} = u_{kj}\bracs{0}.
\end{align*}
If $u$ is continuous at a vertex $v_j$, we will use the notation $u\bracs{v_j}$ for this value.
This will be convenient later, when we work with quantum graphs that have been obtained from graphs embedded in $\reals^2$.
We also require a notion of ``signed derivative" (which is analogous to the ``normal derivative" for PDE problems) for our functions when we approach the endpoints of an edge in the graph $\graph$.
To this end, we adopt the following convention;
\begin{subequations} \label{eq:SignedDerivConvention}
	\begin{align}
		\diff{}{t}u_{jk}\bracs{v_j} &= -u_{jk}'\bracs{v_j} = -\lim_{t\rightarrow0}u_{jk}'(t), \\
		\diff{}{t}u_{jk}\bracs{v_k} &= u_{jk}'\bracs{v_k} = \lim_{t\rightarrow l_{jk}}u_{jk}'(t),
	\end{align}
\end{subequations}
for suitably differentiable functions $u$ on $\graph$. 
In effect the formulae \eqref{eq:SignedDerivConvention} define an ``exterior facing" derivative at each endpoint. \newline
 
Before we discuss differential operators on quantum graphs we introduce some notation to make the equations more readable: $j\conLeft k$ stands for ``an edge with endpoints $v_j$ and $v_k$, with $v_j$ on the left", and $j\con k$ is used to mean ``an edge with endpoints $v_j$ and $v_k$", for use in cases where we do not care about the direction of the connection.
We also use $j \conRight k$ which can be read as meaning $k\conLeft j$, that is ``an edge with endpoints $v_j$ and $v_k$, with $v_j$ on the right".
Formally we have
\begin{align*}
	j\conLeft k &\Leftrightarrow I_{jk}\in \edgeSet, \\
	j\con k &\Leftrightarrow I_{jk}\in \edgeSet \text{ or } I_{kj}\in \edgeSet.
\end{align*}
We can then utilise this notation as shorthand in summations involving edges or vertices in the following manner; for a fixed $j$
\begin{align*}
	\sum_{j\conLeft k} = \sum_{\substack{k \\ I_{jk}\in \edgeSet}}, \quad
	\sum_{j\conRight k} = \sum_{\substack{k \\ I_{kj}\in \edgeSet}}.
\end{align*}
Furthermore, for a fixed $j$ we use the notation
\begin{align*}
	\sum_{j\con k} = \sum_{j\conLeft k} + \sum_{j\conRight k},
\end{align*}
so for example
\begin{align*}
	\sum_{j\con k}u_{jk}\bracs{v_j} &= \sum_{j\conLeft k}u_{jk}\bracs{v_j} + \sum_{j\conRight }u_{kj}\bracs{v_j}.
\end{align*}
For consistency with the possibility of having multiple edges $I_{jk}^l$ connecting two vertices, then these sums should be interpreted as being over all such edges (that is, running over $k$ and the appropriate range of $l$). \newline

Differential operators on quantum graphs can then be defined by specifying their domains and their action on each edge of the graph $\graph$.
A well-posed problem on a graph requires specifying boundary conditions at the vertices, which are identified with the end-points of the corresponding edges $I_{jk}$.
Specifying these ``boundary conditions" or ``vertex conditions" also reflects the connectivity of the graph (which the function spaces \eqref{eq:GraphFuncSpaces} do not convey on their own).
There are a variety of choices one can make for the vertex conditions, and we will be interested in imposing continuity of the solution at each vertex, and a Robin-like condition on the derivatives of the solution.
Such conditions are also referred to as ``$\delta-$type " conditions \tstk{refs to Abelviero et al (exactly solvable) and Berkolainko-Kuchment}.
By way of example, let us demonstrate how we define the graph Laplacian $\mathcal{A}$.
The domain of $\mathcal{A}$ is defined as
\begin{align*}
	\mathrm{dom}\mathcal{A} &= \clbracs{ u\in H^2\bracs{\graph} \ \vert \ \forall j, \ u \text{ is continuous at } v_j, \sum_{j\con k}\diff{u_{jk}}{t}\bracs{v_j} = \alpha_j u\bracs{v_j} },
\end{align*}
and the differential expression on each edge is $-\ddiff{}{t}$.
Then for a function $f\in L^2\bracs{\graph}$ we can pose the resolvent problem of finding $u\in\mathrm{dom}\mathcal{A}$ such that
\begin{align*}
	\mathcal{A}u &= f;
\end{align*}
or alternatively can consider the spectral problem of finding eigenpairs $\bracs{\lambda,u}\in\complex\times\mathrm{dom}\mathcal{A}$ such that
\begin{align*}
	\mathcal{A}u &= \lambda u.
\end{align*}
Each of these problems can be rewritten as a system of ODEs on the edge intervals coupled through the vertex conditions;
\begin{align*}
	\mathcal{A}u = f \quad\Leftrightarrow\quad &
	\begin{cases}
		-\ddiff{u_{jk}}{t} = f_{jk} \ \forall j\conLeft k, \\
		\forall j, \ u \text{ is continuous at } v_j, \\
		\forall j, \sum_{j\con k}\diff{u_{jk}}{t}\bracs{v_j} = \alpha_j u\bracs{v_j}.
	\end{cases} \\
	\mathcal{A}u = \lambda u \quad\Leftrightarrow\quad &
	\begin{cases}
		-\ddiff{u_{jk}}{t} = \lambda u_{jk} \ \forall j\conLeft k, \\
		\forall j, \ u \text{ is continuous at } v_j, \\
		\forall j, \sum_{j\con k}\diff{u_{jk}}{t}\bracs{v_j} = \alpha_j u\bracs{v_j}.
	\end{cases}
\end{align*}

\subsubsection{The $M$-Matrix} \label{sssec:MMatrix}
When solving spectral problems on quantum graphs, recent developments involving an object known as the $M$-matrix will provide us with a useful tool in determining the spectrum, as we discuss in section \ref{ssec:ssec:Derivation}.
The $M$-matrix is in fact a generalisation of the classical Weyl-Titchmarsh $m$-function as it appears in the theory of boundary triples (more information can be found \tstk{refs to Kochuben, Gorbachuk \& Gorbachuk, Brown Marletta Naboko Wood, Kirill Sasha Luis}), however here we restrict ourselves to a short review the specific theory that will be relevant to this work.
Let $\mathcal{A}$ be the graph Laplacian, and define the maps
\begin{align*}
	\dmap, \nmap: \mathrm{dom}\mathcal{A} \rightarrow \complex^{\abs{\vertSet}};
\end{align*}
which send a function $u\in\mathrm{dom}\mathcal{A}$, to its Dirichlet and Neumann data at each of the vertices respectively;
\begin{align*}
	\bracs{\dmap u}_j &= u\bracs{v_j}, \quad &j=1,...,\abs{\vertSet} \\
	\bracs{\nmap u}_j &= \sum_{j\con k}\diff{u_{jk}}{t}\bracs{v_j}, \quad &j=1,...,\abs{\vertSet}. 
\end{align*}
The $M$-matrix is then defined by
\begin{align*}
	\complex^{\abs{\vertSet}} \ni M\bracs{\lambda}\dmap u &= \nmap u \in \complex^{\abs{\vertSet}},
	 &\quad \forall u\in\mathrm{ker}\bracs{\mathcal{A}-\lambda}.
\end{align*}
It can be shown that $\lambda$ is an eigenvalue of $\mathcal{A}$ if and only if
\begin{align*}
	\mathrm{det}\bracs{M\bracs{\lambda} - A}= 0,
\end{align*}
where $A = \mathrm{diag}\bracs{\alpha_1, \alpha_2, ... , \alpha_{\abs{\vertSet}}}$ is the diagonal matrix of the vertex coupling constants.
\tstk{A more general statement can be found in Derkach, Malomund.}
One can see that this effectively reduces a system of ODEs to a matrix eigenvalue problem, at least as far as determining the spectrum is concerned.
We will exploit this fact in section \tstk{ref!}.

\subsubsection{Embedded Graphs and Singular Measures} \label{sssec:EmbeddedGraphs}
We now introduce the concepts that will link the quantum graph problems we study to the measure-theoretic formulation introduced in \tstk{ref to equation. This will be made clearer in the introduction when we first introduce our problem in the measure-theoretic way, and explain how we will go about solving it.}
We say a (directed) graph $\graph = \bracs{\vertSet,\edgeSet}$ is embedded into $\reals^2$ if each vertex $v_j$ is associated to a point which we also label $v_j\in\reals^2$; and each edge $I_{jk}$ associated to a curve $\gamma_{jk}\subset\reals^2$ with (arc-)length $l_{jk}$, such that there is a continuous map
\begin{align*}
	r: \sqbracs{0,l_{jk}} \rightarrow \gamma_{jk}, \quad r(0) = v_j, \quad r\bracs{l_{jk}} = v_k.
\end{align*}
Again, we will drop the distinction between $I_{jk}$ and $\gamma_{jk}$, simply using $I_{jk}$ for both.
Clearly, an embedded graph gives rise to a quantum graph, and a quantum graph can be used to construct embedded graphs.
The choice of $\reals^2$ is simply to cover the situations we will be considering in this work, there are more general definitions depending on the choice of space to ``embed" the graph into.
For a unit vector $x\in\reals^2$, an embedded graph is said to be $T$-periodic in the direction $x$ if it is invariant under the translation $Tx$ applied to its vertices and edges.
In what follows, we always refer to the minimal such $T>0$ (``period") with this property for a given $x\in\reals^2$.
Note that so long as the graph $\graph$ is periodic in two linearly independent directions, a linear transform can be applied to transform the period cell of $\graph$ into a rectangle with sides parallel to the co-ordinate axes, and so without loss of generality we only consider rectangular period cells henceforth.
If $\graph$ is periodic in the (orthogonal) axial directions $e_1, e_2$ (with periods $T_1, T_2$ respectively), then we can define the ``period cell" or ``unit cell" $\mathcal{P}$ of $\graph$ in the obvious manner: take the intersection of the graph $\graph$ with the region $\mathcal{P} = \sqbracs{0,T_1}\times\sqbracs{0,T_2}$ and match the left boundary to the right, and top boundary to the bottom.
That is, view the part $\graph_{\mathcal{P}}$ of the graph $\graph$ contained in the region $\mathcal{P}$ as a set on a torus, see \ref{fig:PeriodCellIllustration}.
\begin{figure}[b!]
	\centering
	\begin{subfigure}[t]{0.45\textwidth}
		\centering
		\includegraphics[height=4.5cm]{Diagram_PeriodCellFullLattice.pdf}
		\caption{\label{fig:Diagram_PeriodCellFullLattice} A periodic graph embedded into $\reals^2$, with the period cell marked.}
	\end{subfigure}
	~
	\begin{subfigure}[t]{0.45\textwidth}
		\centering
		\includegraphics[height=4.5cm]{Diagram_PeriodCellEdgeAssociation.pdf}
		\caption{\label{fig:Diagram_PeriodCellEdgeAssociation} The period cell of the graph in \ref{fig:Diagram_PeriodCellFullLattice}, notice how the edges of the period cell are associated.}
	\end{subfigure}
	\\
	\begin{subfigure}[b]{0.75\textwidth}
		\centering
		\includegraphics[scale=	1.0]{Diagram_PeriodCellOnTorus.pdf}
		\caption{\label{fig:Diagram_PeriodCellOnTorus} An illustration of the period cell as a subset of a torus.}
	\end{subfigure}
	\caption{\label{fig:PeriodCellIllustration} Illustrating a periodic cell of a periodic embedded graph.}
\end{figure} 
We will refer to $\graph_{\mathcal{P}} = \graph \cap \mathcal{P}$ as the ``period graph", or ``unit graph" of $\graph$. \newline

In what follows we consider straight edges between vertices, so the $\gamma_{jk}$ are simply the line segments joining the vertices at either end, with lengths $l_{jk} = \norm{v_j-v_k}_2$. \tstk{suggested just using $\abs{\cdot}$?}
For each edge $I_{jk}$ we will use the parametrisation
\begin{align} \label{eq:EdgeParameterisation}
	r_{jk}:\sqbracs{0, l_{jk}} \ni t \mapsto v_j + te_{jk} \in I_{jk},
\end{align}
where $e_{jk}$ is the unit vector parallel to $I_{jk}$ and directed from $v_j$ to $v_k$ (consistent with the direction of the edge $I_{jk}$ itself).
Note that we also have $r_{jk}'(t) = e_{jk}$.

\subsubsection{Singular Measures} \label{sssec:SingularMeasures}
Whilst discussing embedded graphs, we should introduce the measure which links quantum graph problems to the measure-theoretic problems we first introduced \tstk{reference to introduction. Once written, this sentence can be redone}.
For an embedded graph $\graph = \bracs{\vertSet,\edgeSet}$ and for each $I_{jk}\in \edgeSet$, define the (Borel) measure $\lambda_{jk}$ as the measure which supports the one-dimensional Lebesgue measure on the edge $I_{jk}$.
So for each Borel set $B$ we have
\begin{align*}
	\lambda_{jk}\bracs{B} = \lambda_{1}\bracs{r_{jk}^{-1}\bracs{B \cap I_{jk}}}
\end{align*}
where $\lambda_1$ is the Lebesgue measure on $\reals$, and $r_{jk}$ is the parametrisation of the edge $I_{jk}$ (see \eqref{eq:EdgeParameterisation}).
Then set $\ddmes$ to be the (Borel) measure defined by
\begin{align*}
	\ddmes\bracs{B} = \sum_{v_j\in \vertSet}\sum_{j\conLeft k} \lambda_{jk}\bracs{B}.
\end{align*}
Then $\ddmes$ is the ``singular measure that supports $\graph$"; or alternatively the ``singular measure on $\graph$", or the ``(singular) measure that supports the edges of $\graph$". \tstk{didn't include the mass-on-the-vertices (coupling constant / vertex order) here, which will be needed for when we consider the scaled-vertex example.}
For a graph embedded into a 2D domain, the singular measure is illustrated in figure \ref{fig:Diagram_SingularMeasure2D}.
\begin{figure}[t!]
	\centering
	\includegraphics[scale=0.85]{Diagram_SingularMeasure2D.pdf}
	\caption{\label{fig:Diagram_SingularMeasure2D} For a graph embedded in $\reals^2$, the $\ddmes$-measure of any Borel set $B$ is obtained from summing the contributions of each $\lambda_{jk}$, as indicated by the thickened and coloured lines.
	Sets that do not intersect $\graph$ have zero measure.}
\end{figure} \newline

The singular measure of a graph will be the key component in our measure-theoretic formulations, which enables us to establish ideas of derivatives whilst working on a domain which has no ``area" in the Lebesgue-sense.

\subsection{Measure Theoretic Concepts} \label{ssec:MeasureTheory}
\tstk{per comments, this section (and most of it's content) will be moved to the appendix}
In this section we address the question of how one understands \tstk{starting equations reference}, by outlining the appropriate differential operators and function spaces that are required.
As touched on in section \tstk{sec ref}, attempting to pose a boundary-value problem on a singular-structure, by drawing analogy to the ``ingredients" of a boundary-value problem on a thin structure, runs into problems.
These are due to the singular structure lacking a domain interior from the perspective of the space it is embedded in, and thus the notion of boundary values ceases to make sense. \tstk{Gelfand transform also enters at this point!}
This issue is resolved by not abandoning what we believe are the ingredients of a boundary-value problem, but rather by reworking our concepts of integration and differentiation so that they respect the fact that we are looking at a problem on the singular-structure itself, despite the fact that it is embedded in a higher-dimensional space. \tstk{zhikov mention?}

Let $\graph=\bracs{\vertSet,\edgeSet}$ be a finite graph embedded (as detailed in section \ref{ssec:QuantumGraphs}) into the region $\ddom=\sqbracs{0,T_1}\times\sqbracs{0,T_2}\subset\reals^2$, which will serve as our singular structure.
For $\qm=\bracs{\qm_1,\qm_2}\in\left[-\frac{\pi}{T_1},\frac{\pi}{T_1}\right)\times\left[-\frac{\pi}{T_2},\frac{\pi}{T_2}\right)$ define the shifted gradient operator $\tgrad$ on smooth functions $\phi\in\smooth{\ddom}$ by
\begin{align*}
	\tgrad\phi &= \begin{pmatrix} \partial_1\phi + i\qm_1\phi \\ \partial_2\phi + i\qm_2\phi \end{pmatrix}.
\end{align*}
In sections \tstk{examples, maybe also discussion, or even this is our starting point!} we will consider $\graph$ as the period cell of some periodic, embedded graph in $\reals^2$, and the operator $\tgrad$ arises after using a Gelfand transform \tstk{relevant appendix/introduction section reference} to aid us in solving differential equations posed on such a graph.
The parameter $\qm$ is the quasi-momentum, which we incorporate into our analysis in this section.
Let $\ddmes$ be the singular measure supported on $\graph$, and for each $I_{jk}\in \edgeSet$, let $\lambda_{jk}$ be the singular measure supporting the edge $I_{jk}$, as in section \ref{sssec:SingularMeasures}.
Then one can construct the set
\begin{align*}
W^{\qm}_\graph &= \overline{ \clbracs{ \bracs{\phi, \tgrad\phi} \ vert \ \phi\in\smooth{\ddom} } },
\end{align*}
where the closure is taken in the $\ltwo{\ddom}{\ddmes}\times\ltwo{\ddom}{\ddmes}^2$ norm.
$W^{\qm}_{jk}$ is similarly constructed from the closure of set of pairs $\bracs{\phi, \tgrad\phi}$ in the $\ltwo{\ddom}{\lambda_{jk}}\times\ltwo{\ddom}{\lambda_{jk}}^2$ norm, for each $I_{jk}\in\edgeSet$.
The idea here being to construct an analogy of a Sobolev space for functions defined on our singular-structure, and hence obtain a concept of (weak) derivative.
With this in mind \tstk{smoother link here - why should we be thinking this way? W-construction of sob. spaces. Also review this part of the text, the following lines are ambiguous}, for a pair $\bracs{u,g}\in W^{\qm}_\graph$ it is tempting to call $g$ ``the gradient" of $u$, however this gives the implication that $u$ has only one gradient in this sense, and this is not the case.
Indeed if $\bracs{u,g_1},\bracs{0,g_2}\in W^{\qm}_\graph$ then clearly $\bracs{u, g_1+g_2}\in W^{\qm}_\graph$ too.
This means that $W^{\qm}_\graph$ isn't quite the set of functions we want when it comes to singular-structure problems, however we can obtain a usable set of functions from $W^{\qm}_\graph$. \newline

\tstk{move this to the gradients of zero section in the appendix/analysis section}
We define the ``set of $\ddmes$-gradients of zero" as
\begin{align} \label{eq:GradZeroDef}
	\gradZero{\ddom}{\ddmes} &= \clbracs{ g \ \vert \ \bracs{0,g}\in W^{\qm}_\graph }, \\
	&= \clbracs{ g \ \vert \ \exists\phi_n\in\smooth{\ddom} \text{ s.t. } \phi_n\lconv{\ltwo{\ddom}{\ddmes}}0, \tgrad\phi_n\lconv{\ltwo{\ddom}{\ddmes}^2}g }
\end{align}
which is a closed, linear subspace of $\ltwo{\ddom}{\ddmes}^2$.
It can be shown that $\gradZero{\ddom}{\ddmes}$ does not depend on the value of $\qm$, which is why the notation lacks a $\qm$ symbol:
\begin{prop}[Gradients of Zero are Invariant Under Quasi-Momentum] \label{prop:GradZeroInvarientUnderQM}
	For any fixed $\qm\in[-\pi,\pi)^2$, we have that
	\begin{align*}
		\mathcal{G}^0_{\ddom, \md \ddmes} := \clbracs{ g \ \vert \ \bracs{0,g}\in W^0_\graph} &= 
		\clbracs{ g \ \vert \ bracs{0,g}\in W^{\qm}_\graph } =: \mathcal{G}^{\qm}_{\ddom, \md \ddmes}.
	\end{align*}
\end{prop}
\begin{proof}
	Given that $\tgrad\phi = \grad\phi + i\qm\phi$ for smooth $\phi$, if $g\in\mathcal{G}^0_{\ddom, \md \ddmes}$ there exists a sequence of smooth functions $\phi_n$ such that 
	\begin{align*}
		\phi_n\lconv{\ltwo{\ddom}{\ddmes}}0, &\grad\phi_n\lconv{\ltwo{\ddom}{\ddmes}^2}g.
	\end{align*}
	But then clearly
	\begin{align*}
		\phi_n\lconv{\ltwo{\ddom}{\ddmes}}0, &\tgrad\phi_n = \grad\phi_n + i\qm\phi_n \lconv{\ltwo{\ddom}{\ddmes}^2} g + 0 = g,
	\end{align*}
	so $g\in\mathcal{G}^\qm_{\ddom, \md \ddmes}$.
	The proof of the opposite inclusion is similar.
\end{proof}

For each $u\in\ltwo{\ddom}{\ddmes}$ there exists a $\grad_\ddmes u\in\gradZero{\ddom}{\ddmes}^{\perp}$ such that any pair $\bracs{u,z}\in W^{\qm}_\graph$ can be written as $\bracs{u,\grad_\ddmes u + g}$ where $g\in\gradZero{\ddom}{\ddmes}$.
We call $\grad_\ddmes u$ the ``tangential derivative" of $u$ and it is unique in this sense; which allows us to construct the ``Sobolev space"
\begin{align*}
	\gradSobQM{\ddom}{\ddmes} &= \clbracs{ \bracs{u, \grad_\ddmes u}\in W^{\qm}_\graph \ \vert \ \grad_\ddmes u \in \gradZero{\ddom}{\ddmes}^{\perp} }.
\end{align*}
Since $\grad_\ddmes u$ is unique, we will use the shorthand $u\in\gradSobQM{\ddom}{\ddmes}$ to refer to the pair $\bracs{u, \grad_\ddmes u}$. \tstk{worth a note about how tang gradient isn't always the gradient we want when solving equations with non-identity elliptic $A$}
We can follow a similar process to define the sets $\gradZero{\ddom}{\lambda_{jk}}$ and $\gradSobQM{\ddom}{\lambda_{jk}}$. \newline

Next, \tstk{posing equations in these spaces, just introduce notation and shorthand so that it appears before we type it.}

This construction gives us a concept of ``derivative" to work with; however there is still work to be done towards understanding the nature of these tangential derivatives, which will be key to the derivation of our system \tstk{system reference} from \tstk{m theory system}.
We explore this in section \tstk{section ref!}.
