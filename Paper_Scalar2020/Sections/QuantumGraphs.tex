\section{Concepts from the Study of Metric- and Quantum Graphs} \label{sec:QuantumGraphs}
Here we introduce the objects appearing in the system \eqref{eq:QGFullSystem} and measure-theoretic formulation \eqref{eq:PeriodCellLaplaceStrongForm}.
To this end, we outline some elements of the theory of quantum graphs that we borrow in what follows, and setup the notation that we use throughout this work.
This includes a discussion of what is meant by an embedded graph, in order to provide a foundation for the term ``periodic graph" which represents the geometry of our (physically-motivated) singular structure.
We then define the function spaces we use when dealing with the problem \eqref{eq:QGFullSystem}, as well as an overview of the $M$-matrix, a tool that will be key to our analysis of it's spectrum.
Although the concept of a singular measure is needed to formally link \eqref{eq:PeriodCellLaplaceStrongForm} and \eqref{eq:QGFullSystem}, these details are left to the appendix (section \ref{ssec:SingularMeasures}).

\subsection{Periodic Graphs embedded in the plane} \label{ssec:EmbeddedGraphs}
We begin with the definition of a metric graph.
Let $\graph=\bracs{\vertSet,\edgeSet}$ be a directed graph with vertices $v_j\in \vertSet$ and edges $I_{jk}=\bracs{v_j, v_k}\in \edgeSet$, where $I_{jk}$ is directed from vertex $v_j$ (``on the left") to vertex $v_k$ (``on the right").
Strictly we should use the notation $I_{jk}^l$ for the edges, where the superscript $l$ parametrises the edges sharing the endpoints $v_j$ and $v_k$.
In what follows we drop the superscript $l$ to avoid notational clutter.
Each edge $I_{jk}$ is assigned a length $l_{jk}>0$ and we use the same notation $I_{jk}$ for the interval $\sqbracs{0,l_{jk}}$.
We also assign each vertex $v_j$ a ``coupling constant" $\alpha_j\in\complex$ (although in our examples we will only be considering $\alpha_j\in\reals$).
A graph $\graph$ with the metric structure described above is called a \emph{metric graph}.
Metric graphs convey a sense of size or bulk to the structures they describe - the edges between vertices now represent physical ``wires" as opposed to simple combinatorial links as in graph theory.
One can form a quantum graph by equipping a metric graph with a suitable differential operator (and associated vertex- or ``boundary"-conditions) as discussed in section \ref{ssec:FunctionSpaces} and in greater detail in \cite{berkolaiko2013introduction}.
Before discussing operators on metric graphs though, we first outline some notation and conventions that will be in use throughout this work. \newline

We say a (directed) graph $\graph = \bracs{\vertSet,\edgeSet}$ is embedded into $\reals^2$ if each vertex $v_j$ is associated to a point which we also label $v_j\in\reals^2$; and each edge $I_{jk}$ associated to a curve $\gamma_{jk}\subset\reals^2$ with (arc-)length $l_{jk}$, such that there is a continuous map
\begin{align} \label{eq:GeneralCurveParam}
	r: \sqbracs{0,l_{jk}} \rightarrow \gamma_{jk}, \quad r(0) = v_j, \quad r\bracs{l_{jk}} = v_k.
\end{align}
Again, we will drop the distinction between $I_{jk}$ and $\gamma_{jk}$, simply using $I_{jk}$ for both.
Clearly, an embedded graph gives rise to a metric graph, and a metric graph can be used to construct embedded graphs.
The choice of $\reals^2$ is simply to cover the situations we will be considering in this work, there are more general definitions depending on the choice of space to ``embed" the graph into.
For a unit vector $x\in\reals^2$, an embedded graph is said to be $T$-periodic in the direction $x$ if it is invariant under the translation $Tx$ applied to its vertices and edges.
In what follows, we always refer to the minimal such $T>0$ (``period") with this property for a given $x\in\reals^2$.
Note that so long as the graph $\graph$ is periodic in two linearly independent directions, a linear transform can be applied to transform the period cell of $\graph$ into a rectangle with sides parallel to the co-ordinate axes, and so without loss of generality we only consider rectangular period cells henceforth.
If $\graph$ is periodic in the (orthogonal) axial directions $e_1, e_2$ (with periods $T_1, T_2$ respectively), then we can define the ``period cell" or ``unit cell" $\mathcal{P}$ of $\graph$ in the obvious manner: take the intersection of the graph $\graph$ with the region $\mathcal{P} = \sqbracs{0,T_1}\times\sqbracs{0,T_2}$ and match the left boundary to the right, and the top boundary to the bottom.
That is, view the part $\graph_{\mathcal{P}}$ of the graph $\graph$ contained in the region $\mathcal{P}$ as a set on a torus, see \ref{fig:PeriodCellIllustration}.
\begin{figure}[b!]
	\centering
	\begin{subfigure}[t]{0.45\textwidth}
		\centering
		\includegraphics[height=4.5cm]{Diagram_PeriodCellFullLattice.pdf}
		\caption{\label{fig:Diagram_PeriodCellFullLattice} A periodic graph embedded into $\reals^2$, with the period cell marked.}
	\end{subfigure}
	~
	\begin{subfigure}[t]{0.45\textwidth}
		\centering
		\includegraphics[height=4.5cm]{Diagram_PeriodCellEdgeAssociation.pdf}
		\caption{\label{fig:Diagram_PeriodCellEdgeAssociation} The period cell of the graph in \ref{fig:Diagram_PeriodCellFullLattice}, notice how the edges of the period cell are associated.}
	\end{subfigure}
	\\
	\begin{subfigure}[b]{0.75\textwidth}
		\centering
		\includegraphics[scale=	1.0]{Diagram_PeriodCellOnTorus.pdf}
		\caption{\label{fig:Diagram_PeriodCellOnTorus} An illustration of the period cell as a subset of a torus.}
	\end{subfigure}
	\caption{\label{fig:PeriodCellIllustration} Illustrating a periodic cell of a periodic embedded graph.}
\end{figure} 
We will refer to $\graph_{\mathcal{P}} = \graph \cap \mathcal{P}$ as the ``period graph", or ``unit graph" of $\graph$. \newline

As mentioned in section \ref{ssec:OurSystem}, we restrict ourselves to the consideration of graphs with straight edges.
The details given in the appendix (\tstk{ref}) include the case of curved edges being present in the graph, and we show that curved edges give rise to systems of a similar form to \eqref{eq:QGFullSystem} only with the parameters $\qm_{jk}$ being $t$-dependant (so non-constant).
Operators of this kind have been studied before in works such as \tstk{ref: Shterenberg}, in the context of differential operators on non-singular structures in $\reals^2$.
The additional complication (of having $\qm_{jk}$ $t$-dependent) would require an alternative approach to determining the spectrum of \eqref{eq:QGFullSystem} to that which we take here, and we restrict ourselves to considering straight-edges only henceforth.
We summarise our standing assumptions for the rest of this work below:
\begin{assumption} \label{ass:MeasTheoryProblemSetup}
	Let $\graph=\bracs{\vertSet,\edgeSet}$ be the period graph of an embedded graph in $\reals^2$ with period cell $\ddom$, so $\graph\subset\ddom$.
	We restrict ourselves to considering straight-edges between vertices, so each $I_{jk}\in \edgeSet$ is the line segment joining the vertices at either end, with lengths $l_{jk} = \abs{I_{jk}} = \norm{v_j-v_k}_2$.
	Let $e_{jk}$ be the unit vector parallel to $I_{jk}$ and directed from $v_j$ to $v_k$.
	Set
	\begin{align} \label{eq:EdgeParameterisation}
	r_{jk}:\sqbracs{0, l_{jk}} \ni t \mapsto v_j + te_{jk} \in I_{jk},
	\end{align}
	and note that $r_{jk}'(t) = e_{jk}$.
	Finally, let $n_{jk}$ be the unit normal to $I_{jk}$ such that $y_{jk} = \bracs{e_{jk}, n_{jk}}$ can be obtained by an orthonormal rotation $R_{jk}\in\mathrm{SO}(2)$ of the (canonical) axis vectors $x = \bracs{x_1, x_2}$, formally by $x = R_{jk}y_{jk}$.
\end{assumption}

\subsection{Function Spaces} \label{ssec:FunctionSpaces}
Consider the following function spaces
\begin{subequations} \label{eq:GraphFuncSpaces}
	\begin{align}
		L^2\bracs{\graph} := \bigoplus_{I_{jk}\in \edgeSet} \ltwo{I_{jk}}{t},
		&\quad H^1\bracs{\graph} := \bigoplus_{I_{jk}\in \edgeSet} \gradSob{I_{jk}}{t}, \\
		H^2\bracs{\graph} := \bigoplus_{I_{jk}\in \edgeSet} H^2_\mathrm{grad}\bracs{I_{jk}, \md t}. &
	\end{align}
\end{subequations}
We define $u_{jk} = u\vert_{I_{jk}}$ to be the restriction of a function $u$ defined on $\graph$ to the edge $I_{jk}$, extended by zero.
We will also use the shorthand $u_{jk}\bracs{v_j}$ and $u_{jk}\bracs{v_k}$ for the values $u_{jk}\bracs{0}$ and $u_{jk}\bracs{l_{jk}}$ respectively.
If additionally $u$ is continuous at a vertex $v_j$, we will use the notation $u\bracs{v_j}$ for this value.
We also use a notion of ``signed derivative" for our functions when we approach the endpoints of an edge in the graph $\graph$:
\begin{subequations} \label{eq:SignedDerivConvention}
	\begin{align}
		\diff{}{t}u_{jk}\bracs{v_j} &= -u_{jk}'\bracs{v_j} = -\lim_{t\rightarrow0}u_{jk}'(t), \\
		\diff{}{t}u_{jk}\bracs{v_k} &= u_{jk}'\bracs{v_k} = \lim_{t\rightarrow l_{jk}}u_{jk}'(t),
	\end{align}
\end{subequations}
for suitably differentiable functions $u$ on $\graph$. 
One can think of the formulae \eqref{eq:SignedDerivConvention} as defining an ``exterior facing" derivative at each endpoint.
Before discussing differential operators on metric graphs, we introduce some additional notation for convenience: $j\conLeft k$ stands for ``an edge with endpoints $v_j$ and $v_k$, with $v_j$ on the left", and $j\con k$ is used to mean ``an edge with endpoints $v_j$ and $v_k$", whenever the direction of the edge is not important.
We also use $j \conRight k$ alongside $k\conLeft j$.
In other words we have,
\begin{align*}
	j\conLeft k &\Leftrightarrow I_{jk}\in \edgeSet, \\
	j\con k &\Leftrightarrow I_{jk}\in \edgeSet \text{ or } I_{kj}\in \edgeSet.
\end{align*}
We then utilise this notation as shorthand in summations involving edges or vertices; for example, for a fixed $j$
\begin{align*}
	\sum_{j\conLeft k} = \sum_{\substack{k \\ I_{jk}\in \edgeSet}}, 
	&\quad 	\sum_{j\conRight k} = \sum_{\substack{k \\ I_{kj}\in \edgeSet}}.
	&\quad \sum_{j\con k} = \sum_{j\conLeft k} + \sum_{j\conRight k},
\end{align*}
so for example
\begin{align*}
	\sum_{j\con k}u_{jk}\bracs{v_j} &= \sum_{j\conLeft k}u_{jk}\bracs{v_j} + \sum_{j\conRight k}u_{kj}\bracs{v_j}.
\end{align*}
For consistency with the possibility of having multiple edges $I_{jk}^l$ connecting two vertices, then these sums should be interpreted as running over all such edges. \newline

Differential operators on metric graphs are defined by specifying their domains and their action on each edge of the graph $\graph$.
A well-posed problem on a graph requires specifying boundary conditions at the vertices (also referred to as ``vertex conditions"), which are identified with the end-points of the corresponding edges $I_{jk}$.
Specifying these vertex conditions also reflects the connectivity of the graph, which the function spaces \eqref{eq:GraphFuncSpaces} do not convey on their own.
A \emph{quantum graph} is then simply a metric graph equipped with such a differential operator.
There are a variety of choices one can make for the vertex conditions, and we will be interested in imposing continuity of the solution at each vertex, and a Robin-like condition on the derivatives of the solution.
Such conditions are also referred to as ``$\delta-$type " conditions. \tstk{refs to Abelviero et al (exactly solvable) and \cite{berkolaiko2013introduction}}
By way of example, let us demonstrate how we define the graph Laplacian $\mathcal{A}$ on the period graph $\graph_{\mathcal{P}}$.
The domain of $\mathcal{A}$ is defined as
\begin{align*}
	\mathrm{dom}\bracs{\mathcal{A}} &= \clbracs{ u\in H^2\bracs{\graph_{\mathcal{P}}} \ \vert \ \forall j, \ u \text{ is continuous at } v_j, \sum_{j\con k}\diff{u_{jk}}{t}\bracs{v_j} = \alpha_j u\bracs{v_j} },
\end{align*}
and the differential expression on each edge is $-\ddiff{}{t}$.
Then for a function $f\in L^2\bracs{\graph_{\mathcal{P}}}$ we can pose the resolvent problem of finding $u\in\mathrm{dom}\bracs{\mathcal{A}}$ such that
\begin{align*}
	\mathcal{A}u &= f;
\end{align*}
or alternatively can consider the spectral problem of finding eigenpairs $\bracs{\lambda,u}\in\complex\times\mathrm{dom}\bracs{\mathcal{A}}$ such that
\begin{align*}
	\mathcal{A}u &= \lambda u.
\end{align*}
Each of these problems can be rewritten as a system of ODEs on the edge intervals coupled through the vertex conditions:
\begin{align*}
	\mathcal{A}u = f \quad\Leftrightarrow\quad &
	\begin{cases}
		-\ddiff{u_{jk}}{t} = f_{jk} \ \forall j\conLeft k, \\
		\forall j, \ u \text{ is continuous at } v_j, \\
		\forall j, \sum_{j\con k}\diff{u_{jk}}{t}\bracs{v_j} = \alpha_j u\bracs{v_j}.
	\end{cases} \\
	\mathcal{A}u = \lambda u \quad\Leftrightarrow\quad &
	\begin{cases}
		-\ddiff{u_{jk}}{t} = \lambda u_{jk} \ \forall j\conLeft k, \\
		\forall j, \ u \text{ is continuous at } v_j, \\
		\forall j, \sum_{j\con k}\diff{u_{jk}}{t}\bracs{v_j} = \alpha_j u\bracs{v_j}.
	\end{cases}
\end{align*}

\subsection{The $M$-Matrix} \label{ssec:MMatrix}
When tackling spectral problems on quantum graphs, the so-called $M$-matrix will provide us with a useful tool in determining the spectrum, as we discuss in section \ref{sec:SystemDerivation}.
The $M$-matrix is in fact a generalisation of the classical Weyl-Titchmarsh $m$-function, and a particular case of the abstract $M$-operator, as it appears in the theory of boundary triples (more information can be found \tstk{refs for $M$-operator are: Kochubei, Gorbachuk \& Gorbachuk, Brown Marletta Naboko Wood, Kirill Sasha Luis. Classical references include books by Titchmarsh and Atkinson for the $m$-function}).
Here we restrict ourselves to a short review the specific theory that will be relevant to this work.
Let $\mathcal{A}$ be the graph Laplacian, and define the maps
\begin{align*}
	\dmap, \nmap: \mathrm{dom}\bracs{\mathcal{A}} \rightarrow \complex^{\abs{\vertSet}};
\end{align*}
which send a function $u\in\mathrm{dom}\bracs{\mathcal{A}}$, to its Dirichlet and Neumann data at each of the vertices respectively;
\begin{align*}
	\bracs{\dmap u}_j &= u\bracs{v_j}, \quad &j=1,...,\abs{\vertSet}, \\
	\bracs{\nmap u}_j &= \sum_{j\con k}\diff{u_{jk}}{t}\bracs{v_j}, \quad &j=1,...,\abs{\vertSet}. 
\end{align*}
The $M$-matrix is then defined by
\begin{align*}
	\complex^{\abs{\vertSet}} \ni M\bracs{\lambda}\dmap u &= \nmap u \in \complex^{\abs{\vertSet}},
	 &\quad \forall u\in\mathrm{ker}\bracs{\mathcal{A}-\lambda}.
\end{align*}
It can be shown that $\lambda$ is an eigenvalue of $\mathcal{A}$ if and only if
\begin{align*}
	\mathrm{det}\bracs{M\bracs{\lambda} - A}= 0,
\end{align*}
where $A = \mathrm{diag}\bracs{\alpha_1, \alpha_2, ... , \alpha_{\abs{\vertSet}}}$ is the diagonal matrix of the vertex coupling constants.
\tstk{A more general statement can be found in Derkach, Malamund.}
One can see that this effectively reduces a system of ODEs to a matrix eigenvalue problem, at least as far as determining the spectrum is concerned.
We will exploit this fact in section \ref{sec:Examples}.
