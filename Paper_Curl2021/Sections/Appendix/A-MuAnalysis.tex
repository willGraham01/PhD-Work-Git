\section{Appendix: Sobolev functions on the edges of an embedded graph} \label{app:3DMuAnalysis}
In what follows, we describe the sets $\gradZero{\ddom}{\lambda_{jk}}$ and $\curlZero{\ddom}{\ddmes}$; first for when the edge $I_{jk}$ is parallel to the $x_2$-axis, then for when $I_{jk}$ is non-parallel to the axes, and finally use this understanding to understand $\gradZero{\ddom}{\ddmes}$ and $\curlZero{\ddom}{\ddmes}$.
From this understanding, we will then be able to infer various properties of the functions in $\ktgradSob{\ddom}{\ddmes}$ and $\ktcurlSob{\ddom}{\ddmes}$.
Throughout this section, we will write $\grad^{(0)}$ for the operator $\ktgrad$ with $\kt=\bracs{0,0}$.
Since we have proposition \ref{prop:ZeroInvariantUnderQM-Wavenumber}, it is sufficient for us to only consider this operator when determining the form of gradients (or curls) of zero with respect to $\ddmes$.
\tstk{Throughout this section, assumption \ref{ass:MeasTheoryProblemSetup} is adopted?? Ideally yes, but this assumption has $y=\bracs{e,n}$ and here we always set $y=\bracs{n,e}$ in our examples - so I guess it's time to reconcile the two! This will trim down some of the hypotheses of the results which follow}

\subsection{Gradients of zero with respect to $\ddmes$} \label{apps:3DGradientsOfZero}
We begin by examining the set of gradients of zero on a graph consisting of a single edge parallel to the $x_2$ axis.
\tstk{this is similar to when we just had regular gradients... in fact it's identical basically...}

\begin{prop} \label{prop:3DGradZeroSegment}
	Let $I$ be a segment in $\ddom$ parallel to the $x_2$ axis, with singular measure $\lambda_I$.
	Then 
	\begin{align*}
		\gradZero{\ddom}{\lambda_I} &= \clbracs{ \bracs{g_1, 0, 0}^{\top} \setVert g_1\in\ltwo{\ddom}{\lambda_I} }.
	\end{align*}
\end{prop}
\begin{proof}
	Without loss of generality we assume $x_1=0$ on $I$.
	If $\bracs{0,0,g_3}^\top\in\gradZero{\ddom}{\ddmes}$, we can find an approximating sequence $\phi_n$ as in \eqref{eq:GradZeroSequenceDef}, and since $\phi_n\rightarrow 0$ we have that $\rmi\wavenumber\phi_n\rightarrow0$ and hence $g_3=0$.
	
	Now suppose that $g_1\in\smooth{\ddom}$.
	Set $\phi(x) = x_1 g_1(x)$, so that 
	\begin{align*}
		\grad^{(0)}\phi(x) = \bracs{g_1(x) + x_1 \partial_1 g_1(x), x_1 \partial_2 g_1(x), \rmi\wavenumber\phi(x)}^\top.
	\end{align*}
	Then $\phi = 0$ on $I$, and $\grad^{(0)}\phi = \bracs{g_1, 0, 0}^\top$ on $I$, so the constant sequence $\phi$ serves as the approximating sequence in \eqref{eq:GradZeroSequenceDef}, and thus $\bracs{g_1, 0, 0}^{\top}\in\gradZero{\ddom}{\lambda_I}$.
	Given the definition of $\gradZero{\ddom}{\lambda_I}$, we conclude by a density that $\bracs{g_1, 0, 0}^{\top}\in\gradZero{\ddom}{\lambda_I}$ for $g_1\in\ltwo{\ddom}{\lambda_I}$.
	
	Finally, suppose that $\bracs{0, g_2, 0}\in\gradZero{\ddom}{\lambda_I}$ and let $\phi_n$ be an approximating sequence as in \eqref{eq:GradZeroSequenceDef}.
	Consider the change of variables $r_I$ (analogous to $r_{jk}$ in \ref{ass:MeasTheoryProblemSetup}) for $I$, and notice that
	\begin{align*}
		\int_0^{\abs{I}} \abs{\phi_n\bracs{r_I(t)}}^2 \ \md t &= \integral{\ddom}{ \abs{\phi_n}^2 }{\lambda_I} \rightarrow 0, \\
		\int_0^{\abs{I}} \abs{ \phi_n'\bracs{r_I(t)} - g_2\bracs{r_I(t)} }^2 \ \md t &= \integral{\ddom}{ \abs{\grad^{(0)}\phi_n - g_2}^2 }{\lambda_I} \rightarrow 0.		
	\end{align*}
	Thus, the function $\tilde{\phi}_n = \phi_n\circ r_I \in\smooth{I}$ is such that $\tilde{\phi}_n\rightarrow 0, \tilde{\phi}_n'\rightarrow g_2\circ r_I$ in $\ltwo{\interval{I}}{t}$.
	Therefore, $g_2\circ r_I$ is the derivative of $0$ in the $\gradSob{\interval{I}}{t}$-sense, and so $g_2\circ r_I = 0$, and hence $g_2 = 0$.
\end{proof}

\tstk{remark here about how these gradients are essentially the same as in the 2D (non-$\wavenumber$) case?}

\begin{cory} \label{cory:3DGradSegmentRotated}
	Let $I$ be a segment in $\ddom$ with local orthogonal co-ordinate system  $y=\bracs{y_1,y_2}$ with $y_2$ parallel to $I$, and let $\lambda_I$ be the singular measure supported on $I$.
	Suppose $R\in\mathrm{SO}(2)$ is the change of co-ordinates $x=Ry$ where $x=\bracs{x_1,x_2}$ is the axis co-ordinate system.
	Then
	\begin{align*}
		\gradZero{\ddom}{\lambda_I} &= 
		\clbracs{ \begin{pmatrix} R^\top & 0 \\ 0 & 1 \end{pmatrix} \begin{pmatrix} g_1 \\ 0 \\ 0 \end{pmatrix} }
	\end{align*}
\end{cory}
\begin{proof}
	This is simply an application of the rotation $R$ to the result of proposition \ref{prop:3DGradZeroSegment}.
\end{proof}
Note that the result of corollary \ref{cory:3DGradSegmentRotated} also implies that
\begin{align*}
	\gradZero{\ddom}{\lambda_I} &= 
		\clbracs{ \begin{pmatrix} g_1 \\ g_2 \\ 0 \end{pmatrix}\in\ltwo{\ddom}{\lambda_I}^3 \setVert \bracs{g_1, g_2}^\top e_I = 0 },
\end{align*}
where $e_I$ is the unit (tangent) vector to $I$.
We now have the following characterisation for gradients of zero on a graph $\graph$.

\tstk{what "name" does this deserve - theorem? prop? cory? It's not revolutionary given the scalar work}
\begin{prop} \label{prop:3DGradZeroCharacterisation}
	Let $\graph=\bracs{\vertSet, \edgeSet}$ be a graph embedded into $\ddom$, and $\ddmes$ the singular measure on $\graph$.
	Assume convention \ref{ass:MeasTheoryProblemSetup}.
	Then
	\begin{align*}
		\gradZero{\ddom}{\ddmes} &= \clbracs{ g\in\ltwo{\ddom}{\ddmes}^3 \setVert g\in\gradZero{\ddom}{\lambda_{jk}} \ \forall I_{jk}\in\edgeSet }
	\end{align*}
\end{prop}
\begin{proof}
	\tstk{this proof is literally the same as that in the scalar case for gradients of zero, save we are carrying a 3rd component $=0$ around with us.
	I can type this out, but here it's probably wisest to point back to the arguments made previously in the scalar case.}
\end{proof}
Now that we have established the form of the functions which live in $\gradZero{\ddom}{\ddmes}$, we can move onto examining the gradients of the functions $u\in\ktgradSob{\ddom}{\ddmes}$.

\subsection{Tangential $\kt$-Gradients with respect to $\ddmes$} \label{apps:3DTangentialGradients}
Whilst $\gradZero{\ddom}{\ddmes}$ is invariant under changes in $\kt$, the gradients of functions in $\ktgradSob{\ddom}{\ddmes}$ are not.
However given proposition \ref{prop:3DGradZeroCharacterisation}, we can understand these gradients on a graph $\graph$ by understanding their form on each edge $I_{jk}$, and so our analysis flows in much the same way as in section \ref{apps:3DGradientsOfZero}.
First, we seek to understand the behaviour of a gradient on a single edge parallel to the $x_2$ axis.

\begin{prop} \label{prop:3DTangGradSegment}
	Let $I$ be a segment in $\ddom$ parallel to the $x_2$ axis, with singular measure $\lambda_I$, and take $r_I$ as in convention \ref{ass:MeasTheoryProblemSetup} for the edge $I$.
	Write $\gradSob{\interval{I}}{t}$ for the ``classical" Sobolev space of functions on $\interval{I}$ with respect to the Lebesgue measure.
	Suppose that $u\in\ktgradSob{\ddom}{\lambda_I}$ and set $\widetilde{u} = u \circ r_I$.
	Then $\widetilde{u}\in\gradSob{\interval{I}}{t}$ and
	\begin{align*}
		\ktgrad_{\lambda_I} u &= 
		\begin{pmatrix} 0 \\ u' + \rmi\qm_2 u \\ \rmi\wavenumber u \end{pmatrix},
	\end{align*}
	where $u'(x) = \bracs{\widetilde{u}'\circ r_I^{-1}}(x)$.
\end{prop}
\begin{proof}
	Write $\ktgrad_{\lambda_I} u = \bracs{v_1, v_2, v_3}^\top$, and let $\phi_n\in\smooth{\ddom}$ be an approximating sequence for $\bracs{u, \ktgrad_{\lambda_I}u}$, so
	\begin{align*}
		\phi_n \lconv{\ltwo{\ddom}{\lambda_I}} u, &\qquad \ktgrad\phi_n \lconv{\ltwo{\ddom}{\lambda_I}^3}\ktgrad_{\lambda_I}u.
	\end{align*}
	The requirement that $\ktgrad_{\lambda_I} u \perp \gradZero{\ddom}{\lambda_I}$ and proposition \ref{prop:3DGradZeroSegment} implies that $v_1=0$, as $v_1$ must be orthogonal to all functions in $\ltwo{\ddom}{\lambda_I}$.
	Since $\rmi\wavenumber\phi_n\lconv{\ltwo{\ddom}{\lambda_I}} v_3$ and $\phi_n\lconv{\ltwo{\ddom}{\lambda_I}}u$, we have that $v_3 = \rmi\wavenumber u$.
	
	This leaves the form of the second component to be determined.
	Set $\widetilde{\phi}_n = \phi_n \circ r_I$, and use the change of variables $r_I$ to obtain
	\begin{align*}
		\int_0^{\abs{I}} \abs{ \widetilde{\phi}_n - \widetilde{u} }^2 \ \md t 
		&= \integral{\ddom}{ \abs{ \phi_n - u }^2 }{\lambda_I} \rightarrow 0, \\
		\int_0^{\abs{I}} \abs{ \widetilde{\phi}'_n + \rmi\qm_2\widetilde{\phi}_n - \widetilde{v}_2 }^2 \ \md t 
		&= \integral{\ddom}{ \abs{ \partial_2\phi_n + \rmi\qm_2\phi_n - v_2 }^2 }{\lambda_I} \rightarrow 0,
	\end{align*}
	where $\widetilde{v}_2 = v_2\circ r_I$.
	Thus, $\widetilde{\phi}_n\in\smooth{\interval{I}}$ is such that
	\begin{align*}
		\widetilde{\phi}_n \rightarrow \widetilde{u}, \quad
		\widetilde{\phi}'_n - \rmi\qm_2\widetilde{\phi}_n \rightarrow \widetilde{v}_2, \quad
		\text{in } \ltwo{\interval{I}}{t}.
	\end{align*}
	As a result, we have that $\widetilde{\phi}'_n \rightarrow  \widetilde{v}_2 - \rmi\qm_2\widetilde{u}$, and thus $\widetilde{u}' = \widetilde{v}_2 - \rmi\qm_2\widetilde{u}$ in the $\gradSob{\interval{I}}{t}$ sense.
	Undoing the change of variables $r_I$ and rearranging for $v_2$ then gives the desired expression.
\end{proof}

As before when we handled $\gradZero{\ddom}{\lambda_{jk}}$, we can use a rotation argument to determine the form of the tangential gradient on an edge oriented at an angle to the co-ordinate axes.
\begin{cory} \label{cory:3DTangGradRotated}
	Let $I$ be a segment in $\ddom$ with local orthogonal co-ordinate system  $y=\bracs{y_1,y_2}$ with $y_2$ parallel to $I$, and let $\lambda_I$ be the singular measure supported on $I$.
	Suppose $R\in\mathrm{SO}(2)$ is the change of co-ordinates $x=Ry$ where $x=\bracs{x_1,x_2}$ is the axis co-ordinate system, and take $r_I$ as in convention \ref{ass:MeasTheoryProblemSetup} for the edge $I$.
	Write $\gradSob{\interval{I}}{t}$ for the Sobolev space of functions on $\interval{I}$ with respect to the Lebesgue measure.
	Suppose that $u\in\ktgradSob{\ddom}{\lambda_I}$ and set $\widetilde{u} = u \circ r_I$.
	Then $\widetilde{u}\in\gradSob{\interval{I}}{t}$ and
	\begin{align*}
		\ktgrad_{\lambda_I} u &= 
		\begin{pmatrix} R^{\top} & 0 \\ 0 & 1	\end{pmatrix}
		\begin{pmatrix} 0 \\ u' + \rmi\qm_{I}u \\ \rmi\wavenumber u \end{pmatrix},
	\end{align*}
	where $\qm_I = \bracs{R_I\qm}_2$ and $u'(x) = \bracs{\widetilde{u}'\circ r_I^{-1}}(x)$.
\end{cory}

Now that we understand the form of the tangential gradient of a function on an edge of a graph, we can determine the form of the tangential gradient for a function $u\in\ktgradSob{\ddom}{\ddmes}$.
\begin{cory} \label{cory:3DTangGradGraph}
	Let $\graph=\bracs{\vertSet \edgeSet}$ be an embedded graph in $\ddom$ and let $\ddmes$ be the singular measure supported on $\graph$.
	Then if $u\in\ktgradSob{\ddom}{\ddmes}$, then for each $I_{jk}\in\edgeSet$ we have that
	\begin{align*}
		\widetilde{u}_{jk} &:= u_{jk} \circ r_{jk} \in \gradSob{\interval{I_{jk}}}{t}, \\
		\bracs{ \ktgrad_{\ddmes} u }\vert_{I_{jk}} &= 
		\begin{pmatrix} R^{\top}_{jk} & 0 \\ 0 & 1	\end{pmatrix}
		\begin{pmatrix} 0 \\ u_{jk}' + \rmi\qm_{jk}u_{jk} \\ \rmi\wavenumber u_{jk} \end{pmatrix},
	\end{align*}
	where $\qm_{jk} = \bracs{ R_{jk}\qm }_2$ and $u_{jk}' = \widetilde{u}_{jk}' \circ r_{jk}^{-1}$.
\end{cory}
\begin{proof}
	Since $u\in\ktgradSob{\ddom}{\ddmes}$, we have that $\ktgrad_{\ddmes} u \perp \gradZero{\ddom}{\ddmes}$.
	Fix an edge $I_{jk}$, and let $\phi_n$ be an approximating sequence for $u\in\ktgradSob{\ddom}{\ddmes}$.
	Given proposition \ref{prop:3DGradZeroCharacterisation}, for every function $g_{jk}\in\gradZero{\ddom}{\lambda_{jk}}$, the function $g_{\graph}\in\gradZero{\ddom}{\ddmes}$ where
	\begin{align*}
		g_{\graph}(x) = 
		\begin{cases} 
			g_{jk}(x), & x\in I_{jk}, \\
			0, & x\not\in I_{jk}.
		\end{cases}
	\end{align*}
	Therefore, $\ktgrad_{\ddmes} u$ is orthogonal to each $g_{\graph}$ which implies that $\bracs{ \ktgrad_{\ddmes} u }\vert_{I_{jk}} \perp \gradZero{\ddom}{\lambda_{jk}}$.
	In addition, 
	\begin{align*}
		\norm{\phi_n - u}_{\ltwo{\ddom}{\lambda_{jk}}} &\leq \norm{\phi_n - u}_{\ltwo{\ddom}{\ddmes}} \rightarrow 0, \\
		\norm{\ktgrad\phi_n - \ktgrad_{\ddmes} u}_{\ltwo{\ddom}{\lambda_{jk}}^2} &\leq \norm{\ktgrad\phi_n - \ktgrad_{\ddmes} u}_{\ltwo{\ddom}{\ddmes}^2} \rightarrow 0,
	\end{align*}
	and so $\bracs{u, \ktgrad_{\ddmes} u}\in\ktgradSob{\ddom}{\lambda_{jk}}$.
	Application of corollary \ref{cory:3DTangGradRotated} then implies the stated form for $\bracs{ \ktgrad_{\ddmes} u }\vert_{I_{jk}}$.
\end{proof}

Corollary \ref{cory:3DTangGradGraph} provides us with an understanding of the behaviour of the tangential gradient $\ktgrad_{\ddmes}u$.
In addition to this understanding of the tangential gradients, we can also demonstrate that functions $u\in\ktgradSob{\ddom}{\ddmes}$ are also continuous at the vertices $v_j$.
\begin{prop} \label{prop:3DVertexContinuity-Grad}
	\begin{align*}
		\bracs{u, \ktgrad_{\ddmes}u}\in\ktgradSob{\ddom}{\ddmes} \quad\Rightarrow\quad
		&\mathrm{(i)} \bracs{u, \bracs{\ktgrad_{\ddmes} u}\vert_{I_{jk}} }\in\ktgradSob{\ddom}{\lambda_{jk}} \quad \forall I_{jk}\in\edgeSet, \\
		&\mathrm{(ii)} \ u \text{ is continuous at every } v_j\in\vertSet.
	\end{align*}
\end{prop}
\begin{proof}
	The implication of (i) is immediate from proposition \ref{prop:3DTangGradGraph}, whilst the proof of the implication of (ii) is identical to that in the scalar case \tstk{should this proof be inserted anyway? If this document is standalone, it probably should. If this is forming part of the thesis, then no, since we'll have the proof to hand in the previous chapter.}
\end{proof}
\tstk{mention here that Zhikov's work appears to claim that the converse is true, but this hinges on his ``proposition 5.5" which we don't think is true, or if it is seems to go against his chosen setup of his Sobolev spaces.}

\subsection{Curls of Zero with respect to $\ddmes$} \label{apps:CurlsOfZero}
With our description of $\ktgradSob{\ddom}{\ddmes}$ complete, we move onto investigating functions with $\kt$-curls with respect to the measure $\ddmes$.
Our investigation proceeds almost identically to that for gradients; we first examine the set $\curlZero{\ddom}{\lambda_{jk}}$ for an edge $I_{jk}$, and use this to build-up an understanding of $\curlZero{\ddom}{\ddmes}$.
The objective being to establish a result similar to proposition \ref{prop:3DGradZeroCharacterisation} for curls of zero, which is then used to describe the functions that live in $\ktcurlSob{\ddom}{\ddmes}$.
To that end, we begin by examining curls of zero on a segment parallel to the $x_2$-axis.

\begin{prop} \label{prop:CurlZeroSegment}
	Let $I$ be a segment in $\ddom$ parallel to the $x_2$ axis, and let $\lambda_I$ be the singular measure supported on $I$.
	Then
	\begin{align*}
		\curlZero{\ddom}{\lambda_I} &= 
		\clbracs{ \bracs{0, c_2, c_3}^\top \setVert c_2, c_3\in\ltwo{\ddom}{\lambda_I} }.
	\end{align*}
\end{prop}
\begin{proof}
	We first show that if $c = \bracs{c_1, 0, 0}^\top\in\curlZero{\ddom}{\lambda_I}$, then $c_1=0$.
	To this end, take an approximating sequence $\Phi^n = \bracs{\phi^n_1, \phi_2^n, \phi_3^n}$ as in \eqref{eq:CurlZeroSequenceDef}.
	This provides us with the following convergences in $\ltwo{\ddom}{\lambda_I}$;
	\begin{align*}
		\phi_1^n \rightarrow 0, \quad
		\phi_2^n \rightarrow 0, \quad
		\phi_3^n \rightarrow 0, \quad
		\partial_1\phi_3^n \rightarrow 0, \quad
		\partial_2\phi_3^n \rightarrow c_1.
	\end{align*}
	In particular, we have that
	\begin{align*}
		\phi_3^n \lconv{\ltwo{\ddom}{\lambda_I}} 0, \quad
		\grad^{(0)}\phi_3^n \lconv{ \ltwo{\ddom}{\lambda_I}^3 } \bracs{0, c_1, 0}^\top.
	\end{align*}
	Therefore, $\bracs{0, c_1, 0}^\top\in\gradZero{\ddom}{\lambda_I}$ but by proposition \ref{prop:3DGradZeroSegment} we have that $c_1=0$.
	
	Now we demonstrate that the second and third components of $c\in\curlZero{\ddom}{\lambda_I}$ can be any functions in $\ltwo{\ddom}{\lambda_I}$.
	Without loss of generality we assume that $x_1=0$ on $I$, and let $c_2\in\smooth{\ddom}$.
	Consider the function $\Phi = \bracs{0, x_1 c_2, 0 }^\top$, and set $c=\bracs{0,c_2,0}^\top$.
	Then
	\begin{align*}
		\integral{\ddom}{ \abs{\Phi}^2 }{\lambda_I} &= \integral{I}{ x_1^2 c_2^2 }{\lambda_I} = 0, \\
		\integral{\ddom}{ \abs{\grad^{(0)}\wedge\Phi - c}^2 }{\lambda_I} &= \integral{I}{ \abs{ x_1\partial_1 c_2}^2 }{\lambda_I} = 0,
	\end{align*}
	so we have that $c\in\curlZero{\ddom}{\lambda_I}$.
	By density, we have that $c=\bracs{0,c_2,0}^\top\in\curlZero{\ddom}{\lambda_I}$ for any $c_2\in\ltwo{\ddom}{\lambda_I}$.
	Similarly, the function $\Psi = \bracs{0, 0, -x_1 c_3}^\top$ for $c_3\in\smooth{\ddom}$ leads us to the conclusion that $\bracs{0,0,c_3}^\top\in\curlZero{\ddom}{\lambda_I}$, and thus that $c=\bracs{0,0, c_3}^\top\in\curlZero{\ddom}{\lambda_I}$ for any $c_3\in\ltwo{\ddom}{\lambda_I}$, which completes the proof.
\end{proof}

Of course, we can now determine what the set $\curlZero{\ddom}{\lambda_I}$ looks like for an edge $I$ at an arbitrary angle to the co-ordinate axes.
\begin{cory} \label{cory:CurlZeroRotated}
	Suppose $I$ is a segment in $\ddom$ with orthogonal co-ordinate system $y = \bracs{y_1, y_2}$ with $y_2$ parallel to $I$, and let $\lambda_I$ be the singular measure supporting $I$. 
	Let $R\in\mathrm{SO}(2)$ be the change of co-ordinates $x = Ry$, where $x = \bracs{x_1, x_2}$ is the axis co-ordinate system. 
	Finally, let $e_I$ be the unit vector directed along $I$ and $n_I$ the unit normal vector to $I$(directed as $y_1$ is). 
	Then
	\begin{align*}
		\curlZero{\ddom}{\lambda_I} &= 
		\clbracs{ \begin{pmatrix} R & 0 \\ 0 & 1 \end{pmatrix} \begin{pmatrix} 0 \\ c_2 \\ c_3 \end{pmatrix} \setVert c_2,c_3\in\ltwo{\ddom}{\lambda_I} } \\
		&= \clbracs{ \begin{pmatrix} c_1 \\ c_2 \\ c_3 \end{pmatrix}\in\ltwo{\ddom}{\lambda_I}^3 \setVert \begin{pmatrix} c_1 \\ c_2 \end{pmatrix} \cdot n_I = 0 \text{ on } I }
	\end{align*}
\end{cory}
\begin{proof}
	This is another rotation argument applied to the result of proposition \ref{prop:CurlZeroSegment}, in the same vein as that of corollary \ref{cory:3DGradSegmentRotated} to proposition \ref{prop:3DGradZeroSegment} and corollary \ref{cory:3DTangGradRotated} to proposition \ref{prop:3DTangGradSegment}.
\end{proof}

At this point it can be helpful to provide a geometric interpretation of $\ddmes$-curls of zero (and consequently, the tangential curls).
The conventional interpretation of the curl $c$ of a vector field $u$ is that $c(x)$ is the axis of rotation that an (infinitesimally small) spherical body would undergo if placed in the field $u$ at position $x$.
Next recall that (the edges of a) graph $\graph$ induce a set of planes when extruded into the $x_3$ direction, and the operator $\kgrad$ is the result of applying a Fourier transform in $x_3$.
As such, the ``view" of the measure $\ddmes$ is limited to these planes (through the Fourier transform in $x_3$, plus being the singular measure on edges of the graph).
Now consider what the rotation of a spherical body would look like from the perspective of $\ddmes$ --- place the centre of the sphere at a position $x$ in one of these planes, and mark a point $p$ on the intersection between the surface of the sphere and the plane. 
If $c(x)$ is normal to the plane induced by an edge, the measure will see the point $p$ tracing out a (closed, connected) path in the plane, and thus is able to determine that the field $u$ is undergoing a change.
However if $c(x)$ is tangent to the plane, the path of the point $p$ under this rotation leaves the plane (only returning to the plane at two disconnected points) and the measure $\ddmes$ is unable to follow it, so determines that the field $u$ isn't changing.
One can also envision this in terms of the area of the region that the path of $p$ encloses, illustrated in figure \ref{fig:Diagram_CurlZeroPlane}.
\begin{figure}[b!]
	\centering
	\includegraphics[scale=1.0]{Diagram_CurlZeroPlane.pdf}
	\caption{\label{fig:Diagram_CurlZeroPlane} An illustration of how one interprets $\ddmes$-curls. When the curl $c$ is normal to the plane, rotations enclose regions of non-zero $\ddmes$ area, and changes can be ``seen". By contrast, when $c$ is tangent to the plane, rotations enclose regions of zero $\ddmes$ area, and thus are curls of zero.}
\end{figure}
For the path of $p$ under the rotation about $c(x)$ to project into a region of non-zero ($\ddmes$) measure, $c(x)$ cannot lie entirely tangent to the plane.
So similarly to how the measure $\ddmes$ cannot see changes in functions across the edges of $\graph$, which gives rise to the gradients of zero being normal to each edge, $\ddmes$ cannot see rotations taking points ``outside" the planes and thus these form the set of curls of zero.

Naturally, the behaviour of curls of zero on $\graph$ can be characterised by the behaviour on the individual edges, as was the case with gradients.
\begin{theorem} \label{thm:CurlZeroCharacterisation}
	Let $\graph=\bracs{\vertSet, \edgeSet}$ be a graph embedded into $\ddom$, and $\ddmes$ the singular measure supported on $\graph$.
	Assume convention \ref{ass:MeasTheoryProblemSetup}.
	Then
	\begin{align*}
		\curlZero{\ddom}{\ddmes} &= 
		\clbracs{ c\in\ltwo{\ddom}{\ddmes}^3 \setVert c\in\curlZero{\ddom}{\lambda_{jk}}, \ \forall I_{jk}\in\edgeSet }
	\end{align*}
\end{theorem}
For ease we set $B = \clbracs{ c\in\ltwo{\ddom}{\ddmes}^3 \setVert c\in\curlZero{\ddom}{\lambda_{jk}} \ \forall I_{jk}\in\edgeSet }$ in what follows.
Also, given corollary \ref{cory:CurlZeroRotated}, we can write $B$ as
\begin{align*}
	B &= \clbracs{ \bracs{c_1, c_2, c_3}^\top\in\ltwo{\ddom}{\ddmes}^3 \setVert \bracs{c_1(x), c_2(x)}^\top n_{jk}=0, \ \forall x\in I_{jk}, \ \forall I_{jk}\in\edgeSet } \\
	&= \clbracs{ c\in\ltwo{\ddom}{\ddmes}^3 \setVert c\vert_{I_{jk}} = \begin{pmatrix} R_{jk} & 0 \\ 0 & 1 \end{pmatrix} \begin{pmatrix} 0 \\ c_2 \\ c_3 \end{pmatrix}, \ c_2, c_3\in\ltwo{\ddom}{\lambda_{jk}}, \ \forall I_{jk}\in\edgeSet }.
\end{align*}
The method of proof proceeds in much the same way as for gradients of zero, in proving proposition \ref{prop:3DGradZeroCharacterisation} --- the inclusion $\curlZero{\ddom}{\ddmes}\subset B$ is immediate from a simple observation of the norms.
The reverse inclusion requires us to select an approximating sequence for an element of $B$ and use smooth cut-off functions to produce a sequence which is converging in $\ltwo{\ddom}{\ddmes}$.
We break the proof of theorem \ref{thm:CurlZeroCharacterisation} into smaller results \tstk{which should bear similarity to the gradients of zero case!}, and assume the hypothesis of the theorem throughout.

We first prove the easier of the two inclusions.
\begin{lemma} \label{lem:CurlZeroInB}
	\begin{align*}
		\curlZero{\ddom}{\ddmes}\subset B.
	\end{align*}
\end{lemma}
\begin{proof}
	Let $c\in\curlZero{\ddom}{\ddmes}$, and take an approximating sequence $\Phi^n$.
	Since $\norm{ \cdot }_{\ltwo{\ddom}{\lambda_{jk}}^3} \leq \norm{ \cdot }_{\ltwo{\ddom}{\ddmes}^3}$ for every $I_{jk}\in\edgeSet$, we have that
	\begin{align*}
		\norm{ \Phi^n }_{\ltwo{\ddom}{\lambda_{jk}}^3} &\leq \norm{ \Phi^n }_{\ltwo{\ddom}{\ddmes}^3} \rightarrow 0, \\
		\norm{ \grad^{(0)}\wedge\Phi^n - c }_{\ltwo{\ddom}{\lambda_{jk}}^3} & \leq \norm{ \grad^{(0)}\wedge\Phi^n - c }_{\ltwo{\ddom}{\ddmes}^3} \rightarrow 0,
	\end{align*}
	and so $c\in B$.
\end{proof}

Next, we derive an ``extension lemma" for curls of zero, similarly to how such a result was also derived for gradients of zero.
\tstk{actually, we should state the stronger conclusion here that these functions will also be in $\dddmes$ spaces too when extended by zero the the vertices.}
\begin{lemma}[Extension lemma for $\curlZero{\ddom}{\lambda_{jk}}$] \label{lem:CurlZeroExtensionLemma}
	Fix $n\in\naturals$, $I_{jk}\in\edgeSet$, and let 
	\begin{align*}
		I_{jk}^n &= \clbracs{x\in I_{jk} \setVert \mathrm{dist}\bracs{x, \partial_I} \leq \recip{n}}.
	\end{align*}
	Suppose that $c\in\curlZero{\ddom}{\lambda_{jk}}$ with $c=0$ on $I_{jk}\setminus I_{jk}^n$.
	Then
	\begin{align*}
		c\in\curlZero{\ddom}{\ddmes}.
	\end{align*}
\end{lemma}
\begin{proof}
	Let $\Phi^l$ be an approximating sequence for $c$ as in \eqref{eq:CurlZeroSequenceDef}.
	Take $\chi_{jk}^n\in\smooth{\ddom}$ to be a smooth function with the properties
	\begin{align*}
		\chi_{jk}^n(x) &\in\sqbracs{0,1}, \\
		\chi_{jk}^n(x) = 1 \ &\text{whenever} \ \mathrm{dist}\bracs{x, I_{jk}^n} \leq \recip{3n}, \\
		\chi_{jk}^n(x) = 0 \ &\text{whenever} \ \mathrm{dist}\bracs{x, I_{jk}^n} \geq \frac{2}{3n}.
	\end{align*}
	This function is illustrated in figure \ref{fig:Diagram_ChiFunctionI-n}, and we have that $\chi_{jk}^n\rightarrow \charFunc{I_{jk}}$ in $\ltwo{\ddom}{\ddmes}$, and that $\abs{\grad\chi_{jk}^n}\leq cn$ for some constant $c$ independent of $n$.
	\begin{figure}[t]
		\centering
		\includegraphics[scale=1.0]{Diagram_ChiFunctionI-n.pdf}
		\caption{\label{fig:Diagram_ChiFunctionI-n} The function $\chi_{jk}^n$ in the proximity of the edge $I_{jk}$. If another edge lies in the support of $\chi_{jk}^n$, we can apply a scaling to produce a similar function whose support does not intersect any edges other than $I_{jk}$.}
	\end{figure}
	Now consider the sequence $\Psi^l := \chi_{jk}^n \Phi^l$, by construction we have that
	\begin{align*}
		\integral{\ddom}{ \abs{\Psi^l}^2 }{\ddmes} \leq \integral{I_{jk}}{ \abs{\Phi^l}^2 }{\lambda_{jk}} \rightarrow 0 \toInfty{l}.
	\end{align*}
	Since the following identity holds,
	\begin{align*}
		\grad^{(0)}\wedge\Psi^l &= \grad^{(0)}\chi_{jk}^n\wedge\Phi^l + \chi_{jk}^n\grad^{(0)}\wedge\Phi^l,
	\end{align*}
	we can deduce
	\begin{align*}
		\integral{\ddom}{ \abs{ \grad^{(0)}\wedge\Psi^l - c }^2 }{\ddmes}
		&\leq 2\integral{I_{jk}}{ \abs{ \chi_{jk}^n\grad^{(0)}\wedge\Phi^l - c }^2 }{\lambda_{jk}}
		+ 2\sup\abs{\grad^{(0)}\chi_{jk}^n}^2 \integral{I_{jk}}{ \abs{ \Phi^l }^2 }{\lambda_{jk}} \\
		&\leq2\integral{I_{jk}}{ \abs{ \grad^{(0)}\wedge\Phi^l - c }^2 }{\lambda_{jk}}
		+ 2(cn)^2 \integral{I_{jk}}{ \abs{ \Phi^l }^2 }{\lambda_{jk}} \\
		&\rightarrow 0+0 = 0 \toInfty{l}.
	\end{align*}
	The sequence $\Psi^l$ now serves as the required approximating sequence, giving us that $c\in\curlZero{\ddom}{\ddmes}$.
\end{proof}
\tstk{again, mention here that we've actually proved that $c\in\curlZero{\ddom}{\dddmes}$ since $\Psi^l=0$ and $\grad^{(0)}\wedge\Psi^l=0$ at any vertices.}

This extension lemma \ref{lem:CurlZeroExtensionLemma} now allows us to demonstrate the inclusion $B\subset\curlZero{\ddom}{\ddmes}$.
\begin{lemma} \label{lem:BInCurlZero}
	\begin{align*}
		B \subset \curlZero{\ddom}{\ddmes}.
	\end{align*}
\end{lemma}
\begin{proof}
	First let $\eta\in\smooth{\ddom}$ be such that
	\begin{align*}
		\eta(x) &\in\sqbracs{0,1}, \\
		\eta(x) = 0 \ &\text{whenever } \abs{x} \leq 1, \\
		\eta(x) = 1 \ &\text{whenever } \abs{x} \geq 2,
	\end{align*}
	and for each $v_j\in\vertSet$ and $n\in\naturals$ set
	\begin{align*}
		\eta_j(x) = \eta\bracs{x - v_j} &\qquad \eta_j^n(x) = \eta_j\bracs{nx}.
	\end{align*}
	It can be seen that $\eta_j^n\rightarrow 1$ in $\ltwo{\ddom}{\ddmes}$ since
	\begin{align*}
		\integral{\ddom}{ \abs{\eta_j^n - 1}^2 }{\ddmes} 
		&\leq \integral{ B_{\recip{2}{n}}(v_j)}{ }{\ddmes} \leq \frac{4\abs{\edgeSet}}{n} \rightarrow 0.
	\end{align*}
	Take $c\in B$, and define a family of functions $c_n$ by
	\begin{align*}
		c_n &= \sum_{v_j\in\vertSet}\sum_{j\conLeft k} \eta_j^n \eta_k^n c_{jk}.
	\end{align*}
	For each $n\in\naturals$ and $j\conLeft k$, the function $\eta_j^n \eta_k^n c_{jk}$ satisfies the hypothesis of the extension lemma \ref{lem:CurlZeroExtensionLemma}, and so is an element of $\curlZero{\ddom}{\ddmes}$.
	As $\curlZero{\ddom}{\ddmes}$ is a linear subspace, $c_n\in\curlZero{\ddom}{\ddmes}$ for every $n\in\naturals$.
	Since $\eta_j^n\rightarrow 1$ in $\ltwo{\ddom}{\ddmes}$, we conclude that $c_n\rightarrow c$ in $\ltwo{\ddom}{\ddmes}^3$, and since $\curlZero{\ddom}{\ddmes}$ is closed, we have that $c\in\curlZero{\ddom}{\ddmes}$.
\end{proof}
Lemmas \ref{lem:CurlZeroInB}, \ref{lem:CurlZeroExtensionLemma}, and \ref{lem:BInCurlZero} then form the proof of theorem \ref{thm:CurlZeroCharacterisation}.

\subsection{Tangential $\kt$-Curls with respect to $\ddmes$} \label{apps:TangentialCurls}
With curls of zero now understood, we look to understand $\kt$-tangential curls with respect to $\ddmes$ in much the same way as we did with tangential gradients --- with theorem \ref{thm:CurlZeroCharacterisation} allowing us to obtain an edge-wise characterisation.
As usual, we begin by studying tangential curls on a segment aligned to the $x_2$-axis.

\begin{prop} \label{prop:TangCurlsSegment}
	Let $I$ be a segment in $\ddom$ parallel to the $x_2$ axis, with singular measure $\lambda_I$, and take $r_I$ as in convention \ref{ass:MeasTheoryProblemSetup} for the edge $I$.
	Write $\gradSob{\interval{I}}{t}$ for the ``classical" Sobolev space of functions on $\interval{I}$ with respect to the Lebesgue measure.
	Suppose that $u=\bracs{u_1, u_2, u_3}^\top\in\ktgradSob{\ddom}{\lambda_I}$ and set $\widetilde{u}_j = u_j \circ r_I$ for $j\in\clbracs{1,2,3}$.
	Then $\widetilde{u}_3\in\gradSob{\interval{I}}{t}$ and
	\begin{align*}
		\ktcurl{\lambda_I} u &= 
		\begin{pmatrix} u_3' + \rmi\qm_2 u_3 - \rmi\wavenumber u_2 \\ 0 \\ 0 \end{pmatrix},
	\end{align*}
	where $u_3'(x) = \bracs{\widetilde{u}'_3\circ r_I^{-1}}(x)$.
\end{prop}
\begin{proof}
	Write $\ktcurl{\lambda_I}u = \bracs{v_1, v_2, v_3}^\top$.
	As we know that $\ktcurl{\lambda_I}u \perp \curlZero{\ddom}{\lambda_I}$, given theorem \ref{thm:CurlZeroCharacterisation} we conclude that $v_2 = v_3 = 0$.
	Now take an approximating sequence $\Phi^n = \bracs{\phi^n_1, \phi^n_2, \phi^n_3}^\top$ for $u$, so we have that
	\begin{align*}
		\Phi^n \lconv{ \ltwo{\ddom}{\lambda_I}^3 } u, \qquad \ktcurl{}\Phi^n \lconv{ \ltwo{\ddom}{\lambda_I}^3 } \bracs{v_1, v_2, v_3}^\top.
	\end{align*}
	Using an overhead tilde to denote composition with $r_I$, we notice that in particular,
	\begin{align*}
		\int_0^{\abs{I}} \abs{\widetilde{\phi}^n_j - \widetilde{u}_j}^2 \ \md t
		&= \integral{\ddom}{ \abs{\phi^n_j - u_j}^2 }{\lambda_I} 
		\rightarrow 0, \\
		\int_0^{\abs{I}} \abs{ \bracs{\widetilde{\phi}^n_3}' + \rmi\qm_2\widetilde{\phi}^n_3 - \rmi\wavenumber\widetilde{\phi}^n_2 - \widetilde{v}_1 }^2 \ \md t
		&= \integral{\ddom}{ \abs{ \partial_2\phi^n_3 + \rmi\qm_2\phi^n_3 - \rmi\wavenumber\phi^n_2 - v_1 }^2 }{\lambda_I}
		\rightarrow 0,
	\end{align*}
	for $j\in\clbracs{2,3}$.
	This provides us with the following convergences in $\ltwo{\interval{I}}{t}$;
	\begin{align*}
		\widetilde{\phi}^l_2 \rightarrow \widetilde{u}_2, \quad
		\widetilde{\phi}^l_3 \rightarrow \widetilde{u}_3, \quad
		\bracs{\widetilde{\phi}^n_3}' + \rmi\qm_2\widetilde{\phi}^n_3 - \rmi\wavenumber\widetilde{\phi}^n_2 \rightarrow \widetilde{v}_1.
	\end{align*}
	By the algebra of limits we have that
	\begin{align*}
		\bracs{\widetilde{\phi}^n_3}' \rightarrow -\rmi\qm_2\widetilde{u}_3 + \rmi\wavenumber\widetilde{u}_2 + \widetilde{v}_1,
	\end{align*}
	in $\ltwo{\interval{I}}{t}$ and thus $\widetilde{u}_3\in\gradSob{\interval{I}}{t}$ with $\widetilde{u}_3' = -\rmi\qm_2\widetilde{u}_3 + \rmi\wavenumber\widetilde{u}_2 + \widetilde{v}_1$.
	Rearranging and undoing the change of variables $r_I$ then gives the desired result.
\end{proof}

As we expect from our interpretation of the $\ddmes$-curl (see section \ref{apps:CurlsOfZero}), the $\kt$-tangential curl is directed normal to the segment $I$.
We can derive the form of the tangential curl on an edge at an arbitrary angle to the segment simply by applying a rotation to the result of proposition \ref{prop:TangCurlsSegment}.
\tstk{need to check the form of this! Seems REALLY cumbersome in the way it's currently written... there's got to be a better way! Also, check the quasi-momentum formula on the edge, and the $U$ form - have a feeling this should be $R^\top$ rather than just $R$ here...}
\begin{cory} \label{cory:TangCurlRotated}
	Let $I$ be a segment in $\ddom$ with singular measure $\lambda_I$, and local coordinate system $y=\bracs{y_1, y_2}$ with $y_2$ parallel to $I$.
	Suppose $R\in\mathrm{SO}(2)$ is the rotation $x=Ry$ where $x=\bracs{x_1, x_2}$ is the axis coordinate system.
	Write $r_I$ for the usual parametrisation of $I$ (convention \ref{ass:MeasTheoryProblemSetup}), and let $u=\bracs{u_1,u_2,u_3}\in\ktcurlSob{\ddom}{\lambda_I}$.
	Set
	\begin{align*}
		U(x) = R \begin{pmatrix} u_1(x) \\ u_2(x) \end{pmatrix}, 
		\qquad \qm_I = \bracs{ R\qm }_2,
		\qquad \tilde{u}_3 = u_3 \circ r_I.
	\end{align*}
	Then $\widetilde{u}_3\in\ktgradSob{\ddom}{\lambda_I}$ and 
	\begin{align*}
		\ktcurl{\lambda_I}u &= \begin{pmatrix} R & 0 \\ 0 & 1	\end{pmatrix}
		\begin{pmatrix} u_3' + \rmi\qm_I u_3 - \rmi\wavenumber U_2 \\ 0 \\ 0 \end{pmatrix},
	\end{align*}
	where $u_3' = \widetilde{u}'_3 \circ r_I^{-1}$.
\end{cory}

\tstk{Again, might have $\top$'s everywhere once work through the above results}
\begin{theorem} \label{thm:TangCurlGraph}
	Let $\graph=\bracs{\vertSet, \edgeSet}$ be a graph embedded into $\ddom$ with singular measure $\ddmes$.
	Give each $I_{jk}\in\edgeSet$ local coordinate system $y_{jk}=\bracs{y_{1, jk},y_{2, jk}}$ with $y_{2,jk}$ parallel to $I_{jk}$, and let $R_{jk}\in\mathrm{SO}(2)$ be the change of coordinates $x=R_{jk}y_{jk}$ where $x=\bracs{x_1, x_2}$ is the axis coordinate system.
	Write $r_{jk}$ for the usual parametrisation of the edge $I_{jk}$, and suppose $u=\bracs{u_1, u_2, u_3}^\top\in\ktcurlSob{\ddom}{\ddmes}$, and write $u_{jk}=\bracs{u_{1,jk}, u_{2,jk}, u_{3,jk}}^\top$ for the restriction of $u$ to the edge $I_{jk}$.
	Also define
	\begin{align*}
		U_{jk}(x) = R_{jk}\begin{pmatrix} u_{1, jk}(x) \\ u_{2, jk}(x) \end{pmatrix},
		\qquad \qm_{jk} = \bracs{ R_{jk}\qm }_2,
		\qquad \widetilde{u}_{3,jk} = u_{3,jk} \circ r_{jk},
	\end{align*}
	for each edge $I_{jk}$.
	Then for every $I_{jk}\in\edgeSet$, we have that $\widetilde{u}_{3,jk}\in\ktgradSob{\ddom}{\lambda_{jk}}$ and
	\begin{align*}
		\bracs{ \ktcurl{\ddmes} u }\big\vert_{I_{jk}} &= 
		\begin{pmatrix} R_{jk} & 0 \\ 0 & 1 \end{pmatrix}
		\begin{pmatrix} u_{3,jk}' + \rmi\qm_{jk} u_{3,jk} - \rmi\wavenumber U_{2,jk} \\ 0 \\ 0 \end{pmatrix}
	\end{align*}
	where $u_{3,jk}' = \widetilde{u}_{3,jk}' \circ r_{jk}^{-1}$.
\end{theorem}
\begin{proof}
	Given theorem \ref{thm:CurlZeroCharacterisation}, we know that $\ktcurl{\ddmes}u$ must have the same form as $\ktcurl{\lambda_{jk}}u$ on the edge $I_{jk}$.
	Thus by using corollary \ref{cory:TangCurlRotated}, the result follows.
\end{proof}

This concludes our investigation into the form of the $kt$-tangential curl, however we can infer some further properties about the component $u_3$ of a vector field $u\in\ktgradSob{\ddom}{\ddmes}$.
\begin{prop} \label{prop:CurlImpliesThirdCompGradient}
	We have that
	\begin{align*}
		u:=\bracs{u_1,u_2,u_3}^\top\in\ktcurlSob{\ddom}{\ddmes}
		\quad\Leftrightarrow\quad &
		u_3\in\ktgradSob{\ddom}{\ddmes}.
	\end{align*}
\end{prop}
\begin{proof}
	($\Leftarrow$) Let $u_3\in\ktgradSob{\ddom}{\ddmes}$, take an approximating sequence $\phi_n$ for $u_3$, and define $\Phi^n=\bracs{0,0,\phi_n}^\top$.
	Then as
	\begin{align*}
		\phi_n \lconv{\ltwo{\ddom}{\ddmes}} u_3, \quad \ktgrad\phi_n \lconv{\ltwo{\ddom}{\ddmes}^2} \ktgrad_{\ddmes}u,
	\end{align*}
	clearly
	\begin{align*}
		\Phi^n = \begin{pmatrix} 0 \\ 0 \\ \phi_n \end{pmatrix} \lconv{\ltwo{\ddom}{\ddmes}^3} u, \quad
		\ktcurl{}\Phi^n = \begin{pmatrix} \partial_2\phi_n + \rmi\qm_2\phi_n \\ -\partial_1\phi_n - \rmi\qm_1\phi_n \\ 0 \end{pmatrix} 
		\lconv{\ltwo{\ddom}{\ddmes}^3} \begin{pmatrix} \bracs{\ktgrad_{\ddmes}u_3}_2 \\ -\bracs{\ktgrad_{\ddmes}u_3}_1 \\ 0	\end{pmatrix} =: c.
	\end{align*}
	\tstk{could try to prove that this sequence really is converging to the tangential gradient here, but it's not necessary.}
	Thus, $\bracs{u, c}\in W^{\kt}_{\mathrm{curl}}$ and hence $u$ possess a $\kt$-tangential curl, so $u\in\ktcurlSob{\ddom}{\ddmes}$.
	
	($\Rightarrow$) Now suppose that $u=\bracs{0,0,u_3}^\top\in\ktcurlSob{\ddom}{\ddmes}$, and again take an approximating sequence $\Phi^n = \bracs{\phi^n_1, \phi^n_2, \phi^n_3}^\top$ for $u$.
	Then it can be shown that
	\begin{align*}
		\phi^n_3 \lconv{\ltwo{\ddom}{\ddmes}} u_3
		\quad \ktgrad\phi^n_3 \lconv{\ltwo{\ddom}{\ddmes}^3} \begin{pmatrix} -\bracs{ \ktcurl{\ddmes}u}_2 \\ \bracs{ \ktcurl{\ddmes}u}_1 \\ 0 \end{pmatrix} =: g,
	\end{align*}
	which implies that $\bracs{u_3,g}\in W^{\kt}_{\mathrm{grad}}$ and hence that $u_3$ possesses a $\kt$-tangential gradient, and thus $u_3\in\ktgradSob{\ddom}{\ddmes}$.
\end{proof}
In particular, note that since functions in $\ktgradSob{\ddom}{\ddmes}$ are continuous at the vertices of $\graph$, the component $u_3$ of a vector field $u\in\ktcurlSob{\ddom}{\ddmes}$ is thus continuous at the vertices.

\subsection{The Divergence-Free Condition} \label{apps:DivFreeEdge}
Having established the behaviour of (the curl of) functions $u\in\ktcurlSob{\ddom}{\ddmes}$, and knowing what $\kt$-gradients with respect to $\ddmes$ look like through proposition \ref{prop:3DGradZeroCharacterisation} and corollary \ref{cory:3DTangGradGraph}, we can examine what it means for a vector field to be divergence free.
Again, we begin with an examination of the divergence-free condition on an individual edge, and build up to the study of this condition on the whole graph.

\begin{prop} \label{prop:DivFreeSegment}
	Let $I$ be a segment in $\ddom$ parallel to the $x_2$ axis, with singular measure $\lambda_I$, and take $r_I$ as in convention \ref{ass:MeasTheoryProblemSetup} for the edge $I$.
	Suppose that $u\bracs{u_1, u_2, u_3}^\top\in\ktcurlSob{\ddom}{\lambda_I}$.
	Then
	\begin{align*}
		u \text{ is divergence free } \quad\Leftrightarrow\quad 
		\begin{cases}
			\mathrm{(i)} & u_1 = 0, \\
			\mathrm{(ii)} & \widetilde{u}_2 := u_2 \circ r_I \in\gradSob{\interval{I}}{t}, \\
			\mathrm{(iii)} & u_2' + \rmi\qm_2 u_2 + \rmi\wavenumber u_3 = 0 \text{ on } I, \\
			\mathrm{(iv)} & u_2\vert_{\partial I} = 0,
		\end{cases}
	\end{align*}
	where $u_2' = \widetilde{u}_2'\circ r_I^{-1}$.
\end{prop}
\begin{proof}
	($\Rightarrow$) Suppose $u$ is divergence free.
	Since we require that
	\begin{align*}
		0 &= \integral{I}{ u\cdot \overline{g}}{\lambda_I},
	\end{align*}
	for every $g\in\gradZero{\ddom}{\lambda_I}$ and have proposition \ref{prop:3DGradZeroSegment}, we can conclude that $u_1=0$.
	Next, let $\varphi\in\smooth{\ddom}$ be a smooth function with support that only intersects the interior of $I$.
	Then,
	\begin{align*}
		0 &= \integral{I}{ u\cdot\overline{\ktgrad_{\lambda_I}\varphi} }{\lambda_I}
		= \integral{I}{ u_2\overline{\varphi}' - \rmi\qm_2 u_2\overline{\varphi} - \rmi\wavenumber u_3\overline{\varphi} }{\lambda_I} \\
		&= \int_0^{\abs{I}} \widetilde{u}_2\overline{\widetilde{\varphi}}' - \rmi\qm_2\widetilde{u}_2\overline{\widetilde{\varphi}} - \rmi\wavenumber\widetilde{u}_3\overline{\widetilde{\varphi}} \ \md t,
	\end{align*}
	where we have used overhead tildes to denote composition with $r_I$.
	Upon rearranging, we find that
	\begin{align*}
		\int_0^{\abs{I}} \widetilde{u}_2\overline{\widetilde{\varphi}}' \ \md t
		&= \int_0^{\abs{I}} \bracs{ \rmi\qm_2\widetilde{u}_2 + \rmi\wavenumber\widetilde{u}_3 }\overline{\widetilde{\varphi}} \ \md t,
	\end{align*}
	which holds for every smooth $\widetilde{\varphi}$ with compact support in the interior of $\interval{I}$.
	Since both of $\widetilde{u}_2, \widetilde{u}_3\in\ltwo{\interval{I}}{t}$, the function $\rmi\qm_2\widetilde{u}_2 + \rmi\wavenumber\widetilde{u}_3\in\ltwo{\interval{I}}{t}$ too.
	Thus, we have that $\widetilde{u}_2\in\gradSob{\interval{I}}{t}$ (giving (ii)) with 
	\begin{align*}
		\widetilde{u}'_2 = - \rmi\qm_2\widetilde{u}_2\overline{\widetilde{\varphi}} - \rmi\wavenumber\widetilde{u}_3\overline{\widetilde{\varphi}},
	\end{align*}
	which provides (iii) upon undoing the change of variables $r_I$.
	Finally, let $\varphi\in\smooth{\ddom}$ have support that intersects only one of the two vertices at the endpoints of $I$, and label this vertex $v_I$.
	Without loss of generality, we assume that $r_I(0)=v_I$ (the case $r_I\bracs{\abs{I}}=v_I$ is similar).
	Retaining the notational conventions from before, we have that $\widetilde{u}_2\widetilde{\varphi}\in\gradSob{\interval{I}}{t}$ and thus
	\begin{align*}
		0 
		&= \int_0^{\abs{I}} \widetilde{u}_2\overline{\widetilde{\varphi}}' - \rmi\qm_2\widetilde{u}_2\overline{\widetilde{\varphi}} - \rmi\wavenumber\widetilde{u}_3\overline{\widetilde{\varphi}} \ \md t \\
		&= -\widetilde{u}_2(0)\overline{\widetilde{\varphi}}(0)
		- \int_0^{\abs{I}} \bracs{ \widetilde{u}'_2 + \rmi\qm_2\widetilde{u}_2 + \rmi\wavenumber\widetilde{u}_3 }\overline{\widetilde{\varphi}} \ \md t \\
		&= -\widetilde{u}_2(0)\overline{\widetilde{\varphi}}(0).
	\end{align*}
	Since this holds for all such $\varphi$, we must conclude that $\widetilde{u}_2(0)=0$ which provides (iv).
	
	($\Leftarrow$) For the reverse implication, the conditions (i) ensures that $u$ is orthogonal to all gradients of zero.
	Conditions (ii)-(iv) imply that $0 = \integral{I}{ u\cdot\overline{\ktgrad_{\lambda_I}\varphi} }{\lambda_I}$ for all smooth functions $\varphi\in\smooth{\ddom}$, and so by density imply that
	\begin{align*}
		0 = \integral{I}{ u\cdot\overline{\ktgrad_{\lambda_I}v} }{\lambda_I} \quad\forall v\in\ktgradSob{\ddom}{\lambda_I},
	\end{align*}
	which shows that $u$ is orthogonal to all tangential gradients, completing the proof.
\end{proof}

\tstk{again, this might need checking. It'll be the same form as whatever the previous proposition comes out to be, but there might be $R^\top$ rather than $R$s everywhere, and the method of proof might change.}
\begin{cory} \label{cory:DivFreeRotated}
	Let $I$ be a segment in $\ddom$ with singular measure $\lambda_I$, and local coordinate system $y=\bracs{y_1, y_2}$ with $y_2$ parallel to $I$.
	Suppose $R\in\mathrm{SO}(2)$ is the rotation $x=Ry$ where $x=\bracs{x_1, x_2}$ is the axis coordinate system.
	Write $r_I$ for the usual parametrisation of $I$ (convention \ref{ass:MeasTheoryProblemSetup}), and let $u=\bracs{u_1,u_2,u_3}\in\ktcurlSob{\ddom}{\lambda_I}$.
	Set
	\begin{align*}
		U(x) = R \begin{pmatrix} u_1(x) \\ u_2(x) \end{pmatrix}, 
		\qquad \qm_I = \bracs{ R\qm }_2,
		\qquad \tilde{U} = U \circ r_I.
	\end{align*}
	Then
	\begin{align*}
		u \text{ is divergence free } \quad\Leftrightarrow\quad
		\begin{cases}
			\mathrm{(i)} & U_1 = 0, \\
			\mathrm{(ii)} & \widetilde{U}_2 \in\gradSob{\interval{I}}{t}, \\
			\mathrm{(iii)} & U_2' + \rmi\qm_I U_2 + \rmi\wavenumber U_3 = 0 \text{ on } I, \\
			\mathrm{(iv)} & U_2\vert_{\partial I} = 0,
		\end{cases}
	\end{align*}
	where $U_2' = \widetilde{U}_2'\circ r_I^{-1}$.
\end{cory}
\begin{proof}
	The proof is just a repeat of that in proposition \ref{prop:DivFreeSegment}; the segment now being at an angle to the coordinate axes resulting in different forms of the tangential gradient and gradients of zero, in turn affecting the divergence free condition.
\end{proof}

Characterising the $\kt$-divergence free condition on a graph can then be done as follows.
Note that the condition (iv) differs from its counterparts in the single-segment cases of proposition \ref{prop:DivFreeSegment} and corollary \ref{cory:DivFreeRotated} in that it involves the sum of function values from the edges totalling zero at the vertices, rather than imposing that the function $U$ be zero.
\begin{theorem} \label{thm:DivFreeEdgesCharacterisation}
	Let $\graph=\bracs{\vertSet, \edgeSet}$ be a graph embedded into $\ddom$ with singular measure $\ddmes$.
	Give each $I_{jk}\in\edgeSet$ local coordinate system $y_{jk}=\bracs{y_{1, jk},y_{2, jk}}$ with $y_{2,jk}$ parallel to $I_{jk}$, and let $R_{jk}\in\mathrm{SO}(2)$ be the change of coordinates $x=R_{jk}y_{jk}$ where $x=\bracs{x_1, x_2}$ is the axis coordinate system.
	Write $r_{jk}$ for the usual parametrisation of the edge $I_{jk}$, and suppose $u=\bracs{u_1, u_2, u_3}^\top\in\ktcurlSob{\ddom}{\ddmes}$, and write $u_{jk}=\bracs{u_{1,jk}, u_{2,jk}, u_{3,jk}}^\top$ for the restriction of $u$ to the edge $I_{jk}$.
	Also define
	\begin{align*}
		U_{jk}(x) = R_{jk}\begin{pmatrix} u_{1, jk}(x) \\ u_{2, jk}(x) \end{pmatrix},
		\qquad \qm_{jk} = \bracs{ R_{jk}\qm }_2,
		\qquad \widetilde{U}_{jk} = U_{jk} \circ r_{jk},
	\end{align*}
	for each edge $I_{jk}$.
	Then
	\begin{align*}
		u \text{ is divergence free } \quad\Leftrightarrow\quad
		\begin{cases}
			\mathrm{(i)} & U_{1, jk} = 0, \ \forall I_{jk}\in\edgeSet, \\
			\mathrm{(ii)} & \widetilde{U}_{2,jk} \in\gradSob{\ddom}{\lambda_{jk}}, \ \forall I_{jk}\in\edgeSet, \\
			\mathrm{(iii)} & U_{2,jk}' + \rmi\qm_{jk} U_{2,jk} + \rmi\wavenumber U_{3,jk} = 0 \text{ on } I_{jk}, \ \forall I_{jk}\in\edgeSet, \\
			\mathrm{(iv)} & \sum_{j\conRight k} U_{2,kj}\bracs{v_j} - \sum_{j\conLeft k} U_{2,jk}\bracs{v_j} = 0, \ \forall v_j\in\vertSet,
		\end{cases}
	\end{align*}
	where $U_{2,jk}' = \widetilde{U}_{2,jk}'\circ r_{jk}^{-1}$.
\end{theorem}
\begin{proof}
	($\Rightarrow$) The proof follows the same arguments as corollary \ref{cory:DivFreeRotated} and proposition \ref{prop:DivFreeSegment} when showing (i)-(iii) hold.
	The only difference in the argument is that, for each $I_{jk}$, one has to take $\varphi\in\smooth{\ddom}$ with $\supp\bracs{\varphi}\cap I_{jk}$ a compact subset of $I_{jk}^{\mathrm{o}}$ and $\supp\bracs{\varphi}\cap I_{lm} = \emptyset$ whenever $I_{jk}\neq I_{lm}$.
	To show (iv), again take $\varphi\in\smooth{\ddom}$ whose support contains the vertex $v_j$ and no other vertices of $\graph$.
	Then we will have that
	\begin{align*}
		0
		&= \sum_{j\con k}\int_0^{\abs{I_{jk}}} \widetilde{U}_{2,jk}\overline{\widetilde{\varphi}}' - \rmi\qm_2\widetilde{U}_{2,jk}\overline{\widetilde{\varphi}} - \rmi\wavenumber\widetilde{u}_{3,jk}\overline{\widetilde{\varphi}} \ \md t  \\
		&= \sum_{j\con k}\sqbracs{ \widetilde{U}_{2,jk}\overline{\widetilde{\varphi}} }_0^{\abs{I_{jk}}}
		= \overline{\varphi}\bracs{v_j}\bracs{ \sum_{j\conRight k} U_{2,kj}\bracs{v_j} - \sum_{j\conLeft k} U_{2,jk}\bracs{v_j} },
	\end{align*}
	from which (iv) can be inferred.
	
	($\Leftarrow$) Accounting for the difference in dealing with the condition (iv), presented above, this implication follows in the same way as for proposition \ref{prop:DivFreeSegment}.
\end{proof}