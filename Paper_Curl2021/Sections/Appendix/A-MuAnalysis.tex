\section{Appendix: Sobolev functions on the edges of an embedded graph} \label{app:MuAnalysis}
In what follows, we describe the sets $\gradZero{\ddom}{\lambda_{jk}}$ and $\curlZero{\ddom}{\ddmes}$; first for when the edge $I_{jk}$ is parallel to the $x_2$-axis, then for when $I_{jk}$ is non-parallel to the axes, and finally use this understanding to understand $\gradZero{\ddom}{\ddmes}$ and $\curlZero{\ddom}{\ddmes}$.
From this understanding, we will then be able to infer various properties of the functions in $\ktgradSob{\ddom}{\ddmes}$ and $\ktcurlSob{\ddom}{\ddmes}$.
Throughout this section, we will write $\grad^{(0)}$ for the operator $\ktgrad$ with $\kt=\bracs{0,0}$.
Since we have proposition \ref{prop:ZeroInvariantUnderQM-Wavenumber}, it is sufficient for us to only consider this operator when determining the form of gradients (or curls) of zero with respect to $\ddmes$.
\tstk{Throughout this section, assumption \ref{ass:MeasTheoryProblemSetup} is adopted??}

\subsection{Gradients of zero with respect to $\ddmes$} \label{apps:3DGradientsOfZero}
We begin by examining the set of gradients of zero on a graph consisting of a single edge parallel to the $x_2$ axis.
\tstk{this is similar to when we just had regular gradients... in fact it's identical basically...}

\begin{prop} \label{prop:3DGradZeroSegment}
	Let $I$ be a segment in $\ddom$ parallel to the $x_2$ axis, with singular measure $\lambda_I$.
	Then 
	\begin{align*}
		\gradZero{\ddom}{\lambda_I} &= \clbracs{ \bracs{g_1, 0, 0}^{\top} \setVert g_1\in\ltwo{\ddom}{\lambda_I} }.
	\end{align*}
\end{prop}
\begin{proof}
	Without loss of generality we assume $x_1=0$ on $I$.
	If $\bracs{0,0,g_3}^\top\in\gradZero{\ddom}{\ddmes}$, we can find an approximating sequence $\phi_n$ as in \eqref{eq:GradZeroSequenceDef}, and since $\phi_n\rightarrow 0$ we have that $\rmi\wavenumber\phi_n\rightarrow0$ and hence $g_3=0$.
	
	Now suppose that $g_1\in\smooth{\ddom}$.
	Set $\phi(x) = x_1 g_1(x)$, so that 
	\begin{align*}
		\grad^{(0)}\phi(x) = \bracs{g_1(x) + x_1 \partial_1 g_1(x), x_1 \partial_2 g_1(x), \rmi\wavenumber\phi(x)}^\top.
	\end{align*}
	Then $\phi = 0$ on $I$, and $\grad^{(0)}\phi = \bracs{g_1, 0, 0}^\top$ on $I$, so the constant sequence $\phi$ serves as the approximating sequence in \eqref{eq:GradZeroSequenceDef}, and thus $\bracs{g_1, 0, 0}^{\top}\in\gradZero{\ddom}{\lambda_I}$.
	Given the definition of $\gradZero{\ddom}{\lambda_I}$, we conclude by a density that $\bracs{g_1, 0, 0}^{\top}\in\gradZero{\ddom}{\lambda_I}$ for $g_1\in\ltwo{\ddom}{\lambda_I}$.
	
	Finally, suppose that $\bracs{0, g_2, 0}\in\gradZero{\ddom}{\lambda_I}$ and let $\phi_n$ be an approximating sequence as in \eqref{eq:GradZeroSequenceDef}.
	Consider the change of variables $r_I$ (analogous to $r_{jk}$ in \ref{ass:MeasTheoryProblemSetup}) for $I$, and notice that
	\begin{align*}
		\int_0^{\abs{I}} \abs{\phi_n\bracs{r_I(t)}}^2 \ \md t &= \integral{\ddom}{ \abs{\phi_n}^2 }{\lambda_I} \rightarrow 0, \\
		\int_0^{\abs{I}} \abs{ \phi_n'\bracs{r_I(t)} - g_2\bracs{r_I(t)} }^2 \ \md t &= \integral{\ddom}{ \abs{\grad^{(0)}\phi_n - g_2}^2 }{\lambda_I} \rightarrow 0.		
	\end{align*}
	Thus, the function $\tilde{\phi}_n = \phi_n\circ r_I \in\smooth{I}$ is such that $\tilde{\phi}_n\rightarrow 0, \tilde{\phi}_n'\rightarrow g_2\circ r_I$ in $\ltwo{\interval{I}}{t}$.
	Therefore, $g_2\circ r_I$ is the derivative of $0$ in the $\gradSob{\interval{I}}{t}$-sense, and so $g_2\circ r_I = 0$, and hence $g_2 = 0$.
\end{proof}

\tstk{remark here about how these gradients are essentially the same as in the 2D (non-$\wavenumber$) case?}

\begin{cory} \label{cory:3DGradSegmentRotated}
	Let $I$ be a segment in $\ddom$ with local orthogonal co-ordinate system  $y=\bracs{y_1,y_2}$ with $y_2$ parallel to $I$, and let $\lambda_I$ be the singular measure supported on $I$.
	Suppose $R\in\mathrm{SO}(2)$ is the change of co-ordinates $x=Ry$ where $x=\bracs{x_1,x_2}$ is the axis co-ordinate system.
	Then
	\begin{align*}
		\gradZero{\ddom}{\lambda_I} &= 
		\clbracs{ \begin{pmatrix} R^\top & 0 \\ 0 & 1 \end{pmatrix} \begin{pmatrix} g_1 \\ 0 \\ 0 \end{pmatrix} }
	\end{align*}
\end{cory}
\begin{proof}
	This is simply an application of the rotation $R$ to the result of proposition \ref{prop:3DGradZeroSegment}.
\end{proof}
Note that the result of corollary \ref{cory:3DGradSegmentRotated} also implies that
\begin{align*}
	\gradZero{\ddom}{\lambda_I} &= 
		\clbracs{ \begin{pmatrix} g_1 \\ g_2 \\ 0 \end{pmatrix}\in\ltwo{\ddom}{\lambda_I}^3 \setVert \bracs{g_1, g_2}^\top e_I = 0 },
\end{align*}
where $e_I$ is the unit (tangent) vector to $I$.
We now have the following characterisation for gradients of zero on a graph $\graph$.

\tstk{what "name" does this deserve - theorem? prop? cory? It's not revolutionary given the scalar work}
\begin{prop} \label{prop:3DGradZeroCharacterisation}
	Let $\graph=\bracs{\vertSet, \edgeSet}$ be a graph embedded into $\ddom$, and $\ddmes$ the singular measure on $\graph$.
	Assume convention \ref{ass:MeasTheoryProblemSetup}.
	Then
	\begin{align*}
		\gradZero{\ddom}{\ddmes} &= \clbracs{ g\in\ltwo{\ddom}{\ddmes}^3 \setVert g\in\gradZero{\ddom}{\lambda_{jk}} \ \forall I_{jk}\in\edgeSet }
	\end{align*}
\end{prop}
\begin{proof}
	\tstk{this proof is literally the same as that in the scalar case for gradients of zero, save we are carrying a 3rd component $=0$ around with us.
	I can type this out, but here it's probably wisest to point back to the arguments made previously in the scalar case.}
\end{proof}
Now that we have established the form of the functions which live in $\gradZero{\ddom}{\ddmes}$, we can move onto examining the gradients of the functions $u\in\ktgradSob{\ddom}{\ddmes}$.

\subsection{Tangential $\kt$-Gradients with respect to $\ddmes$} \label{apps:3DTangentialGradients}
Whilst $\gradZero{\ddom}{\ddmes}$ is invariant under changes in $\kt$, the gradients of functions in $\ktgradSob{\ddom}{\ddmes}$ are not.
However given proposition \ref{prop:3DGradZeroCharacterisation}, we can understand these gradients on a graph $\graph$ by understanding their form on each edge $I_{jk}$, and so our analysis flows in much the same way as in section \ref{apps:3DGradientsOfZero}.
First, we seek to understand the behaviour of a gradient on a single edge parallel to the $x_2$ axis.

\begin{prop} \label{prop:3DTangGradSegment}
	Let $I$ be a segment in $\ddom$ parallel to the $x_2$ axis, with singular measure $\lambda_I$, and take $r_I$ as in convention \ref{ass:MeasTheoryProblemSetup} for the edge $I$.
	Write $\gradSob{\interval{I}}{t}$ for the ``classical" Sobolev space of functions on $\interval{I}$ with respect to the Lebesgue measure.
	Suppose that $u\in\ktgradSob{\ddom}{\lambda_I}$ and set $\widetilde{u} = u \circ r_I$.
	Then $\widetilde{u}\in\gradSob{\interval{I}}{t}$ and
	\begin{align*}
		\ktgrad_{\lambda_I} u &= 
		\begin{pmatrix} 0 \\ u' + \rmi\qm_2 u \\ \rmi\wavenumber u \end{pmatrix},
	\end{align*}
	where $u'(x) = \bracs{\widetilde{u}'\circ r_I^{-1}}(x)$.
\end{prop}
\begin{proof}
	Write $\ktgrad_{\lambda_I} u = \bracs{v_1, v_2, v_3}^\top$, and let $\phi_n\in\smooth{\ddom}$ be an approximating sequence for $\bracs{u, \ktgrad_{\lambda_I}u}$, so
	\begin{align*}
		\phi_n \lconv{\ltwo{\ddom}{\lambda_I}} u, &\qquad \ktgrad\phi_n \lconv{\ltwo{\ddom}{\lambda_I}^3}\ktgrad_{\lambda_I}u.
	\end{align*}
	The requirement that $\ktgrad_{\lambda_I} u \perp \gradZero{\ddom}{\lambda_I}$ and proposition \ref{prop:3DGradZeroSegment} implies that $v_1=0$, as $v_1$ must be orthogonal to all functions in $\ltwo{\ddom}{\lambda_I}$.
	Since $\rmi\wavenumber\phi_n\lconv{\ltwo{\ddom}{\lambda_I}} v_3$ and $\phi_n\lconv{\ltwo{\ddom}{\lambda_I}}u$, we have that $v_3 = \rmi\wavenumber u$.
	
	This leaves the form of the second component to be determined.
	Set $\widetilde{\phi}_n = \phi_n \circ r_I$, and use the change of variables $r_I$ to obtain
	\begin{align*}
		\int_0^{\abs{I}} \abs{ \widetilde{\phi}_n - \widetilde{u} }^2 \ \md t 
		&= \integral{\ddom}{ \abs{ \phi_n - u }^2 }{\lambda_I} \rightarrow 0, \\
		\int_0^{\abs{I}} \abs{ \widetilde{\phi}'_n + \rmi\qm_2\widetilde{\phi}_n - \widetilde{v}_2 }^2 \ \md t 
		&= \integral{\ddom}{ \abs{ \partial_2\phi_n + \rmi\qm_2\phi_n - v_2 }^2 }{\lambda_I} \rightarrow 0,
	\end{align*}
	where $\widetilde{v}_2 = v_2\circ r_I$.
	Thus, $\widetilde{\phi}_n\in\smooth{\interval{I}}$ is such that
	\begin{align*}
		\widetilde{\phi}_n \rightarrow \widetilde{u}, \quad
		\widetilde{\phi}'_n - \rmi\qm_2\widetilde{\phi}_n \rightarrow \widetilde{v}_2, \quad
		\text{in } \ltwo{\interval{I}}{t}.
	\end{align*}
	As a result, we have that $\widetilde{\phi}'_n \rightarrow  \widetilde{v}_2 - \rmi\qm_2\widetilde{u}$, and thus $\widetilde{u}' = \widetilde{v}_2 - \rmi\qm_2\widetilde{u}$ in the $\gradSob{\interval{I}}{t}$ sense.
	Undoing the change of variables $r_I$ and rearranging for $v_2$ then gives the desired expression.
\end{proof}

As before when we handled $\gradZero{\ddom}{\lambda_{jk}}$, we can use a rotation argument to determine the form of the tangential gradient on an edge oriented at an angle to the co-ordinate axes.
\begin{cory} \label{cory:3DTangGradRotated}
	Let $I$ be a segment in $\ddom$ with local orthogonal co-ordinate system  $y=\bracs{y_1,y_2}$ with $y_2$ parallel to $I$, and let $\lambda_I$ be the singular measure supported on $I$.
	Suppose $R\in\mathrm{SO}(2)$ is the change of co-ordinates $x=Ry$ where $x=\bracs{x_1,x_2}$ is the axis co-ordinate system, and take $r_I$ as in convention \ref{ass:MeasTheoryProblemSetup} for the edge $I$.
	Write $\gradSob{\interval{I}}{t}$ for the Sobolev space of functions on $\interval{I}$ with respect to the Lebesgue measure.
	Suppose that $u\in\ktgradSob{\ddom}{\lambda_I}$ and set $\widetilde{u} = u \circ r_I$.
	Then $\widetilde{u}\in\gradSob{\interval{I}}{t}$ and
	\begin{align*}
		\ktgrad_{\lambda_I} u &= 
		\begin{pmatrix} R^{\top} & 0 \\ 0 & 1	\end{pmatrix}
		\begin{pmatrix} 0 \\ u' + \rmi\qm_{I}u \\ \rmi\wavenumber u \end{pmatrix},
	\end{align*}
	where $\qm_I = \bracs{R_I\qm}_2$ and $u'(x) = \bracs{\widetilde{u}'\circ r_I^{-1}}(x)$.
\end{cory}

Now that we understand the form of the tangential gradient of a function on an edge of a graph, we can determine the form of the tangential gradient for a function $u\in\ktgradSob{\ddom}{\ddmes}$.
\begin{cory} \label{cory:3DTangGradGraph}
	Let $\graph=\bracs{\vertSet \edgeSet}$ be an embedded graph in $\ddom$ and let $\ddmes$ be the singular measure supported on $\graph$.
	Then if $u\in\ktgradSob{\ddom}{\ddmes}$, then for each $I_{jk}\in\edgeSet$ we have that
	\begin{align*}
		\widetilde{u}_{jk} &:= u_{jk} \circ r_{jk} \in \gradSob{\interval{I_{jk}}}{t}, \\
		\bracs{ \ktgrad_{\ddmes} u }\vert_{I_{jk}} &= 
		\begin{pmatrix} R^{\top}_{jk} & 0 \\ 0 & 1	\end{pmatrix}
		\begin{pmatrix} 0 \\ u_{jk}' + \rmi\qm_{jk}u_{jk} \\ \rmi\wavenumber u_{jk} \end{pmatrix},
	\end{align*}
	where $\qm_{jk} = \bracs{ R_{jk}\qm }_2$ and $u_{jk}' = \widetilde{u}_{jk}' \circ r_{jk}^{-1}$.
\end{cory}
\begin{proof}
	Since $u\in\ktgradSob{\ddom}{\ddmes}$, we have that $\ktgrad_{\ddmes} u \perp \gradZero{\ddom}{\ddmes}$.
	Fix an edge $I_{jk}$, and let $\phi_n$ be an approximating sequence for $u\in\ktgradSob{\ddom}{\ddmes}$.
	Given proposition \ref{prop:3DGradZeroCharacterisation}, for every function $g_{jk}\in\gradZero{\ddom}{\lambda_{jk}}$, the function $g_{\graph}\in\gradZero{\ddom}{\ddmes}$ where
	\begin{align*}
		g_{\graph}(x) = 
		\begin{cases} 
			g_{jk}(x), & x\in I_{jk}, \\
			0, & x\not\in I_{jk}.
		\end{cases}
	\end{align*}
	Therefore, $\ktgrad_{\ddmes} u$ is orthogonal to each $g_{\graph}$ which implies that $\bracs{ \ktgrad_{\ddmes} u }\vert_{I_{jk}} \perp \gradZero{\ddom}{\lambda_{jk}}$.
	In addition, 
	\begin{align*}
		\norm{\phi_n - u}_{\ltwo{\ddom}{\lambda_{jk}}} &\leq \norm{\phi_n - u}_{\ltwo{\ddom}{\ddmes}} \rightarrow 0, \\
		\norm{\ktgrad\phi_n - \ktgrad_{\ddmes} u}_{\ltwo{\ddom}{\lambda_{jk}}^2} &\leq \norm{\ktgrad\phi_n - \ktgrad_{\ddmes} u}_{\ltwo{\ddom}{\ddmes}^2} \rightarrow 0,
	\end{align*}
	and so $\bracs{u, \ktgrad_{\ddmes} u}\in\ktgradSob{\ddom}{\lambda_{jk}}$.
	Application of corollary \ref{cory:3DTangGradRotated} then implies the stated form for $\bracs{ \ktgrad_{\ddmes} u }\vert_{I_{jk}}$.
\end{proof}

Corollary \ref{cory:3DTangGradGraph} provides us with an understanding of the behaviour of the tangential gradient $\ktgrad_{\ddmes}u$.
In addition to this understanding of the tangential gradients, we can also demonstrate that functions $u\in\ktgradSob{\ddom}{\ddmes}$ are also continuous at the vertices $v_j$.
\begin{prop} \label{prop:3DVertexContinuity-Grad}
	\begin{align*}
		\bracs{u, \ktgrad_{\ddmes}u}\in\ktgradSob{\ddom}{\ddmes} \quad\Rightarrow\quad
		&\mathrm{(i)} \bracs{u, \bracs{\ktgrad_{\ddmes} u}\vert_{I_{jk}} }\in\ktgradSob{\ddom}{\lambda_{jk}} \quad \forall I_{jk}\in\edgeSet, \\
		&\mathrm{(ii)} \ u \text{ is continuous at every } v_j\in\vertSet.
	\end{align*}
\end{prop}
\begin{proof}
	The implication of (i) is immediate from proposition \ref{prop:3DTangGradGraph}, whilst the proof of the implication of (ii) is identical to that in the scalar case \tstk{should this proof be inserted anyway? If this document is standalone, it probably should. If this is forming part of the thesis, then no, since we'll have the proof to hand in the previous chapter.}
\end{proof}
\tstk{mention here that Zhikov's work appears to claim that the converse is true, but this hinges on his ``proposition 5.5" which we don't think is true, or if it is seems to go against his chosen setup of his Sobolev spaces.}

\subsection{Curls of Zero with respect to $\ddmes$} \label{apps:CurlsOfZero}
With our description of $\ktgradSob{\ddom}{\ddmes}$ complete, we move onto investigating functions with $\kt$-curls with respect to the measure $\ddmes$.
Our investigation proceeds almost identically to that for gradients; we first examine the set $\curlZero{\ddom}{\lambda_{jk}}$ for an edge $I_{jk}$, and use this to build-up an understanding of $\curlZero{\ddom}{\ddmes}$.
The objective being to establish a result similar to proposition \ref{prop:3DGradZeroCharacterisation} for curls of zero, which is then used to describe the functions that live in $\ktcurlSob{\ddom}{\ddmes}$.
To that end, we begin by examining curls of zero on a segment parallel to the $x_2$-axis.

\begin{prop} \label{prop:CurlZeroSegment}
	Let $I$ be a segment in $\ddom$ parallel to the $x_2$ axis, and let $\lambda_I$ be the singular measure supported on $I$.
	Then
	\begin{align*}
		\curlZero{\ddom}{\lambda_I} &= 
		\clbracs{ \bracs{0, c_2, c_3}^\top \setVert c_2, c_3\in\ltwo{\ddom}{\lambda_I} }.
	\end{align*}
\end{prop}
\begin{proof}
	We first show that if $c = \bracs{c_1, 0, 0}^\top\in\curlZero{\ddom}{\lambda_I}$, then $c_1=0$.
	To this end, take an approximating sequence $\Phi^n = \bracs{\phi^n_1, \phi_2^n, \phi_3^n}$ as in \eqref{eq:CurlZeroSequenceDef}.
	This provides us with the following convergences in $\ltwo{\ddom}{\lambda_I}$;
	\begin{align*}
		\phi_1^n \rightarrow 0, \quad
		\phi_2^n \rightarrow 0, \quad
		\phi_3^n \rightarrow 0, \quad
		\partial_1\phi_3^n \rightarrow 0, \quad
		\partial_2\phi_3^n \rightarrow c_1.
	\end{align*}
	In particular, we have that
	\begin{align*}
		\phi_3^n \lconv{\ltwo{\ddom}{\lambda_I}} 0, \quad
		\grad^{(0)}\phi_3^n \lconv{ \ltwo{\ddom}{\lambda_I}^3 } \bracs{0, c_1, 0}^\top.
	\end{align*}
	Therefore, $\bracs{0, c_1, 0}^\top\in\gradZero{\ddom}{\lambda_I}$ but by proposition \ref{prop:3DGradZeroSegment} we have that $c_1=0$.
	
	Now we demonstrate that the second and third components of $c\in\curlZero{\ddom}{\lambda_I}$ can be any functions in $\ltwo{\ddom}{\lambda_I}$.
	Without loss of generality we assume that $x_1=0$ on $I$, and let $c_2\in\smooth{\ddom}$.
	Consider the function $\Phi = \bracs{0, x_1 c_2, 0 }^\top$, and set $c=\bracs{0,c_2,0}^\top$.
	Then
	\begin{align*}
		\integral{\ddom}{ \abs{\Phi}^2 }{\lambda_I} &= \integral{I}{ x_1^2 c_2^2 }{\lambda_I} = 0, \\
		\integral{\ddom}{ \abs{\grad^{(0)}\wedge\Phi - c}^2 }{\lambda_I} &= \integral{I}{ \abs{ x_1\partial_1 c_2}^2 }{\lambda_I} = 0,
	\end{align*}
	so we have that $c\in\curlZero{\ddom}{\lambda_I}$.
	By density, we have that $c=\bracs{0,c_2,0}^\top\in\curlZero{\ddom}{\lambda_I}$ for any $c_2\in\ltwo{\ddom}{\lambda_I}$.
	Similarly, the function $\Psi = \bracs{0, 0, -x_1 c_3}^\top$ for $c_3\in\smooth{\ddom}$ leads us to the conclusion that $\bracs{0,0,c_3}^\top\in\curlZero{\ddom}{\lambda_I}$, and thus that $c=\bracs{0,0, c_3}^\top\in\curlZero{\ddom}{\lambda_I}$ for any $c_3\in\ltwo{\ddom}{\lambda_I}$, which completes the proof.
\end{proof}

Of course, we can now determine what the set $\curlZero{\ddom}{\lambda_I}$ looks like for an edge $I$ at an arbitrary angle to the co-ordinate axes.
\begin{cory} \label{cory:CurlZeroRotated}
	Suppose $I$ is a segment in $\ddom$ with orthogonal co-ordinate system $y = \bracs{y_1, y_2}$ with $y_2$ parallel to $I$, and let $\lambda_I$ be the singular measure supporting $I$. 
	Let $R\in\mathrm{SO}(2)$ be the change of co-ordinates $x = Ry$, where $x = \bracs{x_1, x_2}$ is the axis co-ordinate system. 
	Finally, let $e_I$ be the unit vector directed along $I$ and $n_I$ the unit normal vector to $I$(directed as $y_1$ is). 
	Then
	\begin{align*}
		\curlZero{\ddom}{\lambda_I} &= 
		\clbracs{ \begin{pmatrix} R & 0 \\ 0 & 1 \end{pmatrix} \begin{pmatrix} 0 \\ c_2 \\ c_3 \end{pmatrix} \setVert c_2,c_3\in\ltwo{\ddom}{\lambda_I} } \\
		&= \clbracs{ \begin{pmatrix} c_1 \\ c_2 \\ c_3 \end{pmatrix}\in\ltwo{\ddom}{\lambda_I}^3 \setVert \begin{pmatrix} c_1 \\ c_2 \end{pmatrix} \cdot n_I = 0 \text{ on } I }
	\end{align*}
\end{cory}
\begin{proof}
	This is another rotation argument applied to the result of proposition \ref{prop:CurlZeroSegment}, in the same vein as that of corollary \ref{cory:3DGradSegmentRotated} to proposition \ref{prop:3DGradZeroSegment} and corollary \ref{cory:3DTangGradRotated} to proposition \ref{prop:3DTangGradSegment}.
\end{proof}

At this point it is helpful to provide a geometric interpretation of $\ddmes$-curls of zero (and consequently, the tangential curls).
The conventional interpretation of the curl 
\begin{figure}[h!]
	\centering
	\includegraphics[scale=1.0]{Diagram_CurlZeroPlane.pdf}
	\caption{\label{fig:Diagram_CurlZeroPlane}}
\end{figure}

\subsection{Tangential $\kt$-Curls with respect to $\ddmes$} \label{apps:TangentialCurls}