\section{Appendix: Sobolev Spaces associated with the measure $\dddmes$} \label{app:SumMeasureAnalysis}
Now that the analysis of sections \ref{app:3DMuAnalysis} and \ref{app:3DVertexAnalysis} is complete, we turn our attention to the notion of $\kt$-gradients, curls, and divergence free with respect to the measure $\dddmes$.
Needless to say, our reason for previously considering the two measures $\ddmes$ and $\nu$ individually is because we expect a link back to the properties that these measures bestowed upon functions.
Indeed, our first result informs us that gradients of zero with respect to $\dddmes$ are formed from combinations of gradients of zero with respect to the measures $\ddmes$ and $\nu$.
\begin{prop} \label{prop:3DThickVertexGradZeroCharacterisation}
	Let $g\in\ltwo{\ddom}{\dddmes}^3$ and let 
	\begin{align*}
		g_{\ddmes}(x) = \begin{cases} g(x) & x\neq v_j \ \forall v_j\in\vertSet, \\ 0 & x=v_j, \ v_j\in\vertSet, \end{cases} 
		&\qquad
		g_{\nu}(x) = \begin{cases} 0 & x\neq v_j \ \forall v_j\in\vertSet, \\ g(x) & x=v_j, \ v_j\in\vertSet. \end{cases}
	\end{align*}		
	Then we have that
	\begin{align*}
		g\in\gradZero{\ddom}{\dddmes} \quad\Leftrightarrow\quad 
		& g_{\ddmes}\in\gradZero{\ddom}{\ddmes} \text{ and } g_{\nu}\in\gradZero{\ddom}{\nu}.
	\end{align*}
\end{prop}
\begin{proof}
	\tstk{this is simply a restatement of the result E1 from the scalar paper - there is literally no difference to the argument when we carry around an additional component that is always 0 anyway.
	I can put the whole proof in, but I just don't think it's necessary here.}
\end{proof}

Knowing the correspondence between the various gradients of zero with respect to the measures $\dddmes$, $\ddmes$, and $\nu$, we can deduce the following about the $\kt$-tangential gradients of functions in $\ktgradSob{\ddom}{\dddmes}$.
\begin{theorem} \label{thm:3DThickVertexTangGradImplication}
	\begin{align*}
		\bracs{u, \ktgrad_{\dddmes}u}\in\ktgradSob{\ddom}{\dddmes} \quad\Rightarrow\quad
		& \ \mathrm{(i)} \ \bracs{u, \ktgrad_{\dddmes}u}\in\ktgradSob{\ddom}{\ddmes}, \\
		& \ \mathrm{(ii)} \ \bracs{u, \ktgrad_{\dddmes}u}\in\ktgradSob{\ddom}{\nu}. \\
	\end{align*}
\end{theorem}
\begin{proof}
	Taking an approximating sequence $\phi_n$ such that
	\begin{align*}
		\phi_n \lconv{\ltwo{\ddom}{\dddmes}} u, \quad \ktgrad\phi_n\lconv{\ltwo{\ddom}{\dddmes}^2}\grad_{\dddmes}u,
	\end{align*}
	then using the fact that
	\begin{align*}
		\norm{\cdot}_{\ltwo{\ddom}{\dddmes}}^2 &= \norm{\cdot}_{\ltwo{\ddom}{\ddmes}}^2 + \norm{\cdot}_{\ltwo{\ddom}{\nu}}^2,
	\end{align*}
	we infer that $\phi_n$ also converges in $\ltwo{\ddom}{\ddmes}$ and $\ltwo{\ddom}{\nu}$, as do its gradients in $\ltwo{\ddom}{\ddmes}^2$ and $\ltwo{\ddom}{\nu}^2$.
	Furthermore, since $\ktgrad_{\dddmes}u \perp \gradZero{\ddom}{\dddmes}$ and given proposition \ref{prop:3DThickVertexGradZeroCharacterisation}, we have that $\ktgrad_{\dddmes}u$ is orthogonal to $\gradZero{\ddom}{\ddmes}$ (in $\ltwo{\ddom}{\ddmes}^2$) and to $\gradZero{\ddom}{\nu}$ (in $\ltwo{\ddom}{\nu}^2$), and we are done. 
\end{proof}
From our analysis of $\ktgradSob{\ddom}{\ddmes}$ we know that the conditions (i) and (ii) in theorem \ref{thm:3DThickVertexTangGradImplication} are sufficient for the following to hold;
\begin{align*}
		\text{(a)} \ & \bracs{u, \ktgrad_{\dddmes}u}\in\gradSob{\ddom}{\lambda_{jk}} \ \forall I_{jk}\in\edgeSet, \\
		\text{(b)} \ & u \text{ is continuous across the vertices} \ v_j\in\vertSet, \\
		\text{(c)} \ & \ktgrad_{\dddmes}u\vert_{v_j} = 0 \ \forall v_j\in\vertSet.
\end{align*}
If the converse to theorem \ref{thm:3DThickVertexTangGradImplication} is true, as implied from results in \cite{zhikov2002homogenization}, then (i)-(ii) are necessary and sufficient for (a)-(c).
Understanding the space $\ktgradSob{\ddom}{\dddmes}$ through the properties of $\ktgradSob{\ddom}{\ddmes}$ and $\ktgradSob{\ddom}{\nu}$ will allow us to understand the $\kt$-divergence-free condition with respect to $\dddmes$.
However, before proceeding we also deduce results similar to proposition \ref{prop:3DThickVertexGradZeroCharacterisation} and theorem \ref{thm:3DThickVertexTangGradImplication}, but concerning curls with respect to $\dddmes$.

\begin{prop} \label{prop:ThickVertexCurlZeroCharacterisation}
	Let $c\in\ltwo{\ddom}{\dddmes}^3$ and set
	\begin{align*}
		c_{\ddmes}(x) = \begin{cases} c(x) & x\neq v_j \ \forall v_j\in\vertSet, \\ 0 & x=v_j, \ v_j\in\vertSet, \end{cases} 
		&\qquad
		c_{\nu}(x) = \begin{cases} 0 & x\neq v_j \ \forall v_j\in\vertSet, \\ c(x) & x=v_j, \ v_j\in\vertSet. \end{cases}
	\end{align*}		
	Then we have that
	\begin{align*}
		c\in\curlZero{\ddom}{\dddmes} \quad\Leftrightarrow\quad 
		& c_{\ddmes}\in\curlZero{\ddom}{\ddmes} \text{ and } c_{\nu}\in\curlZero{\ddom}{\nu}.
	\end{align*}
\end{prop}
\begin{proof}
	For the right-directed implication ($\Rightarrow$), it is sufficient to notice that 
	\begin{align*}
		\norm{\cdot}_{\ltwo{\ddom}{\dddmes}}^2 &= \norm{\cdot}_{\ltwo{\ddom}{\ddmes}}^2 + \norm{\cdot}_{\ltwo{\ddom}{\nu}}^2,
	\end{align*}
	so any approximating sequence for $c$ that converges in $\ltwo{\ddom}{\dddmes}^3$ also converges in $\ltwo{\ddom}{\ddmes}^3$ to $c_{\ddmes}$ and in $\ltwo{\ddom}{\nu}^3$ to $c_{\nu}$.
	
	For the left-directed implication ($\Leftarrow$), it is sufficient for us to demonstrate the implication holds for the case when $c_{\nu}=0$, and the case that $c_{\ddmes}=0$ with $c_{\nu}\neq0$ at precisely one vertex $v$.
	Having shown the implication in these cases, linearity of $\curlZero{\ddom}{\dddmes}$ will then complete the implication.
	As such, first consider the case when $c_{\nu}=0$.
	Notice that the conclusion of lemma \ref{lem:CurlZeroExtensionLemma} can be strengthened to membership of $\curlZero{\ddom}{\dddmes}$, as the approximating sequence $\psi_l$ that is constructed satisfies $\psi_l\bracs{v_j}=0, \grad^{(0)}\wedge\psi_l\bracs{v_j}=0$.
	With this, the argument of lemma \ref{lem:BInCurlZero} can be recycled to conclude that $c_{\ddmes}\in\curlZero{\ddom}{\dddmes}$, and hence $c\in\curlZero{\ddom}{\dddmes}$ too.
	
	Next, consider the case when $c_{\ddmes}=0$, and when $c_{\nu}=0$ at all vertices except $v\in\vertSet$, with $v=\bracs{v_1, v_2}\in\ddom$ and with $c_{\nu}(v) = \bracs{c_1, c_2, c_3}^\top$.
	For each $n\in\naturals$ consider the smooth function $\phi_n:\reals\rightarrow\sqbracs{-1,1}$ as illustrated in figure \ref{fig:Diagram_SmoothFunctionBoundedGrad1DAltAxisLabels}, with the properties
	\begin{align*}
		\phi_n(0) = 0,
		&\quad	\phi'_n(0) = 1, \\
		\phi_n(t) = 0, &\quad t\not\in B_{\frac{2}{n}}(0), \\
		\abs{\phi_n(t)} \leq \recip{n} &\quad t\in B_{\frac{2}{n}}(0).
	\end{align*}
	\begin{figure}[b]
		\centering
		\includegraphics[scale=1.0]{Diagram_SmoothFunctionBoundedGrad1DAltAxisLabels.pdf}
		\caption{\label{fig:Diagram_SmoothFunctionBoundedGrad1DAltAxisLabels} The profile of the functions $\phi_n$.}
	\end{figure}
	Since $\abs{\phi_n(t)} \leq \recip{n}$ when $\abs{t}\leq\recip{n}$, $\phi_n$ can be chosen so that exists a constant $K$ independent of $n$ such that $\abs{\grad\phi_n} \leq K$ when $\recip{n} \leq \abs{x-v} \leq \frac{2}{n}$.
	Define the functions $\Phi^n\in\smooth{\ddom}$ (and compute their curls) as follows;
	\begin{align*}
		\Phi^n(x) = 
		\begin{pmatrix} 
			0 \\ 
			c_3\phi_n\bracs{x_1 - v_1} \\ 
			c_1\phi_n\bracs{x_2-v_2} + c_2\phi_n\bracs{v_1-x_1} 
		\end{pmatrix},
		&\qquad
		\grad^{(0)}\wedge\Phi_n(x) =
		\begin{pmatrix}
			c_1\phi'_n\bracs{x_2-v_2} \\
			c_2\phi'_n\bracs{v_1-x_1} \\
			c_3\phi'_n\bracs{x_1-v_1}
		\end{pmatrix}.
	\end{align*}
	Then we have the following:
	\begin{align*}
		\integral{\ddom}{ \abs{ \Phi^n }^2 }{\dddmes}
		&\leq \bracs{c_3^2 + \bracs{c_1 + c_2}^2} \bracs{ \integral{B_{\frac{2}{n}}(v)}{ \recip{n^2} }{\ddmes}
		+ \alpha_v\abs{\phi_n(0)}^2 } \\
		&= \frac{2\mathrm{deg}(v)}{n^3}\bracs{\abs{c_3}^2 + \abs{c_1 + c_2}^2} \rightarrow 0 \toInfty{n}, \\
		\integral{\ddom}{ \abs{ \grad^{(0)}\wedge\Phi^n - c }^2 }{\dddmes}
		&= \integral{\ddom}{ \abs{ \grad^{(0)}\wedge\Phi^n}^2 }{\ddmes}
		+ \integral{\ddom}{ \abs{ \grad^{(0)}\wedge\Phi^n - c_{\nu} }^2 }{\nu} \\
		&= \abs{c_1}^2\integral{\ddom}{ \abs{\phi'_n\bracs{x_2-v_2}}^2 }{\ddmes}
		+ \abs{c_2}^2\integral{\ddom}{ \abs{\phi'_n\bracs{v_1-x_1}}^2 }{\ddmes} \\
		&\quad + \abs{c_3}^2\integral{\ddom}{ \abs{\phi'_n\bracs{x_1-v_1}}^2 }{\ddmes}
		+ \alpha_v\abs{c(v)}^2\abs{\phi'_n(0)-1}^2 \\
		&\leq \abs{c(v)}^2 \bracs{ K^2\integral{B_{\frac{2}{n}}(v)}{ }{\ddmes}
		+ \alpha_v\abs{\phi'_n(0)-1}^2 } \\
		&= \frac{2\mathrm{deg}(v)}{n}\abs{c(v)}^2 K^2 \rightarrow 0 \toInfty{n},
	\end{align*}
	where $\mathrm{\deg}(v)$ is the degree of the vertex $v$.
	We thus conclude that $c\in\curlZero{\ddom}{\dddmes}$, and given the linearity of $\curlZero{\ddom}{\dddmes}$, the proof is complete.
\end{proof}

\begin{theorem} \label{thm:ThickVertexTangCurlImplication}
	\begin{align*}
		\bracs{u, \ktcurl{\dddmes}u}\in\ktcurlSob{\ddom}{\dddmes}
		\quad\Rightarrow\quad
		\mathrm{(i)} &\ \bracs{u, \ktcurl{\dddmes}u}\in\ktcurlSob{\ddom}{\ddmes} \\
		\mathrm{(ii)} &\ \bracs{u, \ktcurl{\dddmes}u}\in\ktcurlSob{\ddom}{\nu}
	\end{align*}
\end{theorem}
\begin{proof}
	Taking an approximating sequence $\phi_n$ such that
	\begin{align*}
		\phi_n \lconv{\ltwo{\ddom}{\dddmes}} u, \quad \ktcurl{}\phi_n\lconv{\ltwo{\ddom}{\dddmes}^3}\ktcurl{\dddmes}u,
	\end{align*}
	then using the fact that
	\begin{align*}
		\norm{\cdot}_{\ltwo{\ddom}{\dddmes}}^2 &= \norm{\cdot}_{\ltwo{\ddom}{\ddmes}}^2 + \norm{\cdot}_{\ltwo{\ddom}{\nu}}^2,
	\end{align*}
	we infer that $\phi_n$ also converges in $\ltwo{\ddom}{\ddmes}$ and $\ltwo{\ddom}{\nu}$, as do its $\kt$-curls in $\ltwo{\ddom}{\ddmes}^3$ and $\ltwo{\ddom}{\nu}^3$.
	Furthermore, since $\ktcurl{\dddmes}u \perp \curlZero{\ddom}{\dddmes}$ and given proposition \ref{prop:ThickVertexCurlZeroCharacterisation}, we have that $\ktcurl{\dddmes}u$ is orthogonal to $\curlZero{\ddom}{\ddmes}$ (in $\ltwo{\ddom}{\ddmes}^3$) and to $\curlZero{\ddom}{\nu}$ (in $\ltwo{\ddom}{\nu}^3$), and we are done. 
\end{proof}

Theorem \ref{thm:ThickVertexTangCurlImplication} provides us with a description of the space $\ktcurlSob{\ddom}{\dddmes}$.
When combined with the $\kt$-divergence-free condition, we will have enough information to derive the system \eqref{eq:QGFullSystem} from \eqref{eq:PeriodCellCurlCurlStrongForm}.
Of course, given that we have proposition \ref{prop:3DThickVertexGradZeroCharacterisation} and theorem \ref{thm:3DThickVertexTangGradImplication}, we can look to understand the divergence-free condition.

\begin{prop} \label{prop:ThickVertexDivFree}
	Let $\graph=\bracs{\vertSet, \edgeSet}$ be a graph embedded into $\ddom$, and recall assumption \ref{ass:MeasTheoryProblemSetup}.
	Suppose $u\in\ktcurlSob{\ddom}{\dddmes}$ and define
	\begin{align*}
		U^{(jk)} = R_{jk}\begin{pmatrix} u_1^{(jk)} \\ u_2^{(jk)} \end{pmatrix},
		\qquad \qm_{jk} = \bracs{ R_{jk}\qm }_2,
		\qquad \widetilde{U}^{(jk)} = U^{(jk)} \circ r_{jk},
	\end{align*}
	for each edge $I_{jk}$.
	Then $u$ is $\kt$-divergence free with respect to $\dddmes$ if and only if all of the following conditions hold:
	\begin{align*}
			\mathrm{(i)} \quad & U_1^{(jk)} = 0, \ \forall I_{jk}\in\edgeSet, \\
			\mathrm{(ii)} \quad & \widetilde{U}_2^{(jk)} \in\gradSob{\ddom}{\lambda_{jk}}, \ \forall I_{jk}\in\edgeSet, \\
			\mathrm{(iii)} \quad & \bracs{U_2^{(jk)}}' + \rmi\qm_{jk} U_2^{(jk)} + \rmi\wavenumber u_3^{(jk)} = 0 \text{ on } I_{jk}, \ \forall I_{jk}\in\edgeSet, \\
			\mathrm{(iv)} \quad & u_1\bracs{v_j} = u_2\bracs{v_j} = 0, \ \forall v_j\in\vertSet, \\
			\mathrm{(v)} \quad & \sum_{j\conRight k} U_2^{(kj)}\bracs{v_j} - \sum_{j\conLeft k} U_2^{(jk)}\bracs{v_j} = \rmi\wavenumber\alpha_j u_3\bracs{v_j}, \ \forall v_j\in\vertSet,
	\end{align*}
	where $\bracs{U_2^{(jk)}}' = \bracs{\widetilde{U}_2^{(jk)}}'\circ r_{jk}^{-1}$.
\end{prop}
\begin{proof}
	($\Rightarrow$) Suppose that $u$ is $\kt$-divergence-free with respect to $\dddmes$.
	The properties (i) and (iv) are obtained by considering orthogonality of $u$ to $\gradZero{\ddom}{\dddmes}$, and given proposition \ref{prop:3DThickVertexGradZeroCharacterisation} utilise the same arguments that appear in theorem \ref{thm:DivFreeEdgesCharacterisation} and proposition \ref{prop:DivFreeVertex} when considering orthogonality to $\gradZero{\ddom}{\ddmes}$ and $\gradZero{\ddom}{\nu}$ (respectively).
	Properties (ii) and (iii) follow from using theorem \ref{thm:3DThickVertexTangGradImplication} and recycling the argument for properties (ii) and (iii) in theorem \ref{thm:DivFreeEdgesCharacterisation}.
	For (v), fix a vertex $v_j\in\vertSet$ and take a (smooth) function $\varphi$ whose support contains $v_j$ in its interior.
	Then we have that
	\begin{align*}
		0 &= \integral{\ddom}{ u\cdot\overline{\ktgrad_{\dddmes}\varphi} }{\dddmes} \\
		&= \sum_{j\con k}\integral{I_{jk}}{ U_2^{(jk)}\bracs{\overline{\varphi}' - \rmi\qm_{jk}\overline{\varphi}} - \rmi\wavenumber u_3^{(jk)}\overline{\varphi} }{\lambda_{jk}} + \integral{\ddom}{ u\cdot\overline{\ktgrad_{\dddmes}\varphi} }{\nu} \\
		&= \sum_{j\con k}\int_0^{\abs{I_{jk}}} \widetilde{U}_2^{(jk)}\bracs{\overline{\widetilde{\varphi}}' - \rmi\qm_{jk}\overline{\widetilde{\varphi}}} - \rmi\wavenumber u_3^{(jk)}\overline{\widetilde{\varphi}} \ \md y
		-\rmi\wavenumber\alpha_j u_3\bracs{v_j}\overline{\varphi}\bracs{v_j},
	\end{align*}
	where we have used overhead tildes to denote composition with $r_{jk}$.
	Rearranging and using (ii), we find that
	\begin{align*}
		\rmi\wavenumber u_3\bracs{v_j}\overline{\varphi}\bracs{v_j}
		&= - \sum_{j\con k}\int_0^{\abs{I_{jk}}} \overline{\widetilde{\varphi}}\bracs{ \widetilde{U}_2^{(jk)} + \rmi\qm_{jk}\widetilde{U}_2^{(jk)} + \rmi\wavenumber\widetilde{u}_3^{(jk)} } \ \md y \\
		&\quad + \overline{\varphi}\bracs{v_j}\bracs{ \sum_{j\conRight k} U_2^{(jk)}\bracs{v_j} - \sum_{j\conLeft k} U_2^{(jk)}\bracs{v_j} } \\
		&= \overline{\varphi}\bracs{v_j}\bracs{ \sum_{j\conRight k} U_2^{(kj)}\bracs{v_j} - \sum_{j\conLeft k} U_2^{(jk)}\bracs{v_j} }.
	\end{align*}
	This holds for every such smooth $\varphi$, and so we must have that
	\begin{align*}
		\rmi\wavenumber u_3\bracs{v_j} &= \sum_{j\conRight k} U_2^{(kj)}\bracs{v_j} - \sum_{j\conLeft k} U_2^{(jk)}\bracs{v_j},
	\end{align*}
	which is condition (v).
	
	($\Leftarrow$) Given the ideas above and proposition \ref{prop:3DThickVertexGradZeroCharacterisation}, and, this implication employs the arguments above but in reverse to show that the divergence-free condition is satisfied. 
	Conditions (i) and (iv) ensure orthogonality to $\gradZero{\ddom}{\dddmes}$, whilst (iii) through (v) ensure orthogonality to all $\kt$-tangential gradients.
\end{proof}

With theorem \ref{thm:ThickVertexTangCurlImplication} and proposition \ref{prop:ThickVertexDivFree}, we have necessary and sufficient properties of $\kt$-divergence-free functions $u\in\ktcurlSob{\ddom}{\dddmes}$, and will able able to use these in our derivation of the system \eqref{eq:QGFullSystem}.