\section{Derivation of quantum graph problem} \label{sec:3DSystemDerivation}
In this section we provide an overview of how the system \eqref{eq:3DQGFullSystem} is obtained from \eqref{eq:PeriodCellCurlCurlStrongForm}, which will setup our discussion concerning the tools for determining the spectrum of \eqref{eq:3DQGFullSystem} in section \ref{sec:3DDiscussion}.
To reiterate what was said at the end of section \ref{sec:QuantumGraphs}, convention \ref{ass:MeasTheoryProblemSetup} is adopted throughout this section and the rest of this work. 

Precise definition and analysis of the ``Sobolev spaces" defined here can be found in the appendices \ref{app:MeasureTheory}-\ref{app:SumMeasureAnalysis}, but we provide a short intuitive idea of the object $\ktgrad_{\dddmes}$ and its implications for the concepts of gradient, divergence, and curl here.
The central idea surrounding these concepts involves determining the information (about a function) that is ``lost" due to the measure $\dddmes$ only supporting the edges and vertices of an embedded graph $\graph$, which is a set without any interior from the perspective of the region $\ddom$ into which it is embedded.
Another key result to note is that, since $\dddmes$ is the sum of a singular measure supporting the edges of $\graph$ ($\ddmes$) and point-masses at the vertices ($\nu$), functions in $\ktgradSob{\ddom}{\dddmes}$ possess both the properties of functions in $\ktgradSob{\ddom}{\ddmes}$ and $\ktgradSob{\ddom}{\nu}$ (analogous statements hold for the curl spaces too).

With this in mind, to understand $\ktgrad_{\dddmes}u$ we consider the behaviour of $\ktgrad_{\ddmes}u$ (the part of the gradient on the edges) and $\ktgrad_{\nu}u$ (the part on the vertices).
On the edges, the measure $\ddmes$ cannot see any changes in a function $u$ that occur ``across" (specifically, in the direction perpendicular to) the edges of $\graph$.
So at any point $x\in I_{jk}$, the gradient $\ktgrad_{\ddmes}u$ encapsulates the rate of change of the function $u$ \emph{only in the direction along the edge} $I_{jk}$, and it not inaccurate to think of $\ktgrad_{\ddmes}u(y) = \bracs{u^{(jk)}}'(y)e_{jk}$ for $y\in I_{jk}$, where $\bracs{u^{(jk)}}' = \pdiff{u^{(jk)}}{e_{jk}}$.
The curl $\ktcurl{\ddmes}u$ of a vector field $u$ tells a similar story; an in-depth interpretation is provided in appendix \ref{apps:CurlsOfZero}, but the consequence is that the curl is directed normal to the edges of $\graph$ so as to induce a rotation ``within" the edge $I_{jk}$ (appendix \ref{apps:CurlsOfZero} has precise details).
Finally, for a vector field $u$ to be divergence free on the edges of $\graph$, it must be orthogonal to all gradients.
This results the requirement that, on any given edge $I_{jk}$ the components of $u$ ``directed down the edge" must satisfy a particular ODE, and contributions of these components from connected edges must sum to zero at common vertices (see theorem \ref{thm:DivFreeEdgesCharacterisation}).
For the behaviour of functions at the vertices of $\graph$ we can apply similar intuition, but the ``view" of the measure $\nu$ is even more restricted than that of $\ddmes$, only being able to see isolated points in $\ddom$ corresponding to the placement of vertices.
It transpires that because of this, $\nu$ cannot see any changes ``across" the vertices, which results in all gradients $\ktgrad_{\nu}u=0$ and all curls being essentially zero too.
The requirement that a vector field $u$ be divergence-free even transpires to be necessary and sufficient to having $u=0$ at all the vertices.

Before we begin the derivation, let us establish some notation.
Let $u=\bracs{u_1,u_2,u_3}^\top\in\ktcurlSob{\ddom}{\dddmes}$ be divergence-free, and let $\Phi=\bracs{\phi_1,\phi_2,\phi_3}^\top\in\smooth{\ddom}^3$.
Also define for each $I_{jk}\in\edgeSet$, \tstk{could be $R^\top$s here!}
\begin{align*}
	U^{(jk)} := R_{jk} \begin{pmatrix} u_1^{(jk)} \\ u_2^{(jk)} \end{pmatrix},
	\qquad
	\Psi^{(jk)} := R_{jk} \begin{pmatrix} \phi_1^{(jk)} \\ \phi_2^{(jk)} \end{pmatrix}.
\end{align*}
Use an overhead tilde to denote composition with $r_{jk}$, and for a function $u$ with $\widetilde{u}^{(jk)}\in\ktgradSob{\interval{I_{jk}}}{y}$ write $\bracs{u^{(jk)}}' := \bracs{\widetilde{u}^{(jk)}}' \circ r_{jk}^{-1}$.
Finally, for a given quasi-momentum $\qm$, set $\qm_{jk} = \bracs{ R_{jk}\qm }_2$, and let
\begin{align*}
	\ktcurlSobDivFree{\ddom}{\dddmes} &=
	\clbracs{ u\in\ktcurlSob{\ddom}{\dddmes} \setVert u \text{ is } \kt\text{-divergence-free with respect to } \dddmes}.
\end{align*}
A summary of all the important results which will be used in the derivation that follows is provided below, for $u\in\ktcurlSobDivFree{\ddom}{\dddmes}$:
\begin{enumerate}[(a)]
	\item (Theorem \ref{thm:TangCurlGraph}): For every edge $I_{jk}$, $\ktcurl{\dddmes}u$ points normal to the edge $I_{jk}$, and has the form
	\begin{align*}
		\ktcurl{\dddmes}u(x) = \bracs{ \bracs{u_3^{(jk)}}'(x) + \rmi\qm_{jk} u_3^{(jk)}(x) - \rmi\wavenumber U_2^{(jk)}(x) } n_{jk}, \quad x\in I_{jk}.
	\end{align*}
	\item (Proposition \ref{prop:CurlImpliesThirdCompGradient}): $u_3\in\ktgradSob{\ddom}{\ddmes}$ is continuous at every $v_j\in\vertSet$.
	\item (Corollary \ref{cory:VertexCurlSob}): At every vertex $v_j\in\vertSet$, $\ktcurl{\dddmes}u = 0$.
	Note that this \emph{does not} necessitate that
	\begin{align*}
		\lim_{\substack{x\rightarrow v_j \\ x\in I_{jk}, \\ j\con k}}\ktcurl{\dddmes}u(x) = 0,
	\end{align*}
	since this is a behaviour induced by the measure $\nu$ as a part of $\dddmes$, and is disconnected from the behaviour of the curl on the edges.
	\item (Proposition \ref{prop:ThickVertexDivFree}): On every edge $I_{jk}$ we have $U_1^{(jk)}=0$ and
	\begin{align*}
		\bracs{U_2^{(jk)}}' + \rmi\qm_{jk} U_2^{(jk)} + \rmi\wavenumber u_3^{(jk)} = 0.
	\end{align*}
	\item (Proposition \ref{prop:ThickVertexDivFree}): At every vertex $v_j$, we have that $u_1(v_j) = u_2(v_j) = 0$ and
	\begin{align*}
		\sum_{j\conRight k} U_2^{(jk)}\bracs{v_j} - \sum_{j\conLeft k} U_2^{(jk)}\bracs{v_j} = \rmi\wavenumber\alpha_j u_3\bracs{v_j}.
	\end{align*}
\end{enumerate}

We now derive the system \eqref{eq:3DQGFullSystem} from \eqref{eq:PeriodCellCurlCurlStrongForm}.
A function $u\in\ktcurlSob{\ddom}{\dddmes}$ is a solution to \eqref{eq:PeriodCellCurlCurlStrongForm} if
\begin{align} \label{eq:PeriodCellCurlCurlWeakForm}
	\integral{\ddom}{ \ktcurl{\dddmes}u\cdot\overline{\ktcurl{\dddmes}\Phi} }{\dddmes} &= \omega^2 \integral{\ddom}{ u\cdot\overline{\Phi} }{\dddmes},
	\quad\forall \Phi\in\smooth{\ddom}^3.
\end{align}
Notice that we immediately have $u\in\ktcurlSobDivFree{\ddom}{\dddmes}$ from \eqref{eq:PeriodCellCurlCurlWeakForm}, since we can choose $\Phi = \ktgrad\phi$ for any smooth $\phi$ (and apply proposition \ref{prop:CurlOfGradIsZero} and corollary \ref{cory:DivFreeSufficient}).
The equality in \eqref{eq:PeriodCellCurlCurlWeakForm} holds (in particular) whenever we take $\Phi$ to be a smooth function whose support only intersects (the interior of) an edge $I_{jk}$, and no other parts of the graph $\graph$.
In this case, \eqref{eq:PeriodCellCurlCurlWeakForm} reduces to
\begin{align*}
	\integral{I_{jk}}{ \ktcurl{\ddmes}u\cdot\overline{\ktcurl{\ddmes}\Phi} }{\lambda_{jk}} &= \omega^2 \integral{I_{jk}}{ u\cdot\overline{\Phi} }{\lambda_{jk}},
\end{align*}
since the contribution at the vertices is zero due to the choice of $\Phi$.
By (a) we have that
\begin{align*}
	\integral{I_{jk}}{ \bracs{ \bracs{u_3^{(jk)}}' + \rmi\qm_{jk} u_3^{(jk)} - \rmi\wavenumber U_2^{(jk)} }\overline{\bracs{ \bracs{\phi_3^{(jk)}}' + \rmi\qm_{jk} \phi_3^{(jk)} - \rmi\wavenumber \Psi_2^{(jk)} }} }{\lambda_{jk}}
	&= \omega^2 \integral{I_{jk}}{ u\cdot\overline{\Phi} }{\lambda_{jk}},
\end{align*}
and then using the change of variables $r_{jk}$ this implies
\begin{align*} 
	\int_0^{\abs{I_{jk}}} \bracs{ \bracs{\widetilde{u}_3^{(jk)}}' + \rmi\qm_{jk} \widetilde{u}_3^{(jk)} - \rmi\wavenumber \widetilde{U}_2^{(jk)} } 
	& \bracs{ \bracs{\overline{\widetilde{\phi}}_3^{(jk)}}' - \rmi\qm_{jk} \overline{\widetilde{\phi}}_3^{(jk)} + \rmi\wavenumber \overline{\widetilde{\Psi}}_2^{(jk)} } \ \md y 
	\\
	&= \omega^2 \int_0^{\abs{I_{jk}}} \widetilde{U}_2^{(jk)}\overline{\widetilde{\Psi}}_2^{(jk)} + \widetilde{u}_3^{(jk)}\overline{\widetilde{\phi}}_3^{(jk)} \ \md y,
\end{align*}
which holds for all $\widetilde{\Psi}_2^{(jk)}, \widetilde{\phi}_3^{(jk)}\in\smooth{\interval{I_{jk}}}$ with compact support in $\bracs{0,\abs{I_{jk}}}$.
Therefore, for all $\psi\in\smooth{\interval{I_{jk}}}$ with compact support in $\bracs{0,\abs{I_{jk}}}$ we have that,
\begin{subequations}
	\begin{align*}
		0 &= \int_0^{\abs{I_{jk}}} \overline{\psi} \bracs{ \rmi\wavenumber\bracs{\widetilde{u}_3^{(jk)}}' + \bracs{\wavenumber^2 - \omega^2}\widetilde{U}_2^{(jk)} - \wavenumber\qm_{jk}\widetilde{u}_3^{(jk)}  } \ \md y, \labelthis\label{eq:CurlCurlWeakFormPhi2} \\
		0 &= \int_0^{\abs{I_{jk}}} \overline{\psi}' \bracs{ \bracs{\widetilde{u}_3^{(jk)}}'
		- \rmi\wavenumber\widetilde{U}_2^{(jk)} + \rmi\qm_{jk}\widetilde{u}_3^{(jk)} } \\
		&\qquad -\rmi\qm_{jk}\overline{\psi}\bracs{ \bracs{\widetilde{u}_3^{(jk)}}' - \rmi\wavenumber\bracs{\widetilde{U}_2^{(jk)}}' + \rmi\qm_{jk}\widetilde{u}_3^{(jk)} }
		- \omega^2 \widetilde{u}_3^{(jk)}\overline{\psi} \ \md y. \labelthis\label{eq:CurlCurlWeakFormPhi3}
	\end{align*}
\end{subequations}
Given (b) and (d), \eqref{eq:CurlCurlWeakFormPhi3} can be manipulated to demonstrate that
\begin{align*}
	-\int_0^{\abs{I_{jk}}} \overline{\psi}' \widetilde{u}'_{3,jk} \ \md y
	&= \int_0^{\abs{I_{jk}}} \overline{\psi} \bracs{ \rmi\wavenumber\widetilde{U}'_{2,jk} - \wavenumber\qm_{jk}\widetilde{U}_{2,jk} - 2\rmi\qm_{jk}\widetilde{u}'_{3,jk} + \qm_{jk}^2\widetilde{u}_{3,jk} - \omega^2\widetilde{u}_{3,jk} } \ \md y.
\end{align*}
This holds for all such $\psi$, so $\bracs{\widetilde{u}_3^{(jk)}}'\in\gradSob{\interval{I_{jk}}}{y}$  --- that is, $\widetilde{u}$ is twice (weakly) differentiable.
As such, \eqref{eq:CurlCurlWeakFormPhi3} implies
\begin{align*}
	0 &= \int_0^{\abs{I_{jk}}} \overline{\psi} \bracs{ \rmi\wavenumber\bracs{\widetilde{U}_2^{(jk)}}' - \wavenumber\qm_{jk}\widetilde{U}_2^{(jk)} - \bracs{\widetilde{u}_3^{(jk)}}'' - 2\rmi\qm_{jk}\bracs{\widetilde{u}_3^{(jk)}}' + \qm_{jk}^2\widetilde{u}_3^{(jk)} - \omega^2\widetilde{u}_3^{(jk)} } \ \md y,
\end{align*}
and since \eqref{eq:CurlCurlWeakFormPhi2} and the equation above hold for all smooth $\psi$ with compact support, (after some rearranging) we have
\begin{subequations} \label{eq:QGPhiStandalone}
	\begin{align}
		i\wavenumber \bracs{ \diff{}{y} + \rmi\qm_{jk} }\widetilde{u}_3^{(jk)} + \wavenumber^2\widetilde{U}_2^{(jk)} &= \omega^2\widetilde{U}_2^{(jk)}, \label{eq:QGPhi2Standalone} \\
		-\bracs{ \diff{}{y} + \rmi\qm_{jk} }^2\widetilde{u}_3^{(jk)} + \rmi\wavenumber\bracs{ \diff{}{y} + \rmi\qm_{jk} }\widetilde{U}_2^{(jk)} &= \omega^2 \widetilde{u}_3^{(jk)}, \label{eq:QGPhi3Standalone}
	\end{align}
\end{subequations}
on each $I_{jk}$.

Now we return to \eqref{eq:PeriodCellCurlCurlWeakForm}, and fix a vertex $v_j\in\vertSet$.
Consider smooth $\Phi$ with whose support contains the vertex $v_j$ in its interior, and no other vertices of $\graph$.
Since \eqref{eq:PeriodCellCurlCurlWeakForm} holds for each of these $\Phi$ too, we have that
\begin{align*}
	\alpha_j \omega^2 u_3\bracs{v_j} \overline{\phi}_3\bracs{v_j} 
	&= \sum_{j\con k} \integral{I_{jk}}{ \bracs{ \bracs{u_3^{(jk)}}' + \rmi\qm_{jk} u_3^{(jk)} - \rmi\wavenumber U_2^{(jk)} } \overline{ \bracs{ \bracs{\phi_3^{(jk)}}' + \rmi\qm_{jk} \phi_3^{(jk)} - \rmi\wavenumber \Psi_2^{(jk)}} }}{\lambda_{jk}} \\
	&\qquad - \omega^2 \sum_{j\con k}\integral{I_{jk}}{ u\cdot\overline{\Phi} }{\lambda_{jk}}.
\end{align*}
Then changing variables via $r_{jk}$ and using \eqref{eq:QGPhi2Standalone} on each connecting edge, we find that (with $\widetilde{\phi}_3^{(jk)} = \psi^{(jk)}$),
\begin{align*}
	\alpha_j \omega^2 u_3\bracs{v_j} \phi_3\bracs{v_j}
	 &= \sum_{j\con k} \int_0^{\abs{I_{jk}}} 	\overline{\psi^{(jk)}}' \bracs{ \bracs{\widetilde{u}_3^{(jk)}}' - \rmi\wavenumber\widetilde{U}_2^{(jk)} + \rmi\qm_{jk}\widetilde{u}_3^{(jk)} } \\
		&\qquad -\rmi\qm_{jk}\overline{\psi^{(jk)}}\bracs{ \bracs{\widetilde{u}_3^{(jk)}}' - \rmi\wavenumber\widetilde{U}_2^{(jk)} + \rmi\qm_{jk}\widetilde{u}_3^{(jk)} }
		- \omega^2 \widetilde{u}_3^{(jk)}\overline{\psi^{(jk)}} \ \md y \\
	&= \sum_{j\con k}\sqbracs{ \overline{\psi^{(jk)}}\bracs{ \bracs{\widetilde{u}_3^{(jk)}}' - \rmi\wavenumber\widetilde{U}_2^{(jk)} + \rmi\qm_{jk}\widetilde{u}_3^{(jk)} } }_{v_j} \\
	&\quad + \sum_{j\con k}\int_0^{\abs{I_{jk}}} \overline{\psi^{(jk)}} \bracs{ -\bracs{\widetilde{u}_3^{(jk)}}'' + \rmi\wavenumber\bracs{\widetilde{U}_2^{(jk)}}' - \rmi\qm_{jk}\bracs{\widetilde{u}_3^{(jk)}}' } \ \md y \\
	&\quad + \sum_{j\con k}\int_0^{\abs{I_{jk}}} \overline{\psi^{(jk)}} \bracs{ - \rmi\qm_{jk}\bracs{\widetilde{u}_3^{(jk)}}' - \wavenumber\qm_{jk}\widetilde{U}_2^{(jk)} + \qm_{jk}^2\widetilde{u}_3^{(jk)} - \omega^2\widetilde{u}_3^{(jk)} } \ \md y \\
	&= \sum_{j\con k}\sqbracs{ \overline{\psi^{(jk)}}\bracs{ \bracs{\widetilde{u}_3^{(jk)}}' - \rmi\wavenumber\widetilde{U}_2^{(jk)} + \rmi\qm_{jk}\widetilde{u}_3^{(jk)} } }_{v_j},
\end{align*}
using \eqref{eq:QGPhi3Standalone} for the final step.
Recalling the map $r_{jk}$, and that $\phi_3$ is continuous at the vertex $v_j$, this implies that 
\begin{align*}
	\alpha_j \omega^2 u_3\bracs{v_j} \overline{\phi}_3\bracs{v_j} 
	&= \overline{\phi}_3\bracs{v_j} \bracs{ \sum_{j\con k} \bracs{ \pdiff{}{n} + \rmi\qm_{jk} }\widetilde{u}_3^{(jk)}\bracs{v_j} - \rmi\wavenumber\bracs{ \sum_{j\conRight k} \widetilde{U}_2^{(jk)} - \sum_{j\conLeft k} \widetilde{U}_2^{(jk)} } }. %\\
	%\implies \alpha_j \bracs{\omega^2 - \wavenumber^2} u_3\bracs{v_j} \overline{\phi}_3\bracs{v_j} &= \overline{\phi}\bracs{v_j} \sum_{j\con k} \bracs{ \pdiff{}{n} + \rmi\qm_{jk} }\widetilde{u}_3^{(jk)}\bracs{v_j},
\end{align*}
This holds for every such smooth $\phi$, and so (combined with \eqref{eq:QGPhiStandalone}) we arrive at the following system of equations;
%\begin{align*}
%	\alpha_j \bracs{\omega^2 - \wavenumber^2} u_3\bracs{v_j} &= \sum_{j\con k} \bracs{ \pdiff{}{n} + \rmi\qm_{jk} }\widetilde{u}_3^{(jk)}\bracs{v_j},
%\end{align*}
%at every vertex $v_j$.
\begin{subequations} \label{eq:QGRawSystem}
	\begin{align}
		\rmi\wavenumber \bracs{ \diff{}{y} + \rmi\qm_{jk} }\widetilde{u}_3^{(jk)} + \wavenumber^2\widetilde{U}_2^{(jk)} &= \omega^2\widetilde{U}_2^{(jk)}, \label{eq:QGPhi2} \\
		-\bracs{ \diff{}{y} + \rmi\qm_{jk} }^2\widetilde{u}_3^{(jk)} + \rmi\wavenumber\bracs{ \diff{}{y} + \rmi\qm_{jk} }\widetilde{U}_2^{(jk)} &= \omega^2 \widetilde{u}_3^{(jk)}, \label{eq:QGPhi3} \\
		\widetilde{u}_3 \text{ is continuous at } v_j &\quad\forall v_j\in\vertSet, \label{eq:QGContinuity} \\
		\sum_{j\con k} \bracs{ \pdiff{}{n} + \rmi\qm_{jk} }\widetilde{u}_3^{(jk)}\bracs{v_j} - \rmi\wavenumber\bracs{ \sum_{j\conRight k} \widetilde{U}_2^{(jk)} - \sum_{j\conLeft k} \widetilde{U}_2^{(jk)} } 
		&= \alpha_j \omega^2 u_3\bracs{v_j}. \label{eq:QGVertexCondition}
	\end{align}
\end{subequations}
The equations \eqref{eq:QGPhi2}-\eqref{eq:QGPhi3} can be written as a first-order system (on each edge)
\begin{align*}
	\diff{}{y} \begin{pmatrix} \widetilde{U}_2^{(jk)} \\ \widetilde{u}_3^{(jk)} \end{pmatrix}
	&= 
	\begin{pmatrix} 
		-\rmi\qm_{jk} & -\rmi\wavenumber \\	
		\frac{\omega^2-\wavenumber^2}{\rmi\wavenumber} & -\rmi\qm_{jk}
	\end{pmatrix}
	\begin{pmatrix} \widetilde{U}_2^{(jk)} \\ \widetilde{u}_3^{(jk)} \end{pmatrix},
\end{align*}
the differential equation in the second row being that which appears in (d), which is a consequence of $u$ being $\kt$-divergence-free with respect to $\dddmes$.
As such, the form of $\widetilde{U}_2^{(jk)}$ and $\widetilde{u}_3^{(jk)}$ is determined up to 2 constants per edge, giving us a total of $2\abs{\edgeSet}$ constants to be determined by the vertex conditions.
The condition \eqref{eq:QGContinuity} provides us with $\sum_{v\in\vertSet}\clbracs{\mathrm{deg}(v)-1}$ algebraic equations in these constants, whilst \eqref{eq:QGVertexCondition} provides us with an additional $\abs{\vertSet}$.
Since $2\abs{\edgeSet} = \sum_{v\in\vertSet}\mathrm{deg}(v)$, this means that \eqref{eq:QGRawSystem} has a unique solution for each $\omega, \wavenumber$ pair.
However we can eliminate $\widetilde{U}_2$ from \eqref{eq:QGRawSystem}, to obtain a system in terms of $\widetilde{u}_3$ only:
\begin{align*}
	- \bracs{ \diff{}{y} + \rmi\qm_{jk} }^2 u_3^{(jk)} - \bracs{\omega^2-\wavenumber^2}u_3^{(jk)} = 0, &\quad y\in\interval{I_{jk}}, \quad\forall I_{jk}\in\edgeSet, \\
	u_3 \text{ is continuous at each } & v_j\in\vertSet, \\
	\sum_{j\con k}\bracs{ \pdiff{}{n} + \rmi\qm_{jk} }u_3^{(jk)} = \bracs{\omega^2-\wavenumber^2}\alpha_j u_3\bracs{v_j}, &\quad\forall v_j\in\vertSet,
\end{align*}
that is the system \eqref{eq:3DQGFullSystem}.
To do so, we have differentiated \eqref{eq:QGPhi2} and substituted into \eqref{eq:QGPhi3} to obtain a second-order ODE in $\widetilde{u}_3$ on each edge.
Then, given that $u$ is divergence free, we have used the condition (e) to eliminate $\widetilde{U}_2^{(jk)}$ from the vertex conditions.
Of course, upon solving \eqref{eq:3DQGFullSystem} for $\widetilde{u}_3$, we can recover $\widetilde{U}_2^{(jk)}$ from \eqref{eq:QGPhi2} if we so wish.

The benefits of expressing the system as \eqref{eq:3DQGFullSystem} are tied to the utility of the $M$-matrix, which was introduced in section \ref{sec:QuantumGraphs}.
More precisely, the $M$-matrix reduces the task of determining the eigenvalues $\omega^2$ of \eqref{eq:3DQGFullSystem} (and thus of \eqref{eq:PeriodCellCurlCurlStrongForm}) to solution of a matrix-eigenvalue problem.
This will be the topic of discussion in section \ref{sec:3DDiscussion}.