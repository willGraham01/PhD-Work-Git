\section{Introduction} \label{sec:Intro}

\tstk{This section includes our literature review and motivation.
The purpose of the paper is to present the equations that were derived for the TFR/curl-curl system as what can be interpreted as an "effective" or "limit" problem for a fine/thin-structure material.
This is subject to future advances in the field that draw parallel to the scalar-case developments, of course.
This text should be deleted upon completion of the LitReview.}

\subsection{Lit Review} \label{ssec:LitReview}

\subsection{Physical motivation} \label{ssec:PhysMot}
The problem we consider in the present work is motivated by the desire to study wave propagation in a medium that exhibits a periodic (micro) structure.
Understanding the behaviour of waves in such media has applications in the field of photonic crystals, but also extends broadly into the study of wave behaviour in an elastic setting (see section \ref{ssec:LitReview}), although our motivations largely stem from the former.
Maxwell's equations are the typical starting point for describing the propagation of light,
\begin{subequations} \label{eq:MaxwellSystem}
	\begin{align} 
%		\grad \cdot \bracs{\epsilon E} &= \rho_{f}, \label{eq:E-div} \\
%		\grad \cdot \bracs{\mu H} &= 0, \label{eq:H-div} \\
%		\curl{} E &= -\pdiff{\bracs{\mu H}}{t}, \label{eq:E-curlEqn}\\
%		\curl{} H &= J_{f} + \pdiff{\bracs{\epsilon E}}{t}, \label{eq:H-curlEqn}
		\grad \cdot D &= \rho_{f}, \label{eq:D-div} \\
		\grad \cdot B &= 0, \label{eq:B-div} \\
		\curl{} E &= -\pdiff{B}{t}, \label{eq:E-curlEqn}\\
		\curl{} H &= J_{f} + \pdiff{D}{t}, \label{eq:H-curlEqn}
	\end{align}
\end{subequations}
where $E, B, D,$ and $H$ are the electric field, magnetic field, electric displacement field, and magnetising field respectively.
The parameters $\rho_{f}$ and $J_f$ denote the free charge density and current density respectively.
For a linear and isotropic material, these fields are related by the (constitutive) relations $E=\epsilon D$, and $B= \mu H$ where $\epsilon$ and $\mu$ are the electric permittivity and magnetic permeability of the material.
Generally (and in almost all real-world materials), the constitutive relations are only approximately linear, and $\epsilon$ and $\mu$ are non-constant --- potentially depending on the frequency of the propagating waves \tstk{ref to Lorenzt/Drude model} and/or varying in time.
With $\epsilon$ and $\mu$ constant in time and space, and in the absence of free charges and currents, one can take the curl of \eqref{eq:E-curlEqn} (respectively \eqref{eq:H-curlEqn}) and substitute into \eqref{eq:H-curlEqn} (respectively \eqref{eq:E-curlEqn}) the to obtain the ``curl-of-the-curl" equation
\begin{align*}
	- \curl{}\bracs{ \curl{} \hat{u} } &= \epsilon\mu\pddiff{ \hat{u} }{t},
\end{align*}
where the vector field $\hat{u} = H$ (respectively $\hat{u} = E$).
Further assuming a time-harmonic field $\hat{u}(x,t) = e^{-\rmi\omega t}u(x)$ for some frequency $\omega$, one obtains the (eigenvalue) equation
\begin{align} \label{eq:CurlCurlEqn}
	\curl{}\bracs{ \curl{} u } &= zu,
\end{align}
where $z = \epsilon\mu\omega^2$.
\tstk{can also derive this in a more general way right? Using gauges and potentials - should probably give this derivation if possible, Jackson or Cassent might have it in full.
Basically, set $B = \curl{}A$, then \eqref{eq:E-curlEqn} implies $\curl{}(E+\partial_t A) = 0$ so there's some potential $\phi$ such that $-\grad\phi = E + \partial_t A$.
Substitute this into \eqref{eq:H-curlEqn}, then absorb the hanging $\partial_t\grad\phi$ term into $A$ to remove it (via gauge freedoms). 
This leaves you with $\curl{}\curl{}A + \epsilon\mu\partial_t^2 A = \mu J_f$, and you can even use Ohm's law to set $J_f = \sigma E$ to play around with removing the $J_f$ too.
}
We will set $\epsilon=\mu=1$ in this work, and thus write $z=\omega^2$ for the spectral parameter.
This also retains an intuitive link to applications: the spectrum (that is, the set of $z=\omega^2$ that possess an eigenfunction which solves \eqref{eq:CurlCurlEqn}) provides access to the frequencies of waves that can propagate in the singular-structures we examine, which determines the use of the structure itself as a waveguide.
Our focus will be on quantitatively encapsulating the structure of the spectrum for all problems of the form \eqref{eq:CurlCurlEqn} on the class of domains that we consider.

\subsection{Problem formulation} \label{ssec:OurSystem}
We shall now outline the problem to be addressed.
The reader may wish to refer to section \ref{sec:QuantumGraphs} and the appendix (sections \ref{app:MeasureTheory}-\ref{app:SumMeasureAnalysis}) for a precise description of the objects that are mentioned in the discussion below. 

Our work concerns the study of equations of the form \eqref{eq:CurlCurlEqn} on singular domains, that is domains which have no interior from the perspective of the space they are embedded into.
We represent such a singular structure by a graph $\graph$ in $\reals^2$, which we will assume to be periodic in the sense that there exists a region (period cell) $\ddom\subset\reals^2$ and vectors $p_1, p_2$ such that for any $z\in\integers$ the part of $\graph$ contained in the shifted regions $\ddom+p_1$ and $\ddom+p_2$ coincides with the part of $\graph$ contained in $\ddom$.
We call this part of $\graph$ in $\ddom$ as the \emph{period graph} of $\graph$, and denote it by $\graph_{\mathcal{P}}=\bracs{\vertSet, \edgeSet}$, where $\vertSet$ is a finite set of vertices and $\edgeSet$ a finite set of edges.
The graph $\graph$ is then ``extruded" into 3 dimensions to form a domain that consists of a union of planes, $\ddom\times\bracs{0,\infty}$, which is a 3-dimensional singular structure.
For the singular structure above define the gradient operator $\grad_{\dddmes}$, and consider the ``spectral problem"
\begin{align} \label{eq:WholeSpaceCurlCurl}
	\curl{\dddmes} \bracs{ \curl{\dddmes}u } = \omega^2 u.
\end{align}
Here the subscript $\dddmes$ stands for the sum of the singular measure supported by the planes induced by $\graph$ and point mass measures at centred at the vertices of $\graph$, the details of which are available in the appendix (section \ref{app:SingularMeasures}).
It is convenient to take a Fourier transform along in the $x_3$ direction, and replace \eqref{eq:WholeSpaceCurlCurl} with a family of problems on $\ddom$ parametrised by $\wavenumber$ (the Fourier variable, representing a ``wavenumber"), and $\qm$ (the ``quasi-momentum") which varies over the dual-cell of $\ddom$.
This provides us with a family of problems involving ``$\kt$-shifted" gradients $\ktgrad_{\dddmes}$ (see section \ref{app:SMandSGO}),
\begin{align}  \label{eq:PeriodCellCurlCurlStrongForm}
	\ktcurl{\dddmes} \bracs{ \ktcurl{\dddmes} u } = \omega^2 u, & \quad\text{in } \ddom.
\end{align}
The link between \eqref{eq:WholeSpaceCurlCurl} and \eqref{eq:PeriodCellCurlCurlStrongForm} is established by means of a version of the so-called Gelfand transform \cite{gelfand1950expansion}
\begin{align*}
	\hat{u}\bracs{x} &= \sum_{z_1,z_2\in\integers^2}u\bracs{x+zp_1+zp_2}\e^{-\rmi\qm\bracs{x+zp_1+zp_2}}.
\end{align*}
It can then be shown that the spectrum of the problem \eqref{eq:PeriodCellCurlCurlStrongForm} can be determined from the spectra of the following family of problems (see section \ref{sec:3DSystemDerivation}):
\begin{subequations} \label{eq:3DQGFullSystem}
	\begin{align}
		- \bracs{ \diff{}{y} + \rmi\qm_{jk} }^2 u_3^{(jk)} - \bracs{\omega^2-\wavenumber^2}u_3^{(jk)} = 0, &\quad y\in\interval{I_{jk}}, \quad\forall I_{jk}\in\edgeSet, \\
		u_3 \text{ is continuous at each } & v_j\in\vertSet, \\
		\sum_{j\con k}\bracs{ \pdiff{}{n} + \rmi\qm_{jk} }u_3^{(jk)} = \bracs{\omega^2-\wavenumber^2}\alpha_j u_3\bracs{v_j}, &\quad\forall v_j\in\vertSet. \label{eq:3DQGDerivCondition}
	\end{align}
\end{subequations}
The $\alpha_j\in\complex$ are suitable ``coupling constants" at the vertices, and $\qm_{jk}$ are effectively rotations of $\qm$ which can be computed if one knows the orientation of the edge $I_{jk}\in\vertSet$.
The notation $u_3^{(jk)}$ in \eqref{eq:3DQGFullSystem} denotes the restriction of the component $u_3$ to the edge $I_{jk}$ of $\graph_{\mathcal{P}}$, and the ``signed derivative" $\pdiff{}{n}$ is defined in section \ref{ssec:FunctionSpaces}.
Problems like \eqref{eq:3DQGFullSystem} belong to the class of problems with generalised resolvents, since the spectral parameter $\omega^2$ appears in the boundary condition \eqref{eq:3DQGDerivCondition} \cite{strauss1954generalized, strauss1968extensions, strauss1998functional, cherednichenko2018effective}.
We analyse the spectrum of \eqref{eq:WholeSpaceCurlCurl}, equivalently \eqref{eq:3DQGFullSystem}, in terms of the geometry of the structure and the constants $\alpha_j$.
We provide an overview of how \eqref{eq:3DQGFullSystem} can be obtained from \eqref{eq:PeriodCellCurlCurlStrongForm} in section \ref{sec:3DSystemDerivation}, and precise details of the objects involved can be found in the appendices \ref{app:3DMuAnalysis}-\ref{app:SumMeasureAnalysis}.