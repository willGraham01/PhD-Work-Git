\documentclass{beamer}
 
%presentation pre-amble
%The idea behind this file is that it will be used to store all the maths-related macros that I concoct; so that I can import all the commands by \input{this file} in the preamble of any file that I want to use them in.
%This should make the top-level files look a lot cleaner, and the preamble much shorter!

\usepackage{amssymb}
\usepackage{amsmath}
%\usepackage{mathtools}

%begin the macros via newcommand. Try to group them up reasonably!

%standard sets
\newcommand{\naturals}{\mathbb{N}}			%natural numbers
\newcommand{\integers}{\mathbb{Z}}			%integers
\newcommand{\rationals}{\mathbb{Q}}			%rational numbers
\newcommand{\reals}{\mathbb{R}}				%real numbers
\newcommand{\complex}{\mathbb{C}}			%complex numbers

%brackets and norms
\newcommand{\bracs}[1]{\left( #1 \right)}				%encloses input in brackets
\newcommand{\sqbracs}[1]{\left[ #1 \right]}				%encloses input in square brackets
\newcommand{\clbracs}[1]{\left\{ #1 \right\}}			%encloses input in curly bracers
\newcommand{\abs}[1]{\lvert #1 \rvert}					%absolute value
\newcommand{\norm}[1]{\lvert\lvert #1 \rvert\rvert}		%norm 

%function sets
\newcommand{\smooth}[1]{C^{\infty}\bracs{#1}}							%smooth functions
\newcommand{\ltwo}[2]{L^{2}\bracs{#1,\mathrm{d}#2}}						%general L^2 space
\newcommand{\gradSob}[2]{H^1_\mathrm{grad}\bracs{#1, \mathrm{d}#2}}		%gradient Sobolev space
\newcommand{\curlSob}[2]{H^1_\mathrm{curl}\bracs{#1, \mathrm{d}#2}}		%curl Sobolev space
\newcommand{\kSob}[2]{H^1_{k,\mathrm{curl}}\bracs{#1, \mathrm{d}#2}}	%k-curl Sobolev space
\newcommand{\supp}{\mathrm{supp}}										%support of a function

%grad and curl sets
\newcommand{\gradZero}[2]{\mathcal{G}_{ #1, \mathrm{d}#2}\bracs{0}}		%gradients of zero for domain #1 with measure #2
\newcommand{\curlZero}[2]{\mathcal{C}_{ #1, \mathrm{d}#2}\bracs{0}}	%curls of zero for domain #1 with measure #2
\newcommand{\kcurlZero}[2]{\mathcal{C}_{ #1, \mathrm{d}#2}^{(k)}\bracs{0}}	%k-curls of zero for domain #1 with measure #2

%derivatives and grad-like symbols
\newcommand{\diff}[2]{\dfrac{\mathrm{d}#1}{\mathrm{d}#2}}			%complete derivative d#1/d#2
\newcommand{\pdiff}[2]{\dfrac{\partial #1}{\partial #2}}			%partial derivative p#1/p#2
\newcommand{\ddiff}[2]{\dfrac{\mathrm{d}^2 #1}{\mathrm{d}^2 #2}}	%2nd deriv
\newcommand{\grad}{\nabla}											%grad operator
\newcommand{\curl}[1]{\grad^{(#1)}\wedge}							%curl with superscript #1
\newcommand{\kcurl}[1]{\grad_{#1}^{(k)}\wedge}						%k-curl with measure subscript #1

%displaying integrals
\newcommand{\integral}[3]{\int_{#1}#2 \ \mathrm{d}#3}			%integral, domain #1, integrand #2, measure #3
\newcommand{\md}{\ \mathrm{d}}									%differential d

%convergence
%\newcommand{\lconv}[1]{\xrightarrow{#1}}							%convergence with #1 above the rightarrow - requires mathtools
\newcommand{\lconv}[1]{\overset{#1}{\longrightarrow}}				%convergence with #1 above the rightarrow
\newcommand{\toInfty}[1]{ \ \text{as} \ #1 \rightarrow\infty}		%writes out "as #1 tends to infty"

%notation and variable use throughout the file
\renewcommand{\vec}[1]{\mathbf{#1}}				%vectors are bold, not overline arrow
\newcommand{\recip}[1]{\frac{1}{#1}}			%fast reciprocal as a fraction

\newcommand{\dddom}{\widetilde{\Omega}}			%3D domain notation
\newcommand{\ddom}{\Omega}						%2D domain notation
\newcommand{\dddmes}{\widetilde{\mu}}			%3D measure
\newcommand{\ddmes}{\mu}						%2D measure

\newcommand{\graph}{\mathbb{G}}					%graph variable
%Preamble file for all presentation styles but not including maths formatting!

\usepackage[utf8]{inputenc}

\usepackage{graphicx}
\usepackage{subcaption}
\usepackage{tikz}
\usetikzlibrary{arrows}

\usetheme{Madrid}
\usecolortheme{default}
 
%removes figure prefix from captions
\setbeamertemplate{caption}{\raggedright\insertcaption\par}

%sets the directory for image files - note: this assumes the file structure as of 14-10-2019
\graphicspath{{../Diagrams/Diagram_PDFs/} {../Images/}}

\usepackage{soul} %strikethrough text via \st
\usepackage{graphicx}
\graphicspath{{./PSS_Diagrams/}}

\newcommand{\eps}{\varepsilon}
\newcommand{\sM}{\mathcal{M}}
\newcommand{\borel}{\mathcal{B}}
\newcommand{\closedInts}{\mathcal{H}}
\newcommand{\openInts}{\mathcal{I}}

%Information to be included in the title page
\title{What is Measure Theory, and How does it Affect Me?}
\author{William Graham}
\institute{Postgraduate Seminar Series}
\date{\today}
 
\begin{document}
 
\frame{\titlepage}
 
%insert a contents or overview slide at the end
\begin{frame}
	\frametitle{Talk overview}
	
	\begin{columns}
		\begin{column}{0.45\textwidth}
			\begin{enumerate}
				\item Motivation and Objectives
				\item A New Area
				\item What we can't Measure
				\item New Area, New Integral
				\item Why was this Useful?
				\item Conclusion and Freedom
			\end{enumerate}
			\visible<2>{
			\alert{Minimal technical detail is promised.}
			}
		\end{column}
		\begin{column}{0.45\textwidth}
			\visible<2>{
			\begin{figure}
				\centering
				\includegraphics[scale=0.2]{PSS_Disclaimer-GondorGate.pdf}
			\end{figure}
			}
		\end{column}
	\end{columns}
\end{frame}

%Motivation and Objectives
\begin{frame}
	\frametitle{Motivation and Objectives}
	
	\begin{block}{Motiviation}
		\begin{itemize}
			\visible<1->{\item Have you ever wondered why PDFs are such a big deal when working in probability?}
			\visible<2->{\item Have you ever used FEM or Fourier transforms to solve a problem?}
			\visible<3->{\item Have you ever tried to break up a chocolate bar, and then reassemble the pieces into two chocolate bars of equal size to the first????}
		\end{itemize}
	\end{block}
	
	\visible<4->{This talk aims to address all these questions, whilst guiding you through the \st{nightmare hellscape} pleasant field of measure theory.}
	
\end{frame} 

%Introductory concepts 1
\begin{frame}
	\frametitle{New Beginnings}
	
	Measure Theory aims to solidify our understanding of bulk, size, or volume.
	First though, we need to decide on what we are going to measure...
	\visible<2->{
	\begin{block}{Starting Concepts}
		Let $X$ be a set.
		Let $\sM$ be a collection of subsets of $X$.
		Then $\sM$ is a $\sigma$-algebra in $X$ if,
		\begin{itemize}
			\item $\emptyset\in\sM$
			\item $\sM$ is closed under compliments and (countable!) unions.
		\end{itemize}
	\end{block}
	\alert{$\sM$ is the collection of sets we want to be able to say ``have a size"}.	
	}
	\visible<3->{
	\begin{block}{Generated $\sigma$-algebras, $\sigma\bracs{M}$}
		For a collection $M$ of subsets of $X$, we define $\sigma\bracs{M}$ to be the smallest $\sigma$-algebra that contains all the sets in $M$.
	\end{block}
	}
\end{frame}

%Introductory concepts 2
\begin{frame}
	\frametitle{Borel}
	
	The most important $\sigma$-algebra is the Borel $\sigma$-algebra in $\overline{\reals}$, $\borel$.
	\begin{block}{$\borel$}
		Let $\closedInts$ be the set of all the closed sets in $\overline{\reals}$, then
		\begin{align*}
			\borel &:= \sigma\bracs{\closedInts}.
		\end{align*}
		In particular, $\borel$ contains all the closed sets, and all the open sets.
		This means we can measure (or find the ``length" of) these sets!
	\end{block}
	
	\begin{figure}
		\centering
		\includegraphics[scale=1]{ObtainingBorelSets.pdf}
	\end{figure}
\end{frame}

%Introduce measures
\begin{frame}
	\frametitle{Measures}

	\only<1-3>{
	\begin{block}{Measure}
		A measure is a function $\mu:\sM\rightarrow\overline{\reals}$ that sends a set to a number. \newline
		That number is effectively the ``volume" of that set. 
		\only<1>{ \newline
		Properties of $\sM$ translate into $\mu$:
		\begin{align*}
			\mu\bracs{\emptyset} &= 0, \\
			\mu\bracs{A \cup B} &= \mu\bracs{A} + \mu\bracs{B} - \mu\bracs{A\cap B}, \\
			\mu\bracs{X\setminus A} &= \mu\bracs{X} - \mu\bracs{A}.
		\end{align*}
		}
	\end{block}
	\visible<2-3>{
	If $\sM$ is generated, then we can get away with only defining our measure on the generating set.
	This is how the Lebesgue measure is constructed:
	}
	}
	\only<3-4>{
	\visible<3->{
	\begin{block}{Lebesgue Measure}
		Let $\lambda^*:\closedInts\rightarrow\overline{\reals}$ assigns length to the closed sets by
		\begin{align*}
			\lambda^*\bracs{\sqbracs{a,b}} &= b-a.
		\end{align*}
		$\lambda^*$ can then be \emph{uniquely extended} to $\borel=\sigma\bracs{\closedInts}$, forming the Lebesgue measure $\lambda:\borel\rightarrow\overline{\reals}$, with
		\begin{align*}
			\lambda\bracs{\sqbracs{a,b}} &= \lambda^*\bracs{\sqbracs{a,b}}, \quad \forall \sqbracs{a,b}\in\closedInts.
		\end{align*}
	\end{block}
	}
	}
	\visible<4-5>{
	\begin{itemize}
		\item $\closedInts$ is what we know how to measure (with $\lambda^*$),
		\item $\borel$ is the collection of things we can measure by combining sets in $\closedInts$,
		\item $\lambda$ is the resulting ``method of measuring" sets!
	\end{itemize}
	}
	\only<5->{
	\alert{This is even constructive:}
	\begin{align*}
		\lambda\bracs{\bracs{a,b}} &= \lambda\bracs{\sqbracs{a,b}\setminus\bracs{\clbracs{a} \cup \clbracs{b}}} \\
		&= \lambda\bracs{\sqbracs{a,b}} - \bracs{ \lambda\bracs{\clbracs{a}} - \lambda\bracs{\clbracs{b}}} \\
		&= b - a - \bracs{ \lambda\bracs{\sqbracs{a,a}} + \lambda\bracs{\sqbracs{b,b}} } \\
		&= b - a - \bracs{ (a-a) + (b-b)} \\
		&= b - a.
	\end{align*}
	It's actually very hard to construct something we \emph{can't} Lebesgue-measure, but there is a famous example...
	}
\end{frame}


%B-T paradox, non-Lebesgue measurable is ``not physically viable"
\begin{frame}
	\frametitle{Banach-Tarski}
	
	\begin{block}{Banach-Tarski Paradox, 1924}
		A sphere can be broken into finitely many pieces, which can then be reassembled into two identical copies of the original sphere.
	\end{block}
	
	\only<1>{
	\begin{figure}
		\centering
		\includegraphics[scale=0.25]{BT-Meme.pdf}
	\end{figure}
	}
	\only<2>{
	So why can't I create infinite chocolate from one chocolate bar?
	\begin{itemize}
		\item The ``pieces" are not in (the 3D) $\borel$
		\item $\borel$ contains all the sets that are physically measurable
		\item $\implies$ The notion of volume doesn't make sense for these ``pieces".
	\end{itemize}
	\alert{The pieces are ``non-Lebesgue measurable", you couldn't ever assemble them in real life (which sadly means no infinite chocolate :( ).}
	}
	
\end{frame}

%Measure 0 and a.e.
\begin{frame}
	\frametitle{Sets of Measure 0}
	
	\only<1-2>{
	We saw that the Lebesgue measure of a single point is 0, $\lambda\bracs{\clbracs{a}}=0$.
	\begin{block}{Zero Measure}
		For a measure $\mu$ and set $E\in\sM$, if $\mu\bracs{E} = 0$ then $E$ is a ``set of $\mu$-measure 0".
	\end{block}
	Sets of 0 measure are effectively inconsequential when considering volumes.
	}
	\visible<2->{
	\begin{block}{Almost everywhere}
		We say something happens ``($\mu$)-almost everywhere" (a.e.) if it happens everywhere except on a set of zero $\mu$-measure. 
		\only<2>{\newline\alert{If you are a probabilist, you say ``almost surely" (a.s.).}}
	\end{block}
	}
	\only<3-4>{
	Examples include: \newline
	}
	\only<3>{
	The Heaviside-function 
	\begin{align*}
		H:\reals\rightarrow\sqbracs{0,1}, &\quad H(x) = \begin{cases} 0 & x< 0 \\ 1 & x\geq 1 \end{cases}
	\end{align*}
	is continuous at all $x\in\reals\setminus\clbracs{0}$.
	$\clbracs{0}$ is a set of zero Lebesgue measure, so $H$ is ``$\lambda$-almost-everywhere continuous".
	}
	\only<4>{
	The Dirichlet function
	\begin{align*}
		d:\reals\rightarrow\sqbracs{0,1}, &\quad d(x)  = \begin{cases} 0 & x\in\rationals \\ 1 & x\in\reals\setminus\rationals \end{cases}
	\end{align*}
	is ``$\lambda$-almost everywhere equal to $1$".
	This is because we can write 
	\begin{align*}
		\rationals &= \bigcup_{q\in\rationals}\clbracs{q} &\quad\text{(a countable union of single points)},
	\end{align*}
	so
	\begin{align*}
		\lambda\bracs{\rationals} &= \sum_{q\in\rationals}\lambda\bracs{\clbracs{q}} = \sum_{q\in\rationals} 0 = 0.
	\end{align*}
	}
	
\end{frame}

%Measurable and simple functions
\begin{frame}
	\frametitle{Some Functionality}
	
	\only<1>{
	We have redefined ``volume", so naturally we now need to rework integration (``volume under a curve"). \newline
	Our measures let us integrate a wider class of functions:
	\begin{block}{Measurable functions}
		$f:X\rightarrow Y$ is a measurable function between $X$ and $Y$ if the preimage of any measurable set in $Y$ is a measurable set in $X$.
	\end{block}
	First, we try to define integration for some simple (literally) functions...
	}
	\only<2->{
	\begin{block}{Simple Functions}
		$f:X\rightarrow\reals$ is a simple function if $f(X)$ is a finite set.
		We can write $f$ as
		\begin{align*}
			f = \sum_i \alpha_i \charFunc{E_i},
		\end{align*}
		for $\clbracs{E_i}_i$ a partition of $X$, $\alpha_i\in\reals$, and $\charFunc{E_i}$ being the characteristic function of $E_i$.
	\end{block}
	
	\begin{figure}
		\centering
		\includegraphics[scale=0.5]{SimpleFunctionPartition.pdf}
	\end{figure}
	}
\end{frame}

%integration
\begin{frame}
	\frametitle{Int-it-grate?}
	
	\only<1>{
	\begin{block}{Integral of Simple Functions}
		For a simple function $f = \sum_i \alpha_i \charFunc{E_i}$, the integral of $f$ over $X$ wrt $\mu$ is
		\begin{align*}
			\integral{X}{f}{\mu} &= \sum_{i} \alpha_{i}\mu\bracs{E_i},
		\end{align*}
		or over a measurable subset $E\in\sM$ of $X$ is
		\begin{align*}
			\integral{E}{f}{\mu} &= \integral{X}{f\charFunc{E}}{\mu}.
		\end{align*}
	\end{block}
	}
	\only<2>{
	\begin{figure}
		\centering
		\includegraphics[scale=0.85]{SimpleFunctionIntegral.pdf}
	\end{figure}
	}
	
	\only<3>{
	Idea: integrating simple functions is easy, so why don't we recycle this for measurable functions?
	\begin{block}{Simple functions approximate Measurable functions}
		For any measurable function $f$ we can find a sequence of simple functions $f_n$ such that
		\begin{align*}
			f_n \rightarrow f, &\quad n\rightarrow\infty.
		\end{align*}
		We say ``Simple functions are dense in Measurable functions". \newline
		So we can define:
		\begin{align*}
			\integral{E}{f}{\mu} &= \lim_{n\rightarrow\infty}\integral{E}{f_n}{\mu}
		\end{align*}
	\end{block}
	}

\end{frame}

%riemann vs lebesgue
\begin{frame}
	\frametitle{Riemann vs Lebesgue}

	\begin{columns}
		\begin{column}{0.45\textwidth}
			\only<1>{
			\begin{figure}
				\centering
				\includegraphics[scale=0.65]{RiemannIntegration.pdf}
				\caption{Riemann Integration}
			\end{figure}		
			}
			\only<2>{
			\begin{figure}
				\centering
				\includegraphics[scale=0.175]{IntegrationGalaxyBrain.pdf}
			\end{figure}
			}
		\end{column}
		\begin{column}{0.45\textwidth}
			\begin{figure}
				\centering
				\includegraphics[scale=0.65]{MeasureTheoryIntegration.pdf}
				\caption{Lebesgue Integration}
			\end{figure}
		\end{column}
	\end{columns}
\end{frame}

%what was the point (fourier, fem, sobolev spaces)
\begin{frame}
	\frametitle{Why did we bother with that?}

	\only<1-3>{
	This notion of integration lets us integrate a wider range of functions, and define some very useful function spaces.
	\begin{block}{$L^p$ spaces}
		Let $\sim$ be the equivalence relation
		\begin{align*}
			f \sim g \quad\Leftrightarrow\quad f = g \ \mu\text{-a.e.}
		\end{align*}
		\visible<2-3>{
		Let $L^p\sqbracs{\mu}$ be the set of equivalence classes for $\sim$. \newline
		}
		\visible<3>{
		Then for $p\in\left[1,\infty\right)$, the set $L^p\sqbracs{\mu}$ equipped with the norm
		\begin{align*}
			\norm{f}_p &= \bracs{ \integral{X}{f^p}{\mu} }^{\recip{p}}
		\end{align*}		
		is called $L^p\bracs{X,\md\mu}$.
		}
	\end{block}
	}
	\only<4>{
	You have probably all met $L^2$ before - ``the space of square-integrable functions". \newline
	Well not quite, it's the space of equivalence classes of \emph{almost-identical} square-integrable functions.
	\begin{itemize}
		\item Measure theory tells us that functions which are the same a.e. will have the same integral
		\item $L^p$ is a Banach space
		\item $L^2$ is a Hilbert space (these spaces are pretty damn nice)
	\end{itemize}
	Without this equivalence class treatment, we can't obtain a Hilbert space for our functions.
	}
	\only<5>{
	This means that some of your favourite things wouldn't be usable:
	\begin{itemize}
		\item The Fourier transform is only invertible on ``functions" in $\ltwo{\reals}{\lambda}$, so solving PDEs via Fourier transform would be hard if you couldn't undo it at the end!
		\item Convergence of Finite Element Methods requires orthonormal decompositions of function spaces, which we can only be guaranteed to be able to do in a Hilbert space.
		\item Most results on existence, uniqueness and regularity of PDEs require Sobolev spaces, which are built on $L^p$ spaces.
	\end{itemize}
	But let's get to the crux of why we bothered with all that...
	}
\end{frame}

%abs. continuity, and pdfs
\begin{frame}
	\frametitle{Absolute Continuity and Probability}
		
	\only<1-2>{
	Measures, if nice enough, can ``be translated into each other".
	\begin{block}{Absolute continuity}
		$\mu$ is absolutely continuous w.r.t. $\lambda$ (written $\mu \ll \lambda$) if there is some measurable function $F:X\rightarrow\overline{\reals}$ such that
		\begin{align*}
			\mu\bracs{E} &= \integral{E}{F}{\lambda}, \quad\forall \mu\text{-measurable sets }E\in\sM.
		\end{align*}
	\end{block}
	}
	\only<1>{
	\begin{figure}
		\centering
		\includegraphics[scale=0.2]{AbsCont-Meme.pdf}
	\end{figure}
	}
	\only<2>{
	This function $F$ allows us to translate unintuitive descriptions of size into something that is familiar to us - the Lebesgue measure!
	}
	
	\only<3-4>{
	\begin{block}{Probability}
		A probability space $\bracs{\Omega, \mathcal{F}, \mathbb{P}}$ is a set of all possible outcomes $\Omega$, possible events $\mathcal{F}$, and a mysterious function $\mathbb{P}$. \newline
		\visible<4->{$\mathbb{P}$ is assigning events in $\mathcal{F}$ a numerical value - it is a measure in disguise!}
	\end{block}
	}
	\only<5-7>{
	We obtain quantities like the expectation and variance of a random variable $T$ via integrals involving it's probability density function $F_{T}$:
	\begin{align*}
		\mathbb{E}\sqbracs{T} &= \integral{\reals}{t F_{T}(t)}{t}.
	\end{align*}
	\visible<6-7>{
	But if $\mathbb{P}\ll\lambda$, then we can write
	\begin{align*}
		\mathbb{P}\bracs{T} &= \integral{T}{F_{T}}{\lambda},
	\end{align*}
	}
	\visible<7>{
	so the expectation is actually us performing
	\begin{align*}
		\mathbb{E}\sqbracs{T} &= \integral{T}{t}{\mathbb{P}}.
	\end{align*}
	}
	}
	\only<8->{
	\alert{We are just translating the idea of probabilistic ``size" into ``physical" size, in order to be able to do our calculations!} \newline
	
	This would bring us nicely towards the realm of Stochastic Calculus, but I figured that here was a good place to stop before I inflict too much suffering on everyone today.
	}
\end{frame}

\begin{frame}
	\frametitle{Conclusion}
	
	\only<1>{
	\begin{block}{In summary}
		\begin{itemize}
			\item $X$ is our whole space, $\sM$ is what we want to measure
			\item The Lebesgue measure is our intuitive, real-world sense of size
			\item Integration is still done by counting squares, but we can do it on a wider range of functions
			\item The ``function spaces" $L^p$ are Banach/Hilbert spaces, allowing us to define tools for solving PDEs
			\item Probability spaces are just measure spaces, and PDFs translate ``probability size" into ``physical size" for us to work with. 
		\end{itemize}
	\end{block}
	}
	
	\only<2>{
	Thank-you all for listening and not running away... Any questions?
	\begin{figure}
		\centering
		\includegraphics[scale=0.35]{FrodoOverDone-Meme.jpg}
	\end{figure}
	}
\end{frame}

\end{document}
