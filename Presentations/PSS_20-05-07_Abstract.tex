\documentclass[11pt]{report}

\usepackage{url}
\usepackage[margin=2.5cm]{geometry} % See geometry.pdf to learn the layout options. There are lots.
\geometry{a4paper} %or letterpaper or a5paper or ...


%-------------------------------------------------------------------------
%DOCUMENT STARTS

\begin{document}

\chapter*{PDEs on Singular Structures}

PDEs are very popular choices for modelling various physical processes; such as wave propagation, fluid dynamics, and material deformation.
Whilst posing a PDE driven model for a process, we are often limited in the analysis we can perform due to the shape of the domain on which we are working - this is particularly apparent when the domain we want to use is very ``thin", or involves phenomena like cracks or laminates.
Thus we come to the idea of approximating such processes by treating this ``thin structure" as ``singular" (having no width) and asking the question as to whether we can actually make sense of this.
In this talk I will (attempt to) guide you through the thought process that gave rise to the idea of ``PDEs on singular structures", as well as some of the challenges that had to be overcome, and indeed what kinds of problems we end up with if we follow this analysis through. \newline

Disclaimer: there will be analysis, however I promise to be on my best behaviour and tone this down as much as possible.
So long as you remember what a PDE is, you should be fine (so long as you can stand the sound of my voice for an hour).

\end{document}