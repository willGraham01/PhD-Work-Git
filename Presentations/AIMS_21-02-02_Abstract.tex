\documentclass[11pt]{report}

\usepackage{url}
\usepackage[margin=2.5cm]{geometry} % See geometry.pdf to learn the layout options. There are lots.
\geometry{a4paper} %or letterpaper or a5paper or ...

\usepackage{graphicx}


%-------------------------------------------------------------------------
%DOCUMENT STARTS

\begin{document}

\chapter*{Considerations for Quantum Graph Problems Motivated by Photonic Crystals}

Photonic crystals can be thought of as thin, periodic structures that have found applications in the design of waveguides.
This is primarily because the geometry of these crystal structures gives rise to ``frequency band gaps" - ranges of frequencies that cannot support wave propagation in the crystal - meaning one can use the crystal as a cladding for another material that does support waves at those frequencies.
However, determining the manner in which the crystal's geometry influences these ``band gaps" is difficult - as one is typically looking at analysing Maxwell's equations on a geometrically complex and thin domain to find them.
This begs the question; since the crystal structure is typically thin, would it be unreasonable to approximate the structure as singular (having a thickness of zero)?
Doing so means we can borrow ideas from the theory of Quantum Graphs, essentially a framework for differential equations on graph-like structures, obtaining something that's easier to deal with than a system of PDEs (like Maxwell).
In this talk I will discuss the treatment of these ``singular-structures" motivated by photonic crystals, the Quantum Graph problems that they lead us to, and the tools we can employ to handle them.
We will see that the Quantum Graph problems we obtain explicitly encode the geometry of the structure; and give us access to a tool that allows us (both analytically and numerically) to obtain information about the ``band-gaps" of these structures.

\end{document}