\documentclass[11pt]{report}

\usepackage{url}
\usepackage[margin=2.5cm]{geometry} % See geometry.pdf to learn the layout options. There are lots.
\geometry{a4paper} %or letterpaper or a5paper or ...

%for figures and graphics
\usepackage{graphicx}
\usepackage{subcaption} %allows subfigures
\usepackage[bottom]{footmisc} %footnotes go below figures
\usepackage{tikz}
%\usepackage{parskip} %adds line space between paragraphs by default
\usepackage{enumerate} %allows lower case roman numerials in enumerate environments

\DeclareGraphicsRule{.tif}{png}{.png}{`convert #1 `dirname #1`/`basename #1 .tif`.png}
\graphicspath{{../Diagrams/Diagram_PDFs/} {../Diagrams/Numerical_Results/}}

%\input imports all commands from the target files
%The idea behind this file is that it will be used to store all the maths-related macros that I concoct; so that I can import all the commands by \input{this file} in the preamble of any file that I want to use them in.
%This should make the top-level files look a lot cleaner, and the preamble much shorter!

\usepackage{amssymb}
\usepackage{amsmath}
%\usepackage{mathtools}

%theorems and lemma etc setup using amsthm
\usepackage{amsthm}
\theoremstyle{definition}
\newtheorem{definition}{Definition}[section]
\theoremstyle{plain}
\newtheorem{theorem}{Theorem}[section]
\theoremstyle{plain}
\newtheorem{lemma}[theorem]{Lemma}
\theoremstyle{plain}
\newtheorem{prop}[theorem]{Proposition}
\theoremstyle{plain}
\newtheorem{cory}[theorem]{Corollary}
\theoremstyle{definition}
\newtheorem{convention}[theorem]{Convention}
\theoremstyle{definition}
\newtheorem{assumption}[theorem]{Assumption}
\theoremstyle{definition}
\newtheorem{conjecture}[theorem]{Conjecture}

\allowdisplaybreaks %allows equations in the same align environment to split over multiple pages.

%tstk always need be there
\newcommand{\tstk}[1]{\textbf{#1} \newline}

%this adds extra functionality to pmatrix, vmatrix, bmatrix etc by allowing you to pass an optional argument in [FACTOR] to multiply the default spacing between elements by FACTOR
\makeatletter
\renewcommand*\env@matrix[1][\arraystretch]{%
  \edef\arraystretch{#1}%
  \hskip -\arraycolsep
  \let\@ifnextchar\new@ifnextchar
  \array{*\c@MaxMatrixCols c}
  }
\makeatother

%begin the macros via newcommand. Try to group them up reasonably!

%notation and variable use throughout the file
\renewcommand{\vec}[1]{\mathbf{#1}}				%vectors are bold, not overline arrow
\newcommand{\recip}[1]{\frac{1}{#1}}			%fast reciprocal as a fraction
\newcommand{\interval}[1]{\sqbracs{0,\abs{#1}}}	%creates the closed interval from 0 to the length of the input #1, denoted by absolute value
\newcommand{\eps}{\varepsilon}					%pretty epsilons
\newcommand{\charFunc}[1]{\mathcal{I}_{#1}}%{\mathbb{1}_{#1}}		%characteristic function of a set

\newcommand{\dddom}{\widetilde{\Omega}}			%3D domain notation
\newcommand{\ddom}{\Omega}						%2D domain notation
\newcommand{\dddmes}{\widetilde{\mu}}			%3D measure
\newcommand{\ddmes}{\mu}						%2D measure

\newcommand{\graph}{\mathbb{G}}					%graph variable
\newcommand{\vertSet}{\mathcal{V}}					%set of vertices rather than big V
\newcommand{\edgeSet}{\mathcal{E}}					%set of edges rather than large E, \graph = (V,E)
\newcommand{\wavenumber}{\kappa}				%fourier variable or wavenumber, not to confuse with jk subscripts!
\newcommand{\qm}{\theta}						%quasi-momentum parameter
\newcommand{\kt}{\bracs{\wavenumber, \qm}}		%(k, theta) pair
\newcommand{\dmap}{\Gamma_0}					%Dirichlet map
\newcommand{\nmap}{\Gamma_1}					%Neumann map
\newcommand{\effFreq}{\Lambda}					%Effective frequency sqrt(w^2-\wavenumber^2)
\newcommand{\conLeft}{\stackrel{\rightarrow}{\smash{\sim}\rule{0pt}{0.4ex}}} %j connects to k, j left
\newcommand{\conRight}{\stackrel{\leftarrow}{\smash{\sim}\rule{0pt}{0.4ex}}} %j connects to k, j right
\newcommand{\con}{\sim}							%j connects to k, indifferent of direction

%standard sets
\newcommand{\naturals}{\mathbb{N}}			%natural numbers
\newcommand{\integers}{\mathbb{Z}}			%integers
\newcommand{\rationals}{\mathbb{Q}}			%rational numbers
\newcommand{\reals}{\mathbb{R}}				%real numbers
\newcommand{\complex}{\mathbb{C}}			%complex numbers

%brackets and norms
\newcommand{\bracs}[1]{\left( #1 \right)}				%encloses input in brackets
\newcommand{\sqbracs}[1]{\left[ #1 \right]}				%encloses input in square brackets
\newcommand{\clbracs}[1]{\left\{ #1 \right\}}			%encloses input in curly bracers
\newcommand{\abs}[1]{\left\lvert #1 \right\rvert}					%absolute value
\newcommand{\norm}[1]{\lvert\lvert #1 \rvert\rvert}		%norm 

%function sets
\newcommand{\smooth}[1]{C^{\infty}\bracs{#1}}							%smooth functions
\newcommand{\ltwo}[2]{L^{2}\bracs{#1,\mathrm{d}#2}}						%general L^2 space
\newcommand{\gradSob}[2]{H^1_\mathrm{grad}\bracs{#1, \mathrm{d}#2}}		%gradient Sobolev space
\newcommand{\gradSobQM}[2]{H^1_{\qm, \mathrm{grad}}\bracs{#1, \mathrm{d}#2}} %gradient + i\qm Sobolev space
\newcommand{\ktgradSob}[2]{H^1_{\wavenumber,\qm,\mathrm{grad}}\bracs{#1, \mathrm{d}#2}}	%k,\qm-gradient Sobolev space
\newcommand{\curlSob}[2]{H^1_\mathrm{curl}\bracs{#1, \mathrm{d}#2}}		%curl Sobolev space
\newcommand{\tcurlSob}[2]{H^1_{\qm, \mathrm{curl}}\bracs{#1, \mathrm{d}#2}}		%curl + i\qm Sobolev space
\newcommand{\kcurlSob}[2]{H^1_{\wavenumber,\mathrm{curl}}\bracs{#1, \mathrm{d}#2}}	%k-curl Sobolev space
\newcommand{\ktcurlSob}[2]{H^1_{\wavenumber,\qm,\mathrm{curl}}\bracs{#1, \mathrm{d}#2}}	%k,\qm-curl Sobolev space
\newcommand{\ktcurlSobDivFree}[2]{\mathcal{H}^{\kt}\bracs{#1, \mathrm{d}#2}}					%k,\qm-curl, divergence-free Sobolev space
\newcommand{\supp}{\mathrm{supp}}										%support of a function

%grad and curl sets
\newcommand{\gradZero}[2]{\mathcal{G}_{ #1, \mathrm{d}#2}\bracs{0}}		%gradients of zero for domain #1 with measure #2
\newcommand{\kgradZero}[2]{\mathcal{G}_{ #1, \mathrm{d}#2}^{(\wavenumber)}\bracs{0}}	%k-gradients of zero for domain #1 with measure #2
\newcommand{\curlZero}[2]{\mathcal{C}_{ #1, \mathrm{d}#2}\bracs{0}}	%curls of zero for domain #1 with measure #2
\newcommand{\kcurlZero}[2]{\mathcal{C}_{ #1, \mathrm{d}#2}^{(\wavenumber)}\bracs{0}}	%k-curls of zero for domain #1 with measure #2

%derivatives and grad-like symbols
\newcommand{\diff}[2]{\dfrac{\mathrm{d}#1}{\mathrm{d}#2}}			%complete derivative d#1/d#2
\newcommand{\pdiff}[2]{\dfrac{\partial #1}{\partial #2}}			%partial derivative p#1/p#2
\newcommand{\ddiff}[2]{\dfrac{\mathrm{d}^2 #1}{\mathrm{d} {#2}^2}}	%2nd deriv
\newcommand{\grad}{\nabla}											%grad operator
\newcommand{\tgrad}{\nabla^{\qm}}									%grad operator with qm superscript
\newcommand{\kgrad}{\grad^{(\wavenumber)}}							%grad with wavenumber superscript
\newcommand{\ktgrad}{\grad^{\kt}}				%grad with wavenumber, qm superscript
\newcommand{\curl}[1]{\grad_{#1}\wedge}							%curl with subscript #1
\newcommand{\kcurl}[1]{\grad_{#1}^{(\wavenumber)}\wedge}			%k-curl with measure subscript #1
\newcommand{\ktcurl}[1]{\grad_{#1}^{\kt}\wedge}		%k,theta-curl with measure subscript #1
\newcommand{\laplacian}{\Delta}						%laplacian operator, can have subscripts attached

%displaying integrals
\newcommand{\integral}[3]{\int_{#1}#2 \ \mathrm{d}#3}			%integral, domain #1, integrand #2, measure #3
\newcommand{\md}{\mathrm{d}}									%differential d

%convergence
%\newcommand{\lconv}[1]{\xrightarrow{#1}}							%convergence with #1 above the rightarrow - requires mathtools
\newcommand{\lconv}[1]{\overset{#1}{\longrightarrow}}				%convergence with #1 above the rightarrow
\newcommand{\toInfty}[1]{ \ \text{as} \ #1 \rightarrow\infty}		%writes out "as #1 tends to infty" %maths commands, variables, and other packages

%labelling hacks
\newcommand\labelthis{\addtocounter{equation}{1}\tag{\theequation}}

\newtheorem{conj}[theorem]{Conjecture}
\theoremstyle{plain}

\newcommand{\tlambda}{\tilde{\lambda}}
\newcommand{\C}[1]{\mathrm{C}^1\bracs{#1}}

%-------------------------------------------------------------------------
%DOCUMENT STARTS

\begin{document}

\section*{Setup}
The following theorem is given a placeholder reference; it is a result that is known for ``classical" Sobolev spaces that we may need later, and will serve as motivation for what we are trying to prove.
\begin{theorem} \label{thm:MeetingThm}
	Let $\ddom = \overline{\ddom}_1 \cup \overline{\ddom}_2$ be a domain in $\reals^2$, with $I = \partial\ddom_1 \cap \partial\ddom_2$ (sketched in figure \ref{fig:MeetingSituation}).
	Suppose $u\in \C{\overline{\ddom_1}}\cap \C{\overline{\ddom_2}}$.
	Then
	\begin{align*}
		u\in\mathrm{C}\bracs{\overline{\ddom}} \quad\Leftrightarrow\quad & u\in\gradSob{\ddom}{\lambda_2}
	\end{align*}
\end{theorem}

Throughout, we will make a few assumptions on the geometry of our domain (although the arguments we present should easily generalise).
We take 
\begin{align*}
	\ddom = \sqbracs{0,1}^2, &\quad \ddom_1 = \sqbracs{0,1}\times\sqbracs{0,\recip{2}},
	\quad \ddom_2 = \sqbracs{0,1}\times\sqbracs{\recip{2},1}, \\
	\implies &I = \sqbracs{0,1}\times\clbracs{\recip{2}}.	
\end{align*}
So we have the setup that is sketched in figure \ref{fig:MeetingSituation}, and we also define $\lambda_I$ as the singular measure supported on $I$, and $\lambda_2$ is the 2D-Lebesgue measure on $\ddom$.
\begin{figure}[h!]
	\centering
	\begin{tikzpicture}
		\draw (0,0) rectangle (3,3);
		\draw[thick, red] (0,1.5) -- (3,1.5);
		\node[anchor=east] at (0,3) {$\ddom = \overline{\ddom}_1 \cup \overline{\ddom}_2$};
		\node[align=center] at (1.5,0.75) {$\ddom_1$};
		\node[align=center] at (1.5,2.25) {$\ddom_2$};
		\node[red, anchor=east] at (0,1.5) {$I = \sqbracs{0,1}\times\clbracs{\recip{2}}$};
		\draw[->] (5,0.5) -- (5,2.5) node[anchor=south] {$x_2$};
		\draw[->] (5,0.5) -- (7,0.5) node[anchor=west] {$x_1$};
	\end{tikzpicture}
	\caption{\label{fig:MeetingSituation}}
\end{figure}
The composite measure on $\ddom$ is then $\tlambda_I = \lambda_2 + \lambda_I$. \newline

We now look at the following conjecture for our composite Sobolev spaces, motivated by theorem \ref{thm:MeetingThm}.
\begin{conj} \label{conj:CompMeasRegularity}
	Under our standing assumptions on $\ddom$, suppose $u\in \C{\overline{\ddom_1}}\cap \C{\overline{\ddom_2}}$.
	Then
	\begin{align*}
		u\in\mathrm{C}\bracs{\overline{\ddom}} \quad\Leftrightarrow\quad & u\in\gradSob{\ddom}{\tlambda_I}
	\end{align*}
\end{conj}

\subsection*{Proof of the Left-Directed ($\Leftarrow$) implication of conjecture \ref{conj:CompMeasRegularity}}
The right-directed implication holds fairly easily; for explicitness we work with the pair $\bracs{u,v}$ in these Sobolev spaces.
Given $\bracs{u,v}\in\gradSob{\ddom}{\tlambda_I}$ there exists a sequence of smooth functions $\phi_n$ such that
\begin{align*}
	\phi_n \lconv{\ltwo{\ddom}{\tlambda_I}} u, &\quad \grad\phi_n \lconv{\ltwo{\ddom}{\tlambda_I}^2} v.
\end{align*}
In particular,
\begin{align*}
	\integral{\ddom}{\abs{\phi_n - u}^2}{\lambda_2} &\leq \integral{\ddom}{\abs{\phi_n - u}^2}{\tlambda_I} \rightarrow 0 \toInfty{n}, \\
	\integral{\ddom}{\abs{\grad\phi_n - v}^2}{\lambda_2} &\leq \integral{\ddom}{\abs{\grad\phi_n - v}^2}{\tlambda_I} \rightarrow 0 \toInfty{n}.
\end{align*}
Thus,
\begin{align*}
	\phi_n \lconv{\ltwo{\ddom}{\lambda_2}} u, &\quad \grad\phi_n \lconv{\ltwo{\ddom}{\lambda_2}^2} v,
\end{align*}
and so $\bracs{u,v}\in\gradSob{\ddom}{\lambda_2}$.
Thus, by theorem \ref{thm:MeetingThm} we ave that $u\in\mathrm{C}\bracs{\overline{\ddom}}$. \newline

\subsection*{Proof of the Right-Directed ($\Rightarrow$) implication of conjecture \ref{conj:CompMeasRegularity}}
Our approach to this direction of the implication is as follows;
\begin{enumerate}
	\item Demonstrate that the difference in the uniform extensions/traces of $\grad u$ from $\ddom_1$ and $\ddom_2$ to $I$ are consistent in the direction parallel to $I$.
	Hence this difference is also a gradient of zero in the $\lambda_I$-sense, and additionally in the $\tlambda_I$-sense.
	\item Use theorem \ref{thm:MeetingThm} to obtain an approximating sequence for the weak derivative of $u$ in the $\gradSob{\ddom}{\lambda_2}$-sense.
	We will impose further specifications on this approximating sequence, in order to later obtain an approximating sequence converging to $u$ in the $\ltwo{\ddom}{\tlambda_I}$-sense, whose gradients also converge in $\ltwo{\ddom}{\tlambda_I}^2$.
	\item We will present the candidate pair $\bracs{u,w}$, and show that the approximating sequence we constructed is sufficient to demonstrate it is an element of $\gradSob{\ddom}{\tlambda_I}$.
\end{enumerate}

As such, we also present the proof in stages:
\begin{enumerate}
	\item Here our idea is that $u\in\mathrm{C}\bracs{\overline{\ddom}}$ means that $u$ still have a consistent gradient along $I$ from both $\ddom_1$ and $\ddom_2$, otherwise we could use the Fundamental Theorem of Calculus and a limit-to-the-boundary argument to deduce that $u$ takes different values on $I$ according to each extension.
	To this end, let $v^{(1)}$ and $v^{(2)}$ be the uniform extensions/traces of $\grad u$ to $I$, from $\ddom_1$ and $\ddom_2$ respectively.
	Notice that we can write
	\begin{align}
		u\bracs{x_1,x_2} &= \int_0^{x_1} \partial_1 u\bracs{t,x_2} \ \md t,
		\quad \forall x_1\in\bracs{0,1}, \forall x_2\in\bracs{0,1}\setminus\clbracs{\recip{2}}.
	\end{align}
	As $u$ is continuous on $\ddom$ and so in particular is continuous on approach to $I$,
	\begin{align}
		u\bracs{x_1,\recip{2}} &= \lim_{x_2\rightarrow \recip{2}} u\bracs{x_1,x_2}, \\
		&= \lim_{x_2\rightarrow\recip{2}} \int_0^{x_1} \partial_1 u\bracs{t,x_2} \ \md t.
	\end{align}
	In particular, the limits ``from the left" and ``from the right" must coincide, so
	\begin{align}
		\lim_{x_2\rightarrow\recip{2}-} \int_0^{x_1} \partial_1 u\bracs{t,x_2} \ \md t 
		&= \lim_{x_2\rightarrow\recip{2}+} \int_0^{x_1} \partial_1 u\bracs{t,x_2} \ \md t,
	\end{align}
	again for every $x_1\in\bracs{0,1}$.
	Then as $v^{(1)}$ and $v^{(2)}$ are the uniform extensions of $\grad u$ to the boundary from each side,
	\begin{align}
		\lim_{x_2\rightarrow\recip{2}-} \int_0^{x_1} \partial_1 u\bracs{t,x_2} \ \md t
		&= \int_0^{x_1} \lim_{x_2\rightarrow\recip{2}+}\partial_1 u\bracs{t,x_2} \ \md t \\
		&= \int_0^{x_1} v^{(1)}_1\bracs{t,\recip{2}} \ \md t, \\
		\lim_{x_2\rightarrow\recip{2}+} \int_0^{x_1} \partial_1 u\bracs{t,x_2} \ \md t
		&= \int_0^{x_1} \lim_{x_2\rightarrow\recip{2}+}\partial_1 u\bracs{t,x_2} \ \md t \\
		&= \int_0^{x_1} v^{(2)}_1\bracs{t,\recip{2}} \ \md t.	
	\end{align}
	Thus,
	\begin{align} \label{eq:TraceIntegrals}
		0 &= \int_0^{x_1} v^{(1)}_1\bracs{t,\recip{2}} - v^{(2)}_1\bracs{t,\recip{2}} \ \md t,
		\quad \forall x_1\in\bracs{0,1}.
	\end{align}
	As $v^{(1)}$ and $v^{(2)}$ are continuous, we can conclude from \eqref{eq:TraceIntegrals} that $v^{(1)}_1 = v^{(2)}_1$ everywhere on $\bracs{0,1}\times\clbracs{\recip{2}}$, and hence (by continuity arguments) on $\sqbracs{0,1}\times\clbracs{\recip{2}}$.
	This means that on $I$,
	\begin{align}
		v^{(1)} - v^{(2)} &= \begin{pmatrix} 0 \\ v^{(1)}_1 - v^{(2)}_2 \end{pmatrix},
	\end{align}
	and thus $z\in\gradZero{\ddom}{\lambda_I}$ by the characterisation result for this set, where
	\begin{align}
		z &= \begin{cases}
		\begin{pmatrix} 0 \\ v^{(1)}_2 - v^{(2)}_2 \end{pmatrix} & x\in I, \\
		0 & x\in\ddom\setminus I.
		\end{cases}
	\end{align}
	Given that $z\in\gradZero{\ddom}{\lambda_I}$, we can use our characterisation of $\gradZero{\ddom}{\tlambda_I}$ to deduce that $z\in\gradZero{\ddom}{\tlambda_I}$ too.
	\item Using theorem \ref{thm:MeetingThm}, we have that the pair $\bracs{u,v}\in\gradSob{\ddom}{\lambda_2}$ where $v$ is the weak derivative of $u$ that coincides with $\grad u$ on $\ddom_1$ and $\ddom_2$.
	By the ``H=W" theorem, this means that there exists a sequence of smooth functions $\phi_n\in\smooth{\ddom}$ such that
	\begin{align}
		\phi_n \lconv{\ltwo{\ddom}{\lambda_2}} u, \quad \grad\phi_n\lconv{\ltwo{\ddom}{\lambda_2}^2}v.
	\end{align}
	Let
	\begin{align}
		g(x_1) &= 
		\begin{pmatrix}
			v^{(1)}_1\bracs{x_1,\recip{2}} \\
			0
		\end{pmatrix}.
	\end{align}
	In particular we can choose $\phi_n$ such that
	\begin{align} \label{eq:ApproxSeqRequirements}
		\phi_n = u, \quad \grad\phi_n = g, \quad x\in I.
	\end{align}
	\item Define the function $w\in\ltwo{\ddom}{\tlambda_I}^2$ by
	\begin{align}
		w &= \begin{cases} \grad u & x\in\ddom_1 \cup \ddom_2, \\ g & x\in I. \end{cases}
	\end{align}
	We claim that $\bracs{u,w}\in\gradSob{\ddom}{\tlambda_I}$.
	To this end, we take our approximating sequence as in equation \eqref{eq:ApproxSeqRequirements}, and notice that
	\begin{subequations}
		\begin{align}
			\integral{\ddom}{\abs{\phi_n - u}^2}{\tlambda_I}
			&= \integral{\ddom}{\abs{\phi_n - u}^2}{\lambda_2} + \integral{\ddom}{\abs{\phi_n - u}^2}{\lambda_I} \\
			&= \integral{\ddom}{\abs{\phi_n - u}^2}{\lambda_2} + \integral{I}{\abs{\phi_n - u}^2}{\lambda_I} \\
			&= \integral{\ddom}{\abs{\phi_n - u}^2}{\lambda_2} + 0 \\
			&\rightarrow 0 \toInfty{n}.
		\end{align}
	\end{subequations}
	Additionally,
	\begin{subequations}
		\begin{align}
			\integral{\ddom}{\abs{\grad\phi_n - w}^2}{\tlambda_I}
			&= \integral{\ddom}{\abs{\grad\phi_n - w}^2}{\lambda_2} + \integral{\ddom}{\abs{\grad\phi_n - w}^2}{\lambda_I} \\
			&= \integral{\ddom}{\abs{\grad\phi_n - \grad u}^2}{\lambda_2} + \integral{I}{\abs{\grad\phi_n - g}^2}{\lambda_I} \\
			&= \integral{\ddom}{\abs{\grad\phi_n - \grad u}^2}{\lambda_2} + 0 \\
			&\rightarrow 0 \toInfty{n}.
		\end{align}
	\end{subequations}
	Hence
	\begin{align}
		\phi_n \lconv{\ltwo{\ddom}{\tlambda_I}} u, \quad \grad\phi_n \lconv{\ltwo{\ddom}{\tlambda_I}^2} w,
	\end{align}
	and so $\bracs{u,w}\in\gradSob{\ddom}{\tlambda_I}$.
\end{enumerate}

\subsection*{On the choice of functions in \eqref{eq:ApproxSeqRequirements}}
	One of my slight qualms about this proof is that I am just saying ``I can pick the $\phi_n$ as in \eqref{eq:ApproxSeqRequirements}", however I want to be explicitly clear how I can - I can draw these functions but would like to better understand how one could actually construct them.	
	For example, one such choice of the $\phi_n$, which I can sketch, is below:
	\begin{subequations} \label{eq:SpecificReqs}
		\begin{align}
			\phi_n(x) = u(x), &\quad \mathrm{dist}(x,I)\geq\recip{n}, \\
			\abs{\phi_n(x) - u(x)} \leq \recip{n}, &\quad 0<\mathrm{dist}(x,I)<\recip{n}, \\
			\abs{\grad\phi_n(x) - \grad u(x)} \leq c, &\quad 0<\mathrm{dist}(x,I)<\recip{n}, \\
			\phi_n(x) = u(x), &\quad x\in I, \\
			\grad\phi_n(x) = g(x_1), &\quad x\in I,
		\end{align}
	\end{subequations}
	where $c>0$ is a constant independent of $n$.
	To this end, we take our approximating sequence and notice that
	\begin{subequations}
		\begin{align}
			\integral{\ddom}{\abs{\phi_n - u}^2}{\tlambda_I}
			&= \integral{\ddom}{\abs{\phi_n - u}^2}{\lambda_2} + \integral{\ddom}{\abs{\phi_n - u}^2}{\lambda_I} \\
			&= \integral{\ddom}{\abs{\phi_n - u}^2}{\lambda_2} + \integral{I}{\abs{\phi_n - u}^2}{\lambda_I} \\
			&= \integral{\sqbracs{0,1}\times\sqbracs{\recip{2}-\recip{n},\recip{2}+\recip{n}}}{\abs{\phi_n - u}^2}{\lambda_2} \\
			&\leq \recip{n^2}\integral{\sqbracs{0,1}\times\sqbracs{\recip{2}-\recip{n},\recip{2}+\recip{n}}}{}{\lambda_2} = \frac{2}{n^3} \\
			&\rightarrow 0 \toInfty{n}.
		\end{align}
	\end{subequations}
	Additionally,
	\begin{subequations}
		\begin{align}
			\integral{\ddom}{\abs{\grad\phi_n - w}^2}{\tlambda_I}
			&= \integral{\ddom}{\abs{\grad\phi_n - w}^2}{\lambda_2} + \integral{\ddom}{\abs{\grad\phi_n - w}^2}{\lambda_I} \\
			&= \integral{\ddom}{\abs{\grad\phi_n - \grad u}^2}{\lambda_2} + \integral{I}{\abs{\grad\phi_n - g}^2}{\lambda_I} \\
			&= \integral{\sqbracs{0,1}\times\sqbracs{\recip{2}-\recip{n},\recip{2}+\recip{n}}}{\abs{\grad\phi_n - \grad u}^2}{\lambda_2} \\
			&\leq c^2\integral{\sqbracs{0,1}\times\sqbracs{\recip{2}-\recip{n},\recip{2}+\recip{n}}}{}{\lambda_2} = \frac{2c^2}{n} \\
			&\rightarrow 0 \toInfty{n}.
		\end{align}
	\end{subequations}
	
	However, $u$ is only $\mathrm{C}^1$ on $\ddom_1$ and $\ddom_2$, which means that these $\phi_n$'s are also not smooth functions on all of $\ddom$ (as they are equal to $u$ outside a small neighbourhood of $I$).
	We could get around this by instead asking for
	\begin{subequations} \label{eq:MoreSpecificReqs}
		\begin{align}
			\abs{\phi_n(x) - u(x)} \leq \recip{n^2}, &\quad \mathrm{dist}(x,I)\geq\recip{n}, \\
			\abs{\grad\phi_n(x) - u(x)} \leq \recip{n}, &\quad \mathrm{dist}(x,I)\geq\recip{n}, \\
			\abs{\phi_n(x) - u(x)} \leq \recip{n}, &\quad 0<\mathrm{dist}(x,I)<\recip{n}, \\
			\abs{\grad\phi_n(x) - \grad u(x)} \leq c, &\quad 0<\mathrm{dist}(x,I)<\recip{n}, \\
			\phi_n(x) = u(x), &\quad x\in I, \\
			\grad\phi_n(x) = g(x_1), &\quad x\in I.
		\end{align}
	\end{subequations}
	Again, I can sketch a function like this, but how would it actually be constructed for use?
	I'm familiar with constructing smooth ``bump" functions (like those we used in the singular-measure only situation) however have not seen a method for constructing smooth functions like those in \eqref{eq:SpecificReqs} and \eqref{eq:MoreSpecificReqs} where we are asking a lot about not only the functions, but their gradients too.
	
\end{document}