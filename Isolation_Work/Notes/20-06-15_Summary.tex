\documentclass[11pt]{report}

\usepackage{url}
\usepackage[margin=2.5cm]{geometry} % See geometry.pdf to learn the layout options. There are lots.
\geometry{a4paper} %or letterpaper or a5paper or ...

%for figures and graphics
\usepackage{graphicx}
\usepackage{subcaption} %allows subfigures
\usepackage[bottom]{footmisc} %footnotes go below figures
\usepackage{tikz}
%\usepackage{parskip} %adds line space between paragraphs by default
\usepackage{enumerate} %allows lower case roman numerials in enumerate environments

\DeclareGraphicsRule{.tif}{png}{.png}{`convert #1 `dirname #1`/`basename #1 .tif`.png}
\graphicspath{{../Diagrams/Diagram_PDFs/} {../Diagrams/Numerical_Results/}}

%\input imports all commands from the target files
%The idea behind this file is that it will be used to store all the maths-related macros that I concoct; so that I can import all the commands by \input{this file} in the preamble of any file that I want to use them in.
%This should make the top-level files look a lot cleaner, and the preamble much shorter!

\usepackage{amssymb}
\usepackage{amsmath}
%\usepackage{mathtools}

%theorems and lemma etc setup using amsthm
\usepackage{amsthm}
\theoremstyle{definition}
\newtheorem{definition}{Definition}[section]
\theoremstyle{plain}
\newtheorem{theorem}{Theorem}[section]
\theoremstyle{plain}
\newtheorem{lemma}[theorem]{Lemma}
\theoremstyle{plain}
\newtheorem{prop}[theorem]{Proposition}
\theoremstyle{plain}
\newtheorem{cory}[theorem]{Corollary}
\theoremstyle{definition}
\newtheorem{convention}[theorem]{Convention}
\theoremstyle{definition}
\newtheorem{assumption}[theorem]{Assumption}
\theoremstyle{definition}
\newtheorem{conjecture}[theorem]{Conjecture}

\allowdisplaybreaks %allows equations in the same align environment to split over multiple pages.

%tstk always need be there
\newcommand{\tstk}[1]{\textbf{#1} \newline}

%this adds extra functionality to pmatrix, vmatrix, bmatrix etc by allowing you to pass an optional argument in [FACTOR] to multiply the default spacing between elements by FACTOR
\makeatletter
\renewcommand*\env@matrix[1][\arraystretch]{%
  \edef\arraystretch{#1}%
  \hskip -\arraycolsep
  \let\@ifnextchar\new@ifnextchar
  \array{*\c@MaxMatrixCols c}
  }
\makeatother

%begin the macros via newcommand. Try to group them up reasonably!

%notation and variable use throughout the file
\renewcommand{\vec}[1]{\mathbf{#1}}				%vectors are bold, not overline arrow
\newcommand{\recip}[1]{\frac{1}{#1}}			%fast reciprocal as a fraction
\newcommand{\interval}[1]{\sqbracs{0,\abs{#1}}}	%creates the closed interval from 0 to the length of the input #1, denoted by absolute value
\newcommand{\eps}{\varepsilon}					%pretty epsilons
\newcommand{\charFunc}[1]{\mathcal{I}_{#1}}%{\mathbb{1}_{#1}}		%characteristic function of a set

\newcommand{\dddom}{\widetilde{\Omega}}			%3D domain notation
\newcommand{\ddom}{\Omega}						%2D domain notation
\newcommand{\dddmes}{\widetilde{\mu}}			%3D measure
\newcommand{\ddmes}{\mu}						%2D measure

\newcommand{\graph}{\mathbb{G}}					%graph variable
\newcommand{\vertSet}{\mathcal{V}}					%set of vertices rather than big V
\newcommand{\edgeSet}{\mathcal{E}}					%set of edges rather than large E, \graph = (V,E)
\newcommand{\wavenumber}{\kappa}				%fourier variable or wavenumber, not to confuse with jk subscripts!
\newcommand{\qm}{\theta}						%quasi-momentum parameter
\newcommand{\kt}{\bracs{\wavenumber, \qm}}		%(k, theta) pair
\newcommand{\dmap}{\Gamma_0}					%Dirichlet map
\newcommand{\nmap}{\Gamma_1}					%Neumann map
\newcommand{\effFreq}{\Lambda}					%Effective frequency sqrt(w^2-\wavenumber^2)
\newcommand{\conLeft}{\stackrel{\rightarrow}{\smash{\sim}\rule{0pt}{0.4ex}}} %j connects to k, j left
\newcommand{\conRight}{\stackrel{\leftarrow}{\smash{\sim}\rule{0pt}{0.4ex}}} %j connects to k, j right
\newcommand{\con}{\sim}							%j connects to k, indifferent of direction

%standard sets
\newcommand{\naturals}{\mathbb{N}}			%natural numbers
\newcommand{\integers}{\mathbb{Z}}			%integers
\newcommand{\rationals}{\mathbb{Q}}			%rational numbers
\newcommand{\reals}{\mathbb{R}}				%real numbers
\newcommand{\complex}{\mathbb{C}}			%complex numbers

%brackets and norms
\newcommand{\bracs}[1]{\left( #1 \right)}				%encloses input in brackets
\newcommand{\sqbracs}[1]{\left[ #1 \right]}				%encloses input in square brackets
\newcommand{\clbracs}[1]{\left\{ #1 \right\}}			%encloses input in curly bracers
\newcommand{\abs}[1]{\left\lvert #1 \right\rvert}					%absolute value
\newcommand{\norm}[1]{\lvert\lvert #1 \rvert\rvert}		%norm 

%function sets
\newcommand{\smooth}[1]{C^{\infty}\bracs{#1}}							%smooth functions
\newcommand{\ltwo}[2]{L^{2}\bracs{#1,\mathrm{d}#2}}						%general L^2 space
\newcommand{\gradSob}[2]{H^1_\mathrm{grad}\bracs{#1, \mathrm{d}#2}}		%gradient Sobolev space
\newcommand{\gradSobQM}[2]{H^1_{\qm, \mathrm{grad}}\bracs{#1, \mathrm{d}#2}} %gradient + i\qm Sobolev space
\newcommand{\ktgradSob}[2]{H^1_{\wavenumber,\qm,\mathrm{grad}}\bracs{#1, \mathrm{d}#2}}	%k,\qm-gradient Sobolev space
\newcommand{\curlSob}[2]{H^1_\mathrm{curl}\bracs{#1, \mathrm{d}#2}}		%curl Sobolev space
\newcommand{\tcurlSob}[2]{H^1_{\qm, \mathrm{curl}}\bracs{#1, \mathrm{d}#2}}		%curl + i\qm Sobolev space
\newcommand{\kcurlSob}[2]{H^1_{\wavenumber,\mathrm{curl}}\bracs{#1, \mathrm{d}#2}}	%k-curl Sobolev space
\newcommand{\ktcurlSob}[2]{H^1_{\wavenumber,\qm,\mathrm{curl}}\bracs{#1, \mathrm{d}#2}}	%k,\qm-curl Sobolev space
\newcommand{\ktcurlSobDivFree}[2]{\mathcal{H}^{\kt}\bracs{#1, \mathrm{d}#2}}					%k,\qm-curl, divergence-free Sobolev space
\newcommand{\supp}{\mathrm{supp}}										%support of a function

%grad and curl sets
\newcommand{\gradZero}[2]{\mathcal{G}_{ #1, \mathrm{d}#2}\bracs{0}}		%gradients of zero for domain #1 with measure #2
\newcommand{\kgradZero}[2]{\mathcal{G}_{ #1, \mathrm{d}#2}^{(\wavenumber)}\bracs{0}}	%k-gradients of zero for domain #1 with measure #2
\newcommand{\curlZero}[2]{\mathcal{C}_{ #1, \mathrm{d}#2}\bracs{0}}	%curls of zero for domain #1 with measure #2
\newcommand{\kcurlZero}[2]{\mathcal{C}_{ #1, \mathrm{d}#2}^{(\wavenumber)}\bracs{0}}	%k-curls of zero for domain #1 with measure #2

%derivatives and grad-like symbols
\newcommand{\diff}[2]{\dfrac{\mathrm{d}#1}{\mathrm{d}#2}}			%complete derivative d#1/d#2
\newcommand{\pdiff}[2]{\dfrac{\partial #1}{\partial #2}}			%partial derivative p#1/p#2
\newcommand{\ddiff}[2]{\dfrac{\mathrm{d}^2 #1}{\mathrm{d} {#2}^2}}	%2nd deriv
\newcommand{\grad}{\nabla}											%grad operator
\newcommand{\tgrad}{\nabla^{\qm}}									%grad operator with qm superscript
\newcommand{\kgrad}{\grad^{(\wavenumber)}}							%grad with wavenumber superscript
\newcommand{\ktgrad}{\grad^{\kt}}				%grad with wavenumber, qm superscript
\newcommand{\curl}[1]{\grad_{#1}\wedge}							%curl with subscript #1
\newcommand{\kcurl}[1]{\grad_{#1}^{(\wavenumber)}\wedge}			%k-curl with measure subscript #1
\newcommand{\ktcurl}[1]{\grad_{#1}^{\kt}\wedge}		%k,theta-curl with measure subscript #1
\newcommand{\laplacian}{\Delta}						%laplacian operator, can have subscripts attached

%displaying integrals
\newcommand{\integral}[3]{\int_{#1}#2 \ \mathrm{d}#3}			%integral, domain #1, integrand #2, measure #3
\newcommand{\md}{\mathrm{d}}									%differential d

%convergence
%\newcommand{\lconv}[1]{\xrightarrow{#1}}							%convergence with #1 above the rightarrow - requires mathtools
\newcommand{\lconv}[1]{\overset{#1}{\longrightarrow}}				%convergence with #1 above the rightarrow
\newcommand{\toInfty}[1]{ \ \text{as} \ #1 \rightarrow\infty}		%writes out "as #1 tends to infty" %maths commands, variables, and other packages

%labelling hacks
\newcommand\labelthis{\addtocounter{equation}{1}\tag{\theequation}}

\newtheorem{conj}[theorem]{Conjecture}
\theoremstyle{plain}

\newcommand{\tlambda}{\tilde{\lambda}}
\newcommand{\C}[1]{\mathrm{C}^1\bracs{#1}}

%-------------------------------------------------------------------------
%DOCUMENT STARTS

\begin{document}

\section*{Regarding Regularity}

We discussed in the Monday meeting that there is a result for classical Sobolev spaces as follows:
\begin{theorem} \label{thm:MeetingThm}
	Using the setup in Figure \ref{fig:MeetingSituation}, suppose $u\in \C{\overline{\ddom_1}}\cap \C{\overline{\ddom_2}}$.
	Then
	\begin{align*}
		u\in\mathrm{C}\bracs{\overline{\ddom}} \quad\Leftrightarrow\quad & u\in\gradSob{\ddom}{\lambda_2}
	\end{align*}
\end{theorem}
\begin{figure}[h!]
	\centering
	\begin{tikzpicture}
		\draw (0,0) rectangle (3,3);
		\draw[thick, red] (0,1.5) -- (3,1.5);
		\node[anchor=east] at (0,3) {$\ddom = \overline{\ddom_1} \cup \overline{\ddom_2}$};
		\node[align=center] at (1.5,0.75) {$\ddom_1$};
		\node[align=center] at (1.5,2.25) {$\ddom_2$};
		\node[red, anchor=west] at (3,1.5) {$I = \partial\ddom_1 \cap \partial\ddom_2$};
	\end{tikzpicture}
	\caption{\label{fig:MeetingSituation}}
\end{figure}
And we then wondered whether something similar to this would hold for our composite measure Sobolev spaces, again to attempt to address where (if anywhere) continuity of our functions $u$ across the segment $I$ should come from.
That started, the obvious place to look was at trying to prove the following:
\begin{conj} \label{conj:CompMeasRegularity}
	Using the setup in Figure \ref{fig:MeetingSituation}, suppose $u\in \C{\overline{\ddom_1}}\cap \C{\overline{\ddom_2}}$.
	Let $\tlambda_I = \lambda_2 + \lambda_I$ be the composite measure.
	Then
	\begin{align*}
		u\in\mathrm{C}\bracs{\overline{\ddom}} \quad\Leftrightarrow\quad & u\in\gradSob{\ddom}{\tlambda_I}
	\end{align*}
\end{conj}

\subsection*{Proof of the Right-Directed implication of conjecture \ref{conj:CompMeasRegularity}}
The right-directed implication holds fairly easily; for explicitness we work with the pair $\bracs{u,v}$ in these Sobolev spaces.
Given $\bracs{u,v}\in\gradSob{\ddom}{\tlambda_I}$ there exists a sequence of smooth functions $\phi_n$ such that
\begin{align*}
	\phi_n \lconv{\ltwo{\ddom}{\tlambda_I}} u, &\quad \grad\phi_n \lconv{\ltwo{\ddom}{\tlambda_I}^2} v.
\end{align*}
In particular,
\begin{align*}
	\integral{\ddom}{\abs{\phi_n - u}^2}{\lambda_2} &\leq \integral{\ddom}{\abs{\phi_n - u}^2}{\tlambda_I} \rightarrow 0 \toInfty{n}, \\
	\integral{\ddom}{\abs{\grad\phi_n - v}^2}{\lambda_2} &\leq \integral{\ddom}{\abs{\grad\phi_n - v}^2}{\tlambda_I} \rightarrow 0 \toInfty{n}.
\end{align*}
Thus,
\begin{align*}
	\phi_n \lconv{\ltwo{\ddom}{\lambda_2}} u, &\quad \grad\phi_n \lconv{\ltwo{\ddom}{\lambda_2}^2} v,
\end{align*}
and so $\bracs{u,v}\in\gradSob{\ddom}{\lambda_2}$.
Thus, by theorem \ref{thm:MeetingThm} we ave that $u\in\mathrm{C}\bracs{\overline{\ddom}}$. \newline

This proof of the right-directed implication makes me rather uneasy though - at no point do I need to consider what is happening along the segment $I$ (which is additional information that we are provided by $\bracs{u,v}\in\gradSob{\ddom}{\tlambda_I}$ that we haven't used).
It also bypasses another point - to prove theorem \ref{thm:MeetingThm} we use the relevant integral identities for distributional derivatives (or show that they hold for the reverse implication).
Here, I begin with a convergent sequence, show that it belongs to a more familiar space, and then am essentially swapping to using the integral identities to complete the proof.
This means that, when attempting to prove the converse implication, we need to find a way to go from an integral identity to finding a sequence that converges in the right norm.

\subsection*{Attempt to prove the Left-Directed implication of conjecture \ref{conj:CompMeasRegularity}}

Since we are starting with $u\in\mathrm{C}\bracs{\overline{\ddom}}$ as well as $u\in \C{\overline{\ddom_1}}\cap \C{\overline{\ddom_2}}$, we can use theorem \ref{thm:MeetingThm} to deduce that $\bracs{u,v}\in\gradSob{\ddom}{\lambda_2}$.
We know that $v = \grad u$ $\lambda_2$-almost everywhere as well (as we deduce this in the proof of theorem \ref{thm:MeetingThm}).
Now we can transition to a sequence argument - as $\bracs{u,v}\in\gradSob{\ddom}{\lambda_2}$ we can use ``$H=W$" to deduce that there is a smooth sequence of functions $\phi_n$ such that
\begin{align*}
	\phi_n \lconv{\ltwo{\ddom}{\lambda_2}} u, &\quad \grad\phi_n \lconv{\ltwo{\ddom}{\lambda_2}^2} v.
\end{align*}
The question now is how do we obtain a sequence of functions that converges in $\ltwo{\ddom}{\tlambda_I}$?
Our candidates $\phi_n$ can always misbehave on $I$ so we can't naively assume that $\phi_n$ will do.
We also can't use the trick of extracting a subsequence $\phi_{n_k}$ that converges pointwise $\lambda_2$-almost-everywhere, because $I$ is a set of zero $\lambda_2$-measure.
As such, if we want to show that $\bracs{u,w}\in\gradSob{\ddom}{\tlambda_I}$ for some $w$, we need to come up with a candidate for both of $w$ on $\ddom\setminus I$, and $w\vert_I$.
The obvious candidate is to take $w=v$ on $\ddom\setminus I$, given the sequence that we obtain above and so that $w=v$ $\lambda_2$-almost everywhere.
Thinking up what $w\vert_I$ should be is a lot trickier, mostly because we need to decide what to assign when we have a function that looks like the one illustrated in figure \ref{fig:ContButNonDiff}.
\begin{figure}[h!]
	\centering
	\begin{tikzpicture}
		\draw[->] (-3,0) -- (3,0);
		\draw[->] (0,0) -- (0,3);
		\node[anchor=north] at (0,0) {$I$};
		\node[anchor=north] at (-2.5,0) {$\ddom_1$};
		\node[anchor=north] at (2.5,0) {$\ddom_2$};
		\node[anchor=south] at (0,3) {$u$};
		\draw (-3,2) -- (0,1) -- (3,3);
	\end{tikzpicture}
	\caption{\label{fig:ContButNonDiff}}
\end{figure}
These aren't a problem for theorem \ref{thm:MeetingThm}, because our distributional derivative that we look to define can be arbitrary on the segment $I$; but in this case we \textit{have to} care about the value of our ``gradient" $w$ on $I$.
At this point, I have a couple of things that might be worth investigating regarding this matter:
\begin{itemize}
	\item Let $v^{(1)}$ and $v^{(2)}$ be the traces of $\grad u$ onto $I$ from $\ddom_1$ and $\ddom_2$ respectively.
	Is $v^{(1)} - v^{(2)}$ a gradient of zero in the $\lambda_I$ sense?
	The idea being that if it is, then the traces $v^{(1)}$ and $v^{(2)}$ might still be consistent along the segment $I$, and hence can still provide a gradient here (in the $\lambda_I$ sense).
	If it is, is the function
	\begin{align*}
		z &= \begin{cases} v^{(1)} - v^{(2)} & x\in I, \\ 0 & x\in\ddom\setminus I \end{cases}
	\end{align*}
	a gradient of zero in the $\tlambda_I$ sense?
	If it is, then our candidate for $w$ is now $w = v - z$.
	We already have the $\phi_n$ that relate to $v$, and as $z\in\gradZero{\ddom}{\tlambda_I}$ we can find another sequence of functions $\varphi_n$ with the usual convergence properties.
	Then we expect the sequence $\psi_n = \phi_n - \varphi_n$ to be the approximating sequence we need, to show that $\bracs{u,w}\in\gradSob{\ddom}{\tlambda_I}$.
	There is still is issue of showing that $\phi_n-\varphi_n$ is behaving as we expect on $I$ though.
	\item The second question I have is more general - can we say anything at all about the sequence $\phi_n$ or $\grad\phi_n$ on $I$, or about what it does in the $\ltwo{\ddom}{\lambda_I}$ norm?
	For example, if we could show it was Cauchy we would know it had a limit in this norm, which would be our candidate for $w\vert_{I}$.
	Alternatively, if it was a bounded sequence in the appropriate $\mathrm{L}^2$-norm, then it might have a convergent subsequence, which we could use instead to make the same conclusion.
	However these approaches are both hindered by the fact that all we have to work with regarding the $\phi_n$ are that they are smooth and converge in $\ltwo{\ddom}{\lambda_2}$, so no ``nice" behaviour is guaranteed on $I$ (or if it is, I'm missing why).
\end{itemize}

\section*{More general investigations into $\gradSob{\ddom}{\tlambda_I}$}
The idea at the end of the last section also ties into the more general investigation about $\gradSob{\ddom}{\tlambda_I}$.
We know that $\bracs{u,v}\in\gradSob{\ddom}{\tlambda_I}$ implies that $\bracs{u,v}\in\gradSob{\ddom}{\lambda_I}$ and $\bracs{u,v}\in\gradSob{\ddom}{\lambda_2}$, however I've been unsuccessful in proving the converse.
Indeed, for the converse to hold I have come to the conclusion that we would require the existence of a sequence of smooth functions $\theta_n$ such that
\begin{align*}
	\theta_n \rightarrow \charFunc{I} &\text{ in } \ltwo{\ddom}{\tlambda_I}, \\
	\grad\theta_n \rightarrow 0 &\text{ in } \ltwo{\ddom}{\tlambda_I}^2.
\end{align*}
With $\charFunc{I}$ being the characteristic function of the segement $I$,
\begin{align*}
	\charFunc{I} &= \begin{cases} 1 & x\in I, \\ 0 & x\not\in I. \end{cases}
\end{align*}
The reason being we can then take the approximating sequences we get from $\bracs{u,v}\in\gradSob{\ddom}{\lambda_I}$ and $\bracs{u,v}\in\gradSob{\ddom}{\lambda_2}$ and then add them together, weighted by the $\theta_n$.
However I am also unconvinced as to whether such a sequence of functions can ever exist; convergence to $\charFunc{I}$ necessitates a steep jump in the gradients of $\theta_n$, so how can they ever converge to 0?
But maybe that's because we need the additional information that we (may or may not have!) in the above case; if we know that $\bracs{u,v}\in\gradSob{\ddom}{\lambda_I}$ and $\bracs{u,v}\in\gradSob{\ddom}{\lambda_2}$, \textit{and} that $v^{(1)}-v^{(2)}$ is a gradient of zero, we can counteract this?

\end{document}