\documentclass[11pt]{report}

\usepackage{url}
\usepackage[margin=2.5cm]{geometry} % See geometry.pdf to learn the layout options. There are lots.
\geometry{a4paper} %or letterpaper or a5paper or ...

\usepackage{graphicx}
\usepackage{tikz}

%\input imports all commands from the target files
%The idea behind this file is that it will be used to store all the maths-related macros that I concoct; so that I can import all the commands by \input{this file} in the preamble of any file that I want to use them in.
%This should make the top-level files look a lot cleaner, and the preamble much shorter!

\usepackage{amssymb}
\usepackage{amsmath}
%\usepackage{mathtools}

%theorems and lemma etc setup using amsthm
\usepackage{amsthm}
\theoremstyle{definition}
\newtheorem{definition}{Definition}[section]
\theoremstyle{plain}
\newtheorem{theorem}{Theorem}[section]
\theoremstyle{plain}
\newtheorem{lemma}[theorem]{Lemma}
\theoremstyle{plain}
\newtheorem{prop}[theorem]{Proposition}
\theoremstyle{plain}
\newtheorem{cory}[theorem]{Corollary}
\theoremstyle{definition}
\newtheorem{convention}[theorem]{Convention}
\theoremstyle{definition}
\newtheorem{assumption}[theorem]{Assumption}
\theoremstyle{definition}
\newtheorem{conjecture}[theorem]{Conjecture}

\allowdisplaybreaks %allows equations in the same align environment to split over multiple pages.

%tstk always need be there
\newcommand{\tstk}[1]{\textbf{#1} \newline}

%this adds extra functionality to pmatrix, vmatrix, bmatrix etc by allowing you to pass an optional argument in [FACTOR] to multiply the default spacing between elements by FACTOR
\makeatletter
\renewcommand*\env@matrix[1][\arraystretch]{%
  \edef\arraystretch{#1}%
  \hskip -\arraycolsep
  \let\@ifnextchar\new@ifnextchar
  \array{*\c@MaxMatrixCols c}
  }
\makeatother

%begin the macros via newcommand. Try to group them up reasonably!

%notation and variable use throughout the file
\renewcommand{\vec}[1]{\mathbf{#1}}				%vectors are bold, not overline arrow
\newcommand{\recip}[1]{\frac{1}{#1}}			%fast reciprocal as a fraction
\newcommand{\interval}[1]{\sqbracs{0,\abs{#1}}}	%creates the closed interval from 0 to the length of the input #1, denoted by absolute value
\newcommand{\eps}{\varepsilon}					%pretty epsilons
\newcommand{\charFunc}[1]{\mathcal{I}_{#1}}%{\mathbb{1}_{#1}}		%characteristic function of a set

\newcommand{\dddom}{\widetilde{\Omega}}			%3D domain notation
\newcommand{\ddom}{\Omega}						%2D domain notation
\newcommand{\dddmes}{\widetilde{\mu}}			%3D measure
\newcommand{\ddmes}{\mu}						%2D measure

\newcommand{\graph}{\mathbb{G}}					%graph variable
\newcommand{\vertSet}{\mathcal{V}}					%set of vertices rather than big V
\newcommand{\edgeSet}{\mathcal{E}}					%set of edges rather than large E, \graph = (V,E)
\newcommand{\wavenumber}{\kappa}				%fourier variable or wavenumber, not to confuse with jk subscripts!
\newcommand{\qm}{\theta}						%quasi-momentum parameter
\newcommand{\kt}{\bracs{\wavenumber, \qm}}		%(k, theta) pair
\newcommand{\dmap}{\Gamma_0}					%Dirichlet map
\newcommand{\nmap}{\Gamma_1}					%Neumann map
\newcommand{\effFreq}{\Lambda}					%Effective frequency sqrt(w^2-\wavenumber^2)
\newcommand{\conLeft}{\stackrel{\rightarrow}{\smash{\sim}\rule{0pt}{0.4ex}}} %j connects to k, j left
\newcommand{\conRight}{\stackrel{\leftarrow}{\smash{\sim}\rule{0pt}{0.4ex}}} %j connects to k, j right
\newcommand{\con}{\sim}							%j connects to k, indifferent of direction

%standard sets
\newcommand{\naturals}{\mathbb{N}}			%natural numbers
\newcommand{\integers}{\mathbb{Z}}			%integers
\newcommand{\rationals}{\mathbb{Q}}			%rational numbers
\newcommand{\reals}{\mathbb{R}}				%real numbers
\newcommand{\complex}{\mathbb{C}}			%complex numbers

%brackets and norms
\newcommand{\bracs}[1]{\left( #1 \right)}				%encloses input in brackets
\newcommand{\sqbracs}[1]{\left[ #1 \right]}				%encloses input in square brackets
\newcommand{\clbracs}[1]{\left\{ #1 \right\}}			%encloses input in curly bracers
\newcommand{\abs}[1]{\left\lvert #1 \right\rvert}					%absolute value
\newcommand{\norm}[1]{\lvert\lvert #1 \rvert\rvert}		%norm 

%function sets
\newcommand{\smooth}[1]{C^{\infty}\bracs{#1}}							%smooth functions
\newcommand{\ltwo}[2]{L^{2}\bracs{#1,\mathrm{d}#2}}						%general L^2 space
\newcommand{\gradSob}[2]{H^1_\mathrm{grad}\bracs{#1, \mathrm{d}#2}}		%gradient Sobolev space
\newcommand{\gradSobQM}[2]{H^1_{\qm, \mathrm{grad}}\bracs{#1, \mathrm{d}#2}} %gradient + i\qm Sobolev space
\newcommand{\ktgradSob}[2]{H^1_{\wavenumber,\qm,\mathrm{grad}}\bracs{#1, \mathrm{d}#2}}	%k,\qm-gradient Sobolev space
\newcommand{\curlSob}[2]{H^1_\mathrm{curl}\bracs{#1, \mathrm{d}#2}}		%curl Sobolev space
\newcommand{\tcurlSob}[2]{H^1_{\qm, \mathrm{curl}}\bracs{#1, \mathrm{d}#2}}		%curl + i\qm Sobolev space
\newcommand{\kcurlSob}[2]{H^1_{\wavenumber,\mathrm{curl}}\bracs{#1, \mathrm{d}#2}}	%k-curl Sobolev space
\newcommand{\ktcurlSob}[2]{H^1_{\wavenumber,\qm,\mathrm{curl}}\bracs{#1, \mathrm{d}#2}}	%k,\qm-curl Sobolev space
\newcommand{\ktcurlSobDivFree}[2]{\mathcal{H}^{\kt}\bracs{#1, \mathrm{d}#2}}					%k,\qm-curl, divergence-free Sobolev space
\newcommand{\supp}{\mathrm{supp}}										%support of a function

%grad and curl sets
\newcommand{\gradZero}[2]{\mathcal{G}_{ #1, \mathrm{d}#2}\bracs{0}}		%gradients of zero for domain #1 with measure #2
\newcommand{\kgradZero}[2]{\mathcal{G}_{ #1, \mathrm{d}#2}^{(\wavenumber)}\bracs{0}}	%k-gradients of zero for domain #1 with measure #2
\newcommand{\curlZero}[2]{\mathcal{C}_{ #1, \mathrm{d}#2}\bracs{0}}	%curls of zero for domain #1 with measure #2
\newcommand{\kcurlZero}[2]{\mathcal{C}_{ #1, \mathrm{d}#2}^{(\wavenumber)}\bracs{0}}	%k-curls of zero for domain #1 with measure #2

%derivatives and grad-like symbols
\newcommand{\diff}[2]{\dfrac{\mathrm{d}#1}{\mathrm{d}#2}}			%complete derivative d#1/d#2
\newcommand{\pdiff}[2]{\dfrac{\partial #1}{\partial #2}}			%partial derivative p#1/p#2
\newcommand{\ddiff}[2]{\dfrac{\mathrm{d}^2 #1}{\mathrm{d} {#2}^2}}	%2nd deriv
\newcommand{\grad}{\nabla}											%grad operator
\newcommand{\tgrad}{\nabla^{\qm}}									%grad operator with qm superscript
\newcommand{\kgrad}{\grad^{(\wavenumber)}}							%grad with wavenumber superscript
\newcommand{\ktgrad}{\grad^{\kt}}				%grad with wavenumber, qm superscript
\newcommand{\curl}[1]{\grad_{#1}\wedge}							%curl with subscript #1
\newcommand{\kcurl}[1]{\grad_{#1}^{(\wavenumber)}\wedge}			%k-curl with measure subscript #1
\newcommand{\ktcurl}[1]{\grad_{#1}^{\kt}\wedge}		%k,theta-curl with measure subscript #1
\newcommand{\laplacian}{\Delta}						%laplacian operator, can have subscripts attached

%displaying integrals
\newcommand{\integral}[3]{\int_{#1}#2 \ \mathrm{d}#3}			%integral, domain #1, integrand #2, measure #3
\newcommand{\md}{\mathrm{d}}									%differential d

%convergence
%\newcommand{\lconv}[1]{\xrightarrow{#1}}							%convergence with #1 above the rightarrow - requires mathtools
\newcommand{\lconv}[1]{\overset{#1}{\longrightarrow}}				%convergence with #1 above the rightarrow
\newcommand{\toInfty}[1]{ \ \text{as} \ #1 \rightarrow\infty}		%writes out "as #1 tends to infty" %maths commands, variables, and other packages

%labelling hacks
\newcommand\labelthis{\addtocounter{equation}{1}\tag{\theequation}}

\newcommand{\tlambda}{\widetilde{\lambda}}

%-------------------------------------------------------------------------
%DOCUMENT STARTS

\begin{document}

Let $\ddom$ be a domain in which an embedded graph $\graph$ sits.
The graph $\graph$ naturally separates $\ddom$ into disconnected regions $\ddom_i$ for some appropriate index $i$, where $\ddom = \bigcup_i \ddom_i \cup \graph$.
Write $\tlambda_2 = \lambda_2 + \ddmes$ for the singular measure $\ddmes$ on $\graph$ and the 2D Lebesgue measure $\lambda_2$.

\begin{prop}
	Suppose that $u\in\ktgradSob{\ddom}{\tlambda_2}$, and write
	\begin{align*}
		u = \begin{cases} u_{\lambda} & x\in\ddom_i, \\ u_{jk} & x\in I_{jk}, \end{cases}
		&\qquad
		\tgrad_{\tlambda}u = \begin{cases} \tgrad u_{\lambda} & x\in\ddom_i, \\ \tgrad_{\lambda_{jk}}u_{jk} & x\in I_{jk}. \end{cases}
	\end{align*}
	Note that the form of the tangential gradient is guaranteed from our analysis of $\gradZero{\ddom}{\tlambda_{jk}}$.
	Suppose further that for each $i$, $u\in H^2_{\mathrm{grad}}\bracs{\ddom_i,\md\lambda_2}$.
	Then $u_{\lambda}$ is continuous (or precisely, has a continuous representative) across the edges $I_{jk}$ of $\graph$.
\end{prop}
\begin{proof}
	It suffices for us to consider each edge in turn, so we fix some $I_{jk}$.
	We then provide local labelling as shown in figure \ref{fig:EdgeLocalLabels}; the component $\ddom_i$ with boundary intersecting $I_{jk}$ with \emph{exterior} normal $n_{jk}$ we denote by $\ddom^-$, and the component with boundary intersecting $I_{jk}$ with exterior normal $-n_{jk}$ we denote by $\ddom^+$.
	\begin{figure}[h]
		\centering
		\begin{tikzpicture}
			\draw (-3,0) -- (3,0);
			\filldraw (-3,0) circle (1pt) node[anchor=east] at (-3,0) {$v_j$};
			\filldraw (3,0) circle (1pt) node[anchor=west] at (3,0) {$v_k$};
			\begin{scope}[shift={(-0.5,0.5)}]
				\draw[->] (-5,0) -- (-4,0) node[anchor=west] {$e_{jk}$};
				\draw[->] (-5,0) -- (-5,-1) node[anchor=north] {$n_{jk}$};
			\end{scope}
			\node[anchor=south] at (2,0.5) {$\ddom^-$};
			\node[anchor=north] at (2,-0.5) {$\ddom^+$};
			\draw[dashed] (-2.5,-0.35) rectangle (2,0.35);
			\node[anchor=north] at (-2,-0.35) {$\supp\bracs{\phi}$};
		\end{tikzpicture}
		\caption{\label{fig:EdgeLocalLabels} Local labels for the regions bordering the edge $I_{jk}$, and a schematic illustration of the support of the smooth functions $\phi$ that we consider.}
	\end{figure}
	(Note: the sign conventions are because $\ddom^+$ lies in the positive $n_{jk}$ direction from $I_{jk}$, and vice-versa for $\ddom^-$).
	Now take a smooth function $\phi\in\smooth{\ddom}$ with support satisfying
	\begin{align} \label{eq:PhiProperties}
		\supp\bracs{\phi}\cap I_{jk}\subset I_{jk}^{\circ},
		\quad \supp\bracs{\phi} \cap \ddom_i = \emptyset \text{ except when } \ddom_i = \ddom^+ \text{ or } \ddom^-,
	\end{align}
	that is, $\phi$ has compact support in $\ddom$ which only intersects the interior of $I_{jk}$ and the neighbouring regions $\ddom^+$ and $\ddom^-$.
	
	Now, for $l\in\clbracs{1,2}$ we consider the following integrals.
	\begin{align*}
		\integral{\ddom}{ u\overline{\bracs{\tgrad_{\tlambda_2}\phi}}_l + \overline{\phi}\bracs{\tgrad_{\tlambda_2}u}_l }{\lambda_{jk}}
		&= \integral{I_{jk}}{ u_{jk}e^{(jk)}_l\bracs{\overline{\phi}' - \rmi\qm_{jk}\overline{\phi}} + \overline{\phi}e^{(jk)}_l\bracs{u_{jk}' + \rmi\qm_{jk}u_{jk}} }{\lambda_{jk}} \\
		&= \int_0^{\abs{I_{jk}}} e^{(jk)}_l \bracs{ u_{jk}'\overline{\phi} + u_{jk}\overline{\phi}' } \ \md y \\
		&= 0,
	\end{align*}
	which follows since $u\in\gradSob{\ddom}{\lambda_{jk}}$ and $\supp\bracs{\phi}\subset I_{jk}^\circ$.
	Similarly, 
	\begin{align*}
		\integral{\ddom}{ u\overline{\bracs{\tgrad_{\tlambda_2}\phi}}_l + \overline{\phi}\bracs{\tgrad_{\tlambda_2}u}_l }{\lambda_2}
		&= \integral{\ddom}{ u\partial_l\overline{\phi} + \overline{\phi}\partial_l u }{\lambda_2}
		= 0,
	\end{align*}
	since $u\in\gradSob{\ddom}{\lambda_2}$.
	This implies that
	\begin{align*}
		0 &= \integral{\ddom}{ u\overline{\bracs{\tgrad_{\tlambda_2}\phi}}_l + \overline{\phi}\bracs{\tgrad_{\tlambda_2}u}_l }{\tlambda_2},
	\end{align*}
	for any $\phi$ with the properties \eqref{eq:PhiProperties}.
	However, we can manipulate the integral as follows:
	\begin{align*}
		0 &= \integral{\ddom}{ u\overline{\bracs{\tgrad_{\tlambda_2}\phi}}_l + \overline{\phi}\bracs{\tgrad_{\tlambda_2}u}_l }{\tlambda_2}
		= \integral{\ddom}{ u\partial_l\overline{\phi} + \overline{\phi}\partial_l u }{\lambda_2} \\
		&= \integral{\ddom^+}{ u\partial_l\overline{\phi} + \overline{\phi}\partial_l u }{\lambda_2} 
		+ \integral{\ddom^-}{ u\partial_l\overline{\phi} + \overline{\phi}\partial_l u }{\lambda_2},
	\end{align*}
	and since $u\in H^2_{\mathrm{grad}}\bracs{\ddom^+,\md\lambda_2}$ and $u\in H^2_{\mathrm{grad}}\bracs{\ddom^-,\md\lambda_2}$, we find that
	\begin{align*}
		0 &= \integral{\ddom^+}{ u\partial_l\overline{\phi} + \overline{\phi}\partial_l u }{\lambda_2} 
		+ \integral{\ddom^-}{ u\partial_l\overline{\phi} + \overline{\phi}\partial_l u }{\lambda_2} \\
		&= \integral{\partial\ddom^+}{ u^+\overline{\phi}n^+_l }{S}
		+ \integral{\partial\ddom^-}{ u^-\overline{\phi}n^-_l }{S},
	\end{align*}
	where $u^{\pm}$ denotes the trace onto the boundary of $u_{\lambda}$ from the region $\ddom^{\pm}$, and $n^{\pm}$ is the exterior normal to $\ddom^{\pm}$.
	Since $n^+ = -n_{jk}$ and $n^- = n_{jk}$, and \eqref{eq:PhiProperties}, we can further reduce the above to
	\begin{align*}
		0 &= \integral{I_{jk}}{ \overline{\phi}n^{(jk)}_l\bracs{u^- - u^+} }{S} \\
		&= \int_0^{\abs{I_{jk}}} \overline{\phi}n^{(jk)}_l\bracs{u^- - u^+} \ \md y, \\
		\implies 0 &= \int_0^{\abs{I_{jk}}} \overline{\varphi}n^{(jk)}_l\bracs{u^- - u^+} \ \md y, \quad\forall\varphi\in C^{\infty}_{0}\bracs{0, I_{jk}}.
	\end{align*}
	Therefore, we conclude that $u^- = u^+$ on $I_{jk}$ --- implying that $u$ has a continuous representative across the edge $I_{jk}$.
	Repeating the argument for each edge $I_{jk}$ then completes the proof.
\end{proof}

Note that even though $u_\lambda$ has a representative that is continuous across each edge, this argument doesn't guarantee that it coincides with $u_{jk}$ on the edge.
That being said, it doesn't seem unreasonable to expect this to be the case: I'd be very surprised if the two ``parts" $u_\lambda$ and $u_{jk}$ were completely unrelated from each other, especially since they're approximated by the same smooth sequence in lieu of $u\in\ktgradSob{\ddom}{\tlambda_2}$.

Indeed, we can do better than this with only the assumption that $u\in\ktgradSob{\ddom}{\tlambda_2}$.
\begin{prop}
	Let $u\in\ktgradSob{\ddom}{\tlambda_2}$ and take any $\ddom_i$ and $I_{jk}$ such that $\partial\ddom_i\cap I_{jk}\neq 0$.
	Denote the trace of $u$ from the region $\ddom_i$ as $u_i := \mathrm{Tr}_{\ddom_i}u$, and let $u_{jk}$ be the usual component of $u$ on the edge $I_{jk}$.
	Then $u_{jk} = u_i$ ($\lambda_{jk}$-almost-everywhere) on $I_{jk}$.
\end{prop}
\begin{proof}
	Take an approximating sequence $\phi_n$ for $u$ in $\ktgradSob{\ddom}{\tlambda_2}$; clearly this sequence is also such that
	\begin{align*}
		\phi_n \lconv{\ltwo{\ddom_i}{\lambda_2}} u, &\qquad \grad\phi_n \lconv{\ltwo{\ddom_i}{\lambda_2}^2} \grad u, \\
		 \phi_n \lconv{\ltwo{\ddom}{\lambda_{jk}}} u, &\qquad \ktgrad\phi_n \lconv{\ltwo{\ddom}{\lambda_{jk}}^2} \ktgrad_{\lambda_{jk}} u.
	\end{align*}
	Note: we can drop the $\kt$ when handling Lebesgue measures since the only gradients of zero are zero!
	Furthermore, since $\phi_n\in\smooth{\ddom}$ and both $\phi_n,\grad\phi_n$ converge on all edges $I_{lm}\in\edgeSet$ (so in particular on the edges $I_{lm}$ with non-empty intersection with $\partial\ddom_i$) we have $\phi_n\in C\bracs{\overline{\ddom}_i}$.
	The region $\ddom_i$ is Lipschitz (as it is a polygonal region in the plane) and thus the trace operator from 
	\begin{align*}
		\gradSob{\ddom_i}{\lambda_2} \rightarrow \ltwo{\partial\ddom_i}{S},
	\end{align*}
	is (linear and) continuous.
	Furthermore, since $\phi_n \rightarrow u$ in $\gradSob{\ddom_i}{\lambda_2}$ we have that
	\begin{align*}
		\phi_n \lconv{\ltwo{\ddom_i}{S}} u_i,
	\end{align*}
	by definition of the trace operator.
	In particular, we have that $\phi_n \rightarrow u_i$ in $\ltwo{I_{jk}}{S}$ since $I_{jk}\subset\partial\ddom_i$.
	However,
	\begin{align*}
		\norm{\phi_n - u_i}^2_{\ltwo{\ddom}{\lambda_{jk}}}
		&= \integral{\ddom}{\abs{\phi_n - u_i}^2}{\lambda_{jk}}
		= \int_0^{\abs{I_{jk}}} \abs{\phi_n - u_i}^2 \ \md y
		= \integral{I_{jk}}{\abs{\phi_n - u_i}^2}{S} \\
		&= \norm{\phi_n - u_i}^2_{\ltwo{I_{jk}}{S}}
		\rightarrow 0 \toInfty{n},
	\end{align*}
	and thus we must have that $\phi_n\lconv{\ltwo{\ddom}{\lambda_{jk}}}u_i$ too.
	But $\phi_n\lconv{\ltwo{\ddom}{\lambda_{jk}}}u_{jk}$ as well, and thus it must be the case that $u_{jk} = u_i$, and we are done.
\end{proof}
This implies that any function $u\in\ktgradSob{\ddom}{\tlambda_2}$ is ``continuous across the graph edges" --- that is, the traces of $u$ from any $\ddom_i$ onto a boundary edge $I_{jk}$ must coincide with the value of $u_{jk}$.
This is invaluable information, as now we have a solid way of tying the function on the edges to the function in the larger inclusions.
Furthermore, it also means that we can complete the reformulation of our derived problem into a quantum graph problem involving a non-local operator --- we now just have to read off the information from other edges rather than information about a function on the neighbouring inclusions!

\end{document}