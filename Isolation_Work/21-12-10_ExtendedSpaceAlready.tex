\documentclass[11pt]{report}

\usepackage{url}
\usepackage[margin=2.5cm]{geometry} % See geometry.pdf to learn the layout options. There are lots.
\geometry{a4paper} %or letterpaper or a5paper or ...

\usepackage{graphicx}
\usepackage{tikz}

%\input imports all commands from the target files
%The idea behind this file is that it will be used to store all the maths-related macros that I concoct; so that I can import all the commands by \input{this file} in the preamble of any file that I want to use them in.
%This should make the top-level files look a lot cleaner, and the preamble much shorter!

\usepackage{amssymb}
\usepackage{amsmath}

%theorems and lemma etc setup using amsthm
\usepackage{amsthm}
\newcommand{\tstk}[1]{\textbf{#1} \newline}
\theoremstyle{definition}
\newtheorem{definition}{Definition}[section]
\theoremstyle{plain}
\newtheorem{theorem}{Theorem}[section]
\theoremstyle{plain}
\newtheorem{lemma}[theorem]{Lemma}
\theoremstyle{plain}
\newtheorem{prop}[theorem]{Proposition}

\allowdisplaybreaks %allows equations in the same align environment to split over multiple pages.

%begin the macros via newcommand. Try to group them up reasonably!

%standard sets
\newcommand{\naturals}{\mathbb{N}}			%natural numbers
\newcommand{\integers}{\mathbb{Z}}			%integers
\newcommand{\rationals}{\mathbb{Q}}			%rational numbers
\newcommand{\reals}{\mathbb{R}}				%real numbers
\newcommand{\complex}{\mathbb{C}}			%complex numbers

%brackets and norms
\newcommand{\bracs}[1]{\left( #1 \right)}				%encloses input in brackets
\newcommand{\sqbracs}[1]{\left[ #1 \right]}				%encloses input in square brackets
\newcommand{\clbracs}[1]{\left\{ #1 \right\}}			%encloses input in curly bracers
\newcommand{\abs}[1]{\lvert #1 \rvert}					%absolute value
\newcommand{\norm}[2]{\lvert\lvert #1 \rvert\rvert}		%norm (double line)

%function sets
\newcommand{\smooth}[1]{C^{\infty}\bracs{#1}}							%smooth functions
\newcommand{\ltwo}[2]{L^{2}\bracs{#1,\mathrm{d}#2}}						%general L^2 space
\newcommand{\gradSob}[2]{H^1_\mathrm{grad}\bracs{#1, \mathrm{d}#2}}		%gradient Sobolev space
\newcommand{\curlSob}[2]{H^1_\mathrm{curl}\bracs{#1, \mathrm{d}#2}}		%curl Sobolev space
\newcommand{\kSob}[2]{H^1_{k,\mathrm{curl}}\bracs{#1, \mathrm{d}#2}}	%k-curl Sobolev space

%grad and curl sets
\newcommand{\gradZero}[2]{\mathcal{G}_{ #1, \mathrm{d}#2}\bracs{0}}		%gradients of zero for domain #1 with measure #2
\newcommand{\curlZero}[2]{\mathcal{C}_{ #1, \mathrm{d}#2}\bracs{0}}	%curls of zero for domain #1 with measure #2

%derivatives and grad-like symbols
\newcommand{\diff}[2]{\dfrac{\mathrm{d}#1}{\mathrm{d}#2}}			%complete derivative d#1/d#2
\newcommand{\pdiff}[2]{\dfrac{\partial #1}{\partial #2}}			%partial derivative p#1/p#2
\newcommand{\ddiff}[2]{\dfrac{\mathrm{d}^2 #1}{\mathrm{d}^2 #2}}	%2nd deriv
\newcommand{\grad}{\nabla}											%grad operator
\newcommand{\curl}[1]{\grad_{#1}\wedge}								%curl with measure subscript #1

%displaying integrals
\newcommand{\integral}[3]{\int_{#1}#2 \ \mathrm{d}#3}			%integral, domain #1, integrand #2, measure #3

%notation for variable use throughout the file
\newcommand{\dddom}{\widetilde{\Omega}}			%3D domain notation
\newcommand{\ddom}{\Omega}						%2D domain notation
\newcommand{\dddmes}{\widetilde{\mu}}			%3D measure
\newcommand{\ddmes}{\mu}						%2D measure

\newcommand{\graph}{\mathbb{G}}					%graph variable %maths commands, variables, and other packages

%labelling hacks
\newcommand\labelthis{\addtocounter{equation}{1}\tag{\theequation}}

\newcommand{\aop}{\mathcal{A}}
\newcommand{\hcal}{\mathcal{H}}
\newcommand{\lcal}{\mathcal{L}}
%-------------------------------------------------------------------------
%DOCUMENT STARTS

\begin{document}

Let $\ddom=\left[0,1\right)^2$ be our usual domain filled with a singular structure $\graph$, separated by $\graph$ into the pairwise-disjoint connected components $\ddom_i, i\in\Lambda$ for some finite index set $\Lambda$.
Set $N = \abs{\vertSet}$ to be the number of vertices, and $L=\abs{\Lambda}$ be the number of bulk regions.
Also denote by $\compMes = \lambda_2 + \ddmes$, and for coupling constants $\alpha_j>0$ at the vertices $v_j$ let $\nu = \sum_{v_j\in\vertSet}\alpha_j\delta_{v_j}$ be a weighted sum of point-mass measures centred at the vertices.
On an edge $I_{jk}$, we denote by $\ddom_+$ the bulk region in the direction $n_{jk}$ from $I_{jk}$, and $\ddom_-$ the bulk region in the direction $-n_{jk}$ from $I_{jk}$.
Denote by $n^{\pm}$ the unit exterior normal to $\ddom_{\pm}$ (noting that $n^{\pm}=\mp n_{jk}$), and write $\pdiff{u^{\pm}}{n^{\pm}}$ to be the normal derivative on $\partial\ddom_{\pm}$ of the function $u$ restricted to $\ddom_{\pm}$.

The ``strong formulation" of our composite medium problem is
\begin{subequations} \label{eq:StrongForm}
	\begin{align}
		-\laplacian_{\qm}u &= \omega^2 u, &\qquad\text{in } \ddom_i, \ \forall i\in\Lambda, \\
		-\bracs{\diff{}{y}+\rmi\qm_{jk}}^2 u_{jk} - \bracs{\pdiff{u^+}{n^+} + \pdiff{u^-}{n^-}} &= \omega^2 u_{jk},  &\qquad\text{on every } I_{jk}\in\edgeSet, \\
		\sum_{j\con k}\bracs{\pdiff{}{n}+\rmi\qm_{jk}}u_{jk}(v_j) &= 0, &\qquad\text{at every } v_j\in\vertSet,
	\end{align}
\end{subequations}
where the function $u$ is $\gradgradSob{\ddom_i}{\lambda_2}$ for every $i$, $H^2(I_{jk})$ for every $I_{jk}$, is continuous (in the sense of traces) across $I_{jk}$, and is continuous at the vertices $v_j$.
Recall that this problem was derived from the variational problem of finding $u\in\tgradSob{\ddom}{\compMes}$ such that
\begin{align} \label{eq:WeakForm}
	\integral{\ddom}{ \tgrad_{\compMes}u\cdot\overline{\tgrad_{\compMes}} }{\compMes} &=
	\omega^2\integral{\ddom}{ u\overline{\phi} }{\compMes}, \quad\forall\phi\in\smooth{\ddom}.
\end{align}

We are interested in constructing an extended space $\hcal$ in which the problem \eqref{eq:StrongForm} reads as a standard eigenvalue problem $\aop u = \omega^2 u$ for some operator $\aop$ on $\hcal$.
With this in mind, consider the space
\begin{align*}
	\hcal &= \bracs{\bigoplus_{i\in\Lambda}\gradgradSob{\ddom_i}{\lambda_2}} \oplus H^2\bracs{\graph} \oplus \ltwo{\ddom}{\nu},
\end{align*}
viewed as a subspace of
\begin{align*}
	\lcal := \bracs{\bigoplus_{i\in\Lambda}\ltwo{\ddom_i}{\lambda_2}} \oplus L^2\bracs{\graph} \oplus \ltwo{\ddom}{\nu},
\end{align*}
where we denote an element $u$ of $\lcal$ by the $\bracs{L+\abs{\edgeSet}+N}$-``vector"
\begin{align*}
	u &= \bracs{\clbracs{u_{\ddom_i}}_{i\in\Lambda}, \clbracs{u_{jk}}_{I_{jk}\in\edgeSet}, \clbracs{u(v_j)}_{v_j\in\vertSet}}^\top.
\end{align*}
Note that $\ltwo{\ddom}{\nu}\cong\complex^N$.
Define the operator $\aop$ by
\begin{align*}
	\dom\bracs{\aop} &= \hcal, \\
	\aop u &= 
	\begin{pmatrix}	
	\clbracs{-\laplacian_{\qm}u_{\ddom_i}}_{i\in\Lambda} \\
	\clbracs{-\bracs{\diff{}{y}+\rmi\qm_{jk}}^2 u_{jk} - \bracs{\tgrad u_{\ddom_+}\cdot n^+ + \tgrad u_{\ddom_-}\cdot n^-} }_{I_{jk}\in\edgeSet} \\
	\clbracs{\sum_{j\con k}\bracs{\pdiff{}{n}+\rmi\qm_{jk}}u_{jk}(v_j)}_{v_j\in\vertSet}
	\end{pmatrix}
	\in\lcal,
\end{align*}
and (setting aside questions about the nature of the spectrum, etc) consider the eigenvalue problem
\begin{align*}
	\aop u = \omega^2 u,
\end{align*}
which without much effort can be shown to be equivalent to
\begin{subequations} \label{eq:aopEvalProb}
	\begin{align}
		-\laplacian_{\qm}u &= \omega^2 u, &\qquad\text{in } \ddom_i, \ \forall i\in\Lambda, \\
		-\bracs{\diff{}{y}+\rmi\qm_{jk}}^2 u_{jk} - \bracs{\tgrad u_{\ddom_+}\cdot n^+ + \tgrad u_{\ddom_-}\cdot n^-} &= \omega^2 u_{jk},  &\qquad\text{on every } I_{jk}\in\edgeSet, \\
		\sum_{j\con k}\bracs{\pdiff{}{n}+\rmi\qm_{jk}}u_{jk}(v_j) &= \omega^2\alpha_j u(v_j), &\qquad\text{at every } v_j\in\vertSet.		
	\end{align}
\end{subequations}
Here, we denote by $\tgrad u_{\ddom_+}\cdot n^+$ the trace of the $\qm$-shifted gradient $\tgrad$ of the function $u_{\ddom_{\pm}}\cdot n^{\pm}$ onto $I_{jk}$.

Note that \eqref{eq:aopEvalProb} is not equivalent to \eqref{eq:StrongForm}, as functions in $\hcal$ are not required to adhere to continuity across the skeleton (hence the $\tgrad$ terms rather than just normal derivatives) nor continuity at the vertices (the notation $u(v_j)$ is just some value in $\complex$ that $u\in\hcal$ takes at each $v_j$).
We can then ask if there is a form from which $\aop$ can be defined, with this in mind, notice that
\begin{align*}
	\ip{\aop u}{\phi}_{\lcal} &= \sum_{i\in\Lambda}\integral{\ddom_i}{ \tgrad u_{\ddom_i}\cdot\overline{\tgrad\phi} }{\lambda_2}
	+ \sum_{v_j\in\vertSet}\sum_{j\conLeft k}\integral{\ddom}{ \bracs{\diff{}{y}+\rmi\qm_{jk}}u\overline{\bracs{\diff{}{y}+\rmi\qm_{jk}}\phi} }{\lambda_{jk}} \\
	&= \sum_{i\in\Lambda}\ip{ \tgrad u }{ \tgrad\phi }_{\ltwo{\ddom_i}{\lambda_i}}
	+ \sum_{v_j\in\vertSet}\sum_{j\conLeft k}\ip{ \bracs{\diff{}{y}+\rmi\qm_{jk}}u }{ \bracs{\diff{}{y}+\rmi\qm_{jk}}\phi }_{\ltwo{I_{jk}}{\lambda_{jk}}},
\end{align*}
so if we define (for $u\in\hcal$)
\begin{align*}
	\tgrad_{\hcal} u &:= \bracs{\clbracs{\tgrad u_{\ddom_i}}_{i\in\Lambda}, \clbracs{\bracs{\diff{}{y}+\rmi\qm_{jk}}u_{jk}}_{I_{jk}\in\edgeSet}, \clbracs{0}_{v_j\in\vertSet}}^\top,
\end{align*}
we have that
\begin{align*}
	\ip{\aop u}{\phi}_{\lcal} &= \ip{\tgrad_{\hcal} u}{\tgrad_{\hcal}\phi}_{\lcal}.
\end{align*}
Thus, we can define $\aop$ from the bilinear form $b(u,v) = \ip{\tgrad_{\hcal} u}{\tgrad_{\hcal}\phi}_{\lcal}$ for $u,v\in\hcal$, and write the eigenvalue problem for $\aop$ as
\begin{align} \label{eq:aopWeakEvalProb}
	\ip{\tgrad_{\hcal} u}{\tgrad_{\hcal}\phi}_{\lcal} &= \omega^2\ip{u}{\phi}_{\lcal}.
\end{align}

The similarities between \eqref{eq:StrongForm} and \eqref{eq:aopEvalProb} are apparent --- if we take $\alpha_j=0$ for every $j$, then \eqref{eq:WeakForm} and \eqref{eq:aopWeakEvalProb} are the same problem, and our ``definition" of $\tgrad_{\hcal} u$ for $u\in\hcal$ coincides with $\tgrad_{\compMes}u$ for $u\in\tgradSob{\ddom}{\compMes}$.
We even have that the set of $u\in\tgradSob{\ddom}{\compMes}$ that solve \eqref{eq:WeakForm} is a subset of $\hcal$.

Furthermore, if we then ``switch on" the $\alpha_j>0$, then $\tgrad_{\hcal} u$ still coincides with what we expect the tangential gradient in $\tgradSob{\ddom}{(\compMes+\nu)}$ to be.
The equations \eqref{eq:WeakForm} (replacing $\compMes$ with $\compMes+\nu$) and \eqref{eq:aopWeakEvalProb} are identical, and the problem \eqref{eq:aopEvalProb} (restricted to $\tgradSob{\ddom}{(\compMes+\nu)}$) reduces to what my current ``guess" at the analogue of \eqref{eq:StrongForm} with the point masses included would be.

\end{document}