\documentclass[11pt]{report}

\usepackage{url}
\usepackage[margin=2.5cm]{geometry} % See geometry.pdf to learn the layout options. There are lots.
\geometry{a4paper} %or letterpaper or a5paper or ...

\usepackage{graphicx}
\usepackage{tikz}

%\input imports all commands from the target files
%The idea behind this file is that it will be used to store all the maths-related macros that I concoct; so that I can import all the commands by \input{this file} in the preamble of any file that I want to use them in.
%This should make the top-level files look a lot cleaner, and the preamble much shorter!

\usepackage{amssymb}
\usepackage{amsmath}
%\usepackage{mathtools}

%theorems and lemma etc setup using amsthm
\usepackage{amsthm}
\theoremstyle{definition}
\newtheorem{definition}{Definition}[section]
\theoremstyle{plain}
\newtheorem{theorem}{Theorem}[section]
\theoremstyle{plain}
\newtheorem{lemma}[theorem]{Lemma}
\theoremstyle{plain}
\newtheorem{prop}[theorem]{Proposition}
\theoremstyle{plain}
\newtheorem{cory}[theorem]{Corollary}
\theoremstyle{definition}
\newtheorem{convention}[theorem]{Convention}
\theoremstyle{definition}
\newtheorem{assumption}[theorem]{Assumption}
\theoremstyle{definition}
\newtheorem{conjecture}[theorem]{Conjecture}

\allowdisplaybreaks %allows equations in the same align environment to split over multiple pages.

%tstk always need be there
\newcommand{\tstk}[1]{\textbf{#1} \newline}

%this adds extra functionality to pmatrix, vmatrix, bmatrix etc by allowing you to pass an optional argument in [FACTOR] to multiply the default spacing between elements by FACTOR
\makeatletter
\renewcommand*\env@matrix[1][\arraystretch]{%
  \edef\arraystretch{#1}%
  \hskip -\arraycolsep
  \let\@ifnextchar\new@ifnextchar
  \array{*\c@MaxMatrixCols c}
  }
\makeatother

%begin the macros via newcommand. Try to group them up reasonably!

%notation and variable use throughout the file
\renewcommand{\vec}[1]{\mathbf{#1}}				%vectors are bold, not overline arrow
\newcommand{\recip}[1]{\frac{1}{#1}}			%fast reciprocal as a fraction
\newcommand{\interval}[1]{\sqbracs{0,\abs{#1}}}	%creates the closed interval from 0 to the length of the input #1, denoted by absolute value
\newcommand{\eps}{\varepsilon}					%pretty epsilons
\newcommand{\charFunc}[1]{\mathcal{I}_{#1}}%{\mathbb{1}_{#1}}		%characteristic function of a set

\newcommand{\dddom}{\widetilde{\Omega}}			%3D domain notation
\newcommand{\ddom}{\Omega}						%2D domain notation
\newcommand{\dddmes}{\widetilde{\mu}}			%3D measure
\newcommand{\ddmes}{\mu}						%2D measure

\newcommand{\graph}{\mathbb{G}}					%graph variable
\newcommand{\vertSet}{\mathcal{V}}					%set of vertices rather than big V
\newcommand{\edgeSet}{\mathcal{E}}					%set of edges rather than large E, \graph = (V,E)
\newcommand{\wavenumber}{\kappa}				%fourier variable or wavenumber, not to confuse with jk subscripts!
\newcommand{\qm}{\theta}						%quasi-momentum parameter
\newcommand{\kt}{\bracs{\wavenumber, \qm}}		%(k, theta) pair
\newcommand{\dmap}{\Gamma_0}					%Dirichlet map
\newcommand{\nmap}{\Gamma_1}					%Neumann map
\newcommand{\effFreq}{\Lambda}					%Effective frequency sqrt(w^2-\wavenumber^2)
\newcommand{\conLeft}{\stackrel{\rightarrow}{\smash{\sim}\rule{0pt}{0.4ex}}} %j connects to k, j left
\newcommand{\conRight}{\stackrel{\leftarrow}{\smash{\sim}\rule{0pt}{0.4ex}}} %j connects to k, j right
\newcommand{\con}{\sim}							%j connects to k, indifferent of direction

%standard sets
\newcommand{\naturals}{\mathbb{N}}			%natural numbers
\newcommand{\integers}{\mathbb{Z}}			%integers
\newcommand{\rationals}{\mathbb{Q}}			%rational numbers
\newcommand{\reals}{\mathbb{R}}				%real numbers
\newcommand{\complex}{\mathbb{C}}			%complex numbers

%other notations
\newcommand{\rmi}{\mathrm{i}}				%imaginary unit i (RoMan font i)
\newcommand{\e}{\mathrm{e}}					%Euler number, e (Roman font e)

%brackets and norms
\newcommand{\bracs}[1]{\left( #1 \right)}				%encloses input in brackets
\newcommand{\sqbracs}[1]{\left[ #1 \right]}				%encloses input in square brackets
\newcommand{\clbracs}[1]{\left\{ #1 \right\}}			%encloses input in curly bracers
\newcommand{\abs}[1]{\left\lvert #1 \right\rvert}					%absolute value
\newcommand{\norm}[1]{\lvert\lvert #1 \rvert\rvert}		%norm 
\newcommand{\setVert}{\ \middle\vert \ }					%vertical bar for the middle of sets

%function sets
\newcommand{\smooth}[1]{C^{\infty}\bracs{#1}}							%smooth functions
\newcommand{\ltwo}[2]{L^{2}\bracs{#1,\mathrm{d}#2}}						%general L^2 space
\newcommand{\gradSob}[2]{H^1_\mathrm{grad}\bracs{#1, \mathrm{d}#2}}		%gradient Sobolev space
\newcommand{\gradSobQM}[2]{H^1_{\qm, \mathrm{grad}}\bracs{#1, \mathrm{d}#2}} %gradient + i\qm Sobolev space
\newcommand{\ktgradSob}[2]{H^1_{\wavenumber,\qm,\mathrm{grad}}\bracs{#1, \mathrm{d}#2}}	%k,\qm-gradient Sobolev space
\newcommand{\curlSob}[2]{H^1_\mathrm{curl}\bracs{#1, \mathrm{d}#2}}		%curl Sobolev space
\newcommand{\tcurlSob}[2]{H^1_{\qm, \mathrm{curl}}\bracs{#1, \mathrm{d}#2}}		%curl + i\qm Sobolev space
\newcommand{\kcurlSob}[2]{H^1_{\wavenumber,\mathrm{curl}}\bracs{#1, \mathrm{d}#2}}	%k-curl Sobolev space
\newcommand{\ktcurlSob}[2]{H^1_{\wavenumber,\qm,\mathrm{curl}}\bracs{#1, \mathrm{d}#2}}	%k,\qm-curl Sobolev space
\newcommand{\ktcurlSobDivFree}[2]{\mathcal{H}^{\kt}\bracs{#1, \mathrm{d}#2}}					%k,\qm-curl, divergence-free Sobolev space
\newcommand{\supp}{\mathrm{supp}}										%support of a function

%grad and curl sets
\newcommand{\gradZero}[2]{\mathcal{G}_{ #1, \mathrm{d}#2}\bracs{0}}		%gradients of zero for domain #1 with measure #2
\newcommand{\kgradZero}[2]{\mathcal{G}_{ #1, \mathrm{d}#2}^{(\wavenumber)}\bracs{0}}	%k-gradients of zero for domain #1 with measure #2
\newcommand{\curlZero}[2]{\mathcal{C}_{ #1, \mathrm{d}#2}\bracs{0}}	%curls of zero for domain #1 with measure #2
\newcommand{\kcurlZero}[2]{\mathcal{C}_{ #1, \mathrm{d}#2}^{(\wavenumber)}\bracs{0}}	%k-curls of zero for domain #1 with measure #2

%derivatives and grad-like symbols
\newcommand{\diff}[2]{\dfrac{\mathrm{d}#1}{\mathrm{d}#2}}			%complete derivative d#1/d#2
\newcommand{\pdiff}[2]{\dfrac{\partial #1}{\partial #2}}			%partial derivative p#1/p#2
\newcommand{\ddiff}[2]{\dfrac{\mathrm{d}^2 #1}{\mathrm{d} {#2}^2}}	%2nd deriv
\newcommand{\pddiff}[2]{\dfrac{\partial^2 #1}{\partial {#2}^2}}		%2nd partial derivative
\newcommand{\grad}{\nabla}											%grad operator
\newcommand{\tgrad}{\nabla^{\qm}}									%grad operator with qm superscript
\newcommand{\kgrad}{\grad^{(\wavenumber)}}							%grad with wavenumber superscript
\newcommand{\ktgrad}{\grad^{\kt}}				%grad with wavenumber, qm superscript
\newcommand{\curl}[1]{\grad_{#1}\wedge}							%curl with subscript #1
\newcommand{\kcurl}[1]{\grad_{#1}^{(\wavenumber)}\wedge}			%k-curl with measure subscript #1
\newcommand{\ktcurl}[1]{\grad_{#1}^{\kt}\wedge}		%k,theta-curl with measure subscript #1
\newcommand{\laplacian}{\Delta}						%laplacian operator, can have subscripts attached

%displaying integrals
\newcommand{\integral}[3]{\int_{#1}#2 \ \mathrm{d}#3}			%integral, domain #1, integrand #2, measure #3
\newcommand{\md}{\mathrm{d}}									%differential d

%convergence
%\newcommand{\lconv}[1]{\xrightarrow{#1}}							%convergence with #1 above the rightarrow - requires mathtools
\newcommand{\lconv}[1]{\overset{#1}{\longrightarrow}}				%convergence with #1 above the rightarrow
\newcommand{\toInfty}[1]{ \ \text{as} \ #1 \rightarrow\infty}		%writes out "as #1 tends to infty" %maths commands, variables, and other packages

%labelling hacks
\newcommand\labelthis{\addtocounter{equation}{1}\tag{\theequation}}

\newcommand{\psmooth}[1]{C^{\infty}_{\#}\bracs{#1}}
\newcommand{\csmooth}[1]{C^{\infty}_{\mathrm{c}}\bracs{#1}}
%-------------------------------------------------------------------------
%DOCUMENT STARTS

\begin{document}

Let $\ddom=\left[0,1\right)^2$ be our period cell, $\graph$ our usual period graph, and let $Q = \ddom^{\circ} = \bracs{0,1}^2$.
To be precise in this document, we'll use:
\begin{itemize}
	\item $\psmooth{\ddom}$ to denote the set of smooth, $\ddom$-periodic functions on $\reals^2$.
	\item $\csmooth{Q}$ to denote the set of smooth functions on $Q$ with compact support in $Q$.
	\item $\smooth{\ddom}$ to denote the set of smooth functions on $\ddom$.
\end{itemize}
Let $\rho$ be a Borel measure on $\ddom$.
We'll also be precise in our definitions:
\begin{align*}
	W^{\kt}_{\rho, \mathrm{grad}}\bracs{\ddom} &= \overline{ \clbracs{ \bracs{\phi, \ktgrad\phi} \setVert \phi\in\psmooth{\ddom} } },
\end{align*}
where the closure is taken in $\ltwo{\ddom}{\rho}\times\ltwo{\ddom}{\rho}^2$.
Also then define
\begin{align*}
	\gradZero{\ddom}{\rho} &= \clbracs{ g \setVert \bracs{0,g}\in W^{\kt}_{\rho, \mathrm{grad}}\bracs{\ddom} } \\
	&= \clbracs{ g \setVert \exists\phi_n\in\psmooth{\ddom} \text{ s.t. } \phi_n\lconv{\ltwo{\ddom}{\rho}}0, \ \ktgrad\phi_n\lconv{\ltwo{\ddom}{\rho}^2}g }, \\
	\ktgradSob{\ddom}{\rho} &= \clbracs{ \bracs{u,\ktgrad_{\rho}u}\in W^{\kt}_{\rho, \mathrm{grad}}\bracs{\ddom} \setVert \ktgrad_{\rho}u\perp\gradZero{\ddom}{\rho} }.
\end{align*}

For the majority of our graph-only purposes (that is, the scalar and curl problems we've already considered) the measures we use are only concerned with the behaviour of functions on the edges of $\graph$, and what's going on ``off" the graph doesn't particularly matter.
Indeed, if we have an edge that is entirely contained within the interior of the period cell, then that edge does not even feel the affect of the $\ddom$-periodicity:
\begin{prop} \label{prop:p1}
	Let $I_{jk}\subset\ddom$ be such that $d := \mathrm{dist}\bracs{I_{jk},\partial\ddom} >0$.
	Then the following are all equivalent:
	\begin{enumerate}
		\item $\bracs{u,g}\in W^{\kt}_{\lambda_{jk}, \mathrm{grad}}\bracs{\ddom}$,
		\item $\exists\psi_n\in\csmooth{Q} \text{ s.t. } \psi_n\lconv{\ltwo{\ddom}{\lambda_{jk}}}u, \ \ktgrad\psi_n\lconv{\ltwo{\ddom}{\lambda_{jk}}^2}g$,
		\item $\exists\varphi_n\in\smooth{\ddom} \text{ s.t. } \varphi_n\lconv{\ltwo{\ddom}{\lambda_{jk}}}u, \ \ktgrad\varphi_n\lconv{\ltwo{\ddom}{\lambda_{jk}}^2}g$,
	\end{enumerate}
\end{prop}
\begin{proof}
	($1. \Rightarrow 2.$) Since there is positive distance between the edge $I_{jk}$ and the boundary of $\ddom$, we can multiply $\phi_n$ by a (bounded) smooth function that is 1 on the set $\clbracs{x\in\ddom \setVert \mathrm{dist}\bracs{x,I_{jk}}}\leq\frac{d}{3}$ and 0 on the set $\clbracs{x\in\ddom \setVert \mathrm{dist}\bracs{x,I_{jk}}}\geq\frac{2d}{3}$.
	This then forms the sequence $\psi_n$, which is equal to $\phi_n$ on $I_{jk}$. \newline
	($1. \Leftarrow 2.$) Extend $\psi_n$ by zero to $\ddom$, and then by periodicity to $\reals^2$.
	The resulting sequence $\phi_n$ ensures that $\bracs{u,g}\in W^{\kt}_{\lambda_{jk}, \mathrm{grad}}\bracs{\ddom}$, since $\phi_n = \psi_n$ on $I_{jk}$ still. \newline
	($2. \Rightarrow 3.$) Extending $\psi_n$ by zero suffices. \newline
	($2. \Leftarrow 3.$) Multiplying $\varphi_n$ by a smooth function as described in the proof of $1. \Rightarrow 2.$ suffices.
\end{proof}
We can produce a similar result to proposition \ref{prop:p1}, but concerning $\ddmes$ when $\graph$ has positive distance to $\partial\ddom$.

This result also means that, whenever one of the edges we come to consider is contained in $Q$, we have the option of looking for $\csmooth{Q}$-functions, or $\smooth{\ddom}$-functions to approximate $\bracs{u,g}$ rather than $\psmooth{\ddom}$ (then applying the proposition).

We can also produce a similar result to proposition \ref{prop:p1} for when we have an edge with \emph{exactly} one of its vertices on the boundary of $\ddom$.
In such a situation, apply a translation $T$ to $Q$ so that $I_{jk}\subset Q+T$ (which can be done since exactly one of $I_{jk}$'s edges lies on $\partial\ddom$).
We then work in this translated cell $T+Q$, can use the above result, and then ``translate" the resulting smooth functions we find back to functions in $\psmooth{\ddom}$.
So without loss of generality, we can consider graphs without any edges with exactly one vertex on the boundary of $\ddom$ (since the arguments above deal with this special case).

The only remaining possibility is when $I_{jk}$ has both vertices on the boundary of $\ddom$, and we can't use the above proposition (edges like this are what I was referring to as ``long edges").
However we do have some freedoms in our choice of approximating sequence, depending on the orientation of $I_{jk}$.
\begin{itemize}
	\item If $I_{jk}$ is horizontal, it is enough for us to approximate $\bracs{u,g}$ via smooth functions $\psi_n\in\smooth{\reals^2}$ that are $1$-periodic in $x_1$, IE are such that $\psi_n(x_1+1,x_2) = \psi_n(x_1,x_2)$ for any $x_1\in\reals$, to show that $\bracs{u,g}\in W^{\kt}_{\lambda_{jk}, \mathrm{grad}}\bracs{\ddom}$.
	Indeed, suppose that we have such a sequence $\psi_n$.
	Let $d$ be the distance between $I_{jk}$ and the horizontal boundaries of $\ddom$ --- if $d=0$ we can use the ``translation" trick above to circumnavigate this, so we assume $d>0$. 
	Then let $\chi:\ddom\rightarrow\sqbracs{0,1}$ be a smooth function such that $\chi(x)=1$ whenever $\mathrm{dist}(x,I_{jk})<\frac{d}{3}$ and $\chi(x)=0$ whenever $\mathrm{dist}(x,I_{jk})>\frac{2d}{3}$.
	This function $\chi$ can then be extended to a smooth function $\hat{\chi}$ that is $1$-periodic in $x_2$ and constant in $x_1$.
	Then the sequence $\phi_n:=\psi_n\hat{\chi}\in\psmooth{\ddom}$ and approximates $\bracs{u,g}$ in $\ltwo{\ddom}{\lambda_{jk}}$ as required.
	
	A converse statement also holds --- if we have an approximating sequence $\phi_n\in\psmooth{\ddom}$ for $\bracs{u,g}\in W^{\kt}_{\lambda_{jk}, \mathrm{grad}}\bracs{\ddom}$ then without loss of generality can assume it has the form $\phi_n = \psi_n\hat{\chi}$ where $\psi_n$ and $\chi$ have the properties prescribed above.
	\item If $I_{jk}$ is vertical, we can repeat the horizontal argument but interchange $x_1$ and $x_2$.
	\item The only remaining possibility is when $I_{jk}$ is diagonal, so $\clbracs{v_j,v_k}=\clbracs{\bracs{0,0}^\top, \bracs{1,1}^\top}$ or $\clbracs{v_j,v_k}=\clbracs{\bracs{0,1}^\top, \bracs{1,0}^\top}$.
	Suppose that we have a sequence of functions $\psi_n\in\smooth{\reals^2}$ with ``$\abs{I_{jk}}e_{jk}$"-periodicity, that is $\psi_n\bracs{x+\abs{I_{jk}}e_{jk}}=\psi_n(x)$ for any $x\in\reals^2$, and which approximates $\bracs{u,g}$.
	Then we can form an approximating sequence $\phi_n\in\psmooth{\ddom}$ for $\bracs{u,g}$ from $\psi_n$.
	Take $\hat{\psi}_n(x) = \psi_n\bracs{x - \bracs{x-v_j}\cdot n_{jk} n_{jk}}$, and let $\chi:\reals\rightarrow\sqbracs{0,1}$ be a smooth function such that $\chi(y)=1$ when $\abs{y}<\recip{8}$ and $\chi(x)=0$ when $\abs{y}>\recip{4}$.
	Define $\hat{\chi}(x) = \chi\bracs{\bracs{x-v_j}\cdot n_{jk}}$.
	By construction, the function $\hat{\psi}_n\hat{\chi}\in\smooth{\reals^2}$ and has support $\supp\bracs{\hat{\psi}_n\hat{\chi}}=\clbracs{x = v_j + \alpha e_{jk} + \beta n_{jk} \setVert \alpha\in\reals, \beta\in\sqbracs{-\recip{4},\recip{4}}}$ --- that is a ``strip" of width $\recip{2}$ whose centre is the line $L := \clbracs{v_j + \alpha e_{jk} \setVert \alpha\in\reals}$.
	This construction ensures that $\hat{\psi}_n\hat{\chi} = \psi_n$ on $I_{jk}$, and we can define
	\begin{align*}
		\phi_n:\reals^2\rightarrow\complex, &\qquad \phi_n(x) = \hat{\psi}_n(y)\hat{\chi}(y) \\
		\text{ where } x=y+Z, \ Z\in\integers^2, 
		&\qquad y\in\clbracs{x = v_j + \alpha e_{jk} + \beta n_{jk} \setVert \alpha\in\reals, \beta\in\sqbracs{-\recip{2},\recip{2}}}
	\end{align*}
	The function $\phi_n$ is an element of $\psmooth{\ddom}$ by our construction, and since $\phi_n=\psi_n$ on $I_{jk}$, it will also approximate $\bracs{u,g}$.
	
	Again, a converse of sorts holds in that any approximating sequence $\phi_n\in\psmooth{\ddom}$ for $\bracs{u,g}$ can without loss of generality be assumed to have the form $\phi_n(x) = \hat{\psi}_n(y)\hat{\chi}(y)$ as above.
\end{itemize}

These arguments essentially tell us that the bulk of the work in finding approximating sequences lies in what the approximating sequence is doing \emph{on the edge $I_{jk}$}.
Once we know the behaviour we need from our approximating sequences on an edge $I_{jk}$, the above results/remarks reduce our task to simply finding a smooth function on $\ddom$ which has these desired properties on $I_{jk}$, and from there we could follow the reasoning above to create our sequence in $\psmooth{\ddom}$.
As such, for the majority of the scalar and curl chapters, I stuck to the notation ``$\smooth{\ddom}$" for the set that in this document is denoted as $\psmooth{\ddom}$ --- see the sections where $\ktgrad{\ddom}{\ddmes}$ and $\gradZero{\ddom}{\ddmes}$ are defined, for example.
Some of the proofs in those chapters also implicitly assume the remarks and reasoning above; for example when examining gradients of zero on $I_{jk}$, we show that functions of the form $g(x)n_{jk}$ are gradients of zero via the approximating sequence $\phi(x) = \bracs{x\cdot n_{jk}}g(x)$.
Obviously without qualification, $\phi$ here is not in $\psmooth{\ddom}$, but given the remarks above we know we \emph{can find} $\hat{\phi}\in\psmooth{\ddom}$ from $\phi$.

\end{document}