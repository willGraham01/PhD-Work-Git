\documentclass[11pt]{report}

\usepackage{url}
\usepackage[margin=2.5cm]{geometry} % See geometry.pdf to learn the layout options. There are lots.
\geometry{a4paper} %or letterpaper or a5paper or ...

%for figures and graphics
\usepackage{graphicx}
\usepackage{subcaption} %allows subfigures
\usepackage[bottom]{footmisc} %footnotes go below figures
%\usepackage{parskip} %adds line space between paragraphs by default
\usepackage{enumerate} %allows lower case roman numerials in enumerate environments

\DeclareGraphicsRule{.tif}{png}{.png}{`convert #1 `dirname #1`/`basename #1 .tif`.png}
\graphicspath{{../Diagrams/Diagram_PDFs/} {../Diagrams/Numerical_Results/}}

%\input imports all commands from the target files
%The idea behind this file is that it will be used to store all the maths-related macros that I concoct; so that I can import all the commands by \input{this file} in the preamble of any file that I want to use them in.
%This should make the top-level files look a lot cleaner, and the preamble much shorter!

\usepackage{amssymb}
\usepackage{amsmath}
%\usepackage{mathtools}

%theorems and lemma etc setup using amsthm
\usepackage{amsthm}
\theoremstyle{definition}
\newtheorem{definition}{Definition}[section]
\theoremstyle{plain}
\newtheorem{theorem}{Theorem}[section]
\theoremstyle{plain}
\newtheorem{lemma}[theorem]{Lemma}
\theoremstyle{plain}
\newtheorem{prop}[theorem]{Proposition}
\theoremstyle{plain}
\newtheorem{cory}[theorem]{Corollary}
\theoremstyle{definition}
\newtheorem{convention}[theorem]{Convention}
\theoremstyle{definition}
\newtheorem{assumption}[theorem]{Assumption}
\theoremstyle{definition}
\newtheorem{conjecture}[theorem]{Conjecture}

\allowdisplaybreaks %allows equations in the same align environment to split over multiple pages.

%tstk always need be there
\newcommand{\tstk}[1]{\textbf{#1} \newline}

%this adds extra functionality to pmatrix, vmatrix, bmatrix etc by allowing you to pass an optional argument in [FACTOR] to multiply the default spacing between elements by FACTOR
\makeatletter
\renewcommand*\env@matrix[1][\arraystretch]{%
  \edef\arraystretch{#1}%
  \hskip -\arraycolsep
  \let\@ifnextchar\new@ifnextchar
  \array{*\c@MaxMatrixCols c}
  }
\makeatother

%begin the macros via newcommand. Try to group them up reasonably!

%notation and variable use throughout the file
\renewcommand{\vec}[1]{\mathbf{#1}}				%vectors are bold, not overline arrow
\newcommand{\recip}[1]{\frac{1}{#1}}			%fast reciprocal as a fraction
\newcommand{\interval}[1]{\sqbracs{0,\abs{#1}}}	%creates the closed interval from 0 to the length of the input #1, denoted by absolute value
\newcommand{\eps}{\varepsilon}					%pretty epsilons
\newcommand{\charFunc}[1]{\mathcal{I}_{#1}}%{\mathbb{1}_{#1}}		%characteristic function of a set

\newcommand{\dddom}{\widetilde{\Omega}}			%3D domain notation
\newcommand{\ddom}{\Omega}						%2D domain notation
\newcommand{\dddmes}{\widetilde{\mu}}			%3D measure
\newcommand{\ddmes}{\mu}						%2D measure

\newcommand{\graph}{\mathbb{G}}					%graph variable
\newcommand{\vertSet}{\mathcal{V}}					%set of vertices rather than big V
\newcommand{\edgeSet}{\mathcal{E}}					%set of edges rather than large E, \graph = (V,E)
\newcommand{\wavenumber}{\kappa}				%fourier variable or wavenumber, not to confuse with jk subscripts!
\newcommand{\qm}{\theta}						%quasi-momentum parameter
\newcommand{\kt}{\bracs{\wavenumber, \qm}}		%(k, theta) pair
\newcommand{\dmap}{\Gamma_0}					%Dirichlet map
\newcommand{\nmap}{\Gamma_1}					%Neumann map
\newcommand{\effFreq}{\Lambda}					%Effective frequency sqrt(w^2-\wavenumber^2)
\newcommand{\conLeft}{\stackrel{\rightarrow}{\smash{\sim}\rule{0pt}{0.4ex}}} %j connects to k, j left
\newcommand{\conRight}{\stackrel{\leftarrow}{\smash{\sim}\rule{0pt}{0.4ex}}} %j connects to k, j right
\newcommand{\con}{\sim}							%j connects to k, indifferent of direction

%standard sets
\newcommand{\naturals}{\mathbb{N}}			%natural numbers
\newcommand{\integers}{\mathbb{Z}}			%integers
\newcommand{\rationals}{\mathbb{Q}}			%rational numbers
\newcommand{\reals}{\mathbb{R}}				%real numbers
\newcommand{\complex}{\mathbb{C}}			%complex numbers

%brackets and norms
\newcommand{\bracs}[1]{\left( #1 \right)}				%encloses input in brackets
\newcommand{\sqbracs}[1]{\left[ #1 \right]}				%encloses input in square brackets
\newcommand{\clbracs}[1]{\left\{ #1 \right\}}			%encloses input in curly bracers
\newcommand{\abs}[1]{\left\lvert #1 \right\rvert}					%absolute value
\newcommand{\norm}[1]{\lvert\lvert #1 \rvert\rvert}		%norm 

%function sets
\newcommand{\smooth}[1]{C^{\infty}\bracs{#1}}							%smooth functions
\newcommand{\ltwo}[2]{L^{2}\bracs{#1,\mathrm{d}#2}}						%general L^2 space
\newcommand{\gradSob}[2]{H^1_\mathrm{grad}\bracs{#1, \mathrm{d}#2}}		%gradient Sobolev space
\newcommand{\gradSobQM}[2]{H^1_{\qm, \mathrm{grad}}\bracs{#1, \mathrm{d}#2}} %gradient + i\qm Sobolev space
\newcommand{\ktgradSob}[2]{H^1_{\wavenumber,\qm,\mathrm{grad}}\bracs{#1, \mathrm{d}#2}}	%k,\qm-gradient Sobolev space
\newcommand{\curlSob}[2]{H^1_\mathrm{curl}\bracs{#1, \mathrm{d}#2}}		%curl Sobolev space
\newcommand{\tcurlSob}[2]{H^1_{\qm, \mathrm{curl}}\bracs{#1, \mathrm{d}#2}}		%curl + i\qm Sobolev space
\newcommand{\kcurlSob}[2]{H^1_{\wavenumber,\mathrm{curl}}\bracs{#1, \mathrm{d}#2}}	%k-curl Sobolev space
\newcommand{\ktcurlSob}[2]{H^1_{\wavenumber,\qm,\mathrm{curl}}\bracs{#1, \mathrm{d}#2}}	%k,\qm-curl Sobolev space
\newcommand{\ktcurlSobDivFree}[2]{\mathcal{H}^{\kt}\bracs{#1, \mathrm{d}#2}}					%k,\qm-curl, divergence-free Sobolev space
\newcommand{\supp}{\mathrm{supp}}										%support of a function

%grad and curl sets
\newcommand{\gradZero}[2]{\mathcal{G}_{ #1, \mathrm{d}#2}\bracs{0}}		%gradients of zero for domain #1 with measure #2
\newcommand{\kgradZero}[2]{\mathcal{G}_{ #1, \mathrm{d}#2}^{(\wavenumber)}\bracs{0}}	%k-gradients of zero for domain #1 with measure #2
\newcommand{\curlZero}[2]{\mathcal{C}_{ #1, \mathrm{d}#2}\bracs{0}}	%curls of zero for domain #1 with measure #2
\newcommand{\kcurlZero}[2]{\mathcal{C}_{ #1, \mathrm{d}#2}^{(\wavenumber)}\bracs{0}}	%k-curls of zero for domain #1 with measure #2

%derivatives and grad-like symbols
\newcommand{\diff}[2]{\dfrac{\mathrm{d}#1}{\mathrm{d}#2}}			%complete derivative d#1/d#2
\newcommand{\pdiff}[2]{\dfrac{\partial #1}{\partial #2}}			%partial derivative p#1/p#2
\newcommand{\ddiff}[2]{\dfrac{\mathrm{d}^2 #1}{\mathrm{d} {#2}^2}}	%2nd deriv
\newcommand{\grad}{\nabla}											%grad operator
\newcommand{\tgrad}{\nabla^{\qm}}									%grad operator with qm superscript
\newcommand{\kgrad}{\grad^{(\wavenumber)}}							%grad with wavenumber superscript
\newcommand{\ktgrad}{\grad^{\kt}}				%grad with wavenumber, qm superscript
\newcommand{\curl}[1]{\grad_{#1}\wedge}							%curl with subscript #1
\newcommand{\kcurl}[1]{\grad_{#1}^{(\wavenumber)}\wedge}			%k-curl with measure subscript #1
\newcommand{\ktcurl}[1]{\grad_{#1}^{\kt}\wedge}		%k,theta-curl with measure subscript #1
\newcommand{\laplacian}{\Delta}						%laplacian operator, can have subscripts attached

%displaying integrals
\newcommand{\integral}[3]{\int_{#1}#2 \ \mathrm{d}#3}			%integral, domain #1, integrand #2, measure #3
\newcommand{\md}{\mathrm{d}}									%differential d

%convergence
%\newcommand{\lconv}[1]{\xrightarrow{#1}}							%convergence with #1 above the rightarrow - requires mathtools
\newcommand{\lconv}[1]{\overset{#1}{\longrightarrow}}				%convergence with #1 above the rightarrow
\newcommand{\toInfty}[1]{ \ \text{as} \ #1 \rightarrow\infty}		%writes out "as #1 tends to infty" %maths commands, variables, and other packages

%labelling hacks
\newcommand\labelthis{\addtocounter{equation}{1}\tag{\theequation}}

\renewcommand{\curl}[1]{\mathrm{curl}\bracs{#1}}
\renewcommand{\ktcurl}[1]{\mathrm{curl}^{\kt}\bracs{#1}}
\newcommand{\ktdcurl}[2]{\mathrm{curl}^{\kt}_{#1}\bracs{#2}}
\newcommand{\dgmap}{\dmap^{\graph}}
\newcommand{\ngmap}{\nmap^{\graph}}
\newcommand{\sgn}{\mathrm{sgn}}
\newcommand{\ag}{\mathcal{A}_\graph}
\newcommand{\ip}[2]{\left\langle #1 , #2 \right\rangle}

%-------------------------------------------------------------------------
%DOCUMENT STARTS

\begin{document}

Throughout, let $\curl{u}$ denote the classical curl of a $C^1$ function, let $\ktcurl{u}$ denote the operation $\nabla^{\kt}\wedge u$ for $\nabla^{\kt} = \bracs{ \partial_1 + \rmi\qm_1, \partial_2 + \rmi\qm_2, \rmi\wavenumber}^\top$, and let $\ktdcurl{\rho}{u}$ denote the $\kt$-tangential curl of $u$ with respect to the measure $\rho$.

\section*{The Calderon Operator (2nd order, curl-of-the-curl)}

\subsection*{PDE problem}
Let $\dddom\subset\reals^3$ be a domain, and let $\mathcal{A}$ be the operator which acts as
\begin{align*}
	\mathcal{A}u = \curl{\curl{u}},
\end{align*}
on some appropriate domain of functions.
Then consider the problem
\begin{subequations} \label{eq:Maxwell3D}
\begin{align} 
	\mathcal{A}u - \beta u &= 0 \qquad\text{in } \dddom, \\
	\hat{n}\wedge\curl{u} &= m \qquad\text{on } \partial\dddom,
\end{align}
\end{subequations}
where $\beta>0$ and $m$ is a given function, and $\hat{n}$ is the exterior normal to the surface $\partial\dddom$.
The Calderon operator for (associated with?) the problem \eqref{eq:Maxwell3D} is then the operator $\mathcal{C}$ that acting on solutions $u$ to \eqref{eq:Maxwell3D}, sending
\begin{align*}
	u \rightarrow \hat{n}\wedge\curl{u},
\end{align*}
(so in general $\mathcal{C}$ depends on both $\beta$ and $m$?).

We can validate Green's identity for the triple $\bracs{\mathcal{A}, \dmap, \nmap}$ where
\begin{align*}
	\dmap u = u\vert_{\partial\dddom}, \qquad
	\nmap u = \hat{n}\wedge\curl{u}\vert_{\partial\dddom},
\end{align*}
since
\begin{align*}
	\integral{\dddom}{ \mathcal{A}u \cdot \overline{v} - u \cdot \overline{\mathcal{A}v} }{x}
	&= \integral{\dddom}{ \curl{\curl{u}}\cdot\overline{v} - u\cdot\overline{\curl{\curl{v}}} }{x} \\
	&= \integral{\dddom}{ \curl{u}\cdot\overline{\curl{v}} - \curl{u}\cdot\overline{\curl{v}} }{x} \\
	&\quad + \integral{\partial\dddom}{ \hat{n}\wedge\curl{u}\cdot\overline{v} - u\cdot\hat{n}\wedge\overline{\curl{v}} }{S} \\
	&= \integral{\partial\dddom}{ \nmap u \cdot \overline{\dmap v} - \dmap u \cdot \overline{\nmap v} }{S}.
\end{align*}

\subsection*{Graph Problem}
Now let's transition to our graph problem, on $\graph$ inside $\ddom\subset\reals^2$.
We expect that the boundary condition that we obtain from the problem
\begin{align} \label{eq:SingStrucCurlCurl}
	\ktdcurl{\dddmes}{\ktdcurl{\dddmes}{u}} &= \omega^2 u \qquad\text{in } \ddom,
\end{align}
by which we mean
\begin{align*}
	\integral{\ddom}{ \ktdcurl{\dddmes}{u}\cdot\overline{\ktdcurl{\dddmes}{\phi}} -\omega^2 u\cdot\overline{\phi} }{\dddmes},
	\quad\forall\phi\in\smooth{\ddom},
\end{align*}
to be of the form
\begin{align} \label{eq:DispersiveBC}
	\dgmap u &= -\omega^2 \alpha \ngmap u.
\end{align}
Here $\alpha$ is (akin to) the diagonal matrix of coupling constants and $\dgmap, \ngmap$ play the role of our Dirichlet and Neumann maps.
Note that in the scalar paper, our vertex conditions also appeared in the form \eqref{eq:DispersiveBC} due to our conventions on exterior normals and signs of the coupling constants $\alpha_j$.
Given the definitions for the PDE problem, we should expect that
\begin{align} \label{eq:DGMapDef}
	\dgmap u &= 
	\begin{pmatrix}
		u\bracs{v_1} \\ u\bracs{v_2} \\ \vdots \\ u\bracs{v_N}
	\end{pmatrix}
	\in\complex^{3N},
\end{align}
(where we have stacked the 3-vectors on top of each other), whilst $\ngmap u$ should bear resemblance to $\hat{n}\wedge\curl{u}\vert_{\partial\graph}$ --- the boundary ``$\partial\graph$" is the vertices of $\graph$.
For each edge $I_{jk}$, let $e_{jk}$ be the unit vector directed along $I_{jk}$ and $n_{jk}$ the unit normal (use the standard convention from the scalar and curl papers), and set $\widehat{e}_{jk} = \bracs{e_{jk}, 0}^\top\in\reals^3$, $\widehat{n}_{jk} = \bracs{n_{jk}, 0}^\top\in\reals^3$.
Define the function
\begin{align*}
	\sgn_{jk}: \clbracs{v_j, v_k} \rightarrow \clbracs{-1,0,1}, 
	&\qquad
	\sgn_{jk}(x) = \begin{cases} -1 & x=v_j, \\ 1 & x=v_k. \end{cases}
\end{align*}
and 
\begin{align*}
	\hat{\sigma}_{jk} &= \sgn_{jk}\widehat{e}_{jk},
\end{align*}
so $\hat{\sigma}_{jk}$ is playing the role of the ``exterior normal" for the edge $I_{jk}$.
The natural candidate for $\ngmap$ is then 
\begin{align} \label{eq:NGMapDef}
	\ngmap u &= 
	\begin{pmatrix}
		\sum_{1\con k} \hat{\sigma}_{1k}\wedge\ktdcurl{\dddmes}{u}\vert_{v_1} \\
		\sum_{2\con k} \hat{\sigma}_{2k}\wedge\ktdcurl{\dddmes}{u}\vert_{v_2} \\
		\vdots \\
		\sum_{N\con k} \hat{\sigma}_{Nk}\wedge\ktdcurl{\dddmes}{u}\vert_{v_N}
	\end{pmatrix}
	\in\complex^{3N}.
\end{align}
We know from our analysis that
\begin{align*}
	\ktdcurl{\dddmes}{u} &= \bracs{ \bracs{ u_3^{(jk)} }' + \rmi\qm_{jk}u_3^{(jk)} - \rmi\wavenumber U_2^{(jk)} }\widehat{n}_{jk},
\end{align*}
on each edge $I_{jk}$.
So we can deduce that
\begin{align*}
	\widehat{e}_{jk}\wedge\ktdcurl{\dddmes}{u} &= -
	\begin{pmatrix} 
	0 \\
	0 \\
	\bracs{ u_3^{(jk)} }' + \rmi\qm_{jk}u_3^{(jk)} - \rmi\wavenumber U_2^{(jk)}
	\end{pmatrix},
\end{align*}
on $I_{jk}$, and hence that (for a fixed $v_j\in\vertSet$)
\begin{align*}
	\sum_{j\con k} \hat{\sigma}_{jk} \ \wedge \ &\ktdcurl{\dddmes}{u}\vert_{v_j} = \\ 
	&\begin{pmatrix}
	0 \\
	0 \\	
	- \sum_{j\con k}\bracs{\pdiff{}{n} + \rmi\qm_{jk}}u_3^{(jk)}\bracs{v_j}
	+ \rmi\wavenumber\bracs{ \sum_{j\conRight k} U_2^{(kj)}\bracs{v_j} - \sum_{j\conLeft k} U_2^{(jk)}\bracs{v_j} }
	\end{pmatrix}.
\end{align*}

In the curl paper, we derive the vertex conditions
\begin{align} \label{eq:VertConditionExplicit}
	\alpha_j\omega^2 u\bracs{v_j} &=
	\begin{pmatrix}
	0 \\
	0 \\	
	\bracs{\pdiff{}{n} + \rmi\qm_{jk}}u_3^{(jk)}\bracs{v_j}
	- \rmi\wavenumber\bracs{ \sum_{j\conRight k} U_2^{(kj)}\bracs{v_j} - \sum_{j\conLeft k} U_2^{(jk)}\bracs{v_j} }
	\end{pmatrix},
\end{align}
at each $v_j\in\vertSet$ (now expressed in vector form).
We can identify \eqref{eq:VertConditionExplicit} as being of the form \eqref{eq:DispersiveBC} where
\begin{align*}
	\alpha = 
	\mathrm{diag}\bracs{\alpha_1, \alpha_1, \alpha_1, \alpha_2, \alpha_2, \alpha_2, ..., \alpha_N, \alpha_N, \alpha_N} \in \complex^{3N\times 3N},
\end{align*}
and $\dgmap, \ngmap$ are as in \eqref{eq:DGMapDef}, \eqref{eq:NGMapDef}.
This implies that we have already found the ``correct" (or at least, expected) analogue of the Calderon operator for the problem \eqref{eq:SingStrucCurlCurl}, and the following section demonstrates that we indeed have a suitable boundary triple for the resulting quantum graph problem.

\subsection*{Current Curl-Curl Quantum Graph Problem}
Let $\graph$ be (an embedded) metric graph, and recall that for a function $u=\bracs{u_1,u_2,u_3}^\top$ and an edge $I_{jk}$ we define
\begin{align*}
	U = \bracs{u_1, u_2}^\top, \quad U_1^{(jk)} = U^{(jk)}\cdot n_{jk}, \quad U_2^{(jk)} = U^{(jk)}\cdot e_{jk}.
\end{align*}
Now define the operator $\ag$ via the action
\begin{align*}
	\ag u &= 
	\begin{pmatrix}
		\sqbracs{ \rmi\wavenumber\bracs{\diff{}{y} + \rmi\qm_{jk} }u_3^{(jk)} + \wavenumber^2 U_2^{(jk)} }e_{jk}
		+ U_1^{(jk)} n_{jk} \\
		- \bracs{\diff{}{y} + \rmi\qm_{jk} }^2 u_3^{(jk)} + \rmi\wavenumber \bracs{\diff{}{y} + \rmi\qm_{jk} }U_2^{(jk)}
	\end{pmatrix}
\end{align*}
on each edge, where $\mathrm{dom}\bracs{\ag}$ consists of all functions $u$ with the following properties:
\begin{align*}
	u\in\mathrm{dom}\bracs{\ag} \quad\Leftrightarrow\quad &
	\begin{cases}
	u\in L^2\bracs{\graph}\times L^2\bracs{\graph}\times H^2\bracs{\graph}, \\
	\begin{pmatrix} u_1 \\ u_2 \end{pmatrix}\cdot e_{jk}\in \gradSob{I_{jk}}{y}, & \forall I_{jk}\in\edgeSet, \\
	u \text{ is continuous at } v_j, & \forall v_j\in\vertSet, \\
	\text{\eqref{eq:VertConditionExplicit} is satisfied at } v_j, & \forall v_j\in\vertSet.
	\end{cases}
\end{align*}
Let $\dgmap, \ngmap$ be as defined in \eqref{eq:DGMapDef}, \eqref{eq:NGMapDef}.
Then we have that
\begin{align*}
	\integral{I_{jk}}{ \ag u \cdot \overline{v} }{y} - \integral{I_{jk}}{ u \cdot \overline{\ag v} }{y}
	&= \sqbracs{ -u'_3 v_3 + u_3 v_3' - 2\rmi\qm_{jk}u_3 v_3 + \rmi\wavenumber\bracs{U_2 v_3 + u_3 V_2} }_{v_j}^{v_k} \\
	&= -\sqbracs{ \overline{v}_3\bracs{ \bracs{\diff{}{y} + \rmi\qm_{jk} }u_3 - \rmi\wavenumber U_2 } }_{v_j}^{v_k} \\
	&\qquad + \sqbracs{ u_3\overline{\bracs{ \bracs{\diff{}{y} + \rmi\qm_{jk} }v_3 - \rmi\wavenumber V_2 }} }_{v_j}^{v_k}.
\end{align*}
Ergo,
\begin{align*}
	&\ip{\ag u}{v}_{L^2\bracs{\graph}^3} - \ip{u}{\ag v}_{L^2\bracs{\graph}^3}
	= \sum_{v_j\in\vertSet}\sum_{j\conLeft k} \integral{I_{jk}}{ \ag u \cdot \overline{v} - u \cdot \overline{\ag v} }{y} \\
	&\quad = \sum_{v_j\in\vertSet}\sum_{j\conLeft k} -\sqbracs{ \overline{v}_3\bracs{ \bracs{\diff{}{y} + \rmi\qm_{jk} }u_3 - \rmi\wavenumber U_2 } }_{v_j}^{v_k}
	+ \sqbracs{ u_3\overline{\bracs{ \bracs{\diff{}{y} + \rmi\qm_{jk} }v_3 - \rmi\wavenumber V_2 }} }_{v_j}^{v_k} \\
	&\quad = \sum_{v_j\in\vertSet} u_3\bracs{v_j}\overline{\bracs{ \sum_{j\con k}\bracs{\pdiff{}{n} + \rmi\qm_{jk}}v_3 - \rmi\wavenumber\bracs{ \sum_{j\conRight k} V_2^{(kj)}\bracs{v_j} - \sum_{j\conLeft k} V_2^{(jk)}\bracs{v_j} } }} \\
	&\quad + \sum_{v_j\in\vertSet} \overline{v}_3\bracs{v_j}\bracs{ \sum_{j\con k}\bracs{\pdiff{}{n} + \rmi\qm_{jk}}u_3 - \rmi\wavenumber\bracs{ \sum_{j\conRight k} U_2^{(kj)}\bracs{v_j} - \sum_{j\conLeft k} U_2^{(jk)}\bracs{v_j} } } \\
	&\quad = \ngmap u \cdot \overline{\dgmap v} - \dgmap u \cdot \overline{\ngmap v}
	= \ip{\ngmap u}{\dgmap v}_{\complex^{3N}} - \ip{\dgmap u}{\ngmap v}_{\complex^{3N}},
\end{align*}
and so Green's identity holds.
As such, $\bracs{\complex^{3N}, \dgmap, \ngmap}$ is a boundary triple for the operator $\ag$.

\end{document}