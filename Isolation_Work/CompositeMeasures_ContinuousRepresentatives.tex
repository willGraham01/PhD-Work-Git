\documentclass[11pt]{report}

\usepackage{url}
\usepackage[margin=2.5cm]{geometry} % See geometry.pdf to learn the layout options. There are lots.
\geometry{a4paper} %or letterpaper or a5paper or ...

%for figures and graphics
\usepackage{graphicx}
\usepackage{subcaption} %allows subfigures
\usepackage[bottom]{footmisc} %footnotes go below figures
%\usepackage{parskip} %adds line space between paragraphs by default
\usepackage{enumerate} %allows lower case roman numerials in enumerate environments

\DeclareGraphicsRule{.tif}{png}{.png}{`convert #1 `dirname #1`/`basename #1 .tif`.png}
\graphicspath{{../Diagrams/Diagram_PDFs/} {../Diagrams/Numerical_Results/}}

%\input imports all commands from the target files
%The idea behind this file is that it will be used to store all the maths-related macros that I concoct; so that I can import all the commands by \input{this file} in the preamble of any file that I want to use them in.
%This should make the top-level files look a lot cleaner, and the preamble much shorter!

\usepackage{amssymb}
\usepackage{amsmath}
%\usepackage{mathtools}

%theorems and lemma etc setup using amsthm
\usepackage{amsthm}
\theoremstyle{definition}
\newtheorem{definition}{Definition}[section]
\theoremstyle{plain}
\newtheorem{theorem}{Theorem}[section]
\theoremstyle{plain}
\newtheorem{lemma}[theorem]{Lemma}
\theoremstyle{plain}
\newtheorem{prop}[theorem]{Proposition}
\theoremstyle{plain}
\newtheorem{cory}[theorem]{Corollary}
\theoremstyle{definition}
\newtheorem{convention}[theorem]{Convention}
\theoremstyle{definition}
\newtheorem{assumption}[theorem]{Assumption}
\theoremstyle{definition}
\newtheorem{conjecture}[theorem]{Conjecture}

\allowdisplaybreaks %allows equations in the same align environment to split over multiple pages.

%tstk always need be there
\newcommand{\tstk}[1]{\textbf{#1} \newline}

%this adds extra functionality to pmatrix, vmatrix, bmatrix etc by allowing you to pass an optional argument in [FACTOR] to multiply the default spacing between elements by FACTOR
\makeatletter
\renewcommand*\env@matrix[1][\arraystretch]{%
  \edef\arraystretch{#1}%
  \hskip -\arraycolsep
  \let\@ifnextchar\new@ifnextchar
  \array{*\c@MaxMatrixCols c}
  }
\makeatother

%begin the macros via newcommand. Try to group them up reasonably!

%notation and variable use throughout the file
\renewcommand{\vec}[1]{\mathbf{#1}}				%vectors are bold, not overline arrow
\newcommand{\recip}[1]{\frac{1}{#1}}			%fast reciprocal as a fraction
\newcommand{\interval}[1]{\sqbracs{0,\abs{#1}}}	%creates the closed interval from 0 to the length of the input #1, denoted by absolute value
\newcommand{\eps}{\varepsilon}					%pretty epsilons
\newcommand{\charFunc}[1]{\mathcal{I}_{#1}}%{\mathbb{1}_{#1}}		%characteristic function of a set

\newcommand{\dddom}{\widetilde{\Omega}}			%3D domain notation
\newcommand{\ddom}{\Omega}						%2D domain notation
\newcommand{\dddmes}{\widetilde{\mu}}			%3D measure
\newcommand{\ddmes}{\mu}						%2D measure

\newcommand{\graph}{\mathbb{G}}					%graph variable
\newcommand{\vertSet}{\mathcal{V}}					%set of vertices rather than big V
\newcommand{\edgeSet}{\mathcal{E}}					%set of edges rather than large E, \graph = (V,E)
\newcommand{\wavenumber}{\kappa}				%fourier variable or wavenumber, not to confuse with jk subscripts!
\newcommand{\qm}{\theta}						%quasi-momentum parameter
\newcommand{\kt}{\bracs{\wavenumber, \qm}}		%(k, theta) pair
\newcommand{\dmap}{\Gamma_0}					%Dirichlet map
\newcommand{\nmap}{\Gamma_1}					%Neumann map
\newcommand{\effFreq}{\Lambda}					%Effective frequency sqrt(w^2-\wavenumber^2)
\newcommand{\conLeft}{\stackrel{\rightarrow}{\smash{\sim}\rule{0pt}{0.4ex}}} %j connects to k, j left
\newcommand{\conRight}{\stackrel{\leftarrow}{\smash{\sim}\rule{0pt}{0.4ex}}} %j connects to k, j right
\newcommand{\con}{\sim}							%j connects to k, indifferent of direction

%standard sets
\newcommand{\naturals}{\mathbb{N}}			%natural numbers
\newcommand{\integers}{\mathbb{Z}}			%integers
\newcommand{\rationals}{\mathbb{Q}}			%rational numbers
\newcommand{\reals}{\mathbb{R}}				%real numbers
\newcommand{\complex}{\mathbb{C}}			%complex numbers

%brackets and norms
\newcommand{\bracs}[1]{\left( #1 \right)}				%encloses input in brackets
\newcommand{\sqbracs}[1]{\left[ #1 \right]}				%encloses input in square brackets
\newcommand{\clbracs}[1]{\left\{ #1 \right\}}			%encloses input in curly bracers
\newcommand{\abs}[1]{\left\lvert #1 \right\rvert}					%absolute value
\newcommand{\norm}[1]{\lvert\lvert #1 \rvert\rvert}		%norm 

%function sets
\newcommand{\smooth}[1]{C^{\infty}\bracs{#1}}							%smooth functions
\newcommand{\ltwo}[2]{L^{2}\bracs{#1,\mathrm{d}#2}}						%general L^2 space
\newcommand{\gradSob}[2]{H^1_\mathrm{grad}\bracs{#1, \mathrm{d}#2}}		%gradient Sobolev space
\newcommand{\gradSobQM}[2]{H^1_{\qm, \mathrm{grad}}\bracs{#1, \mathrm{d}#2}} %gradient + i\qm Sobolev space
\newcommand{\ktgradSob}[2]{H^1_{\wavenumber,\qm,\mathrm{grad}}\bracs{#1, \mathrm{d}#2}}	%k,\qm-gradient Sobolev space
\newcommand{\curlSob}[2]{H^1_\mathrm{curl}\bracs{#1, \mathrm{d}#2}}		%curl Sobolev space
\newcommand{\tcurlSob}[2]{H^1_{\qm, \mathrm{curl}}\bracs{#1, \mathrm{d}#2}}		%curl + i\qm Sobolev space
\newcommand{\kcurlSob}[2]{H^1_{\wavenumber,\mathrm{curl}}\bracs{#1, \mathrm{d}#2}}	%k-curl Sobolev space
\newcommand{\ktcurlSob}[2]{H^1_{\wavenumber,\qm,\mathrm{curl}}\bracs{#1, \mathrm{d}#2}}	%k,\qm-curl Sobolev space
\newcommand{\ktcurlSobDivFree}[2]{\mathcal{H}^{\kt}\bracs{#1, \mathrm{d}#2}}					%k,\qm-curl, divergence-free Sobolev space
\newcommand{\supp}{\mathrm{supp}}										%support of a function

%grad and curl sets
\newcommand{\gradZero}[2]{\mathcal{G}_{ #1, \mathrm{d}#2}\bracs{0}}		%gradients of zero for domain #1 with measure #2
\newcommand{\kgradZero}[2]{\mathcal{G}_{ #1, \mathrm{d}#2}^{(\wavenumber)}\bracs{0}}	%k-gradients of zero for domain #1 with measure #2
\newcommand{\curlZero}[2]{\mathcal{C}_{ #1, \mathrm{d}#2}\bracs{0}}	%curls of zero for domain #1 with measure #2
\newcommand{\kcurlZero}[2]{\mathcal{C}_{ #1, \mathrm{d}#2}^{(\wavenumber)}\bracs{0}}	%k-curls of zero for domain #1 with measure #2

%derivatives and grad-like symbols
\newcommand{\diff}[2]{\dfrac{\mathrm{d}#1}{\mathrm{d}#2}}			%complete derivative d#1/d#2
\newcommand{\pdiff}[2]{\dfrac{\partial #1}{\partial #2}}			%partial derivative p#1/p#2
\newcommand{\ddiff}[2]{\dfrac{\mathrm{d}^2 #1}{\mathrm{d} {#2}^2}}	%2nd deriv
\newcommand{\grad}{\nabla}											%grad operator
\newcommand{\tgrad}{\nabla^{\qm}}									%grad operator with qm superscript
\newcommand{\kgrad}{\grad^{(\wavenumber)}}							%grad with wavenumber superscript
\newcommand{\ktgrad}{\grad^{\kt}}				%grad with wavenumber, qm superscript
\newcommand{\curl}[1]{\grad_{#1}\wedge}							%curl with subscript #1
\newcommand{\kcurl}[1]{\grad_{#1}^{(\wavenumber)}\wedge}			%k-curl with measure subscript #1
\newcommand{\ktcurl}[1]{\grad_{#1}^{\kt}\wedge}		%k,theta-curl with measure subscript #1
\newcommand{\laplacian}{\Delta}						%laplacian operator, can have subscripts attached

%displaying integrals
\newcommand{\integral}[3]{\int_{#1}#2 \ \mathrm{d}#3}			%integral, domain #1, integrand #2, measure #3
\newcommand{\md}{\mathrm{d}}									%differential d

%convergence
%\newcommand{\lconv}[1]{\xrightarrow{#1}}							%convergence with #1 above the rightarrow - requires mathtools
\newcommand{\lconv}[1]{\overset{#1}{\longrightarrow}}				%convergence with #1 above the rightarrow
\newcommand{\toInfty}[1]{ \ \text{as} \ #1 \rightarrow\infty}		%writes out "as #1 tends to infty" %maths commands, variables, and other packages

%labelling hacks
\newcommand\labelthis{\addtocounter{equation}{1}\tag{\theequation}}

\DeclareMathOperator*{\dw}{width}
\newcommand{\dirWidth}[2]{\dw_{#2}\bracs{#1}}
\newcommand{\tlambda}{\tilde{\lambda}}

%-------------------------------------------------------------------------
%DOCUMENT STARTS

\begin{document}

Let $\ddom\subset\reals^2$ be our domain and let $\graph=\bracs{V,E}$ be an embedded graph in $\ddom$.
For each edge $I_{jk}=\bracs{v_j,v_k}\in E$ with $v_j,v_k\in V$ we write $\lambda_{jk}$ for the singular measure supported on the edge $I_{jk}$, and we use $\ddmes$ to denote the singular measure supported on the graph $\graph$, where
\begin{align*}
	\ddmes\bracs{B} &= \sum_{I_{jk}\in E}\lambda_{jk}\bracs{B}.
\end{align*}
The edge $I_{jk}$ is directed from $v_j$ to $v_k$, and we denote by $e_{jk}$ the unit vector parallel to $I_{jk}$ in this direction, and $n_{jk}$ the unit normal to $I_{jk}$.

\begin{convention}[Notation for Composite Measures] \label{conv:CompMeasNotation}
	If $\ddmes$ is a singular measure supporting a graph $\graph$ (including a single-edge graph, or segment), embedded in $\ddom\subset\reals^2$, use an overhead tilde to denote the composite measure of $\graph$ in $\ddom$, so
	\begin{align*}
		\dddmes(B) &= \lambda_2(B) + \ddmes(B).
	\end{align*}
\end{convention}

\begin{theorem}[Characterisation of $\gradZero{\ddom}{\dddmes}$] \label{thm:CharGradZeroGraphCompositeMeasure}
	Let $G=\bracs{V,E}$ be an embedded graph in $\ddom\subset\reals^2$, with singular measure $\ddmes$.
	Then the following are all equivalent:
	\begin{enumerate}[(i)]
		\item $g\in\gradZero{\ddom}{\dddmes}$,
		\item $g\in\gradZero{\ddom}{\lambda_2}$, and $g\in\gradZero{\ddom}{\lambda_{jk}} \ \forall I_{jk}\in E$,
		\item $g$ is such that
		\begin{align*}
			g\cdot e_{jk} = 0 &\quad \forall I_{jk}\in E, \\
			\bracs{g\cdot n_{jk}}\vert_{I_{jk}} \in \ltwo{\ddom}{\lambda_{jk}} &\quad \forall I_{jk}\in E, \\
			g\cdot n_{jk} = 0 \ \lambda_2\text{-a.e.} &\quad \forall I_{jk}\in E.
		\end{align*}
	\end{enumerate}
\end{theorem}
\begin{proof}
	There is a proof of this in the document CompositeMeasures.tex.
\end{proof}
Note that theorem \ref{thm:CharGradZeroGraphCompositeMeasure} essentially says that any gradient of zero is equal to the zero function away from the graph $\graph$, and on $\graph$ adheres to the same structure as the set $\gradZero{\ddom}{\ddmes}$.

\begin{prop}[Membership of Sobolev spaces with smaller domains] \label{prop:MembershipSmallerDomains}
	Let $D\subset\ddom$ be such that $D\cap\graph = \emptyset$.
	Then if $u\in\gradSob{\ddom}{\dddmes}$ with gradient $\grad_{\dddmes}u$, we have that
	\begin{align*}
		u\in \gradSob{D}{\lambda_2} = \gradSob{D}{\dddmes},
	\end{align*}
	and the gradient of $u$ (in the $\gradSob{D}{\lambda_2}$-sense) is simply the restriction of $\grad_{\dddmes}u$ to the set $D$.
\end{prop}
\begin{proof}
	Since $u\in\gradSob{\ddom}{\dddmes}$ we have that there exist smooth functions $\phi_n$ such that
	\begin{align*}
		\phi_n \lconv{\ltwo{\ddom}{\dddmes}} u, &\quad \grad\phi_n \lconv{\ltwo{\ddom}{\dddmes}^2} \grad_{\dddmes}u. \\
		\implies \integral{\ddom}{\abs{\phi_n-u}^2}{\dddmes} \rightarrow 0, 
		&\quad \integral{\ddom}{\abs{\grad\phi_n - \grad_{\dddmes}u}^2}{\dddmes} \rightarrow 0 \toInfty{n}.
	\end{align*}
	But then clearly
	\begin{align*}
		\integral{D}{\abs{\phi_n - u}^2}{\lambda_2} 
		&\leq \integral{\ddom}{\abs{\phi_n-u}^2}{\dddmes} \rightarrow 0, \\
		\integral{D}{\abs{\grad\phi_n - \grad_{\dddmes}u}^2}{\lambda_2} 
		&\leq \integral{\ddom}{\abs{\grad\phi_n - \grad_{\dddmes}u}^2}{\dddmes} \rightarrow 0.
	\end{align*}
	Thus,
	\begin{align*}
		\phi_n \lconv{\ltwo{D}{\lambda_2}} u, &\quad \grad\phi_n \lconv{\ltwo{D}{\lambda_2}^2} \grad_{\dddmes}u,
	\end{align*}
	and so we have that $u\in\gradSob{D}{\lambda_2}$.
\end{proof}

\begin{convention}[] \label{conv:ConsistencyOfConstructionH1}
Our construction of $\gradSob{D}{\lambda_2}$ uses approximating sequences to define (classes of) functions and their gradients, which is akin to the ``$W^{1,2}$" construction of classical Sobolev spaces.
When the measure is Lebesgue, what we denote by $\gradSob{D}{\lambda_2}$ coincides with $W^{1,2}\bracs{D}$.
This is because of the ``$H^1 = W^{1,2}$" theorem from Sobolev space theory, showing that we have $\gradSob{D}{\lambda_2}$ coinciding with the classical $H^1\bracs{D}$, where derivatives are defined as those functions which satisfy an ``integration-by-parts" style result when integrated against each member of a suitable class of test functions. \newline

However we can also show that if $\bracs{u,\grad_{\dddmes} u}\in\gradSob{\ddom}{\ddmes}$, then we also have $\bracs{u,\grad_{\dddmes} u}\in\gradSob{\ddom}{\lambda_2}$.
However given the result of theorem \ref{thm:CharGradZeroGraphCompositeMeasure}, we know that $\grad_{\dddmes} u$ is \emph{almost everywhere} equal to $\grad_{\lambda_2}u = \grad u$ with respect to the $\lambda_2$-measure, and thus $\grad_{\dddmes} u$ is a representative of $\grad u$.
\end{convention}

\chapter*{Continuity Result}
Suppose that the graph $\graph$ consists of one edge $I=I_{jk}$.
In this section we replace the usual subscript $jk$ notation on $e_{jk}, n_{jk}$ and $\lambda_{jk}$ with a subscript $I$, and note that $\ddmes = \lambda_I$.
Suppose further (for ease) that this edge lies parallel to the $x_1$-axis (although rotating the edge has minimal impact on the following result and proof).
We aim to prove that a function $u\in\gradSob{\ddom}{\tlambda_I}$ ``is continuous across the edge $I$", by which we mean to show that the function $u$ has a representative that is continuous when approaching $I$ in the directions $n_I$ and $-n_I$.

\begin{theorem}
	Let $u\in\gradSob{\ddom}{\tlambda_I}$.
	Then $u$ has a representative that is continuous across the edge $I$ in the direction $\pm n_I$.
\end{theorem}
\begin{proof}
	Take an approximating sequence
	\begin{align*}
		\phi_n \lconv{\ltwo{\ddom}{\tlambda_I}} u, &\quad \grad\phi_n \lconv{\ltwo{\ddom}{\tlambda_I}^2} \grad_{\tlambda_I}u.
	\end{align*}
	Let $\psi$ be a smooth function with support $D\subset\ddom$, where $D$ is connected and $D\setminus I$ is disconnected, but has precisely two connected components (essentially $I$ splits the support of $D$ in two).
	As $I$ is assumed parallel to the $x_1$ axis, denote the parts of $D$ that lie ``above" and ``below" $I$ by
	\begin{align*}
		D^+ &:= \clbracs{x=\bracs{x_1,x_2}\in D \ \vert \ x_2>y_2 \ \forall y=\bracs{y_1,y_2}\in I }, \\
		D^- &:= \clbracs{x=\bracs{x_1,x_2}\in D \ \vert \ x_2<y_2 \ \forall y=\bracs{y_1,y_2}\in I }.
	\end{align*}
	This is illustrated in figure \ref{fig:Diagram_CompositeMeasureEdgeContinuity} for clarity.
	\begin{figure}[h]
		\centering
		\includegraphics[scale=1.0]{Diagram_CompositeMeasureEdgeContinuity.pdf}
		\caption{\label{fig:Diagram_CompositeMeasureEdgeContinuity}}
	\end{figure}
	By proposition \ref{prop:MembershipSmallerDomains}, we know that $u\in\gradSob{D^+}{\lambda_2}$ and $u\in\gradSob{D^-}{\lambda_2}$.
	As these are Sobolev spaces with respect to the Lebesgue measure (``classical Sobolev spaces") we know that
	\begin{align*}
		\partial_j\phi_n \rightarrow \partial_j u &\quad\text{in } \ltwo{D^+}{\lambda_2} \text{ and } \ltwo{D^-}{\lambda_2},
	\end{align*}
	for each $j\in\clbracs{1,2}$.
	Let us consider
	\begin{align*}
		\integral{\ddom}{\psi\partial_j\phi_n}{\lambda_2}
		&= \integral{D^+}{\psi\partial_j\phi_n}{\lambda_2} + \integral{D^-}{\psi\partial_j\phi_n}{\lambda_2} \\
		&= -\integral{D^+}{\phi_n\partial_j\psi}{\lambda_2} -\integral{D^-}{\phi_n\partial_j\psi}{\lambda_2} + \integral{\partial D^+}{\phi_n\psi n^+}{s} + \integral{\partial D^-}{\phi_n\psi n^-}{s},
	\end{align*}
	where $n_j^+$ and $n_j^-$ are the $j^{\text{th}}$ components of the outward-facing normals to the boundaries of $D^+$ and $D^-$, respectively.
	We can perform this integration by parts due to $u$'s membership of $\gradSob{D^+}{\lambda_2}$ and $\gradSob{D^-}{\lambda_2}$, and convention \ref{conv:ConsistencyOfConstructionH1}.
	We can also note that the support of $\psi$ causes the integrand of the boundary integrals to be zero except on the portion of the boundary that intersects the segment $I$, and thus
	\begin{align*}
		\integral{\ddom}{\psi\partial_j\phi_n}{\lambda_2}
		&= -\integral{D^+}{\phi_n\partial_j\psi}{\lambda_2} -\integral{D^-}{\phi_n\partial_j\psi}{\lambda_2} + \integral{D \cap I}{\phi_n\psi n^+}{s} + \integral{D \cap I}{\phi_n\psi n^-}{s}.
	\end{align*}
	If we now pass to the limit $n\rightarrow\infty$ in the above equation, and use the compactness of the embeddings
	\begin{align*}
		\gradSob{D^+}{\lambda_2}\hookrightarrow\ltwo{\partial D^+}{s}, &\quad \gradSob{D^-}{\lambda_2}\hookrightarrow\ltwo{\partial D^-}{s},
	\end{align*}
	we obtain
	\begin{align*}
		\integral{\ddom}{\psi\partial_j u}{\lambda_2}
		&= -\integral{D^+}{u\partial_j\psi}{\lambda_2} -\integral{D^-}{u\partial_j\psi}{\lambda_2} + \integral{D \cap I}{u^{+}\psi n^+}{s} + \integral{D \cap I}{u^{-}\psi n^-}{s},
	\end{align*}
	where $u^+$ (respectively $u^-$) is the trace of $u\vert_{D^+}$ (respectively $u\vert_{D^-}$) onto it's boundary.
	It will be more useful to us to write
	\begin{align*}
		\integral{\ddom}{\psi\partial_j u}{\lambda_2}
		&= \integral{D^+}{\psi\partial_j u}{\lambda_2} + \integral{D^-}{\psi\partial_j u}{\lambda_2},
	\end{align*}
	so that we have
	\begin{align*}
		\integral{D^+}{\psi\partial_j u}{\lambda_2} + \integral{D^-}{\psi\partial_j u}{\lambda_2}
		&= -\integral{D^+}{u\partial_j\psi}{\lambda_2} -\integral{D^-}{u\partial_j\psi}{\lambda_2} \\
		&\quad + \integral{D \cap I}{u^{+}\psi n^+}{s} + \integral{D \cap I}{u^{-}\psi n^-}{s} \\
		&= -\integral{D^+}{u\partial_j\psi}{\lambda_2} -\integral{D^-}{u\partial_j\psi}{\lambda_2} \\
		&\quad + \integral{D \cap I}{\psi n^+\bracs{u^{+} - u^{-}}}{s}
	\end{align*}
	Here we would like to invoke the fact that $u$ is an element of $\gradSob{D^+}{\lambda_2}$ and $\gradSob{D^-}{\lambda_2}$, then use convention \ref{conv:ConsistencyOfConstructionH1} to deduce that
	\begin{align*}
		\integral{D^+}{\psi\partial_j u}{\lambda_2} &= -\integral{D^+}{u\partial_j\psi}{\lambda_2}, \\
		\integral{D^+}{\psi\partial_j u}{\lambda_2} &= -\integral{D^+}{u\partial_j\psi}{\lambda_2}
	\end{align*}
	however the function $\psi$ is of the wrong class (it has support $D$ which includes the a portion of the segment $I$, which in turn is part of the boundary of $D^+$ and $D^-$).
	\tstk{Can we find functions that do this and still ensure the validity of our argument in the limit?}
	To this end, we would be required to find a suitable sequence of (smooth) functions $\psi_l$ that have support (contained in) $D^+ \cup D^-$ such that $\psi_l\rightarrow\psi$.
	Then we would repeat the above steps, but beginning from
	\begin{align*}
		\integral{\ddom}{\psi_l\partial_j\phi_n}{\lambda_2}
	\end{align*}
	and reaching the conclusion that
	\begin{align} \label{eq:TempLabel1}
		\integral{D^+}{\psi_l\partial_j u}{\lambda_2} + \integral{D^-}{\psi_l\partial_j u}{\lambda_2}
		&= -\integral{D^+}{u\partial_j\psi_l}{\lambda_2} -\integral{D^-}{u\partial_j\psi_l}{\lambda_2} \\
		&\quad + \integral{D \cap I}{\psi_l n^+\bracs{u^{+} - u^{-}}}{s}.
	\end{align}
	Then
	\begin{align*}
		\integral{D^+}{\psi_l\partial_j u}{\lambda_2} &= -\integral{D^+}{u\partial_j\psi_l}{\lambda_2}, \\
		\integral{D^+}{\psi_l\partial_j u}{\lambda_2} &= -\integral{D^+}{u\partial_j\psi_l}{\lambda_2},
	\end{align*}
	would hold for each $l\in\naturals$, and thus passing to the limit $l\rightarrow\infty$ would let us conclude from \eqref{eq:TempLabel1} that
	\begin{align*}
		0 = \integral{D \cap I}{\psi n^+\bracs{u^{+} - u^{-}}}{s}.
	\end{align*}
	This would then hold for each smooth $\psi$ such that $\supp\bracs{\psi}\subset\ddom$, and thus we could conclude by (a version of) the fundamental lemma of calculus of variations that $u^{+} - u^{-}=0$, and thus $u^+ = u^-$.
	Hence, $u$ would have a representative that is continuous across the edge $I$ when approaching in the directions $\pm n_I$.
\end{proof}

Presenting this proof in this way means that this method also suffices as a proof when the segment $I$ is rotated (it just means that defining $D^+$ and $D^-$ involves a bit more notation).
As for when $\graph$ consists of multiple edges, I guess we can just repeat the above argument for each edge (as we can always ensure that the support of our function $\psi$ only intersects precisely one edge for finite graphs).


\end{document}