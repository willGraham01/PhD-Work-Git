\documentclass[11pt]{report}

\usepackage{url}
\usepackage[margin=2.5cm]{geometry} % See geometry.pdf to learn the layout options. There are lots.
\geometry{a4paper} %or letterpaper or a5paper or ...

%for figures and graphics
\usepackage{graphicx}
\usepackage{subcaption} %allows subfigures
\usepackage[bottom]{footmisc} %footnotes go below figures
%\usepackage{parskip} %adds line space between paragraphs by default
\usepackage{enumerate} %allows lower case roman numerials in enumerate environments

\DeclareGraphicsRule{.tif}{png}{.png}{`convert #1 `dirname #1`/`basename #1 .tif`.png}
\graphicspath{{../Diagrams/Diagram_PDFs/} {../Diagrams/Numerical_Results/}}

%\input imports all commands from the target files
%The idea behind this file is that it will be used to store all the maths-related macros that I concoct; so that I can import all the commands by \input{this file} in the preamble of any file that I want to use them in.
%This should make the top-level files look a lot cleaner, and the preamble much shorter!

\usepackage{amssymb}
\usepackage{amsmath}
%\usepackage{mathtools}

%theorems and lemma etc setup using amsthm
\usepackage{amsthm}
\theoremstyle{definition}
\newtheorem{definition}{Definition}[section]
\theoremstyle{plain}
\newtheorem{theorem}{Theorem}[section]
\theoremstyle{plain}
\newtheorem{lemma}[theorem]{Lemma}
\theoremstyle{plain}
\newtheorem{prop}[theorem]{Proposition}
\theoremstyle{plain}
\newtheorem{cory}[theorem]{Corollary}
\theoremstyle{definition}
\newtheorem{convention}[theorem]{Convention}
\theoremstyle{definition}
\newtheorem{assumption}[theorem]{Assumption}
\theoremstyle{definition}
\newtheorem{conjecture}[theorem]{Conjecture}

\allowdisplaybreaks %allows equations in the same align environment to split over multiple pages.

%tstk always need be there
\newcommand{\tstk}[1]{\textbf{#1} \newline}

%this adds extra functionality to pmatrix, vmatrix, bmatrix etc by allowing you to pass an optional argument in [FACTOR] to multiply the default spacing between elements by FACTOR
\makeatletter
\renewcommand*\env@matrix[1][\arraystretch]{%
  \edef\arraystretch{#1}%
  \hskip -\arraycolsep
  \let\@ifnextchar\new@ifnextchar
  \array{*\c@MaxMatrixCols c}
  }
\makeatother

%begin the macros via newcommand. Try to group them up reasonably!

%notation and variable use throughout the file
\renewcommand{\vec}[1]{\mathbf{#1}}				%vectors are bold, not overline arrow
\newcommand{\recip}[1]{\frac{1}{#1}}			%fast reciprocal as a fraction
\newcommand{\interval}[1]{\sqbracs{0,\abs{#1}}}	%creates the closed interval from 0 to the length of the input #1, denoted by absolute value
\newcommand{\eps}{\varepsilon}					%pretty epsilons
\newcommand{\charFunc}[1]{\mathcal{I}_{#1}}%{\mathbb{1}_{#1}}		%characteristic function of a set

\newcommand{\dddom}{\widetilde{\Omega}}			%3D domain notation
\newcommand{\ddom}{\Omega}						%2D domain notation
\newcommand{\dddmes}{\widetilde{\mu}}			%3D measure
\newcommand{\ddmes}{\mu}						%2D measure

\newcommand{\graph}{\mathbb{G}}					%graph variable
\newcommand{\vertSet}{\mathcal{V}}					%set of vertices rather than big V
\newcommand{\edgeSet}{\mathcal{E}}					%set of edges rather than large E, \graph = (V,E)
\newcommand{\wavenumber}{\kappa}				%fourier variable or wavenumber, not to confuse with jk subscripts!
\newcommand{\qm}{\theta}						%quasi-momentum parameter
\newcommand{\kt}{\bracs{\wavenumber, \qm}}		%(k, theta) pair
\newcommand{\dmap}{\Gamma_0}					%Dirichlet map
\newcommand{\nmap}{\Gamma_1}					%Neumann map
\newcommand{\effFreq}{\Lambda}					%Effective frequency sqrt(w^2-\wavenumber^2)
\newcommand{\conLeft}{\stackrel{\rightarrow}{\smash{\sim}\rule{0pt}{0.4ex}}} %j connects to k, j left
\newcommand{\conRight}{\stackrel{\leftarrow}{\smash{\sim}\rule{0pt}{0.4ex}}} %j connects to k, j right
\newcommand{\con}{\sim}							%j connects to k, indifferent of direction

%standard sets
\newcommand{\naturals}{\mathbb{N}}			%natural numbers
\newcommand{\integers}{\mathbb{Z}}			%integers
\newcommand{\rationals}{\mathbb{Q}}			%rational numbers
\newcommand{\reals}{\mathbb{R}}				%real numbers
\newcommand{\complex}{\mathbb{C}}			%complex numbers

%brackets and norms
\newcommand{\bracs}[1]{\left( #1 \right)}				%encloses input in brackets
\newcommand{\sqbracs}[1]{\left[ #1 \right]}				%encloses input in square brackets
\newcommand{\clbracs}[1]{\left\{ #1 \right\}}			%encloses input in curly bracers
\newcommand{\abs}[1]{\left\lvert #1 \right\rvert}					%absolute value
\newcommand{\norm}[1]{\lvert\lvert #1 \rvert\rvert}		%norm 

%function sets
\newcommand{\smooth}[1]{C^{\infty}\bracs{#1}}							%smooth functions
\newcommand{\ltwo}[2]{L^{2}\bracs{#1,\mathrm{d}#2}}						%general L^2 space
\newcommand{\gradSob}[2]{H^1_\mathrm{grad}\bracs{#1, \mathrm{d}#2}}		%gradient Sobolev space
\newcommand{\gradSobQM}[2]{H^1_{\qm, \mathrm{grad}}\bracs{#1, \mathrm{d}#2}} %gradient + i\qm Sobolev space
\newcommand{\ktgradSob}[2]{H^1_{\wavenumber,\qm,\mathrm{grad}}\bracs{#1, \mathrm{d}#2}}	%k,\qm-gradient Sobolev space
\newcommand{\curlSob}[2]{H^1_\mathrm{curl}\bracs{#1, \mathrm{d}#2}}		%curl Sobolev space
\newcommand{\tcurlSob}[2]{H^1_{\qm, \mathrm{curl}}\bracs{#1, \mathrm{d}#2}}		%curl + i\qm Sobolev space
\newcommand{\kcurlSob}[2]{H^1_{\wavenumber,\mathrm{curl}}\bracs{#1, \mathrm{d}#2}}	%k-curl Sobolev space
\newcommand{\ktcurlSob}[2]{H^1_{\wavenumber,\qm,\mathrm{curl}}\bracs{#1, \mathrm{d}#2}}	%k,\qm-curl Sobolev space
\newcommand{\ktcurlSobDivFree}[2]{\mathcal{H}^{\kt}\bracs{#1, \mathrm{d}#2}}					%k,\qm-curl, divergence-free Sobolev space
\newcommand{\supp}{\mathrm{supp}}										%support of a function

%grad and curl sets
\newcommand{\gradZero}[2]{\mathcal{G}_{ #1, \mathrm{d}#2}\bracs{0}}		%gradients of zero for domain #1 with measure #2
\newcommand{\kgradZero}[2]{\mathcal{G}_{ #1, \mathrm{d}#2}^{(\wavenumber)}\bracs{0}}	%k-gradients of zero for domain #1 with measure #2
\newcommand{\curlZero}[2]{\mathcal{C}_{ #1, \mathrm{d}#2}\bracs{0}}	%curls of zero for domain #1 with measure #2
\newcommand{\kcurlZero}[2]{\mathcal{C}_{ #1, \mathrm{d}#2}^{(\wavenumber)}\bracs{0}}	%k-curls of zero for domain #1 with measure #2

%derivatives and grad-like symbols
\newcommand{\diff}[2]{\dfrac{\mathrm{d}#1}{\mathrm{d}#2}}			%complete derivative d#1/d#2
\newcommand{\pdiff}[2]{\dfrac{\partial #1}{\partial #2}}			%partial derivative p#1/p#2
\newcommand{\ddiff}[2]{\dfrac{\mathrm{d}^2 #1}{\mathrm{d} {#2}^2}}	%2nd deriv
\newcommand{\grad}{\nabla}											%grad operator
\newcommand{\tgrad}{\nabla^{\qm}}									%grad operator with qm superscript
\newcommand{\kgrad}{\grad^{(\wavenumber)}}							%grad with wavenumber superscript
\newcommand{\ktgrad}{\grad^{\kt}}				%grad with wavenumber, qm superscript
\newcommand{\curl}[1]{\grad_{#1}\wedge}							%curl with subscript #1
\newcommand{\kcurl}[1]{\grad_{#1}^{(\wavenumber)}\wedge}			%k-curl with measure subscript #1
\newcommand{\ktcurl}[1]{\grad_{#1}^{\kt}\wedge}		%k,theta-curl with measure subscript #1
\newcommand{\laplacian}{\Delta}						%laplacian operator, can have subscripts attached

%displaying integrals
\newcommand{\integral}[3]{\int_{#1}#2 \ \mathrm{d}#3}			%integral, domain #1, integrand #2, measure #3
\newcommand{\md}{\mathrm{d}}									%differential d

%convergence
%\newcommand{\lconv}[1]{\xrightarrow{#1}}							%convergence with #1 above the rightarrow - requires mathtools
\newcommand{\lconv}[1]{\overset{#1}{\longrightarrow}}				%convergence with #1 above the rightarrow
\newcommand{\toInfty}[1]{ \ \text{as} \ #1 \rightarrow\infty}		%writes out "as #1 tends to infty" %maths commands, variables, and other packages

%labelling hacks
\newcommand\labelthis{\addtocounter{equation}{1}\tag{\theequation}}

\DeclareMathOperator*{\dw}{width}
\newcommand{\dirWidth}[2]{\dw_{#2}\bracs{#1}}
\newcommand{\tlambda}{\tilde{\lambda}}

\renewcommand{\curl}[1]{\mathrm{curl}\bracs{#1}}
\renewcommand{\ktcurl}[1]{\mathrm{curl}^{\kt}_{\dddmes}\bracs{#1}}
\newcommand{\hone}{\mathcal{H}}
\newcommand{\aop}{\mathcal{A}}

%-------------------------------------------------------------------------
%DOCUMENT STARTS

\begin{document}

Let $\graph$ be a graph embedded into $\ddom$ with the usual singular measure $\dddmes = \ddmes + \nu$.
Use $\ktcurl{u}$ to denote the $\kt$-tangential curl (wrt $\dddmes$) of $u\in\ltwo{\ddom}{\dddmes}$, and use $\curl{v}$ to denote the (classical) curl of a $C^1$ function $v$.

The (classical) Maxwell system is
\begin{subequations} \label{eq:MaxwellSystem}
	\begin{align} 
		- \curl{E} &= \rmi\omega\mu H, \\
		\curl{H} &= \rmi\omega\epsilon E,
	\end{align}
\end{subequations}
we will use as our basis the ``weak formulation"
\begin{align*}
	\integral{\ddom}{ \recip{\epsilon}H\cdot\overline{\ktcurl{\phi}} - \recip{\mu}E\cdot\overline{\ktcurl{\psi}} + \rmi\omega\bracs{H\cdot\overline{\psi} + E\cdot\overline{\phi}} }{\dddmes}
	&= 0.
\end{align*}

So let's define the ``Maxwell operator" $\aop$ as the operator with domain
\begin{align*}
	\mathrm{dom}\aop = \clbracs{ u = \bracs{E,H}\in\sqbracs{\ltwo{\ddom}{\dddmes}^3}^2 \setVert \exists f=\bracs{f^E, f^H}\in\sqbracs{\ltwo{\ddom}{\dddmes}^3}^2, \right. \\
	\left. \ktgrad_{\dddmes}\cdot f^E = 0, \ \ktgrad_{\dddmes}\cdot f^H = 0, \right. \\
	\left. \integral{\ddom}{ \recip{\epsilon}H\cdot\overline{\ktcurl{\phi}} - \recip{\mu}E\cdot\overline{\ktcurl{\psi}} + \rmi\omega\bracs{H\cdot\overline{\psi} + E\cdot\overline{\phi}} }{\dddmes}
	= \integral{\ddom}{f^E\cdot\overline{\phi} + f^H\cdot\overline{\psi}}{\dddmes}, \right. \\
	\left. \forall\phi,\psi\in\smooth{\ddom}^3.  },
\end{align*}
with action $\aop u = f - \rmi\omega u$, where $u,f$ are as in the definition of the domain of $\aop$.

\section*{Problems with this ``weak form"}
The manner in which we define the concept of divergence-free greatly impacts whether we obtain an interesting problem, at least in this context.
For explicit clarity, we (currently) define divergence-free as follows:
\begin{definition}
	A vector field $u\in\ltwo{\ddom}{\dddmes}^3$ is $\kt$-divergence-free (with respect to $\dddmes$) if
	\begin{align*}
		\integral{\ddom}{ u\cdot\overline{g} }{\dddmes} &=0 \quad\forall\bracs{v,g}\in W^{\kt}_{\dddmes, \mathrm{grad}},
	\end{align*}
	and we write $\ktgrad_{\dddmes}\cdot u = 0$.
\end{definition}
Notice that it is actually sufficient to simply test the condition
\begin{align*}
	\integral{\ddom}{ u\cdot\overline{\ktgrad\phi} }{\dddmes} &=0 , \quad\forall\phi\in\smooth{\ddom},
\end{align*}
to show that $u$ is divergence free (under this definition), by the definition of $W^{\kt}_{\dddmes, \mathrm{grad}}$.

This results in the following conclusion:
\begin{prop}
	Suppose that $u$ is $\kt$-divergence-free with respect to $\dddmes$.
	Then for every $v\in\ktcurlSob{\ddom}{\dddmes}$, we have that $u$ is orthogonal to $\ktcurl{v}$.
\end{prop}
\begin{proof}
	Since $u$ is divergence-free, Proposition E.5 of Paper\_Curl2021 implies that $U^{(jk)}_1=0$ on every edge $I_{jk}$.
	Furthermore for any $v\in\ktcurlSob{\ddom}{\dddmes}$, 
	\begin{align*}
		\ktcurl{v} &= 
		\begin{cases} 
			\bracs{v_3' +\rmi\qm_{jk} v_3 - \rmi\wavenumber V_2^{(jk)}}\hat{n}_{jk} & x\in I_{jk}, \\
			0 & x\in\vertSet.
		\end{cases}
	\end{align*}
	As a result,
	\begin{align*}
		\integral{\ddom}{ u\cdot\overline{\ktcurl{v}} }{\dddmes}
		&= \integral{\ddom}{u\cdot\overline{\ktcurl{v}} }{\ddmes} + 0 \\
		&= \sum_{j\con k}\integral{I_{jk}}{ u\cdot\bracs{\overline{v}_3' +\rmi\qm_{jk} \overline{v}_3 - \rmi\wavenumber\overline{V}_2^{(jk)}}\hat{n}_{jk} }{\ddmes} \\
		&= \sum_{j\con k}\integral{I_{jk}}{ \bracs{\overline{v}_3' +\rmi\qm_{jk} \overline{v}_3 - \rmi\wavenumber\overline{V}_2^{(jk)}}U_1^{(jk)} }{\ddmes} \\
		&= \sum_{j\con k}\integral{I_{jk}}{ 0 }{\ddmes} = 0.
	\end{align*}
\end{proof}
It is worth remarking here that this result is entirely due to the requirement that a divergence-free function $u$ be orthogonal to $\gradZero{\ddom}{\dddmes}$.
Even though the definition of divergence-free requires that $u$ be orthogonal to both gradients of zero and tangential gradients, we only need this assumption to deduce that $u$ is orthogonal to tangential curls.

We next observe that:
\begin{prop}
	Suppose $u=\bracs{E,H}\in\mathrm{dom}\aop$. 
	Then $E$ and $H$ are $\kt$-divergence free with respect to $\dddmes$.
\end{prop} 
\begin{proof}
	There exists some $f$ such that the function $u=\bracs{E,H}$ satisfies
	\begin{align*}
		\integral{\ddom}{ \recip{\epsilon}H\cdot\overline{\ktcurl{\phi}} - \recip{\mu}E\cdot\overline{\ktcurl{\psi}} + \rmi\omega\bracs{H\cdot\overline{\psi} + E\cdot\overline{\phi}} }{\dddmes}
		= \integral{\ddom}{f^E\cdot\overline{\phi} + f^H\cdot\overline{\psi}}{\dddmes},
	\end{align*}
	for all $\phi,\psi\in\smooth{\ddom}^3$.
	For the time being, let us set $\psi=0$.
	Let $\varphi\in\smooth{\ddom}$ and set $\phi = \ktgrad\varphi$.
	Then we have that
	\begin{align*}
		0 &= \integral{\ddom}{ \recip{\epsilon}H\cdot\overline{\ktcurl{\phi}} + \rmi\omega E\cdot\overline{\phi} - f^E\cdot\overline{\phi} }{\dddmes} \\
		&= \integral{\ddom}{ \rmi\omega E\cdot\overline{\ktgrad\varphi} - f^E\cdot\overline{\ktgrad\varphi} }{\dddmes} \\
		&= \integral{\ddom}{ \rmi\omega E\cdot\overline{\ktgrad\varphi} }{\dddmes},
	\end{align*}	 
	since $f^E$ is $\kt$-divergence-free.
	Clearly, this forces us to conclude that $E$ is $\kt$-divergence-free too.
	
	The proof for $H$ is identical, just using $\phi=0$ and $\psi=\ktgrad\varphi$ instead.
\end{proof}

This means that $u\in\mathrm{dom}\aop$ implies that $E,H$ are divergence-free.
Thus, they are orthogonal to all $\kt$-tangential curls by the preceding result.
As such, the ``weak" form defining $\aop$ reduces (since the curl-parts vanish) and we just get that $\bracs{E,H}=\recip{\rmi\omega}\bracs{f^E,f^H}$.
Thus, the operator $\aop u = f - \rmi\omega u = 0$ --- that is, everything in the domain is mapped to 0!

\section*{Changing the notion of ``divergence-free"}
Given that it is a part of the notion of divergence-free that results in any divergence-free function being orthogonal to $\kt$-tangential curls, we might have some luck in redefining the notion of divergence-free --- in particular, restricting it to orthogonality to $\kt$-gradients.
Explicitly,
\begin{definition}[Alternative Divergence-Free Notion:]
	A vector field $u\in\ltwo{\ddom}{\dddmes}^3$ is $\kt$-divergence-free (with respect to $\dddmes$) if
	\begin{align*}
		\integral{\ddom}{ u\cdot\overline{\ktgrad_{\dddmes}v} }{\dddmes} &=0 \quad\forall v\in\ktgradSob{\ddom}{\dddmes},
	\end{align*}
	and we write $\ktgrad_{\dddmes}\cdot u = 0$.	
\end{definition}
This means that we loose the condition that $U_1^{(jk)}=0$ on the edges from Proposition E.5 in Paper\_Curl2021 (as well as one of the other vertex conditions).
However, it does mean that a divergence-free function is no longer orthogonal to all tangential curls.
We can also try to motivate this alternative definition by reasoning that the tangential gradients are really the only gradients we care about, whilst the gradients of zero are essentially redundant --- as such, being divergence free should only entail being orthogonal to the gradients that we will actually be considering in the first place.

We are still able to deduce that $u=\bracs{E,H}\in\mathrm{dom}\aop$ implies that $E,H$ are divergence-free (in this new sense).
Also, we can actually deduce the action of the operator $\aop$ again - and it has the potential to be non-trivial!
Suppose that $u\in\mathrm{dom}\aop$ and let $f$ be the divergence-free function such that
\begin{align*}
	\integral{\ddom}{ \recip{\epsilon}H\cdot\overline{\ktcurl{\phi}} - \recip{\mu}E\cdot\overline{\ktcurl{\psi}} + \rmi\omega\bracs{H\cdot\overline{\psi} + E\cdot\overline{\phi}} }{\dddmes}
	= \integral{\ddom}{f^E\cdot\overline{\phi} + f^H\cdot\overline{\psi}}{\dddmes}, \labelthis\label{eq:OperatorWeakForm} \\
	\forall\phi,\psi\in\smooth{\ddom}^3.
\end{align*}
Now employ the usual trick --- first take smooth functions $\phi, \psi$ whose support only intersects the interior of one edge $I_{jk}$.
This allows us to deduce that
\begin{align*}
	0 &= \integral{I_{jk}}{ \recip{\epsilon}\bracs{\overline{\phi}_3' - \rmi\qm_{jk}\overline{\phi}_3 + \rmi\wavenumber\Phi^{(jk)}_2}H\cdot\hat{n}_{jk} - \recip{\mu}\bracs{\overline{\psi}_3' - \rmi\qm_{jk}\overline{\psi}_3 + \rmi\wavenumber\Psi^{(jk)}_2}E\cdot\hat{n}_{jk} + \\
	&\qquad \bracs{\rmi\omega E - f^E}\cdot\overline{\phi} + \bracs{\rmi\omega H - f^H}\cdot\overline{\psi} }{\lambda_{jk}},
\end{align*}
from which we can gather components, deducing that on each edge $I_{jk}$
\begin{align*}
	\bracs{\phi_1}: \quad & 0 = \frac{\rmi\wavenumber}{\epsilon}H\cdot\hat{n}_{jk} e^{(jk)}_1 + \rmi\omega E_1 - f^E_1, \\
	\bracs{\phi_2}: \quad & 0 = \frac{\rmi\wavenumber}{\epsilon}H\cdot\hat{n}_{jk} e^{(jk)}_2 + \rmi\omega E_2 - f^E_2, \\
	\bracs{\psi_1}: \quad & 0 = \frac{\rmi\wavenumber}{\mu}E\cdot\hat{n}_{jk} e^{(jk)}_1 + \rmi\omega H_1 - f^H_1, \\
	\bracs{\psi_2}: \quad & 0 = \frac{\rmi\wavenumber}{\mu}E\cdot\hat{n}_{jk} e^{(jk)}_2 + \rmi\omega H_2 - f^H_2,
\end{align*}
using the fundamental lemma of variations.
We can perform the following manipulations to the equations $\bracs{\phi_1}$ through $\bracs{\psi_2}$ to deduce that:
\begin{align*}
	\bracs{\phi_1}n^{(jk)}_1 + \bracs{\phi_2}n^{(jk)}_2
	&\quad\implies\quad 0 = \bracs{\rmi\omega E - f^E}\cdot\hat{n}_{jk}, \\
	\bracs{\phi_1}e^{(jk)}_1 + \bracs{\phi_2}e^{(jk)}_2
	&\quad\implies\quad 0 = \bracs{\rmi\omega E - f^E}\cdot\hat{e}_{jk} + \frac{\rmi\wavenumber}{\epsilon}H\cdot\hat{n}_{jk}, \\
	\bracs{\psi_1}n^{(jk)}_1 + \bracs{\psi_2}n^{(jk)}_2
	&\quad\implies\quad 0 = \bracs{\rmi\omega H - f^H}\cdot\hat{n}_{jk}, \\
	\bracs{\psi_1}e^{(jk)}_1 + \bracs{\psi_2}e^{(jk)}_2
	&\quad\implies\quad 0 = \bracs{\rmi\omega H - f^H}\cdot\hat{e}_{jk} - \frac{\rmi\wavenumber}{\mu}E\cdot\hat{n}_{jk}.
\end{align*}
Incidently, the normal-components being zero is what we would deduce if we took $\phi,\psi$ as a sequence of smooth functions converging to a gradient of zero in \eqref{eq:OperatorWeakForm}.
Then since we know that
\begin{align*}
	\cdot\begin{pmatrix} a_1 \\ a_2 \\ 0 \end{pmatrix}
	&= \bracs{a\cdot\hat{e}_{jk}}\hat{e}_{jk} + \bracs{a\cdot\hat{n}_{jk}}\hat{n}_{jk},
	\qquad\forall a\in\complex^3,
\end{align*}
we can deduce that
\begin{align*}
	\begin{pmatrix} f^E_1 - \rmi\omega E_1 \\ f^E_2 - \rmi\omega E_2 \\ 0 \end{pmatrix}
	&= \bracs{\frac{\rmi\wavenumber}{\epsilon}H\cdot\hat{n}_{jk}}\hat{e}_{jk}, \\
	\begin{pmatrix} f^H_1 - \rmi\omega H_1 \\ f^H_2 - \rmi\omega H_2 \\ 0 \end{pmatrix}
	&= -\bracs{\frac{\rmi\wavenumber}{\mu}E\cdot\hat{n}_{jk}}\hat{e}_{jk}.
\end{align*}

The components $\phi_3$ and $\psi_3$ yield the relations
\begin{align*}
	0 &= \integral{I_{jk}}{ \recip{\epsilon}H\cdot\hat{n}_{jk}\overline{\phi}_3' - \frac{\rmi\qm_{jk}}{\epsilon}H\cdot\hat{n}_{jk}\overline{\phi}_3 + \bracs{\rmi\omega E_3 - f^E_3}\overline{\phi}_3 }{\lambda_{jk}}, \\
	0 &= \integral{I_{jk}}{ -\recip{\mu}E\cdot\hat{n}_{jk}\overline{\psi}_3' - \frac{\rmi\qm_{jk}}{\mu}E\cdot\hat{n}_{jk}\overline{\psi}_3 + \bracs{\rmi\omega H_3 - f^H_3}\overline{\psi}_3 }{\lambda_{jk}},	
\end{align*}
which hold for every smooth $\phi, \psi$ with support intersecting the interior of the edge $I_{jk}$.
As such, we can deduce that $\recip{\epsilon}H\cdot\hat{n}_{jk}$ and $\recip{\mu}E\cdot\hat{n}_{jk}$ are differentiable in the $\gradSob{\interval{I_{jk}}}{y}$-sense, and that
\begin{align*}
	\bracs{\recip{\epsilon}H\cdot\hat{n}_{jk}}' &= \rmi\omega E_3 - f^E_3 - \frac{\rmi\qm_{jk}}{\epsilon}H\cdot\hat{n}_{jk}, \\
	\bracs{\recip{\mu}E\cdot\hat{n}_{jk}}' &= -\bracs{\rmi\omega H_3 - f^H_3} - \frac{\rmi\qm_{jk}}{\mu}E\cdot\hat{n}_{jk}.
\end{align*}
As such, we now know the form of the function $f-\rmi\omega u$ on the edge $I_{jk}$.

To obtain the form of $f-\rmi\omega u$ at the vertices, we now take $\phi, \psi$ to be smooth functions supported on a ball centred on the vertex $v_j$ and containing no other vertices.
Returning to \eqref{eq:OperatorWeakForm}, noting that the curls are zero at the vertices themselves, and the relations that hold on the edges that were just deduced, we obtain
\begin{align*}
	\alpha_j\begin{pmatrix} f^E - \rmi\omega E \\ f^H - \rmi\omega H \end{pmatrix} \cdot\begin{pmatrix} \overline{\phi} \\ \overline{\psi} \end{pmatrix}\big\vert_{v_j} &= 
	\sum_{j\conRight k} \clbracs{\recip{\epsilon}H^{(jk)}\cdot\hat{n}_{jk}\overline{\phi}_3\big\vert_{v_j}	 - \recip{\mu}E\cdot\hat{n}_{jk}\overline{\psi}_3\big\vert_{v_j}}	 \\
	&\quad - \sum_{j\conLeft k} \clbracs{\recip{\epsilon}H^{(jk)}\cdot\hat{n}_{jk}\overline{\phi}_3\big\vert_{v_j}	 - \recip{\mu}E\cdot\hat{n}_{jk}\overline{\psi}_3\big\vert_{v_j}}	.
\end{align*}
Matching components of $\phi$ and $\psi$ then provides us with the form of $f-\rmi\omega u$ at the vertices too.
As such, we can deduce that
\begin{align*}
	\aop u = f - \rmi\omega u &=
	\begin{cases}
		\begin{pmatrix}
			\bracs{\frac{\rmi\wavenumber}{\epsilon}H^{(jk)}\cdot\hat{n}_{jk}}\hat{e}_{jk} \\
			-\bracs{\recip{\epsilon}H^{(jk)}\cdot\hat{n}_{jk}}' - \frac{\rmi\qm_{jk}}{\epsilon}H^{(jk)}\cdot\hat{n}_{jk} \\
			-\bracs{\frac{\rmi\wavenumber}{\mu}E^{(jk)}\cdot\hat{n}_{jk}}\hat{e}_{jk} \\
			\bracs{\recip{\mu}E^{(jk)}\cdot\hat{n}_{jk}}' + \frac{\rmi\qm_{jk}}{\mu}E^{(jk)}\cdot\hat{n}_{jk}
		\end{pmatrix}
		& x\in I_{jk}, \\
		\begin{pmatrix}
			0 \\
			0 \\
			\recip{\alpha_j}\clbracs{ \sum_{j\conRight k}\recip{\epsilon}H^{(jk)}\cdot\hat{n}_{jk}\big\vert_{v_j} - \sum_{j\conLeft k}\recip{\epsilon}H^{(jk)}\cdot\hat{n}_{jk}\big\vert_{v_j} } \\
			0 \\
			0 \\
			-\recip{\alpha_j}\clbracs{ \sum_{j\conRight k}\recip{\mu}E^{(jk)}\cdot\hat{n}_{jk}\big\vert_{v_j} - \sum_{j\conLeft k}\recip{\mu}E^{(jk)}\cdot\hat{n}_{jk}\big\vert_{v_j} }
		\end{pmatrix}
		& x=v_j\in\vertSet.
	\end{cases}
\end{align*}
And now, we know what the action of $\aop$ is on functions in it's domain.
This still seems off though --- notice how the components $E_3,H_3$ are essentially ignored by the operator $\aop$, as well as the parts $E\cdot\hat{e}_{jk}$ and $H\cdot\hat{e}_{jk}$.
However, for a given (divergence-free) $g=\bracs{g^E,g^H}^\top\in\sqbracs{\ltwo{\ddom}{\dddmes}^3}^2$, the problem $\aop u = g$ does make sense now, and does define a quantum graph problem.

\end{document}