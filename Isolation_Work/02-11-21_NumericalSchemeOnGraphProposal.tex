\documentclass[11pt]{report}

\usepackage{url}
\usepackage[margin=2.5cm]{geometry} % See geometry.pdf to learn the layout options. There are lots.
\geometry{a4paper} %or letterpaper or a5paper or ...

\usepackage{graphicx}
\usepackage{tikz}

%\input imports all commands from the target files
%The idea behind this file is that it will be used to store all the maths-related macros that I concoct; so that I can import all the commands by \input{this file} in the preamble of any file that I want to use them in.
%This should make the top-level files look a lot cleaner, and the preamble much shorter!

\usepackage{amssymb}
\usepackage{amsmath}
%\usepackage{mathtools}

%theorems and lemma etc setup using amsthm
\usepackage{amsthm}
\theoremstyle{definition}
\newtheorem{definition}{Definition}[section]
\theoremstyle{plain}
\newtheorem{theorem}{Theorem}[section]
\theoremstyle{plain}
\newtheorem{lemma}[theorem]{Lemma}
\theoremstyle{plain}
\newtheorem{prop}[theorem]{Proposition}
\theoremstyle{plain}
\newtheorem{cory}[theorem]{Corollary}
\theoremstyle{definition}
\newtheorem{convention}[theorem]{Convention}
\theoremstyle{definition}
\newtheorem{assumption}[theorem]{Assumption}
\theoremstyle{definition}
\newtheorem{conjecture}[theorem]{Conjecture}

\allowdisplaybreaks %allows equations in the same align environment to split over multiple pages.

%tstk always need be there
\newcommand{\tstk}[1]{\textbf{#1} \newline}

%this adds extra functionality to pmatrix, vmatrix, bmatrix etc by allowing you to pass an optional argument in [FACTOR] to multiply the default spacing between elements by FACTOR
\makeatletter
\renewcommand*\env@matrix[1][\arraystretch]{%
  \edef\arraystretch{#1}%
  \hskip -\arraycolsep
  \let\@ifnextchar\new@ifnextchar
  \array{*\c@MaxMatrixCols c}
  }
\makeatother

%begin the macros via newcommand. Try to group them up reasonably!

%notation and variable use throughout the file
\renewcommand{\vec}[1]{\mathbf{#1}}				%vectors are bold, not overline arrow
\newcommand{\recip}[1]{\frac{1}{#1}}			%fast reciprocal as a fraction
\newcommand{\interval}[1]{\sqbracs{0,\abs{#1}}}	%creates the closed interval from 0 to the length of the input #1, denoted by absolute value
\newcommand{\eps}{\varepsilon}					%pretty epsilons
\newcommand{\charFunc}[1]{\mathcal{I}_{#1}}%{\mathbb{1}_{#1}}		%characteristic function of a set

\newcommand{\dddom}{\widetilde{\Omega}}			%3D domain notation
\newcommand{\ddom}{\Omega}						%2D domain notation
\newcommand{\dddmes}{\widetilde{\mu}}			%3D measure
\newcommand{\ddmes}{\mu}						%2D measure

\newcommand{\graph}{\mathbb{G}}					%graph variable
\newcommand{\vertSet}{\mathcal{V}}					%set of vertices rather than big V
\newcommand{\edgeSet}{\mathcal{E}}					%set of edges rather than large E, \graph = (V,E)
\newcommand{\wavenumber}{\kappa}				%fourier variable or wavenumber, not to confuse with jk subscripts!
\newcommand{\qm}{\theta}						%quasi-momentum parameter
\newcommand{\kt}{\bracs{\wavenumber, \qm}}		%(k, theta) pair
\newcommand{\dmap}{\Gamma_0}					%Dirichlet map
\newcommand{\nmap}{\Gamma_1}					%Neumann map
\newcommand{\effFreq}{\Lambda}					%Effective frequency sqrt(w^2-\wavenumber^2)
\newcommand{\conLeft}{\stackrel{\rightarrow}{\smash{\sim}\rule{0pt}{0.4ex}}} %j connects to k, j left
\newcommand{\conRight}{\stackrel{\leftarrow}{\smash{\sim}\rule{0pt}{0.4ex}}} %j connects to k, j right
\newcommand{\con}{\sim}							%j connects to k, indifferent of direction

%standard sets
\newcommand{\naturals}{\mathbb{N}}			%natural numbers
\newcommand{\integers}{\mathbb{Z}}			%integers
\newcommand{\rationals}{\mathbb{Q}}			%rational numbers
\newcommand{\reals}{\mathbb{R}}				%real numbers
\newcommand{\complex}{\mathbb{C}}			%complex numbers

%brackets and norms
\newcommand{\bracs}[1]{\left( #1 \right)}				%encloses input in brackets
\newcommand{\sqbracs}[1]{\left[ #1 \right]}				%encloses input in square brackets
\newcommand{\clbracs}[1]{\left\{ #1 \right\}}			%encloses input in curly bracers
\newcommand{\abs}[1]{\left\lvert #1 \right\rvert}					%absolute value
\newcommand{\norm}[1]{\lvert\lvert #1 \rvert\rvert}		%norm 

%function sets
\newcommand{\smooth}[1]{C^{\infty}\bracs{#1}}							%smooth functions
\newcommand{\ltwo}[2]{L^{2}\bracs{#1,\mathrm{d}#2}}						%general L^2 space
\newcommand{\gradSob}[2]{H^1_\mathrm{grad}\bracs{#1, \mathrm{d}#2}}		%gradient Sobolev space
\newcommand{\gradSobQM}[2]{H^1_{\qm, \mathrm{grad}}\bracs{#1, \mathrm{d}#2}} %gradient + i\qm Sobolev space
\newcommand{\ktgradSob}[2]{H^1_{\wavenumber,\qm,\mathrm{grad}}\bracs{#1, \mathrm{d}#2}}	%k,\qm-gradient Sobolev space
\newcommand{\curlSob}[2]{H^1_\mathrm{curl}\bracs{#1, \mathrm{d}#2}}		%curl Sobolev space
\newcommand{\tcurlSob}[2]{H^1_{\qm, \mathrm{curl}}\bracs{#1, \mathrm{d}#2}}		%curl + i\qm Sobolev space
\newcommand{\kcurlSob}[2]{H^1_{\wavenumber,\mathrm{curl}}\bracs{#1, \mathrm{d}#2}}	%k-curl Sobolev space
\newcommand{\ktcurlSob}[2]{H^1_{\wavenumber,\qm,\mathrm{curl}}\bracs{#1, \mathrm{d}#2}}	%k,\qm-curl Sobolev space
\newcommand{\ktcurlSobDivFree}[2]{\mathcal{H}^{\kt}\bracs{#1, \mathrm{d}#2}}					%k,\qm-curl, divergence-free Sobolev space
\newcommand{\supp}{\mathrm{supp}}										%support of a function

%grad and curl sets
\newcommand{\gradZero}[2]{\mathcal{G}_{ #1, \mathrm{d}#2}\bracs{0}}		%gradients of zero for domain #1 with measure #2
\newcommand{\kgradZero}[2]{\mathcal{G}_{ #1, \mathrm{d}#2}^{(\wavenumber)}\bracs{0}}	%k-gradients of zero for domain #1 with measure #2
\newcommand{\curlZero}[2]{\mathcal{C}_{ #1, \mathrm{d}#2}\bracs{0}}	%curls of zero for domain #1 with measure #2
\newcommand{\kcurlZero}[2]{\mathcal{C}_{ #1, \mathrm{d}#2}^{(\wavenumber)}\bracs{0}}	%k-curls of zero for domain #1 with measure #2

%derivatives and grad-like symbols
\newcommand{\diff}[2]{\dfrac{\mathrm{d}#1}{\mathrm{d}#2}}			%complete derivative d#1/d#2
\newcommand{\pdiff}[2]{\dfrac{\partial #1}{\partial #2}}			%partial derivative p#1/p#2
\newcommand{\ddiff}[2]{\dfrac{\mathrm{d}^2 #1}{\mathrm{d} {#2}^2}}	%2nd deriv
\newcommand{\grad}{\nabla}											%grad operator
\newcommand{\tgrad}{\nabla^{\qm}}									%grad operator with qm superscript
\newcommand{\kgrad}{\grad^{(\wavenumber)}}							%grad with wavenumber superscript
\newcommand{\ktgrad}{\grad^{\kt}}				%grad with wavenumber, qm superscript
\newcommand{\curl}[1]{\grad_{#1}\wedge}							%curl with subscript #1
\newcommand{\kcurl}[1]{\grad_{#1}^{(\wavenumber)}\wedge}			%k-curl with measure subscript #1
\newcommand{\ktcurl}[1]{\grad_{#1}^{\kt}\wedge}		%k,theta-curl with measure subscript #1
\newcommand{\laplacian}{\Delta}						%laplacian operator, can have subscripts attached

%displaying integrals
\newcommand{\integral}[3]{\int_{#1}#2 \ \mathrm{d}#3}			%integral, domain #1, integrand #2, measure #3
\newcommand{\md}{\mathrm{d}}									%differential d

%convergence
%\newcommand{\lconv}[1]{\xrightarrow{#1}}							%convergence with #1 above the rightarrow - requires mathtools
\newcommand{\lconv}[1]{\overset{#1}{\longrightarrow}}				%convergence with #1 above the rightarrow
\newcommand{\toInfty}[1]{ \ \text{as} \ #1 \rightarrow\infty}		%writes out "as #1 tends to infty" %maths commands, variables, and other packages

%labelling hacks
\newcommand\labelthis{\addtocounter{equation}{1}\tag{\theequation}}

\newcommand{\DtN}{\mathcal{D}}
\newcommand{\di}{\DtN_{\qm, \omega}^i}
\newcommand{\dplus}{\DtN_{\qm, \omega}^+}
\newcommand{\dminus}{\DtN_{\qm, \omega}^-}
\newcommand{\dom}{\mathrm{dom}}
\newcommand{\gradgradSob}[1]{H^2\bracs{#1}}

%-------------------------------------------------------------------------
%DOCUMENT STARTS

\begin{document}

Let $\ddom=\left[0,1\right)^2$ be our usual domain filled with a singular structure $\graph$, separated by $\graph$ into the pairwise-disjoint connected components $\ddom_i$.
For each $\ddom_i$, $\qm\in\left[-\pi,\pi\right)^2$, $\omega^2>0$ define the ``Dirichlet to Neumann" (DtN) map $\di$ via
\begin{align*}
	\dom\bracs{\di} = \clbracs{ \bracs{g,h}\in\ltwo{\partial\ddom_i}{S}\times\ltwo{\partial\ddom_i}{S} \setVert \exists v\in \gradgradSob{\ddom_i} \text{ s.t. }  \laplacian_{\qm}v+\omega^2v=0, \right. \\
	 \left. v\vert_{\partial\ddom_i}=g, \ \bracs{\tgrad v\cdot n}\vert_{\partial\ddom_i}=h }.
\end{align*}
Since the identity
\begin{align*}
	\integral{D}{\tgrad u\cdot\overline{\tgrad v}}{x} 
	&= -\integral{D}{\overline{v}\laplacian_{\qm}u}{x}
	+ \integral{\partial D}{\overline{v}\tgrad u\cdot n}{S},
\end{align*}
holds, $\di$ satisfies the Green formula and is the Dirichlet to Neumann map for the ``$\qm$-shifted Helmholtz" operator $\laplacian_{\qm}+\omega^2$ on $\ddom_i$.

Our goal is to propose a numerical method which will solve
\begin{align*}
	-\bracs{\diff{}{y} + \rmi\qm_{jk}}^2 u + \bracs{\dplus u + \dminus u} &= \omega^2 u, &\qquad\text{on each } I_{jk}\in\edgeSet, \\
	\sum_{j\con k} \bracs{\pdiff{}{n} + \rmi\qm_{jk}}u_{jk}(v_j) &= 0, &\qquad\text{at each } v_j\in\vertSet, \\
	u \text{ is continuous at } & \text{each } v_j, &\qquad\forall v_j\in\vertSet,
\end{align*}
for $u\in\gradgradSob{\graph}$.
Note: if we want to find $u,\omega^2$ simultaneously (that is, actually solve the eigenvalue problem) we just adapt the final form of our numerical scheme, see below).

Let $V\subset\gradgradSob{\graph}$ be a finite-dimensional subspace with dimension $M$ and basis functions $\clbracs{\psi_m}_{m=1}^{M}$.
In practice, we would choose these to be ``hat" functions on some nodes that we place along the edges of $\graph$.
Write the approximate solution to the above problem
\begin{align*}
	u_V &= \sum_{m=1}^M u_m\psi_m, \qquad u_m\in\complex,
\end{align*}
for the basis coefficients $u_m$ to be determined.
The operator $\di$ is self-adjoint and has compact resolvent (as its inverse is the Neumann to Dirichlet map), and thus possesses a sequence of eigenvalues $\lambda^i_n$ and eigenfunctions $\varphi_n^i$, where we list the $\lambda^i_n$ in ascending order (in $n$, for each $i$).
These eigenfunctions also form a basis of the space $\ltwo{\partial\ddom_i}{S}$, and can be extended by zero to functions $\hat{\varphi}_n^i$ in $L^2\bracs{\graph}$.
This means that we can represent each $\psi_m\vert_{\partial\ddom_i}$ as a sum of the $\varphi_n^i$ as
\begin{align*}
	\psi_m = \sum_{n=1}^{\infty} c_{m,n}^i \varphi_n^i, \quad c_{m,n}^i = \ip{\psi_m}{\varphi_n^i}_{\ltwo{\partial\ddom_i}{S}},
\end{align*}
and each of the $\hat{\varphi}_n^i$ as
\begin{align*}
	\hat{\varphi}_n^i = \sum_{n=1}^{\infty} \hat{c}_{n,m}^i \psi_m, \quad \hat{c}_{n,m}^i = \ip{\varphi_n^i}{\psi_m}_{L^2\bracs{\graph}}.
\end{align*}
However, extending $\varphi_n^i$ by zero implies that
\begin{align*}
	\hat{c}_{n,m}^i = \ip{\varphi_n^i}{\psi_m}_{L^2\bracs{\graph}} = \ip{\varphi_n^i}{\psi_m}_{\ltwo{\partial\ddom_i}{S}} = \overline{\ip{\psi_m}{\varphi_n^i}}_{\ltwo{\partial\ddom_i}{S}} = \overline{c}_{m,n}^i,
\end{align*}
which cuts down on the number of constants that need to be computed.
Choose a ``truncation index" $N_i$ for each $\ddom_i$, and define the matrices $B, C, L$ via
\begin{align*}
	B_{n,m} &= \ip{\tgrad\psi_m}{\tgrad\psi_n}_{L^2\bracs{\graph}}, \\
	C_{n,m} &= \ip{\psi_m}{\psi_n}_{L^2\bracs{\graph}}, \\
	L_{n,m} &= \sum_{v_j\in\vertSet}\sum_{j\conLeft k}
	\sqbracs{ \sum_{p=1}^{N_+}c_{m,p}^+\lambda^+_p \sum_{q=1}^M \hat{c}_{p,q}^+ \ip{\psi_q}{\psi_n}_{L^2\bracs{I_{jk}}} + \sum_{p=1}^{N_-}c_{m,p}^-\lambda^-_p \sum_{q=1}^M \hat{c}_{p,q}^- \ip{\psi_q}{\psi_n}_{L^2\bracs{I_{jk}}} },
\end{align*}
where we use our usual $\pm$ notation for the regions $\ddom^{\pm}$ adjacent to an edge $I_{jk}$.
Setting $U = \bracs{u_1, ..., u_M}^\top$, our approximate solution $u_V$ can then be found by determining the solution to
\begin{align*}
	B U &= \bracs{\omega^2 C - L} U.
\end{align*}
Note that $B$ is the term in the above equation which does not depend on $\omega^2$ --- $L$ depends on $\omega^2$ through the Steklov eigenfunctions associated to $\laplacian_{\qm}-\omega^2$.
This provides us with a system of $M$ algebraic equations in $M$ unknowns which we can solve for $U$, or if $\omega^2$ is unknown, we can solve as a generalised eigenvalue problem.
However, at each step of the generalised eigenvalue problem, we will need to compute $L_{n,m}$ again, since $\omega$ will be iteratively updated, which in turn will require us to compute new Steklov eigenfunctions.
Note that, if we are interested in the resolvent equation (replace $\omega^2 u$ with $f$ in the original formulation) then we just replace $omega^2 C$ with the column vector $F=\bracs{f_1,...,f_M}^\top$ where $f = \sum_{m=1}^M f_m\psi_m$.

\section*{Computing the DtN eigenvalues and eigenfunctions}
We can compute the DtN operator's eigenvalues (and eigenfunctions) via the ``max-min" principle;
\begin{align*}
	\lambda^i_n &= \max_{S_{n-1}}\min_{\varphi\in S_{n-1}}\clbracs{ \frac{\ip{\varphi}{\di\varphi}_{\ltwo{\partial\ddom_i}{S}}}{\norm{\varphi}_{\ltwo{\partial\ddom_i}{S}}} \setVert \varphi\perp S_{n-1}},
\end{align*}
where $S_{n-1}$ is a subspace of $\ltwo{\partial\ddom_i}{S}$ with dimension $n-1$.
The eigenfunction $\varphi_n^i$ is the $\varphi$ for which the ``max-min" is attained.
Given the domain of $\di$, we can demonstrate that
\begin{align*}
	\ip{\varphi}{\di\varphi}_{\ltwo{\partial\ddom_i}{S}}
	&= \integral{\ddom_i}{ \tgrad \varphi\cdot\overline{\tgrad \varphi} + \varphi\overline{\laplacian_{\qm} \varphi} }{x} \\
	&= \integral{\ddom_i}{ \tgrad \varphi\cdot\overline{\tgrad \varphi} - \omega^2 \varphi\overline{\varphi} }{x}
	= \norm{\tgrad \varphi}_{\ltwo{\ddom_i}{x}} - \omega^2 \norm{\varphi}_{\ltwo{\ddom_i}{x}},
\end{align*}
and therefore
\begin{align*}
	\lambda^i_n 
	&= \max_{S'_{n-1}}\min_{\varphi\in S'_{n-1}}\clbracs{ \frac{\norm{\tgrad \varphi}_{\ltwo{\ddom_i}{x}} - \omega^2 \norm{\varphi}_{\ltwo{\ddom_i}{x}}}{\norm{\varphi}_{\ltwo{\partial\ddom_i}{S}}} \setVert \varphi\perp S'_{n-1} }, \\
	&= \max_{S'_{n-1}}\min_{\varphi\in S'_{n-1}}\clbracs{ \norm{\tgrad \varphi}_{\ltwo{\ddom_i}{x}} - \omega^2 \norm{\varphi}_{\ltwo{\ddom_i}{x}} \setVert \norm{\varphi}_{\ltwo{\partial\ddom_i}{S}}=1, \ \varphi\perp S'_{n-1} },
\end{align*}
where $S'_{n-1}$ is a subspace of $\gradgradSob{\ddom_i}$ with dimension $n-1$.
We then have the following procedure available to extract the $\lambda_n^i, \varphi_n^i$:
\begin{enumerate}
	\item Solve
	\begin{align*}
		\lambda_1^i &= \min_{\substack{\varphi\in\gradgradSob{\ddom_i} \\ \norm{\varphi}_{\ltwo{\partial\ddom_i}{S}}=1}} \clbracs{ \norm{\tgrad \varphi}_{\ltwo{\ddom_i}{x}} - \omega^2 \norm{\varphi}_{\ltwo{\ddom_i}{x}} },
	\end{align*}
	to obtain $\lambda_1^i$.
	The argmin of the above expression is the eigenfunction $\varphi_1^i$.
	\item For $n>1$, the eigenfunctions $\varphi_k^i$ are known for $1\leq k\leq n-1$.
	Furthermore, we also know that $\varphi_n^i$ is orthogonal to each of the $\varphi_k^i$, and so we know that the subspace in which the maximum will be attained is $S'_{n-1} = \mathrm{span}\clbracs{\varphi_k^i \setVert 1\leq k\leq n-1}$.
	Thus, we solve
	\begin{align*}
		\lambda_n^i &= \min_{\substack{\varphi\in\gradgradSob{\ddom_i} \\ \norm{\varphi}_{\ltwo{\partial\ddom_i}{S}}=1}} \clbracs{ \norm{\tgrad \varphi}_{\ltwo{\ddom_i}{x}} - \omega^2 \norm{\varphi}_{\ltwo{\ddom_i}{x}} \setVert \varphi\perp\varphi_k^i, \text{ for each } 1\leq k\leq n-1 },
	\end{align*}
	with the argmin being the eigenfunction $\varphi_n^i$.
\end{enumerate}

We can numerically solve these minimisation problems via the method of Lagrange multipliers for example, but this would require us to settle for approximate $\varphi_n^i$ and eigenvalues.

\section*{Theory Needed, and Cost Estimations/ Summary}
We summarise here the overall algorithm that we are proposing and where the largest ``computational costs" will come from.

The procedure is as follows:
\begin{enumerate}
	\item Choose a finite-dimensional subspace $V\subset\gradgradSob{\graph}$ and a basis $\clbracs{\psi_m}_{m=1}^M$.
	Recommend a basis of locally-defined functions, like ``tent" or ``hat" functions placed on nodes along the edges of $\graph$.
	\item For each region $i$, compute the first $N_i$ eigenvalues $\lambda_n^i$ and eigenfunctions $\varphi_n^i$ for the map $\di$.
	The choice of $N_i$ will be important here --- too high might lead to heavy computational cost, whilst too small might lead to inaccuracy.
	A happy medium might n\"iavely seem to be choosing $N_i = M$ for every region, given that we are already working in a finite dimensional subspace for $u$, its shouldn't hurt to throw away the higher eigenvalues, but having less than $M$ might cause issues for the change of basis calculations.
	\item Compute the coefficients $c_{m,n}^i$ and the entries of the matrices $B, C, L$.
	The entries of $B$ and $C$ can be computed immediately after step 1, or analytically to save on numerical approximation of integrals.
	It is also worth noting that $c_{m,n}^i=0$ whenever $\supp\bracs{\psi_m}\cap\partial\ddom_i = \emptyset$, and that $\hat{c}_{m,n}^i=\overline{c}_{n,m}^i$.
	\item Solve the system $B U = \bracs{\omega^2 C - L} U$.
	Note that if we are solving for $\omega^2$ and $u_V$, $L$ will need to be recomputed at each iteration of the solve.
	Also, $B, C$, and $L$ are Hermitian, as can be seen from their (and the basis coefficients') definitions in terms of inner products.
\end{enumerate}

The key idea here is that $B,C$ only need to be computed once, and are only $M\times M$ in size (although will in general be dense matrices).
The two \emph{major} costs will come in from steps 2 and 4:
\begin{itemize}
	\item Step 4 requires a matrix solve.
	If we are using $\omega$ as a constant, or are looking at the resolvent problem, then this cost is that of solving an $M\times M$ system.
	If however we are trying to find $\omega^2$ and $u_V$, then this cost is multiplied by the cost of computing the Steklov functions in step 2.
	\item Step 2 requires us to determine the first $N_i$ of the Steklov eigenfunctions (and eigenvalues), by solving $N_i$ constrained optimisation problems.
	We also have to introduce a finite dimensional space $W\subset\gradgradSob{\ddom_i}$ to approximate the $\varphi_n^i$ in, which will affect the accuracy of the approximate solution $u_V$.
	Let $W$ have dimension $p$ for the cost estimation that follows.
	The cost for each optimisation problem should be of the order of another linear-system solve, but I'll need to double check.
\end{itemize}
Thus, I estimate the (major) cost of the numerical scheme to be (worst case, assuming we want to find $\omega^2$ too);
\begin{align*}
	\text{ cost } 
	&= \bracs{ M\times M \text{ eigenvalue or linear system solve} } \\
	& \quad\times \sum_{N_i}\bracs{ N_i \times (p\times p-\text{constrained optimisation problem solves}) }.
\end{align*}
The memory usage would be of the order of (worst case, assuming no $c_{m,n}^i$ are zero)
\begin{align*}
	\text{ memory }
	&= (M+1)\sum_{i}N_i  &\qquad\text{for the } c_{m,n}^i \text{ and } \lambda_n^i, \\
	& \ + 3M^2 &\qquad\text{for the entries of } B, C, L, \\
	& \ + M^2\abs{\edgeSet} &\qquad\text{for each } \ip{\psi_q}{\psi_n}_{L^2\bracs{I_{jk}}},
\end{align*}
so would be of the order $M^2$ if we are choosing $N_i$ of the order $M$.
There is also a memory cost accrued for the representation of the solution to each optimisation problem for the $\lambda_n^i$, however once we are done with computing the $\lambda_n^i$ for the region $\ddom_i$, we can compute any associated $c_{m,n}^i$ and then reuse the memory assigned to the representations.

The theory that would then need to be filled in or investigated would be
\begin{itemize}
	\item Assurance that $u_V$ gets close to $u$ as the dimension of $V$ increases, or at least is optimal in $V$ (cf Cea's lemma).
	\item The accuracy loss from truncation at $N_i$.
	\item Accuracy loss from the representation of the $\varphi_n^i$.
	\item Optimal (or necessary) conditions on $N_i$ and $p$ given $M$.
	\item Any timesaves that we can make computationally or analytically --- for example, identical $\ddom_i$ have identical eigenfunctions (computational), and we can compute some of the inner products if we choose our $\psi_m$ cleverly (analytical).
\end{itemize}
\end{document}