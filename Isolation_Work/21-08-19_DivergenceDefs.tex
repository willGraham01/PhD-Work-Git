\documentclass[11pt]{report}

\usepackage{url}
\usepackage[margin=2.5cm]{geometry} % See geometry.pdf to learn the layout options. There are lots.
\geometry{a4paper} %or letterpaper or a5paper or ...

%for figures and graphics
\usepackage{graphicx}
\usepackage{subcaption} %allows subfigures
\usepackage[bottom]{footmisc} %footnotes go below figures
%\usepackage{parskip} %adds line space between paragraphs by default
\usepackage{enumerate} %allows lower case roman numerials in enumerate environments

\DeclareGraphicsRule{.tif}{png}{.png}{`convert #1 `dirname #1`/`basename #1 .tif`.png}
\graphicspath{{../Diagrams/Diagram_PDFs/} {../Diagrams/Numerical_Results/}}

%\input imports all commands from the target files
%The idea behind this file is that it will be used to store all the maths-related macros that I concoct; so that I can import all the commands by \input{this file} in the preamble of any file that I want to use them in.
%This should make the top-level files look a lot cleaner, and the preamble much shorter!

\usepackage{amssymb}
\usepackage{amsmath}
%\usepackage{mathtools}

%theorems and lemma etc setup using amsthm
\usepackage{amsthm}
\theoremstyle{definition}
\newtheorem{definition}{Definition}[section]
\theoremstyle{plain}
\newtheorem{theorem}{Theorem}[section]
\theoremstyle{plain}
\newtheorem{lemma}[theorem]{Lemma}
\theoremstyle{plain}
\newtheorem{prop}[theorem]{Proposition}
\theoremstyle{plain}
\newtheorem{cory}[theorem]{Corollary}
\theoremstyle{definition}
\newtheorem{convention}[theorem]{Convention}
\theoremstyle{definition}
\newtheorem{assumption}[theorem]{Assumption}
\theoremstyle{definition}
\newtheorem{conjecture}[theorem]{Conjecture}

\allowdisplaybreaks %allows equations in the same align environment to split over multiple pages.

%tstk always need be there
\newcommand{\tstk}[1]{\textbf{#1} \newline}

%this adds extra functionality to pmatrix, vmatrix, bmatrix etc by allowing you to pass an optional argument in [FACTOR] to multiply the default spacing between elements by FACTOR
\makeatletter
\renewcommand*\env@matrix[1][\arraystretch]{%
  \edef\arraystretch{#1}%
  \hskip -\arraycolsep
  \let\@ifnextchar\new@ifnextchar
  \array{*\c@MaxMatrixCols c}
  }
\makeatother

%begin the macros via newcommand. Try to group them up reasonably!

%notation and variable use throughout the file
\renewcommand{\vec}[1]{\mathbf{#1}}				%vectors are bold, not overline arrow
\newcommand{\recip}[1]{\frac{1}{#1}}			%fast reciprocal as a fraction
\newcommand{\interval}[1]{\sqbracs{0,\abs{#1}}}	%creates the closed interval from 0 to the length of the input #1, denoted by absolute value
\newcommand{\eps}{\varepsilon}					%pretty epsilons
\newcommand{\charFunc}[1]{\mathcal{I}_{#1}}%{\mathbb{1}_{#1}}		%characteristic function of a set

\newcommand{\dddom}{\widetilde{\Omega}}			%3D domain notation
\newcommand{\ddom}{\Omega}						%2D domain notation
\newcommand{\dddmes}{\widetilde{\mu}}			%3D measure
\newcommand{\ddmes}{\mu}						%2D measure

\newcommand{\graph}{\mathbb{G}}					%graph variable
\newcommand{\vertSet}{\mathcal{V}}					%set of vertices rather than big V
\newcommand{\edgeSet}{\mathcal{E}}					%set of edges rather than large E, \graph = (V,E)
\newcommand{\wavenumber}{\kappa}				%fourier variable or wavenumber, not to confuse with jk subscripts!
\newcommand{\qm}{\theta}						%quasi-momentum parameter
\newcommand{\kt}{\bracs{\wavenumber, \qm}}		%(k, theta) pair
\newcommand{\dmap}{\Gamma_0}					%Dirichlet map
\newcommand{\nmap}{\Gamma_1}					%Neumann map
\newcommand{\effFreq}{\Lambda}					%Effective frequency sqrt(w^2-\wavenumber^2)
\newcommand{\conLeft}{\stackrel{\rightarrow}{\smash{\sim}\rule{0pt}{0.4ex}}} %j connects to k, j left
\newcommand{\conRight}{\stackrel{\leftarrow}{\smash{\sim}\rule{0pt}{0.4ex}}} %j connects to k, j right
\newcommand{\con}{\sim}							%j connects to k, indifferent of direction

%standard sets
\newcommand{\naturals}{\mathbb{N}}			%natural numbers
\newcommand{\integers}{\mathbb{Z}}			%integers
\newcommand{\rationals}{\mathbb{Q}}			%rational numbers
\newcommand{\reals}{\mathbb{R}}				%real numbers
\newcommand{\complex}{\mathbb{C}}			%complex numbers

%other notations
\newcommand{\rmi}{\mathrm{i}}				%imaginary unit i (RoMan font i)
\newcommand{\e}{\mathrm{e}}					%Euler number, e (Roman font e)

%brackets and norms
\newcommand{\bracs}[1]{\left( #1 \right)}				%encloses input in brackets
\newcommand{\sqbracs}[1]{\left[ #1 \right]}				%encloses input in square brackets
\newcommand{\clbracs}[1]{\left\{ #1 \right\}}			%encloses input in curly bracers
\newcommand{\abs}[1]{\left\lvert #1 \right\rvert}					%absolute value
\newcommand{\norm}[1]{\lvert\lvert #1 \rvert\rvert}		%norm 
\newcommand{\setVert}{\ \middle\vert \ }					%vertical bar for the middle of sets

%function sets
\newcommand{\smooth}[1]{C^{\infty}\bracs{#1}}							%smooth functions
\newcommand{\ltwo}[2]{L^{2}\bracs{#1,\mathrm{d}#2}}						%general L^2 space
\newcommand{\gradSob}[2]{H^1_\mathrm{grad}\bracs{#1, \mathrm{d}#2}}		%gradient Sobolev space
\newcommand{\gradSobQM}[2]{H^1_{\qm, \mathrm{grad}}\bracs{#1, \mathrm{d}#2}} %gradient + i\qm Sobolev space
\newcommand{\ktgradSob}[2]{H^1_{\wavenumber,\qm,\mathrm{grad}}\bracs{#1, \mathrm{d}#2}}	%k,\qm-gradient Sobolev space
\newcommand{\curlSob}[2]{H^1_\mathrm{curl}\bracs{#1, \mathrm{d}#2}}		%curl Sobolev space
\newcommand{\tcurlSob}[2]{H^1_{\qm, \mathrm{curl}}\bracs{#1, \mathrm{d}#2}}		%curl + i\qm Sobolev space
\newcommand{\kcurlSob}[2]{H^1_{\wavenumber,\mathrm{curl}}\bracs{#1, \mathrm{d}#2}}	%k-curl Sobolev space
\newcommand{\ktcurlSob}[2]{H^1_{\wavenumber,\qm,\mathrm{curl}}\bracs{#1, \mathrm{d}#2}}	%k,\qm-curl Sobolev space
\newcommand{\ktcurlSobDivFree}[2]{\mathcal{H}^{\kt}\bracs{#1, \mathrm{d}#2}}					%k,\qm-curl, divergence-free Sobolev space
\newcommand{\supp}{\mathrm{supp}}										%support of a function

%grad and curl sets
\newcommand{\gradZero}[2]{\mathcal{G}_{ #1, \mathrm{d}#2}\bracs{0}}		%gradients of zero for domain #1 with measure #2
\newcommand{\kgradZero}[2]{\mathcal{G}_{ #1, \mathrm{d}#2}^{(\wavenumber)}\bracs{0}}	%k-gradients of zero for domain #1 with measure #2
\newcommand{\curlZero}[2]{\mathcal{C}_{ #1, \mathrm{d}#2}\bracs{0}}	%curls of zero for domain #1 with measure #2
\newcommand{\kcurlZero}[2]{\mathcal{C}_{ #1, \mathrm{d}#2}^{(\wavenumber)}\bracs{0}}	%k-curls of zero for domain #1 with measure #2

%derivatives and grad-like symbols
\newcommand{\diff}[2]{\dfrac{\mathrm{d}#1}{\mathrm{d}#2}}			%complete derivative d#1/d#2
\newcommand{\pdiff}[2]{\dfrac{\partial #1}{\partial #2}}			%partial derivative p#1/p#2
\newcommand{\ddiff}[2]{\dfrac{\mathrm{d}^2 #1}{\mathrm{d} {#2}^2}}	%2nd deriv
\newcommand{\pddiff}[2]{\dfrac{\partial^2 #1}{\partial {#2}^2}}		%2nd partial derivative
\newcommand{\grad}{\nabla}											%grad operator
\newcommand{\tgrad}{\nabla^{\qm}}									%grad operator with qm superscript
\newcommand{\kgrad}{\grad^{(\wavenumber)}}							%grad with wavenumber superscript
\newcommand{\ktgrad}{\grad^{\kt}}				%grad with wavenumber, qm superscript
\newcommand{\curl}[1]{\grad_{#1}\wedge}							%curl with subscript #1
\newcommand{\kcurl}[1]{\grad_{#1}^{(\wavenumber)}\wedge}			%k-curl with measure subscript #1
\newcommand{\ktcurl}[1]{\grad_{#1}^{\kt}\wedge}		%k,theta-curl with measure subscript #1
\newcommand{\laplacian}{\Delta}						%laplacian operator, can have subscripts attached

%displaying integrals
\newcommand{\integral}[3]{\int_{#1}#2 \ \mathrm{d}#3}			%integral, domain #1, integrand #2, measure #3
\newcommand{\md}{\mathrm{d}}									%differential d

%convergence
%\newcommand{\lconv}[1]{\xrightarrow{#1}}							%convergence with #1 above the rightarrow - requires mathtools
\newcommand{\lconv}[1]{\overset{#1}{\longrightarrow}}				%convergence with #1 above the rightarrow
\newcommand{\toInfty}[1]{ \ \text{as} \ #1 \rightarrow\infty}		%writes out "as #1 tends to infty" %maths commands, variables, and other packages

%labelling hacks
\newcommand\labelthis{\addtocounter{equation}{1}\tag{\theequation}}

\newcommand{\diver}{\mathrm{div}}
\newcommand{\ktdiv}[1]{\ktgrad_{#1}\cdot}
\newcommand{\divZero}[2]{\mathcal{D}_{#1, \md #2}\bracs{0}}
\newcommand{\dist}{\mathrm{dist}}

%-------------------------------------------------------------------------
%DOCUMENT STARTS

\begin{document}

Choosing how to define the notion of divergence for our $\ltwo{\ddom}{\dddmes}^3$-functions presents two possible options.
This document contains a summary of each, and some results and musings about which one seems like the ``correct" notion to use.
We'll begin with the less promising notion, since it's fairly easy to see why the ``divergence" we end up with is undesirable!
Then, we will move onto the alternative notion.

Throughout, let $\ddom$ be our usual domain with an embedded graph $\graph$.
Use our standard notation on $\graph$, and for the singular measures $\dddmes, \ddmes, \nu$.

\section{Via Smooth Approximations}
We could choose to follow a similar definition of divergence-free that is motivated by our previous definitions for curl and gradient, which we do here.
\begin{definition}[$W^{\kt}_{\diver}$]
	For a Borel measure $\rho$, define the set
	\begin{align*}
		W^{\kt}_{\diver} &= \overline{\clbracs{ \bracs{\phi, \ktdiv{}\phi} \setVert \phi\in\smooth{\ddom}^3 }}.
	\end{align*}
	where the closure is taken in $\ltwo{\ddom}{\rho}^3\times\ltwo{\ddom}{\rho}$.
	Next, define the set of ``divergences of zero" with respect to $\rho$ as
	\begin{align*}
		\divZero{\ddom}{\rho} &= \clbracs{ d \setVert \bracs{0,d}\in W^{\kt}_{\diver}}.
	\end{align*}
\end{definition}
We can quickly deduce that the definition of $\divZero{\ddom}{\rho}$ does not depend on the value of $\kt$.
It is also not hard to deduce that $\divZero{\ddom}{\rho}$ is a closed linear subspace of $\ltwo{\ddom}{\rho}$, and so we can decompose
\begin{align*}
	\ltwo{\ddom}{\rho} &= \divZero{\ddom}{\rho} \oplus \divZero{\ddom}{\rho}^\perp.
\end{align*}
We can then deduce (akin to how we do so for gradients and curls) that a function $u$ will have only one unique ``$\kt$-tangential divergence".
As such, we can define the ``$\kt$-tangential divergence" of a function $u$ as being the unique element $d$ of the pair $\bracs{u,d}\in W^{\kt}_{\diver}$ such that $d\perp\divZero{\ddom}{\rho}$.

The trouble with these definitions is that, very quickly, we see that our divergences are going to be trivial.
\begin{lemma}
	For an edge $I_{jk}$, we have that $\divZero{\ddom}{\lambda_{jk}} = \ltwo{\ddom}{\lambda_{jk}}$.
\end{lemma}
\begin{proof}
	Given any smooth function $f\in\smooth{\ddom}$, set $\phi = \bracs{x\cdot n_{jk}}f\hat{n}_{jk}$.
	It can clearly be seen that $\phi=0$ on $I_{jk}$ and that $\ktdiv{}\phi = f$ on $I_{jk}$, and so we have that $f\in\divZero{ddom}{\lambda_{jk}}$.
	By a density argument, the result follows.
\end{proof}

This essentially kills off any interesting behaviour an individual edge might have, and the story gets worse once we realise the extension lemma holds.
\begin{lemma}[Extension lemma for divergences]
	Suppose that $d\in\ltwo{\ddom}{\lambda_{jk}}$ is such that $d=0$ whenever $\dist\bracs{x,\partial I_{jk}}\leq\recip{n}$, for some $n\in\naturals$.
	Then the function
	\begin{align*}
		\tilde{d} &= \begin{cases} d & x\in I_{jk}, \\ 0 & x\not\in I_{jk}, \end{cases}
	\end{align*}
	is an element of both $\divZero{\ddom}{\ddmes}$ and $\divZero{\ddom}{\dddmes}$.
\end{lemma}
\begin{proof}
	Take the usual function $\chi_{jk}^n$ that we used in the other two extension lemmas, and an approximating sequence $\phi^l$ for $d$.
	Setting $\varphi = \chi_{jk}^n\phi^l$ will enable us to deduce the result --- note that this approximating sequence is 0 at every vertex of the graph, for every $l$ (and every $n$, which matters later) so the convergence with respect to $\dddmes$ is obtained since the contribution from the measure $\nu$ is zero.
\end{proof}
As such, we now have that we can elevate any divergence of zero on an edge to the whole graph, including when we care about the vertices.
By linearity (and the trivial opposite implication), we thus have that
\begin{align*}
	\divZero{\ddom}{\ddmes} = \clbracs{ d \setVert d_{jk}\in\divZero{\ddom}{\lambda_{jk}} }.
\end{align*}

At the vertices, we also have a similar story.
\begin{lemma}
	We have that
	\begin{align*}
		\divZero{\ddom}{\nu} &= \ltwo{\ddom}{\nu}.
	\end{align*}
	Furthermore, if $d\in\divZero{\ddom}{\nu}$ then we have that the function
	\begin{align*}
		\tilde{d} = \begin{cases} d & x\in\vertSet, \\ 0 & x\not\in\vertSet, \end{cases}
	\end{align*}
	is an element of $\divZero{\ddom}{\dddmes}$.
\end{lemma}
\begin{proof}
	In both cases it suffices to show the result for when $d$ supports only a single vertex, and then apply linearity.
	For the first result: take smooth functions $\phi$ where, on a suitably small ball around the vertex $v_j$, we have that $\phi = \frac{d(v_j)}{2}\bracs{x - v_j}$.
	For the second result: use smooth functions $\phi_n$ like in Proposition E.3 of the curl paper and then use the approximating sequence $\Phi^n = \bracs{\phi_n\bracs{x_1-v_j^{(1)}}, 0, 0}^\top$, starting the sequence at suitably large $n$.
\end{proof}

This forces us to deduce that $\divZero{\ddom}{\dddmes} = \ltwo{\ddom}{\dddmes}$, and thus for any function $u$ we have that $\ktdiv{\dddmes}u = 0$.
This is not ideal --- not least because it would imply that all functions are divergence free.

\section{Via ``Integration by Parts"}
An alternative definition of ``$\kt$-divergence" can be given as below.
\begin{definition}
	Let $u\in\ltwo{\ddom}{\rho}^3$ and suppose that there exists a function $d\in\ltwo{\ddom}{\rho}$ such that
	\begin{align*}
		\integral{\ddom}{ u\cdot\overline{\ktgrad_{\rho}\phi} }{\rho}
		&= -\integral{\ddom}{ d\overline{\phi} }{\rho}
		\qquad \forall\phi\in\smooth{\ddom}.
	\end{align*}
	Then we call $d =: \ktdiv{\rho}u$ the ``$\kt$-divergence of $u$ with respect to $\rho$.
	If $\ktdiv{\rho}u=0$, we say that $u$ is ``$\kt$-divergence-free".
\end{definition}

This alternative definition immediately coincides with the notion of ``divergence-free" that we have been using thus far in the curl-curl problem.
However, it is not clear whether $\ktdiv{\rho}u$ is unique for all Borel measures $\rho$\footnote{For this to be true, we would need a version of the Fundamental Lemma of the Calculus of Variations to hold for the measure $\rho$, or to know that smooth functions are dense in $\ltwo{\ddom}{\rho}$.}, however we can safely deduce that when $\rho=\dddmes, \ddmes$, or $\nu$, it is.
\begin{lemma}
	$\ktdiv{\rho}u$ is unique when $\rho=\dddmes, \ddmes$, or $\nu$.
\end{lemma}
\begin{proof}
	Suppose $u$ has two divergences and form the difference $d=d_1-d_2$, and we'll see that $d$ integrated against $\phi$ is zero for any smooth function $\phi$.
	Choose the smooth functions $\phi$ in the definition to first support only the interior of one edge, which by the (normal!) FTC gives us that $d=0$ on the edges.
	Then use smooth functions that support balls around each vertex to deduce that $d=0$ at the vertices too, so $d=0$ and thus the original two divergences were equal.
\end{proof}

Now we can ask questions about what $\ktdiv{\dddmes}u$ is (when it exists).
\begin{prop}
	Suppose that $u\in\ltwo{\ddom}{\dddmes}$ has divergence $\ktdiv{\dddmes}u$.
	Then,
	\begin{align*}
		(i) \ & U_2^{(jk)} \in\gradSob{\ddom}{\lambda_{jk}} \ \forall I_{jk}\in\edgeSet, \\
		(ii) \ & \ktdiv{\dddmes}u = \bracs{U_2^{(jk)}}' + \rmi\qm_{jk}U_2^{(jk)} + \rmi\wavenumber u_3 \ \text{on each } I_{jk}, \\
		(iii) \ & \bracs{\ktdiv{\dddmes}u}(v_j) = -\recip{\alpha_j}\bracs{ \sum_{j\conRight k}U_2^{(kj)}(v_j) - \sum_{j\conLeft k}U_2^{(jk)}(v_j) } + \rmi\wavenumber u_3(v_j), \ \forall v_j\in\vertSet.
	\end{align*}
\end{prop}
Note that a converse of this theorem holds; provided that condition $(i)$ is assumed, one can show that the divergence of $u$ exists (and it will be defined by $(ii)$ and $(iii)$).
\begin{proof}
	Again, this is just the standard choice of smooth functions to consider.
	First on the edges with interior support, which'll give the weak derivative for $U_2^{(jk)}$ and the condition $(ii)$.
	Then for smooth functions that support the vertices, which'll provide us with the condition $(iii)$ given our knowledge of $\dddmes$-tangential gradients.
\end{proof}

Using this definition provides us with a form of divergence that also matches our intuitive expectations, like for gradient and curl.
Divergence should indicate how much a vector field is ``spreading out" from a given point, or rather how much of a source/sink a given point is in a vector field.
On the edges, condition $(ii)$ indicates that the ``spread" of a function is given by the rate of change of it's components along the edge $I_{jk}$ (that is, $U_2^{(jk)}$) and into the planes that have been Fourier-transformed out ($u_3$).
Then at each of the vertices, the ``spread" is again dictated by $u_3$ (into the direction of travel) and the incoming contributions of the $U_2^{(jk)}$ to the vertex.
If $u_1, u_2$ happen to be continuous, then one can determine whether there is any ``spread" in the plane by simply looking at the unit normals at the vertex $v_j$ and the value $u_3(v_j)$.

This notion of divergence also plays well with our knowledge of regular calculus --- the divergence of a curl is always zero.
\begin{prop}
Let $u\in\ktcurlSob{\ddom}{\dddmes}$.
Then $\ktdiv{\dddmes}u$ exists and is zero.
\end{prop}
\begin{proof}
	Let $\phi\in\smooth{\ddom}{\dddmes}$, and recall that we know that
	\begin{align*}
		\overline{\ktgrad_{\dddmes}\phi} &= 
		\begin{cases}
			\bracs{ \overline{\phi}' - \rmi\qm_{jk}\overline{\phi} }\hat{e}_{jk} - \rmi\wavenumber\overline{\phi}\hat{x}_3 & x\in I_{jk}, \\
			-\rmi\wavenumber\overline{\phi}\bracs{v_j}\hat{x}_3 & x=v_j, \ v_j\in\vertSet.
		\end{cases}
	\end{align*}
	We also know that
	\begin{align*}
		\ktcurl{\dddmes}u &= 
		\begin{cases}
			\bracs{ u_3' + \rmi\qm_{jk}u_3 - \rmi\wavenumber U_2^{(jk)} }\hat{n}_{jk} & x\in I_{jk}, \\
			0 & x = v_j, \ v_j\in\vertSet.
		\end{cases}
	\end{align*}
	As such, we observe that for any $\phi\in\smooth{\ddom}$,
	\begin{align*}
		\integral{\ddom}{ \ktcurl{\dddmes}u\cdot\overline{\ktgrad_{\dddmes}\phi} }{\dddmes}
		&= \integral{\ddom}{ \ktcurl{\dddmes}u\cdot\overline{\ktgrad_{\dddmes}\phi} }{\ddmes} + 0 \\
		&= \integral{\ddom}{ \bracs{ u_3' + \rmi\qm_{jk}u_3 - \rmi\wavenumber U_2^{(jk)} }n_{jk} \cdot \bracs{ \overline{\phi}' - \rmi\qm_{jk}\overline{\phi} }e_{jk} }{\ddmes} \\
		&= 0.
	\end{align*}
	Therefore, we conclude that $\ktdiv{\dddmes}u = 0$.
\end{proof}

\end{document}