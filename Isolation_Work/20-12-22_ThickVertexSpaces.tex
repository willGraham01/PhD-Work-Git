\documentclass[11pt]{report}

\usepackage{url}
\usepackage[margin=2.5cm]{geometry} % See geometry.pdf to learn the layout options. There are lots.
\geometry{a4paper} %or letterpaper or a5paper or ...

%for figures and graphics
\usepackage{graphicx}
\usepackage{subcaption} %allows subfigures
\usepackage[bottom]{footmisc} %footnotes go below figures
%\usepackage{parskip} %adds line space between paragraphs by default
\usepackage{enumerate} %allows lower case roman numerials in enumerate environments

\DeclareGraphicsRule{.tif}{png}{.png}{`convert #1 `dirname #1`/`basename #1 .tif`.png}
\graphicspath{{../Diagrams/Diagram_PDFs/} {../Diagrams/Numerical_Results/}}

%\input imports all commands from the target files
%The idea behind this file is that it will be used to store all the maths-related macros that I concoct; so that I can import all the commands by \input{this file} in the preamble of any file that I want to use them in.
%This should make the top-level files look a lot cleaner, and the preamble much shorter!

\usepackage{amssymb}
\usepackage{amsmath}
%\usepackage{mathtools}

%theorems and lemma etc setup using amsthm
\usepackage{amsthm}
\theoremstyle{definition}
\newtheorem{definition}{Definition}[section]
\theoremstyle{plain}
\newtheorem{theorem}{Theorem}[section]
\theoremstyle{plain}
\newtheorem{lemma}[theorem]{Lemma}
\theoremstyle{plain}
\newtheorem{prop}[theorem]{Proposition}
\theoremstyle{plain}
\newtheorem{cory}[theorem]{Corollary}
\theoremstyle{definition}
\newtheorem{convention}[theorem]{Convention}
\theoremstyle{definition}
\newtheorem{assumption}[theorem]{Assumption}
\theoremstyle{definition}
\newtheorem{conjecture}[theorem]{Conjecture}

\allowdisplaybreaks %allows equations in the same align environment to split over multiple pages.

%tstk always need be there
\newcommand{\tstk}[1]{\textbf{#1} \newline}

%this adds extra functionality to pmatrix, vmatrix, bmatrix etc by allowing you to pass an optional argument in [FACTOR] to multiply the default spacing between elements by FACTOR
\makeatletter
\renewcommand*\env@matrix[1][\arraystretch]{%
  \edef\arraystretch{#1}%
  \hskip -\arraycolsep
  \let\@ifnextchar\new@ifnextchar
  \array{*\c@MaxMatrixCols c}
  }
\makeatother

%begin the macros via newcommand. Try to group them up reasonably!

%notation and variable use throughout the file
\renewcommand{\vec}[1]{\mathbf{#1}}				%vectors are bold, not overline arrow
\newcommand{\recip}[1]{\frac{1}{#1}}			%fast reciprocal as a fraction
\newcommand{\interval}[1]{\sqbracs{0,\abs{#1}}}	%creates the closed interval from 0 to the length of the input #1, denoted by absolute value
\newcommand{\eps}{\varepsilon}					%pretty epsilons
\newcommand{\charFunc}[1]{\mathcal{I}_{#1}}%{\mathbb{1}_{#1}}		%characteristic function of a set

\newcommand{\dddom}{\widetilde{\Omega}}			%3D domain notation
\newcommand{\ddom}{\Omega}						%2D domain notation
\newcommand{\dddmes}{\widetilde{\mu}}			%3D measure
\newcommand{\ddmes}{\mu}						%2D measure

\newcommand{\graph}{\mathbb{G}}					%graph variable
\newcommand{\vertSet}{\mathcal{V}}					%set of vertices rather than big V
\newcommand{\edgeSet}{\mathcal{E}}					%set of edges rather than large E, \graph = (V,E)
\newcommand{\wavenumber}{\kappa}				%fourier variable or wavenumber, not to confuse with jk subscripts!
\newcommand{\qm}{\theta}						%quasi-momentum parameter
\newcommand{\kt}{\bracs{\wavenumber, \qm}}		%(k, theta) pair
\newcommand{\dmap}{\Gamma_0}					%Dirichlet map
\newcommand{\nmap}{\Gamma_1}					%Neumann map
\newcommand{\effFreq}{\Lambda}					%Effective frequency sqrt(w^2-\wavenumber^2)
\newcommand{\conLeft}{\stackrel{\rightarrow}{\smash{\sim}\rule{0pt}{0.4ex}}} %j connects to k, j left
\newcommand{\conRight}{\stackrel{\leftarrow}{\smash{\sim}\rule{0pt}{0.4ex}}} %j connects to k, j right
\newcommand{\con}{\sim}							%j connects to k, indifferent of direction

%standard sets
\newcommand{\naturals}{\mathbb{N}}			%natural numbers
\newcommand{\integers}{\mathbb{Z}}			%integers
\newcommand{\rationals}{\mathbb{Q}}			%rational numbers
\newcommand{\reals}{\mathbb{R}}				%real numbers
\newcommand{\complex}{\mathbb{C}}			%complex numbers

%other notations
\newcommand{\rmi}{\mathrm{i}}				%imaginary unit i (RoMan font i)
\newcommand{\e}{\mathrm{e}}					%Euler number, e (Roman font e)

%brackets and norms
\newcommand{\bracs}[1]{\left( #1 \right)}				%encloses input in brackets
\newcommand{\sqbracs}[1]{\left[ #1 \right]}				%encloses input in square brackets
\newcommand{\clbracs}[1]{\left\{ #1 \right\}}			%encloses input in curly bracers
\newcommand{\abs}[1]{\left\lvert #1 \right\rvert}					%absolute value
\newcommand{\norm}[1]{\lvert\lvert #1 \rvert\rvert}		%norm 
\newcommand{\setVert}{\ \middle\vert \ }					%vertical bar for the middle of sets

%function sets
\newcommand{\smooth}[1]{C^{\infty}\bracs{#1}}							%smooth functions
\newcommand{\ltwo}[2]{L^{2}\bracs{#1,\mathrm{d}#2}}						%general L^2 space
\newcommand{\gradSob}[2]{H^1_\mathrm{grad}\bracs{#1, \mathrm{d}#2}}		%gradient Sobolev space
\newcommand{\gradSobQM}[2]{H^1_{\qm, \mathrm{grad}}\bracs{#1, \mathrm{d}#2}} %gradient + i\qm Sobolev space
\newcommand{\ktgradSob}[2]{H^1_{\wavenumber,\qm,\mathrm{grad}}\bracs{#1, \mathrm{d}#2}}	%k,\qm-gradient Sobolev space
\newcommand{\curlSob}[2]{H^1_\mathrm{curl}\bracs{#1, \mathrm{d}#2}}		%curl Sobolev space
\newcommand{\tcurlSob}[2]{H^1_{\qm, \mathrm{curl}}\bracs{#1, \mathrm{d}#2}}		%curl + i\qm Sobolev space
\newcommand{\kcurlSob}[2]{H^1_{\wavenumber,\mathrm{curl}}\bracs{#1, \mathrm{d}#2}}	%k-curl Sobolev space
\newcommand{\ktcurlSob}[2]{H^1_{\wavenumber,\qm,\mathrm{curl}}\bracs{#1, \mathrm{d}#2}}	%k,\qm-curl Sobolev space
\newcommand{\ktcurlSobDivFree}[2]{\mathcal{H}^{\kt}\bracs{#1, \mathrm{d}#2}}					%k,\qm-curl, divergence-free Sobolev space
\newcommand{\supp}{\mathrm{supp}}										%support of a function

%grad and curl sets
\newcommand{\gradZero}[2]{\mathcal{G}_{ #1, \mathrm{d}#2}\bracs{0}}		%gradients of zero for domain #1 with measure #2
\newcommand{\kgradZero}[2]{\mathcal{G}_{ #1, \mathrm{d}#2}^{(\wavenumber)}\bracs{0}}	%k-gradients of zero for domain #1 with measure #2
\newcommand{\curlZero}[2]{\mathcal{C}_{ #1, \mathrm{d}#2}\bracs{0}}	%curls of zero for domain #1 with measure #2
\newcommand{\kcurlZero}[2]{\mathcal{C}_{ #1, \mathrm{d}#2}^{(\wavenumber)}\bracs{0}}	%k-curls of zero for domain #1 with measure #2

%derivatives and grad-like symbols
\newcommand{\diff}[2]{\dfrac{\mathrm{d}#1}{\mathrm{d}#2}}			%complete derivative d#1/d#2
\newcommand{\pdiff}[2]{\dfrac{\partial #1}{\partial #2}}			%partial derivative p#1/p#2
\newcommand{\ddiff}[2]{\dfrac{\mathrm{d}^2 #1}{\mathrm{d} {#2}^2}}	%2nd deriv
\newcommand{\pddiff}[2]{\dfrac{\partial^2 #1}{\partial {#2}^2}}		%2nd partial derivative
\newcommand{\grad}{\nabla}											%grad operator
\newcommand{\tgrad}{\nabla^{\qm}}									%grad operator with qm superscript
\newcommand{\kgrad}{\grad^{(\wavenumber)}}							%grad with wavenumber superscript
\newcommand{\ktgrad}{\grad^{\kt}}				%grad with wavenumber, qm superscript
\newcommand{\curl}[1]{\grad_{#1}\wedge}							%curl with subscript #1
\newcommand{\kcurl}[1]{\grad_{#1}^{(\wavenumber)}\wedge}			%k-curl with measure subscript #1
\newcommand{\ktcurl}[1]{\grad_{#1}^{\kt}\wedge}		%k,theta-curl with measure subscript #1
\newcommand{\laplacian}{\Delta}						%laplacian operator, can have subscripts attached

%displaying integrals
\newcommand{\integral}[3]{\int_{#1}#2 \ \mathrm{d}#3}			%integral, domain #1, integrand #2, measure #3
\newcommand{\md}{\mathrm{d}}									%differential d

%convergence
%\newcommand{\lconv}[1]{\xrightarrow{#1}}							%convergence with #1 above the rightarrow - requires mathtools
\newcommand{\lconv}[1]{\overset{#1}{\longrightarrow}}				%convergence with #1 above the rightarrow
\newcommand{\toInfty}[1]{ \ \text{as} \ #1 \rightarrow\infty}		%writes out "as #1 tends to infty" %maths commands, variables, and other packages

%labelling hacks
\newcommand\labelthis{\addtocounter{equation}{1}\tag{\theequation}}

\DeclareMathOperator*{\dw}{width}
\newcommand{\dirWidth}[2]{\dw_{#2}\bracs{#1}}
\newcommand{\tlambda}{\tilde{\lambda}}

%-------------------------------------------------------------------------
%DOCUMENT STARTS

\begin{document}

\begin{assumption} \label{ass:MeasTheoryProblemSetup}
	Let $\graph=\bracs{\vertSet,\edgeSet}$ be the period graph of an embedded graph in $\reals^2$ with period cell $\ddom$, so $\graph\subset\ddom$.
	Assign coupling constants $\alpha_j\in\complex$ to each vertex $v_j\in\vertSet$ respectively.
	We restrict ourselves to considering straight-edges between vertices, so each $I_{jk}\in \edgeSet$ is the line segment joining the vertices at either end, with lengths $l_{jk} = \abs{I_{jk}} = \norm{v_j-v_k}_2$.
	Let $e_{jk}$ be the unit vector parallel to $I_{jk}$ and directed from $v_j$ to $v_k$.
	Set
	\begin{align} \label{eq:EdgeParameterisation}
	r_{jk}:\sqbracs{0, l_{jk}} \ni t \mapsto v_j + te_{jk} \in I_{jk},
	\end{align}
	and note that $r_{jk}'(t) = e_{jk}$.
	Finally, let $n_{jk}$ be the unit normal to $I_{jk}$ such that $y_{jk} = \bracs{e_{jk}, n_{jk}}$ can be obtained by an orthonormal rotation $R_{jk}\in\mathrm{SO}(2)$ of the (canonical) axis vectors $x = \bracs{x_1, x_2}$, formally by $x = R_{jk}y_{jk}$.
\end{assumption}

Throughout, let $\ddmes$ be the singular measure that supports the edges of the graph $\graph$, and let
\begin{align*}
	\nu &= \sum_{j\in\vertSet} \alpha_j\delta_{j},
\end{align*}
be the measure resulting from the weighted sum of point-mass measures $\delta_j$ placed at the vertices $v_j$.
Assume that $\graph\subset\sqbracs{0,1}^2$ for ease, and set $\dddmes = \ddmes + \nu$.
We look to examine the ``Sobolev Space" $\gradSob{\ddom}{\dddmes}$.

\section{Objective}
Our desire is to show that
\begin{conjecture} \label{conj:ThickVertexSpaceCharacterisation}
	\begin{align*}
		\bracs{u, \grad_{\dddmes}u}\in\gradSob{\ddom}{\dddmes} \quad\Leftrightarrow\quad
		& \ \text{(i)} \ \bracs{u, \grad_{\dddmes}u}\in\gradSob{\ddom}{\ddmes}, \\
		& \ \text{(ii)} \ \bracs{u, \grad_{\dddmes}u}\in\gradSob{\ddom}{\nu}. \\
		\\
		\quad\Leftrightarrow\quad
		& \ \text{(a)} \ \bracs{u, \grad_{\dddmes}u}\in\gradSob{\ddom}{\lambda_{jk}} \ \forall I_{jk}\in\edgeSet, \\
		& \ \text{(b)} \ u \text{ is continuous across the vertices} \ v_j\in\vertSet, \\
		& \ \text{(c)} \ \grad_{\dddmes}u\vert_{v_j} = 0 \ \forall v_j\in\vertSet.
	\end{align*}
\end{conjecture}
Note that from our analysis of $\gradSob{\ddom}{\ddmes}$ we know that (i) is necessary and sufficient for (a) and (b).
Our analysis of $\gradSob{\ddom}{\nu}$ will demonstrate that (ii) is necessary and sufficient for (c), so it suffices to show that membership of $\gradSob{\ddom}{\dddmes}$ is equivalent to one of these sets of conditions.

\subsection{At present}
At present we are able to prove the following proposition.
\begin{prop}
	\begin{align*}
		\bracs{u, \grad_{\dddmes}u}\in\gradSob{\ddom}{\dddmes} \quad\Rightarrow\quad
		& \ \text{(i)} \ \bracs{u, \grad_{\dddmes}u}\in\gradSob{\ddom}{\ddmes}, \\
		& \ \text{(ii)} \ \bracs{u, \grad_{\dddmes}u}\in\gradSob{\ddom}{\nu}. \\
	\end{align*}
	Hence,
	\begin{align*}
		\bracs{u, \grad_{\dddmes}u}\in\gradSob{\ddom}{\dddmes} \quad\Rightarrow\quad
		& \ \text{(a)} \ \bracs{u, \grad_{\dddmes}u}\in\gradSob{\ddom}{\lambda_{jk}} \ \forall I_{jk}\in\edgeSet, \\
		& \ \text{(b)} \ u \text{ is continuous across the vertices} \ v_j\in\vertSet, \\
		& \ \text{(c)} \ \grad_{\dddmes}u\vert_{v_j} = 0 \ \forall v_j\in\vertSet.
	\end{align*}
\end{prop}
\begin{proof}
	Clearly, if we have smooth functions such that
	\begin{align*}
		\phi_n \lconv{\ltwo{\ddom}{\dddmes}} u, \quad \grad\phi_n\lconv{\ltwo{\ddom}{\dddmes}^2}\grad_{\dddmes}u,
	\end{align*}
	then using the fact that
	\begin{align*}
		\norm{\cdot}_{\ltwo{\ddom}{\dddmes}}^2 &= \norm{\cdot}_{\ltwo{\ddom}{\ddmes}}^2 + \norm{\cdot}_{\ltwo{\ddom}{\nu}}^2,
	\end{align*}
	we have that $\phi_n$ also converges in $\ltwo{\ddom}{\ddmes}$ and $\ltwo{\ddom}{\nu}$, as do it's gradients.	
	The conditions (a) through (c) follow from our analysis of $\gradSob{\ddom}{\ddmes}$ and $\gradSob{\ddom}{\nu}$ (see section \ref{sec:PointMassSpaceAnalysis}).
\end{proof}
Whilst this is only one-half of conjecture \ref{conj:ThickVertexSpaceCharacterisation}, it is the only direction we need to employ in our derivation of the quantum-graph problem.
Since we pose the (measure-theoretic) variational problem in the space $\gradSob{\ddom}{\dddmes}$, we then know that every function in that space also lives in $\gradSob{\ddom}{\ddmes}$ and $\gradSob{\ddom}{\nu}$, which gives us the properties we need to derive the edge-ODEs and vertex conditions.
That said, it would be ideal from an analytical perspective to show that the reverse implication holds (to complete conjecture \ref{conj:ThickVertexSpaceCharacterisation}) so that our intuition is validated; or alternatively a counterexample would be instructive in showing why our intuition isn't correct.

\section{Analysis of $\gradSob{\ddom}{\nu}$} \label{sec:PointMassSpaceAnalysis}
This section contains our characterisation of $\gradSob{\ddom}{\nu}$.
If we fix $N = \abs{\vertSet}$ as the number of vertices, we will see that $\gradZero{\ddom}{\nu} \cong \complex^{2N}$, and thus that any $\bracs{u,\grad_{\nu}u}\in\gradSob{\ddom}{\nu}$ is such that $\grad_{\nu}u=0$.
Don't forget that this is only $\nu$-almost everywhere, so there won't be any contradictions when we piece this space together with $\gradSob{\ddom}{\ddmes}$.

\begin{definition}[$d$, $\varphi_c$, and $g^j$] \label{def:UsefulObjects}
	Let 
	\begin{align*}
		d=\recip{2}\min\clbracs{\norm{v_j-v_k}_2 \ \vert \ v_j,v_k\in\vertSet}
	\end{align*}
	be half the minimum distance between any two vertices in the graph (note that this may occur between two vertices that do not share a single edge).
	$d$ exists since the graph $\graph$ is assumed finite.
	For $c\in\complex$, let $\varphi_c:\reals^2\rightarrow\complex$ be the smooth function such that
	\begin{align*}
		\varphi_c\bracs{0} = 0, &\quad \grad\varphi_c\bracs{0} = c, \\
		\supp\bracs{\varphi_c} &\subset B_{d}\bracs{0},
	\end{align*}
	where $B_{d}\bracs{0}$ denotes the ball of radius $d$ centred at the origin.
	Finally, for each $v_j\in\vertSet$ define
	\begin{align*}
		g^j_1\bracs{x} &=
		\begin{cases}
			\begin{pmatrix} 1 \\ 0 \end{pmatrix} & x=v_j, \\
			0 & x\neq v_j. \\
		\end{cases}
		&\quad
		g^j_2\bracs{x} &=
		\begin{cases}
			\begin{pmatrix} 0 \\ 1 \end{pmatrix} & x=v_j, \\
			0 & x\neq v_j. \\
		\end{cases}
	\end{align*}
\end{definition}

Notice that we have the following result:
\begin{lemma}
	The space $\ltwo{\ddom}{\nu}$ is isomorphic to $\complex^{2N}$.
	Moreover, the collection 
	\begin{align*}
		\clbracs{g_1^j, g_2^j \ \vert \ v_j\in\vertSet}
	\end{align*}
	forms a basis of $\ltwo{\ddom}{\nu}^2$.
\end{lemma}
\begin{proof}
	It is sufficient to notice that any function $f\in\ltwo{\ddom}{\nu}^2$ is entirely determined by the values it takes at the vertices $v_j$.
	Each of these values is a $\complex^2$-vector, and thus we may define the function
	\begin{align*}
		\iota:\ltwo{\ddom}{\nu}^2 \rightarrow\complex^{2N}, &\quad
		\iota\bracs{f} = \begin{pmatrix} f\bracs{v_1} \\ f\bracs{v_2} \\ \vdots \\ f\bracs{v_N} \end{pmatrix}.
	\end{align*}
	Clearly $\iota$ is a bijection, and additionally for $f,g\in\ltwo{\ddom}{\nu}$ we have that
	\begin{align*}
		\integral{\ddom}{f\cdot \overline{g}}{\nu} &= \sum_{v_j\in\vertSet} f\bracs{v_j}\cdot\overline{g\bracs{v_j}} \\
		&= \iota\bracs{f}\cdot\overline{\iota\bracs{g}}.
	\end{align*}
	So $\iota$ is an isometry.
	Moreover the inverse of the canonial basis $\clbracs{e_k \ \vert \ k\in\clbracs{1,...,2N}}$ under $\iota$ is the collection $\clbracs{g_1^j, g_2^j \ \vert \ v_j\in\vertSet}$, and hence $\clbracs{g_1^j, g_2^j \ \vert \ v_j\in\vertSet}$ forms a basis of $\ltwo{\ddom}{\nu}^2$.
\end{proof}

We now characterise the set of $\nu$-gradients of zero, $\gradZero{\ddom}{\nu}$, which will turn out to be the entire space $\ltwo{\ddom}{\nu}$.
\begin{prop}{Characterisation of $\gradZero{\ddom}{\nu}$} \label{prop:CharPointMassGradZero}
	We have that $\gradZero{\ddom}{\nu} = \ltwo{\ddom}{\nu}^2$.
\end{prop}
\begin{proof}
	Since $\gradZero{\ddom}{\nu}$ is a closed, linear subspace of $\ltwo{\ddom}{\nu}^2$ by definition, it is sufficient to show that $\gradZero{\ddom}{\nu}$ contains the basis $\clbracs{g_1^j, g_2^j \ \vert \ v_j\in\vertSet}$.
	We demonstrate inclusion of the elements $g^j_1$ (as that of $g^j_2$ is similar, with the obvious alternative choice of $c$, below).
	Take $c=\bracs{1,0}^{\top}$, fix $v_j\in\vertSet$, and set
	\begin{align*}
		\phi\bracs{x} &= \varphi_c\bracs{x-v_j},
	\end{align*}
	where $\varphi_c$ is as in definition \ref{def:UsefulObjects}.
	Note that $\phi$ is smooth by composition of smooth functions, $\supp\phi\subset B_{d}\bracs{v_j}$, and that
	\begin{align*}
		\grad\phi\bracs{x} &= \bracs{\grad\varphi_c}\bracs{x-v_j}.
	\end{align*}
	Then we have that
	\begin{align*}
		\integral{\ddom}{\abs{\phi}^2}{\nu} &= \sum_{v_l\in\vertSet} \alpha_l\abs{\phi\bracs{v_l}}^2 \\
		&= \alpha_j\abs{\varphi_c\bracs{0}}^2 + \sum_{v_l\neq v_j}\alpha_l\abs{\phi\bracs{v_l}}^2 \\
		&= 0 + \sum_{v_l\neq v_j} \alpha_l \times 0 = 0,
	\end{align*}
	and
	\begin{align*}
		\integral{\ddom}{\abs{\grad\phi - g^j_1}^2}{\nu} &= \sum_{v_l\in\vertSet} \alpha_l\abs{\grad\phi\bracs{v_l} - g^j_1\bracs{v_l}}^2 \\
		&= \alpha_j\abs{\grad\varphi_c\bracs{0} - \bracs{1,0}^{\top}}^2 + \sum_{v_l\neq v_j} \alpha_l\abs{\grad\varphi_c\bracs{v_l} - 0}^2 \\
		&= 0 + \sum_{v_l\neq v_j} \alpha_l\abs{0 - 0}^2 = 0.
	\end{align*}
	Hence, the constant sequence of smooth functions $\phi$ is such that
	\begin{align*}
		\phi\lconv{\ltwo{\ddom}{\nu}}0, \quad \grad\phi\lconv{\ltwo{\ddom}{\nu}^2} g^j_1,
	\end{align*}
	and hence $g^j_1\in\gradZero{\ddom}{\nu}$.
\end{proof}

This then provides us with the following characterisation of $\gradSob{\ddom}{\nu}$:
\begin{cory} \label{eq:CharPointMassSpace}
	We have that
	\begin{align*}
		\bracs{u,\grad_{\nu}u}\in\gradSob{\ddom}{\nu} \quad\Leftrightarrow\quad 
		& \ \text{(i)} \ u\in\ltwo{\ddom}{\nu}, \\
		& \ \text{(ii)} \ \grad_{\nu}u\in\ltwo{\ddom}{\nu}^2 \ \text{with} \ \grad_{\nu}u = 0 \ \nu\text{-almost everywhere}.
	\end{align*}
\end{cory}
\begin{proof}
	($\Rightarrow$) For the right-directed implication; $\grad_{\nu}u\in\ltwo{\ddom}{\nu}^2$ is an element of $\ltwo{\ddom}{\nu}^2$ by definition and is orthogonal to $\gradZero{\ddom}{\nu}$, but by proposition \ref{prop:CharPointMassGradZero} we know that $\gradZero{\ddom}{\nu}=\ltwo{\ddom}{\nu}^2$, we must conclude that $\grad_{\nu}u = 0$. \newline
	($\Leftarrow$) For the left-directed implication, take smooth ``bump" functions $\psi_j$ (for each $v_j\in\vertSet$) with the properties
	\begin{align*}
		\psi_j\bracs{v_j} = 1, &\quad \grad\psi_j\bracs{v_j} = 0, \\
		\supp\bracs{\psi_j} &\subset B_{d}\bracs{v_j}.
	\end{align*}
	Then consider the smooth function
	\begin{align*}
		\phi\bracs{x} &= \sum_{v_j\in\vertSet} u\bracs{v_j}\psi_j\bracs{x}, \\
		\implies \grad\phi\bracs{x} &= \sum_{v_j\in\vertSet} u\bracs{v_j}\grad\psi_j\bracs{x}.
	\end{align*}
	Then we have that
	\begin{align*}
		\integral{\ddom}{\abs{\phi - u}^2}{\nu} &= \sum_{v_j\in\vertSet} \alpha_j\abs{\phi\bracs{v_j} - u\bracs{v_j}}^2 \\
		&= \sum_{v_j\in\vertSet} \alpha_j\abs{u\bracs{v_j}}^2\abs{\psi_j\bracs{v_j}-1}^2 \\
		&= \sum_{v_j\in\vertSet} \alpha_j\abs{u\bracs{v_j}}^2 \times 0 = 0,
	\end{align*}
	and
	\begin{align*}
		\integral{\ddom}{\abs{\grad\phi - 0}^2}{\nu} &= \sum_{v_j\in\vertSet} \alpha_j\abs{ \sum_{v_l\in\vertSet} \grad\psi_l\bracs{v_j} }^2 \\
		&= \sum_{v_j\in\vertSet} \alpha_j \times 0 = 0.
	\end{align*}
	Thus, the constant sequence $\phi$ is such that
	\begin{align*}
		\phi \lconv{\ltwo{\ddom}{\nu}} u, \quad \grad\phi \lconv{\ltwo{\ddom}{\nu}^2} 0,
	\end{align*}
	and thus $\bracs{u,0}\in\gradSob{\ddom}{\nu}$.
\end{proof}

As a final remark, notice that corollary \ref{conj:ThickVertexSpaceCharacterisation} means that the space $\gradSob{\ddom}{\nu}$ is essentially isomorphic to $\complex^N$, namely functions in this space are entirely determined by their values at the vertices, and their gradients are always zero.
This matches our intuitive expectations, as the notion of a gradient (or rate of change) at an isolated point being non-zero implies that there is a small neighbourhood around the point in which we can observe the function values changing, but in the case of a point-mass measure this is not the case.

\end{document}