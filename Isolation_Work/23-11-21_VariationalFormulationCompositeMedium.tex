\documentclass[11pt]{report}

\usepackage{url}
\usepackage[margin=2.5cm]{geometry} % See geometry.pdf to learn the layout options. There are lots.
\geometry{a4paper} %or letterpaper or a5paper or ...

\usepackage{graphicx}
\usepackage{tikz}

%\input imports all commands from the target files
%The idea behind this file is that it will be used to store all the maths-related macros that I concoct; so that I can import all the commands by \input{this file} in the preamble of any file that I want to use them in.
%This should make the top-level files look a lot cleaner, and the preamble much shorter!

\usepackage{amssymb}
\usepackage{amsmath}

%theorems and lemma etc setup using amsthm
\usepackage{amsthm}
\newcommand{\tstk}[1]{\textbf{#1} \newline}
\theoremstyle{definition}
\newtheorem{definition}{Definition}[section]
\theoremstyle{plain}
\newtheorem{theorem}{Theorem}[section]
\theoremstyle{plain}
\newtheorem{lemma}[theorem]{Lemma}
\theoremstyle{plain}
\newtheorem{prop}[theorem]{Proposition}

\allowdisplaybreaks %allows equations in the same align environment to split over multiple pages.

%begin the macros via newcommand. Try to group them up reasonably!

%standard sets
\newcommand{\naturals}{\mathbb{N}}			%natural numbers
\newcommand{\integers}{\mathbb{Z}}			%integers
\newcommand{\rationals}{\mathbb{Q}}			%rational numbers
\newcommand{\reals}{\mathbb{R}}				%real numbers
\newcommand{\complex}{\mathbb{C}}			%complex numbers

%brackets and norms
\newcommand{\bracs}[1]{\left( #1 \right)}				%encloses input in brackets
\newcommand{\sqbracs}[1]{\left[ #1 \right]}				%encloses input in square brackets
\newcommand{\clbracs}[1]{\left\{ #1 \right\}}			%encloses input in curly bracers
\newcommand{\abs}[1]{\lvert #1 \rvert}					%absolute value
\newcommand{\norm}[2]{\lvert\lvert #1 \rvert\rvert}		%norm (double line)

%function sets
\newcommand{\smooth}[1]{C^{\infty}\bracs{#1}}							%smooth functions
\newcommand{\ltwo}[2]{L^{2}\bracs{#1,\mathrm{d}#2}}						%general L^2 space
\newcommand{\gradSob}[2]{H^1_\mathrm{grad}\bracs{#1, \mathrm{d}#2}}		%gradient Sobolev space
\newcommand{\curlSob}[2]{H^1_\mathrm{curl}\bracs{#1, \mathrm{d}#2}}		%curl Sobolev space
\newcommand{\kSob}[2]{H^1_{k,\mathrm{curl}}\bracs{#1, \mathrm{d}#2}}	%k-curl Sobolev space

%grad and curl sets
\newcommand{\gradZero}[2]{\mathcal{G}_{ #1, \mathrm{d}#2}\bracs{0}}		%gradients of zero for domain #1 with measure #2
\newcommand{\curlZero}[2]{\mathcal{C}_{ #1, \mathrm{d}#2}\bracs{0}}	%curls of zero for domain #1 with measure #2

%derivatives and grad-like symbols
\newcommand{\diff}[2]{\dfrac{\mathrm{d}#1}{\mathrm{d}#2}}			%complete derivative d#1/d#2
\newcommand{\pdiff}[2]{\dfrac{\partial #1}{\partial #2}}			%partial derivative p#1/p#2
\newcommand{\ddiff}[2]{\dfrac{\mathrm{d}^2 #1}{\mathrm{d}^2 #2}}	%2nd deriv
\newcommand{\grad}{\nabla}											%grad operator
\newcommand{\curl}[1]{\grad_{#1}\wedge}								%curl with measure subscript #1

%displaying integrals
\newcommand{\integral}[3]{\int_{#1}#2 \ \mathrm{d}#3}			%integral, domain #1, integrand #2, measure #3

%notation for variable use throughout the file
\newcommand{\dddom}{\widetilde{\Omega}}			%3D domain notation
\newcommand{\ddom}{\Omega}						%2D domain notation
\newcommand{\dddmes}{\widetilde{\mu}}			%3D measure
\newcommand{\ddmes}{\mu}						%2D measure

\newcommand{\graph}{\mathbb{G}}					%graph variable %maths commands, variables, and other packages

%labelling hacks
\newcommand\labelthis{\addtocounter{equation}{1}\tag{\theequation}}

\newcommand{\DtN}{\mathcal{D}}
\newcommand{\di}{\DtN_{\qm, \omega}^i}
\newcommand{\dplus}{\DtN_{\qm, \omega}^+}
\newcommand{\dminus}{\DtN_{\qm, \omega}^-}

%-------------------------------------------------------------------------
%DOCUMENT STARTS

\begin{document}

Let $\ddom=\left[0,1\right)^2$ be our usual domain filled with a singular structure $\graph$, separated by $\graph$ into the pairwise-disjoint connected components $\ddom_i$.

Consider the constrained problem of finding $\omega^2$ such that
\begin{align} \label{eq:MinProblem}
	\omega^2 &= \inf\clbracs{ \integral{\ddom}{\abs{\tgrad_{\compMes}u}^2}{\compMes}, \ u\in\tgradSob{\ddom}{\compMes} \setVert \norm{u}_{\ltwo{\ddom}{\compMes}} = 1}.
\end{align}

\begin{theorem}
	Suppose that \eqref{eq:MinProblem} has a minimising function $v\in\tgradSob{\ddom}{\compMes}$.
	Then this $v$ satisfies the system
	\begin{subequations} \label{eq:GraphSystem}
		\begin{align} 
			\bracs{\laplacian_\qm + \omega^2}v &= 0 \quad &\text{in each } \ddom_i, \\
			-\bracs{\diff{}{y}+\rmi\qm_{jk}}^2 v_{jk} &= \omega^2 v_{jk} + \bracs{\grad v\vert_{\partial\ddom^+} - \grad v\vert_{\partial\ddom^-}}\cdot n_{jk}, \quad &\text{on each } I_{jk}, \\
			\sum_{j\con k}\bracs{\pdiff{}{n} + \rmi\qm_{jk}}v_{jk}\bracs{v_j} &= 0, \quad &\text{at each } v_j\in\vertSet.
		\end{align}
	\end{subequations}
\end{theorem}
\begin{proof}
	Our approach utilises the method of Lagrange multipliers.
	To this end, let $\xi$ be our Lagrange multiplier, and consider the functional
	\begin{align*}
		J\sqbracs{u,\xi} &= \integral{\ddom}{\abs{\tgrad_{\compMes}u}^2}{\compMes} - \xi\bracs{ \norm{u}_{\ltwo{\ddom}{\compMes}} - 1 }. 
	\end{align*}
	The variation from $v$ of $J$ is then the functional
	\begin{align*}
		V\sqbracs{\phi,\xi} &= \diff{}{\eps} J\sqbracs{v+\eps\phi,\xi}\big\vert_{\eps=0}
		= \integral{\ddom}{ \tgrad_{\compMes}v\cdot\overline{\tgrad_{\compMes}\phi} + \overline{\tgrad_{\compMes} u}\cdot\tgrad_{\compMes}\phi }{\compMes}
		- \xi\int_{\ddom}{ v\cdot\overline{\phi} + \overline{v}\cdot\phi }{\compMes},
	\end{align*}
	and we have that $V\sqbracs{\phi,\xi} = 0$ for every $\phi\in\smooth{\ddom}$, since $v$ is the minimising function of \eqref{eq:MinProblem}.
	
	In particular, notice that $V\sqbracs{v,\xi}=0$, and thus
	\begin{align*}
		0 &= 2\bracs{ \integral{\ddom}{ \abs{\tgrad_{\compMes}v}^2 }{\compMes} - \xi\integral{\ddom}{\abs{v}^2}{\compMes}}
		= 2\bracs{\omega^2 - \xi\times1}, \\
		\implies \xi &= \omega^2,
	\end{align*}
	providing the value of the Lagrange multiplier.
	
	Next, we take $\phi$ whose support is contained in one of the bulk regions $\ddom_i$.
	By a method similar to our derivation of the quantum graph problem, we will obtain that \tstk{check by parts formula for when conjugations are reversed! Also confirm that we get that $v$ has sufficient regularity for us to do the by parts in the first place!}
	\begin{align*}
		V\sqbracs{\phi,\omega^2} = 0
		&= -\integral{\ddom_i}{ \overline{\phi}\bracs{\laplacian_{\qm}+\omega^2}v + \phi\overline{\bracs{\laplacian_{\qm}+\omega^2}v} }{\lambda_2},
	\end{align*}
	which will hold for every such $\phi$ with compact support in $\ddom_i$, which provides the first (bulk) equation we seek (take $\phi$ to have no real, then no imaginary part).
	
	Now, take $\phi$ to have support that straddles one of the edges $I_{jk}$.
	Again following a similar process to that in our derivation, we will arrive at \tstk{check signs of traces!}
	\begin{align*}
		V\sqbracs{\phi,\omega^2} = 0
		&= \integral{I_{jk}}{ \tgrad_{jk}v\cdot\overline{\tgrad_{jk}\phi} - \omega^2 u\overline{\phi} + \overline{\tgrad_{jk}v}\cdot\tgrad_{jk}\phi - \omega^2\overline{v}\phi }{\lambda_{jk}} \\
		&\quad + \integral{I_{jk}}{ \bracs{\tgrad v\cdot n^+}^+\overline{\phi} + \bracs{\overline{\tgrad_{jk}v}\cdot n^+}^+\phi - \bracs{\tgrad v\cdot n^-}^-\overline{\phi} - \bracs{\overline{\tgrad_{jk}v}\cdot n^-}^-\phi }{\lambda_{jk}},
	\end{align*}
	which will reduce to our second equation.
	
	Finally, consider $\phi$ that have support now containing the vertices, and we'll end up with the final equation after following the steps in our derivation again.
	This completes the proof.
\end{proof}

What can we say now?
Well, if our operator $-\laplacian_{\qm}$ is self-adjoint (in some appropriate space, so we'd need to give $\tgradSob{\ddom}{\compMes}$ a formal norm etc) then we can just apply the min-max principle to obtain the eigenvalues!
We have something of a compromise here: we still have to construct the solution in the bulk regions (so aren't entirely on the graph) but don't have to expand the majority of our computational effort to handle them --- in fact, we just approximate with a basis and solve the min-max problem for $-\laplacian_{\qm}$!

\end{document}