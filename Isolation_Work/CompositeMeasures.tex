\documentclass[11pt]{report}

\usepackage{url}
\usepackage[margin=2.5cm]{geometry} % See geometry.pdf to learn the layout options. There are lots.
\geometry{a4paper} %or letterpaper or a5paper or ...

%for figures and graphics
\usepackage{graphicx}
\usepackage{subcaption} %allows subfigures
\usepackage[bottom]{footmisc} %footnotes go below figures
%\usepackage{parskip} %adds line space between paragraphs by default
\usepackage{enumerate} %allows lower case roman numerials in enumerate environments

\DeclareGraphicsRule{.tif}{png}{.png}{`convert #1 `dirname #1`/`basename #1 .tif`.png}
\graphicspath{{../Diagrams/Diagram_PDFs/} {../Diagrams/Numerical_Results/}}

%\input imports all commands from the target files
%The idea behind this file is that it will be used to store all the maths-related macros that I concoct; so that I can import all the commands by \input{this file} in the preamble of any file that I want to use them in.
%This should make the top-level files look a lot cleaner, and the preamble much shorter!

\usepackage{amssymb}
\usepackage{amsmath}
%\usepackage{mathtools}

%theorems and lemma etc setup using amsthm
\usepackage{amsthm}
\theoremstyle{definition}
\newtheorem{definition}{Definition}[section]
\theoremstyle{plain}
\newtheorem{theorem}{Theorem}[section]
\theoremstyle{plain}
\newtheorem{lemma}[theorem]{Lemma}
\theoremstyle{plain}
\newtheorem{prop}[theorem]{Proposition}
\theoremstyle{plain}
\newtheorem{cory}[theorem]{Corollary}
\theoremstyle{definition}
\newtheorem{convention}[theorem]{Convention}
\theoremstyle{definition}
\newtheorem{assumption}[theorem]{Assumption}
\theoremstyle{definition}
\newtheorem{conjecture}[theorem]{Conjecture}

\allowdisplaybreaks %allows equations in the same align environment to split over multiple pages.

%tstk always need be there
\newcommand{\tstk}[1]{\textbf{#1} \newline}

%this adds extra functionality to pmatrix, vmatrix, bmatrix etc by allowing you to pass an optional argument in [FACTOR] to multiply the default spacing between elements by FACTOR
\makeatletter
\renewcommand*\env@matrix[1][\arraystretch]{%
  \edef\arraystretch{#1}%
  \hskip -\arraycolsep
  \let\@ifnextchar\new@ifnextchar
  \array{*\c@MaxMatrixCols c}
  }
\makeatother

%begin the macros via newcommand. Try to group them up reasonably!

%notation and variable use throughout the file
\renewcommand{\vec}[1]{\mathbf{#1}}				%vectors are bold, not overline arrow
\newcommand{\recip}[1]{\frac{1}{#1}}			%fast reciprocal as a fraction
\newcommand{\interval}[1]{\sqbracs{0,\abs{#1}}}	%creates the closed interval from 0 to the length of the input #1, denoted by absolute value
\newcommand{\eps}{\varepsilon}					%pretty epsilons
\newcommand{\charFunc}[1]{\mathcal{I}_{#1}}%{\mathbb{1}_{#1}}		%characteristic function of a set

\newcommand{\dddom}{\widetilde{\Omega}}			%3D domain notation
\newcommand{\ddom}{\Omega}						%2D domain notation
\newcommand{\dddmes}{\widetilde{\mu}}			%3D measure
\newcommand{\ddmes}{\mu}						%2D measure

\newcommand{\graph}{\mathbb{G}}					%graph variable
\newcommand{\vertSet}{\mathcal{V}}					%set of vertices rather than big V
\newcommand{\edgeSet}{\mathcal{E}}					%set of edges rather than large E, \graph = (V,E)
\newcommand{\wavenumber}{\kappa}				%fourier variable or wavenumber, not to confuse with jk subscripts!
\newcommand{\qm}{\theta}						%quasi-momentum parameter
\newcommand{\kt}{\bracs{\wavenumber, \qm}}		%(k, theta) pair
\newcommand{\dmap}{\Gamma_0}					%Dirichlet map
\newcommand{\nmap}{\Gamma_1}					%Neumann map
\newcommand{\effFreq}{\Lambda}					%Effective frequency sqrt(w^2-\wavenumber^2)
\newcommand{\conLeft}{\stackrel{\rightarrow}{\smash{\sim}\rule{0pt}{0.4ex}}} %j connects to k, j left
\newcommand{\conRight}{\stackrel{\leftarrow}{\smash{\sim}\rule{0pt}{0.4ex}}} %j connects to k, j right
\newcommand{\con}{\sim}							%j connects to k, indifferent of direction

%standard sets
\newcommand{\naturals}{\mathbb{N}}			%natural numbers
\newcommand{\integers}{\mathbb{Z}}			%integers
\newcommand{\rationals}{\mathbb{Q}}			%rational numbers
\newcommand{\reals}{\mathbb{R}}				%real numbers
\newcommand{\complex}{\mathbb{C}}			%complex numbers

%brackets and norms
\newcommand{\bracs}[1]{\left( #1 \right)}				%encloses input in brackets
\newcommand{\sqbracs}[1]{\left[ #1 \right]}				%encloses input in square brackets
\newcommand{\clbracs}[1]{\left\{ #1 \right\}}			%encloses input in curly bracers
\newcommand{\abs}[1]{\left\lvert #1 \right\rvert}					%absolute value
\newcommand{\norm}[1]{\lvert\lvert #1 \rvert\rvert}		%norm 

%function sets
\newcommand{\smooth}[1]{C^{\infty}\bracs{#1}}							%smooth functions
\newcommand{\ltwo}[2]{L^{2}\bracs{#1,\mathrm{d}#2}}						%general L^2 space
\newcommand{\gradSob}[2]{H^1_\mathrm{grad}\bracs{#1, \mathrm{d}#2}}		%gradient Sobolev space
\newcommand{\gradSobQM}[2]{H^1_{\qm, \mathrm{grad}}\bracs{#1, \mathrm{d}#2}} %gradient + i\qm Sobolev space
\newcommand{\ktgradSob}[2]{H^1_{\wavenumber,\qm,\mathrm{grad}}\bracs{#1, \mathrm{d}#2}}	%k,\qm-gradient Sobolev space
\newcommand{\curlSob}[2]{H^1_\mathrm{curl}\bracs{#1, \mathrm{d}#2}}		%curl Sobolev space
\newcommand{\tcurlSob}[2]{H^1_{\qm, \mathrm{curl}}\bracs{#1, \mathrm{d}#2}}		%curl + i\qm Sobolev space
\newcommand{\kcurlSob}[2]{H^1_{\wavenumber,\mathrm{curl}}\bracs{#1, \mathrm{d}#2}}	%k-curl Sobolev space
\newcommand{\ktcurlSob}[2]{H^1_{\wavenumber,\qm,\mathrm{curl}}\bracs{#1, \mathrm{d}#2}}	%k,\qm-curl Sobolev space
\newcommand{\ktcurlSobDivFree}[2]{\mathcal{H}^{\kt}\bracs{#1, \mathrm{d}#2}}					%k,\qm-curl, divergence-free Sobolev space
\newcommand{\supp}{\mathrm{supp}}										%support of a function

%grad and curl sets
\newcommand{\gradZero}[2]{\mathcal{G}_{ #1, \mathrm{d}#2}\bracs{0}}		%gradients of zero for domain #1 with measure #2
\newcommand{\kgradZero}[2]{\mathcal{G}_{ #1, \mathrm{d}#2}^{(\wavenumber)}\bracs{0}}	%k-gradients of zero for domain #1 with measure #2
\newcommand{\curlZero}[2]{\mathcal{C}_{ #1, \mathrm{d}#2}\bracs{0}}	%curls of zero for domain #1 with measure #2
\newcommand{\kcurlZero}[2]{\mathcal{C}_{ #1, \mathrm{d}#2}^{(\wavenumber)}\bracs{0}}	%k-curls of zero for domain #1 with measure #2

%derivatives and grad-like symbols
\newcommand{\diff}[2]{\dfrac{\mathrm{d}#1}{\mathrm{d}#2}}			%complete derivative d#1/d#2
\newcommand{\pdiff}[2]{\dfrac{\partial #1}{\partial #2}}			%partial derivative p#1/p#2
\newcommand{\ddiff}[2]{\dfrac{\mathrm{d}^2 #1}{\mathrm{d} {#2}^2}}	%2nd deriv
\newcommand{\grad}{\nabla}											%grad operator
\newcommand{\tgrad}{\nabla^{\qm}}									%grad operator with qm superscript
\newcommand{\kgrad}{\grad^{(\wavenumber)}}							%grad with wavenumber superscript
\newcommand{\ktgrad}{\grad^{\kt}}				%grad with wavenumber, qm superscript
\newcommand{\curl}[1]{\grad_{#1}\wedge}							%curl with subscript #1
\newcommand{\kcurl}[1]{\grad_{#1}^{(\wavenumber)}\wedge}			%k-curl with measure subscript #1
\newcommand{\ktcurl}[1]{\grad_{#1}^{\kt}\wedge}		%k,theta-curl with measure subscript #1
\newcommand{\laplacian}{\Delta}						%laplacian operator, can have subscripts attached

%displaying integrals
\newcommand{\integral}[3]{\int_{#1}#2 \ \mathrm{d}#3}			%integral, domain #1, integrand #2, measure #3
\newcommand{\md}{\mathrm{d}}									%differential d

%convergence
%\newcommand{\lconv}[1]{\xrightarrow{#1}}							%convergence with #1 above the rightarrow - requires mathtools
\newcommand{\lconv}[1]{\overset{#1}{\longrightarrow}}				%convergence with #1 above the rightarrow
\newcommand{\toInfty}[1]{ \ \text{as} \ #1 \rightarrow\infty}		%writes out "as #1 tends to infty" %maths commands, variables, and other packages

%labelling hacks
\newcommand\labelthis{\addtocounter{equation}{1}\tag{\theequation}}

\DeclareMathOperator*{\dw}{width}
\newcommand{\dirWidth}[2]{\dw_{#2}\bracs{#1}}
\newcommand{\tlambda}{\tilde{\lambda}}

%-------------------------------------------------------------------------
%DOCUMENT STARTS

\begin{document}

\chapter{Preliminary Results} \label{ch:PrelimResults}
This first lemma is just a reality check by me, for when I want to manipulate integrals with respect to the composite measure in parts.
For example, when I want to deduce that
\begin{align*}
	\integral{\ddom}{\abs{\phi_n}^2}{\dddmes} \rightarrow 0
	&\implies
	\begin{cases}
		\integral{\ddom}{\abs{\phi_n}^2}{\ddmes} \rightarrow 0, \\
		\integral{\ddom}{\abs{\phi_n}^2}{\lambda_2} \rightarrow 0.
	\end{cases}
\end{align*}
using the fact that by definition
\begin{align*}
	\integral{\ddom}{\abs{\phi_n}^2}{\dddmes} &= \integral{\ddom}{\abs{\phi_n}^2}{\lambda_2} + \integral{\ddom}{\abs{\phi_n}^2}{\ddmes}.
\end{align*}
\begin{lemma}
Suppose $\bracs{a_n}_{n\in\naturals}$ and $\bracs{b_n}_{n\in\naturals}$ are non-negative real sequences such that the sequence
\begin{align*}
	c_n = a_n + b_n \rightarrow 0 \toInfty{n}.
\end{align*} 
Then $a_n\rightarrow0, b_n\rightarrow0 \toInfty{n}$.
\end{lemma}
\begin{proof}
	Let $\eps>0$, then as $c_n\rightarrow0$ there exists some $N\in\naturals$ such that $\abs{c_n}<\eps$ for all $n\geq N$.
	Thus $\abs{a_n+b_n}\leq\eps$ for every $n\geq N$, and since $a_n,b_n\geq0 \ \forall n\in\naturals$ this implies that
	\begin{align*}
		\abs{a_n}<\eps, \quad \abs{b_n}<\eps, \quad \forall n\geq N,
	\end{align*}
	and so $a_n\rightarrow0, b_n\rightarrow0 \toInfty{n}$.
\end{proof}

The next result is to do with manipulating integrals over measurable sets.
\begin{lemma}
	Let $\ddom\subset\reals^2$ be bounded and $\nu$ a Borel measure on $\ddom$.
	Suppose that $f:\ddom\rightarrow\reals$ is measurable.
	Then
	\begin{enumerate}[(i)]
		\item If $A,B\subset\ddom$ are disjoint and measurable, then
		\begin{align*}
			\integral{A\cup B}{f}{\nu} &= \integral{A}{f}{\nu} + \integral{B}{f}{\nu}.
		\end{align*}
		\item If $E\subset\ddom$ is such that $\nu\bracs{E}=0$, then $\int_E f \md\nu = 0$.
		\item If $A,B\subset\ddom$ are measurable and $A\cap B = E$ where $E\subset\ddom$ is measurable with $\nu\bracs{E}=0$, then
		\begin{align*}
			\integral{A\cup B}{f}{\nu} &= \integral{A}{f}{\nu} + \integral{B}{f}{\nu}.
		\end{align*}
	\end{enumerate}
\end{lemma}
\begin{proof}
	\begin{enumerate}[(i)]
		\item This is just additivity of the integral; every function $f_n$ in an approximating sequence of simple functions for $f$ can be divided up between $A$ and $B$ (as they are disjoint), hence the separate integrals exist and their value is the same as the integral with the combined domain.
		\item If $s$ is a simple function with
		\begin{align*}
			s(x) &= \alpha_j, \quad x\in A_j, \quad \bigcup_{j=1}^{N} A_j=\ddom, \quad \alpha_j\in\reals,
		\end{align*}
		then we have that
		\begin{align*}
			\integral{E}{s}{\nu} &= \integral{\ddom}{s\charFunc{E}}{\nu} = \sum_{j=1}^{N} \alpha_j \nu{E\cap A_j} \\
			&= \sum_{j=1}^{N} \alpha_j\times0 = 0.
		\end{align*}
		Hence for an approximating sequence of simple functions $s_n\rightarrow f$,
		\begin{align*}
			\integral{E}{f}{\nu} &= \integral{\ddom}{f\charFunc{E}}{\nu} \\
			&= \lim_{n\rightarrow\infty}\integral{\ddom}{s_n\charFunc{E}}{\nu} = \lim_{n\rightarrow\infty}0 \\
			&= 0.
		\end{align*}
		\item This follows by the results of (i) and (ii); and as
		\begin{align*}
			A\cup B &= \bracs{A\setminus\bracs{A\cap B}} \cup \bracs{B\setminus\bracs{A\cap B}} \cup \bracs{A\cap B}, \\
			&= \bracs{A\setminus E} \cup \bracs{B\setminus E} \cup E.
		\end{align*}
		Thus
		\begin{align*}
			\integral{A\cup B}{f}{\nu} &= \integral{A\setminus E}{f}{\nu} + \integral{B\setminus E}{f}{\nu} + \integral{E}{f}{\nu},
		\end{align*}
		and as $\nu\bracs{E}=0$,
		\begin{align*}
			\integral{A\setminus E}{f}{\nu} + \integral{B\setminus E}{f}{\nu} + \integral{E}{f}{\nu}
			&= \integral{A\setminus E}{f}{\nu} + \integral{B\setminus E}{f}{\nu} + 2\integral{E}{f}{\nu} \\
			&= \integral{A\setminus E}{f}{\nu} + \integral{E}{f}{\nu} + \integral{B\setminus E}{f}{\nu} + \integral{E}{f}{\nu} \\
			&= \integral{A}{f}{\nu} + \integral{B}{f}{\nu}.
		\end{align*}
		Hence we are done.
	\end{enumerate}
\end{proof}

Next we need to setup some specific smooth functions that will be utilised in our analysis of sets of gradients of zero.
\begin{definition}[Smooth function $\xi$ with Gradient 1 at 0] \label{def:xiFuncDef}
	Let $\xi(t)$ be the smooth function on $\reals$ with
	\begin{align*}
		-1 \leq \xi(t) \leq 1 &\quad \forall t\in\sqbracs{-1,1}, \\
		\xi(t) = 0 &\quad \forall t\in\reals\setminus\sqbracs{-1,1}, \\
		\xi(0) = 0, &\quad \xi'(0) = 1.
	\end{align*}
	It is important to note that $\abs{\xi'(t)}$ is bounded by some constant $c_{\xi}>0$.
\end{definition}

\begin{definition}[Width in the Direction $e_\ddom$]
	Let $\ddom\subset\reals^2$ be bounded and let $\bracs{e_\ddom, n_\ddom}$ be a pair of orthonormal vectors in $\ddom$.
	For each $\alpha\in\reals$, let
	\begin{align*}
		L_\alpha &= \clbracs{x\in\ddom \ \vert \ x=\alpha n_\ddom + \beta e_\ddom, \beta\in\reals}
	\end{align*}
	be the line in $\ddom$ parallel to $e_\ddom$, a ``signed distance" $\alpha$ away from the ``diagonal" line $L_0$ which contains $e_\ddom$.
	Let $\lambda_\alpha$ be the singular measure supported on $L_\alpha$.
	Then the width of $\ddom$ in the direction $e_\ddom$ is
	\begin{align*}
		\dirWidth{\ddom}{e_\ddom} &= \sup_{\alpha\in\reals} \lambda_\alpha\bracs{L_\alpha}.
	\end{align*}
\end{definition}

Now we show that we can obtain convergence of the gradients of $\xi$ in definition \ref{def:xiFuncDef} in $\ltwo{\ddom}{\dddmes}$.
\begin{lemma} \label{lem:SmoothFuncConvCharFuncCompositeMeasure}
	Let $\ddom\subset\reals^2$ be bounded and $I\subset\ddom$ be a segment with unit vector $e_I$ and unit normal $n_I$.
	Let $\lambda_I$ be the singular measure supported on $I$, $\lambda_2$ the Lebesgue measure on $\ddom$, and $\dddmes = \lambda_2 + \lambda_I$ be the composite measure of $I$ in $\ddom$.
	Take $\charFunc{I}$ to be the characteristic function of $I$ and define
	\begin{align*}
		\phi_n(x) &= \recip{n}\xi\bracs{n x\cdot n_I},
	\end{align*}
	for each $n\in\naturals$ and $\xi$ as in definition \ref{def:xiFuncDef}.
	Note that $\phi_n\in\smooth{\ddom}$.
	Then
	\begin{align*}
		\phi_n \lconv{\ltwo{\ddom}{\dddmes}} 0, &\quad
		\grad\phi_n \lconv{\ltwo{\ddom}{\ddmes}^2} \charFunc{I}n_i.
	\end{align*}
\end{lemma}
\begin{proof}
	First we can compute
	\begin{align*}
		\grad\phi_n(x) &= n_I \xi'\bracs{n x\cdot n_I},
	\end{align*}
	and also should note that $x=\alpha e_I$ for some $\alpha\in\reals$ when $x\in I$, so
	\begin{align*}
		\xi\bracs{nx\cdot n_I} &= 0, \quad \xi'\bracs{nx\cdot n_I} = 1
	\end{align*}
	whenever $x\in I$.
	The proof is now a case of estimating the various integrals we need to consider;
	\begin{align*}
		\integral{\ddom}{\abs{\phi_n}^2}{\lambda_I}
		&= \recip{n^2}\integral{\ddom}{\abs{\xi\bracs{nx\cdot n_I}}^2}{\lambda_I} \\
		&= \recip{n^2}\integral{I}{\abs{\xi\bracs{nx\cdot n_I}}^2}{\lambda_I} \\
		&= \recip{n^2}\integral{I}{0}{\lambda_I} = 0, \\
		\integral{\ddom}{\abs{\phi_n}^2}{\lambda_2}
		&= \recip{n^2}\integral{\ddom}{\abs{\xi\bracs{nx\cdot n_I}}^2}{\lambda_2} \\
		&\leq \recip{n^2}\integral{\ddom}{}{\lambda_2} \\
		&= \recip{n^2}\lambda_2\bracs{\ddom} \rightarrow 0 \toInfty{n}.
	\end{align*}
	Thus,
	\begin{align*}
		\integral{\ddom}{\abs{\phi_n}^2}{\dddmes} &= \integral{\ddom}{\abs{\phi_n}^2}{\lambda_2} + \integral{\ddom}{\abs{\phi_n}^2}{\lambda_I} \\
		&\rightarrow 0 \toInfty{n},
	\end{align*}
	so $\phi_n\rightarrow0$ in $\ltwo{\ddom}{\dddmes}$.
	Additionally,
	\begin{align*}
		\integral{\ddom}{\abs{\grad\phi_n - \charFunc{I}n_I}^2}{\lambda_I}
		&= \integral{I}{\abs{\xi'\bracs{nx\cdot n_I}n_I - n_I}^2}{\lambda_I} \\
		&= \integral{I}{\abs{\xi'\bracs{0} - 1}^2}{\lambda_I}
		= \integral{I}{0}{\lambda_I} = 0, \\
		\integral{\ddom}{\abs{\grad\phi_n - \charFunc{I}n_I}^2}{\lambda_2}
		&= \integral{\ddom}{\abs{\xi'\bracs{nx\cdot n_I}}^2}{\lambda_2} &\quad(\charFunc{I}=0 \ \lambda_2-\text{a.e.}) \\
		&= \integral{B_n}{\abs{\xi'\bracs{nx\cdot n_I}}^2}{\lambda_2},
	\end{align*}
	where
	\begin{align*}
		B_n &= \supp\bracs{\xi'\bracs{nx\cdot n_I}}
		= \ddom \cap \bracs{\clbracs{\alpha e_I \ \vert \ \alpha\in\reals} \times \clbracs{\alpha n_I \ \vert \ \alpha\in\sqbracs{-\recip{n},\recip{n}}}}.
	\end{align*}
	Additionally,
	\begin{align*}
		\integral{B_n}{\abs{\xi'\bracs{nx\cdot n_I}}^2}{\lambda_2}
		&\leq \integral{B_n}{c_{\xi}^2}{\lambda_2} \\
		&= 2c_{\xi}^2\dirWidth{\ddom}{e_I}\times\recip{n} \\
		&\rightarrow 0 \toInfty{n}.
	\end{align*}
	Thus we have that
	\begin{align*}
		\integral{\ddom}{\abs{\grad\phi_n - \charFunc{I}n_I}^2}{\dddmes}
		&= \integral{\ddom}{\abs{\grad\phi_n - \charFunc{I}n_I}^2}{\lambda_2} + \integral{\ddom}{\abs{\grad\phi_n - \charFunc{I}n_I}^2}{\lambda_I} \\
		&\rightarrow0 \toInfty{n},
	\end{align*}
	so $\grad\phi_n\rightarrow \charFunc{I}n_I$ in $\ltwo{\ddom}{\dddmes}^2$ too, and we are done.
\end{proof}

Also we will need to use the function $\eta^n_j$ as defined in the ScalarEqns chapter of the confirmation report, and we will quickly prove this result:
\begin{lemma} \label{lem:EtaConvCompositeMeasureCase}
	Let $\ddom\subset\reals^2$ and $\graph=\bracs{V,E}$ be an embedded graph in $\ddom$.
	Let $\ddmes$ be the singular measure supported on $\graph$ and $\dddmes$ the composite measure of $\graph$ in $\ddom$.
	Take $\eta_j^n$ as in that definition from ScalarEqns.
	Then
	\begin{align*}
		\eta_j^n \lconv{\ltwo{\ddom}{\dddmes}} 1.
	\end{align*}
\end{lemma}
\begin{proof}
	We know that $\eta_j^n\rightarrow 1$ in $\ltwo{\ddom}{\ddmes}$ by the result in the confirmation report, so we just need to show that $\eta_j^n\rightarrow 1$ in $\ltwo{\ddom}{\lambda_2}$ too.
	But we know that
	\begin{align*}
		\integral{\ddom}{\abs{\eta_j^n - 1}^2}{\lambda_2}
		&\leq \integral{B_{2/n}\bracs{v_j}}{}{\lambda_2} = \frac{4\pi}{n^2} \\
		&\rightarrow 0 \toInfty{n},
	\end{align*}
	because
	\begin{align*}
		\supp\abs{\eta_j^n - 1}^2 \subset B_{2/n}\bracs{v_j}, \quad\text{and } \max\abs{\eta_j^n - 1}^2 = 1.
	\end{align*}
\end{proof}

\chapter{Gradients of Zero}

This first result is colloquially summarised as ``being a gradient of zero with respect to the composite measure implies that you are also a gradient of zero with respect to each of the individual measures".
\begin{lemma} \label{lem:GradZeroCompositeImpliesGradZeroOthers}
	Let $\ddom\subset\reals^2$ and $\graph=\bracs{V,E}$ be an embedded graph in $\ddom$.
	Let $\ddmes$ be the singular measure supported on $\graph$ and $\dddmes$ the composite measure of $\graph$ in $\ddom$.
	Then we have that
	\begin{align*}
		g\in\gradZero{\ddom}{\dddmes} &\implies g\in\gradZero{\ddom}{\lambda_2}\cap\gradZero{\ddom}{\ddmes}.
	\end{align*}
\end{lemma}
\begin{proof}
	As $g\in\gradZero{\ddom}{\dddmes}$ there exists a sequence of smooth functions $\phi_n$ such that
	\begin{align*}
		\phi_n \lconv{\ltwo{\ddom}{\dddmes}} 0, &\quad \grad\phi_n \lconv{\ltwo{\ddom}{\dddmes}^2} g.
	\end{align*}
	But then we have that
	\begin{align*}
		\integral{\ddom}{\abs{\phi_n}^2}{\dddmes}
		&= \integral{\ddom}{\abs{\phi_n}^2}{\lambda_2} + \integral{\ddom}{\abs{\phi_n}^2}{\ddmes} \rightarrow 0, \\
		\implies \integral{\ddom}{\abs{\phi_n}^2}{\lambda_2}\rightarrow0, &\quad \integral{\ddom}{\abs{\phi_n}^2}{\ddmes} \rightarrow 0.
	\end{align*}
	And similarly
	\begin{align*}
		\integral{\ddom}{\abs{\grad\phi_n-g}^2}{\dddmes}
		&= \integral{\ddom}{\abs{\grad\phi_n-g}^2}{\lambda_2} + \integral{\ddom}{\abs{\grad\phi_n-g}^2}{\ddmes} \rightarrow 0, \\
		\implies \integral{\ddom}{\abs{\grad\phi_n-g}^2}{\lambda_2}\rightarrow0, &\quad \integral{\ddom}{\abs{\grad\phi_n-g}^2}{\ddmes} \rightarrow 0.
	\end{align*}
	Which means that
	\begin{align*}
		\phi_n \lconv{\ltwo{\ddom}{\lambda_2}} 0, &\quad \grad\phi_n \lconv{\ltwo{\ddom}{\lambda_2}^2} g, \\
		\phi_n \lconv{\ltwo{\ddom}{\ddmes}} 0, &\quad \grad\phi_n \lconv{\ltwo{\ddom}{\ddmes}^2} g,
	\end{align*}
	so $g\in\gradZero{\ddom}{\lambda_2}$ and $g\in\gradZero{\ddom}{\ddmes}$.
\end{proof}
Proving the reverse inclusion requires us to work in stages however, as we will need some method of ``stitching" together approximating sequences from $\gradZero{\ddom}{\lambda_2}$ and $\gradZero{\ddom}{\ddmes}$ to obtain an approximating sequence showing membership of $\gradZero{ddom}{\dddmes}$.

\begin{convention}[Notation for Composite Measures] \label{conv:CompMeasNotation}
	If $\nu$ is a singular measure supporting a graph $\graph$ (including a single-edge graph, or segment), embedded in $\ddom\subset\reals^2$, use an overhead tilde to denote the composite measure of $\graph$ in $\ddom$, so
	\begin{align*}
		\tilde{\nu}(B) &= \lambda_2(B) + \nu(B).
	\end{align*}
\end{convention}

\begin{lemma}[Sufficient conditions for membership of $\gradZero{\ddom}{\tlambda_I}$] \label{lem:SuffCondForGradZeroSegmentCompositeMeasure}
	Let $\ddom\subset\reals^2$ and $I$ be a segment in $\ddom$ with singular measure $\lambda_I$.
	Suppose that $g$ satisfies
	\begin{align*}
		g\cdot e_I =0 &\quad\tlambda_I\text{-a.e.}, \\
		\bracs{g\cdot n_I}\vert_I &\in\ltwo{\ddom}{\lambda_I}, \\
		\bracs{g\cdot n_I}\vert_{\ddom\setminus I} =0 &\quad\lambda_2\text{-a.e.}
	\end{align*}
	Then
	\begin{align*}
		g\in\gradZero{\ddom}{\tlambda_I}.
	\end{align*}
\end{lemma}
\begin{proof}
	We assume first that $g\cdot n_I$ is smooth, as we can then apply a density argument to extend the result to all functions in $\ltwo{\ddom}{\lambda_I}$.
	Set
	\begin{align*}
		\phi_n &= \recip{n}\xi\bracs{nx\cdot n_I}\bracs{g\cdot n_I}
	\end{align*}
	for $\xi$ as in definition \ref{def:xiFuncDef}, so $\phi_n\in\smooth{\ddom}$ for each $n\in\naturals$.
	Then by lemma \ref{lem:SmoothFuncConvCharFuncCompositeMeasure} we know that
	\begin{align*}
		\phi_n &\lconv{\ltwo{\ddom}{\tlambda_I}} 0, \\
		\grad\phi_n &= \underbrace{n_I\xi'\bracs{nx\cdot n_I}\bracs{g\cdot n_I}}_{\rightarrow \charFunc{I}n_I} + \underbrace{n^{-1}\xi\bracs{nx\cdot n_I}\grad\bracs{g\cdot n_I}}_{\rightarrow0} \\
		&\rightarrow n_I\bracs{g\cdot n_I}\vert_I = g\vert_I &(\text{as } g\cdot e_I=0) \\
		&= g &(\text{as } \bracs{g\cdot n_I}\vert_{\ddom\setminus I} =0 &\quad\lambda_2\text{-a.e.}).
	\end{align*}
	Hence we have that $g\in\gradZero{\ddom}{\tlambda_I}$.
\end{proof}

Now we have necessary and sufficient conditions for membership of $\gradZero{\ddom}{tlambda_I}$.
\begin{prop}[Characterisation of $\gradZero{\ddom}{tlambda_I}$] \label{prop:CharGradZeroSegmentCompositeMeasure}
	Let $\ddom\subset\reals^2$ be bounded, and $I\subset\ddom$ be a segment with unit vector $e_I$ and unit normal $n_I$.
	Let $\lambda_I$ be the singular measure supported on $I$.
	Then the following are all equivalent:
	\begin{enumerate}[(i)]
		\item $g\in\gradZero{\ddom}{\tlambda_I}$,
		\item $g\in\gradZero{\ddom}{\lambda_2} \cap \gradZero{\ddom}{\lambda_I}$,
		\item $g$ is such that
		\begin{align*}
			g\cdot e_I =0 &\quad\tlambda_I\text{-a.e.}, \\
			\bracs{g\cdot n_I}\vert_I &\in\ltwo{\ddom}{\lambda_I}, \\
			\bracs{g\cdot n_I}\vert_{\ddom\setminus I} =0 &\quad\lambda_2\text{-a.e.}
		\end{align*}
	\end{enumerate}
\end{prop}
\begin{proof}
	(i) $\implies$ (ii) is a special case of lemma \ref{lem:GradZeroCompositeImpliesGradZeroOthers}. \newline
	(ii) $\implies$ (iii) is a consequence of characterising $\gradZero{\ddom}{\lambda_I}$ as in ScalarEqns, combined with the fact that $g\in\gradZero{\ddom}{\lambda_2}$ implies $g=0 \ \lambda_2-$a.e. \newline
	(iii) $\implies$ (i) is dealt with by lemma \ref{lem:SuffCondForGradZeroSegmentCompositeMeasure}.
\end{proof}

Now we need an extension lemma to bring gradients of zero with respect to $\tlambda_I$ into context as gradients of zero with respect to $\dddmes$.
\begin{lemma}[Extension Lemma for $\gradZero{\ddom}{tlambda_I}$] \label{lem:ExtLemmaGradZeroCompositeMeasure}
	Let $G=\bracs{V,E}$ be an embedded graph in $\ddom\subset\reals^2$, and suppose $g\in\gradZero{\ddom}{\tlambda_{jk}}$ is such that $g\cdot n_{jk}=0$ on $\graph\setminus I^n_{jk}$.
	See ScalarEqns chapter of confirmation report for a reminder of $I^n_{jk}, \chi^n_{jk}$, etc.
	Then
	\begin{align*}
		g\in\gradZero{\ddom}{\dddmes}.
	\end{align*}
\end{lemma}
\begin{proof}
	We can find a smooth sequence $\phi_l$ such that
	\begin{align*}
		\phi_l \lconv{\ltwo{\ddom}{\tlambda_{jk}}} 0, \quad
		\grad\phi_l \lconv{\ltwo{\ddom}{\tlambda_{jk}}^2} g.
	\end{align*}
	Set $\varphi_l = \chi^n_{jk}\phi_l$ for each $l\in\naturals$.
	Then we have that
	\begin{align*}
		\integral{\ddom}{\abs{\varphi_l}^2}{\dddmes}
		&= \integral{\ddom}{\abs{\varphi_l}^2}{\lambda_2} + \integral{\ddom}{\abs{\varphi_l}^2}{\ddmes} \\
		&= \integral{\ddom}{\abs{\chi^n_{jk}\phi_l}^2}{\lambda_2} + \integral{I_{jk}}{\abs{\chi^n_{jk}\phi_l}^2}{\lambda_{jk}} \\
		&\leq \integral{\ddom}{\abs{\phi_l}^2}{\lambda_2} + \integral{\ddom}{\abs{\phi_l}^2}{\lambda_{jk}} \\
		&= \integral{\ddom}{\abs{\phi_l}^2}{\tlambda_{jk}} \rightarrow 0.
	\end{align*}
	Additionally,
	\begin{align*}
		\recip{2}\integral{\ddom}{\abs{\grad\varphi_l - g}^2}{\dddmes}
		&= \recip{2}\integral{\ddom}{\abs{\chi^n_{jk}\grad\phi_l + \phi_l\grad\chi^n_{jk} - g}^2}{\dddmes} \\
		&\leq \integral{\ddom}{\abs{\grad\phi_l - g}^2}{\tlambda_{jk}} + \sup\abs{\grad\chi^n_{jk}}^2\integral{\ddom}{\abs{\phi_l}^2}{\tlambda_{jk}} \\
		&\rightarrow 0+0 = 0.
	\end{align*}
	[N.B: Full working on paper, dated 15/4/20].
	Hence, we have found an appropriate approximating sequence $\varphi_l$, demonstrating that $g\in\gradZero{\ddom}{\dddmes}$.
\end{proof}	

And we also will want this corollary which follows immediately from the Extension lemma \ref{lem:ExtLemmaGradZeroCompositeMeasure}.
\begin{cory}
	Let $G=\bracs{V,E}$ be an embedded graph in $\ddom\subset\reals^2$, and suppose that for some edge $I_{jk}\in E$ the function $g$ satisfies
	\begin{align*}
		g\cdot e_{jk} =0 &\quad\tlambda_{jk}\text{-a.e.}, \\
		\bracs{g\cdot n_{jk}}\vert_{I_{jk}} &\in\ltwo{\ddom}{\lambda_{jk}}, \\
		\bracs{g\cdot n_{jk}}\vert_{\ddom\setminus I^n_{jk}} =0 &\quad\lambda_2\text{-a.e.}
	\end{align*}
	Then
	\begin{align*}
		g\in\gradZero{\ddom}{\dddmes}.
	\end{align*}
\end{cory}
\begin{proof}
	These properties ensure that $g\in\gradZero{\ddom}{\tlambda_{jk}}$, and hence we can then apply lemma \ref{lem:ExtLemmaGradZeroCompositeMeasure}.
\end{proof}

We can now characterise the set $\gradZero{\ddom}{\dddmes}$.
\begin{theorem}[Characterisation of $\gradZero{\ddom}{\dddmes}$] \label{thm:CharGradZeroGraphCompositeMeasure}
	Let $G=\bracs{V,E}$ be an embedded graph in $\ddom\subset\reals^2$, with singular measure $\ddmes$.
	Then the following are all equivalent:
	\begin{enumerate}[(i)]
		\item $g\in\gradZero{\ddom}{\dddmes}$,
		\item $g\in\gradZero{\ddom}{\lambda_2}$, and $g\in\gradZero{\ddom}{\lambda_{jk}} \ \forall I_{jk}\in E$,
		\item $g$ is such that
		\begin{align*}
			g\cdot e_{jk} = 0 &\quad \forall I_{jk}\in E, \\
			\bracs{g\cdot n_{jk}}\vert_{I_{jk}} \in \ltwo{\ddom}{\lambda_{jk}} &\quad \forall I_{jk}\in E, \\
			g\cdot n_{jk} = 0 \ \lambda_2\text{-a.e.} &\quad \forall I_{jk}\in E.
		\end{align*}
	\end{enumerate}
\end{theorem}
\begin{proof}
	(i) $\implies$ (ii) is the result of lemma \ref{lem:GradZeroCompositeImpliesGradZeroOthers}, and characterisation of $\gradZero{\ddom}{\ddmes}$. \newline
	(ii) $\implies$ (iii) is the result of characterisation of $\gradZero{\ddom}{\lambda_{jk}}$ combined with $g\in\gradZero{\ddom}{\lambda_2}$. \newline
	(iii) $\implies$ (i): define, for each $n\in\naturals$,
	\begin{align*}
		g_n &= \sum_{v_j\in V}\sum_{j\conLeft k} \eta^n_j \eta^n_k g_{jk}.
	\end{align*}
	Each $\eta^n_j \eta^n_k g_{jk}$ satisfies the Extension lemma \ref{lem:ExtLemmaGradZeroCompositeMeasure}, and thus is an element of $\gradZero{\ddom}{\dddmes}$.
	As $\gradZero{\ddom}{\dddmes}$ is a linear subspace of $\ltwo{\ddom}{\dddmes}^2$, we have that $g_n\in\gradZero{\ddom}{\dddmes}$ for each $n\in\naturals$.
	Furthermore, as $\gradZero{\ddom}{\dddmes}$ is closed, $\lim_{n\rightarrow\infty}g_n\in\gradZero{\ddom}{\dddmes}$ if it exists at all.
	But because $\eta^n_j \rightarrow 1$ in $\ltwo{\ddom}{\dddmes}$ (lemma \ref{lem:EtaConvCompositeMeasureCase}), we can conclude that $g_n\rightarrow g \toInfty{n}$, and so we are done.
\end{proof}

\end{document}