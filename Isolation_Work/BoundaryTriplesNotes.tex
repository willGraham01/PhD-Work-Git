\documentclass[11pt]{report}

\usepackage{url}
\usepackage[margin=2.5cm]{geometry} % See geometry.pdf to learn the layout options. There are lots.
\geometry{a4paper} %or letterpaper or a5paper or ...

%for figures and graphics
\usepackage{graphicx}
\usepackage{subcaption} %allows subfigures
\usepackage[bottom]{footmisc} %footnotes go below figures
\usepackage{tikz}
%\usepackage{parskip} %adds line space between paragraphs by default
\usepackage{enumerate} %allows lower case roman numerials in enumerate environments

\DeclareGraphicsRule{.tif}{png}{.png}{`convert #1 `dirname #1`/`basename #1 .tif`.png}
\graphicspath{{../Diagrams/Diagram_PDFs/} {../Diagrams/Numerical_Results/}}

%\input imports all commands from the target files
%The idea behind this file is that it will be used to store all the maths-related macros that I concoct; so that I can import all the commands by \input{this file} in the preamble of any file that I want to use them in.
%This should make the top-level files look a lot cleaner, and the preamble much shorter!

\usepackage{amssymb}
\usepackage{amsmath}
%\usepackage{mathtools}

%theorems and lemma etc setup using amsthm
\usepackage{amsthm}
\theoremstyle{definition}
\newtheorem{definition}{Definition}[section]
\theoremstyle{plain}
\newtheorem{theorem}{Theorem}[section]
\theoremstyle{plain}
\newtheorem{lemma}[theorem]{Lemma}
\theoremstyle{plain}
\newtheorem{prop}[theorem]{Proposition}
\theoremstyle{plain}
\newtheorem{cory}[theorem]{Corollary}
\theoremstyle{definition}
\newtheorem{convention}[theorem]{Convention}
\theoremstyle{definition}
\newtheorem{assumption}[theorem]{Assumption}
\theoremstyle{definition}
\newtheorem{conjecture}[theorem]{Conjecture}

\allowdisplaybreaks %allows equations in the same align environment to split over multiple pages.

%tstk always need be there
\newcommand{\tstk}[1]{\textbf{#1} \newline}

%this adds extra functionality to pmatrix, vmatrix, bmatrix etc by allowing you to pass an optional argument in [FACTOR] to multiply the default spacing between elements by FACTOR
\makeatletter
\renewcommand*\env@matrix[1][\arraystretch]{%
  \edef\arraystretch{#1}%
  \hskip -\arraycolsep
  \let\@ifnextchar\new@ifnextchar
  \array{*\c@MaxMatrixCols c}
  }
\makeatother

%begin the macros via newcommand. Try to group them up reasonably!

%notation and variable use throughout the file
\renewcommand{\vec}[1]{\mathbf{#1}}				%vectors are bold, not overline arrow
\newcommand{\recip}[1]{\frac{1}{#1}}			%fast reciprocal as a fraction
\newcommand{\interval}[1]{\sqbracs{0,\abs{#1}}}	%creates the closed interval from 0 to the length of the input #1, denoted by absolute value
\newcommand{\eps}{\varepsilon}					%pretty epsilons
\newcommand{\charFunc}[1]{\mathcal{I}_{#1}}%{\mathbb{1}_{#1}}		%characteristic function of a set

\newcommand{\dddom}{\widetilde{\Omega}}			%3D domain notation
\newcommand{\ddom}{\Omega}						%2D domain notation
\newcommand{\dddmes}{\widetilde{\mu}}			%3D measure
\newcommand{\ddmes}{\mu}						%2D measure

\newcommand{\graph}{\mathbb{G}}					%graph variable
\newcommand{\vertSet}{\mathcal{V}}					%set of vertices rather than big V
\newcommand{\edgeSet}{\mathcal{E}}					%set of edges rather than large E, \graph = (V,E)
\newcommand{\wavenumber}{\kappa}				%fourier variable or wavenumber, not to confuse with jk subscripts!
\newcommand{\qm}{\theta}						%quasi-momentum parameter
\newcommand{\kt}{\bracs{\wavenumber, \qm}}		%(k, theta) pair
\newcommand{\dmap}{\Gamma_0}					%Dirichlet map
\newcommand{\nmap}{\Gamma_1}					%Neumann map
\newcommand{\effFreq}{\Lambda}					%Effective frequency sqrt(w^2-\wavenumber^2)
\newcommand{\conLeft}{\stackrel{\rightarrow}{\smash{\sim}\rule{0pt}{0.4ex}}} %j connects to k, j left
\newcommand{\conRight}{\stackrel{\leftarrow}{\smash{\sim}\rule{0pt}{0.4ex}}} %j connects to k, j right
\newcommand{\con}{\sim}							%j connects to k, indifferent of direction

%standard sets
\newcommand{\naturals}{\mathbb{N}}			%natural numbers
\newcommand{\integers}{\mathbb{Z}}			%integers
\newcommand{\rationals}{\mathbb{Q}}			%rational numbers
\newcommand{\reals}{\mathbb{R}}				%real numbers
\newcommand{\complex}{\mathbb{C}}			%complex numbers

%brackets and norms
\newcommand{\bracs}[1]{\left( #1 \right)}				%encloses input in brackets
\newcommand{\sqbracs}[1]{\left[ #1 \right]}				%encloses input in square brackets
\newcommand{\clbracs}[1]{\left\{ #1 \right\}}			%encloses input in curly bracers
\newcommand{\abs}[1]{\left\lvert #1 \right\rvert}					%absolute value
\newcommand{\norm}[1]{\lvert\lvert #1 \rvert\rvert}		%norm 

%function sets
\newcommand{\smooth}[1]{C^{\infty}\bracs{#1}}							%smooth functions
\newcommand{\ltwo}[2]{L^{2}\bracs{#1,\mathrm{d}#2}}						%general L^2 space
\newcommand{\gradSob}[2]{H^1_\mathrm{grad}\bracs{#1, \mathrm{d}#2}}		%gradient Sobolev space
\newcommand{\gradSobQM}[2]{H^1_{\qm, \mathrm{grad}}\bracs{#1, \mathrm{d}#2}} %gradient + i\qm Sobolev space
\newcommand{\ktgradSob}[2]{H^1_{\wavenumber,\qm,\mathrm{grad}}\bracs{#1, \mathrm{d}#2}}	%k,\qm-gradient Sobolev space
\newcommand{\curlSob}[2]{H^1_\mathrm{curl}\bracs{#1, \mathrm{d}#2}}		%curl Sobolev space
\newcommand{\tcurlSob}[2]{H^1_{\qm, \mathrm{curl}}\bracs{#1, \mathrm{d}#2}}		%curl + i\qm Sobolev space
\newcommand{\kcurlSob}[2]{H^1_{\wavenumber,\mathrm{curl}}\bracs{#1, \mathrm{d}#2}}	%k-curl Sobolev space
\newcommand{\ktcurlSob}[2]{H^1_{\wavenumber,\qm,\mathrm{curl}}\bracs{#1, \mathrm{d}#2}}	%k,\qm-curl Sobolev space
\newcommand{\ktcurlSobDivFree}[2]{\mathcal{H}^{\kt}\bracs{#1, \mathrm{d}#2}}					%k,\qm-curl, divergence-free Sobolev space
\newcommand{\supp}{\mathrm{supp}}										%support of a function

%grad and curl sets
\newcommand{\gradZero}[2]{\mathcal{G}_{ #1, \mathrm{d}#2}\bracs{0}}		%gradients of zero for domain #1 with measure #2
\newcommand{\kgradZero}[2]{\mathcal{G}_{ #1, \mathrm{d}#2}^{(\wavenumber)}\bracs{0}}	%k-gradients of zero for domain #1 with measure #2
\newcommand{\curlZero}[2]{\mathcal{C}_{ #1, \mathrm{d}#2}\bracs{0}}	%curls of zero for domain #1 with measure #2
\newcommand{\kcurlZero}[2]{\mathcal{C}_{ #1, \mathrm{d}#2}^{(\wavenumber)}\bracs{0}}	%k-curls of zero for domain #1 with measure #2

%derivatives and grad-like symbols
\newcommand{\diff}[2]{\dfrac{\mathrm{d}#1}{\mathrm{d}#2}}			%complete derivative d#1/d#2
\newcommand{\pdiff}[2]{\dfrac{\partial #1}{\partial #2}}			%partial derivative p#1/p#2
\newcommand{\ddiff}[2]{\dfrac{\mathrm{d}^2 #1}{\mathrm{d} {#2}^2}}	%2nd deriv
\newcommand{\grad}{\nabla}											%grad operator
\newcommand{\tgrad}{\nabla^{\qm}}									%grad operator with qm superscript
\newcommand{\kgrad}{\grad^{(\wavenumber)}}							%grad with wavenumber superscript
\newcommand{\ktgrad}{\grad^{\kt}}				%grad with wavenumber, qm superscript
\newcommand{\curl}[1]{\grad_{#1}\wedge}							%curl with subscript #1
\newcommand{\kcurl}[1]{\grad_{#1}^{(\wavenumber)}\wedge}			%k-curl with measure subscript #1
\newcommand{\ktcurl}[1]{\grad_{#1}^{\kt}\wedge}		%k,theta-curl with measure subscript #1
\newcommand{\laplacian}{\Delta}						%laplacian operator, can have subscripts attached

%displaying integrals
\newcommand{\integral}[3]{\int_{#1}#2 \ \mathrm{d}#3}			%integral, domain #1, integrand #2, measure #3
\newcommand{\md}{\mathrm{d}}									%differential d

%convergence
%\newcommand{\lconv}[1]{\xrightarrow{#1}}							%convergence with #1 above the rightarrow - requires mathtools
\newcommand{\lconv}[1]{\overset{#1}{\longrightarrow}}				%convergence with #1 above the rightarrow
\newcommand{\toInfty}[1]{ \ \text{as} \ #1 \rightarrow\infty}		%writes out "as #1 tends to infty" %maths commands, variables, and other packages

%labelling hacks
\newcommand\labelthis{\addtocounter{equation}{1}\tag{\theequation}}
\newcommand{\dom}[1]{\mathrm{dom}\bracs{#1}}
\newcommand{\amin}{A_{\mathrm{min}}}
\newcommand{\amax}{A_{\mathrm{max}}}
\newcommand{\ab}{A_B}
\newcommand{\hilb}{\mathcal{H}}

%-------------------------------------------------------------------------
%DOCUMENT STARTS

\begin{document}

\section{Notes on Boundary Triples, Operator Extensions, and the $M$-Operator}

Let $H$ be a Hilbert space with inner product $\ip{\cdot}{\cdot}_H$.
Let $A$ be an operator on $H$, with $A$ not necessarily bounded.
\begin{definition}[Densely defined]
	The operator $A$ is densely defined if $\dom{A}$ is a dense subset of $H$.
\end{definition}
\begin{definition}[(Operator) Graph]
	The graph $G(A)$ of the operator $A$ is the set 
	\begin{align*}
		\clbracs{(x,Ax) \setVert x\in\dom{A}}.
	\end{align*}
\end{definition}
\begin{definition}[Symmetric]
	The operator $A$ is symmetric if $\ip{Ax}{y}_H = \ip{x}{Ay}_H$ for every $x,y\in\dom{A}$.
\end{definition}
\begin{definition}[Deficiency Indices]
	The deficiency indices of $A$ are the dimensions of the orthogonal compliments of its range and domain, that is
	\begin{align*}
		n_+ := \mathrm{dim}\bracs{\dom{A}^\perp},  \qquad
		n_- := \mathrm{dim}\bracs{\mathrm{range}\bracs{A}^\perp}.
	\end{align*}
\end{definition}
\begin{definition}[Extension]
	An operator $B$ is an extension of $A$, written $A\subset B$, if $G(A)\subset G(B)$.
\end{definition}
\begin{definition}[Adjoint, Self-adjoint]
	The adjoint of $A$, written $A^*$, is the operator with domain
	\begin{align*}
		\dom{A^*} &= \clbracs{y\in H \setVert \exists z\in H \text{ s.t. } \ip{Ax}{y}_H = \ip{x}{z} },
	\end{align*}
	with $A^*y := z$. \newline
	A is self-adjoint if $G(A) = G\bracs{A^*}$, that is if $\dom{A}=\dom{A^*}$ and (the actions of) $A=A^*$.
\end{definition}
\begin{definition}[Simple]
	$A$ is simple if there does not exist a reducing subspace $H_0\subset H$ such that the restriction $A\vert_{H_0}$ is self-adjoint in $H_0$.
\end{definition}
Note that if $A$ is bounded, things get a lot nicer (think 4th year functional analysis course nicer).
However, some results that are true in general are:
\begin{itemize}
	\item $A$ is self-adjoint $\implies$ $A$ is symmetric.
	\item $A$ is symmetric with $\dom{A}=\dom{A^*}$ $\implies$ $A$ is self-adjoint.
	\item In general, $A$ is symmetric $\Leftrightarrow$ $A \subset A^*$.
\end{itemize}
\begin{definition}[Closure]
	If the closure of $G(A)$ is the graph of another operator $B$ on $H$, then the closure of $A$ is $B$.
	Note that by definition, $B$ is an extension of $A$.
\end{definition}
\begin{itemize}
	\item A symmetric operator $A$ is always closable --- that is, $A$ has some closure $B$.
\end{itemize}
\begin{definition}[Essentially self-adjoint]
	A symmetric operator $A$ is essentially self-adjoint if its closure is self-adjoint.
\end{definition}
\begin{itemize}
	\item A symmetric operator has a unique self-adjoint extension if and only if $n_+=n_-$.
\end{itemize}

We now have the terminology to describe boundary triples and the $M$-function.
From here on out, the setups of the definitions are carried through, unless explicitly stated otherwise.
\begin{definition}[Boundary Triple, Dirichlet and Neumann map, (Weyl-Titchmarsh) $M$-function]
	Let $\amin$ be a symmetric, densely defined operator on $H$ with equal deficiency indices (so $\amin$ possesses a unique self-adjoint extension) and set $\amax = \amin^*$.
	Suppose that $\hilb$ is a separable Hilbert space with inner product $\ip{\cdot}{\cdot}_\hilb$, and that $\dmap, \nmap:\dom{\amax}\rightarrow\hilb$ are linear maps.
	Then the triple $\bracs{\hilb, \dmap, \nmap}$ is a boundary triple for the operator $\amax$ if
	\begin{enumerate}[(i)]
		\item The ``Green's identity"
		\begin{align*}
			\ip{\amax u}{v}_H - \ip{u}{\amax v}_H &= \ip{\nmap u}{\dmap v}_\hilb - \ip{\dmap u}{\nmap v}_\hilb,
		\end{align*}
		holds $\forall u,v\in\dom{\amax}$.
		\item The map $u\in\dom{\amax}, \ u\mapsto\bracs{\dmap u, \nmap u}$ is surjective.
		That is, $\forall u\in\dom{\amax}$ there exist $U_0, U_1\in\hilb$ such that $\dmap u = U_0$, $\nmap u = U_1$.
	\end{enumerate}
	We will refer to the map $\dmap$ as the Dirichlet map and $\nmap$ as the Neumann map. \newline
	The operator-valued function $M=M(z)$ defined by
	\begin{align*}
		M(x)\dmap u &= \nmap u, \quad u\in\mathrm{ker}\bracs{\amax - z}, \ z\in\complex_+\cup\complex_-,
	\end{align*}
	is called the (Weyl-Titchmarsh) $M$-function of $\amax$ with respect to the triple $\bracs{\hilb, \dmap, \nmap}$.
\end{definition}
\begin{definition}[Almost-solvable]
	A non-trivial extension $\ab$ of $\amin$ such that $\amin\subset\ab\subset\amax$ is called almost-solvable if there exists a boundary triple $\bracs{\hilb, \dmap, \nmap}$ for $\amax$ and a bounded linear operator $B$ on $\hilb$ such that
	\begin{align*}
		\forall u\in\dom{\amax}, \quad u\in\dom{\ab} \Leftrightarrow \nmap u = B\dmap u.
	\end{align*}
\end{definition}
\begin{itemize}
	\item If $\ab$ is an almost-solvable extension of a simple symmetric operator $\amin$, then we have that $z_0$ is in the spectrum of $\ab$ if and only if $\bracs{M(z)-B}^{-1}$ does not admit analytic continuation into $z_0$.
\end{itemize}
\end{document}