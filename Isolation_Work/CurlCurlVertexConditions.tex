\documentclass[11pt]{report}

\usepackage{url}
\usepackage[margin=2.5cm]{geometry} % See geometry.pdf to learn the layout options. There are lots.
\geometry{a4paper} %or letterpaper or a5paper or ...

%for figures and graphics
\usepackage{graphicx}
\usepackage{subcaption} %allows subfigures
\usepackage[bottom]{footmisc} %footnotes go below figures
%\usepackage{parskip} %adds line space between paragraphs by default
\usepackage{enumerate} %allows lower case roman numerials in enumerate environments

\DeclareGraphicsRule{.tif}{png}{.png}{`convert #1 `dirname #1`/`basename #1 .tif`.png}
\graphicspath{{../Diagrams/Diagram_PDFs/} {../Diagrams/Numerical_Results/}}

%\input imports all commands from the target files
%The idea behind this file is that it will be used to store all the maths-related macros that I concoct; so that I can import all the commands by \input{this file} in the preamble of any file that I want to use them in.
%This should make the top-level files look a lot cleaner, and the preamble much shorter!

\usepackage{amssymb}
\usepackage{amsmath}
%\usepackage{mathtools}

%theorems and lemma etc setup using amsthm
\usepackage{amsthm}
\theoremstyle{definition}
\newtheorem{definition}{Definition}[section]
\theoremstyle{plain}
\newtheorem{theorem}{Theorem}[section]
\theoremstyle{plain}
\newtheorem{lemma}[theorem]{Lemma}
\theoremstyle{plain}
\newtheorem{prop}[theorem]{Proposition}
\theoremstyle{plain}
\newtheorem{cory}[theorem]{Corollary}
\theoremstyle{definition}
\newtheorem{convention}[theorem]{Convention}
\theoremstyle{definition}
\newtheorem{assumption}[theorem]{Assumption}
\theoremstyle{definition}
\newtheorem{conjecture}[theorem]{Conjecture}

\allowdisplaybreaks %allows equations in the same align environment to split over multiple pages.

%tstk always need be there
\newcommand{\tstk}[1]{\textbf{#1} \newline}

%this adds extra functionality to pmatrix, vmatrix, bmatrix etc by allowing you to pass an optional argument in [FACTOR] to multiply the default spacing between elements by FACTOR
\makeatletter
\renewcommand*\env@matrix[1][\arraystretch]{%
  \edef\arraystretch{#1}%
  \hskip -\arraycolsep
  \let\@ifnextchar\new@ifnextchar
  \array{*\c@MaxMatrixCols c}
  }
\makeatother

%begin the macros via newcommand. Try to group them up reasonably!

%notation and variable use throughout the file
\renewcommand{\vec}[1]{\mathbf{#1}}				%vectors are bold, not overline arrow
\newcommand{\recip}[1]{\frac{1}{#1}}			%fast reciprocal as a fraction
\newcommand{\interval}[1]{\sqbracs{0,\abs{#1}}}	%creates the closed interval from 0 to the length of the input #1, denoted by absolute value
\newcommand{\eps}{\varepsilon}					%pretty epsilons
\newcommand{\charFunc}[1]{\mathcal{I}_{#1}}%{\mathbb{1}_{#1}}		%characteristic function of a set

\newcommand{\dddom}{\widetilde{\Omega}}			%3D domain notation
\newcommand{\ddom}{\Omega}						%2D domain notation
\newcommand{\dddmes}{\widetilde{\mu}}			%3D measure
\newcommand{\ddmes}{\mu}						%2D measure

\newcommand{\graph}{\mathbb{G}}					%graph variable
\newcommand{\vertSet}{\mathcal{V}}					%set of vertices rather than big V
\newcommand{\edgeSet}{\mathcal{E}}					%set of edges rather than large E, \graph = (V,E)
\newcommand{\wavenumber}{\kappa}				%fourier variable or wavenumber, not to confuse with jk subscripts!
\newcommand{\qm}{\theta}						%quasi-momentum parameter
\newcommand{\kt}{\bracs{\wavenumber, \qm}}		%(k, theta) pair
\newcommand{\dmap}{\Gamma_0}					%Dirichlet map
\newcommand{\nmap}{\Gamma_1}					%Neumann map
\newcommand{\effFreq}{\Lambda}					%Effective frequency sqrt(w^2-\wavenumber^2)
\newcommand{\conLeft}{\stackrel{\rightarrow}{\smash{\sim}\rule{0pt}{0.4ex}}} %j connects to k, j left
\newcommand{\conRight}{\stackrel{\leftarrow}{\smash{\sim}\rule{0pt}{0.4ex}}} %j connects to k, j right
\newcommand{\con}{\sim}							%j connects to k, indifferent of direction

%standard sets
\newcommand{\naturals}{\mathbb{N}}			%natural numbers
\newcommand{\integers}{\mathbb{Z}}			%integers
\newcommand{\rationals}{\mathbb{Q}}			%rational numbers
\newcommand{\reals}{\mathbb{R}}				%real numbers
\newcommand{\complex}{\mathbb{C}}			%complex numbers

%brackets and norms
\newcommand{\bracs}[1]{\left( #1 \right)}				%encloses input in brackets
\newcommand{\sqbracs}[1]{\left[ #1 \right]}				%encloses input in square brackets
\newcommand{\clbracs}[1]{\left\{ #1 \right\}}			%encloses input in curly bracers
\newcommand{\abs}[1]{\left\lvert #1 \right\rvert}					%absolute value
\newcommand{\norm}[1]{\lvert\lvert #1 \rvert\rvert}		%norm 

%function sets
\newcommand{\smooth}[1]{C^{\infty}\bracs{#1}}							%smooth functions
\newcommand{\ltwo}[2]{L^{2}\bracs{#1,\mathrm{d}#2}}						%general L^2 space
\newcommand{\gradSob}[2]{H^1_\mathrm{grad}\bracs{#1, \mathrm{d}#2}}		%gradient Sobolev space
\newcommand{\gradSobQM}[2]{H^1_{\qm, \mathrm{grad}}\bracs{#1, \mathrm{d}#2}} %gradient + i\qm Sobolev space
\newcommand{\ktgradSob}[2]{H^1_{\wavenumber,\qm,\mathrm{grad}}\bracs{#1, \mathrm{d}#2}}	%k,\qm-gradient Sobolev space
\newcommand{\curlSob}[2]{H^1_\mathrm{curl}\bracs{#1, \mathrm{d}#2}}		%curl Sobolev space
\newcommand{\tcurlSob}[2]{H^1_{\qm, \mathrm{curl}}\bracs{#1, \mathrm{d}#2}}		%curl + i\qm Sobolev space
\newcommand{\kcurlSob}[2]{H^1_{\wavenumber,\mathrm{curl}}\bracs{#1, \mathrm{d}#2}}	%k-curl Sobolev space
\newcommand{\ktcurlSob}[2]{H^1_{\wavenumber,\qm,\mathrm{curl}}\bracs{#1, \mathrm{d}#2}}	%k,\qm-curl Sobolev space
\newcommand{\ktcurlSobDivFree}[2]{\mathcal{H}^{\kt}\bracs{#1, \mathrm{d}#2}}					%k,\qm-curl, divergence-free Sobolev space
\newcommand{\supp}{\mathrm{supp}}										%support of a function

%grad and curl sets
\newcommand{\gradZero}[2]{\mathcal{G}_{ #1, \mathrm{d}#2}\bracs{0}}		%gradients of zero for domain #1 with measure #2
\newcommand{\kgradZero}[2]{\mathcal{G}_{ #1, \mathrm{d}#2}^{(\wavenumber)}\bracs{0}}	%k-gradients of zero for domain #1 with measure #2
\newcommand{\curlZero}[2]{\mathcal{C}_{ #1, \mathrm{d}#2}\bracs{0}}	%curls of zero for domain #1 with measure #2
\newcommand{\kcurlZero}[2]{\mathcal{C}_{ #1, \mathrm{d}#2}^{(\wavenumber)}\bracs{0}}	%k-curls of zero for domain #1 with measure #2

%derivatives and grad-like symbols
\newcommand{\diff}[2]{\dfrac{\mathrm{d}#1}{\mathrm{d}#2}}			%complete derivative d#1/d#2
\newcommand{\pdiff}[2]{\dfrac{\partial #1}{\partial #2}}			%partial derivative p#1/p#2
\newcommand{\ddiff}[2]{\dfrac{\mathrm{d}^2 #1}{\mathrm{d} {#2}^2}}	%2nd deriv
\newcommand{\grad}{\nabla}											%grad operator
\newcommand{\tgrad}{\nabla^{\qm}}									%grad operator with qm superscript
\newcommand{\kgrad}{\grad^{(\wavenumber)}}							%grad with wavenumber superscript
\newcommand{\ktgrad}{\grad^{\kt}}				%grad with wavenumber, qm superscript
\newcommand{\curl}[1]{\grad_{#1}\wedge}							%curl with subscript #1
\newcommand{\kcurl}[1]{\grad_{#1}^{(\wavenumber)}\wedge}			%k-curl with measure subscript #1
\newcommand{\ktcurl}[1]{\grad_{#1}^{\kt}\wedge}		%k,theta-curl with measure subscript #1
\newcommand{\laplacian}{\Delta}						%laplacian operator, can have subscripts attached

%displaying integrals
\newcommand{\integral}[3]{\int_{#1}#2 \ \mathrm{d}#3}			%integral, domain #1, integrand #2, measure #3
\newcommand{\md}{\mathrm{d}}									%differential d

%convergence
%\newcommand{\lconv}[1]{\xrightarrow{#1}}							%convergence with #1 above the rightarrow - requires mathtools
\newcommand{\lconv}[1]{\overset{#1}{\longrightarrow}}				%convergence with #1 above the rightarrow
\newcommand{\toInfty}[1]{ \ \text{as} \ #1 \rightarrow\infty}		%writes out "as #1 tends to infty" %maths commands, variables, and other packages

%labelling hacks
\newcommand\labelthis{\addtocounter{equation}{1}\tag{\theequation}}

\DeclareMathOperator*{\dw}{width}
\newcommand{\dirWidth}[2]{\dw_{#2}\bracs{#1}}
\newcommand{\tlambda}{\tilde{\lambda}}

%-------------------------------------------------------------------------
%DOCUMENT STARTS

\begin{document}

Throughout, write $\Phi = \bracs{\phi_1, \phi_2, \phi_3}^\top$ and define 
\begin{align*}
	\Psi^{(jk)} &= R_{jk} \begin{pmatrix} \phi_1^{(jk)} \\ \phi_2^{(jk)} \end{pmatrix}.
\end{align*}
Seek functions $u\in\ktcurlSob{\ddom}{\dddmes}$ such that
\begin{align} \label{eq:PeriodCellCurlCurlWeakForm}
	\integral{\ddom}{ \ktcurl{\dddmes}u\cdot\overline{\ktcurl{\dddmes}\Phi} }{\dddmes} &= \omega^2 \integral{\ddom}{ u\cdot\overline{\Phi} }{\dddmes},
	\quad\forall \Phi\in\smooth{\ddom}^3.
\end{align}
Notice that we immediately have that $u$ is $\kt$-divergence free, since we can choose $\Phi = \ktgrad\phi$ for any smooth $\phi$. \newline

The equality in \eqref{eq:PeriodCellCurlCurlWeakForm} holds (in particular) whenever we take $\Phi$ to be a smooth function whose support only intersects (the interior of) an edge $I_{jk}$, and no other parts of the graph $\graph$.
In this case, \eqref{eq:PeriodCellCurlCurlWeakForm} reduces to
\begin{align*}
	\integral{I_{jk}}{ \ktcurl{\ddmes}u\cdot\overline{\ktcurl{\ddmes}\Phi} }{\lambda_{jk}} &= \omega^2 \integral{I_{jk}}{ u\cdot\overline{\Phi} }{\lambda_{jk}},
\end{align*}
since the contribution at the vertices is zero due to the choice of $\Phi$.
Using our knowledge of $\kt$-tangential curls, we obtain
\begin{align*}
	\integral{I_{jk}}{ \bracs{ \bracs{u_3^{(jk)}}' + \rmi\qm_{jk} u_3^{(jk)} - \rmi\wavenumber U_2^{(jk)} }\overline{\bracs{ \bracs{\phi_3^{(jk)}}' + \rmi\qm_{jk} \phi_3^{(jk)} - \rmi\wavenumber \Psi_2^{(jk)} }} }{\lambda_{jk}}
	&= \omega^2 \integral{I_{jk}}{ u\cdot\overline{\Phi} }{\lambda_{jk}},
\end{align*}
and then using the change of variables $r_{jk}$ this implies
\begin{align*} 
	\int_0^{\abs{I_{jk}}} \bracs{ \bracs{\widetilde{u}_3^{(jk)}}' + \rmi\qm_{jk} \widetilde{u}_3^{(jk)} - \rmi\wavenumber \widetilde{U}_2^{(jk)} } 
	& \bracs{ \bracs{\overline{\widetilde{\phi}}_3^{(jk)}}' - \rmi\qm_{jk} \overline{\widetilde{\phi}}_3^{(jk)} + \rmi\wavenumber \overline{\widetilde{\Psi}}_2^{(jk)} } \ \md y 
	\\
	&= \omega^2 \int_0^{\abs{I_{jk}}} \widetilde{U}_1^{(jk)}\overline{\widetilde{\Psi}}_1^{(jk)} + \widetilde{U}_2^{(jk)}\overline{\widetilde{\Psi}}_2^{(jk)} + \widetilde{u}_3^{(jk)}\overline{\widetilde{\phi}}_3^{(jk)} \ \md y,
\end{align*}
which holds for all $\widetilde{\Psi}_1^{(jk)}, \widetilde{\Psi}_2^{(jk)}, \widetilde{\phi}_3^{(jk)}\in\smooth{\interval{I_{jk}}}$ with compact support in $\bracs{0,\abs{I_{jk}}}$.
Therefore, for all $\psi\in\smooth{\interval{I_{jk}}}$ with compact support in $\bracs{0,\abs{I_{jk}}}$ we have that
\begin{subequations}
	\begin{align*}
		0 &= \widetilde{U}_1^{(jk)}, \labelthis\label{eq:CurlCurlStrongFormPhi1} \\
		0 &= \int_0^{\abs{I_{jk}}} \overline{\psi} \bracs{ \rmi\wavenumber\bracs{\widetilde{u}_3^{(jk)}}' + \bracs{\wavenumber^2 - \omega^2}\widetilde{U}_2^{(jk)} - \wavenumber\qm_{jk}\widetilde{u}_3^{(jk)}  } \ \md y, \labelthis\label{eq:CurlCurlWeakFormPhi2} \\
		0 &= \int_0^{\abs{I_{jk}}} \overline{\psi}' \bracs{ \bracs{\widetilde{u}_3^{(jk)}}'
		- \rmi\wavenumber\widetilde{U}_2^{(jk)} + \rmi\qm_{jk}\widetilde{u}_3^{(jk)} } \\
		&\qquad -\rmi\qm_{jk}\overline{\psi}\bracs{ \bracs{\widetilde{u}_3^{(jk)}}' - \rmi\wavenumber\bracs{\widetilde{U}_2^{(jk)}}' + \rmi\qm_{jk}\widetilde{u}_3^{(jk)} }
		- \omega^2 \widetilde{u}_3^{(jk)}\overline{\psi} \ \md y. \labelthis\label{eq:CurlCurlWeakFormPhi3}
	\end{align*}
\end{subequations}
The equation \eqref{eq:CurlCurlWeakFormPhi3} can be manipulated to demonstrate that
\begin{align*}
	-\int_0^{\abs{I_{jk}}} \overline{\psi}' \widetilde{u}'_{3,jk} \ \md y
	&= \int_0^{\abs{I_{jk}}} \overline{\psi} \bracs{ \rmi\wavenumber\widetilde{U}'_{2,jk} - \wavenumber\qm_{jk}\widetilde{U}_{2,jk} - 2\rmi\qm_{jk}\widetilde{u}'_{3,jk} + \qm_{jk}^2\widetilde{u}_{3,jk} - \omega^2\widetilde{u}_{3,jk} } \ \md y.
\end{align*}
This holds for all such $\psi$, so $\bracs{\widetilde{u}_3^{(jk)}}'\in\gradSob{\interval{I_{jk}}}{y}$  --- that is, $\widetilde{u}$ is twice (weakly) differentiable.
As such, \eqref{eq:CurlCurlWeakFormPhi3} implies
\begin{align*}
	0 &= \int_0^{\abs{I_{jk}}} \overline{\psi} \bracs{ \rmi\wavenumber\bracs{\widetilde{U}_2^{(jk)}}' - \wavenumber\qm_{jk}\widetilde{U}_2^{(jk)} - \bracs{\widetilde{u}_3^{(jk)}}'' - 2\rmi\qm_{jk}\bracs{\widetilde{u}_3^{(jk)}}' + \qm_{jk}^2\widetilde{u}_3^{(jk)} - \omega^2\widetilde{u}_3^{(jk)} } \ \md y,
\end{align*}
and since \eqref{eq:CurlCurlWeakFormPhi2} and the equation above hold for all smooth $\psi$ with compact support, (after some rearranging) we have
\begin{subequations} \label{eq:QGPhiStandalone}
	\begin{align}
		i\wavenumber \bracs{ \diff{}{y} + \rmi\qm_{jk} }\widetilde{u}_3^{(jk)} + \wavenumber^2\widetilde{U}_2^{(jk)} &= \omega^2\widetilde{U}_2^{(jk)}, \label{eq:QGPhi2Standalone} \\
		-\bracs{ \diff{}{y} + \rmi\qm_{jk} }^2\widetilde{u}_3^{(jk)} + \rmi\wavenumber\bracs{ \diff{}{y} + \rmi\qm_{jk} }\widetilde{U}_2^{(jk)} &= \omega^2 \widetilde{u}_3^{(jk)}, \label{eq:QGPhi3Standalone}
	\end{align}
\end{subequations}
on each $I_{jk}$. \newline

Now we return to \eqref{eq:PeriodCellCurlCurlWeakForm}, and fix a vertex $v_j\in\vertSet$.
Consider smooth $\Phi$ with whose support contains the vertex $v_j$ in its interior, and no other vertices of $\graph$.
In which case, we have that
\begin{align*}
	\integral{\ddom}{ \ktcurl{\dddmes}u\cdot\overline{\ktcurl{\dddmes}\Phi} - \omega^2 u\cdot\overline{\Phi} }{\nu}
	&= - \omega^2 \alpha_j u\bracs{v_j}\cdot\overline{\Phi}\bracs{v_j},
\end{align*}
since tangential curls are zero at the vertices, and
\begin{align*}
	& \integral{\ddom}{ \ktcurl{\dddmes}u\cdot\overline{\ktcurl{\dddmes}\Phi} - \omega^2 u\cdot\overline{\Phi} }{\ddmes} \\
	&= \sum_{j\con k}\integral{I_{jk}}{ \bracs{ \bracs{u_3^{(jk)}}' + \rmi\qm_{jk} u_3^{(jk)} - \rmi\wavenumber U_2^{(jk)} }\overline{\bracs{ \bracs{\phi_3^{(jk)}}' + \rmi\qm_{jk} \phi_3^{(jk)} - \rmi\wavenumber \Psi_2^{(jk)} }} - \omega^2 u\cdot\overline{\Phi} }{\lambda_{jk}} \\
	&= \sum_{j\con k}\integral{I_{jk}}{ \bracs{ \bracs{u_3^{(jk)}}' + \rmi\qm_{jk} u_3^{(jk)} - \rmi\wavenumber U_2^{(jk)} }\overline{\bracs{ \bracs{\phi_3^{(jk)}}' + \rmi\qm_{jk} \phi_3^{(jk)} }} - \omega^2 u_3^{(jk)}\overline{\phi}_3^{(jk)} }{\lambda_{jk}},
\end{align*}
upon using \eqref{eq:CurlCurlStrongFormPhi1} and \eqref{eq:CurlCurlWeakFormPhi2}.
Then changing variables via $r_{jk}$ and using \eqref{eq:QGPhi2Standalone} on each connecting edge, we find that (with $\widetilde{\phi}_3^{(jk)} = \psi^{(jk)}$),
\begin{align*}
	& \integral{\ddom}{ \ktcurl{\dddmes}u\cdot\overline{\ktcurl{\dddmes}\Phi} - \omega^2 u\cdot\overline{\Phi} }{\ddmes} \\
	&= \sum_{j\con k} \int_0^{\abs{I_{jk}}} 	\overline{\psi^{(jk)}}' \bracs{ \bracs{\widetilde{u}_3^{(jk)}}' - \rmi\wavenumber\widetilde{U}_2^{(jk)} + \rmi\qm_{jk}\widetilde{u}_3^{(jk)} } \\
		&\qquad -\rmi\qm_{jk}\overline{\psi^{(jk)}}\bracs{ \bracs{\widetilde{u}_3^{(jk)}}' - \rmi\wavenumber\widetilde{U}_2^{(jk)} + \rmi\qm_{jk}\widetilde{u}_3^{(jk)} }
		- \omega^2 \widetilde{u}_3^{(jk)}\overline{\psi^{(jk)}} \ \md y \\
	&= \sum_{j\con k}\sqbracs{ \overline{\psi^{(jk)}}\bracs{ \bracs{\widetilde{u}_3^{(jk)}}' - \rmi\wavenumber\widetilde{U}_2^{(jk)} + \rmi\qm_{jk}\widetilde{u}_3^{(jk)} } }_{v_j} \\
	&\quad + \sum_{j\con k}\int_0^{\abs{I_{jk}}} \overline{\psi^{(jk)}} \bracs{ -\bracs{\widetilde{u}_3^{(jk)}}'' + \rmi\wavenumber\bracs{\widetilde{U}_2^{(jk)}}' - \rmi\qm_{jk}\bracs{\widetilde{u}_3^{(jk)}}' } \ \md y \\
	&\quad + \sum_{j\con k}\int_0^{\abs{I_{jk}}} \overline{\psi^{(jk)}} \bracs{ - \rmi\qm_{jk}\bracs{\widetilde{u}_3^{(jk)}}' - \wavenumber\qm_{jk}\widetilde{U}_2^{(jk)} + \qm_{jk}^2\widetilde{u}_3^{(jk)} - \omega^2\widetilde{u}_3^{(jk)} } \ \md y \\
	&= \sum_{j\con k}\sqbracs{ \overline{\psi^{(jk)}}\bracs{ \bracs{\widetilde{u}_3^{(jk)}}' - \rmi\wavenumber\widetilde{U}_2^{(jk)} + \rmi\qm_{jk}\widetilde{u}_3^{(jk)} } }_{v_j},
\end{align*}
where we have used \eqref{eq:CurlCurlWeakFormPhi3} for the final step.
Inserting these into \eqref{eq:PeriodCellCurlCurlWeakForm}, we have that
\begin{align*}
	& \alpha_j \omega^2 \bracs{ u_1\bracs{v_j} \overline{\phi}_1\bracs{v_j} + u_2\bracs{v_j} \overline{\phi}_2\bracs{v_j} + u_3\bracs{v_j} \overline{\phi}_3\bracs{v_j} } \\
	&= \sum_{j\con k}\sqbracs{ \overline{\psi^{(jk)}}\bracs{ \bracs{\widetilde{u}_3^{(jk)}}' - \rmi\wavenumber\widetilde{U}_2^{(jk)} + \rmi\qm_{jk}\widetilde{u}_3^{(jk)} } }_{v_j} \\
	&= \overline{\phi}_3\bracs{v_j}\sum_{j\con k}\sqbracs{ \bracs{\widetilde{u}_3^{(jk)}}' - \rmi\wavenumber\widetilde{U}_2^{(jk)} + \rmi\qm_{jk}\widetilde{u}_3^{(jk)} }_{v_j},
\end{align*}
after recalling the map $r_{jk}$, and that $\phi_3$ is continuous at the vertex $v_j$.
Given that this holds for all $\Phi$, we have the following conditions for each $v_j\in\vertSet$: 
\begin{align*}
	u_1\bracs{v_j}\overline{\phi}_1\bracs{v_j} &= 0, \\
	u_2\bracs{v_j}\overline{\phi}_2\bracs{v_j} &= 0, \\
	\alpha_j \omega^2 u_3\bracs{v_j} \overline{\phi}_3\bracs{v_j} 
	&= \overline{\phi}_3\bracs{v_j} \bracs{ \sum_{j\con k} \bracs{ \pdiff{}{n} + \rmi\qm_{jk} }\widetilde{u}_3^{(jk)}\bracs{v_j} - \rmi\wavenumber\bracs{ \sum_{j\conRight k} \widetilde{U}_2^{(jk)} - \sum_{j\conLeft k} \widetilde{U}_2^{(jk)} } }.
\end{align*}
Note that we recover the conditions of $u_1=u_2=0$ at the vertices, as we had from the divergence-free condition.
Given what we have deduced holds for every such smooth $\phi$, and combined with \eqref{eq:QGPhiStandalone}, we arrive at the following system of equations;
\begin{subequations} \label{eq:QGRawSystem}
	\begin{align}
		\rmi\wavenumber \bracs{ \diff{}{y} + \rmi\qm_{jk} }\widetilde{u}_3^{(jk)} + \wavenumber^2\widetilde{U}_2^{(jk)} &= \omega^2\widetilde{U}_2^{(jk)}, \label{eq:QGPhi2} \\
		-\bracs{ \diff{}{y} + \rmi\qm_{jk} }^2\widetilde{u}_3^{(jk)} + \rmi\wavenumber\bracs{ \diff{}{y} + \rmi\qm_{jk} }\widetilde{U}_2^{(jk)} &= \omega^2 \widetilde{u}_3^{(jk)}, \label{eq:QGPhi3} \\
		\widetilde{u}_3 \text{ is continuous at } v_j &\quad\forall v_j\in\vertSet, \label{eq:QGContinuity} \\
		\sum_{j\con k} \bracs{ \pdiff{}{n} + \rmi\qm_{jk} }\widetilde{u}_3^{(jk)}\bracs{v_j} - \rmi\wavenumber\bracs{ \sum_{j\conRight k} \widetilde{U}_2^{(jk)} - \sum_{j\conLeft k} \widetilde{U}_2^{(jk)} } 
		&= \alpha_j \omega^2 u_3\bracs{v_j}. \label{eq:QGVertexCondition}
	\end{align}
\end{subequations}
Note that we have neglected to explicitly include equation \eqref{eq:CurlCurlStrongFormPhi1} and conditions $u_1(v_j)=u_2(v_j)=0$ in \eqref{eq:QGRawSystem}.
These are implicitly still there, but do not influence $\widetilde{u}_3^{(jk)}, \widetilde{U}_2^{(jk)}$, so we don't bother to write them down.

\end{document}