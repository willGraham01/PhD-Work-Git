\documentclass[11pt]{report}

\usepackage{url}
\usepackage[margin=2.5cm]{geometry} % See geometry.pdf to learn the layout options. There are lots.
\geometry{a4paper} %or letterpaper or a5paper or ...

%for figures and graphics
\usepackage{graphicx}
\usepackage{subcaption} %allows subfigures
\usepackage[bottom]{footmisc} %footnotes go below figures
\usepackage{tikz}
%\usepackage{parskip} %adds line space between paragraphs by default
\usepackage{enumerate} %allows lower case roman numerials in enumerate environments

\DeclareGraphicsRule{.tif}{png}{.png}{`convert #1 `dirname #1`/`basename #1 .tif`.png}
\graphicspath{{../Diagrams/Diagram_PDFs/} {../Diagrams/Numerical_Results/}}

%\input imports all commands from the target files
%The idea behind this file is that it will be used to store all the maths-related macros that I concoct; so that I can import all the commands by \input{this file} in the preamble of any file that I want to use them in.
%This should make the top-level files look a lot cleaner, and the preamble much shorter!

\usepackage{amssymb}
\usepackage{amsmath}
%\usepackage{mathtools}

%theorems and lemma etc setup using amsthm
\usepackage{amsthm}
\theoremstyle{definition}
\newtheorem{definition}{Definition}[section]
\theoremstyle{plain}
\newtheorem{theorem}{Theorem}[section]
\theoremstyle{plain}
\newtheorem{lemma}[theorem]{Lemma}
\theoremstyle{plain}
\newtheorem{prop}[theorem]{Proposition}
\theoremstyle{plain}
\newtheorem{cory}[theorem]{Corollary}
\theoremstyle{definition}
\newtheorem{convention}[theorem]{Convention}
\theoremstyle{definition}
\newtheorem{assumption}[theorem]{Assumption}
\theoremstyle{definition}
\newtheorem{conjecture}[theorem]{Conjecture}

\allowdisplaybreaks %allows equations in the same align environment to split over multiple pages.

%tstk always need be there
\newcommand{\tstk}[1]{\textbf{#1} \newline}

%this adds extra functionality to pmatrix, vmatrix, bmatrix etc by allowing you to pass an optional argument in [FACTOR] to multiply the default spacing between elements by FACTOR
\makeatletter
\renewcommand*\env@matrix[1][\arraystretch]{%
  \edef\arraystretch{#1}%
  \hskip -\arraycolsep
  \let\@ifnextchar\new@ifnextchar
  \array{*\c@MaxMatrixCols c}
  }
\makeatother

%begin the macros via newcommand. Try to group them up reasonably!

%notation and variable use throughout the file
\renewcommand{\vec}[1]{\mathbf{#1}}				%vectors are bold, not overline arrow
\newcommand{\recip}[1]{\frac{1}{#1}}			%fast reciprocal as a fraction
\newcommand{\interval}[1]{\sqbracs{0,\abs{#1}}}	%creates the closed interval from 0 to the length of the input #1, denoted by absolute value
\newcommand{\eps}{\varepsilon}					%pretty epsilons
\newcommand{\charFunc}[1]{\mathcal{I}_{#1}}%{\mathbb{1}_{#1}}		%characteristic function of a set

\newcommand{\dddom}{\widetilde{\Omega}}			%3D domain notation
\newcommand{\ddom}{\Omega}						%2D domain notation
\newcommand{\dddmes}{\widetilde{\mu}}			%3D measure
\newcommand{\ddmes}{\mu}						%2D measure

\newcommand{\graph}{\mathbb{G}}					%graph variable
\newcommand{\vertSet}{\mathcal{V}}					%set of vertices rather than big V
\newcommand{\edgeSet}{\mathcal{E}}					%set of edges rather than large E, \graph = (V,E)
\newcommand{\wavenumber}{\kappa}				%fourier variable or wavenumber, not to confuse with jk subscripts!
\newcommand{\qm}{\theta}						%quasi-momentum parameter
\newcommand{\kt}{\bracs{\wavenumber, \qm}}		%(k, theta) pair
\newcommand{\dmap}{\Gamma_0}					%Dirichlet map
\newcommand{\nmap}{\Gamma_1}					%Neumann map
\newcommand{\effFreq}{\Lambda}					%Effective frequency sqrt(w^2-\wavenumber^2)
\newcommand{\conLeft}{\stackrel{\rightarrow}{\smash{\sim}\rule{0pt}{0.4ex}}} %j connects to k, j left
\newcommand{\conRight}{\stackrel{\leftarrow}{\smash{\sim}\rule{0pt}{0.4ex}}} %j connects to k, j right
\newcommand{\con}{\sim}							%j connects to k, indifferent of direction

%standard sets
\newcommand{\naturals}{\mathbb{N}}			%natural numbers
\newcommand{\integers}{\mathbb{Z}}			%integers
\newcommand{\rationals}{\mathbb{Q}}			%rational numbers
\newcommand{\reals}{\mathbb{R}}				%real numbers
\newcommand{\complex}{\mathbb{C}}			%complex numbers

%brackets and norms
\newcommand{\bracs}[1]{\left( #1 \right)}				%encloses input in brackets
\newcommand{\sqbracs}[1]{\left[ #1 \right]}				%encloses input in square brackets
\newcommand{\clbracs}[1]{\left\{ #1 \right\}}			%encloses input in curly bracers
\newcommand{\abs}[1]{\left\lvert #1 \right\rvert}					%absolute value
\newcommand{\norm}[1]{\lvert\lvert #1 \rvert\rvert}		%norm 

%function sets
\newcommand{\smooth}[1]{C^{\infty}\bracs{#1}}							%smooth functions
\newcommand{\ltwo}[2]{L^{2}\bracs{#1,\mathrm{d}#2}}						%general L^2 space
\newcommand{\gradSob}[2]{H^1_\mathrm{grad}\bracs{#1, \mathrm{d}#2}}		%gradient Sobolev space
\newcommand{\gradSobQM}[2]{H^1_{\qm, \mathrm{grad}}\bracs{#1, \mathrm{d}#2}} %gradient + i\qm Sobolev space
\newcommand{\ktgradSob}[2]{H^1_{\wavenumber,\qm,\mathrm{grad}}\bracs{#1, \mathrm{d}#2}}	%k,\qm-gradient Sobolev space
\newcommand{\curlSob}[2]{H^1_\mathrm{curl}\bracs{#1, \mathrm{d}#2}}		%curl Sobolev space
\newcommand{\tcurlSob}[2]{H^1_{\qm, \mathrm{curl}}\bracs{#1, \mathrm{d}#2}}		%curl + i\qm Sobolev space
\newcommand{\kcurlSob}[2]{H^1_{\wavenumber,\mathrm{curl}}\bracs{#1, \mathrm{d}#2}}	%k-curl Sobolev space
\newcommand{\ktcurlSob}[2]{H^1_{\wavenumber,\qm,\mathrm{curl}}\bracs{#1, \mathrm{d}#2}}	%k,\qm-curl Sobolev space
\newcommand{\ktcurlSobDivFree}[2]{\mathcal{H}^{\kt}\bracs{#1, \mathrm{d}#2}}					%k,\qm-curl, divergence-free Sobolev space
\newcommand{\supp}{\mathrm{supp}}										%support of a function

%grad and curl sets
\newcommand{\gradZero}[2]{\mathcal{G}_{ #1, \mathrm{d}#2}\bracs{0}}		%gradients of zero for domain #1 with measure #2
\newcommand{\kgradZero}[2]{\mathcal{G}_{ #1, \mathrm{d}#2}^{(\wavenumber)}\bracs{0}}	%k-gradients of zero for domain #1 with measure #2
\newcommand{\curlZero}[2]{\mathcal{C}_{ #1, \mathrm{d}#2}\bracs{0}}	%curls of zero for domain #1 with measure #2
\newcommand{\kcurlZero}[2]{\mathcal{C}_{ #1, \mathrm{d}#2}^{(\wavenumber)}\bracs{0}}	%k-curls of zero for domain #1 with measure #2

%derivatives and grad-like symbols
\newcommand{\diff}[2]{\dfrac{\mathrm{d}#1}{\mathrm{d}#2}}			%complete derivative d#1/d#2
\newcommand{\pdiff}[2]{\dfrac{\partial #1}{\partial #2}}			%partial derivative p#1/p#2
\newcommand{\ddiff}[2]{\dfrac{\mathrm{d}^2 #1}{\mathrm{d} {#2}^2}}	%2nd deriv
\newcommand{\grad}{\nabla}											%grad operator
\newcommand{\tgrad}{\nabla^{\qm}}									%grad operator with qm superscript
\newcommand{\kgrad}{\grad^{(\wavenumber)}}							%grad with wavenumber superscript
\newcommand{\ktgrad}{\grad^{\kt}}				%grad with wavenumber, qm superscript
\newcommand{\curl}[1]{\grad_{#1}\wedge}							%curl with subscript #1
\newcommand{\kcurl}[1]{\grad_{#1}^{(\wavenumber)}\wedge}			%k-curl with measure subscript #1
\newcommand{\ktcurl}[1]{\grad_{#1}^{\kt}\wedge}		%k,theta-curl with measure subscript #1
\newcommand{\laplacian}{\Delta}						%laplacian operator, can have subscripts attached

%displaying integrals
\newcommand{\integral}[3]{\int_{#1}#2 \ \mathrm{d}#3}			%integral, domain #1, integrand #2, measure #3
\newcommand{\md}{\mathrm{d}}									%differential d

%convergence
%\newcommand{\lconv}[1]{\xrightarrow{#1}}							%convergence with #1 above the rightarrow - requires mathtools
\newcommand{\lconv}[1]{\overset{#1}{\longrightarrow}}				%convergence with #1 above the rightarrow
\newcommand{\toInfty}[1]{ \ \text{as} \ #1 \rightarrow\infty}		%writes out "as #1 tends to infty" %maths commands, variables, and other packages

%labelling hacks
\newcommand\labelthis{\addtocounter{equation}{1}\tag{\theequation}}
\newcommand{\dom}[1]{\mathrm{dom}\bracs{#1}}
\newcommand{\dtn}{\mathcal{D}_\omega}

%-------------------------------------------------------------------------
%DOCUMENT STARTS

\begin{document}

Let $\ddom\subset\reals^2$ be the unit cell of a periodic (embedded) metric graph with period graph $\graph$.
The graph $\graph$ naturally separates $\ddom$ into a collection of (connected, Lipschitz) subdomains $\ddom_i$ for $i\in\Lambda$, for a suitable index set $\Lambda$.
Let $\qm$ denote the quasi-momentum (taking values in the dual cell of $\ddom$), and for $\omega>0$, define the set
\begin{align*}
	D_{\omega}^i = \clbracs{(g,h)\in L^2\bracs{\partial\ddom_i}\times L^2\bracs{\partial\ddom_i} \setVert \exists u\in H^2_{\mathrm{grad}}\bracs{\ddom_i} \text{ s.t. } \bracs{\laplacian_\qm + \omega^2}u = 0, \ u\vert_{\partial\ddom_i} = g, \ \pdiff{u}{n}\vert_{\partial\ddom_i} = h.}
\end{align*}
Then define an operator (the Dirichlet-to-Neumann map) $\dtn^i$ via
\begin{align*}
	\dom{\dtn^i} = D_\omega^i, \quad
	\dtn^i g = h,
\end{align*}
where $g,h$ are related as in $D_\omega^i$.
\tstk{NB: $\dtn^i$ is only self-adjoint if 0 is not in the spectrum of $\laplacian_\qm+\omega^2$ right?}

\emph{If} this definition is fine, and makes sense, let's proceed.
Our proposed ``strong form" for our problem on a medium with singular inclusions is to find $u\in H^2_{\mathrm{grad}}\bracs{\ddom_i}$ for every $i\in\Lambda$ and $u\in\ktgradSob{\interval{I_{jk}}}{y}$ for each edge $I_{jk}$ such that:
\begin{align*}
	\bracs{\laplacian_\qm + \omega^2}u &= 0 \quad &\text{in each } \ddom_i, \\
	-\bracs{\diff{}{y}+\rmi\qm_{jk}}^2 u_{jk} &= \omega^2 u_{jk} + \bracs{\grad u\vert_{\partial\ddom^+} - \grad u\vert_{\partial\ddom^-}}\cdot n_{jk}, \quad &\text{on each } I_{jk}, \\
	\sum_{j\con k}\bracs{\pdiff{}{n} + \rmi\qm_{jk}}u_{jk}\bracs{v_j} &= 0, \quad &\text{at each } v_j\in\vertSet, \\
	u \text{ is continuous } & \text{at } v_j, \quad &\text{at each } v_j\in\vertSet.
\end{align*}
Note that $\ddom^+$ and $\ddom^-$ in these contexts refer to the two regions $\ddom_i$ whose boundary has non-empty intersection with $I_{jk}$, and vertical bars denote traces.
Now we can show (subject to checking the ``derivation" of this problem, at the least assume) that $u$ is also continuous across the edges $I_{jk}$, so $u\vert_{\partial\ddom^{+}} = u\vert_{\partial\ddom^{-}}$ on $I_{jk} = \partial\ddom^{+} \cap \partial\ddom^{-}$.
We can reformulate the above as
\begin{align*}
	\bracs{\laplacian_\qm + \omega^2}u &= 0 \quad &\text{in each } \ddom_i, \\
	-\bracs{\diff{}{y}+\rmi\qm_{jk}}^2 u_{jk} &= \omega^2 u_{jk} + \dtn^+ \bracs{u\vert_{\partial\ddom^+}} - \dtn^- \bracs{u\vert_{\partial\ddom^-}}, \quad &\text{on each } I_{jk}, \\
	\sum_{j\con k}\bracs{\pdiff{}{n} + \rmi\qm_{jk}}u_{jk}\bracs{v_j} &= 0, \quad &\text{at each } v_j\in\vertSet, \\
	u \text{ is continuous } & \text{at } v_j, \quad &\text{at each } v_j\in\vertSet.
\end{align*}
This is still not super accessible for us to solve, however if we further assume that the traces of $u$ onto the graph $\graph$ match the edge-functions $u_{jk}$, we can make some progress.
That is, assume that for any $I_{jk}\in\edgeSet$ such that $\partial\ddom_i\cap I_{jk}\neq\emptyset$ we have that $u\vert_{\partial\ddom_i} = u_{jk}$ on $I_{jk}$.
This allows us to replace the traces from either side of the edges, and drop the $\bracs{\laplacian_\qm + \omega^2}u = 0$ equation (since under our new assumptions this will automatically be satisfied by the solution for $u_{jk}$ that we find), giving
\begin{align*}
	-\bracs{\diff{}{y}+\rmi\qm_{jk}}^2 u_{jk} &= \omega^2 u_{jk} + \dtn^+ \bracs{u} - \dtn^- \bracs{u}, \quad &\text{on each } I_{jk}, \\
	\sum_{j\con k}\bracs{\pdiff{}{n} + \rmi\qm_{jk}}u_{jk}\bracs{v_j} &= 0, \quad &\text{at each } v_j\in\vertSet, \\
	u \text{ is continuous } & \text{at } v_j, \quad &\text{at each } v_j\in\vertSet.
\end{align*}
Here, can now think of $u\in H^2\bracs{\graph}$; the terms $\dtn^{\pm}\bracs{u}$ require knowledge of the entire function $u$, rather than just the function on the edge $u_{jk}$, since $\dtn^{\pm}$ takes as an argument the Dirichlet data on all of $\partial\ddom^{\pm}$, not just $I_{jk}$.

Provided all of these things are justifiable (or reasonable assumptions), we have a well-defined quantum graph problem of finding $\omega, u$ such that the above system is satisfied.
The (action of the) maps $\dtn^i$ would need to be computed --- which is definitely non-trivial! 
Otherwise, we have $2\abs{\edgeSet}$ unknown constants from the differential equation and $\sum_j\clbracs{ \mathrm{deg}(v_j)-1} + \abs{\vertSet}$ boundary conditions, which is an exact match by the handshake lemma.
Ergo, the things that need to be checked/addressed for this to work are:
\begin{itemize}
	\item The $\dtn^i$ maps can be sensibly defined in this way.
	\item The traces of $u$ match the values of the $u_{jk}$ --- this would have to involve going back to the ``weak formulation".
	\item Can the action of the $\dtn^i$ maps actually be computed in the first place?
\end{itemize}
Then, we need to come up with a numerical scheme to handle this thing.

\end{document}