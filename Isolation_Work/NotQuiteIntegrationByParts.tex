\documentclass[11pt]{report}

\usepackage{url}
\usepackage[margin=2.5cm]{geometry} % See geometry.pdf to learn the layout options. There are lots.
\geometry{a4paper} %or letterpaper or a5paper or ...

%for figures and graphics
\usepackage{graphicx}
\usepackage{subcaption} %allows subfigures
\usepackage[bottom]{footmisc} %footnotes go below figures
%\usepackage{parskip} %adds line space between paragraphs by default
\usepackage{enumerate} %allows lower case roman numerials in enumerate environments

\DeclareGraphicsRule{.tif}{png}{.png}{`convert #1 `dirname #1`/`basename #1 .tif`.png}
\graphicspath{{../Diagrams/Diagram_PDFs/} {../Diagrams/Numerical_Results/}}

%\input imports all commands from the target files
%The idea behind this file is that it will be used to store all the maths-related macros that I concoct; so that I can import all the commands by \input{this file} in the preamble of any file that I want to use them in.
%This should make the top-level files look a lot cleaner, and the preamble much shorter!

\usepackage{amssymb}
\usepackage{amsmath}
%\usepackage{mathtools}

%theorems and lemma etc setup using amsthm
\usepackage{amsthm}
\theoremstyle{definition}
\newtheorem{definition}{Definition}[section]
\theoremstyle{plain}
\newtheorem{theorem}{Theorem}[section]
\theoremstyle{plain}
\newtheorem{lemma}[theorem]{Lemma}
\theoremstyle{plain}
\newtheorem{prop}[theorem]{Proposition}
\theoremstyle{plain}
\newtheorem{cory}[theorem]{Corollary}
\theoremstyle{definition}
\newtheorem{convention}[theorem]{Convention}
\theoremstyle{definition}
\newtheorem{assumption}[theorem]{Assumption}
\theoremstyle{definition}
\newtheorem{conjecture}[theorem]{Conjecture}

\allowdisplaybreaks %allows equations in the same align environment to split over multiple pages.

%tstk always need be there
\newcommand{\tstk}[1]{\textbf{#1} \newline}

%this adds extra functionality to pmatrix, vmatrix, bmatrix etc by allowing you to pass an optional argument in [FACTOR] to multiply the default spacing between elements by FACTOR
\makeatletter
\renewcommand*\env@matrix[1][\arraystretch]{%
  \edef\arraystretch{#1}%
  \hskip -\arraycolsep
  \let\@ifnextchar\new@ifnextchar
  \array{*\c@MaxMatrixCols c}
  }
\makeatother

%begin the macros via newcommand. Try to group them up reasonably!

%notation and variable use throughout the file
\renewcommand{\vec}[1]{\mathbf{#1}}				%vectors are bold, not overline arrow
\newcommand{\recip}[1]{\frac{1}{#1}}			%fast reciprocal as a fraction
\newcommand{\interval}[1]{\sqbracs{0,\abs{#1}}}	%creates the closed interval from 0 to the length of the input #1, denoted by absolute value
\newcommand{\eps}{\varepsilon}					%pretty epsilons
\newcommand{\charFunc}[1]{\mathcal{I}_{#1}}%{\mathbb{1}_{#1}}		%characteristic function of a set

\newcommand{\dddom}{\widetilde{\Omega}}			%3D domain notation
\newcommand{\ddom}{\Omega}						%2D domain notation
\newcommand{\dddmes}{\widetilde{\mu}}			%3D measure
\newcommand{\ddmes}{\mu}						%2D measure

\newcommand{\graph}{\mathbb{G}}					%graph variable
\newcommand{\vertSet}{\mathcal{V}}					%set of vertices rather than big V
\newcommand{\edgeSet}{\mathcal{E}}					%set of edges rather than large E, \graph = (V,E)
\newcommand{\wavenumber}{\kappa}				%fourier variable or wavenumber, not to confuse with jk subscripts!
\newcommand{\qm}{\theta}						%quasi-momentum parameter
\newcommand{\kt}{\bracs{\wavenumber, \qm}}		%(k, theta) pair
\newcommand{\dmap}{\Gamma_0}					%Dirichlet map
\newcommand{\nmap}{\Gamma_1}					%Neumann map
\newcommand{\effFreq}{\Lambda}					%Effective frequency sqrt(w^2-\wavenumber^2)
\newcommand{\conLeft}{\stackrel{\rightarrow}{\smash{\sim}\rule{0pt}{0.4ex}}} %j connects to k, j left
\newcommand{\conRight}{\stackrel{\leftarrow}{\smash{\sim}\rule{0pt}{0.4ex}}} %j connects to k, j right
\newcommand{\con}{\sim}							%j connects to k, indifferent of direction

%standard sets
\newcommand{\naturals}{\mathbb{N}}			%natural numbers
\newcommand{\integers}{\mathbb{Z}}			%integers
\newcommand{\rationals}{\mathbb{Q}}			%rational numbers
\newcommand{\reals}{\mathbb{R}}				%real numbers
\newcommand{\complex}{\mathbb{C}}			%complex numbers

%other notations
\newcommand{\rmi}{\mathrm{i}}				%imaginary unit i (RoMan font i)
\newcommand{\e}{\mathrm{e}}					%Euler number, e (Roman font e)

%brackets and norms
\newcommand{\bracs}[1]{\left( #1 \right)}				%encloses input in brackets
\newcommand{\sqbracs}[1]{\left[ #1 \right]}				%encloses input in square brackets
\newcommand{\clbracs}[1]{\left\{ #1 \right\}}			%encloses input in curly bracers
\newcommand{\abs}[1]{\left\lvert #1 \right\rvert}					%absolute value
\newcommand{\norm}[1]{\lvert\lvert #1 \rvert\rvert}		%norm 
\newcommand{\setVert}{\ \middle\vert \ }					%vertical bar for the middle of sets

%function sets
\newcommand{\smooth}[1]{C^{\infty}\bracs{#1}}							%smooth functions
\newcommand{\ltwo}[2]{L^{2}\bracs{#1,\mathrm{d}#2}}						%general L^2 space
\newcommand{\gradSob}[2]{H^1_\mathrm{grad}\bracs{#1, \mathrm{d}#2}}		%gradient Sobolev space
\newcommand{\gradSobQM}[2]{H^1_{\qm, \mathrm{grad}}\bracs{#1, \mathrm{d}#2}} %gradient + i\qm Sobolev space
\newcommand{\ktgradSob}[2]{H^1_{\wavenumber,\qm,\mathrm{grad}}\bracs{#1, \mathrm{d}#2}}	%k,\qm-gradient Sobolev space
\newcommand{\curlSob}[2]{H^1_\mathrm{curl}\bracs{#1, \mathrm{d}#2}}		%curl Sobolev space
\newcommand{\tcurlSob}[2]{H^1_{\qm, \mathrm{curl}}\bracs{#1, \mathrm{d}#2}}		%curl + i\qm Sobolev space
\newcommand{\kcurlSob}[2]{H^1_{\wavenumber,\mathrm{curl}}\bracs{#1, \mathrm{d}#2}}	%k-curl Sobolev space
\newcommand{\ktcurlSob}[2]{H^1_{\wavenumber,\qm,\mathrm{curl}}\bracs{#1, \mathrm{d}#2}}	%k,\qm-curl Sobolev space
\newcommand{\ktcurlSobDivFree}[2]{\mathcal{H}^{\kt}\bracs{#1, \mathrm{d}#2}}					%k,\qm-curl, divergence-free Sobolev space
\newcommand{\supp}{\mathrm{supp}}										%support of a function

%grad and curl sets
\newcommand{\gradZero}[2]{\mathcal{G}_{ #1, \mathrm{d}#2}\bracs{0}}		%gradients of zero for domain #1 with measure #2
\newcommand{\kgradZero}[2]{\mathcal{G}_{ #1, \mathrm{d}#2}^{(\wavenumber)}\bracs{0}}	%k-gradients of zero for domain #1 with measure #2
\newcommand{\curlZero}[2]{\mathcal{C}_{ #1, \mathrm{d}#2}\bracs{0}}	%curls of zero for domain #1 with measure #2
\newcommand{\kcurlZero}[2]{\mathcal{C}_{ #1, \mathrm{d}#2}^{(\wavenumber)}\bracs{0}}	%k-curls of zero for domain #1 with measure #2

%derivatives and grad-like symbols
\newcommand{\diff}[2]{\dfrac{\mathrm{d}#1}{\mathrm{d}#2}}			%complete derivative d#1/d#2
\newcommand{\pdiff}[2]{\dfrac{\partial #1}{\partial #2}}			%partial derivative p#1/p#2
\newcommand{\ddiff}[2]{\dfrac{\mathrm{d}^2 #1}{\mathrm{d} {#2}^2}}	%2nd deriv
\newcommand{\pddiff}[2]{\dfrac{\partial^2 #1}{\partial {#2}^2}}		%2nd partial derivative
\newcommand{\grad}{\nabla}											%grad operator
\newcommand{\tgrad}{\nabla^{\qm}}									%grad operator with qm superscript
\newcommand{\kgrad}{\grad^{(\wavenumber)}}							%grad with wavenumber superscript
\newcommand{\ktgrad}{\grad^{\kt}}				%grad with wavenumber, qm superscript
\newcommand{\curl}[1]{\grad_{#1}\wedge}							%curl with subscript #1
\newcommand{\kcurl}[1]{\grad_{#1}^{(\wavenumber)}\wedge}			%k-curl with measure subscript #1
\newcommand{\ktcurl}[1]{\grad_{#1}^{\kt}\wedge}		%k,theta-curl with measure subscript #1
\newcommand{\laplacian}{\Delta}						%laplacian operator, can have subscripts attached

%displaying integrals
\newcommand{\integral}[3]{\int_{#1}#2 \ \mathrm{d}#3}			%integral, domain #1, integrand #2, measure #3
\newcommand{\md}{\mathrm{d}}									%differential d

%convergence
%\newcommand{\lconv}[1]{\xrightarrow{#1}}							%convergence with #1 above the rightarrow - requires mathtools
\newcommand{\lconv}[1]{\overset{#1}{\longrightarrow}}				%convergence with #1 above the rightarrow
\newcommand{\toInfty}[1]{ \ \text{as} \ #1 \rightarrow\infty}		%writes out "as #1 tends to infty" %maths commands, variables, and other packages

%labelling hacks
\newcommand\labelthis{\addtocounter{equation}{1}\tag{\theequation}}

\DeclareMathOperator*{\dw}{width}
\newcommand{\dirWidth}[2]{\dw_{#2}\bracs{#1}}
\newcommand{\tlambda}{\tilde{\lambda}}

%-------------------------------------------------------------------------
%DOCUMENT STARTS

\begin{document}

Throughout, let $I\subset\ddom$ be the segment $I=\sqbracs{v_j,v_k}$.
Let $\lambda_I$ be the singular measure supported on the edge $I$.
Set $e_I := \frac{v_k-v_j}{\norm{v_k-v_j}}$ as the unit vector parallel to $I$, and $r:\interval{I}\rightarrow I$ as the change of variables $r(t) = v_j + te_I$.
Let $n_I$ be the normal to the segment $I$ so $e_I, n_I$ are an orthonormal basis for $\reals^2$.
Denote composition of functions with $r$ by use of an overhead tilde, so $\tilde{\phi}(t) := \phi\bracs{r(t)}$. \newline

\begin{lemma}[Membership of $\gradSob{\ddom}{\lambda_I}$ implies membership of $\gradSob{\interval{I}}{t}$.] \label{lem:SingularSobSpaceImpliesIntervalSobSpace}
	Suppose $\bracs{u,w}\in\gradSob{\ddom}{\lambda_I}$.
	Then $\bracs{\tilde{u},\tilde{w}\cdot e_I}\in\gradSob{\interval{I}}{t}$.
\end{lemma}
\begin{proof}
	As $\bracs{u,w}\in\gradSob{\ddom}{\lambda_I}$ there exist smooth functions $\phi_n\in\smooth{\ddom}$ such that
	\begin{align*}
		\phi_n \lconv{\ltwo{\ddom}{\lambda_I}} u, &\quad \grad\phi_n \lconv{\ltwo{\ddom}{\lambda_I}^2} w.
	\end{align*}
	Thus,
	\begin{align*}
		0 &\leftarrow \integral{\ddom}{\abs{\phi_n - u}^2}{\lambda_I}
		= \integral{I}{\abs{\phi_n - u}^2}{\lambda_I} \\
		&= \int_0^{\abs{I}}{\abs{\tilde{\phi}_n - \tilde{u}}^2} \ \md t, \\
		0 &\leftarrow \integral{\ddom}{\abs{\grad\phi_n - w}^2}{\lambda_I}
		= \int_0^{\abs{I}}{\abs{\widetilde{\grad\phi_n} - \tilde{w}}^2} \ \md t \\
		&= \int_0^{\abs{I}}{\abs{\widetilde{\grad\phi_n}_1 - \tilde{w}_1}^2} \ \md t + \int_0^{\abs{I}}{\abs{\widetilde{\grad\phi_n}_2 - \tilde{w}_2}^2} \ \md t.
	\end{align*}
	In particular,
	\begin{align*}
		\widetilde{\grad\phi_n}_1 \lconv{\ltwo{\interval{I}}{t}} \tilde{w}_1, 
		&\quad \widetilde{\grad\phi_n}_2 \lconv{\ltwo{\interval{I}}{t}} \tilde{w}_2.
	\end{align*}
	Hence
	\begin{align*}
		\widetilde{\grad\phi_n}\cdot e_I \lconv{\ltwo{\interval{I}}{t}} \tilde{w}\cdot e_I.
	\end{align*}
	However
	\begin{align*}
		\diff{\tilde{\phi_n}}{t} &= \widetilde{\grad\phi_n}\cdot r'(t)
		= \widetilde{\grad\phi_n}\cdot e_I,
	\end{align*}
	and thus we conclude that
	\begin{align*}
		\tilde{\phi}_n \lconv{\ltwo{\interval{I}}{t}} \tilde{u}, 
		&\quad \diff{\tilde{\phi}}{t} \lconv{\ltwo{\interval{I}}{t}} \tilde{w}\cdot e_I.
	\end{align*}
	Hence we have that $\bracs{\tilde{u}, \tilde{w}\cdot e_I}\in\gradSob{\interval{I}}{t}$.
\end{proof}

\begin{cory} \label{cory:EdgeTangentialByParts}
	If $\bracs{u,w}\in\gradSob{\ddom}{\lambda_I}$ and $\phi\in\smooth{\ddom}$ then we have that
	\begin{align*}
		\integral{\ddom}{u\grad\phi\cdot e_I}{\lambda_I} &= -\integral{\ddom}{\phi w\cdot e_I}{\lambda_I}.
	\end{align*}
\end{cory}
\begin{proof}
	Performing manipulations and using the change of variables $r(t)$ yields
	\begin{align*}
		\integral{\ddom}{u\grad\phi\cdot e_I}{\lambda_I} &= \integral{I}{u\grad\phi\cdot e_I}{\lambda_I} \\
		&= \int_0^{\abs{I}} \tilde{u}\widetilde{\grad\phi}\cdot e_I \ \md t \\
		&= \int_0^{\abs{I}} \tilde{u}\diff{\tilde{\phi}}{t} \ \md t.
	\end{align*}
	Then by lemma \ref{lem:SingularSobSpaceImpliesIntervalSobSpace} and the ``$H=W$" theorem for $\gradSob{\interval{I}}{t}$, we conclude that
	\begin{align*}
		\integral{\ddom}{u\grad\phi\cdot e_I}{\lambda_I} &= \int_0^{\abs{I}} \tilde{u}\diff{\tilde{\phi}}{t} \ \md t 
		= -\int_0^{\abs{I}} \tilde{\phi}\tilde{w}\cdot e_I \ \md t \\
		&= -\integral{\ddom}{\phi w\cdot e_I}{\lambda_I}.
	\end{align*}	
\end{proof}

Now let $\tlambda_I$ be the composite measure for the domain $\ddom$ with respect to the segment $I$, namely
\begin{align*}
	\tlambda_I &= \lambda_2 + \lambda_I,
\end{align*}
where $\lambda_2$ is the Lebesgue measure.

\begin{lemma}[Membership of $\gradSob{\ddom}{\tlambda_I}$ implies membership of the individual spaces.] \label{lem:CompositeSobSpaceImpliesIndividualSobSpace}
	If $\bracs{u,v}\in\gradSob{\ddom}{\tlambda_I}$ then $\bracs{u,v}\in\gradSob{\ddom}{\lambda_2}$ and $\bracs{u,v}\in\gradSob{\ddom}{\lambda_I}$.
\end{lemma}
\begin{proof}
	We can find a sequence of smooth functions $\phi_n$ such that
	\begin{align*}
		\phi_n \lconv{\ltwo{\ddom}{\tlambda_I}} u, &\quad \grad\phi_n \lconv{\ltwo{\ddom}{\tlambda_I}^2} w.
	\end{align*}
	However this means that
	\begin{align*}
		0 &\leftarrow \integral{\ddom}{\abs{\phi_n - u}^2}{\tlambda_I}
		= \underbrace{\integral{\ddom}{\abs{\phi_n - u}^2}{\lambda_2}}_{\rightarrow0} + \underbrace{\integral{\ddom}{\abs{\phi_n - u}^2}{\lambda_I}}_{\rightarrow0}, \\
		0 &\leftarrow \integral{\ddom}{\abs{\grad\phi_n - v}^2}{\tlambda_I}
		= \underbrace{\integral{\ddom}{\abs{\grad\phi_n - v}^2}{\lambda_2}}_{\rightarrow0} + \underbrace{\integral{\ddom}{\abs{\grad\phi_n - v}^2}{\lambda_I}}_{\rightarrow0}.
	\end{align*}
	From which it follows that
	\begin{align*}
		\phi_n \lconv{\ltwo{\ddom}{\lambda_2}} u, &\quad \grad\phi_n \lconv{\ltwo{\ddom}{\lambda_2}^2} w, \\
		\phi_n \lconv{\ltwo{\ddom}{\lambda_I}} u, &\quad \grad\phi_n \lconv{\ltwo{\ddom}{\lambda_I}^2} w,
	\end{align*}
	and hence $\bracs{u,v}\in\gradSob{\ddom}{\lambda_2}$, $\bracs{u,v}\in\gradSob{\ddom}{\lambda_I}$.
\end{proof}

\begin{cory} \label{cory:CompositeLebesgueByParts}
	If $\bracs{u,v}\in\gradSob{\ddom}{\tlambda_I}$ and $\phi\in\smooth{\ddom}$ then we have that
	\begin{align*}
		\integral{\ddom}{u\grad\phi\cdot e_I}{\lambda_2} &= -\integral{\ddom}{\phi v\cdot e_I}{\lambda_2}, \\
		\integral{\ddom}{u\grad\phi\cdot n_I}{\lambda_2} &= -\integral{\ddom}{\phi v\cdot n_I}{\lambda_2}.
	\end{align*}
\end{cory}
\begin{proof}
	We know by lemma \ref{lem:CompositeSobSpaceImpliesIndividualSobSpace} that $\bracs{u,v}\in\gradSob{\ddom}{\lambda_2}$, and so by the ``$H=W$" theorem we have that
	\begin{align*}
		\integral{\ddom}{u\partial_j\phi}{\lambda_2} &= -\integral{\ddom}{\phi v_j}{\lambda_2}
	\end{align*}
	for $j\in\clbracs{1,2}$.
	The identities above then follow from considering suitable linear combinations of the $j=1$ and $j=2$ cases, above.
\end{proof}

\begin{theorem}
	If $\bracs{u,v}\in\gradSob{\ddom}{\tlambda_I}$ and $\phi\in\smooth{\ddom}$ then we have that
	\begin{align*}
		\integral{\ddom}{u\grad\phi\cdot e_I}{\tlambda_I} &= -\integral{\ddom}{\phi v\cdot e_I}{\tlambda_I}.
	\end{align*}
	In the event that $\supp\bracs{\phi}\cap I = \emptyset$, we have that
	\begin{align*}
		\integral{\ddom}{u\partial_j\phi}{\tlambda_I} &= -\integral{\ddom}{\phi v_j}{\tlambda_I},
	\end{align*}
	for $j\in\clbracs{1,2}$.
\end{theorem}
\begin{proof}
	Given the definition of the composite measure $\tlambda_I$, this is just a combination of the results of lemma \ref{lem:CompositeSobSpaceImpliesIndividualSobSpace}, and then corollaries \ref{cory:EdgeTangentialByParts} and \ref{cory:CompositeLebesgueByParts}.
	The case when $\supp\bracs{\phi}\cap I = \emptyset$ follows from this because in this case the contribution from the singular measure part of $\tlambda_I$ is zero.
\end{proof}

\end{document}